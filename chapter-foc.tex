
\chapter{Linear logic}

In this chapter, we present linear logic
as a logic with the ability to express
aspects of state and state transition in a natural way. In
Section~\ref{sec:introlinlog} we discuss a traditional account of 
linear logic, and in
Section~\ref{sec:linlogicalframeworks}, we discuss why this
account is insufficient as a {\it logical
  framework} -- derivations in linear logic suffice to establish the
existence of a series of state transitions but do not adequately
capture the structure of those transitions. 

Our remedy for this insufficiency comes in the form of {\it focusing},
Andreoli's restricted normal form for derivations in linear logic. We
discuss focusing for a polarized presentation of linear logic in
Section~\ref{sec:foclinlog}. With focusing, we can describe {\it
  synthetic inference rules} (Section~\ref{sec:linsynthetic}) that do
a much better job of capturing the structure of focused
transitions. In Section~\ref{sec:linhack}, we justify the particular
structure of our focusing system by discussing a few variations and
their particular problems. In Section~\ref{sec:linconcurrenteq}, we
describe {\it concurrent equality}, an equivalence relation on focused
derivations introduced by Watkins et.~al that is motivated by the
desire to capture independent concurrent transitions. We also discuss
the limitations of concurrent equality in a fully focused setting.

In Chapter 3, we will define a much richer ordered linear lax logic and prove
a focusing result. In Chapter 4 we will carve out a
fragment of this logic that forms the basis of the logical framework
of substructural logical specifications; this framework will also
incorporate the concurrent equality sketched out in
Section~\ref{sec:linconcurrenteq}. The purpose of this chapter is to
consider the issues that arise in our rich substructural setting
within the comparatively simple setting of intuitionistic linear logic.

\section{Introduction to linear logic}
\label{sec:introlinlog}

Logic as it has been traditionally understood and studied -- both in
its classical and intuitionistic varieties -- treats the truth of a
proposition as a {\it persistent resource}. That is, if we have
evidence for the truth of a proposition, we can ignore that evidence
if it is not needed and reuse the evidence as many times as we need
to. Throughout this thesis, ``logic as it has been traditionally
understood as studied'' will be referred to as {\it persistent} logic
to emphasize this treatment of evidence. 

Linear logic, which was studied and
popularized by Girard \cite{girard87linear},
treats evidence as an {\it ephemeral} resource; the use of an
ephemeral resource consumes it, at which point it is unavailable for
further use.  Linear logic, like persistent logic, comes in classical
and intuitionistic flavors. We will favor intuitionistic linear logic
in part because the propositions of intuitionistic linear logic
(written $A$, $B$, $C$, \ldots) have a more natural correspondence
with our physical intuitions about consumable resources. Linear
conjunction $A \tensor B$ represents the resource built from the
resources $A$ and $B$; if you have both a bowl of soup {\it and} a
sandwich, that resource can be represented by the proposition ${\sf
  soup} \otimes {\sf sandwich}$. Linear implication $A \lolli B$
represents a resource that can interact with another resource $A$ to
produce a resource $B$. One robot with batteries not included could be
represented as the linear resource $({\sf battery} \lolli {\sf
  robot})$, and the linear resource $({\sf 6bucks} \lolli {\sf soup}
\tensor {\sf sandwich})$ represents the ability to use \$6 to obtain
lunch -- but only once!\footnote{Conjunction will always bind more
  tightly than implication, so this is equivalent to the proposition
  ${\sf 6bucks} \lolli ({\sf soup} \tensor {\sf sandwich})$.} Linear
logic also has a modal connective ${!}A$ representing a persistent
resource that can be
used to generate any number of $A$ resources, including zero. The
Panera ``You Pick Two'' menu might be represented as
\[ {!}({\sf 6bucks} \lolli {\sf soup} \tensor {\sf sandwich}) \otimes
{!}({\sf 6bucks} \lolli {\sf soup} \tensor {\sf salad}) \otimes
{!}({\sf 6bucks} \lolli {\sf sandwich} \tensor {\sf salad}),\] as the
menu gives you the opportunity to exchange six dollars for two
distinct members of the set $\{ {\sf soup}, {\sf salad}, {\sf
  sandwich} \}$ any number of times.

\begin{figure}[t]
\begin{tabbing}
\quad $A$ \,\, \=  $::= p \mid {!}A \mid A \lolli B \mid \one \mid A \tensor B$\\
\quad $\Gamma$ \> $::= \cdot \mid \Gamma, A$ \qquad \= {\it (multiset)}\\
\quad $\Delta$ \> $::= \cdot \mid \Delta, A$ \> {\it (multiset)}\\
\end{tabbing}
%
%
\quad \fbox{$\seq{\Gamma}{\Delta}{A}$}
\[
\infer[{\it init}]
{\seq{\Gamma}{p}{p}}
{}
\qquad
\infer[{\it copy}]
{\seq{\Gamma, A}{\Delta}{C}}
{\seq{\Gamma, A}{\Delta, A}{C}}
%
\]

\[
%
\infer[{!}_R]
{\seq{\Gamma}{\cdot}{{!}A}}
{\seq{\Gamma}{\cdot}{A}}
\qquad
\infer[{!}_L]
{\seq{\Gamma}{\Delta, {!}A}{C}}
{\seq{\Gamma, A}{\Delta}{C}}
\qquad
\infer[\one_R]
{\seq{\Gamma}{\cdot}{\one}}
{}
\qquad
\infer[\one_L]
{\seq{\Gamma}{\Delta, \one}{C}}
{\seq{\Gamma}{\Delta}{C}}
\]

\[
%
\infer[{\tensor}_R]
{\seq{\Gamma}{\Delta_1,\Delta_2}{A \tensor B}}
{\seq{\Gamma}{\Delta_1}{A}
 &
 \seq{\Gamma}{\Delta_2}{B}}
\qquad
\infer[{\tensor}_L]
{\seq{\Gamma}{\Delta, A \tensor B}{C}}
{\seq{\Gamma}{\Delta, A, B}{C}}
\]

\[
%
\infer[{\lolli}_R]
{\seq{\Gamma}{\Delta}{A \lolli B}}
{\seq{\Gamma}{\Delta, A}{B}}
\qquad
\infer[{\lolli}_L]
{\seq{\Gamma}{\Delta_1,\Delta_2, A \lolli B}{C}}
{\seq{\Gamma}{\Delta_1}{A}
 &
 \seq{\Gamma}{\Delta_2, B}{C}}
%
\]
\caption{Intuitionstic linear logic}
\label{fig:linear}
\end{figure}


Figure~\ref{fig:linear} presents a standard sequent calculus for
linear logic, in particular the so-called {\it multiplicative,
  exponential} fragment of intuitionistic linear logic (or {\it
  MELL}). It corresponds most closely to Barber's dual intuitionistic
linear logic \cite{barber96dual}, but also to Andreoli's dyadic system
\cite{andreoli92logic} and Chang et al.'s judgmental analysis of
intuitionistic linear logic \cite{chang03judgmental}.

\subsection{Transitions in linear logic}

The propositions of intuitionistic linear logic, and linear implication
in particular, capture a notion of state change: we can {\it
  transition} from a state where we have both a ${\sf battery}$ and
the battery-less robot (represented, as before, by the linear
implication ${\sf battery} \lolli {\sf robot}$) to a state where we
have the battery-endowed (and therefore presumably functional) robot
(represented by the proposition ${\sf robot}$). In other words, the
proposition
%
\[{\sf battery} \otimes ({\sf battery} \lolli {\sf robot}) \lolli
{\sf robot}\] 
%
is provable in linear logic. These transitions can be chained
together as well: if we start out with ${\sf
  6bucks}$ instead of ${\sf battery}$ but we also have the
persistent ability to turn ${\sf 6bucks}$ into a ${\sf battery}$ --
just like we turned \$6 into a bowl of soup and a salad at Panera --
then we can ultimately get our working robot as well.
Written as a series of transitions, the picture looks like this:
\[
\begin{array}{ccccc}
\begin{array}{c}
\mbox{\it \$6 (1)}\medskip\\ 
\mbox{\it battery-free robot (1)} \medskip\\ 
\mbox{\it turn \$6 into a battery}\\
\mbox{\it (all you want)}
\end{array}
& \leadsto &
\begin{array}{c}
\mbox{\it battery  (1)}\medskip\\ 
\mbox{\it battery-free robot (1)} \medskip\\ 
\mbox{\it turn \$6 into a battery}\\
\mbox{\it (all you want)}
\end{array}
& \leadsto &
\begin{array}{c}
\mbox{\it robot (1)} \medskip\\ 
\mbox{\it turn \$6 into a battery}\\
\mbox{\it (all you want)}\medskip\\~\\
\end{array}
\end{array}
\]
In linear logic, these transitions correspond to the provability
of the proposition
\[{!}({\sf 6bucks} \lolli {\sf battery}) \otimes {\sf 6bucks} \otimes
({\sf battery} \lolli {\sf robot}) \lolli {\sf robot}.\] 
A derivation of this proposition is given in
Figure~\ref{fig:unfocused-robot}.\footnote{In Chapter 4, I will
  argue that this view isn't quite precise enough, and that the most
  natural representation of state change from the state $A$ to the
  state $B$ isn't really captured by derivations of the proposition $A
  \lolli B$ or by derivations of the sequent
  $\seq{\cdot}{A}{B}$.  However, this view remains a simple and useful
  one; Cervesato and Scedrov cover it thoroughly in the context of
  intuitionistic linear logic \cite{cervesato09relating}.}  

\begin{figure}
\[
\infer[{\lolli}_R]
{\seq{\cdot}{\cdot}{{!}({\sf 6bucks} \lolli {\sf battery}) \otimes
                    {\sf 6bucks} \otimes 
                    ({\sf battery} \lolli {\sf robot}) \lolli {\sf robot}}}
{\infer[{\otimes}_L]
{\seq{\cdot}{{!}({\sf 6bucks} \lolli {\sf battery}) \otimes
                    {\sf 6bucks} \otimes 
                    ({\sf battery} \lolli {\sf robot})}{{\sf robot}}}
{\infer[{!}_L]
{\seq{\cdot}{{!}({\sf 6bucks} \lolli {\sf battery}),
                    {\sf 6bucks} \otimes 
                    ({\sf battery} \lolli {\sf robot})}{{\sf robot}}}
{\infer[{\otimes}_L]
{\seq{\Gamma}{{\sf 6bucks} \otimes 
                    ({\sf battery} \lolli {\sf robot})}{{\sf robot}}}
{\infer[{\lolli}_L]
{\seq{\Gamma}{{\sf 6bucks}, {\sf battery} \lolli {\sf robot}}{{\sf robot}}}
{\infer[{\it copy}]
 {\seq{\Gamma}{{\sf 6bucks}}{{\sf battery}}}
 {\infer[{\lolli}_L] 
  {\seq{\Gamma}{{\sf 6bucks}, {\sf 6bucks} \lolli {\sf battery}}{{\sf battery}}}
  {\infer[{\it init}]
   {\seq{\Gamma}{{\sf 6bucks}}{{\sf 6bucks}}}
   {}
   &
   \infer[{\it init}]
   {\seq{\Gamma}{{\sf battery}}{{\sf battery}}}
   {}}}
 &
 \infer[{\it init}]
 {\seq{\Gamma}{{\sf robot}}{{\sf robot}}}
 {}}}}}}
\] 
\caption{Proving that a transition is possible 
(where we let $\Gamma = {\sf 6bucks} \lolli {\sf battery}$)}
\label{fig:unfocused-robot}
\end{figure}


It is precisely because linear logic contains this natural notion of
state and state transition that a rich line of work, dating back to
Chirmar's 1995 Ph.D. thesis, has sought to use linear logic as a {\it
  logical framework} for describing stateful systems
\cite{chirimar95proof,cervesato02linear,
  cervesato02concurrent,pfenning04substructural,miller09formalizing,
  pfenning09substructural,cervesato09relating}.  

\section{Logical frameworks}
\label{sec:linlogicalframeworks}

Generally speaking, logical frameworks use the {\it structure} of
proofs in a logic (like linear logic) to describe the structures we're
really interested in (like the process of obtaining a robot).  There
are two related reasons why linear logic as described in
Figure~\ref{fig:linear} is not immediately useful as a logical
framework. First, the structure of the derivation in
Figure~\ref{fig:unfocused-robot} doesn't really match the intuitive
two-step transition that we sketched out above. Second, there are {\it
  lots} of derivations of our example proposition according to the
rules in Figure~\ref{fig:linear}, even though there's only one
``real'' series of transitions that get us to a working robot. The use
of ${!}L$, for instance, could be permuted up past the ${\otimes}L$
and then past the ${\lolli}L$ into the left branch of the proof. These
differences represent inessential nondeterminism in proof construction
or in proof search -- they just get in the way of the structure that
we are trying to capture. 

This is a general problem in the construction of logical frameworks.
We'll discuss two solutions in the context of LF, a logical
framework based on dependent type theory that has proved to be a
suitable means of encoding a wide variety of deductive systems, such
as logics and programming languages \cite{harper93framework}.  The
first solution is to define an appropriate equivalence class of
proofs, and the second solution is to define a complete set
of canonical proofs.

Using an appropriate equivalence class of proofs can be an effective
way of defining away the problem of inessential nondeterminism.  In
linear logic as presented above, if the permutability of rules like
${!}_L$ and ${\otimes}_L$ is problematic, we can instead reason about
{\it equivalence classes} of derivations where proofs that differ only
in the ordering of ${!}_L$ and ${\otimes}_L$ rules are treated as
equivalent (that is, as members of the same equivalence class):
\[
\infer[{!}_L]
{\seq{\Gamma}{\Delta, {!}A, B \otimes C}{D}}
{\infer[{\otimes}_L]
 {\seq{\Gamma,A}{\Delta, B \otimes C}{D}}
 {\deduce{\seq{\Gamma,A}{\Delta, B, C}{D}}{\mathcal D}}}
\quad
\deduce{\mathstrut}{\mathstrut{\equiv}}
\quad
\infer[{\otimes}_L]
{\seq{\Gamma}{\Delta, {!}A, B \otimes C}{D}}
{\infer[{!}_L]
 {\seq{\Gamma}{\Delta, {!}A, B, C}{D}}
 {\deduce{\seq{\Gamma,A}{\Delta, B, C}{D}}{\mathcal D}}}
\]

In LF, lambda calculus terms (which correspond to derivations by the
Curry-Howard correspondence) are considered modulo the least
equivalence class that includes
\begin{itemize}
\item $\alpha$-equivalence ($\lambda x.N \equiv \lambda y.N[y/x]$ if 
$y \not\in {\it FV}(N)$), 
\item $\beta$-equivalence 
($(\lambda x.\,M)N \equiv M[N/x]$ if $x \not\in {\it FV}(N)$), and 
\item $\eta$-equivalence ($N \equiv \lambda x.N\,x$).
\end{itemize}
The weak normalization property for LF establishes that, given any
typed LF term, we can find an equivalent term that is $\beta$-normal
(no $\beta$-redexes of the form $(\lambda x.M) N$ exist) and
$\eta$-long (replacing $N$ with $\lambda x.N\,x$ anywhere would
introduce a $\beta$-redex or make the term ill-typed).  Furthermore,
in any given equivalence class of typed LF terms, all the
$\beta$-normal and $\eta$-long terms are $\alpha$-equivalent.
Therefore, because $\alpha$-equivalence is decidable, the equivalence
of typed LF terms is also decidable. 

The uniqueness of $\beta$-normal and $\eta$-long terms within an
equivalence class of lambda calculus terms (modulo
$\alpha$-equivalence, which we will henceforth take for granted) makes
these terms useful as canonical representatives of equivalence
classes. In Harper, Honsell, and Plotkin's original formulation
of LF, a deductive system was said to be {\it adequately encoded} as
an LF type family in the case that there is a compositional bijection
between the formal objects in the deductive system and these
$\beta$-normal, $\eta$-long representatives of equivalence classes
\cite{harper93framework}.

More modern presentations of LF, such as Harper and Licata's
\cite{harper07mechanizing}, follow the approach developed by Watkins
et al.~\cite{watkins02concurrent} and define the logical framework so
that it only contains these $\beta$-normal, $\eta$-long {\it canonical
  forms} of LF. This presentation of LF is called Canonical LF to
distinguish it from the original presentation of LF in which the
$\beta$-normal, $\eta$-long terms are just a subset of the possible
terms. A central component in this approach is {\it hereditary
  substitution}.  Hereditary substitution also establishes a
normalization property for LF; using hereditary substitution we can
easily take a regular LF term and transform it into a Canonical LF
term.\footnote{This process is the same as the way we use cut
  admissibility to prove cut elimination.} An oft-overlooked point,
which we will return to in Section~\ref{sec:warning}, is that the
normalization theorem we prove this way is a strictly weaker theorem
than so-called weak normalization.

Our analogue to the canonical forms of LF will be the {\it focused
  derivations} of linear logic that are presented in the next
section. In Section~\ref{sec:foclinlog} below, we will present 
focused linear logic and see that there is exactly 
one focused derivation that derives the proposition
\[{!}({\sf 6bucks} \lolli {\sf battery}) \otimes {\sf 6bucks} \otimes
({\sf battery} \lolli {\sf robot}) \lolli {\sf robot}.\] 
%
We will furthermore see that the structure of this derivation matches
the intuitive transition interpretation of the proposition, a point
that is reinforced by the discussion of {\it synthetic inference
  rules} in Section~\ref{sec:linsynthetic}. 

\section{Focused linear logic}
\label{sec:foclinlog}

Andreoli's original motivation for introducing focusing was not to
describe a logical framework, it was to describe a 
foundation for logic
programming based on proof search in classical linear logic
\cite{andreoli92logic}. The existence of multiple proofs that differ
in inessential ways is particularly problematic for proof search, as
inessential differences between derivations correspond to unnecessary
choice points that a proof search procedure will need to backtrack
over. The presentation of focusing for intuitionistic linear logic in
this section most closely resembles Chaudhuri's focused intuitionistic
linear logic \cite{chaudhuri06focused} and my presentation of
polarized intuitionistic persistent logic
\cite{simmons11structural}. The major exception is the treatment of
asynchronous rules as confluent rather than fixed and arbitrary
(discussed in Section~\ref{sec:confluent-v-fixed}).

\subsection{Polarization}
\label{sec:linpolar}

The first step in describing a focused sequent calculus is to classify
connectives into two groups.  Some connectives, such as linear
implication $A \lolli B$, are called {\it asynchronous} because their
right rules can always be applied eagerly, without backtracking,
during bottom-up proof search. Other connectives, such as disjunction
$A \tensor B$, are called {\it synchronous} because their right rules
cannot be applied eagerly. For instance, if we are trying to prove
$\seq{\Gamma}{A \tensor B}{B \tensor A}$, the ${\tensor}R$ rule cannot
be applied eagerly; we first have to decompose $A \tensor B$ on the
left using the ${\tensor}L$ rule.\footnote{Andreoli dealt with a
  one-sided classical sequent calculus; in intuitionistic logics, it
  is common to call asynchronous connectives {\it right}-asynchronous
  and {\it left}-synchronous. Similarly, it is common to call
  synchronous connectives {\it right}-synchronous and {\it
    left}-asynchronous.

  Synchronicity, a property of connectives, is closely connected to
  (and sometimes conflated with) a property of rules called {\it
    invertibility}; a rule is invertible if the conclusion of the rule
  implies the premises. So ${\lolli}R$ is invertible
  ($\seq{\Gamma}{\Delta}{A \lolli B}$ implies $\seq{\Gamma}{\Delta,
    A}{B}$) but ${\lolli}L$ is not ($\seq{\Gamma}{\Delta, A \lolli
    B}{C}$ does not imply that $\Delta = \Delta_1, \Delta_2$ such that
  $\seq{\Gamma}{\Delta_1}{A}$ and $\seq{\Gamma}{\Delta_2, B}{C}$).
  Rules that can be applied eagerly need to be invertible, so
  asynchronous connectives have invertible right rules and synchronous
  connectives have invertible left rules. Therefore, another synonym
  for asynchronous/negative is {\it right-invertible}, and another
  synonym for synchronous/positive is {\it left-invertible}.}  
The nontrivial result of focusing is that it is possible to separate a
proof into phases: inversion phases in which all asynchronous rules
are applied exhaustively, and focused phases where synchronous rules
are applied repeatedly and exhaustively to a single proposition (the
proposition {\it in focus}). 

We call the asynchronous connectives {\it negative} ($\lolli$, $\top$
and $\with$ in full propositional linear logic) and call the
synchronous connectives {\it positive} ($\zero$, $\oplus$, $\one$, and
$\otimes$ in full propositional linear logic). Each atomic proposition
must be assigned to be either positive or negative, though this
assignment can be arbitrary. At this point, there is an important
choice to make. One way forward is to treat positive and negative
propositions as a syntactic refinements of the set of all
propositions, in which case we end up focusing a standard
intuitionistic linear logic. The other way forward is to treat
positive and negative propositions as distinct syntactic classes $A^+$
and $A^-$ with explicit inclusions, called {\it shifts}, between
them. In this second case, we end up focusing a {\it polarized} linear
logic.  The positive proposition ${\downarrow}A^-$, pronounced
``downshift $A$,'' has a subterm that is a negative proposition; the
negative proposition ${\uparrow}A^+$, pronounced ``upshift $A$,'' has
a subterm that is a positive proposition.

The choice ultimately doesn't make an enormous difference for our
purposes.  Polarized logics are interesting, and polarized linear
logic is a bit more expressive than regular linear logic, as
heavily-shifted propositions like ${\downarrow}{\uparrow}A^+$ and
${\uparrow}{\downarrow}{\uparrow}{\downarrow}A^-$ can be
expressed. This extra expressiveness won't help us in the design of
logical frameworks, but the use of shifts is helpful when explaining
identity expansion in Section~\ref{sec:linindentity} and completeness
in Section~\ref{sec:lincorrectness}, so we will describe focusing
for a polarized linear logic with shifts.

\begin{figure}
{\small \[
\begin{array}{rcl|rcl|rcl}
({\downarrow}A^-)^\circ & \!\!\!=\!\!\! & (A^-)^\circ & & & & & & 
\\
(p^+)^\circ & \!\!\!=\!\!\! & p^+ &
(p^+)^\oplus & \!\!\!=\!\!\! & p^+ &
(p^+)^\ominus & \!\!\!=\!\!\! & {\uparrow}p^+
\\
({!}A^-)^\circ & \!\!\!=\!\!\! & {!}(A^-)^\circ &
({!}A)^\oplus & \!\!\!=\!\!\! & {!}A^\ominus &
({!}A)^\ominus & \!\!\!=\!\!\! & {\uparrow}({!}A^\ominus)
\\
(\one)^\circ & \!\!\!=\!\!\! & \one &
(\one)^\oplus & \!\!\!=\!\!\! & \one &
(\one)^\ominus & \!\!\!=\!\!\! & {\uparrow}\one 
\\
(A^+ \otimes B^+)^\circ & \!\!\!=\!\!\! & (A^+)^\circ \otimes (B^+)^\circ &
(A \otimes B)^\oplus & \!\!\!=\!\!\! & A^\oplus \otimes B^\oplus &
(A \otimes B)^\ominus & \!\!\!=\!\!\! & {\uparrow}(A^\oplus \otimes B^\oplus)
\\
({\uparrow}A^+)^\circ & \!\!\!=\!\!\! & (A^+)^\circ & & & & & & 
\\
(p^-)^\circ & \!\!\!=\!\!\! & p^- &
(p^-)^\oplus & \!\!\!=\!\!\! & {\downarrow}p^- &
(p^-)^\ominus & \!\!\!=\!\!\! & p^- 
\\
(A^+ \lolli B^-)^\circ & \!\!\!=\!\!\! & (A^+)^\circ \lolli (B^-)^\circ &
(A \lolli B)^\oplus & \!\!\!=\!\!\! & {\downarrow}(A^\oplus \lolli B^\ominus) &
(A \lolli B)^\ominus & \!\!\!=\!\!\! & A^\oplus \lolli B^\ominus
\end{array}
\]}

\caption{De-polarizing and polarizing (with minimal shifts) propositions of MELL}
\label{fig:lin-shift}
\end{figure}


The relationship between unpolarized and polarized linear logic is
given by two erasure functions $(A^+)^\circ$ and $(A^-)^\circ$ that
wipe away all the shifts; this function is defined in
Figure~\ref{fig:lin-shift}. In the other direction, every 
proposition in unpolarized linear logic has an obvious polarized
analouge with a minimal number of shifts, which is given by by
the functions $A^\oplus$ and $A^\ominus$ in Figure~\ref{fig:lin-shift}.
Both of these functions are partial inverses of erasure, since
$(A^\oplus)^\circ = (A^\ominus)^\circ = A$; we will generally refer
to partial inverses of erasure as {\it polarization strategies}. 

While shifts turn out to have a profound impact on the structure of
focused proofs, they are intended to have no impact on
provability. Therefore, the strongest statement of the correctness of
focusing is based on erasure: there is an unfocused derivation of
$(A^+)^\circ$ or $(A^-)^\circ$ if and only if there is a focused
derivation of $A^+$ or $A^-$.  However, existing proofs of
focalization are of a weaker property: that there is an unfocused
derivation of $A$ if and only if there is a focused derivation of
$A^\bullet$, where $A^\bullet$ is some polarization strategy.  The
only two exceptions I am aware of are Zeilberger's completeness proof
of classical persistent logic \cite{zeilberger08unity} and my proof in
intuitionistic persistent logic \cite{simmons11structural}); I will
follow the latter development in this section.

\subsection{Focused sequent calculus}

Usually, focused logics are described as having multiple
sequent forms. For intuitionistic logics, there are at least three
sequent forms:
\begin{itemize}
\item $\mildrfoc{\Gamma}{\Delta}{A^+}$ (the {\it right focus} sequent, where
the proposition $A^+$ is in focus),
\item $\mildinv{\Gamma}{\Delta}{C}$ (the {\it inversion} sequent), and
\item $\mildlfoc{\Gamma}{\Delta}{A^-}{C}$ (the {\it left focus} sequent,
where the proposition $A^-$ is in focus).
\end{itemize}
Another reasonable presentation of linear logic, and the one we will
adopt in this section, uses only one sequent form
$\mildseq{\Gamma}{\Delta}{U}$, but generalizes what is to allowed to
to appear in the linear context $\Delta$ or in the succeedant $U$. We
will use this interpretation to understand the logic described in
Figure~\ref{fig:kaustuv-focused}.

\begin{figure}[t]
\begin{tabbing}
\quad $A^+$ \= $::= p^+ 
              \mid {\downarrow}A^- 
              \mid {!}A^- 
              \mid \one
              \mid A \otimes B$\\
\quad $A^-$ \> $::= p^-
              \mid {\uparrow}A^+
              \mid A \lolli B$\\
\quad $\Gamma$ \> $::= \cdot \mid \Gamma, A^-$ \qquad\qquad\qquad\qquad\qquad\qquad\quad \= {\it (multiset)}\\
\quad $\Delta$ \> $::= \cdot \mid \Delta, A^+ \mid \Delta, A^- \mid \Delta, [A^-] \mid \Delta, \langle A^+ \rangle$ \> {\it (multiset)}\\
\quad $U$ \> $::= A^- \mid A^+ \mid [ A^+ ] \mid \langle A^- \rangle$\\
\end{tabbing}
%
%
\quad \fbox{$\mildseq{\Gamma}{\Delta}{U}$}
\[
\infer[{\it focus}^*_R]
{\mildseq{\Gamma}{\Delta}{A^+}}
{\mildseq{\Gamma}{\Delta}{[A^+]}}
\quad
\infer[{\it focus}^*_L]
{\mildseq{\Gamma}{\Delta,A^-}{U}}
{\mildseq{\Gamma}{\Delta,[A^-]}{U}}
\quad
\infer[{\it copy}^*]
{\mildseq{\Gamma, A^-}{\Delta}{U}}
{\mildseq{\Gamma, A^-}{\Delta, [A^-]}{U}}
\]

\[
\infer[\eta^+]
{\mildseq{\Gamma}{\Delta, p^+}{U}}
{\mildseq{\Gamma}{\Delta, \langle p^+ \rangle}{U}}
\quad
\infer[{\it id}^+]
{\mildseq{\Gamma}{\langle A^+ \rangle}{[A^+]}}
{}
\quad
\infer[\eta^-]
{\mildseq{\Gamma}{\Delta}{p^-}}
{\mildseq{\Gamma}{\Delta}{\langle p^- \rangle}}
\quad
\infer[{\it id}^-]
{\mildseq{\Gamma}{[A^-]}{\langle A^- \rangle}}
{}
\]

\[
\infer[{\uparrow}_R]
{\mildseq{\Gamma}{\Delta}{{\uparrow}A^+}}
{\mildseq{\Gamma}{\Delta}{A^+}}
\quad
\infer[{\uparrow}_L]
{\mildseq{\Gamma}{\Delta, [{\uparrow}A^+]}{U}}
{\mildseq{\Gamma}{\Delta, A^+}{U}}
\quad
\infer[{\downarrow}_R]
{\mildseq{\Gamma}{\Delta}{[{\downarrow}A^-]}}
{\mildseq{\Gamma}{\Delta}{A^-}}
\quad
\infer[{\downarrow}_L]
{\mildseq{\Gamma}{\Delta, {\downarrow}A^-}{U}}
{\mildseq{\Gamma}{\Delta, A^-}{U}}
\]

\[
%
\infer[{!}_R]
{\mildseq{\Gamma}{\cdot}{[{!}A^-]}}
{\mildseq{\Gamma}{\cdot}{A^-}}
\quad
\infer[{!}_L]
{\mildseq{\Gamma}{\Delta, {!}A^-}{U}}
{\mildseq{\Gamma, A^-}{\Delta}{U}}
\quad
\infer[\one_R]
{\mildseq{\Gamma}{\cdot}{[\one]}}
{}
\quad
\infer[\one_L]
{\mildseq{\Gamma}{\Delta, \one}{U}}
{\mildseq{\Gamma}{\Delta}{U}}
\]

\[
%
\infer[{\tensor}_R]
{\mildseq{\Gamma}{\Delta_1,\Delta_2}{[A^+ \tensor B^+]}}
{\mildseq{\Gamma}{\Delta_1}{[A^+]}
 &
 \mildseq{\Gamma}{\Delta_2}{[B^+]}}
\quad
\infer[{\tensor}_L]
{\mildseq{\Gamma}{\Delta, A^+ \tensor B^+}{U}}
{\mildseq{\Gamma}{\Delta, A^+, B^+}{U}}
\]

\[
%
\infer[{\lolli}_R]
{\mildseq{\Gamma}{\Delta}{A^+ \lolli B^-}}
{\mildseq{\Gamma}{\Delta, A^+}{B^-}}
\quad
\infer[{\lolli}_L]
{\mildseq{\Gamma}{\Delta_1,\Delta_2, [A^+ \lolli B^-]}{U}}
{\mildseq{\Gamma}{\Delta_1}{[A^+]}
 &
 \mildseq{\Gamma}{\Delta_2, [B^-]}{U}}
%
\]
\caption{Focused intuitionstic linear logic.}
\label{fig:kaustuv-focused}
\end{figure}


By adding a side condition to the three rules ${\it focus}_R$, ${\it
  focus}_L$, and ${\it copy}$ that neither the context $\Delta$ nor
the succeedant $U$ can contain an in-focus proposition $[A^+]$ or
$[A^-]$, derivations can maintain the invariant that there is always
at most one proposition in focus, effectively restoring the situation
in which there are three distinct judgments. From this point on, we
will only consider sequents with at most one focus; we will write
$\underline{\Delta}$ or $\underline{U}$ to indicate when $\Delta$ may
contain a focus $[A^-]$ and when $U$ may be a focus $[A^+]$, though
the sequent form $\mildseq{\Gamma}{\underline{\Delta}}{\underline{U}}$
is still restricted to contain at most one focus.  Pfenning, who
developed this construction in \cite{pfenning12chaining}, calls this
the {\it focusing constraint}, and writes $\delta$ and $\gamma$
instead of $\underline{\Delta}$ and $\underline{U}$.

The focusing constraint alone
gives us what Pfenning calls a {\it chaining} logic
\cite{pfenning12chaining} and which Laurent calls a {\it weakly
  focused} logic \cite{laurent04proof}.\footnote{This is not what I
  called a weakly focused logic \cite{simmons09weak}. That weakly
  focused system had an additional restriction that invertible rules
  could not be applied when any other proposition was in focus; this
  corresponded to what Laurent called a strongly $+$-focused logic.
  Oops.}
We obtain a fully focused logic by further restricting these three
rules so that they only apply when the sequent below the line is {\it
  stable}.  A sequent $\mildseq{\Gamma}{\Delta}{U}$ is stable if the
context $\Delta$ contains only negative propositions $A^-$ and
suspended positive propositions $\langle A^+ \rangle$ and the
succeedant $U$ is either a positive proposition $A^+$ or a suspended
negative propositions $\langle A^- \rangle$. In light of this restriction,
whenever we consider a focused sequent
$\mildseq{\Gamma}{\Delta, [A^-]}{U}$ or 
$\mildseq{\Gamma}{\Delta}{[A^+]}$, we can assume that $\Delta$ and $U$ 
are stable.

We will now turn our attention to the meaning of these suspended
propositions and the four rules that interact with them: ${\it id}^+$,
${\it id}^-$, $\eta^+$, and $\eta^-$.

\subsection{Suspended propositions}

In unfocused sequent calculi, such as the one for linear logic in
Figure~\ref{fig:linear}, initial sequents are restricted to atomic
propositions. All sequent calculi, focused or unfocused, have the
subformula property: every rule breaks down a proposition, either on
the left or the right of the turnstile ``$\vdash$'', 
when read from bottom to top. 
Since the logical interpretation of atomic
propositions is that they are stand-ins for unknown propositions, we
are unable to break them down any further. We are therefore only able
to derive an atomic conclusion or use an atomic premise with the {\it
  init} rule that concludes $\seq{\Gamma}{p}{p}$ and has no premises.
While the {\it init} rule must be explicitly included as a proof rule,
it is an instance of an admissible identity theorem:
$\seq{\Gamma}{A}{A}$ for all $A$. If we substitute a concrete
proposition for some atomic proposition, the structure of the proof
stays exactly the same, except that instances of initial sequents
become admissible instances of the identity theorem.

To my knowledge, all published proof systems for focused logic have
attempted to replicate this initial rule {\it init}. I believe this to
be a design error, and it is one that has historically made it
enormously (and unnecessarily) difficult to prove the identity theorem
for focused systems. This presentation uses {\it suspensions}:
suspended positive propositions $\langle A^+ \rangle$ only appear in
the linear context $\Delta$, and suspended negative propositions
$\langle A^- \rangle$ only appear as succeedants. They are treated as
stable (we never break down a suspended proposition) and are only used
to immediately prove a proposition in focus with one of the identity
rules ${\it id}^+$ or ${\it id}^-$.

\paragraph{Suspended positive propositions} act much like regular variables in a
natural deduction system. The positive identity rule ${\it id}^+$
allows us to prove any positive proposition given that the positive
proposition appears suspended in the context.  There is a
corresponding substitution principle for focal substitutions that has
a natural-deduction-like flavor: we can substitute a derivation
right-focused on $A^+$ for a suspended positive proposition $\langle
A^+ \rangle$ in a context.

\bigskip
\begin{theorem}[Focal substitution (positive)]\label{thm:fsubst-pos}~\\
If $\mildseq{\Gamma}{\Delta}{[A^+]}$ 
and $\mildseq{\Gamma}{\underline{\Delta'}, \langle A^+ \rangle}
      {\underline{U}}$, 
then $\mildseq{\Gamma}{\underline{\Delta'}, \Delta}{\underline{U}}$.
\end{theorem}

\begin{proof}
  Straightforward induction over the second given derivation, as in a
  proof of regular substitution in a natural deduction system. If the
  second derivation is the axiom ${\it id}^+$, the result follows
  immediately using the first given derivation.
\end{proof}

\noindent
Recall that, in the statement of Theorem~\ref{thm:fsubst-pos}, we assume
that $\Delta$ is stable by virtue of it appearing in the 
focused sequent $\mildseq{\Gamma}{\Delta}{[A^+]}$, and the
second premise 
$\mildseq{\Gamma}{\underline{\Delta'}, \langle A^+ \rangle}{\underline{U}}$ may
be a right-focused sequent $\mildseq{\Gamma}{\Delta', \langle A^+
  \rangle}{[B^+]}$, a left-focused sequent $\mildseq{\Gamma}{\Delta'',
  [B^-], \langle A^+ \rangle}{U}$, or an inverting sequent. 

\paragraph{Suspended negative propositions} are a bit weirder. While a derivation
of $\mildseq{\Gamma}{\underline{\Delta'}, \langle A^+ \rangle}{\underline{U}}$
is missing a premise that can be satisfied by a derivation of
$\mildseq{\Gamma}{\Delta}{[A^+]}$, a derivation of 
$\mildseq{\Gamma}{\underline{\Delta}}{\langle A^- \rangle}$ is missing a 
{\it continuation} that can be satisfied by a derivation of
$\mildseq{\Gamma}{\Delta', [A^-]}{U}$. The focal substitution principle,
however, still takes the basic form of a substitution principle.

\bigskip
\begin{theorem}[Focal substitution (negative)]\label{thm:fsubst-neg}~\\
If $\mildseq{\Gamma}{\underline{\Delta}}{\langle A^- \rangle}$
and $\mildseq{\Gamma}{\Delta', [A^-]}{U}$, 
then $\mildseq{\Gamma}{\Delta', \underline{\Delta}}{U}$. 
\end{theorem}

\begin{proof}
  Straightforward induction over the {\it first} given derivation; if
  the first derivation is the axiom ${\it id}^-$, the result follows
  immediately using the second given derivation.
\end{proof}

\noindent
As a regular substitution principle that is inductive over the structure
of the first given proposition, focal substitution is reminiscent of 
the {\it leftist substitutions} introduced by Pfenning and Davies in the 
context of the possibility modality \cite{pfenning01judgmental}.

Unlike cut admissibility, which we discuss in
Section~\ref{sec:lincut}, both of the focal substitution principles
are straightforward inductions over the structure of the derivation
containing the suspended proposition. In the development of structural
focalization, I discuss how, in a focused presentation of persistent
intuitionistic logic that is encoded in LF, a suspended positive
premise can be encoded as a hypothetical right focus. This encoding
makes the ${\it id}^+$ rule an instance of the hypothesis rule
provided by LF and establishes Theorem~\ref{thm:fsubst-pos} ``for
free'' as an instance of LF substitution. This is possible to do for
negative focal substitution as well, but it is somewhat
counterintuitive and relies on a very higher-order use of LF's uniform
function space \cite{simmons11structural}.

The two substitution
principles can be phrased as admissible rules for building derivations,
which we indicate using a dashed line:
\[
\infer-[{\it subst}^+]
{\mildseq{\Gamma}{\underline{\Delta'}, \Delta}{\underline{U}}}
{\mildseq{\Gamma}{\Delta}{[A^+]}
 &
 \mildseq{\Gamma}{\underline{\Delta'}, \langle A^+ \rangle}{\underline{U}}}
\qquad
\infer-[{\it subst}^-]
{\mildseq{\Gamma}{\Delta', \underline{\Delta}}{U}}
{\mildseq{\Gamma}{\underline{\Delta}}{\langle A^- \rangle}
 &
 \mildseq{\Gamma}{\Delta', [A^-]}{U}}
\]

\subsection{Identity expansion}
\label{sec:linindentity}

Suspended propositions appear in Figure~\ref{fig:kaustuv-focused} in
two places: in the identity rules, which we have just discussed
and connected with the focal substitution principles, and in
the rules marked $\eta^+$ and $\eta^-$, which are also the only
mention of atomic propositions in the presentation. It is here that we
need to make an absolutely critical shift of perspective from
unfocused to focused logic. In an unfocused logic, the rules
nondeterministically break down propositions, and the initial rule
{\it init} puts an end to this process when an atomic proposition is
reached. In a focused logic, the focus and inversion phases must break
down a proposition {\it all the way} until a shift is reached. The two
$\eta$ rules are what put an end to this when an atomic proposition is
reached, and they work with to the two ${\it id}$ rules that allow
these necessarily suspended propositions to successfully conclude a
right or left focus.

\begin{figure}
{\small 
\[
\infer-[\eta^+]
{\mildseq{\Gamma}{\Delta, {\downarrow}A}{U}}
{\deduce
 {\mildseq{\Gamma}{\Delta, \langle {\downarrow}A \rangle}{U}}
 {\mathcal D}}
\quad
\deduce{\mathstrut}{\Longrightarrow}
\quad
\infer[{\downarrow}_L]
{\mildseq{\Gamma}{\Delta, {\downarrow}A}{U}}
{\infer-[{\it subst}^+]
 {\mildseq{\Gamma}{\Delta, A}{U}}
 {\infer[{\downarrow}_R]
  {\mildseq{\Gamma}{A}{[ {\downarrow}A ]}}
  {\infer-[{\eta}^-]
   {\mildseq{\Gamma}{A}{A}}
   {\infer-[{\it focus}_L]
    {\mildseq{\Gamma}{A}{\langle A \rangle}}
    {\infer[{\it id}^-]
     {\mildseq{\Gamma}{[ A ]}{\langle A \rangle}}
     {}}}}
  &
  \deduce
  {\mildseq{\Gamma}{\Delta, \langle {\downarrow}A \rangle}{U}}
  {\mathcal D}}}
\]

\[
\infer-[\eta^+]
{\mildseq{\Gamma}{\Delta,{!}A}{U}}
{\deduce
 {\mildseq{\Gamma}{\Delta, \langle {!}A \rangle}{U}}
 {\mathcal D}}
\quad
\deduce{\mathstrut}{\Longrightarrow}
\infer[{!}_L]
{\mildseq{\Gamma}{\Delta,{!}A}{U}}
{\infer-[{\it subst}^+]
 {\mildseq{\Gamma,A}{\Delta}{U}}
 {\infer[{!}_R]
  {\mildseq{\Gamma, A}{\cdot}{[{!}A]}}
  {\infer-[\eta^-]
   {\mildseq{\Gamma, A}{\cdot}{A}}
   {\infer[\it copy]
    {\mildseq{\Gamma, A}{\cdot}{\langle A \rangle}}
    {\infer[{\it id}^-]
     {\mildseq{\Gamma, A}{[ A ]}{\langle A \rangle}}
     {}}}}
  &
  \infer-[{\it weaken}]
  {\mildseq{\Gamma,A}{\Delta, \langle {!}A \rangle}{U}}
  {\deduce
   {\mildseq{\Gamma}{\Delta, \langle {!}A \rangle}{U}}
   {\mathcal D}}}}
\]

\[
\infer-[\eta^+]
{\mildseq{\Gamma}{\Delta, A \otimes B}{U}}
{\deduce{\mildseq{\Gamma}{\Delta, \langle A \otimes B \rangle}{U}}{\mathcal D}}
\quad
\deduce{\mathstrut}{\Longrightarrow}
\!\!\!\!\!\!\!\!\!
\infer[{\otimes}_L]
{\mildseq{\Gamma}{\Delta, A \otimes B}{U}}
{\infer-[\eta^+]
 {\mildseq{\Gamma}{\Delta, A, B}{U}}
 {\infer-[\eta^+]
 {\mildseq{\Gamma}{\Delta, \langle A \rangle, B}{U}}
 {\infer-[{\it subst}^+]
  {\mildseq{\Gamma}{\Delta, \langle A \rangle, \langle B \rangle}{U}}
  {\infer
   {\mildseq{\Gamma}{\langle A \rangle, \langle B \rangle}{[A \otimes B]}}
   {\infer[{\it id}^+]
    {\mildseq{\Gamma}{\langle A \rangle}{[A]}}
    {}
    & 
    \infer[{\it id}^+]
    {\mildseq{\Gamma}{\langle B \rangle}{[B]}}
    {}}
   & 
   \deduce
   {\mildseq{\Gamma}{\Delta, \langle A \otimes B \rangle}{U}}
   {\mathcal D}}}}}
\]

\[
\infer-[\eta^-]
{\mildseq{\Gamma}{\Delta}{{\uparrow}A}}
{\deduce
 {\mildseq{\Gamma}{\Delta}{\langle {\uparrow}A \rangle}}
 {\mathcal D}}
\quad
\deduce{\mathstrut}{\Longrightarrow}
\quad
\infer[{\uparrow}_R]
{\mildseq{\Gamma}{\Delta}{{\uparrow}A}}
{\infer-[{\it subst}^-]
 {\mildseq{\Gamma}{\Delta}{A}}
 {\deduce
  {\mildseq{\Gamma}{\Delta}{\langle {\uparrow}A \rangle}}
  {\mathcal D}
  &
  \infer[{\uparrow}_L]
  {\mildseq{\Gamma}{[{\uparrow}A]}{A}}
  {\infer-[\eta^+]
   {\mildseq{\Gamma}{A}{A}}
   {\infer[{\it focus}_R]
    {\mildseq{\Gamma}{\langle A \rangle}{A}} 
    {\infer[{\it id}^+]
     {\mildseq{\Gamma}{\langle A \rangle}{[ A ]}}
     {}}}}}}
\]

\[
\infer-[\eta^-]
{\mildseq{\Gamma}{\Delta}{A \lolli B}}
{\deduce
 {\mildseq{\Gamma}{\Delta}{\langle A \lolli B \rangle}}
 {\mathcal D}}
\quad
\deduce{\mathstrut}{\Longrightarrow}
\!\!\!\!\!\!
\infer[{\lolli}_R]
{\mildseq{\Gamma}{\Delta}{A \lolli B}}
{\infer-[\eta^+]
 {\mildseq{\Gamma}{\Delta, A}{B}}
 {\infer-[\eta^-]
  {\mildseq{\Gamma}{\Delta, \langle A \rangle}{B}}
  {\infer-[{\it subst}^-]
   {\mildseq{\Gamma}{\Delta, \langle A \rangle}{\langle B \rangle}}
   {\deduce
    {\mildseq{\Gamma}{\Delta}{\langle A \lolli B \rangle}}
    {\mathcal D}
    &
    \infer[{\lolli}_L]
    {\mildseq{\Gamma}{\langle A \rangle, [ A \lolli B ]}{\langle B \rangle}}
    {\infer[{\it id}^+]
     {\mildseq{\Gamma}{\langle A \rangle}{[ A ]}}
     {}
     &
     \infer[{\it id}^-]
     {\mildseq{\Gamma}{[ B ]}{\langle B \rangle}}
     {}}}}}}
\]}
\caption{Identity expansion -- restricting $\eta^+$ and $\eta^-$ to atomic 
 propositions}
\label{fig:lineta-1}
\end{figure}

\begin{figure}[t]
{\small

\[
\infer-[\eta^+]
{\mildseq{\Gamma}{\Delta, \one}{U}}
{\deduce
 {\mildseq{\Gamma}{\Delta, \langle \one \rangle}{U}}
 {\mathcal D}}
\quad
\Longrightarrow
\infer[{\one}_L]
{\mildseq{\Gamma}{\Delta, \one}{U}}
{\infer-[{\it subst}^+]
 {\mildseq{\Gamma}{\Delta}{U}}
 {\infer[{\one}_R]
  {\mildseq{\Gamma}{\cdot}{[ \one ]}}
  {}
  &
  \deduce
  {\mildseq{\Gamma}{\Delta, \langle \one \rangle}{U}}
  {\mathcal D}}}
\]

\[
\infer-[\eta^+]
{\mildseq{\Gamma}{\Delta, \zero}{U}}
{\deduce
 {\mildseq{\Gamma}{\Delta, \langle \zero \rangle}{U}}
 {\mathcal D}}
\quad
\Longrightarrow
\infer[\zero_L]
{\mildseq{\Gamma}{\Delta, \zero}{U}}
{}
\]

\[
\infer-[\eta^+]
{\mildseq{\Gamma}{\Delta, A \oplus B}{U}}
{\deduce
 {\mildseq{\Gamma}{\Delta, \langle A \oplus B \rangle}{U}}
 {\mathcal D}}
\quad
\Longrightarrow
\!\!\!\!\!\!\!\!\!\!\!\!\!\!\!\!
\infer[{\oplus}_L]
{\mildseq{\Gamma}{\Delta, A \oplus B}{U}}
{\infer-[\eta^+]
 {\mildseq{\Gamma}{\Delta, A}{U}}
 {\infer-[{\it subst}^+]
  {\mildseq{\Gamma}{\Delta, \langle A \rangle}{U}}
  {\infer[{\oplus}_{R1}]
   {\mildseq{\Gamma}{\Delta, \langle A \rangle}{[ A \oplus B ]}}
   {\infer[{\it id}^+]
    {\mildseq{\Gamma}{\langle A \rangle}{[ A ]}}
    {}}
   &
   \deduce
   {\mildseq{\Gamma}{\Delta, \langle A \oplus B \rangle}{U}}
   {\mathcal D}}}
 &
 \deduce
 {\mildseq{\Gamma}{\Delta, B}{U}}
 {\vdots}
 }
\]


\[
\infer-[\eta^-]
{\mildseq{\Gamma}{\Delta}{\top}}
{\deduce
 {\mildseq{\Gamma}{\Delta}{\langle \top \rangle}}
 {\mathcal D}}
\quad
\Longrightarrow
\quad
\infer[{\top}_R]
{\mildseq{\Gamma}{\Delta}{\top}}
{}
\]

\[
\infer-[\eta^-]
{\mildseq{\Gamma}{\Delta}{A \with B}}
{\deduce
 {\mildseq{\Gamma}{\Delta}{\langle A \with B \rangle}}
 {\mathcal D}}
\quad
\Longrightarrow
\!\!\!\!
\infer[{\with}_R]
{\mildseq{\Gamma}{\Delta}{A \with B}}
{\infer[\eta^-]
 {\mildseq{\Gamma}{\Delta}{A}}
 {\infer-[{\it subst}^-]
  {\mildseq{\Gamma}{\Delta}{\langle A \rangle}}
  {\deduce
   {\mildseq{\Gamma}{\Delta}{\langle A \with B \rangle}}
   {\mathcal D}
   &
   \infer[{\with}_{L1}]
   {\mildseq{\Gamma}{[A \with B]}{\langle A \rangle}}
   {\infer[{\it id}^-]
    {\mildseq{\Gamma}{[A]}{\langle A \rangle}}
    {}}}}
 & 
 \deduce
 {\mildseq{\Gamma}{\Delta}{B}}
 {\vdots}}
\]}

\caption{Identity expansion for units and additive connectives}
\label{fig:lineta-2}
\end{figure}


Just as the {\it init} rule is a particular instance of the admissible
identity sequent $\seq{\Gamma}{A}{A}$ in unfocused linear logic, the
atomic suspension rules $\eta^+$ and $\eta^-$ are instances of an admissible
{\it identity expansion} rule in focused linear logic:
\[
\infer-[\eta^+]
{\mildseq{\Gamma}{\Delta, A^+}{U}}
{\mildseq{\Gamma}{\Delta, \langle A^+ \rangle}{U}}
\qquad
\infer-[\eta^-]
{\mildseq{\Gamma}{\Delta}{A^-}}
{\mildseq{\Gamma}{\Delta}{\langle A^- \rangle}}
\]
Or, phrased as a theorem:

\bigskip
\begin{theorem}[Identity expansion]~
\begin{itemize}
\item 
If $\mildseq{\Gamma}{\Delta, \langle A^+ \rangle}{U}$, 
then $\mildseq{\Gamma}{\Delta, A^+}{U}$.
\item
If $\mildseq{\Gamma}{\Delta}{\langle A^- \rangle}$, 
then $\mildseq{\Gamma}{\Delta}{A^-}$.
\end{itemize}
\end{theorem}

\begin{proof}
Mutual induction over the structure of the proposition $A^+$ or $A^-$,
with a critical use of focal substitution in each case.

Most of the cases of
this proof are represented in Figure~\ref{fig:lineta-1}. (Note that in
this figure we omit polarity annotations from propositions as they are
always clear from the context.) The remaining case (for the
multiplicative unit $\one$) is presented in Figure~\ref{fig:lineta-2}
along with the cases for the additive connectives $\zero$, $\oplus$,
$\top$, and $\with$, which are neglected elsewhere in this chapter.
\end{proof}

The admissible identity expansion rules fit with an interpretation of
positive atomic propositions as stand-ins for arbitrary positive
propositions and of negative atomic propositions as stand-ins for
negative atomic propositions: if we substitute a proposition in for
some atomic proposition, all the instances of atomic suspension
corresponding to that rule become admissible instances of identity
expansion. 

The usual identity principles are 
corollaries of identity expansion:
\[
\infer-[\eta^+]
{\mildseq{\Gamma}{A^+}{A^+}}
{\infer[{\it focus}_R]
 {\mildseq{\Gamma}{\langle A^+ \rangle}{A^+}}
 {\infer[{\it id}^+]
  {\mildseq{\Gamma}{\langle A^+ \rangle}{[A^+]}}
  {}}}
\qquad
\infer-[\eta^-]
{\mildseq{\Gamma}{A^-}{A^-}}
{\infer[{\it focus}_L]
 {\mildseq{\Gamma}{A^-}{\langle A^- \rangle}}
 {\infer[{\it id}^-]
  {\mildseq{\Gamma}{[A^-]}{\langle A^- \rangle}}
  {}}}
\]

\subsection{Cut admissibility}
\label{sec:lincut}

Theorem~\ref{thm:lincut} mostly follows the well-worn contours of a
structural cut admissibility argument \cite{pfenning00structural}, so
we defer a full discussion of cut admissibility until the next
chapter, where we will give a tidier proof by incorporating more of
the machinery of structural focalization.\footnote{The main ``untidy''
  aspect of Theorem~\ref{thm:lincut} is that the lack of a forced
  inversion order means that the right commutative cases dealing with
  invertible rules must be repeated in parts 1 and 4. The repetition
  of almost all right commutative cases between parts 4 and 5 will
  also be addressed in the next chapter.}
%
The only important caveat about 
cut admissibility is that it is only applicable in the absence of any
non-atomic suspended propositions. If we did not make this
restriction, then in Theorem~\ref{thm:lincut}, part 1, we might encounter
a derivation of $\mildseq{\Gamma}{\langle A \tensor B \rangle}{[ A \tensor B ]}$
being cut into the derivation
\[
\infer[{\otimes}_R]
{\mildseq{\Gamma}{\Delta',A \tensor B}{U}}
{\deduce{\mildseq{\Gamma}{\Delta', A, B}{U}}{\mathcal E}}
\]
in which case there is no clear way to proceed and prove 
$\mildseq{\Gamma}{\Delta', \langle A \tensor B \rangle}{U}$. 

\bigskip
\begin{theorem}[Cut admissibility]\label{thm:lincut}
For all $\Gamma$, $A^+$, $A^-$, $\Delta$, $\Delta'$, and $U$ that
do not contain any non-atomic suspended propositions:
\begin{enumerate}
\item If $\mildseq{\Gamma}{\Delta}{[A^+]}$
      and $\mildseq{\Gamma}{\Delta',A^+}{U}$
      (where $\Delta$ is stable), 
      then $\mildseq{\Gamma}{\Delta',\Delta}{U}$.
\item If $\mildseq{\Gamma}{\Delta}{A^-}$
      and $\mildseq{\Gamma}{\Delta', [A^-]}{U}$
      (where $\Delta$, $\Delta'$, and $U$ are stable),
      then $\mildseq{\Gamma}{\Delta',\Delta}{U}$. 
\item If $\mildseq{\Gamma}{\underline{\Delta}}{A^+}$
      and $\mildseq{\Gamma}{\Delta', A^+}{U}$,
      (where $\Delta'$ and $U$ are stable),
      then $\mildseq{\Gamma}{\Delta',\underline{\Delta}}{U}$. 
\item If $\mildseq{\Gamma}{\Delta}{A^-}$
      and $\mildseq{\Gamma}{\underline{\Delta'}, A^-}{\underline{U}}$,
      (where $\Delta$ is stable),
      then $\mildseq{\Gamma}{\underline{\Delta'},\Delta}{\underline{U}}$. 
\item If $\mildseq{\Gamma}{\cdot}{A^-}$
      and $\mildseq{\Gamma, A^-}{\underline{\Delta'}}{\underline{U}}$,
      then $\mildseq{\Gamma}{\underline{\Delta'}}{\underline{U}}$. 
\end{enumerate}
\end{theorem}

\begin{proof}
  The structure of the proof in is detailed in
  \cite{simmons12cut}.  In each invocation of the induction
  hypothesis, either the principal cut formula $A^+$ or $A^-$ gets
  smaller or else it stays the same and the ``part size'' (1-5) gets
  smaller. When the principal cut formula and the part size remain the
  same, either the first given derivation gets smaller (part 3)
  or the second given derivation gets smaller (parts 1, 4 and 5).

  Standard structural cut admissibility proofs do not isolate the
  left commutative cuts (where the first given derivaiton gets
  smaller) from the right commutative cuts (where the second given
  derivation gets smaller); therefore, the termination metric must
  stipulate that the derivation that does not get smaller must stay
  the same size \cite{pfenning00structural}. As noted in
  \cite{simmons11structural}, our more specific termination argument
  has a potential advantage -- it is not necessary to show that 
  the uses of weakening (if $\mildseq{\Gamma}{\Delta}{U}$ 
  then $\mildseq{\Gamma, A^-}{\Delta}{U}$), which is 
  a lemma needed to deal with all 
  commutative cuts involving ${!}_L$, preserve the
  structure or size of proofs.
\end{proof}

\subsection{Correctness of focusing}
\label{sec:lincorrectness}

\begin{figure}
{\small \[
\begin{array}{rcl|rcl|rcl}
{(\Gamma)^\circ} & & &
{(\underline{\Delta})^\circ} & & &
{(\underline{U})^\circ} & & 
\\
(\cdot)^\circ & \!\!\!=\!\!\! & \cdot &
(\cdot)^\circ & \!\!\!=\!\!\! & \cdot &
(A^-)^\circ & \!\!\!=\!\!\! & (A^-)^\circ
\\
(\Gamma, A^-)^\circ & \!\!\!=\!\!\! & (\Gamma)^\circ, (A^-)^\circ &
(\Delta, A^+)^\circ & \!\!\!=\!\!\! & (\Delta)^\circ, (A^+)^\circ &
(A^+)^\circ & \!\!\!=\!\!\! & (A^+)^\circ
\\
& & & 
(\Delta, A^-)^\circ & \!\!\!=\!\!\! & (\Delta)^\circ, (A^-)^\circ &
([A^+])^\circ & \!\!\!=\!\!\! & (A^+)^\circ 
\\
& & &
(\Delta, [ A^- ])^\circ & \!\!\!=\!\!\! & (\Delta)^\circ, (A^-)^\circ & 
(\langle p^- \rangle)^\circ & \!\!\!=\!\!\! & p^-
\\
& & &
(\Delta, \langle p^+ \rangle)^\circ & \!\!\!=\!\!\! & (\Delta)^\circ, p^+ & 
& &
\end{array}\]}
\caption{Lifting erasure and polarization (Figure~\ref{fig:lin-shift}) to
contexts and succeedents}
\label{fig:lin-shift-ctx}
\end{figure}


Now we will make precise the correctness for a focused, polarized
logic that was discussed in Section~\ref{sec:linpolar}: that 
there is an unfocused derivation of
$(A^+)^\circ$ or $(A^-)^\circ$ if and only if there is a focused
derivation of $A^+$ or $A^-$.
The proof require lifting of our erasure function 
to contexts and succeedants, which is
done in Figure~\ref{fig:lin-shift-ctx}. Note that erasure is only
defined on focused sequents $\mildseq{\Gamma}{\Delta}{U}$ when
$\Delta$ and $U$ contain only atomic suspended propositions; therefore
Theorems~\ref{thm:linfocsound}~and~\ref{thm:linfoccomplete}, like
the cut admissibility theorem, 
have an extra condition that $\Delta$ and $U$ contain 
only atomic suspended propositions. Soundness and completeness
are established on the basis of erasure as in \cite{simmons11structural}.

\bigskip
\begin{theorem}[Soundness of focusing]\label{thm:linfocsound}
If $\mildseq{\Gamma}{\Delta}{U}$, 
then $\seq{\Gamma^\circ}{\Delta^\circ}{U^\circ}$.
\end{theorem}

\begin{proof}
  By straightforward induction on the given derivation; in each case,
  the result either follows directly by invoking the induction
  hypothesis or by invoking the induction hypothesis and applying one
  rule from Figure~\ref{fig:linear}.
\end{proof}

\begin{theorem}[Completeness of focusing]\label{thm:linfoccomplete}
If $\seq{\Gamma^\circ}{\Delta^\circ}{C^\circ}$, where $\Delta$ and $U$ are
stable,\footnote{Frank Pfenning's 
  variant of this proof does not require that inversion sequents be stable,
  but the tradeoff is that he must use a more general statement of 
  cut admissibility (Theorem~\ref{thm:lincut}) that also makes fewer
  stability demands.} 
then $\mildseq{\Gamma}{\Delta}{C}$. 
\end{theorem}

\begin{proof}
  By induction on the first given derivation. Each rule in 
  Figure~\ref{fig:linear} corresponds to one {\it unfocused admissibility 
  lemma} that we must prove in the focused system, some extra steps 
  to take care of leftover shifts. These leftover steps 
  can be characterized as derivable rules of inference:
  \[
  \infer[{\downarrow}{\uparrow}_R]
  {\mildseq{\Gamma}{\Delta}{{\downarrow}{\uparrow}A^+}}
  {\mildseq{\Gamma}{\Delta}{A^+}}
  \quad
  \deduce{\mathstrut}{=}
  \quad
  \infer[{\it focus}_R]
  {\mildseq{\Gamma}{\Delta}{{\downarrow}{\uparrow}A^+}}
  {\infer[{\downarrow}_R]
   {\mildseq{\Gamma}{\Delta}{[{\downarrow}{\uparrow}A^+}]}
   {\infer[{\uparrow}_R]
    {\mildseq{\Gamma}{\Delta}{{\uparrow}A^+}}
    {\mildseq{\Gamma}{\Delta}{A^+}}}}
  \]\[
  \infer[{\uparrow}{\downarrow}_L]
  {\mildseq{\Gamma}{\Delta, {\uparrow}{\downarrow}A^-}{U}}
  {\mildseq{\Gamma}{\Delta, A^-}{U}}
  \quad
  \deduce{\mathstrut}{=}
  \quad
  \infer[{\it focus}_L]
  {\mildseq{\Gamma}{\Delta, {\uparrow}{\downarrow}A^-}{U}}
  {\infer[{\uparrow}_L]
   {\mildseq{\Gamma}{\Delta, [{\uparrow}{\downarrow}A^-]}{U}}
   {\infer[{\downarrow}_R]
    {\mildseq{\Gamma}{\Delta, {\downarrow}A^-}{U}}
    {\mildseq{\Gamma}{\Delta, A^-}{U}}}}
  \]
%   The other two are admissible:
%   \[
%   \infer-[\dedownup_R]
%   {\mildseq{\Gamma}{\Delta}{A^+}}
%   {\mildseq{\Gamma}{\Delta}{{\downarrow}{\uparrow}A^+}}
%   \quad
%   \deduce{\mathstrut}{=}
%   \quad
%   \infer-[{\it cut}(3)]
%   {\mildseq{\Gamma}{\Delta}{A^+}}
%   {\mildseq{\Gamma}{\Delta}{{\downarrow}{\uparrow}A^+}
%    &
%    \infer[{\downarrow}_L]
%    {\mildseq{\Gamma}{{\downarrow}{\uparrow}A^+}{A^+}}
%    {\infer[{\it focus}_L]
%     {\mildseq{\Gamma}{{\uparrow}A^+}{A^+}}
%     {\infer[{\uparrow}_L]
%      {\mildseq{\Gamma}{[{\uparrow}A^+]}{A^+}}
%      {\infer-[\eta^+]
%       {\mildseq{\Gamma}{A^+}{A^+}}
%       {\infer[{\it focus}_R]
%        {\mildseq{\Gamma}{\langle A^+ \rangle}{A^+}}
%        {\infer[{\it id}^+]
%         {\mildseq{\Gamma}{\langle A^+ \rangle}{[A^+]}}
%         {}}}}}}}
%   \]\[
%   \infer-[\deupdown_L]
%   {\mildseq{\Gamma}{\Delta, A^-}{U}}
%   {\mildseq{\Gamma}{\Delta, {\uparrow}{\downarrow}A^-}{U}}
%   \quad
%   \deduce{\mathstrut}{=}
%   \quad
%   \infer-[{\it cut}(4)] 
%   {\mildseq{\Gamma}{\Delta, A^-}{U}}
%   {\infer[{\uparrow}_R]
%    {\mildseq{\Gamma}{\Delta, A^-}{{\uparrow}{\downarrow}A^-}}
%    {\infer[{\it focus}_R]
%     {\mildseq{\Gamma}{\Delta, A^-}{{\downarrow}A^-}}
%     {\infer[{\downarrow}_R]
%      {\mildseq{\Gamma}{\Delta, A^-}{[{\downarrow}A^-]}}
%      {\infer-[\eta^-]
%       {\mildseq{\Gamma}{\Delta, A^-}{A^-}}
%       {\infer[{\it focus}_R]
%        {\mildseq{\Gamma}{\Delta, A^-}{\langle A^- \rangle}}
%        {\infer[{\it id}^-]
%         {\mildseq{\Gamma}{\Delta, [ A^- ]}{\langle A^- \rangle}}
%         {}}}}}}
%    &
%    \mildseq{\Gamma}{\Delta, {\uparrow}{\downarrow}A^-}{U}}
%   \]
We will describe a few cases to illustrate how unfocused admissibility
lemmas work.

  Rule {\it copy}: We are given 
  $\seq{\Gamma^\circ, A}{\Delta^\circ, A}{U^\circ}$, which is
  used to derive $\seq{\Gamma^\circ, A}{\Delta^\circ, A}{U^\circ}$.
  We know $A = (A^-)^\circ$. By the induction hypothesis, we have
  $\mildseq{\Gamma, A^-}{\Delta, A^-}{U}$, and we conclude
  with the unfocused admissibility lemma ${\it copy}_u$:
  \[
  \infer-[{\it cut}(4)]
  {\mildseq{\Gamma, A^-}{\Delta}{U}}
  {\infer-[\eta^-]
   {\mildseq{\Gamma, A^-}{\cdot}{A^-}}
   {\infer[{\it id}^-]
    {\mildseq{\Gamma, A^-}{\cdot}{\langle A^- \rangle}}
    {}}
   &
   \mildseq{\Gamma, A^-}{\Delta, A^-}{U}}
  \]

  Rule ${!}_L$: We are given 
  $\seq{\Gamma^\circ, A}{\Delta^\circ}{U^\circ}$,
  which is used to derive
  $\seq{\Gamma^\circ}{\Delta^\circ, {!}A}{U^\circ}$.
  We know ${!}A = (C^-)^\circ$; by induction on the structure of 
  $C^-$ there exists $A^-$ such that 
  $C^- = {\uparrow}{\downarrow}\ldots{\downarrow}{\uparrow}{!}A^-$.
  By the induction hypothesis, we have 
  $\mildseq{\Gamma, A^-}{\Delta}{U}$, and we conclude by the 
  unfocused admissibility lemma ${!}_{uL}$, which is derivable: 
  \[
  \infer=[{\uparrow}{\downarrow}_L]
  {\mildseq{\Gamma}
   {\Delta, {\uparrow}{\downarrow}\ldots{\downarrow}{\uparrow}{!}A}{U}}
  {\infer[{\it focus}_L]
   {\mildseq{\Gamma}{\Delta, {\uparrow}{!}A^-}{U}}
   {\infer[{\uparrow}_L]
    {\mildseq{\Gamma}{\Delta, [{\uparrow}{!}A^-]}{U}}
    {\infer[{!}_L]
     {\mildseq{\Gamma}{\Delta, {!}A^-}{U}}
     {\mildseq{\Gamma, A^-}{\Delta}{U}}}}}
  \]

  Rule ${!}_R$: We are given
  $\seq{\Gamma^\circ}{\cdot}{A}$,
  which is used to derive
  $\seq{\Gamma^\circ}{\cdot}{{!}A}$. 
  We know ${!}A = (C^+)^\circ$; by induction on the structure of
  $C^+$ there exists $A^-$ such that 
  $C^+ = {\downarrow}{\uparrow}\ldots{\downarrow}{\uparrow}!A^-$.
  By the induction hypothesis, we have
  $\mildseq{\Gamma}{\cdot}{{\downarrow}A^-}$, and we conclude by the unfocused
  admissibility lemma ${!}_{uR}$, which is derivable:
  \[
  \infer=[{\downarrow}{\uparrow}_R]
  {\mildseq{\Gamma}{\cdot}
   {{\downarrow}{\uparrow}\ldots{\downarrow}{\uparrow}!A^-}}
  {\infer-[{\it cut}(5)]
   {\mildseq{\Gamma}{\cdot}{!A^-}}
   {\infer[{\uparrow}_R]
    {\mildseq{\Gamma}{\cdot}{{\uparrow}{\downarrow}A^-}}
    {\mildseq{\Gamma}{\cdot}{{\downarrow}A^-}}
    &
    \infer[{\it focus}_R]
    {\mildseq{\Gamma, {\uparrow}{\downarrow}A^-}{\cdot}{!A^-}}
    {\infer[{!}_R]
     {\mildseq{\Gamma, {\uparrow}{\downarrow}A^-}{\cdot}{[!A^-]}}
     {\infer-[\eta^-]
      {\mildseq{\Gamma, {\uparrow}{\downarrow}A^-}{\cdot}{A^-}}
      {\infer[{\it copy}]
       {\mildseq{\Gamma, {\uparrow}{\downarrow}A^-}{\cdot}{\langle A^-\rangle}}
       {\infer[{\uparrow}_L]
        {\mildseq{\Gamma, {\uparrow}{\downarrow}A^-}
         {[{\uparrow}{\downarrow}A^-]}{\langle A^-\rangle}}
        {\infer[{\downarrow}_L]
         {\mildseq{\Gamma, {\uparrow}{\downarrow}A^-}
          {{\downarrow}A^-}{\langle A^-\rangle}}
         {\infer[{\it focus}_L]
          {\mildseq{\Gamma, {\uparrow}{\downarrow}A^-}
           {A^-}{\langle A^-\rangle}}
          {\infer[{\it id}^-]
           {\mildseq{\Gamma, {\uparrow}{\downarrow}A^-}
           {[ A^- ]}{\langle A^-\rangle}}
           {}}}}}}}}}}
  \]

  Rule $\lolli_L$: We are given 
  $\seq{\Gamma^\circ}{\Delta_A^\circ}{A}$ and
  $\seq{\Gamma^\circ}{\Delta^\circ, B}{U^\circ}$, which are used 
  to derive $\seq{\Gamma^\circ}{\Delta_A^\circ, \Delta^\circ, A \lolli B}{U}$.
  We know $A \lolli B = (C^-)^\circ$; by induction on the structure of
  $C^-$ there exist $A^+$ and
  $B^-$ such that $A = (A^+)^\circ$, $B = (B^-)^\circ$, and 
  $C^- = 
   {\uparrow}{\downarrow}\ldots{\uparrow}{\downarrow}(A^+ \lolli B^-)$.
  By the induction hypothesis, we have
  $\mildseq{\Gamma}{\Delta_A}{A}$ and
  $\mildseq{\Gamma}{\Delta, B}{U}$, and we conclude
  by the unfocused admissibility lemma ${\lolli}_{uL}$:
  \[
  \infer=[{\uparrow}{\downarrow}_L]
  {\mildseq{\Gamma}
   {\Delta_A, \Delta, 
    {\uparrow}{\downarrow}\ldots{\uparrow}{\downarrow}(A^+ \lolli B^-)}{U}}
  {\infer-[{\it cut}(3)]
   {\mildseq{\Gamma}{\Delta_A, \Delta, A^+ \lolli B^-}{U}}
   {\infer-[{\it cut}(3)]
    {\mildseq{\Gamma}{\Delta_A, A^+ \lolli B^-}{{\downarrow}B^-}}  
    {\mildseq{\Gamma}{\Delta_A}{A^+}
     &
     \infer-[\eta^+]
     {\mildseq{\Gamma}{A^+, A^+ \lolli B^-}{{\downarrow}B^-}}
     {\infer[{\it focus}_R]
      {\mildseq{\Gamma}{\langle A^+ \rangle, A^+ \lolli B^-}{{\downarrow}B^-}}
      {\infer[{\downarrow}_R]
       {\mildseq{\Gamma}{\langle A^+ \rangle, A^+ \lolli B^-}
        {[{\downarrow}B^-}]}
       {\infer-[\eta^-]
        {\mildseq{\Gamma}{\langle A^+ \rangle, A^+ \lolli B^-}
         {B^-}}
        {\infer[{\it focus}_L]
         {\mildseq{\Gamma}{\langle A^+ \rangle, A^+ \lolli B^-}
          {\langle B^- \rangle}}
         {\infer[{\lolli}_L]
          {\mildseq{\Gamma}{\langle A^+ \rangle, [ A^+ \lolli B^- ]}
           {\langle B^- \rangle}}
          {\infer[{\it id}^+]
           {\mildseq{\Gamma}{\langle A^+ \rangle}
            {[A^+]}}
           {}
           &
           \infer[{\it id}^-]
           {\mildseq{\Gamma}{[B^-]}{\langle B^- \rangle}}
           {}}}}}}}}
    &
    \infer[B^-]
    {\mildseq{\Gamma}{\Delta, {\downarrow}B^-}{U}}
    {\mildseq{\Gamma}{\Delta, B^-}{U}}}} 
  \]

  Rule $\lolli_R$: We are given 
  $\seq{\Gamma^\circ}{\Delta^\circ, A}{B}$, which is used 
  to derive $\seq{\Gamma^\circ}{\Delta^\circ}{A \lolli B}$.
  We know $A \lolli B = (C^+)^\circ$; by induction on the structure of
  $C^+$ there exist $A^+$ and
  $B^-$ such that $A = (A^+)^\circ$, $B = (B^-)^\circ$, and 
  $C^+ = 
   {\downarrow}{\uparrow}\ldots{\uparrow}{\downarrow}
    (A^+ \lolli B^-)$.
  By the induction hypothesis, we have
  $\mildseq{\Gamma}{\Delta, {\downarrow}A^+}{{\uparrow}B^+}$, 
  and we conclude by the 
  unfocused admissibility lemma ${\lolli}_{uR}$:
  \[
  \infer=[{\downarrow}{\uparrow}_R]
  {\mildseq{\Gamma}{\Delta}{{\downarrow}{\uparrow}\ldots{\uparrow}{\downarrow}
    (A^+ \lolli B^-)}}
  {\infer-[{\it cut}(4)]
   {\mildseq{\Gamma}{\Delta}{{\downarrow}(A^+ \lolli B^-)}}
   {\infer[{\lolli}_R]
    {\mildseq{\Gamma}{\Delta}
     {{\downarrow}{\uparrow}A^+ \lolli {\uparrow}{\downarrow}B^-}}
    {\infer[{\downarrow}_L]
     {\mildseq{\Gamma}{\Delta, {\downarrow}{\uparrow}A^+}
      {{\uparrow}{\downarrow}B^-}}
     {\infer[{\uparrow}_R]
      {\mildseq{\Gamma}{\Delta, {\uparrow}A^+}
       {{\uparrow}{\downarrow}B^-}}
      {\mildseq{\Gamma}{\Delta, {\uparrow}A^+}
       {{\downarrow}B^-}}}}
    & 
    \infer[{\it focus}_R]
    {\mildseq{\Gamma}
     {{\downarrow}{\uparrow}A^+ \lolli {\uparrow}{\downarrow}B^-}
     {{\downarrow}(A^+ \lolli B^-)}}
    {\infer[{\downarrow}_R]
     {\mildseq{\Gamma}
      {{\downarrow}{\uparrow}A^+ \lolli {\uparrow}{\downarrow}B^-}
      {[{\downarrow}(A^+ \lolli B^-)]}}
     {\infer[{\lolli}_R]
      {\mildseq{\Gamma}
       {{\downarrow}{\uparrow}A^+ \lolli {\uparrow}{\downarrow}B^-}
       {A^+ \lolli B^-}}
      {\infer-[\eta^+]
       {\mildseq{\Gamma}
        {{\downarrow}{\uparrow}A^+ \lolli {\uparrow}{\downarrow}B^-, A^+}
        {B^-}}
       {\infer-[\eta^-]
        {\mildseq{\Gamma}
         {{\downarrow}{\uparrow}A^+ \lolli {\uparrow}{\downarrow}B^-, 
          \langle A^+ \rangle}
         {B^-}}
        {\infer[{\it focus}_L]
         {\mildseq{\Gamma}
          {{\downarrow}{\uparrow}A^+ \lolli {\uparrow}{\downarrow}B^-, 
           \langle A^+ \rangle}
          {\langle B^- \rangle}}
         {\infer[{\lolli}_L]
          {\mildseq{\Gamma}
           {[{\downarrow}{\uparrow}A^+ \lolli {\uparrow}{\downarrow}B^-], 
            \langle A^+ \rangle}
           {\langle B^- \rangle}}
          {\infer[{\downarrow}_R]
           {\mildseq{\Gamma}{\langle A^+ \rangle}
            {[ {\downarrow}{\uparrow}A^+ ]}}
           {\infer[{\uparrow}_R]
            {\mildseq{\Gamma}{\langle A^+ \rangle}{{\uparrow}A^+}}
            {\infer[{\it focus}_R]
             {\mildseq{\Gamma}{\langle A^+ \rangle}{A^+}}
             {\infer[{\it id}^+]
              {\mildseq{\Gamma}{\langle A^+ \rangle}{[ A^+ ]}}
              {}}}}
           &
           \infer[{\uparrow}_L]
           {\mildseq{\Gamma}{[ {\uparrow}{\downarrow}B^- ]}
            {\langle B^- \rangle}}
           {\infer[{\downarrow}_L]
            {\mildseq{\Gamma}{{\downarrow}B^-}{\langle B^- \rangle}}
            {\infer[{\it focus}_L]
             {\mildseq{\Gamma}{B^-}{\langle B^- \rangle}}
             {\infer[{\it id}^-]
              {\mildseq{\Gamma}{[B^-]}{\langle B^- \rangle}}
              {}}}}}}}}}}}}}
  \]

\noindent
All the other cases follow the same pattern.
\end{proof}

\subsection{Confluent versus fixed inversion}
\label{sec:confluent-v-fixed}

A salient feature of this presentation of focusing is that invertible,
non-focused rules need not be applied in any particular order.
Therefore, the last step in a proof of $\mildseq{\Gamma}{\Delta, A
  \tensor B, \one, {!}C}{D \lolli E}$ could be ${\otimes}_L$,
${\one}_L$ ${!}_L$, or ${\lolli}_R$. Allowing for this inessential
nondeterminism simplifies the presentation a bit, but it also gets in
the way of effective proof search and canonical derivations if we do
not address it in some way.  Addressing this nondeterminism within an
inversion phase echos the discussion of LF from the beginning of the
chapter.

We can, as suggested in that introduction, declare that all
proofs which differ only by the order of their invertible, non-focused
rules be treated as equivalent. It is possible to establish that all
possible inversion orderings will lead to the same set of stable
sequents, which lets us know that all of these reorderings do not
fundamentally change the structure of the rest of the proof, and it
should furthermore be possible to show that this equivalence relation
on derivations is decidable. This is fundamentally a confluence
property, and so we can call this style of focusing a {\it confluent}
presentation. The style is exemplified by Liang and Miller's LJF
\cite{liang09focusing}, and the confluent presentation in this chapter
is closely faithful to Pfenning's course notes on linear logic
\cite{pfenning12chaining}.  

%; one of them
%is Proposition~\ref{prop:confluence-lin}:
%
%\bigskip
%\begin{proposition}
%\label{prop:confluence-lin}
%Let $\Xi$ be a set of derivations of stable sequents. If 
%$\mildseq{\Gamma}{\Delta}{U}$, where $\Delta$ and $U$ contain no 
%focused propositions, can be derived from $\Xi$ using only non-focused
%rules, then for all non-focused rules that can be used to derive
%$\mildseq{\Gamma}{\Delta}{U}$, its premises can be derived
%from $\Xi$ using only non-focused rules.
%\end{proposition}
%\bigskip

Given this decidable equivalence on proofs, we can pick some member of
each equivalence class
to serve as a canonical representative; this will suffice
to solve the problems with proof search, as we can search for 
the canonical representatives of focused proofs rather than the 
much larger set of all focused proofs. The most common canonical
representatives force all invertible rules to be applied so that
propositions are decomposed in a depth-first ordering. 

Then, reminiscent of the move from LF to Canonical LF, the logic
itself can be restricted so that only the canonical representatives
are admitted. The most convenient way of forcing a left-most,
depth-first ordering is to isolate the invertible propositions ($A^+$
on the left and $A^-$ on the right) in separate, ordered inversion
contexts, and then to only work on the left-most proposition in the
context. This is the way most focused logics are defined, including
those by Andreoli, Chaudhuri, and myself (both in the Structural
Focalization development and in the next chapter). This style of 
presenting a focusing logic can be called a {\it fixed} presentation,
as the inversion phase is fixed in a particular, though
fundamentally arbitrary, shape. 

The completeness of focusing for a fixed presentation of focusing is
implied by the completeness of focusing for a confluent presentation
of the same logic along with the appropriate confluence property for
that logic, whereas the reverse is not true. Therefore, the confluent
presentation, in a certain sense, provides a stronger theorem than the
fixed presentation, though the fixed presentation will be sufficient
for our purposes.

\subsection{Running example}

\begin{figure}[t]
\[
\infer[{\it focus}_R]
{\mildseq{\cdot}{\cdot}
   {{\downarrow}({!}({\sf 6bucks} \lolli {\uparrow}{\sf battery}) \otimes
                 {\sf 6bucks} \otimes 
                 {\downarrow}({\sf battery} \lolli {\uparrow}{\sf robot}) \lolli 
                 {\uparrow}{\sf robot})}}
{\infer[{\downarrow}_R]
{\mildseq{\cdot}{\cdot}
   {[{\downarrow}({!}({\sf 6bucks} \lolli {\uparrow}{\sf battery}) \otimes
                 {\sf 6bucks} \otimes 
                 {\downarrow}({\sf battery} \lolli {\uparrow}{\sf robot}) \lolli 
                 {\uparrow}{\sf robot})]}}
{\infer[{\lolli}_R]
{\mildseq{\cdot}{\cdot}
   {{!}({\sf 6bucks} \lolli {\uparrow}{\sf battery}) \otimes
                 {\sf 6bucks} \otimes 
                 {\downarrow}({\sf battery} \lolli {\uparrow}{\sf robot}) \lolli 
                 {\uparrow}{\sf robot}}}
{\infer[{\otimes}_L]
{\mildseq{\cdot}{{!}({\sf 6bucks} \lolli {\uparrow}{\sf battery}) \otimes
                    {\sf 6bucks} \otimes 
                    {\downarrow}({\sf battery} \lolli {\uparrow}{\sf robot})}
                    {{\uparrow}{\sf robot}}}
{\infer[{!}_L]
{\mildseq{\cdot}{{!}({\sf 6bucks} \lolli {\uparrow}{\sf battery}),
                    {\sf 6bucks} \otimes 
                    {\downarrow}({\sf battery} \lolli {\uparrow}{\sf robot})}
                    {{\uparrow}{\sf robot}}}
{\infer[{\otimes}_L]
{\mildseq{\Gamma}{{\sf 6bucks} \otimes 
                  {\downarrow}({\sf battery} \lolli {\uparrow}{\sf robot})}
                  {{\uparrow}{\sf robot}}}
{\infer[\eta^+]
{\mildseq{\Gamma}{{\sf 6bucks},
                  {\downarrow}({\sf battery} \lolli {\uparrow}{\sf robot})}
                  {{\uparrow}{\sf robot}}}
{\infer[{\downarrow}_L]
{\mildseq{\Gamma}{\langle {\sf 6bucks} \rangle,
                  ({\sf battery} \lolli {\uparrow}{\sf robot})}
                  {{\uparrow}{\sf robot}}}
{\infer[{\uparrow}_R]
{\mildseq{\Gamma}{\langle {\sf 6bucks} \rangle,
                  {\downarrow}({\sf battery} \lolli {\uparrow}{\sf robot})}
                  {{\uparrow}{\sf robot}}}
{\infer[{\it copy}]
{\mildseq{\Gamma}{\langle {\sf 6bucks} \rangle, 
                  {\sf battery} \lolli {\uparrow}{\sf robot}}
                  {{\sf robot}}}
{\infer[{\lolli}_L]
{\mildseq{\Gamma}{\langle {\sf 6bucks} \rangle, 
                  {\sf battery} \lolli {\uparrow}{\sf robot}, 
                  [{\sf 6bucks} \lolli {\uparrow}{\sf battery}]}{{\sf robot}}}
{\infer[{\it id}^+]
 {\mildseq{\Gamma}{\langle {\sf 6bucks} \rangle}{[{\sf 6bucks}]}}
 {}
 &
 \infer[{\uparrow}_L] 
{\mildseq{\Gamma}{{\sf battery} \lolli {\uparrow}{\sf robot}, [{\uparrow}{\sf battery}]}{{\sf robot}}}
 {\infer[\eta^+]
 {\mildseq{\Gamma}{{\sf battery} \lolli {\uparrow}{\sf robot}, {\sf battery}}{{\sf robot}}}
 {\infer[{\it focus}_L]
 {\mildseq{\Gamma}{{\sf battery} \lolli {\uparrow}{\sf robot}, \langle {\sf battery} \rangle}{{\sf robot}}}
 {\infer[{\lolli}_L]
 {\mildseq{\Gamma}{\langle {\sf battery} \rangle, [{\sf battery} \lolli {\uparrow}{\sf robot}]}{{\sf robot}}}
 {\infer[{\it id}^+]
  {\mildseq{\Gamma}{\langle {\sf battery} \rangle}{[{\sf battery}]}}
  {}
  &
  \infer[{\uparrow}_L]
  {\mildseq{\Gamma}{[{\uparrow}{\sf robot}]}{{\sf robot}}}
  {\infer[\eta^+]
  {\mildseq{\Gamma}{{\sf robot}}{{\sf robot}}}
  {\infer[{\it focus}_R]
  {\mildseq{\Gamma}{\langle {\sf robot} \rangle}{{\sf robot}}}
  {\infer[{\it id}^+]
  {\mildseq{\Gamma}{\langle {\sf robot} \rangle}{[{\sf robot}]}}
  {}}}}}}}}}}}}}}}}}}}
\] 
\caption{The single focused transition is possible 
(where we let $\Gamma = {\sf 6bucks} \lolli {\uparrow}{\sf battery}$).}
\label{fig:focused-robot}
\end{figure}


Figure~\ref{fig:focused-robot} gives the result of taking our 
robot example, Figure~\ref{fig:unfocused-robot}, through the 
polarization and focalization process described by 
Theorem~\ref{thm:linfoccomplete}. There is indeed only one 
proof of this focused proposition up to the reordering of 
invertible rules, and only one proof period if we always
decompose invertible propositions in a left-most, depth-first
ordering (as we do in Figure~\ref{fig:focused-robot}). 

We have therefore successfully used focusing to get a canonical
proof structure that correctly corresponds to our 
informal series of transitions:
\[
\begin{array}{ccccc}
\begin{array}{c}
\mbox{\it \$6 (1)}\medskip\\ 
\mbox{\it battery-free robot (1)} \medskip\\ 
\mbox{\it turn \$6 into a battery}\\
\mbox{\it (all you want)}
\end{array}
& \leadsto &
\begin{array}{c}
\mbox{\it battery  (1)}\medskip\\ 
\mbox{\it battery-free robot (1)} \medskip\\ 
\mbox{\it turn \$6 into a battery}\\
\mbox{\it (all you want)}
\end{array}
& \leadsto &
\begin{array}{c}
\mbox{\it robot (1)} \medskip\\ 
\mbox{\it turn \$6 into a battery}\\
\mbox{\it (all you want)}\medskip\\~\\
\end{array}
\end{array}
\]
But at what cost? Figure~\ref{fig:focused-robot} definitely contains a
fair amount of bureaucracy compared to the original
Figure~\ref{fig:unfocused-robot}, even if does a better job of
matching, when read from bottom to top, the series of transitions. A
less cluttered way of looking at these proofs is in terms of what
Andreoli called {\it bipoles} \cite{andreoli01focussing} and what we,
following Chaudhuri, call {\it synthetic inference rules}
\cite{chaudhuri08focusing}.

\section{Synthetic inference rules}
\label{sec:linsynthetic}

Synthetic inference rules were introduced by Andreoli as {\it bipoles}
for the purpose of giving a more abstract way of looking at focused
logics and focused derivations \cite{andreoli01focussing}. The first
idea behind synthetic inference rules is that the stable
sequents are the ones that are most clearly endowed with meaning in a
focused sequent calculus; this was reflected by the polarization
strategy used in the completeness theorem for linear logic
(Theorem~\ref{thm:linfoccomplete}), which only dealt with
stable sequents. The second idea is that, once we know 
whether the last rule in the proof of a stable sequent is
\begin{itemize}
\item ${\it copy}$ on some proposition $A^-$ from $\Gamma$, 
\item ${\it focus}_L$ on some proposition $A^-$ in $\Delta$, or
\item ${\it focus}_R$ on the succeedant $A^+$
\end{itemize} (it must be one of these
three), then the structure of the proof is completely determined
up to the next occurrence of a stable sequent. 

For example, consider the act of focusing on the proposition
$a^+ \lolli {\uparrow}b^+$ in $\Gamma$ using the ${\it copy}$ rule,
where $a^+$ and $b^+$ are positive atomic propositions. 
This must mean that a suspended atomic proposition $a^+$ appears
in the context $\Delta$, or else the proof could not be completed:
\[
\infer[{\it copy}]
{\mildseq{\Gamma, a^+ \lolli {\uparrow}b^+}{\Delta, \langle a^+ \rangle}{U}}
{\infer[{\lolli}_L]
 {\mildseq{\Gamma, a^+ \lolli {\uparrow}b^+}
   {\Delta, \langle a^+ \rangle, [a^+ \lolli {\uparrow}b^+]}{U}}
 {\infer[{\it id}^+]
  {\mildseq{\Gamma, a^+ \lolli {\uparrow}b^+}
   {\langle a^+ \rangle}{[ a^+ ]}}
  {}
  &
  \infer[{\uparrow}_L]
  {\mildseq{\Gamma, a^+ \lolli {\uparrow}b^+}{\Delta, [{\uparrow}b^+]}{U}}
  {\infer[\eta^+]
   {\mildseq{\Gamma, a^+ \lolli {\uparrow}b^+}{\Delta, b^+}{U}}
   {\mildseq{\Gamma, a^+ \lolli {\uparrow}b^+}
    {\Delta, \langle b^+ \rangle}{U}}}}}
\]
The non-stable sequents in the middle are not interesting parts 
of the structure of the proof, as they are fully determined by the
choice of focus, so we can collapse this series of transitions
into a single synthetic rule:
\[
\infer[{\sf CP}_{a^+ \lolli {\uparrow}b^+}]
{\mildseq{\Gamma, a^+ \lolli {\uparrow}b^+}{\Delta, \langle a^+ \rangle}{U}}
{\mildseq{\Gamma, a^+ \lolli {\uparrow}b^+}{\Delta, \langle b^+ \rangle}{U}}
\]
Similar rules can be given for focusing phases associated with the
${\it focus}_L$ and ${\it focus}_R$ rules:
\[
\infer[{\sf LP}_{a^+ \lolli {\uparrow}b^+}]
{\mildseq{\Gamma}{\Delta, \langle a^+ \rangle, a^+ \lolli {\uparrow}b^+}{U}}
{\mildseq{\Gamma}{\Delta, \langle b^+ \rangle}{U}}
\]
\[
\infer[{\sf RF}_{{\downarrow}({!}A^- \otimes b^+ \otimes {\downarrow}C^- \lolli 
   {\uparrow}D^+)}]
{\mildseq{\Gamma}{\Delta}
  {{\downarrow}({!}A^- \otimes b^+ \otimes {\downarrow}C^- \lolli 
   {\uparrow}D^+)}}
{\mildseq{\Gamma, A^-}{\Delta, \langle b^+ \rangle, C^-}{D^+}}
\quad
\infer[{\sf RF}_{a^+}]
{\mildseq{\Gamma}{\langle a^+ \rangle}{a^+}}
{}
\]

\begin{figure}
{\small\[
\infer[{\sf RF}_{{\downarrow}({!}A^- \otimes b^+ \otimes {\downarrow}C^- \lolli 
   {\uparrow}D^+)}]
{\mildseq{\cdot}{\cdot}
   {{\downarrow}({!}({\sf 6bucks} \lolli {\uparrow}{\sf battery}) \otimes
             {\sf 6bucks} \otimes 
             {\downarrow}({\sf battery} \lolli {\uparrow}{\sf robot}) \lolli 
             {\uparrow}{\sf robot})}}
{\infer[{\sf CP}_{a^+ \lolli {\uparrow}b^+}]
 {\mildseq{{\sf 6bucks} \lolli {\uparrow}{\sf battery}\quad}
    {\quad\langle {\sf 6bucks} \rangle, ~~
     {\sf battery} \lolli {\uparrow}{\sf robot}\quad}{\quad{\sf robot}}}
 {\infer[{\sf LF}_{a^+\lolli{\uparrow}b^+}]
  {\mildseq{{\sf 6bucks} \lolli {\uparrow}{\sf battery}\quad}
    {\quad{\sf battery} \lolli {\uparrow}{\sf robot}, ~~
     \langle {\sf battery} \rangle\quad}{\quad{\sf robot}}}
  {\infer[{\sf RF}_{a^+}]
   {\mildseq{{\sf 6bucks} \lolli {\uparrow}{\sf battery}\quad}
       {\quad\langle {\sf robot} \rangle\quad}{\quad{\sf robot}}}
   {}}}}
\]}
\caption{Our running example, presented with synthetic rules.}
\label{fig:synthetic-robot}
\end{figure}

Focused proofs of stable sequents are, by definition, in a 1-to-1
correspondence with proofs using synthetic inference rules. If we look
at our running example as a derivation using synthetic inference rules
(as demonstrated in Figure~\ref{fig:synthetic-robot}), we see that the
system takes four steps. The last three steps, furthermore, correspond
precisely to the three steps in our informal description of the
robot-battery-store system.

In important side note: the fact that the structure of propositions is {\it
  entirely} determined is an artifact of our restriction to MALL. If
we also consider additive connectives, then we would be in a similar
situation except that we would be able to identify some number of
synthetic rules for each right focus, left focus, or copy (possibly
zero; there's no way to successfully right focus on a proposition
like $\zero \otimes {\uparrow}{\downarrow}A^+$).


\section{Hacking the focusing system}
\label{sec:linhack}

Despite the novel treatment of suspended propositions in
Section~\ref{sec:foclinlog}, the presentation of linear logic given
there is essentially the same as the presentation in Chaudhuri's
thesis \cite{chaudhuri06focused}, in the sense that the logic gives
rise to the same synthetic inference rules. It is {\it not} a faithful
intuitionistic analogue to Andreoli's original presentation of focusing
\cite{andreoli92logic}, though the presentation in Pfenning's course notes is
\cite{pfenning12chaining}.\footnote{We will blur the lines, in this
  section, between Andreoli's original presentation of focused
  classical linear logic and Pfenning's adaptation to intuitionistic
  linear logic. In particular We will mostly use the notation of
  Pfenning's presentation, but the observations are equally applicable
  in Andreoli's focused triadic system.}  Nor does it have the same
synthetic inference rules as the focused presentation used in the
language of ordered logical specifications presented by Pfenning and I
\cite{pfenning09substructural}.

The difference, in each case, lies in the treatment of positive atomic
propositions. My justification for presenting
Chaudhuri's system as the canonical focusing system for linear logic
in Section~\ref{sec:foclinlog} is because the treatment of suspended
positive propositions fits beautifully with the idea that atomic
propositions are just placeholders for unspecified
propositions. However, our current objective is not simply to study
focused logic; we want to use the structure of proofs in focused
logics as a logical framework for encoding systems we are interested
in.

In this section, we will discuss two modifications to the treatment 
of positive propositions, Andreoli's atom
optimization and the bang optimization. These changes give the focused
system more expressiveness at the cost of losing the elegant
interpretation of positive atoms as stand-ins for positive atomic
propositions; the implications of this are discussed in
Section~\ref{sec:moreprim}.


\subsection{Atom optimization}

Andreoli's original focused system isn't polarized, so propositions
that are syntactically invalid in a polarized presentation, like
${!}(p^+ \otimes q^+)$ or ${!}p^+$ are valid in his system (we would
have to write ${!}{\uparrow}{(p^+ \otimes q^+)}$ and
${!}{\uparrow}p^+$). It's therefore possible, in an unpolarized
presentation, to use the ${\it copy}$ rule to copy a positive
proposition out of the context and into left focus, but the focus
immediately blurs, as in this proof fragment:\footnote{We will use
  the sequent form $\andseq{\Gamma}{\Delta}{C}$ in this section for
  focused but unpolarized systems.}
\[
\infer[{\it copy}]
{\andseq{p^+ \otimes q^+}{\cdot}{q^+ \otimes p^+}}
{\infer[{\it blur}_L]
 {\andseq{p^+ \otimes q^+}{[p^+ \otimes q^+]}{q^+ \otimes p^+}}
 {\infer[{\otimes}_L]
  {\andseq{p^+ \otimes q^+}{p^+ \otimes q^+}{q^+ \otimes p^+}}
  {\deduce
   {\andseq{p^+ \otimes q^+}{p^+, q^+}{q^+ \otimes p^+}}
   {\vdots}}}}
\]
Note that, in the polarized setting, the
effect of the ${\it blur}_L$ rule is accomplished by the
${\downarrow}_L$ rule.

Andreoli's system makes a single restriction to the ${\it copy}$ rule:
it cannot apply to a positive atomic proposition in the persistent
context. On its own, this restriction would make the system incomplete
with respect to unfocused linear logic -- there would be no focused
proof of ${!}p^+ \lolli p^+$ -- and so Andreoli-style focusing systems
restore completeness by creating a second initial sequent for positive
atomic propositions that allows a positive right focus on an atomic
proposition to succeed if the atomic proposition appears in the
persistent context:
\[
\infer[{\it init}^+]
{\andseq{\Gamma}{p^+}{[p^+]}}
{}
\qquad
\infer[{\it init}^+_p]
{\andseq{\Gamma, p^+}{\cdot}{[p^+]}}
{}
\]
With the second initial rule, we can once again prove ${!}p^+ \lolli p^+$,
and the system becomes complete with respect to unfocused linear
logic again.
\[
\infer[{\lolli}_R]
{\andseq{\cdot}{\cdot}{{!}p^+ \lolli p^+}}
{\infer[{!}_L]
 {\andseq{\cdot}{{!}p^+}{p^+}}
 {\infer[{\it focus}_R]
  {\andseq{p^+}{\cdot}{p^+}}
  {\infer[{\it init}^+_2]
   {\andseq{p^+}{\cdot}{[p^+]}}
   {}}}}
\]
This modified treatment of positive atoms will be called the 
{\it atom optimization}, as it reduces the number of focusing steps that 
need to be applied: it takes only one right focus to prove
${!}p^+ \lolli p^+$ in Andreoli's system, but it would take two focusing
steps to prove the same proposition in Chaudhuri's system (or to prove
${!}{\uparrow}p^+ \lolli {\uparrow}p^+$ in the focusing system we have
presented). 

There seem to be three ways of adapting the atom optimization to a polarized
setting. The first option would be to add an initial sequent that 
directly mimics the one in Andreoli's system, while adding an additional
requirement to the {\it copy} rule that $A^-$ is not a shifted positive
atomic proposition:
\[
\infer[{\it init}^+]
{\mildseq{\Gamma,{\uparrow}p^+}{\cdot}{[p^+]}}
{}
\quad
\infer[{\it copy}^*]
{\mildseq{\Gamma, A^-}{\Delta}{U}}
{A \neq {\uparrow}p^+
 &
 \mildseq{\Gamma, A^-}{\Delta, [A^-]}{U}}
\]
The second approach is to extend suspended propositions to
the persistent context, add a corresponding rule for right focus,
and modify the left rule for ${!}$ to notice
the presence of a positive atomic proposition:
\[
\infer[{!}_{L1}]
{\mildseq{\Gamma}{\Delta, {!}A^-}{U}}
{A^- \neq {\uparrow}p^+
 &
 \mildseq{\Gamma, A^-}{\Delta}{U}}
\quad
\infer[{!}_{L2}]
{\mildseq{\Gamma}{\Delta, {!}{\uparrow}p^+}{U}}
{\mildseq{\Gamma, \langle p^+ \rangle}{\Delta}{U}}
\quad
\infer[{\it id}^+_p]
{\mildseq{\Gamma, \langle A^+ \rangle}{\cdot}{[A^+]}}
{}
\]
The third approach is to introduce a third connective, $\pbang$, that
can only be applied to positive atomic propositions, just as ${!}$ can
only be applied to negative propositions. We can initially view this
option as equivalent to the previous one by defining ${\pbang}p^+$ as
a notational abbreviation for ${!}{\uparrow}p^+$ and styling rules
according to the second approach above:
\[
\infer[{\pbang}_R]
{\mildseq{\Gamma}{\cdot}{[{\pbang}p^+]}}
{\mildseq{\Gamma}{\cdot}{p^+}}
\quad
\infer[{\pbang}_L]
{\mildseq{\Gamma}{\Delta, {\pbang}p^+}{U}}
{\mildseq{\Gamma, \langle p^+ \rangle}{\Delta}{U}}
\quad
\infer[{\it id}^+_p]
{\mildseq{\Gamma, \langle A^+ \rangle}{\cdot}{[A^+]}}
{}
\]

All three of these options are essentially similar; we will go with the
last, as it allows us to preserve the original meaning of ${!}{\uparrow}p^+$
if that is our actual intent. 



\subsection{Bang optimization}
\label{sec:bangopt}

The choice of adding ${\pbang}p^+$ as a special new connective instead
of defining it as ${!}{\uparrow}p^+$ paves the way for us to modify
its meaning further. For instance, there turns out to be no particular
need for the ${\pbang}_R$ rule to lose focus in its premise, even
though it is critical that ${!}_R$ lose focus in its
premise,\footnote{If we fail to do so propositions like ${!}(A \otimes
  B) \lolli {!}(B \otimes A)$ will have no proof.} and we can revise
${\pbang}_R$ accordingly.
\[
\infer[{\pbang}_R]
{\mildseq{\Gamma}{\cdot}{[{\pbang}p^+]}}
{\mildseq{\Gamma}{\cdot}{[p^+]}}
\quad
\infer[{\pbang}_L]
{\mildseq{\Gamma}{\Delta, {\pbang}p^+}{U}}
{\mildseq{\Gamma, \langle p^+ \rangle}{\Delta}{U}}
\quad
\infer[{\it id}^+_p]
{\mildseq{\Gamma, \langle A^+ \rangle}{\cdot}{[A^+]}}
{}
\]
This further optimization
will be called the {\it bang optimization}, as it, like the atom 
optimization, potentially reduces the number of focusing phases
in a proof.

If we think of ${\pbang}p^+$ as being naturally defined as 
${!}{\uparrow}p^+$, then the atom optimization modifies the left rule
and the bang optimization modifies the right rule. It is not, however,
possible to think about the bang optimization independently from the
atom optimization: identity expansion would fail in a system that 
only provided a modified right rule for ${\pbang}p^+$. 

\subsection{Fixing the metatheory}

It is quite straightforward to fix up the 
identity expansion, cut admissibility, soundness, and completeness
theorems by adding new cases to each theorem

There is one new case of identity
expansion:
\[
\infer-[\eta^+]
{\mildseq{\Gamma}{\Delta, {\pbang}p^+}{U}}
{\deduce
 {\mildseq{\Gamma}{\Delta, \langle {\pbang}p^+ \rangle}{U}}
 {\mathcal D}}
\quad
\Longrightarrow
\infer[{\pbang}_L]
{\mildseq{\Gamma}{\Delta, {\pbang}p^+}{U}}
{\infer-[{\it subst}^+]
 {\mildseq{\Gamma, \langle p^+ \rangle}{\Delta}{U}}
 {\infer[{\pbang}_R]
  {\mildseq{\Gamma, \langle p^+ \rangle}{\cdot}{[{\pbang}p^+}]}
  {{\infer[{\it id}^+_p]
    {\mildseq{\Gamma, \langle p^+ \rangle}{\cdot}{[ p^+ ]}}
    {}}}
  &
  \infer-[{\it weaken}]
  {\mildseq{\Gamma, \langle p^+ \rangle}{\Delta, \langle {\pbang}p^+ \rangle}
     {U}}
  {\deduce
   {\mildseq{\Gamma}{\Delta, \langle {\pbang}p^+ \rangle}{U}}
   {\mathcal D}}}}
\]
While we could also prove another focal substitution principle, namely
that $\mildseq{\Gamma}{\cdot}{[A^+]}$ and $\mildseq{\Gamma, \langle
  A^+ \rangle}{\Delta}{U}$ imply $\mildseq{\Gamma}{\Delta}{U}$, it's
notable that such a principle isn't necessary in order to establish
the identity expansion property in this system.

We must add one principal cut to part (1) of the cut admissibility theorem,
one commutative case to part (3), and two commutative cases to 
parts (4) and (5).

Modifying the soundness and completeness theorems requires that we
extend the erasure function: $({\pbang}p^+)^\circ = {!}p^+$,
and $(\Gamma, \langle p^+ \rangle)^\circ = \Gamma^\circ, p^+$. Modifying
the soudness theorem is straightforward. For completeness, we must 
first consider the case of ${\it copy}$ when 
$\Gamma^\circ, A = (\Gamma, \langle p^+ \rangle)^\circ$ and so 
$A = p^+$. We can get 
$\mildseq{\Gamma, \langle p^+ \rangle}{\Delta, \langle p^+ \rangle}{U}$
from the induction hypothesis and then use the focal substitution principle:
\[
\infer-[{\it subst}^+]
{\mildseq{\Gamma, \langle p^+ \rangle}{\Delta}{U}}
{\infer[{\it id}^+_p]
 {\mildseq{\Gamma, \langle p^+ \rangle}{\cdot}{[p^+]}}
 {}
 &
 \mildseq{\Gamma, \langle p^+ \rangle}{\Delta, \langle p^+ \rangle}{U}}
\]
We also need two more unfocused admissibility lemmas:
\[
\infer-[{\pbang}_{uL}]
{\mildseq{\Gamma}{\Delta, {\uparrow}{\pbang}p^+}{U}}
{\mildseq{\Gamma, \langle p^+ \rangle}{\Delta}{U}}
\quad
\infer-[{\pbang}_{uR}]
{\mildseq{\Gamma}{\cdot}{{\pbang}p^+}}
{\mildseq{\Gamma}{\cdot}{p^+}}
\]
The first is straightforwardly derivable, but the second is quite
problemantic: it requires us to prove a new cut principle,
that $\mildseq{\Gamma}{\cdot}{p^+}$ and 
$\mildseq{\Gamma}{\Delta}{}$



Our change does force us to add a new part to the cut admissibility
theorem that is closely related to the new focal substitution
principle, proving that $\mildseq{\Gamma}{\cdot}{p^+}$ and
$\mildseq{\Gamma, \langle p^+ \rangle}{\Delta}{U}$ imply
$\mildseq{\Gamma}{\Delta}{U}$ by induction on the second given
derivation. This is used to satisfy the principal cut for
${\pbang}p^+$. All other necessary modifications to the cut
admissibility theorem are straightforward.


\subsection{Breaking the substitution interpretation of atomic propositions}

Both the atom and bang optimizations are 
unsatisfactory from the point of view that positive atomic
propositions are stand-ins for arbitrary positive propositions. None
of the three alternatives described for characterizing the atom optimization  
-- ${\it init}^+$
in the first proposed alternative, ${!}_{L2}$ in the second proposed
alternative, or ${\pbang}_L$ in the third alternative -- can be
gracefully generalized (as $\eta^+$ and $\eta^-$ can be) to account
for the substitution of a positive atomic proposition with a positive
proposition. 

The key issue, which is illustrated by each of these three
alternatives, is that linear logic does not allow positive
propositions to decomposed once they enter the persistent context.
When we substitute a positive atomic proposition $A^+$ for a positive
atomic proposition $p^+$, all instances of $\langle p^+
\rangle$ in the persistent context must become instances of
${\uparrow}A^+$, and in the third proposal, all instances of
${\pbang}p^+$ must become instances of ${!}{\uparrow}A^+$. The structure
of proofs must change in an unusual way, as well: before a focusing
phase begins, it is necessary to insert, below that focusing phase,
a series of focusing phases that copy a shifted positive proposition
into the linear context and decompose it, one for each occurrence of
${\it id}^+_p$ for a substituted proposition in the focusing phase.
One instance of this phenomena 
is illustrated in Figure~\ref{fig:replacement-breaks}.

\begin{figure}
\[
\infer[{\pbang}_L]
{\mildseq{\cdot}{{\pbang}p^+}{p^+ \otimes p^+}}
{\infer[{\it focus}_R]
 {\mildseq{\langle p^+ \rangle}{\cdot}{p^+ \otimes p^+}}
 {\infer[{\otimes}_R]
  {\mildseq{\langle p^+ \rangle}{\cdot}{[ p^+ \otimes p^+ ]}}
  {\infer[{\it id}^+_p]
   {\mildseq{\langle p^+ \rangle}{\cdot}{[p^+]}}
   {}
   &
   \infer[{\it id}^+_p]
   {\mildseq{\langle p^+ \rangle}{\cdot}{[p^+]}}
   {}}}}
~~
\mbox{vs.}
\!\!\!
\infer[{!}_L]
{\mildseq{\cdot}{{!}{\uparrow}{A^+}}{A^+ \otimes A^+}}
{\infer[\it copy]
 {\mildseq{{\uparrow}{A^+}}{\cdot}{A^+ \otimes A^+}}
 {\infer[{\uparrow}_L]
  {\mildseq{{\uparrow}{A^+}}{[{\uparrow}A^+]}{A^+ \otimes A^+}}
  {\infer-[\eta^+]
   {\mildseq{{\uparrow}{A^+}}{A^+}{A^+ \otimes A^+}}
   {\infer[{\it copy}]
    {\mildseq{{\uparrow}{A^+}}{\langle A^+ \rangle}{A^+ \otimes A^+}}
    {\infer[{\uparrow}_L]
     {\mildseq{{\uparrow}{A^+}}{\langle A^+ \rangle, [{\uparrow}A^+]}
        {A^+ \otimes A^+}}
     {\infer-[\eta^+]
      {\mildseq{{\uparrow}{A^+}}{\langle A^+ \rangle, A^+}
         {A^+ \otimes A^+}}
      {\infer[{\it focus}_R]
       {\mildseq{{\uparrow}{A^+}}{\langle A^+ \rangle, \langle A^+ \rangle}
          {A^+ \otimes A^+}}
       {\infer[{\otimes}_R]
        {\mildseq{{\uparrow}{A^+}}{\langle A^+ \rangle, \langle A^+ \rangle}
           {[A^+ \otimes A^+]}}
        {\infer[{\it id}^+]
         {\mildseq{{\uparrow}{A^+}}{\langle A^+ \rangle}
            {[A^+]}}
         {}
         &
         \infer[{\it id}^+]
         {\mildseq{{\uparrow}{A^+}}{\langle A^+ \rangle}
            {[A^+]}}
         {}}}}}}}}}}
\]
\caption{In systems with the atom optimization, substituting a 
positive proposition $A^+$ for a position atomic proposition $p^+$ 
is more complicated than just replacing atomic instances of $\eta^+$
with admissible non-atomic instances of $\eta^+$.}
\label{fig:replacement-breaks}
\end{figure}

To emphasize, the polarized formulation is not at fault here:
Andreoli's focusing system exhibits the same issues, and this is
presumably why there has been very little discussion of the meaning of
atomic propositions within focused logics; the only example we are
aware of is Miller \cite{miller08proof}. Our new ${\pbang}p^+$ connective
is justified on the grounds that it is {\it useful} for the
construction of expressive synthetic connectives even if it is not,
in a certain sense, canonical for presentations of linear logic.


\subsection{A more primitive logic?}
\label{sec:moreprim}

We introduced ${\pbang}p^+$ as a connective defined as
${!}{\downarrow}p^+$ -- that is, the regular ${!}A^-$ connective plus
a little something extra, the shift. Having modified the rules of
${\pbang}$ significantly, we will now take a crack at viewing
${\pbang}$ as a more primitive connective -- that is, we will view
${!}$ as ${\pbang}$ plus a little something extra.

It is frequently observed that the exponential ${!}A$ of linear logic
appears to have two or more parts; the general idea is that ${\pbang}$
represents just one of those pieces. Accounts of linear logic that
follow the judgmental methodology of Martin-L{\"o}f
\cite{lof96meanings}, such as the analysis by Chang et al.
\cite{chang03judgmental}, emphasize that the regular hypothetical
sequent $\seq{\Gamma}{\Delta}{A}$ of linear logic is establishing the
judgment that $A$ is ephemerally true: we can write
$\seq{\Gamma}{\Delta}{\iseph{A}}$ to emphasize this. Persistent truth,
represented by the judgment $\ispers{A}$, is defined as ephemeral
truth using no ephemeral resources, and ${!}A$ is understood as the
internalization of persistent truth:
\[
\infer[{\it pers}]
{\pseq{\Gamma}{\ispers{A}}}
{\seq{\Gamma}{\cdot}{\iseph{A}}}
\quad
\infer[{!}'_R]
{\seq{\Gamma}{\Delta}{\iseph{{!}A}}}
{\Delta = \cdot ~~~ & \pseq{\Gamma}{\ispers{A}}}
\]
The ${\it pers}$ rule is invertible, so if we ever need to prove
$\pseq{\Gamma}{\ispers{A}}$, we asynchronously transition to proving
$\seq{\Gamma}{\cdot}{\iseph{A}}$. This observation is used to explain
why we don't normally consider persistent truth on the right in linear
logic (we consider it on the left, of course, as all the propositions
of $\Gamma$ are judged as being persistently true). Our more familiar
rule for ${!}_R$ is derivable using these two rules:
\[
\infer[{!}'_R]
{\Gamma; \Delta \vdash {!}\iseph{A} \mathstrut}
{\Delta = \cdot
 &
 \infer[{\it pers}]
 {\Gamma \vdash \ispers{A} \mathstrut}
 {\Gamma; \cdot \vdash \iseph{A}} \mathstrut}
\]

Note that the ${!}'_R$ rule is naturally synchronous (positive),
because it forces the linear context to be empty. The ${\it pers}$
rule, on the other hand, is invertible and so naturally negative,
because it represents the invertible step of deciding to prove that
$A$ is {\it persistently} true by proving that it is {\it ephemerally}
true (in a context with no ephemeral resources). This combination of
positive and negative actions explains why ${!}A^-$ is a positive
proposition with a negative subformula, and similarly explains why we
must break focus when we reach ${!}A$ on the right and why we must
stop decomposing the proposition when we reach ${!}A$ on the left.
The salient feature of our modified rules for ${\pbang}p^+$, of
course, is that they do {\it not} break focus on the right and that
they continue to decompose the proposition on the left (into a
suspended proposition $\langle p^+ \rangle$ in the persistent
context). This is the reason for arguing that ${\pbang}$ captures only
the first, purely positive, component of the ${!}$ connective.

If the $\pbang$ connective is the first part of the $!$ connective,
can we characterize the rest of the connective? Giving a reasonable
answer necessarily requires a more general account of the $\pbang$
connective -- a logic where it is generally applicable rather than
restricted to positive atomic propositions. In other words, to account
for the behavior of $\pbang$, we must give a more primitive logic into
which focused linear logic, with or without the atom and bang
optimizations, may be faithfully encoded. 

\begin{figure}
\[
\infer[G_R]
{\pseq{\Gamma}{G A}}
{\seq{\Gamma}{\cdot}{A}}
\quad
\infer[G_L]
{\seq{\Gamma, G A}{\Delta}{C}}
{\seq{\Gamma, G A}{\Delta, A}{C}}
\quad
\infer[{\it init}_x]
{\pseq{\Gamma, x}{x}}
{}
\]

\[
\infer[{\supset}_R]
{\pseq{\Gamma}{X \supset Y}}
{\pseq{\Gamma, X}{Y}}
\quad
\infer[{\supset}_L]
{\pseq{\Gamma, X \supset Y}{Z}}
{\pseq{\Gamma, X \supset Y}{X}
 &
 \pseq{\Gamma, X \supset Y, Y}{Z}}
\]

\[
\infer[{\supset}'_L]
{\seq{\Gamma, X \supset Y}{\Delta}{C}}
{\pseq{\Gamma, X \supset Y}{X}
 & 
 \seq{\Gamma, X \supset Y, Y}{\Delta}{C}}
\]

\[
\infer[F_R]
{\seq{\Gamma}{\cdot}{F X}}
{\pseq{\Gamma}{X}}
\quad
\infer[F_L]
{\seq{\Gamma}{\Delta, F X}{C}}
{\seq{\Gamma, X}{\Delta}{C}}
\quad
\infer[{\it init}_a]
{\seq{\Gamma}{a}{a}}
{}
\]

\[
\infer[{\lolli}_R]
{\seq{\Gamma}{\Delta}{A \lolli B}}
{\seq{\Gamma}{\Delta, A}{B}}
\quad
\infer[{\lolli}_L]
{\seq{\Gamma}{\Delta_A, \Delta, A \lolli B}{C}}
{\seq{\Gamma}{\Delta_A}{A}
 &
 \seq{\Gamma}{\Delta, B}{C}}
\]
\caption{Some relevant sequent calculus rules for adjoint logic}
\label{fig:fragment-adjoint}
\end{figure}


A candidate for a more primitive logic, and one that has tacitly
formed the basis of much of my previous work on logic programming in
substructural logic
\cite{pfenning09substructural,simmons09weak,simmons11logical}, is {\it
  adjoint logic}.  Adjoint logic was first characterized by Benton and
Wadler as a natural deduction system \cite{benton96linear} and was
substantially generalized by Reed in a sequent calculus setting
\cite{reed09judgmental}. The logic generalizes both linear logic and
lax logic as sub-languages of a common logic, whose propositions come
in two syntactically distinct categories that are connected by the
adjoint operators $F$ and $G$:\footnote{Note this syntactic
  distinction is very different than the syntactic distinction between
  positive and negative propositions. A polarized presentation of
  adjoint logic would have four syntactic categories: $X^+$, $X^-$,
  $A^+$, and $A^-$, with one pair of shifts mediating between $X^+$
  and $X^-$ and another pair of shifts mediating between $A^+$ and
  $A^-$. To make matters worse, in Levy's Call-By-Push-Value language,
  the programming language formalism that corresponds to polarized
  logic, ${\uparrow}$ and ${\downarrow}$ are characterized as adjoints
  as well ($F$ and $U$, respectively), so a fully polarized adjoint
  logic has {\it three} distinct pairs of unary connectives that can
  be characterized as adjoints!}
\begin{align*}
\mbox{\it Persistent propositions} & &
X, Y, Z & ::= G A \mid x \mid X \supset Y \mid X \times Y\\
\mbox{\it Linear propositions} & & 
A, B, C & ::= F X \mid a \mid A \lolli B \mid A \otimes B
\end{align*}
In adjoint logic, persistent propositions $X$ are contained in the
persistent context $\Gamma$ and as the succeedants of sequents
$\pseq{\Gamma}{X}$, whereas linear propositions $A$ are contained in
the linear context $\Delta$ and as the succeedants of sequents
$\seq{\Gamma}{\Delta}{A}$. A fragment of the logic is shown in
Figure~\ref{fig:fragment-adjoint}.  Note the similarity between the
$G_L$ rule and our unfocused ${\it copy}$ rule, as well as the
similarity between $F_R$ and $G_R$ in
Figure~\ref{fig:fragment-adjoint} and the rules ${!}_R$ and ${\it
  pers}$ in the previous discussion.  Linear logic is recovered as a
fragment of adjoint logic by removing all of the persistent
propositions except for $GA$; the usual ${!}A$ is then definable as
$FGA$. Lax logic, on the other hand, is recovered by removing all of
the linear propositions except for $FX$; the usual lax modality
$\ocircle X$ is then definable as $GFX$.

My previous work has used the language of normal linear logic,
but enforced an extra condition of {\it separation}. In our
current development, the separation condition says that each atomic
proposition either {\it always} appears as a subformula of
${\pbang}p^+$ or it {\it never} appears as a subformula of
${\pbang}p^+$. If we enforce separation, then it is easy to encode our
optimized focused logic as a fragment of a naturally focused adjoint
logic. Positive atomic propositions that are always associated with
${\pbang}$ are encoded as {\it persistent} positive atomic
propositions $x^+$, and positive atomic propositions that are never
associated with ${\pbang}$ are encoded as {\it linear} positive atomic
propositions $a^+$. The proposition ${\pbang}p^+$ can then be
translated as $F x^+$, where $x^+$ is the translation of $p^+$ as a
persistent positive atomic proposition.

In this way, adjoint logic explains why, in linear logic, we can't
naturally substitute positive propositions for positive atomic
propositions when those positive atomic propositions appear suspended
in the persistent linear context: because these propositions are
actually stand-ins for {\it persistent} propositions, not for linear
propositions, and we are working in a fragment of the logic that has
no interesting persistent propositions other than atomic propositions
$x$ and the negative inclusion $G A$ back into linear propositions.

Adjoint logic, because it requires a syntactic differentiation of
persistent and linear atomic propositions, still does not allow us to
pleasantly embed the structure of focused linear logic with the atom
and bang optimizations when separation is not enforced.  The flaw is
due to the atom optimization, which allows us to successfully right
focus on an positive atomic proposition $p^+$ if the proposition
appears suspended in the linear context {\it or} in the persistent
context. This is not simple to do in adjoint logic, which forces us to
syntactically specify the context where we expect to find any given
proposition. A more complicated translation in to adjoint logic that
associated each atomic proposition $p^+$ with a persistent proposition
$x_p^+$ and a linear proposition $a_p^+$ would suffice; a subgoal
$p^+$ could be represented as $a_p^+ \oplus F x_p^+$, for instance.
Such a translation would be unsatisfactory both because it further
complicates the previously simple interpretation of atomic
propositions and because the translation of an atomic proposition is
non-uniform -- it depends on knowing whether
an atomic proposition will ultimately appear on the left or the right
side of the sequent.

I conjecture that a variant of adjoint logic which does {\it
  not} syntactically differentiate between persistent and linear
propositions might be a better target for faithfully embedding focused
linear logic with and without the atom and bang optimizations; but
this is outside the scope of this thesis.

\section{Revisiting our notation}
\label{sec:linnote}

Andreoli, in his 2001 paper introducing the idea of synthetic
inference rules \cite{andreoli01focussing}, observed that the atom
optimization can lead to an exponential explosion in the number of
synthetic rules associated with a proposition.  For instance, if $a^+
\otimes b^+ \lolli {\uparrow}c^+$ appears in $\Gamma$, the atom
optimization means that the following are all synthetic inference
rules for that proposition:
\[
\infer
{\mildseq{\Gamma}{\Delta, \langle a^+ \rangle, \langle b^+ \rangle}{U}}
{\mildseq{\Gamma}{\Delta, \langle c^+ \rangle}{U}}
\quad
\infer
{\mildseq{\Gamma, \langle a^+ \rangle}{\Delta, \langle b^+ \rangle}{U}}
{\mildseq{\Gamma, \langle a^+ \rangle}{\Delta, \langle c^+ \rangle}{U}}
\]\[
\infer
{\mildseq{\Gamma, \langle b^+ \rangle}{\Delta, \langle a^+ \rangle}{U}}
{\mildseq{\Gamma, \langle b^+ \rangle}{\Delta, \langle c^+ \rangle}{U}}
\quad
\infer
{\mildseq{\Gamma, \langle a^+ \rangle, \langle b^+ \rangle}{\Delta}{U}}
{\mildseq{\Gamma, \langle a^+ \rangle, \langle b^+ \rangle}
   {\Delta, \langle c^+ \rangle}{U}}
\]
Andreoli suggests coping with this problem by restricting the form of
propositions so that positive atoms never appear in the persistent
context. From our perspective, this is a rather unusual
recommendation, since it just returns us to linear logic without the
atom optimization!

The proliferation of inference rules under the atom optimization is a
problem if, for instance, we need to represent synthetic inference
rules on a computer. Correctly viewed, however, the problem is merely
one of notation. It's already the case that, in writing sequent
calculus rules, we generally tacit use of a fairly large number of
notational conventions, at least relative to Gentzen's original
formulation where all contexts were treated as sequences of
propositions \cite{gentzen35untersuchungen}.  For instance, the
bottom-up reading of the ${\one}_R$ rule's conclusion,
$\mildseq{\Gamma}{\cdot}{[\one]}$, indicates the presence of an
additional premise checking that the linear context is empty, and the
conclusion $\mildseq{\Gamma}{\Delta_1, \Delta_2}{[A \otimes B]}$ of
the ${\tensor}_R$ rule indicates the condition that the context can be
split into two parts.

We deal with the apparent proliferation of rules by adding a new
matching construct for the conclusion of rules: we can say that
$\Gamma; \Delta$ matches $\Gamma; \Delta' / \langle p^+ \rangle$
either when $\langle p^+ \rangle \in \Gamma$ and $\Delta = \Delta'$ or
when $\Delta = \Delta', \langle p^+ \rangle$. We can also iterate
this construction, so that $\Gamma; \Delta$ matches
$\Gamma; \Delta_n / \langle p^+_1 \rangle, \ldots, \langle p^+_n \rangle$
if $\Gamma; \Delta$ matches $\Gamma; \Delta_1 / \langle p^+_1 \rangle$,
$\Gamma; \Delta_1$ matches $\Gamma; \Delta_2 / \langle p^+_2 \rangle$,
\ldots and $\Gamma; \Delta_{n-1}$ matches 
$\Gamma; \Delta_n / \langle p^+_n \rangle$.  Armed with this notation,
we can create a concise synthetic connective:
\[
\infer
{\mildseq{\Gamma}{\Delta/\langle a^+ \rangle, \langle b^+ \rangle}{U}}
{\mildseq{\Gamma}{\Delta, \langle c^+ \rangle}{U}}
\]

This modified notation need not only be used in synthetic connectives: we 
can also use it to combine the two positive identity rules. Furthermore, 
by giving $\Gamma; \Delta / A^-$ the obviously analogous me meaning, we can
fuse the ${\it focus}_L$ rule and the ${\it copy}_L$ rule into a single
rule that is unconcerned with whether the proposition in question came
from the persistent or linear contexts: 
\[
\infer[{\it id}^+]
{\mildseq{\Gamma}{\cdot/\langle A^+ \rangle}{[A^+]}}
{}
\quad
\infer[{\it focus}^*_L]
{\mildseq{\Gamma}{\Delta/A^-}{U}}
{\mildseq{\Gamma}{\Delta, [A^-]}{U}}
\]

\begin{figure}
\[
\infer[{\it init}]
{\altseq{\Gamma}{\cdot/p}{p}}
{}
\]

\[
\infer[{!}_R]
{\altseq{\Gamma}{\cdot}{{!}A}}
{\altseq{\Gamma}{\cdot}{A}}
\quad
\infer[{!}_L]
{\altseq{\Gamma}{\Delta/{!}A}{C}}
{\altseq{\Gamma, A}{\Delta}{C}}
\quad
\infer[{\one}_R]
{\altseq{\Gamma}{\cdot}{\one}}
{}
\quad
\infer[{\one}_L]
{\altseq{\Gamma}{\Delta/{\one}}{C}}
{\altseq{\Gamma}{\Delta}{C}}
\]

\[
\infer[{\otimes}_R]
{\altseq{\Gamma}{\Delta_1, \Delta_2}{A \otimes B}}
{\altseq{\Gamma}{\Delta}{A} & \altseq{\Gamma}{\Delta}{B}}
\quad
\infer[{\otimes}_L]
{\altseq{\Gamma}{\Delta/A \otimes B}{C}}
{\altseq{\Gamma}{\Delta, A, B}{C}}
\]

\[
\infer[{\lolli}_R]
{\altseq{\Gamma}{\Delta}{A \lolli B}}
{\altseq{\Gamma}{\Delta, A}{B}}
\quad
\infer[{\lolli}_L]
{\altseq{\Gamma}{\Delta_1, \Delta_2 / A \lolli B}{C}}
{\altseq{\Gamma}{\Delta_1}{A}
 &
 \altseq{\Gamma}{\Delta_2, B}{C}}
\]

\caption{Alternate presentation of intuitionstic linear logic}
\label{fig:linear-alt}
\end{figure}




Going yet one more step, we could use this notation to revise
the original definition of linear logic in Figure~\ref{fig:linear}.
The {\it copy} rule in that presentation sticks out as the only 
rule that doesn't deal directly with a connective, but we can eliminate
it by using the $\Gamma; \Delta/A$ matching construct. The resulting
presentation, shown in Figure~\ref{fig:linear-alt}, is equivalent
to the presentation in Figure~\ref{fig:linear}.

\bigskip
\begin{theorem}
$\seq{\Gamma}{\Delta}{C}$ if and only if $\altseq{\Gamma}{\Delta}{C}$.
\end{theorem}

\begin{proof}
The reverse direction is a straightforward induction: each rule in 
Figure~\ref{fig:linear-alt} can be translated as the related rule
in Figure~\ref{fig:linear} along with (potentially) an instance of 
the ${\it copy}$ rule.

The forward direction requires a lemma that the {\it copy} rule is
admissible according to the rules of Figure~\ref{fig:linear-alt}; this
lemma can be established by straightforward induction. Having
established the lemma, the forward direction is a straightforward
induction on derivations, applying the admissibility lemma whenever the 
{\it copy} rule is encountered.
\end{proof}




\section{Concurrent equality}
\label{sec:linconcurrenteq}

Concurrent equality, is a
notion of equivalence that operates on synthetic derivations.
It represents an intermediate point between focusing and
multifocusing \cite{chaudhuri08canonical}. 
Consider the sequent in focused linear logic:
\[
\mildseq{a^+ \lolli {\uparrow}(b^+ \otimes c^+), ~
  b^+ \lolli {\uparrow}d^+, ~
  c^+ \lolli {\uparrow}e^+, ~
  d^+ \otimes e^+ \lolli {\uparrow}f^+ ~~}
  {~~
  \langle a^+ \rangle
  ~~}
  {~~f^+}
\]
Let $\Gamma = \left(a^+ \lolli {\uparrow}(b^+ \otimes c^+), ~
  b^+ \lolli {\uparrow}d^+, ~
  c^+ \lolli {\uparrow}e^+, ~
  d^+ \otimes e^+ \lolli {\uparrow}f^+ \right)$.
There are two different focused derivations of this
sequent, the one that transitions $\langle b^+ \rangle$ to $\langle
d^+ \rangle$ first, and the one that transitions 
$\langle c^+ \rangle$ to $\langle e^+ \rangle$ first:
\[
\infer
{\mildseq{\Gamma}{\langle a^+ \rangle}{f^+}}
{\infer
{\mildseq{\Gamma}{\langle b^+ \rangle, \langle c^+ \rangle}{f^+}}
{\infer
{\mildseq{\Gamma}{\langle d^+ \rangle, \langle c^+ \rangle}{f^+}}
{\infer
{\mildseq{\Gamma}{\langle d^+ \rangle, \langle e^+ \rangle}{f^+}}
{\infer
{\mildseq{\Gamma}{\langle f^+ \rangle}{f^+}}
{}}}}}
\qquad
\deduce
{\mathstrut}
{\deduce
{\mathstrut}
{\deduce
{\mathstrut}
{\mbox{\it vs.}}}}
\qquad
\infer
{\mildseq{\Gamma}{\langle a^+ \rangle}{f^+}}
{\infer
{\mildseq{\Gamma}{\langle b^+ \rangle, \langle c^+ \rangle}{f^+}}
{\infer
{\mildseq{\Gamma}{\langle b^+ \rangle, \langle e^+ \rangle}{f^+}}
{\infer
{\mildseq{\Gamma}{\langle d^+ \rangle, \langle e^+ \rangle}{f^+}}
{\infer
{\mildseq{\Gamma}{\langle f^+ \rangle}{f^+}}
{}}}}}
\]
If we think about these two proofs in terms of the series of
transitions they embody, it's not so clear we want to think of them as
different. In both cases, there is an $a^+$ resource that transitions
to a $b^+$ resource and a $c^+$ resource, and then $b^+$ transitions
to $d^+$ while, independently, the $c^+$ transitions to $e^+$. Then,
finally, the $d^+$ and $e^+$ combine to transition to $f^+$, which
completes the trace. The independence here is key: if two focusing
phases consume different resources and both end focus with
${\uparrow}_L$ (as opposed to ${\it id}^-$), then we can treat them as
independent and concurrent steps in the process of proving the same
right hand side; {\it concurrent equality} is the equivalence relation
on focused proofs that treats all proofs that differ only in the
interleaving of independent and concurrent steps as equal.  This
equivalence relation was used in the definition of CLF
\cite{watkins02concurrent}, but only in conjunction with the lax
modality, an approach we will follow in Chapter 4.

The equivalence relation on focused derivations that concurrent
equality gives rise to is related to the equivalence relation induced
by {\it multifocusing} \cite{chaudhuri08canonical}.  Multifocusing
appears to provide an even coarser notion of equivalence on focused
proofs than concurrent equality does. In particular, these two
distinct focusing proofs are not concurrently equal: the proof
on the right succeeds at proving $\langle c^- \rangle$ in one step,
but leaves a subgoal in which $b^+$ is proved indirectly, whereas the
proof at the right first transitions from having $\langle a^+ \rangle$
and $a^+ \lolli {\uparrow} b^+$ resources to having a $\langle b^+
\rangle$ resource, and only then proves $\langle c^- \rangle$, leaving
a subgoal in which $b^+$ is proved directly.
\[
\infer
{\mildseq{\cdot}
  {~~
   \langle a^+ \rangle, ~
   a^+ \lolli {\uparrow}b^+, ~
   {\downarrow}{\uparrow}b^+ \lolli c^-
   ~~}
  {~~\langle c^- \rangle}}
{\infer
{\mildseq{\cdot}
  {~~
   \langle a^+ \rangle, ~
   a^+ \lolli {\uparrow}b^+
   ~~}
  {b^+}}
{\infer
{\mildseq{\cdot}
  {~~
   \langle b^+ \rangle
   ~~}
  {b^+}}
{}}}
\deduce{\mathstrut}
{\deduce{\mathstrut}
{\mbox{\it vs.}\mathstrut}}
\infer
{\mildseq{\cdot}
  {~~
   \langle a^+ \rangle, ~
   a^+ \lolli {\uparrow}b^+, ~
   {\downarrow}{\uparrow}b^+ \lolli c^-
   ~~}
  {~~\langle c^- \rangle}}
{\infer
{\mildseq{\cdot}
  {~~
   \langle b^+ \rangle, ~
   {\downarrow}{\uparrow}b^+ \lolli c^-
   ~~}
  {~~\langle c^- \rangle}}
{\infer
{\mildseq{\cdot}
  {~~
   \langle b^+ \rangle
   ~~}
  {~~b^+}}
{}}}
\]
While there is no full account of multifocusing in intuitionistic
logic, the analogue of this sequent in classical linear logic has only
one multifocused proof. In classical linear logic, multifocusing
offers a very fundamental normal form: any two proofs that can be made
equal by locally permuting inference rules have the same multifocused
proof. 

CLF's concurrent equality, restricted to an association with the lax
modality, will be sufficient for the logical framework in Chapter
4. In fact, for the fragment of the the logic in Chapter 3 that
comprises our logical framework in Chapter 4, I conjecture that
concurrent equality and the equality given by multifocusing
coincide.\footnote{This obviously means that the example above will be
  outside the logical that comprises the logical framework.}  This
conjecture is obviously difficult to make precise, much less prove,
without a general theory of multifocusing in intuitionistic logic.


\subsection{A warning about normalization}
\label{sec:warning}

In our earlier discussion of hereditary substitution and canonical
forms, we mentioned that the normalization theorem provided by
hereditary substitution was weaker than the so-called weak
normalization theorem for LF. That is because the weak normalization
theorem says that any well-typed term can be converted into a
canonical ($\beta$-normal and $\eta$-long) term by a particular series
of $\beta$ and $\eta$ conversions. It is self-evident, by this statement
of the theorem, that the resulting canonical term is equivalent
to the original term. 

On the other hand, when we use hereditary substitution in the obvious
way to obtain a Canonical LF term from an arbitrary non-canonical LF
term, we gain {\it no guarantees} about the relationship between the
non-canonical LF term and the Canonical LF term. The statement of the
theorem does not preclude taking a $\beta$-normal, $\eta$-long LF term
(like $\lambda x. \lambda y. x$ of type $p \rightarrow p \rightarrow
p$ for some atomic type $p$) into a structurally different Canonical
LF term (like $\lambda x. \lambda y. y$, which also has type $p
\rightarrow p \rightarrow p$). It is possible to gain such a guarantee
for LF, as Martens and Crary have shown in unpublished work
\cite{martens11mechanizing}, but this result is a non-trivial statement
about the constructive content of the normalization theorem. 

In our setting, we should be concerned that we might take a focused
proof, turn it into an unfocused proof by way of the obvious
translation (the constructive content of
Theorem~\ref{thm:linfocsound}), and then turn it back into a focused
proof by focalization (the constructive content of
Theorem~\ref{thm:linfoccomplete}) only to obtain a proof that was not
identical or even related. This is not at all a merely hypothetical
concern. We can run the mechanized structural focalization result from
\cite{simmons11structural} on an persistent proposition,
%
   $a^+ \supset 
   {\downarrow}(a^+ \supset {\uparrow}b^+) \supset
   {\downarrow}({\downarrow}{\uparrow}b^+ \lolli c^-) \supset
   c^-$, 
%
that is similar to the example from
Section~\ref{sec:linconcurrenteq}.  In persistent logic (as in
linear logic) that proposition has two focused propositions that
are probably multifocusing equivalent (given a reasonable intuitionistic
notion of multifocusing) but that are not concurrently equivalent
under the proposed definition of concurrent equality. 
However, if we take the focused proof that focuses 
first on $a^+ \supset {\uparrow}b^+$, transform it into an unfocused 
proof, and then re-focus it, we will get the proof that focuses 
first on ${\downarrow}{\uparrow}b^+ \supset c^-$. Focalization,
in other words, is not a partial inverse of erasure in the structural
focalization development, except maybe modulo the (as yet undefined)
equivalence relation established by multifocusing. 

This illustrates why we must be careful, but it is not a fatal flaw
for two reasons. The first is the aforementioned conjecture
that, for the restricted logical fragment defined in
Chapter 4 as the basis of our logical framework, the focalizations of 
two proofs are concurrently
equal if and only if the original proofs are convertible by local
permutations of rules, the same condition that
multifocusing satisfies; it ought to be the case that
focalization is a partial inverse of erasure modulo this courser
equivalence. Second, what is really at stake here is our
ability to write down non-normal proofs in a logical framework that
then normalizes them -- which is what the Twelf implementation of LF
and the Celf implementation of CLF do -- with the confidence that we
can look at a non-normal proof and know its corresponding
canonical form. In this thesis, we will be content to work throughout
with focused proofs and their analogues, so we can afford to leave
questions about convertability and weak normalization to future work.
