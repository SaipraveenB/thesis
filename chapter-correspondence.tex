\chapter{On logical correspondence}

In Part 1, we defined \sls, the logical framework of substructural
logical specifications. For the purposes of this thesis, we are
primarily interested in using \sls~as a framework for specifying the
operational semantics of programming languages, especially stateful
and concurrent programming languages. This is not a new idea: one of
the original case studies on CLF specification described the semantics
of Concurrent ML \cite{cervesato02concurrent} in a specification style
termed {\it substructural operational semantics} by Pfenning
\cite{pfenning04substructural}. The general idea of representing the
intermediate states of a computation as substructural process states
dates back to Miller \cite{miller92pi} and his Ph.D. student Chirimar
\cite{chirimar95proof}, who encoded the intermediate states of a
$\pi$-calculus and of a low-level RISC machine (respectively) as
contexts in focused classical linear logic.

The logical framework \sls~in general provides an extremely rich set
of tools for specifying properties of programming langauges. In this
chapter and the next chapter, we will consider three styles of
specification that can be adequately represented in the
\sls~framework; each specification style is strictly more expressive
than the last.

\begin{itemize}
\item The {\it natural semantics}, or big-step operational semantics,
  is an existing well-known specification style that is convienent for
  the specification of pure and determinstic programming
  languages.

\item The {\it ordered abstract machine semantics} is a generalization
  of abstract machine semantics that can be naturally specified in
  \sls; this specification style naturally handles stateful and
  parallel programming language features, as well as nondeterminism
  \cite{pfenning09substructural}.

\item The {\it destination-passing semantics} is the style of
  substructural operational semantics first explored in CLF by
  Cervesato et al.~\cite{cervesato02concurrent}. It allows for the
  natural specification of features that incorporate syncronous
  communication and non-local transfer of control.
\end{itemize}

\noindent
The statement that each specification style is strictly more
expressive than the last is formal: natural semantics can be
mechnically transformed to ordered abstract machine semantics, and
ordered abstract machine semantics can be mechanically transformed
into destination-passing semantics, by fully atomatic and
provably-correct transformations. These transformations are, in turn,
instances of general transformations on \sls~specifications. In this
chapter, we explore the first transformation, which transforms natural
semantics to ordered abstract machine semantics. The transformation
that does so is a general transformation on deductive
\sls~specifications that operationalizes deductive proof search as
concurrent proof search.

\subsection*{The logical correspondance as an organizing principle}

In order to make sense of the design space of substructural
operational semantics, we need some design principles that allow us to
both {\it classify} different styles of presentation and {\it predict}
what style(s) we should adopt based on what our goals are.  The
organizing principle that we put forth in this chapter and the next
one is called the {\it logical correspondence}, in analogy.

Using general, provabably correct transformations on logical
specifications is the 


: the goal of research into logical frameworks can, at some level  is to find {\it
  right tool for the job}, and to create effe

In the world of
programming language research, the key is to 


, can argue that we should just do all of our work using
the destination passing semantics, as .

, and one that forms the
core of the idea that we will use to classify 


In this chapter and the subsequent chapter
we discuss three ; 

\noindent
Only the latter two of these specefication styles are properly
classified as substructural operational semantics. While natural
semantics can be encoded in \sls, they do not require any of its
features, and can be encoded just as well in LF or the purely
persistent, non-modal fragment of \sls. Furthermore, natural semantics
is not very good as an {\it operational semantics}, as it describes
{\it what} a program may compute, but not how: many issues, such as
the order of operations, are unspecified. 

When we say that each of the three specification styles is strictly
more expressive than the last, this is true in a completely formal
sense: in this chapter, we show that natural semantics can be turned
into ordered abstract machine semantics by a general, provably-correct
transformation on logical sepecifications. These 

In fact, we will describe
three such transformations, one 

 we will explore three styles of specification,
where each is strictly more expressive than the last.

. The first style, 



\subsection*{Functional correspondence}

To make a carefully-chosen analogy, various on-paper styles of
specifying the operational semantics are well-known and visually
recongizable.  To anyone literate in the conventions of programming
languages researchers, a quick glance should be sufficent to classify
the following two rules as natural semantics (or big-step semantics)
for the call-by-value lambda calculus:
\[
\infer
{\lambda x. e \Downarrow \lambda x. e \mathstrut}
{}
\quad
\infer
{e_1\,e_2 \Downarrow v \mathstrut}
{e_1 \Downarrow \lambda x.e
 &
 e_2 \Downarrow v_2
 &
 [v_2/x]e \Downarrow v \mathstrut}
\]
Natural semantics are a clean, high-level, and declarative way of
describing the semantics of a simple, pure programming langauges, but
they do not scale particularly well with the addition of effects like
state and exceptions. Worse, natural semantics are mostly hopeless in
the face of languages features that incoroprate nondeterminism or
advanced control (such as first-class continuations). 

Thus, a researcher interested in a simple, high level specification of
the core features of a functional programming language might
reasonably predict that natural semantics would be a good solution to
their problem; one example is Murphy VII, who used natural semantics
for the high-level formalization of Lambda 5 in his thesis
\cite{murphy08modal}.

Another style used to specify the operational semantics of programming
languages, he {\it abstract
  machine} semantics, is slightly less canonical but nevertheless has
an identifiable set of conventions. The following is an abstract 
machine semantics for our call-by-value lambda calculus:
\[
\infer
{k \rhd (\lambda x. e) ~\mapsto~ k \lhd (\lambda x. e) \mathstrut} 
{}
\quad
\infer
{k \rhd (e_1\,e_2) ~\mapsto~ (k; \Box\,e_2) \rhd e_1 \mathstrut}
{}
\]\[
\infer
{(k; \Box\,e_2) \lhd (\lambda x.e) ~\mapsto~ (k; (\lambda x.e)\Box) \rhd e_2
 \mathstrut}
{}
\quad
\infer
{(k; (\lambda x.e)\Box) \lhd v_2 ~\mapsto~ k \rhd [v_2/x]e
 \mathstrut}
{}
\]

 for programming languages, t, is slightly less canonical, but abstract machine
specifications nevertheless also have a set of common
conventions. There are two states in an abstract machine
specification, $k \rhd e$ (the expression $e$ is evaluating on top of
stack $k$) and $k \lhd v$ (the value $v$ is being returned to the top
of the stack $k$). The first rule says that, as a funtion $\lambda x.e$
is already a value, we proceed by returning it to the stack, whereas
for an application $e_1\,e_2$, 


There needs to be some structure and
at least


the design space
of 

 the general idea
of representing intermedeiate states of a computation as 
substructural process states 

This chapter represents joint work with Ian Zerny.

\section{Logical transformation: compilation}

Deductive computation versus concurrent computation 
Section~\ref{sec:framework-logicprog}

Deductive computation

\begin{figure}

%\fvset{fontsize=\small,boxwidth=229pt}
\fvset{fontsize=\small,boxwidth=187pt}
\BVerbatimInput{sls/cbv-ev.sls}
\fvset{fontsize=\small,boxwidth=auto}
\BVerbatimInput{sls/cbv-ev-ssos.sls}

\caption{A natural semantics for CBV and the corresponding abstract machine.}
\end{figure}

\subsection{Tail-recursion}

\subsection{Parallelism}

\section{Logical transformation: defunctionalization}

\section{Logical transformation: factoring}

Example: exceptions

\section{Exploring the richer fragment}

\subsection{Mutable storage}
\label{sec:mutable-storage}

No check for pointer inequality! This is a fundamental restriction of
the fact that we're using existential quantificaiton rather than some
form of nominal quantification. (Hack due to Favonia and Bob, personal
communication.)

\subsection{Call-by-need}

\subsection{Environment semantics}

\subsection{Looking back at natural semantics}
\label{sec:enriching-natsem}

\section{Partial transformation}


\subsection{Evaluation contexts}

Thus far, we have considered big-step operational semantics and abstract
machines, neglecting the third great tradition of programming language
specification, {\it structural operational semantics}. Structural
operational semantics (SOS) define single-step evaluation inductively over
the structure of expressions; the SOS semantics for our running example
language is the following:
\[
\infer
{\lambda x.e\,{\sf value} \mathstrut}
{}
\quad
\infer
{e_1\,e_2 \mapsto e_1'\,e_2 \mathstrut}
{e_1 \mapsto e_1' \mathstrut}
\quad
\infer
{e_1\,e_2 \mapsto e_1\,e_2' \mathstrut}
{e_1\,{\sf value}
 &
 e_2 \mapsto e_2' \mathstrut}
\quad
\infer
{(\lambda x. e)v \mapsto [v/x]e \mathstrut}
{v\,{\sf value} \mathstrut}
\]
This inductive specification is adequately encoded on the left-hand
side of Figure~\ref{fig:cbv-sos}, along with the proposition \Verb|ev|
that describes a big-step operational semantics in terms of repeated
application of the small-step operational semantics.

\begin{figure}[tp]
\fvset{fontsize=\small,boxwidth=229pt}
\BVerbatimInput{sls/cbv-sos.sls}
\BVerbatimInput{sls/cbv-sos-eval.sls}
\caption{Small-step evaluation, and one corresponding abstract machine.}
\label{fig:cbv-sos}
\end{figure}

\fvset{fontsize=\small}

There are a couple of possibilities for how the 
One obvious way to proceed is to simply translate the big-step portion
of our semantics as encoded 


If we just translate the {\it steps} portion of the semantics (using
the tail-recursion optimizing translation), then we will get what is
probabily fair to call the most boring possible substructural
operational semantics: 

\smallskip
\VerbatimInput{sls/cbv-sos-proc.sls}
\smallskip

\noindent
Under this semantics, the substructural context contains a single
resource, \Verb|eval-steps(E)|, which takes steps according to the
rules of the small-step structural operational semantics until a value
is reached, at which point the context contains \Verb|retn-steps(V)|.


\begin{figure}[t]
\VerbatimInput{sls/cbv-sos-defun.sls}
\caption{The defunctionalized abstract machine from Figure~\ref{fig:cbv-sos}.}
\label{fig:cbv-sos-defun}
\end{figure}

The interesting observations are to be had from the other direction: what if

\subsection{Temporal logic}

The natural semantics of \rowan~are not, on a superficial level,
significantly more complex than other natural semantics. However, it
turns out that the usual set of techniques for adding state to a
natural semantics break down, and discussing a \rowan-like logic with
state remained a challenge for many years.\robnote{Figure out from
  Rowan what the recent work he told you about was.} Through the
logical correspondance, it is easy to see why: the natural SSOS
specification of \rowan~integrates both concurrent and deductive
reasoning in an arbitrarily nested way. In fact, Figure XXX is the
only SLS specification in this thesis that exhibits this form of
recursive dependency between concurrent and deductive reasoning.  In
particular, the \rowan~specification is way out of the image of the
extended natural semantics we considered in
Section~\ref{sec:enriching-natsem}. The natural encoding in state lies
in the ambient substructural context of a concurrent computation, but
that ambient computation cannot properly enter into a deductive
sub-computation. If we tried to add state to \rowan~the same way we
added it in Section~\ref{sec:mutable-storage}, the entire store
would effectively leave scope whenever computation considered
the subterm $e$ of ${\sf next}(e)$. That consideration happens
as deductive reasoning, not as concurrent reasoning!

 it is the only we
will consider in this thesis that has with property.

It's hard to include state in temporal logic! But the logical correspondence
helps us understand why: the natural SSOS specification of 