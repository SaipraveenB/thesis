\usepackage{times}
\usepackage{dashrule}
\usepackage{proof-dashed}
\usepackage{fullpage}
\usepackage{graphicx}
\usepackage{amsthm}
\usepackage{amsmath}
\usepackage{amssymb}
\usepackage{latexsym}
\usepackage{amssymb}            % for \multimap (-o)
\usepackage{stmaryrd}           % for \binampersand (&), \bindnasrepma (\paar)
\usepackage{wasysym}            % for \ocircle
\usepackage[numbers,sort]{natbib}
\usepackage[backref,pageanchor=true,plainpages=false, pdfpagelabels, bookmarks,bookmarksnumbered,
%pdfborder=0 0 0,  %removes outlines around hyper links in online display
]{hyperref}
\usepackage{subfigure}

% Approximately 1" margins, more space on binding side
%\usepackage[letterpaper,twoside,vscale=.8,hscale=.75,nomarginpar]{geometry}
%for general printing (not binding)
\usepackage[letterpaper,twoside,vscale=.8,hscale=.75,nomarginpar,hmarginratio=1:1]{geometry}

% Provides a draft mark at the top of the document. 

\hypersetup{colorlinks=true,citecolor=blue,urlcolor=blue,linkcolor=black}

\newcommand{\robnote}[1]{\footnote{{\bf NOTE TO SELF:} ~~ {#1}}}
\newcommand{\futurework}[1]{}

\newcommand{\ollll}{OL$_3$}

% Theorems
\newtheorem{theorem}{Theorem}
\newtheorem{proposition}{Proposition}
\newtheorem{definition}{Definition}

% symbols of linear logic
\newcommand{\lolli}{\multimap}
\newcommand{\tensor}{\otimes}
\newcommand{\with}{\mathbin{\binampersand}}
\newcommand{\paar}{\mathbin{\bindnasrepma}}
\newcommand{\one}{\mathbf{1}}
\newcommand{\zero}{\mathbf{0}}
\newcommand{\bang}{{!}}
\newcommand{\pbang}{\mbox{\hspace{2pt}$\mathbb !$\hspace{-5pt}$^+$\hspace{-1pt}}}
\newcommand{\whynot}{{?}}
\newcommand{\bilolli}{\mathrel{\raisebox{1pt}{\ensuremath{\scriptstyle\circ}}{\lolli}}}
% \oplus, \top, \bot
\newcommand{\deupdown}{\mbox{${\uparrow}{\downarrow}\hspace{-11.4pt}\diagup$}}
\newcommand{\dedownup}{\mbox{${\downarrow}{\uparrow}\hspace{-11.4pt}\diagdown$}}

\newcommand{\restrictto}[2]{\ensuremath{{#1}{\upharpoonright}_{#2}}}
\newcommand{\restrictfrom}[2]{\ensuremath{{#1}{\downharpoonright}^{#2}}}

% symbols of ordered logic
\newcommand{\fuse}{\mathbin{\bullet}}
\newcommand{\righti}{\twoheadrightarrow}
\newcommand{\lefti}{\rightarrowtail}
\newcommand{\gnab}{\mbox{\textexclamdown}}

\newcommand{\mlax}{{\it lax}}
\newcommand{\mtrue}{{\it ord}}
\newcommand{\meph}{{\it eph}}
\newcommand{\mpers}{{\it pers}}
\newcommand{\mlvl}{{\it lvl}}

\newcommand{\islax}[1]{{#1}\,{\mlax}}
\newcommand{\istrue}[1]{{#1}\,{\mtrue}}
\newcommand{\iseph}[1]{{#1}\,{\meph}}
\newcommand{\ispers}[1]{{#1}\,{\mpers}}
\newcommand{\islvl}[1]{{#1}\,{\mlvl}}

% judgments of linear logic
\newcommand{\seq}[3]{{#1};{#2} \longrightarrow {#3} \mathstrut}
\newcommand{\altseq}[3]{{#1};{#2} \Longrightarrow {#3} \mathstrut}
\newcommand{\pseq}[2]{{#1} \longrightarrow {#2} \mathstrut}

\newcommand{\mildseq}[3]{{#1};{#2} \vdash {#3} \mathstrut}
\newcommand{\andseq}[3]{{#1};{#2} \Vdash {#3} \mathstrut}
\newcommand{\mildrfoc}[3]{{#1};{#2} \vdash [{#3}] \mathstrut}
\newcommand{\mildinv}[3]{{#1};{#2} \vdash {#3} \mathstrut}
\newcommand{\mildlfoc}[4]{{#1};{#2}, [{#3}] \vdash {#4} \mathstrut}

\newcommand{\foc}[3]{{#1};{#2} \vdash {#3}}
\newcommand{\rfoc}[3]{{#1};{#2} \vdash [{#3}]}
\newcommand{\ifoc}[4]{{#1};{#2} {\mid} {#3} \vdash {#4}}
\newcommand{\lfoc}[4]{{#1};{#2}[{#3}] \vdash {#4}}

\newcommand{\foct}[4]{{#1}; {#2} \vdash {#3} : {#4}}
\newcommand{\rfoct}[4]{{#1};{#2} \vdash {#3} : [{#4}]}
\newcommand{\lfoct}[4]{{#1};{#2}[{#3}] \vdash {#3} : [{#4}]}

\newcommand{\tfocusr}[1]{{#1}}
\newcommand{\tfocusl}[2]{{#1} \circ {#2}}
\newcommand{\tetan}[1]{\langle {#1} \rangle}
\newcommand{\tetapp}[2]{\langle {#1} \rangle_p . {#2} }
\newcommand{\tetapl}[2]{\langle {#1} \rangle_l . {#2} }
\newcommand{\tetap}[2]{\langle {#1} \rangle . {#2}}
\newcommand{\tupr}[1]{{\uparrow}{#1}}
\newcommand{\tupl}[1]{{\uparrow}{#1}}
\newcommand{\tlaxr}[1]{\{{#1}\}}
\newcommand{\tlaxl}[1]{\{{#1}\}}
\newcommand{\tdownr}[1]{{\downarrow}{#1}}
\newcommand{\tdownl}[2]{{\downarrow}{#1}.{#2}}
\newcommand{\tbangr}[1]{{!}{#1}}
\newcommand{\tbangl}[2]{{!}{#1}.{#2}}
\newcommand{\tgnabr}[1]{{\gnab}{#1}}
\newcommand{\tgnabl}[2]{{\gnab}{#1}.{#2}}
\newcommand{\tnil}{\textsc{nil}}
\newcommand{\tabort}{\textsc{abort}}
\newcommand{\tinl}[1]{\textsc{inl}({#1})}
\newcommand{\tinr}[1]{\textsc{inr}({#1})}
\newcommand{\toplusl}[2]{[{#1},{#2}]}
\newcommand{\toner}{()}
\newcommand{\ttopr}{\top}
\newcommand{\tfuser}[2]{{#1} \fuse {#2}}
\newcommand{\twithr}[2]{{#1} \with {#2}}
\newcommand{\tpione}[1]{\pi_1;{#1}}
\newcommand{\tpitwo}[1]{\pi_2;{#1}}
\newcommand{\tfusel}[1]{{\fuse}{#1}}
\newcommand{\tlamr}[1]{{\lambda}^{>}{#1}}
\newcommand{\tappr}[2]{{#1}^{>}{#2}}
\newcommand{\tlaml}[1]{{\lambda}^{<}{#1}}
\newcommand{\tappl}[2]{{#1}^{<}{#2}}
\newcommand{\tonel}[1]{().{#1}}
\newcommand{\Sp}{{\it Sp}}
\newcommand{\texistsr}[2]{{#1},{#2}}
\newcommand{\texistsl}[2]{{#1}.{#2}}
\newcommand{\tforallr}[2]{[{#1}].{#2}}
\newcommand{\tforalll}[2]{[{#1}]; {#2}}
\newcommand{\tunifr}{{\doteq}}

\newcommand{\susp}[1]{\langle {#1} \rangle}

\newcommand{\ofirstseq}[5]{{#1};{#2};{#3};{#4} \Longrightarrow {#5} \mathstrut}
\newcommand{\oseq}[4]{{#1};{#2};{#3} \Longrightarrow {#4} \mathstrut}
\newcommand{\oiseq}[2]{\oseq{#1}{\cdot}{/{#2}/}{/{#2}/}}
\newcommand{\orseq}[4]{\oseq{#1}{#2}{{#3}}{/{#4}/}}
\newcommand{\orfseq}[4]{\ofirstseq{\Psi}{#1}{#2}{{#3}}{/{#4}/}}
\newcommand{\otseq}[4]{\oseq{#1}{#2}{{#3}}{\istrue{#4}}}
\newcommand{\olseq}[5]{\oseq{#1}{#2}{{#3}/{#4}/{#5}}{U}}
\newcommand{\olfseq}[5]{\ofirstseq{\Psi}{#1}{#2}{{#3}/{#4}/{#5}}{U}}
\newcommand{\opseq}[4]{\oseq{#1}{#2}{{#3},{#4}}{U}}
\newcommand{\opfseq}[4]{\ofirstseq{\Psi}{#1}{#2}{{#3},{#4}}{U}}

\newcommand{\mkunit}{\cdot}
\newcommand{\matchunit}{\cdot}
\newcommand{\mkconj}[2]{{#1},{#2}}
\newcommand{\matchconj}[2]{{#1},{#2}}

\newcommand{\invoff}[2]{{#1}\{{#2}\mbox\}}
\newcommand{\tackon}[2]{{#1}\{{#2}\}}
\newcommand{\frameoff}[2]{{#1}\mbox{$\{\hspace{-4pt}\{$}{#2}\mbox{$\}\hspace{-4pt}\}$}}

\newcommand{\urfoc}[3]{{#1};{#2} \longrightarrow [{#3}] \mathstrut}
\newcommand{\ulfoc}[4]{{#1};{#2} \,[#3] \longrightarrow {#4} \mathstrut}
\newcommand{\uinv}[4]{{#1};{#2};{#3} \longrightarrow {#4} \mathstrut}

\newcommand{\stableR}[1]{{#1}\,\mathit{stable_R} \mathstrut}
\newcommand{\stableL}[1]{{#1}\,\mathit{stable_L} \mathstrut}

\newcommand{\subst}[2]{{#1} \bullet {#2}}
\newcommand{\rsubsta}[4]{\llbracket {#1}/{#2} \rrbracket^{#4} {#3}}
\newcommand{\lsubsta}[3]{\llbracket {#1} \rrbracket^{#3} {#2}}
\newcommand{\rsubst}[3]{\rsubsta{#1}{#2}{#3}{}}
\newcommand{\lsubst}[2]{\lsubsta{#1}{#2}{}}

