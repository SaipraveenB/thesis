\usepackage{times}
\usepackage{dashrule}
\usepackage{proof-dashed}
\usepackage{fullpage}
\usepackage{graphicx}
\usepackage{amsmath}
\usepackage{amssymb}
\usepackage{latexsym}
\usepackage{amssymb}            % for \multimap (-o)
\usepackage{stmaryrd}           % for \binampersand (&), \bindnasrepma (\paar)
\usepackage{wasysym}            % for \ocircle
\usepackage[numbers,sort]{natbib}
\usepackage[backref,pageanchor=true,plainpages=false, pdfpagelabels, bookmarks,bookmarksnumbered,
%pdfborder=0 0 0,  %removes outlines around hyper links in online display
]{hyperref}
\usepackage{subfigure}

% Approximately 1" margins, more space on binding side
%\usepackage[letterpaper,twoside,vscale=.8,hscale=.75,nomarginpar]{geometry}
%for general printing (not binding)
\usepackage[letterpaper,twoside,vscale=.8,hscale=.75,nomarginpar,hmarginratio=1:1]{geometry}

% Provides a draft mark at the top of the document. 
\draftstamp{\today}{DRAFT}

\hypersetup{colorlinks=true,citecolor=blue,urlcolor=blue,linkcolor=black}


% Theorems
\newtheorem{theorem}{Theorem}
\newtheorem{proposition}{Proposition}

% symbols of linear logic
\newcommand{\lolli}{\multimap}
\newcommand{\tensor}{\otimes}
\newcommand{\with}{\mathbin{\binampersand}}
\newcommand{\paar}{\mathbin{\bindnasrepma}}
\newcommand{\one}{\mathbf{1}}
\newcommand{\zero}{\mathbf{0}}
\newcommand{\bang}{{!}}
\newcommand{\whynot}{{?}}
\newcommand{\bilolli}{\mathrel{\raisebox{1pt}{\ensuremath{\scriptstyle\circ}}{\lolli}}}
% \oplus, \top, \bot

% symbols of ordered logic
\newcommand{\fuse}{\mathbin{\bullet}}
\newcommand{\righti}{\twoheadrightarrow}
\newcommand{\lefti}{\rightarrowtail}
\newcommand{\gnab}{\mbox{\textexclamdown}}
\newcommand{\islax}[1]{{#1}~{\it lax}}
\newcommand{\istrue}[1]{{#1}~{\it true}}

% judgments of linear logic
\newcommand{\seq}[3]{{#1};{#2} \longrightarrow {#3} \mathstrut}

\newcommand{\mildseq}[3]{{#1};{#2} \vdash {#3} \mathstrut}
\newcommand{\mildrfoc}[3]{{#1};{#2} \vdash [{#3}] \mathstrut}
\newcommand{\mildinv}[3]{{#1};{#2} \vdash {#3} \mathstrut}
\newcommand{\mildlfoc}[4]{{#1};{#2}, [{#3}] \vdash {#4} \mathstrut}

\newcommand{\ofirstseq}[5]{{#1};{#2};{#3};{#4} \Rightarrow {#5} \mathstrut}
\newcommand{\oseq}[4]{{#1};{#2};{#3} \Rightarrow {#4} \mathstrut}
\newcommand{\oiseq}[2]{\oseq{#1}{\cdot}{/{#2}/}{/{#2}/}}
\newcommand{\orseq}[4]{\oseq{#1}{#2}{{#3}}{/{#4}/}}
\newcommand{\otseq}[4]{\oseq{#1}{#2}{{#3}}{\istrue{#4}}}
\newcommand{\olseq}[5]{\oseq{#1}{#2}{{#3}/{#4}/{#5}}{U}}
\newcommand{\opseq}[4]{\oseq{#1}{#2}{{#3},{#4}}{U}}



\newcommand{\urfoc}[3]{{#1};{#2} \longrightarrow [{#3}] \mathstrut}
\newcommand{\ulfoc}[4]{{#1};{#2} \,[#3] \longrightarrow {#4} \mathstrut}
\newcommand{\uinv}[4]{{#1};{#2};{#3} \longrightarrow {#4} \mathstrut}

\newcommand{\stableR}[1]{{#1}\,\mathit{stable_R} \mathstrut}
\newcommand{\stableL}[1]{{#1}\,\mathit{stable_L} \mathstrut}


