\usepackage{times}
\usepackage{dashrule}
\usepackage{proof-dashed}
\usepackage{fullpage}
\usepackage{graphicx}
\usepackage{amsthm}
\usepackage{amsmath}
\usepackage{amssymb}
\usepackage{latexsym}
\usepackage{amssymb}            % for \multimap (-o)
\usepackage{stmaryrd}           % for \binampersand (&), \bindnasrepma (\paar)
\usepackage{wasysym}            % for \ocircle
\usepackage[numbers,sort]{natbib}
\usepackage[backref,pageanchor=true,plainpages=false, pdfpagelabels, bookmarks,bookmarksnumbered,
%pdfborder=0 0 0,  %removes outlines around hyper links in online display
]{hyperref}
\usepackage{subfigure}

% Approximately 1" margins, more space on binding side
%\usepackage[letterpaper,twoside,vscale=.8,hscale=.75,nomarginpar]{geometry}
%for general printing (not binding)
\usepackage[letterpaper,twoside,vscale=.8,hscale=.75,nomarginpar,hmarginratio=1:1]{geometry}

% Provides a draft mark at the top of the document. 

\hypersetup{colorlinks=true,citecolor=blue,urlcolor=blue,linkcolor=black}

\newcommand{\robnote}[1]{\footnote{{\bf NOTE TO SELF:} ~~ {#1}}}
\newcommand{\futurework}[1]{}

\newcommand{\ollll}{OL$_4$}

% Theorems
\newtheorem{theorem}{Theorem}
\newtheorem{proposition}{Proposition}
\newtheorem{definition}{Definition}

% symbols of linear logic
\newcommand{\lolli}{\multimap}
\newcommand{\tensor}{\otimes}
\newcommand{\with}{\mathbin{\binampersand}}
\newcommand{\paar}{\mathbin{\bindnasrepma}}
\newcommand{\one}{\mathbf{1}}
\newcommand{\zero}{\mathbf{0}}
\newcommand{\bang}{{!}}
\newcommand{\pbang}{\mbox{\hspace{2pt}$\mathbb !$\hspace{-5pt}$^+$\hspace{-1pt}}}
\newcommand{\whynot}{{?}}
\newcommand{\bilolli}{\mathrel{\raisebox{1pt}{\ensuremath{\scriptstyle\circ}}{\lolli}}}
% \oplus, \top, \bot
\newcommand{\deupdown}{\mbox{${\uparrow}{\downarrow}\hspace{-11.4pt}\diagup$}}
\newcommand{\dedownup}{\mbox{${\downarrow}{\uparrow}\hspace{-11.4pt}\diagdown$}}

\newcommand{\restrictto}[2]{\ensuremath{{#1}{\downharpoonright}^{#2}}}

% symbols of ordered logic
\newcommand{\fuse}{\mathbin{\bullet}}
\newcommand{\righti}{\twoheadrightarrow}
\newcommand{\lefti}{\rightarrowtail}
\newcommand{\gnab}{\mbox{\textexclamdown}}

\newcommand{\mlax}{{\it lax}}
\newcommand{\mtrue}{{\it true}}
\newcommand{\meph}{{\it ephemeral}}
\newcommand{\mpers}{{\it persistent}}
\newcommand{\mlvl}{{\it lvl}}

\newcommand{\islax}[1]{{#1}\,{\mlax}}
\newcommand{\istrue}[1]{{#1}\,{\mtrue}}
\newcommand{\iseph}[1]{{#1}\,{\meph}}
\newcommand{\ispers}[1]{{#1}\,{\mpers}}
\newcommand{\islvl}[1]{{#1}\,{\mlvl}}

% judgments of linear logic
\newcommand{\seq}[3]{{#1};{#2} \longrightarrow {#3} \mathstrut}
\newcommand{\altseq}[3]{{#1};{#2} \Longrightarrow {#3} \mathstrut}
\newcommand{\pseq}[2]{{#1} \longrightarrow {#2} \mathstrut}

\newcommand{\mildseq}[3]{{#1};{#2} \vdash {#3} \mathstrut}
\newcommand{\andseq}[3]{{#1};{#2} \Vdash {#3} \mathstrut}
\newcommand{\mildrfoc}[3]{{#1};{#2} \vdash [{#3}] \mathstrut}
\newcommand{\mildinv}[3]{{#1};{#2} \vdash {#3} \mathstrut}
\newcommand{\mildlfoc}[4]{{#1};{#2}, [{#3}] \vdash {#4} \mathstrut}

\newcommand{\foc}[3]{{#1};{#2} \vdash {#3}}
\newcommand{\rfoc}[3]{{#1};{#2} \vdash [{#3}]}
\newcommand{\ifoc}[4]{{#1};{#2};{#3} \vdash {#4}}
\newcommand{\lfoc}[4]{{#1};{#2}[{#3}] \vdash {#4}}

\newcommand{\foct}[4]{{#1};{#2} \vdash {#3} : {#4}}

\newcommand{\ofirstseq}[5]{{#1};{#2};{#3};{#4} \Longrightarrow {#5} \mathstrut}
\newcommand{\oseq}[4]{{#1};{#2};{#3} \Longrightarrow {#4} \mathstrut}
\newcommand{\oiseq}[2]{\oseq{#1}{\cdot}{/{#2}/}{/{#2}/}}
\newcommand{\orseq}[4]{\oseq{#1}{#2}{{#3}}{/{#4}/}}
\newcommand{\orfseq}[4]{\ofirstseq{\Psi}{#1}{#2}{{#3}}{/{#4}/}}
\newcommand{\otseq}[4]{\oseq{#1}{#2}{{#3}}{\istrue{#4}}}
\newcommand{\olseq}[5]{\oseq{#1}{#2}{{#3}/{#4}/{#5}}{U}}
\newcommand{\olfseq}[5]{\ofirstseq{\Psi}{#1}{#2}{{#3}/{#4}/{#5}}{U}}
\newcommand{\opseq}[4]{\oseq{#1}{#2}{{#3},{#4}}{U}}
\newcommand{\opfseq}[4]{\ofirstseq{\Psi}{#1}{#2}{{#3},{#4}}{U}}

\newcommand{\invoff}[2]{{#1}\{{#2}\mbox\}}
\newcommand{\tackon}[2]{{#1}\{{#2}\}}
\newcommand{\frameoff}[2]{{#1}\mbox{$\{\hspace{-4pt}\{$}{#2}\mbox{$\}\hspace{-4pt}\}$}}

\newcommand{\urfoc}[3]{{#1};{#2} \longrightarrow [{#3}] \mathstrut}
\newcommand{\ulfoc}[4]{{#1};{#2} \,[#3] \longrightarrow {#4} \mathstrut}
\newcommand{\uinv}[4]{{#1};{#2};{#3} \longrightarrow {#4} \mathstrut}

\newcommand{\stableR}[1]{{#1}\,\mathit{stable_R} \mathstrut}
\newcommand{\stableL}[1]{{#1}\,\mathit{stable_L} \mathstrut}


