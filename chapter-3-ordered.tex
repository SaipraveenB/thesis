\chapter{Substructural logic}

Linear logic is the most famous of the {\it substructural logics}.
Persistent logic admits the three so-called {\it structural rules} of
weakening (premises need not be used), contraction (premises may be
used multiple times) and exchange (the ordering of premises are
irrelevant). Substructural logics, then, are logics that do not admit
these structural rules -- linear logic has only exchange, {\it affine}
logic (which is frequently conflated with linear logic by programming
language designers) has exchange and weakening, and {\it ordered}
logic, first investigated as a proof theory by Lambek
\cite{lambek58mathematics}, lacks all three.  

Calling logics like
linear, affine, and ordered logic \underline{sub}structural relative
to persistent logic (which is structural) is greatly unfair to the
substructural logics. Girard's linear logic can express persistent
provability using the exponential connective ${!}A$, and this idea is
generally applicable in substructural logics -- for instance, it was
applied by Polakow and Pfenning to Lambek's ordered logic
\cite{polakow99natural}. It is certainly too
late to advocate for these logics to be understood as
\underline{super}structural logics, but that is undoubtedly what they
are: generalizations of persistent logic that introduce more 
expressive power.

In this chapter, I define a first-order ordered linear logic with a
lax modality (henceforth \ollll, for ordered linear lax logic), in
both unfocused (Section~\ref{sec:ord-unfocused}) and focused
(Section~\ref{sec:ord-focused}) flavors. Then, following the
structural focalization methodology \cite{simmons11structural}, I
establish cut admissibility (Section~\ref{sec:ord-cut}), and identity
expansion (Section~\ref{sec:ord-identity}) for focused~\ollll; with
these results, it is possible to prove the soundness and completeness
of focusing (Section~\ref{sec:ord-correctness}) for \ollll.  A
fragment of this system will form the basis of the logical framework
in Chapter 4, and that framework will, in turn, underpin the rest of
this thesis.

It is worth clarifying why I am presenting the much richer \ollll~here
if only the fragment detailed in Chapter~4 is needed. There are two
main reasons.  First, while we will use only a fragment of this logic
in Chapter 4, other fragments of the logic may well be interesting and
useful for other purposes. Second, the presentation in this chapter,
and in particular the discussion of substructural contexts in
Section~\ref{sec:contexts}, introduces a presentation style and
infrastructure that I believe will generalize to richer logics,
such as the logic of bunched implications~\cite{pym02semantics},
non-commutative linear logic (or ``rigid logic'')
\cite{simmons09linear}, subexponential logics
\cite{nigam09algorithmic}, and so on. 

Furthermore, the choice to present a full account of focusing in
\ollll~is in keeping with as Andreoli's instance that we should seek,
where possible, to avoid the confusion that results from ambiguity as
to whether we are ``defining a foundational paradigm or a [logic]
programming language (two objectives that should clearly be kept
separate)'' \cite{andreoli01focussing}. Both the full logic \ollll~and
the general methodology followed in this chapter are general,
foundational paradigms within which it is possible to instantiate
families of logic programming languages and logical frameworks, even
though we will focus on the a particular logical framework starting in
Chapter 4.

\section{Ordered linear lax logic}
\label{sec:ord-unfocused}

Ordered linear logic was the subject of Polakow's thesis
\cite{polakow01ordered}, and the adaptation of Fairtlough and
Mendler's lax logic \cite{fairtlough95propositional} (as reconstructed
by Pfenning and Davies \cite{pfenning01judgmental}) to linear logic is
the basis of the CLF logical framework
\cite{watkins02concurrent}. Putting the pieces together is a
relatively straightforward proof-theoretic exercise. There are three
contexts relevant to the propositional presentation of ordered linear
lax logic.  The persistent context $\Gamma$ and the linear context
$\Delta$ are multisets as before (so we think of $\Delta_1, \Delta_2$
as being equal to $\Delta_2, \Delta_1$, for instance). The ordered
context $\Omega$ is a sequence of propositions, as in Gentzen's
original presentation of sequent calculi, and {\it not} a multiset.
This means that the two ordered contexts $\Omega_1, \Omega_2$ and
$\Omega_2, \Omega_1$ are, in general, not the same.

\begin{figure}
\small
{\it Atomic propositions}
\[
\infer[{\it init}^+]
{\oiseq{\Gamma}{p}}
{}
\]

\medskip
{\it Modalities}
\[
\infer[{\gnab}_R]
{\orseq{\Gamma}{\Delta}{\cdot}{{\gnab}A}}
{\otseq{\Gamma}{\Delta}{\cdot}{A}}
\quad
\infer[{\gnab}_L]
{\olseq{\Gamma}{\Delta}{\Omega_L}{{\gnab}A}{\Omega_R}}
{\opseq{\Gamma}{\Delta, A}{\Omega_L}{\Omega_R}}
\]
\vspace{-5pt}
\[
\infer[{!}_R]
{\orseq{\Gamma}{\cdot}{\cdot}{{!}A}}
{\otseq{\Gamma}{\cdot}{\cdot}{A}}
\quad
\infer[{!}_L]
{\olseq{\Gamma}{\Delta}{\Omega_L}{{!}A}{\Omega_R}}
{\opseq{\Gamma,A}{\Delta}{\Omega_L}{\Omega_R}}
\]
\vspace{-5pt}
\[
\infer[{\ocircle}_R]
{\orseq{\Gamma}{\Delta}{\Omega}{{\ocircle}A}}
{\oseq{\Gamma}{\Delta}{\Omega}{\islax{A}}}
\quad
\infer[{\ocircle}_L]
{\oseq{\Gamma}{\Delta}{\Omega_L /{\ocircle}A/ \Omega_R}{\islax{C}}}
{\oseq{\Gamma}{\Delta}{\Omega_L, A, \Omega_R}{\islax{C}}}
\]

\medskip
{\it Multiplicative connectives}
\[
\infer[{\one}_R]
{\orseq{\Gamma}{\cdot}{\cdot}{\one}}
{}
\quad
\infer[{\one}_L]
{\olseq{\Gamma}{\Delta}{\Omega_L}{\one}{\Omega_R}}
{\opseq{\Gamma}{\Delta}{\Omega_L}{\Omega_R}}
\]
\[
\infer[{\fuse}_R]
{\orseq{\Gamma}{\Delta_1,\Delta_2}{\Omega_L,\Omega_R}{A \fuse B}}
{\otseq{\Gamma}{\Delta}{\Omega_L}{A}
 &
 \otseq{\Gamma}{\Delta}{\Omega_R}{B}}
\quad
\infer[{\fuse}_L]
{\olseq{\Gamma}{\Delta}{\Omega_L}{A \fuse B}{\Omega_R}}
{\opseq{\Gamma}{\Delta}{\Omega_L}{A,B,\Omega_R}}
\]
\vspace{-5pt}
\[
\infer[{\lefti}_R]
{\orseq{\Gamma}{\Delta}{\Omega}{A \lefti B}}
{\otseq{\Gamma}{\Delta}{A, \Omega}{B}}
\quad
\infer[{\lefti}_L]
{\olseq{\Gamma}{\Delta_A, \Delta}{\Omega_L, \Omega_A}{A \lefti B}{\Omega_R}}
{\otseq{\Gamma}{\Delta_A}{\Omega_A}{A}
 &
 \opseq{\Gamma}{\Delta}{\Omega_L}{B,\Omega_R}}
\]
\vspace{-5pt}
\[
\infer[{\righti}_R]
{\orseq{\Gamma}{\Delta}{\Omega}{A \righti B}}
{\otseq{\Gamma}{\Delta}{\Omega, A}{B}}
\quad
\infer[{\righti}_L]
{\olseq{\Gamma}{\Delta_A, \Delta}{\Omega_L}{A \righti B}{\Omega_A, \Omega_R}}
{\otseq{\Gamma}{\Delta_A}{\Omega_A}{A}
 &
 \opseq{\Gamma}{\Delta}{\Omega_L}{B, \Omega_R}}
\]

\medskip
{\it Additive connectives}
\[
\infer[{\zero}_L]
{\olseq{\Gamma}{\Delta}{\Omega_L}{\zero}{\Omega_R}}
{}
\quad
\infer[{\oplus}_{R1}]
{\orseq{\Gamma}{\Delta}{\Omega}{A \oplus B}}
{\otseq{\Gamma}{\Delta}{\Omega}{A}}
\quad
\infer[{\oplus}_{R2}]
{\orseq{\Gamma}{\Delta}{\Omega}{A \oplus B}}
{\otseq{\Gamma}{\Delta}{\Omega}{B}}
\]
\vspace{-5pt}
\[
\infer[{\oplus}_{L}]
{\olseq{\Gamma}{\Delta}{\Omega_L}{A \oplus B}{\Omega_R}}
{\opseq{\Gamma}{\Delta}{\Omega_L}{A,\Omega_R}
 &
 \opseq{\Gamma}{\Delta}{\Omega_L}{B,\Omega_R}}
\]
\vspace{-5pt}
\[
\infer[{\top}_R]
{\orseq{\Gamma}{\Delta}{\Omega}{\top}}
{}
\quad
\infer[{\with}_{L1}]
{\olseq{\Gamma}{\Delta}{\Omega_L}{A \with B}{\Omega_R}}
{\opseq{\Gamma}{\Delta}{\Omega_L}{A, \Omega_R}}
\quad
\infer[{\with}_{L2}]
{\olseq{\Gamma}{\Delta}{\Omega_L}{A \with B}{\Omega_R}}
{\opseq{\Gamma}{\Delta}{\Omega_L}{B, \Omega_R}}
\]
\vspace{-5pt}
\[
\infer[{\with}_R]
{\orseq{\Gamma}{\Delta}{\Omega}{A \with B}}
{\otseq{\Gamma}{\Delta}{\Omega}{A}
 &
 \otseq{\Gamma}{\Delta}{\Omega}{B}}
\]


\caption{Propositional ordered linear lax logic.}
\label{fig:ordered-prop}
\end{figure}


There are two sequents in \ollll.  The primary sequent is
$\oseq{\Gamma}{\Delta}{\Omega}{\istrue{A}}$, which says that $A$ is an
(ephemeral, ordered) resource derivable from the persistent resources
in $\Gamma$, the ephemeral resources in $\Delta$, and the ephemeral,
ordered resources in $\Omega$. There is also a second judgment,
$\oseq{\Gamma}{\Delta}{\Omega}{\islax{A}}$. The judgment $\islax{A}$
is usually interpreted as truth under an unspecified constraint; one
defining characteristic of the lax modality is that it if $\istrue{A}$
is derivable with some resources then $\islax{A}$ is derivable with
the same resources.

Compare the relationship between $\mtrue$ and $\mlax$ 
to the relationship between persistent and linear truth
in linear logic, where the defining characteristic is that a
persistent resource (associated with judgments of the form
$\ispers{A}$ in $\Gamma$) can always satisfy the need for an ephemeral
resource (associated with judgments of the form $\iseph{A}$ in
$\Delta$). In the previous chapter, we first encoded this relationship
as an explicit rule ${\it copy}$:
\[
\infer[{\it copy}]
{\seq{\Gamma, \ispers{A}}{\Delta}{\iseph{C}}}
{\seq{\Gamma, \ispers{A}}{\Delta, \iseph{A}}{\iseph{C}}}
\]
In Section~\ref{sec:linnote}, based on a discussion of synthetic
connectives under the atom optimization, we considered a revision
in which the ${\it copy}$ rule was admissible and 
left rules had conclusions that used the matching
construct $\altseq{\Gamma}{\Delta/A}{C}$,
which matches a sequent of the form $\altseq{\Gamma}{\Delta'}{C}$ 
if either $A \in \Gamma$ and $\Delta' = \Delta$ or if
$\Delta' = \Delta, A$. The $\oplus_L$ rule in this logic is as follows:
\[
\infer[{\oplus}_L]
{\altseq{\Gamma}{\Delta/A \oplus B}{\iseph{C}}}
{\altseq{\Gamma}{\Delta, \iseph{A}}{\iseph{C}}
 &
 \altseq{\Gamma}{\Delta, \iseph{B}}{\iseph{C}}}
\]

Lax truth can be considered along the same lines, accounting for the
fact that our primary judgment is $\istrue{A}$ ($A$ is an ordered
ephemeral resource) instead of $\iseph{A}$ ($A$ is an ephemeral
resource).  To follow existing judgmental presentations of lax logic,
we would include a distinct rule ${\it lax}$ that derives lax truth
from regular truth.
\[
\infer[{\it lax}]
{\Gamma; \Delta; \Omega \longrightarrow \islax{A}}
{\Gamma; \Delta; \Omega \longrightarrow \istrue{A}}
\]
The alternative is to make ${\it lax}$ admissible just as we made
${\it copy}$ admissible: we modify all the right rules with a construct
$\orseq{\Gamma}{\Delta}{\Omega}{A}$ that matches both 
sequents of the form $\otseq{\Gamma}{\Delta}{\Omega}{A}$
and sequents of the form $\oseq{\Gamma}{\Delta}{\Omega}{\islax{A}}$.
The use of this construct gives us right rules for 
$A \oplus B$ that look like this:
\[
\infer[{\oplus}_{R1}]
{\orseq{\Gamma}{\Delta}{\Omega}{A \oplus B}}
{\otseq{\Gamma}{\Delta}{\Omega}{A}}
\qquad
\infer[{\oplus}_{R2}]
{\orseq{\Gamma}{\Delta}{\Omega}{A \oplus B}}
{\otseq{\Gamma}{\Delta}{\Omega}{B}}
\]
The related notation on the left-hand side is the construct 
$\olseq{\Gamma}{\Delta}{\Omega_L}{A}{\Omega_R}$, which matches
the sequent $\oseq{\Gamma}{\Delta'}{\Omega'}{U}$ if
\begin{itemize}
\item $\Omega' = \Omega_L, A, \Omega_R$ and $\Delta' = \Delta$;
\item $\Omega' = \Omega_L, \Omega_R$ and $\Delta' = \Delta, A$;
\item $\Omega' = \Omega_L, \Omega_R$, $\Delta' = \Delta$, and $A \in \Gamma$.
\end{itemize}
As in the alternate presentation of linear logic where ${\it copy}$ was
admissible, both the ${\it copy}$ rule and a rule Polakow called ${\it
  place}$ are admissible in the logic described in
Figure~\ref{fig:ordered-prop}.
\[
\infer-[{\it copy}]
{\oseq{\Gamma, {A}}{\Delta}{\Omega_L, \Omega_R}{U}}
{\oseq{\Gamma, {A}}{\Delta}{\Omega_L, {A}, \Omega_R}{U}}
\qquad
\infer-[{\it place}]
{\oseq{\Gamma}{\Delta, {A}}{\Omega_L, \Omega_R}{U}}
{\oseq{\Gamma}{\Delta}{\Omega_L, {A}, \Omega_R}{U}}
\]

\subsection{First-order logic}

The presentation in Figure~\ref{fig:ordered-prop} is propositional; by
uniformly adding a first-order context $\Psi$ to all sequents,
however, it can be treated as first-order. We define quantification
(existential and universal), as well as first-order
equality,\footnote{That is, equality of terms from the domain of
  first-order quantification. The domain of first-order equality is
  LF, so an implementation of this equality would involve higher-order
  unification.} in Figure~\ref{fig:ordered-fo}.

\begin{figure}
\[
\infer[{\exists}_R]
{\orfseq{\Gamma}{\Delta}{\Omega}{\exists \lf{a}{:}\tau.B}}
{\Psi \vdash \lf{t} : \tau 
 &
 \otseq{\Gamma}{\Delta}{\Omega}{\lf{[t/a]}B}}
\quad
\infer[{\exists}_L]
{\olfseq{\Gamma}{\Delta}{\Omega_L}{\exists \lf{a}{:}\tau.B}{\Omega_R}}
{\ofirstseq{\Psi,\lf{a}{:}\tau}{\Gamma}{\Delta}{\Omega_L,B,\Omega_R}{U}}
\]
\[
\infer[{\forall}_R]
{\orfseq{\Gamma}{\Delta}{\Omega}{\forall \lf{a}{:}\tau.B}}
{\ofirstseq{\Psi,\lf{a}{:}\tau}{\Gamma}{\Delta}{\Omega}{\isconc{B}}}
\quad
\infer[{\forall}_L]
{\olfseq{\Gamma}{\Delta}{\Omega_L}{\forall \lf{a}{:}\tau.B}{\Omega_R}}
{\Psi \vdash \lf{t} : \tau
 &
 \opfseq{\Gamma}{\Delta}{\Omega_L}{\lf{[t/a]}B,\Omega_R}}
\]
\[
\infer[{\doteq}_R]
{\orfseq{\Gamma}{\Delta}{\Omega}{\lf{t} \doteq_\tau \lf{t}}}
{}
\]
\[
\infer[{\doteq}_L]
{\olfseq{\Gamma}{\Delta}{\Omega_L}{\lf{t} \doteq_\tau \lf{s}}{\Omega_R}}
{\forall(\Psi' \vdash \lf{\sigma} : \Psi). 
 &
 \lf{\sigma t} = \lf{\sigma s}
 &
 \longrightarrow
 &
 \ofirstseq{\Psi'}{\lf{\sigma}\Gamma}{\lf{\sigma}\Delta}{\lf{\sigma}\Omega_L,\lf{\sigma}\Omega_R}{\lf{\sigma} U}}
\]

\caption{First-order ordered linear lax logic}
\label{fig:ordered-fo}
\end{figure}


The equality judgment $t \doteq s$ deserves some attention. It is a
higher-order judgment, in the sense that it reflects over the
definition of simultaneous term substitutions $\Psi' \vdash \sigma :
\Psi$ and over the syntactic equality judgment for first-order terms
$t = s$. This is a rule that, in general, will have countably many
premises; in the case of a trivially satisfiable equality problem like
$x \doteq x$ it will have one premise for each well-formed
substitution that substitutes a term of the appropriate type for
$x$. This exact style of presentation was used previously in
\cite{simmons09weak}, but the approach is based on Schroeder-Heister's
treatment of definitional reflection \cite{schroeder93rules}.

There are two important special cases. First, an unsatisfiable 
equation on the left implies a contradiction, and makes the left rule
for equality equivalent to one with no premises. For instance, this
means that
\[
\infer[\doteq_{\it no}]
{\olfseq{\Gamma}{\Delta}{\Omega_L}{t \doteq s}{\Omega_R}}
{{\it no~unifier~for~} t {\it ~and~} s}
\]
is derivable. The other important special case is
when $t$ and $s$ have a {\it most general unifier} $\sigma_{\it mgu}$,
which just means that for all $\Psi' \vdash \sigma : \Psi$ such that
$\sigma t = \sigma s$, it is the case that $\sigma = \sigma' \circ
\sigma_{\it mgu}$ for some $\sigma'$.\footnote{Where $\circ$ is
  composition -- $(\sigma' \circ \sigma_{\it mgu})t = \sigma'(\sigma
  t)$.} In this case, the left rule for equality is equivalent to the
following rule:
\[
\infer[\doteq_{\it yes}]
{\olfseq{\Gamma}{\Delta}{\Omega_L}{t \doteq s}{\Omega_R}}
{{\it mgu}(t, s) = \Psi' \vdash \sigma : \Psi
 &
 \ofirstseq{\Psi'}{\sigma\Gamma}{\sigma\Delta}{\sigma\Omega_L, \sigma\Omega_R}{\sigma U}}
\]
Therefore, given a first-order domain in which any two terms are
decidably either non-unifiable or unifiable with a most general
unifier, defining the logic with two rules $\doteq_{\it no}$ and
$\doteq_{\it yes}$ is equivalent to defining the logic with
$\doteq_L$.

We have not yet thoroughly specified the type and term structure of
first-order individuals; in the next chapter we clarify that these
types and terms will actually be types and terms of Canonical LF.

\section{Substructural contexts}
\label{sec:contexts}

First-ordered linear lax logic has a lot of contexts -- the persistent
context $\Gamma$ the linear context $\Delta$, and the ordered context
$\Omega$, not to mention the first-order context $\Psi$. In most rules
these contexts just hang around, obscuring the logic's presentation
and ensuring that the {\LaTeX} code of figures and displays remains
permanently unreadable. And there are yet more contexts we might want to 
add, such as the affine contexts present in the Celf implementation
\cite{schacknielsen08celf}.

In this section, we will consider a more compact way of dealing with
the contexts which we think of as containing resources (persistent,
affine, linear, or ordered), though we will maintain the distinction
between resource contexts and first-order variable 
contexts $\Psi$.  The particular way we define substructural contexts
can be generalized substantially: it would be trivial to extend
this presentation to the affine modality or to the subexponentials
discussed by Nigam and Miller \cite{nigam09algorithmic}, and it should
be able to richer logics, such as the logic of bunched implication.

We write unified substructural contexts as either $\Delta$ or $\Xi$,
preferring the latter when there is a chance of confusing them with
linear contexts $\Delta$. For the purposes of encoding \ollll, we can
see these contexts as sequences, defined by the grammar
\[
\Xi ::= \cdot 
  \mid \Xi, x{:}\ispers{T}
  \mid \Xi, x{:}\iseph{T}
  \mid \Xi, x{:}\istrue{T}
\]
where each of the {\em variables} $x$ are distinct, so that the
context also represents a finite map from variables $x$ to {\it
  judgments} $\islvl{T}$, where $\mlvl$ is either $\mpers$, $\meph$,
or $\mtrue$.  By separating out a substructural context into three
subsequences of persistent, linear, and ordered judgments, we can
recover the presentations of contexts for \ollll~given in
Figure~\ref{fig:ordered-prop}. We will use this observation informally
throughout the chapter, writing $\Xi = \Gamma; \Delta; \Omega$.

The domain represented by the metavariable $T$ is arbitrary: when
discussing the unfocused logic given in Figure~\ref{fig:ordered-prop},
$T$ varies over unpolarized propositions $A$, but when discussing a
focused logic in Section~\ref{sec:ord-focused} it will vary over
stable negative propositions $A^-$, positive suspended propositions
$\susp{A^+}$, focused negative propositions $[A^-]$, and inverting
positive propositions $A^+$.



% The first step towards this understanding 
% has already been take in Figure~\ref{fig:linear-alt} and
% Figure~\ref{fig:ordered-prop}, which make it quite obvious that the
% context-{\it matching} notation that we perform in the conclusion of
% an inference rule may not be the same as the context-{\it extending}
% notation we use in the premise of that rule.

The key observation of this presentation was already present in the
unfocused logic presented in Figure~\ref{sec:ord-unfocused}: we need
to differentiate {\it constructions}, which appear in the premises of
rules, and {\it matching constructs}, which appear in the conclusions
of rules.  The notation $\Gamma; \Delta; \Omega_L/A\fuse B/\Omega_R$
that appears in the conclusion of ${\fuse}_L$ is a matching construct;
as discussed in Section~\ref{sec:ord-unfocused}, there are multiple
ways in which a context $\Gamma'; \Delta'; \Omega'$ could match this
context, because $A \fuse B$ could come from any of the three
contexts. However, $\Gamma; \Delta; \Omega_L,{A},{B},
\Omega_R$ in the premise of ${\fuse}_L$ is a construction, and is
unambiguously equal to only one context $\Gamma'; \Delta'; \Omega'$.

\subsection{Fundamental operations on contexts}

The first fundamental idea we consider is {\it singleton} contexts.
We construct a single-element context by writing $x{:}\islvl{T}$.
The corresponding matching construct on contexts is 
$x{:}{T}$. In unfocused \ollll, we say that $\Xi$ matches 
$x{:}{A}$ if its decomposition into persistent, linear, and 
ordered contexts matches $\Gamma; \cdot; /A/$. Specifically,

\bigskip
\begin{definition}[Sole membership]
  $\Xi$ matches $x{:}T$ if
\begin{itemize}
\item $\Xi$ contains no linear judgments and contains exactly
one
ordered judgment $x{:}\istrue{T}$ (corresponding to the situation where
$\Xi = \Gamma; \cdot; T$), 
\item $\Xi$ contains no ordered judgments and contains exactly
one linear judgement $x{:}\iseph{T}$ (corresponding to the situation where
$\Xi = \Gamma; T; \cdot$), or 
\item $\Xi$ contains only persistent judgments, including
$x{:}\ispers{T}$ (corresponding to the situation where
$\Xi = \Gamma, T; \cdot; \cdot$). 
\end{itemize}
\end{definition}
\bigskip

Sole membership is related to the initial sequents and the 
matching construct $\Gamma; \cdot;/A/$ for contexts that was used
in Figure~\ref{fig:ordered-prop}.
We could rewrite the {\it init} rule
from that figure as follows:
\[
\infer[{\it init}]
{x{:}p \Rightarrow /{p}/}
{}
\]
As in all rules involving matching constructs in the conclusion,
it is fair to view the matching construct as an extra premise; thus,
the ${\it init}$ rule above is the same as the ${\it init}$ rule 
below:
\[
\infer[{\it init}]
{\Xi \Rightarrow /{p}/}
{\Xi \textit{~matches~} x{:}p}
\]

The second basic operation on contexts requires a new concept, {\it
  frames} $\Theta$. Intuitively, we can look frames as substructural
contexts for \ollll~as a series of persistent, linear, and ordered
contexts where the ordered context is missing a particular piece. 
Informally, we
can write this missing piece as a box: $\Gamma; \Delta; \Omega_L,
\Box, \Omega_R$. Alternatively we can think of a frame as a one-hole
context or Huet-style zipper \cite{huet97zipper} over the structure of
substructural contexts. We will actually think of them essentially
as linear LF functions $(\lambda\Xi.\, \Xi_L, \Xi, \Xi_R)$
\cite{simmons09linear}, or, more generally, as linear representational
functions defined by substitution.

The construction associated with frames, $\tackon{\Theta}{\Xi}$,
is just a straightforward operation of filling in the hole or 
beta-reducing the linear function; doing this requires that the 
variables in $\Theta$ and $\Xi$ be distinct. If we
think of $\Theta$ informally as $\Gamma; \Delta; \Omega_L, \Box,
\Omega_R$, then this is {\it almost} like the operation of filling in
the hole, as $\tackon{\Theta}{x{:}\istrue{A}} = \Gamma; \Delta;
\Omega_L, A, \Omega_R$. The main difference is that we can also use
the operation to insert linear propositions
($\tackon{\Theta}{x{:}\iseph{A}} = \Gamma; \Delta, A; \Omega_L,
\Omega_R$) and persistent propositions
($\tackon{\Theta}{x{:}\ispers{A}} = \Gamma, A; \Delta; \Omega_L,
\Omega_R$).

The matching construct for frames is a bit more complicated,
Informally, if we treat linear contexts as multisets and say that $\Xi
= \Gamma; \Delta, \Delta'; \Omega_L, \Omega', \Omega_R$, then we can
say $\Xi = \frameoff{\Theta}{\Xi'}$ in the case that $\Theta = \Gamma;
\Delta; \Omega_L, \Box, \Omega_R$ and $\Xi' = \Gamma; \Delta';
\Omega'$. The sub-context $\Xi'$, then, has been {\it framed off} from
$\Xi$, its frame is $\Theta$. If we only had ordered judgments
$\istrue{T}$, then the framing-off matching construct
$\frameoff{\Theta}{\Xi'}$ would be essentially the same as the
construction form $\tackon{\Theta}{\Xi'}$. However, persistent and
linear judgments can be reordered in the process of matching, and
persistent judgments always end up in both the frame and the 
framed-off context. 

\bigskip
\begin{definition}[Framing off]
$\Xi$ matches $\frameoff{\Theta}{\Xi'}$ if the union of the variables in 
$\Theta$ and $\Xi'$ is exactly the variables in $\Xi$ and
\begin{itemize}
\item if $x{:}\ispers{T} \in \Xi$, then the same mapping appears in 
  $\Theta$ and $\Xi'$;
\item if $x{:}\iseph{T} \in \Xi$ or $x{:}\istrue{T} \in \Xi$, 
  then the same mapping appears in $\Theta$ or $\Xi'$ (but not both); 
\item in both $\Theta$ and $\Xi'$, 
  the sequence of mappings $x{:}\istrue{T}$ 
  is a subsequence of $\Xi$; and 
\item if $x{:}\istrue{T} \in \Theta$, then either
  \begin{itemize}
  \item for all $y{:}\istrue{T'} \in \Xi'$, the mapping for $x$ appeared before
    the mapping for $y$ in $\Xi$, or
  \item for all $y{:}\istrue{T'} \in \Xi'$, the mapping for $x$ appeared after
    the mapping for $y$ in $\Xi$. 
  \end{itemize}
\end{itemize}
\end{definition}
\bigskip

An important derived matching construct is $\frameoff{\Theta}{x{:}T}$,
which matches $\Xi$ if $\Xi$ matches $\frameoff{\Theta}{\Xi'}$ for
some $\Xi'$ such that $\Xi'$ matches $x{:}T$.  This pattern is needed
to describe almost every left rule from Figure~\ref{fig:ordered-prop},
for instance:
\[
\infer[]
{\frameoff{\Theta}{x{:}A \oplus B} \Rightarrow U}
{\tackon{\Theta}{y{:}\istrue{A}} \Rightarrow U
 &
 \tackon{\Theta}{z{:}\istrue{B}} \Rightarrow U}
\quad
\infer[]
{\frameoff{\Theta}{x{:}A \with B} \Rightarrow U}
{\tackon{\Theta}{y{:}\istrue{A}} \Rightarrow U}
\quad
\infer[]
{\frameoff{\Theta}{x{:}A \with B} \Rightarrow U}
{\tackon{\Theta}{y{:}\istrue{B}} \Rightarrow U}
\]

We can also use this notation to describe the one of the
cut principles for ordered linear lax logic. 
\[
\infer-[{\it cut}]
{\frameoff{\Theta}{\Xi} \Rightarrow \istrue{C}}
{\Xi \Rightarrow \istrue{A}
 &
 \tackon{\Theta}{x{:}A\,{\it true}} \Rightarrow \istrue{C}}
\]
Especially for the eventual proof of this cut principle, it is important
to consider that this cut principle is equivalent to the one that describes
the matching as an explicit extra premise:
\[
\infer-[{\it cut}]
{\Xi' \Rightarrow \istrue{C}}
{\Xi \Rightarrow \istrue{A}
 &
 \tackon{\Theta}{x{:}A\,{\it true}} \Rightarrow \istrue{C}
 &
 \Xi' \textit{~matches~} \frameoff{\Theta}{\Xi}}
\]

\futurework{
As an aside: it need not always be the case that the same operation
used to describe 

The idea that the operators $\frameoff{\Theta}{\Xi}$
and $\tackon{\Theta}{\Xi}$ are sufficient to describe the 
cut principle is related to the display property, which fails
for some reasonable logics, such as Reed's queue logic
\cite{reed09queue}.}

\subsection{Multiplicative operations}

To describe the multiplicative connectives of \ollll, including the critical
connective of implication, we need to have multiplicative operations on 
contexts. As a construction, $\mkconj{\Xi_L}{\Xi_R}$ is just the
syntactic concatenation of two contexts with distinct variable domains, and
the unit $\mkunit$ is just the empty sequence. The matching constructs
are more complicated to define, but the intuition is, again, 
uncomplicated: if 
$\Xi = \Gamma; \Delta, \Delta'; \Omega_L, \Omega_R$, where linear contexts
are multisets and ordered contexts are sequences, then 
$\Xi = \mkconj{\Xi_L}{\Xi_R}$ if $\Xi_L = \Gamma; \Delta; \Omega_L$ and
$\Xi_R = \Gamma; \Delta'; \Omega_R$. Note that we are giving up on
having different notations for constructions and matching constructs: 
$\matchconj{\Xi_L}{\Xi_R}$ is a matching construct when it appears
in the conclusion of a rule, $\mkconj{\Xi_L}{\Xi_R}$ is a construction
when it appears in the premise of a rule.

\bigskip
\begin{definition}[Conjunction]
$\Xi$ matches $\matchunit$ if $\Xi$ contains only
persistent judgments.

\smallskip
\noindent
$\Xi$ matches $\matchconj{\Xi_L}{\Xi_R}$ if the union 
of the variables in $\Xi_L$ and $\Xi_R$ is exactly the variables in $\Xi$
and 
\begin{itemize}
\item if $x{:}\ispers{T} \in \Xi$, then the same mapping appears in $\Xi_L$
  and $\Xi_R$;
\item if $x{:}\iseph{T} \in \Xi$ or $x{:}\istrue{T} \in \Xi$, then the
  same mapping appears in $\Xi_L$ or $\Xi_R$ (but not both); 
\item in both $\Xi_L$ and $\Xi_R$, 
  the sequence of mappings $x{:}\istrue{T}$ 
  is a subsequence of $\Xi$; and 
\item if $x{:}\istrue{T} \in \Xi_L$ and $y{:}\istrue{T'} \in \Xi_R$, then
  the mapping for $x$ appeared before the mapping for $y$ in $\Xi$. 
\end{itemize}
\end{definition}
\bigskip

The constructs for multiplicative conjunction are put to obvious use
in the description of multiplicative conjunction, which is essentially
just the propositional internalization of context conjunction:
\[
\infer
{\matchconj{\Xi_L}{\Xi_R} \Rightarrow /A \fuse B/}
{\Xi_L \Rightarrow \istrue{A} & \Xi_R \Rightarrow \istrue{B}}
\quad
\infer
{\frameoff{\Theta}{x{:}A \fuse B} \Rightarrow U}
{\tackon{\Theta}{\mkconj{y{:}A}{z{:}B}} \Rightarrow U}
\quad
\infer
{\cdot \Rightarrow \one}
{}
\quad
\infer
{\frameoff{\Theta}{x{:}\one} \Rightarrow U}
{\tackon{\Theta}{\cdot} \Rightarrow U}
\]\[
\infer
{\Xi \Rightarrow / A \lefti B /}
{x{:}\istrue{A}, \Xi \Rightarrow \istrue{B}}
\quad
\infer
{\frameoff{\Theta}{\Xi_A, x{:}A \lefti B} \Rightarrow U}
{\Xi_A \Rightarrow \istrue{A} & \tackon{\Theta}{y{:}\istrue{B}} \Rightarrow U}
\]
Implication makes deeper use of context conjunction:
% we can look at 
$\Xi$ matches
$\frameoff{\Theta}{\matchconj{\Xi_A}{x{:}A \lefti B}}$ 
%in two ways.
%One way of reading this notation 
exactly when there exist $\Xi'$ and $\Xi''$ such that 
% is 
$\Xi$ matches $\frameoff{\Theta}{\Xi'}$, 
$\Xi'$ matches $\matchconj{\Xi_A}{\Xi''}$,
and $x{:}A$ matches $\Xi''$. 

% The other
% way of reading the notation 
% is that $\Xi$ matches $\frameoff{\Theta'}{x{:}A \lefti B}$, and
% $\Theta'$ matches $\frameoff{\Theta}{\matchconj{\Xi_A}{\Box}}$, 
% It is critical that these two ways of 

% \bigskip
% \begin{definition}[Matching into frames]
% $\Theta$ matches $\frameoff{\Theta'}{\matchconj{\Xi_A}{\Box}}$ if, 
% for all $\Xi'$ and $\Xi_A$, $\Xi'$ matches $\matchconj{\Xi_A}{\Xi_B}$
% if and only if 

% For all $\Xi$, $\Xi_A$, $\Xi_B$ and $\Theta$, 
% \begin{itemize}
% \item there exists $\Xi'$ such that $\Xi$ matches $\frameoff{\Theta}{\Xi}$
%   and $\Xi$ matches $\matchconj{\Xi_A}{\Xi_B}$
%   if and only if there exists $\Delta'$ such that $\Xi$ matches 
%   $\frameoff{\Theta}{\Xi_A}$ and $\Theta$ matches 
%   $\frameoff{\Theta'}{}$
% \end{itemize}
% \end{definition}


% As a matching construct, $\Xi = \matchconj{\Xi_L}{\Xi_R}$ if every 
% $x{:}\ispers{T}$ in $\Xi$ appears in both $\Xi_L$ and $\Xi_R$, every 
% $x{:}\iseph{T}$ or $x{:}\istrue{T}$ 
% in $\Xi$ appears in exactly one of $\Xi_L$ or $\Xi_R$, and
% if furthermore every 
% $x{:}\iseph{T}$ $\Xi_L$


% As a construction form, 


\subsection{Exponential operations}

The exponentials ${!}$ and ${\gnab}$ do not have a construction form
associated with them, unless we view the singleton
construction forms $x{:}\ispers{T}$ and $x{:}\iseph{T}$ as
being associated with these exponentials. The matching
operation is quite simple: $\Xi$ matches $\restrictto{\Xi}{\mpers}$ if
$\Xi$ contains no ephemeral or ordered judgments -- in other words,
it says that $\Xi = \Gamma; \cdot; \cdot$. This form can then be used
to describe the right rule for ${!}A$ in unfocused \ollll:
\[
\infer
{\restrictto{\Xi}{\mpers} \Rightarrow /{!}A/}
{\Xi \Rightarrow \istrue{A}}
\]
Similarly, $\Xi$ matches $\restrictto{\Xi}{\meph}$ if $\Xi$ contains
no ordered judgments (that is, if $\Xi = \Gamma; \Delta; \cdot$).
$\Xi$ always matches $\restrictto{\Xi}{\mtrue}$; we don't ever
explicitly use this construct, but it allows us to generally refer to
$\restrictto{\Xi}{\mlvl}$ in the statement of theorems like cut admissibility.

The exponential matching constructs 
don't actually modify contexts in the way
other matching constructs do, but this is a consequence of the 
particular choice of logic we're considering. Given affine resources,
for instance, the matching construct associated with the 
affine modality would
clear the context of affine facts: 
$\Xi$ would match $\restrictto{\Xi'}{\mpers}$ if 
$\Xi$ has only persistent and affine resources and $\Xi'$ 
contains the same persistent resources but none of the affine ones.

We can describe a mirror-image operation on succeedants $U$.  $U$
matches $\restrictfrom{U}{\mlax}$ only if it has the form $\islax{T}$,
and $U$ always matches $\restrictfrom{U}{\mtrue}$. The latter matching
construct is another degenerate form that similarly allows us to refer
to $\restrictfrom{U}{\mlvl}$ as a generic matching construct.

\section{Focused sequent calculus}
\label{sec:ord-focused}

A sequent in the focusing calculus has the form
$\foc{\Psi}{\Delta}{U}$, where $\Psi$ is the first-order variable
context, $\Delta$ is a substructural context as described in the previous
section, and $U$ is a succeedant. The domain $T$ of the substructural
context consists of stable negative propositions $A^-$, positive
suspended propositions $\susp{A^+}$, focused negative propositions
$[A^-]$, and inverting positive propositions $A^+$.

The form of the succeedant $U$ is $\islvl{T}$, where $\mlvl$ is either
$\mtrue$ or $\mlax$; in this way, $U$ is just a like a special
substructural context with exactly one element -- we don't need
to care about the name of the variable, because there's only one.  The
domain of $T$ for succeedants is complementary to the domain of $T$
for contexts: stable positive propositions $A^+$, negative suspended
propositions $\susp{A^-}$, focused positive propositions $[A^+]$, and
inverting negative propositions $A^-$.

\subsection{Restrictions on the form of sequents}

A sequent $\foc{\Psi}{\Delta}{U}$ is {\it stable} when the context
$\Delta$ and succeedant $U$ contain only stable propositions ($A^-$ in
the context, $A^+$ in the succeedant) and suspended propositions
($\susp{A^+}$ in the context, $\susp{A^-}$ in the succeedant). We
repeat the focusing constraint discussed in Chapter 2: there is only
ever at most one focused proposition in a sequent, and if there is
focused proposition in the sequent, then the sequent is otherwise
stable. A restriction on the rules ${\it focus}_L$ and ${\it focus}_R$
is sufficient to enforce this restriction: reading rules from
top down, we can only use a rule ${\it focus}_L$ or ${\it focus}_R$ to
prove a stable sequent, and reading rules from bottom up, we can only
apply ${\it focus}_L$ or ${\it focus}_R$ when we are searching for a
proof of a stable sequent.

Because there is always a distinct focused proposition in a sequent,
we do not need a variable name to reference the focused proposition in
a context $\Delta$ any more than we need a variable name to reference
the unique member of the context-like succeedant $U$. Therefore, we
write $\istrue{[B^-]}$ instead of $x{:}\istrue{[B^-]}$. Furthermore,
we restrict focused propositions and inverting propositions so that
they are always associated with the judgment $\mtrue$. With this
restriction, we can write $[A^-]$ and $x{:}A^+$ instead of
$\istrue{[A^-]}$ and $x{:}\istrue{A^+}$ in $\Delta$, and we can write
$[A^+]$ and $A^-$ instead of $\istrue{[A^+]}$ and $\istrue{A^-}$ for
$U$.

In a confluent presentation of focused logic like the one given for
linear logic in Chapter 2, that would be as far as we could take our
simplifications. However, this presentation will match the fixed
presentation of logic from the structural focalization development
\cite{simmons11structural}. Specifically, if there is more than one
invertible proposition in a sequent, {\it only} the leftmost one will
be treated as eligible to have a rule applied to it. All the
propositions in $\Delta$ are treated as being to the left of the
succeedant $U$, so we always prioritize inversion on positive
propositions in $\Delta$. With this additional restriction, it is
always unambiguous which proposition we are referring to in an
invertible rule, and we write $A^+$ instead of $x{:}A^+$ or
$x{:}\istrue{A^+}$.

We will maintain the notational convention in this chapter that
first-order variables are written as $a$, variables associated
with stable negative propositions are written as $x$, and variables
associated with suspended positive propositions are written as 
$z$. This convention will be abandoned in future chapters.

\subsection{Polarized propositions}

The propositions of ordered logic are fundamentally sorted into
positive propositions $A^+$ and negative propositions $A^-$; both
classes, and the inclusions between them, 
are shown in Figure~\ref{fig:ordered}. The
positive propositions have a refinement, {\it permeable} propositions
$A^+_\mpers$, that is analogous to the refinement discussed for linear
logic in Section~\ref{sec:permeable}. There is also a more generous
refinement, the {\it mobile} propositions, $A^+_\meph$, for positive
propositions that bottom out with one of the two modalities ${!}$ and
${\gnab}$. We introduce atomic propositions $p^+$ that stand for
arbitrary positive propositions, $p^+_\meph$ that stand for arbitrary
mobile propositions, and $p^+_\mpers$ that stand for arbitrary
permeable propositions. We treat $A^+_\mtrue$ and $p^+_\mtrue$ as synonymous
with $A^+$ and $p^+$, respectively, which allows us to generically
refer to $A^+_\mlvl$ and $p^+_\mlvl$ in rules like $\eta^+$ and in the
statement of the identity expansion theorem.

\begin{figure}
\begin{align*}
A^+ & ::= p^+ \mid p^+_\meph \mid p^+_\mpers
        \mid {\downarrow}A^- \mid {\gnab}A^- \mid {!}A^- 
        \mid \one \mid A^+ \fuse B^+ \mid \zero \mid A^+ \oplus B^+ 
        \mid \exists x. A^+ \mid t \doteq s
\\
A^+_\meph & ::= p^+_\meph \mid p^+_\mpers
        \mid {\gnab}A^- \mid {!}A^- 
        \mid \one \mid A^+_\meph \fuse B^+_\meph
        \mid \zero \mid A^+_\meph \oplus B^+_\meph
        \mid \exists x. A^+_\meph \mid t \doteq s
\\
A^+_\mpers & ::= p^+_\mpers 
        \mid {!}A^- 
        \mid \one \mid A^+_\mpers \fuse B^+_\mpers 
        \mid \zero \mid A^+_\mpers \oplus B^+_\mpers
        \mid \exists x. A^+_\mpers \mid t \doteq s
\\
A^- & ::= p^- \mid p^-_\mlax 
        \mid {\uparrow}A^+ \mid {\ocircle}A^+
        \mid A^+ \lefti B^- \mid A^+ \righti B^-
        \mid \top \mid A^- \with B^-
        \mid \forall x.A^-
\\
A^-_\mlax & ::= p^-_\mlax 
        \mid {\ocircle}A^+
        \mid A^+ \lefti B^-_\mlax \mid A^+ \righti B^-_\mlax
        \mid \top \mid A^-_\mlax \with B^-_\mlax
        \mid \forall x.A^-_\mlax
\end{align*}
\caption{Propositions of focused \ollll.}
\label{fig:ordered}
\end{figure}

Negative propositions also have a refinement, $A^-_\mlax$, for
negative propositions that always bottom out as a modal
${\ocircle}A^+$.  This is interesting as a formal artifact and there
is very little overhead involved in putting it into our development,
but I don't claim to fully understand what the refinement means or
what the inclusion of right-permeable atomic propositions $p^-_\mlax$
means in terms of the structure of proofs. Certainly we do {\it not}
want to include such propositions in the logical framework, as to do
so would break the distinction between concurrent and deductive proofs
that will be discussed in Section~\ref{sec:framework-logicprog}.

The presentation of the modalities, and the logic that we now present,
emphasises the degree to which the shifts ${\uparrow}$ and
${\downarrow}$ have much of the character of modalities in a focused
substructural logic; this point was explored in depth by Laurent in
his thesis \cite{laurent02etude}. The upshift ${\uparrow}A^+$ is like
an ordered variant of the lax modality ${\ocircle}A^+$ that puts no
constraints on the form of the succeedant, and the downshift
${\downarrow}A^-$ is like an ordered variant of the persistent and
linear modalities ${!}A^-$ and ${\gnab}A^-$ that puts no constraints
on the form of the context.

\subsection{Derivations and proof terms}
\label{sec:ord-proof-terms}

\renewcommand{\foct}[4]{{#2} \vdash {#3} : {#4}}

\begin{figure}
\small
{\it Focus, identity, and atomic propositions}

\[
\infer[{\it focus}_R^*]
{\foct{\Psi}{\Delta}{\tfocusr{V}}{\islvl{A^+}}}
{\foct{\Psi}{\Delta}{V}{[A^+]}}
\quad
\infer[{\it focus}_L^*]
{\foct{\Psi}{\frameoff{\Theta}{x{:}{A^-}}}
  {\tfocusl{x}{\Sp}}{U}}
{\foct{\Psi}{\tackon{\Theta}{[A^-]}}{\Sp}{U}}
\]
\vspace{-5pt}
\[
\infer[\eta^+]
{\foct{\Psi}{\frameoff{\Theta}{p^+_\mlvl}}{\tetap{z}{N}}{U}}
{\foct{\Psi}{\tackon{\Theta}{z{:}\islvl{\susp{p^+_\mlvl}}}}{N}{U}}
\quad
\infer[{\it id}^+]
{\foct{\Psi}{z{:}\langle A^+ \rangle}{z}{[A^+]}}
{}
\]
\vspace{-5pt}
\[
\infer[\eta^-]
{\foct{\Psi}{\Delta}{\tetan{N}}{p^-_\mlvl}}
{\foct{\Psi}{\Delta}{N}{\islvl{\susp{p^-_\mlvl}}}}
\quad
\infer[{\it id}^-]
{\foct{\Psi}{[A^-]}{\tnil}{\islvl{\susp{A^-}}}}
{}
\]

\medskip
{\it Shifts and modalities}
\[
\infer[{\downarrow}_R]
{\foct{\Psi}{\Delta}{\tdownr{N}}{[{\downarrow}A^-}]}
{\foct{\Psi}{\Delta}{N}{A^-}}
\quad
\infer[{\downarrow}_L]
{\foct{\Psi}{\frameoff{\Theta}{[{\downarrow}A^-]}}{\tdownl{x}{N}}{U}}
{\foct{\Psi}{\tackon{\Theta}{x{:}\istrue{A^-}}}{N}{U}}
\]
\vspace{-5pt}
\[
\infer[{\gnab}_R]
{\foct{\Psi}{\restrictto{\Delta}{\meph}}{\tgnabr{N}}{[{\gnab}A^-}]}
{\foct{\Psi}{\Delta}{N}{A^-}}
\quad
\infer[{\gnab}_L]
{\foct{\Psi}{\frameoff{\Theta}{[{\gnab}A^-]}}{\tgnabl{x}{N}}{U}}
{\foct{\Psi}{\tackon{\Theta}{x{:}\iseph{A^-}}}{N}{U}}
\]
\vspace{-5pt}
\[
\infer[{!}_R]
{\foct{\Psi}{\restrictto{\Delta}{\mpers}}{\tbangr{N}}{[{!}A^-}]}
{\foct{\Psi}{\Delta}{N}{A^-}}
\quad
\infer[{!}_L]
{\foct{\Psi}{\frameoff{\Theta}{[{!}A^-]}}{\tbangl{x}{N}}{U}}
{\foct{\Psi}{\tackon{\Theta}{x{:}\ispers{A^-}}}{N}{U}}
\]
\vspace{-5pt}
\[
\infer[{\uparrow}_R]
{\foct{\Psi}{\Delta}{\tupr{N}}{{\uparrow}A^+}}
{\foct{\Psi}{\Delta}{N}{\istrue{A^+}}}
\quad
\infer[{\uparrow}_L]
{\foct{\Psi}{\frameoff{\Theta}{[{\uparrow}A^+]}}{\tupl{N}}{U}}
{\foct{\Psi}{\tackon{\Theta}{A^+}}{N}{U}}
\]
\vspace{-5pt}
\[
\infer[{\ocircle}_R]
{\foct{\Psi}{\Delta}{\tlaxr{N}}{{\ocircle}A^+}}
{\foct{\Psi}{\Delta}{N}{\islax{A^+}}}
\quad
\infer[{\ocircle}_L]
{\foct{\Psi}{\frameoff{\Theta}{[{\ocircle}A^+]}}
   {\tlaxl{N}}{\restrictfrom{U}{\mlax}}}
{\foct{\Psi}{\tackon{\Theta}{A^+}}{N}{U}}
\]


\medskip
{\it Multiplicative connectives}
\[
\infer[{\one}_R]
{\foct{\Psi}{\matchunit}{\toner}{[\one]}}
{}
\quad
\infer[{\one}_L]
{\foct{\Psi}{\frameoff{\Theta}{\one}}{\tonel{N}}{U}}
{\foct{\Psi}{\tackon{\Theta}{\mkunit}}{N}{U}}
\]
\vspace{-5pt}
\[
\infer[{\fuse}_R]
{\foct{\Psi}{\matchconj{\Delta_1}{\Delta_2}}
   {\tfuser{V_1}{V_2}}{[A^+ \fuse B^+]}}
{\foct{\Psi}{\Delta_1}{V_1}{[A^+]}
 &
 \foct{\Psi}{\Delta_2}{V_2}{[B^+]}}
\quad
\infer[{\fuse}_L]
{\foct{\Psi}{\frameoff{\Theta}{A^+ \fuse B^+}}{\tfusel{N}}{U}}
{\foct{\Psi}{\tackon{\Theta}{\mkconj{A^+}{B^+}}}{N}{U}}
\]
\vspace{-5pt}
\[
\infer[{\lefti}_R]
{\foct{\Psi}{\Delta}{\tlaml{N}}{A^+ \lefti B^-}}
{\foct{\Psi}{\mkconj{A^+}{\Delta}}{N}{B^-}}
\quad
\infer[{\lefti}_L]
{\foct{\Psi}{\frameoff{\Theta}{\matchconj{\Delta_A}{[A \lefti B]}}}
   {\tappl{V}{\Sp}}{U}}
{\foct{\Psi}{\tackon{\Theta}{[B]}}{\Sp}{U}}
\]
\vspace{-5pt}
\[
\infer[{\righti}_R]
{\foct{\Psi}{\Delta}{\tlamr{N}}{A^+ \righti B^-}}
{\foct{\Psi}{\mkconj{\Delta}{A^+}}{N}{B^-}}
\quad
\infer[{\righti}_L]
{\foct{\Psi}{\frameoff{\Theta}{\matchconj{[A \righti B]}{\Delta_A}}}
   {\tappr{V}{\Sp}}{U}}
{\foct{\Psi}{\Delta_A}{V}{[A^+]}
 &
 \foct{\Psi}{\tackon{\Theta}{[B]}}{\Sp}{U}}
\]
\caption{Multiplicative, exponential fragment of focused \ollll.
(Contexts $\Psi$ suppressed.)}
\label{fig:foc-mall}
\end{figure}

\renewcommand{\foct}[4]{{#1}; {#2} \vdash {#3} : {#4}}

\renewcommand{\foct}[4]{{#2} \vdash {#3} : {#4}}

\begin{figure}
\small

\[
\infer[{\zero}_L]
{\foct{\Psi}{\frameoff{\Theta}{\zero}}{\tabort}{U}}
{}
\quad
\infer[{\oplus}_{R1}]
{\foct{\Psi}{\Delta}{\tinl{V}}{[A^+ \oplus B^+]}}
{\foct{\Psi}{\Delta}{V}{[A^+]}}
\quad
\infer[{\oplus}_{R2}]
{\foct{\Psi}{\Delta}{\tinr{V}}{[A^+ \oplus B^+]}}
{\foct{\Psi}{\Delta}{V}{[B^+]}}
\]
\vspace{-5pt}
\[
\infer[{\oplus}_L]
{\foct{\Psi}{\frameoff{\Theta}{A^+ \oplus B^+}}{\toplusl{N_1}{N_2}}{U}}
{\foct{\Psi}{\tackon{\Theta}{A^+}}{N_1}{U}
 &
 \foct{\Psi}{\tackon{\Theta}{B^+}}{N_2}{U}}
\]
\vspace{-5pt}
\[
\infer[\top_R]
{\foct{\Psi}{\Delta}{\ttopr}{\top}}
{}
\quad
\infer[{\with}_R]
{\foct{\Psi}{\Delta}{\twithr{N_1}{N_2}}{A^- \with B^-}}
{\foct{\Psi}{\Delta}{N_1}{A^-}
 &
 \foct{\Psi}{\Delta}{N_2}{B^-}}
\]
\vspace{-5pt}
\[
\infer[{\with}_{L1}]
{\foct{\Psi}{\frameoff{\Theta}{[A^- \with B^-]}}{\tpione{\Sp}}{U}}
{\foct{\Psi}{\tackon{\Theta}{[A^-]}}{\Sp}{U}}
\quad
\infer[{\with}_{L2}]
{\foct{\Psi}{\frameoff{\Theta}{[A^- \with B^-]}}{\tpitwo{\Sp}}{U}}
{\foct{\Psi}{\tackon{\Theta}{[B^-]}}{\Sp}{U}}
\]


\caption{Additive connectives of focused \ollll.
(Contexts $\Psi$ suppressed.)}
\label{fig:foc-add}
\end{figure}

\renewcommand{\foct}[4]{{#1}; {#2} \vdash {#3} : {#4}}

\begin{figure}[t]
\small

\[
\infer[{\exists}_{R}]
{\foct{\Psi}{\Delta}{\texistsr{t}{V}}{[\exists a{:}\tau.A^+]}}
{\Psi \vdash t : \tau
 &
 \foct{\Psi}{\Delta}{V}{[A^+[t/a]]}}
\quad
\infer[{\exists}_L]
{\foct{\Psi}{\frameoff{\Theta}{\exists a{:}\tau.A^+}}{\texistsl{a}{N}}{U}}
{\foct{\Psi, a{:}\tau}{\tackon{\Theta}{A^+}}{N}{U}}
\]
\vspace{-5pt}
\[
\infer[{\doteq}_R]
{\foct{\Psi}{\cdot}{\phi}{t \doteq t}}
{}
\quad
\infer[{\doteq}_L]
{\foct{\Psi}{\frameoff{\Theta}{t \doteq s}}{\phi}{U}}
{\forall(\Psi' \vdash \sigma : \Psi).
 &
 \sigma t = \sigma s
 &
 \longrightarrow
 &
 \foct{\Psi'}{\tackon{\sigma\Theta}{\cdot}}{\phi(\sigma)}{\sigma U}
 }
\]
\vspace{-5pt}
\[
\infer[\forall_R]
{\foct{\Psi}{\Delta}{\tforallr{a}{N}}{\forall a{:}\tau.A^-}}
{\foct{\Psi, a{:}\tau}{\Delta}{N}{A^-}}
\quad
\infer[{\forall}_L]
{\foct{\Psi}{\frameoff{\Theta}{[\forall a{:}\tau.A^-]}}{\tforalll{t}{\Sp}}{U}}
{\Psi \vdash t : \tau
 &
 \foct{\Psi}{\tackon{\Theta}{[A^-[t/a]]}}{\Sp}{U}}
\]


\caption{First-order connectives of focused \ollll.}
\label{fig:foc-fo}
\end{figure}


The multiplicative and exponential fragment of focused \ollll~is given
in Figure~\ref{fig:foc-mall}, the additive fragment is given in
Figure~\ref{fig:foc-add}, and the first-order connectives are treated
in Figure~\ref{fig:foc-fo}. These rules are all written with sequents
of the form $\foct{\Psi}{\Delta}{E}{U}$, where $E$ is a {\it proof
  term} that corresponds to a derivation of that sequent. Just
as sequent forms are divided into the right-focused, inverting, and
left-focused sequents, we divide expressions into {\it values} $V$,
derivations of right-focused sequents; {\it terms} $N$,
derivations of inverting sequents; and {\it spines} $\Sp$,
derivations of left-focused sequents. The structure of
values, terms, and spines is as follows:
\begin{align*}
V & ::= z                     %% z
   \mid \tbangr{N}            %% !N
   \mid \tgnabr{N}            %% $N
   \mid \tdownr{N}            %% N
%
   \mid \toner
   \mid \tfuser{V_1}{V_2}     %% V1 * V2
%  
   \mid \tinl{V}
   \mid \tinr{V}
   \mid \texistsr{t}{V}
   \mid \tunifr
 \\
N & ::= \tfocusr{V}           %% V
   \mid \tfocusl{x}{\Sp}      %% x @ Sp
   \mid \tetap{z}{N}          %% <z>.N
   \mid \tetan{N}             %% <N>
   \mid \tdownl{x}{N}         %% x.N
   \mid \tgnabl{x}{N}         %% $x.N
   \mid \tbangl{x}{N}         %% !x.N
   \mid \tupr{N}              %% N
   \mid \tlaxr{N}             %% {N}
\\ & ~~~~ %
   \mid \tfusel{N}            %% *N
   \mid \tlaml{N}             %% <N
   \mid \tlamr{N}             %% >N
%
   \mid \tabort
   \mid \toplusl{N_1}{N_2}    %% [N1,N2]
   \mid \ttopr 
   \mid \twithr{N_1}{N_2}     %% N1 & N2
   \mid \texistsl{a}{N}
   \mid \tforallr{a}{N}
   \mid \phi
\\
\Sp & ::= \tnil               %% nil
   \mid \tupl{N}              %% N
   \mid \tlaxl{N}             %% {N} 
   \mid \tappl{V}{\Sp}        %% V < Sp
   \mid \tappr{V}{\Sp}        %% V > Sp
   \mid \tpione{\Sp}          %% fst Sp
   \mid \tpitwo{\Sp}          %% snd Sp
   \mid \tforalll{t}{\Sp}
\end{align*}


Expressions are treated as in the structural focalization methodology
\cite{simmons11structural}. It is possible to take a ``Curry-style''
view of expressions as {\it extrinsically} typed, which means we
consider both well-typed and ill-typed expressions; the well-typed
expressions are those for which the sequent
$\foct{\Psi}{\Delta}{E}{U}$ is derivable. However, we will take the
``Church-style'' view that expressions are intrinsically typed
representatives of derivations: that is, $\foct{\Psi}{\Delta}{E}{U}$
expresses that $E$ is a derivation of the sequent
$\foc{\Psi}{\Delta}{U}$. Justifying this close correspondence requires
us to take care that the structure of expressions is faithful to the
structure of proofs; this is the primary reason that we don't
introduce the patterns that are common in other proof term assignments
for focused logic
\cite{watkins02concurrent,licata08focusing,krishnaswami09focusing}.

There are two caveats to the idea that expressions are representatives
of derivations. One caveat is that, in order for there to be an actual
correspondence between expressions and terms, we need to annotate all
variables with the judgment they are associated with, and we need to
annotate $\tinr{V}$, $\tinl{V}$, $\tpione{\Sp}$, and $\tpitwo{\Sp}$
terms with the type of the branch not taken. Pfenning writes these as
superscripts \cite{pfenning08church}, but which we will follow Girard
in leaving them implicit \cite{girard89proofs}. The second caveat is
that, because we do not explicitly represent the significant
bookkeeping associated with matching constructs in proof terms, if
$\foct{\Psi}{\Delta}{E}{U}$, then $\foct{\Psi, a{:}\tau}{\Delta,
  x{:}\ispers{A^+}}{E}{U}$ as well. Therefore, even given appropriate
type annotations, when we say that some expression $E$ is a derivation
of $\foc{\Psi}{\Delta}{U}$, it is only {\it uniquely} a derivation of
that sequent if we account for the implicit bookkeeping.

Our ability to imagine proof terms as being fully independnetly typed
is an advantage to presening a proof theory that is not fully
dependent (which is to say, the domain of first-order quantification
does {\it not} include expressions $E$). Presentations of hereditary
substitution for LF (which is the computational conent of cut
admissibility) need to present a simply typed term language
\cite{watkins02concurrent,harper07mechanizing} or an untyped term
language \cite{reed07properties,martens11mechanizing} and then impose
a Curry-style dependent typing on top of them.

The ability of being able to give derivations fully intrinsic types
is one of the advantages to presenting a proof theory that is not
fully dependent; the proof terms that we present are {\it not} 

% Fundamentally, this is just a statement of the fact that
% weakening holds for persistent facts and first-order variables in our
% logic.

The use of proof terms allows for a greatly compressed presentation of
cut admissibility and identity expansion. Furthermore, we will reuse
much of this notation in our formulation of a logical framework in the
next chapter. 

\subsection{Variable substitution}

The first-order variables introduced by universal quantifiers (on the
right) and existential quantifiers (on the left) are {\it variables},
and as this is the case, the meaning of first-order variables is given
by substitution \cite[Chapter 1]{harper12practical}. A sequent with
free variables is thus a {\it generic} representative of all the
sequents that can be obtained by plugging terms in for those free
variables through the operation of substitution. This intuition is
formalized by the variable substitution theorem,
Theorem~\ref{thm:varsubst}.

\bigskip
\begin{theorem}[Variable substitution]
\label{thm:varsubst}
If $\Psi' \vdash \sigma : \Psi$ and $\foc{\Psi}{\Delta}{U}$, then 
$\foc{\Psi'}{\sigma\Delta}{\sigma{U}}$.
\end{theorem}

\begin{proof}
On the level of proof terms, 
we are given $E$, a expression corresponding to a derivation of
$\foc{\Psi}{\Delta}{U}$; we are defining the operation $\sigma{E}$,
an expression corresponding to a derivation of 
$\foc{\Psi'}{\sigma\Delta}{\sigma{U}}$.

\paragraph{\it Propositional fragment}
For the exponential, multiplicative, and additive fragments, this
operation is simple to define at the level of proof terms, and we will
omit most of the cases: $\sigma(\tfuser{V_1}{V_2}) =
\tfuser{\sigma{V_1}}{\sigma{V_2}}$, $\sigma(\tdownl{x}{N}) =
\tdownl{x}{\sigma N}$, and so on. (Note that first-order variables
$a$ do not interact with variables $x$ and $z$ in the substructural
context.) However, this compact notation does
capture a great deal of complexity. In particular, it is important to
emphasize that we need lemmas saying that variable substitution is
compatible with all the context matching operations from
Section~\ref{sec:contexts}.  In full detail, these two simple cases
would be:

\begin{itemize}

\item[--]
$\sigma(\tfuser{V_1}{V_2}) = \tfuser{\sigma{V_1}}{\sigma{V_2}}$\smallskip\\
We are given a proof of $\foc{\Psi}{\Delta}{[A^+ \fuse B^+]}$ that
ends with the ${\fuse}_R$ rule; the subderivations are
$V_1$, a derivation of $\foc{\Psi}{\Delta_1}{[A^+]}$, and
$V_2$, a derivation of $\foc{\Psi}{\Delta}{[B^+]}$. Furthermore, we know that
$\Delta$ matches $\Delta_1, \Delta_2$. We need a lemma that
tells us that $\sigma\Delta$ matches $\sigma\Delta_1, \sigma\Delta_2$;
then, by rule ${\fuse}_R$, it suffices to show that
$\foc{\Psi'}{\sigma\Delta_1}{\sigma{A^+}}$ (which we have by the 
induction hypothesis on $\sigma$ and $V_1$) and that
$\foc{\Psi'}{\sigma\Delta_2}{B^+}$ (which we have by the induction hypothesis
on $\sigma$ and $V_2$). \smallskip

\item[--]
$\sigma(\tdownl{x}{N}) = \tdownl{x}{\sigma N}$ \smallskip\\ 
We are given a proof
of $\foc{\Psi}{\Delta}{U}$ that ends with ${\downarrow}_L$; 
the subderivation is $N$, a derivation of
$\foc{\Psi}{\tackon{\Theta}{x{:}\istrue{A^-}}}{U}$. Furthermore, we know that
$\Delta$ matches $\frameoff{\Theta}{{\downarrow}A^-}$. We need a lemma
that tells us that $\sigma\Delta$ matches
$\frameoff{\sigma\Theta}{{\downarrow}\sigma A^-}$; then, by 
rule ${\downarrow}_L$, it suffices to show 
$\foc{\Psi'}{\tackon{\sigma\Theta}{x{:}\istrue{\sigma{A^-}}}}{U}$ (which
we have by the induction hypothesis on $\sigma$ and $N$).

\end{itemize}

\paragraph{\it First-order fragment} For the first-order fragment, we need
to be more careful, especially when considering the 
$\doteq_L$ rule, which notably
does {\it not} require an invocation of the induction hypothesis. The cases
for the $\exists$ quantifier mimic the ones we give for the $\forall$
quantifier, and so the discussion of these cases.

\begin{itemize}

\item[--]
$\sigma(\texistsr{t}{N}) = \texistsr{t}{\sigma{N}}$ 

\item[--]
$\sigma(\texistsl{a}{\Sp}) = \tforalll{a}{(\sigma, a/a)\Sp}$ 

\item[--]
$\sigma(\tunifr) = \tunifr$

\item[--]
$\sigma(\phi) = \left( \mathsf{fn}~\sigma' \Rightarrow \phi (\sigma'
  \circ \sigma) \right)$ \smallskip\\ We are given a proof of
$\foc{\Psi}{\Delta}{U}$ that ends with ${\doteq}_L$; we know that
$\Delta$ matches $\frameoff{\Theta}{t \doteq s}$, and the
subderivation is $\phi$, a function from substitutions 
$\Psi'' \vdash \sigma'' : \Psi$ that unify $t$ and $s$ to derivations
of $\foc{\Psi''}{\tackon{\sigma''\Theta}{\cdot}}{\sigma''{U}}$. We need a lemma
that tells us that $\sigma\Delta$ matches 
$\frameoff{\sigma\Theta}{\sigma t \doteq \sigma s}$; then, by rule
${\doteq}_L$, it suffices to show that for all 
$\Psi'' \vdash \sigma' : \Psi'$ that unify $\sigma t$ and $\sigma s$, 
there exists a derivation of 
$\foc{\Psi''}{\tackon{\sigma'(\sigma \Theta)}{\cdot}}{\sigma'(\sigma U)}$,
which is the same thing as a derivation of 
$\foc{\Psi''}{\tackon{(\sigma' \circ \sigma) \Theta}{\cdot}}
    {(\sigma' \circ \sigma) U}$. We have that 
$\Psi'' \vdash \sigma' \circ \sigma : \Psi$, and certainly 
$\sigma' \circ \sigma$ unifies $t$ and $s$, so we can finish by
applying $\sigma' \circ \sigma$ to $\phi$.\smallskip

\item[--]
$\sigma(\tforallr{a}{N}) = \tforallr{a}{(\sigma, a/a)N}$ \smallskip\\ We are
given a proof of $\foc{\Psi}{\Delta}{\forall a{:}\tau.A^-}$ 
that ends with $\forall_R$; the subderivation
is $N$, a derivation of $\foc{\Psi,a{:}\tau}{\Delta}{A^-}$. Because
$\sigma(\forall{a}{:}{\tau}.A^-) 
 = \forall{a}{:}{\sigma\tau}.(\sigma, a/a){A^-}$,
by rule $\forall_R$ it suffices to show 
$\foc{\Psi', a{:}\sigma\tau}{\sigma\Delta}{(\sigma, a/a)A^-}$, 
which is the same thing
as $\foc{\Psi', a{:}\sigma\tau}{(\sigma, a/a)\Delta}{(\sigma, a/a)A^-}$.
The result
follows by the induction hypothesis on $(\sigma, a/a)$ and $N$. \smallskip

\item[--]
$\sigma(\tforalll{t}{\Sp}) = \tforalll{\sigma{t}}{\sigma\Sp}$ \smallskip\\
We are given a proof of $\foc{\Psi}{\Delta}{U}$ that ends with 
$\forall_L$; the subderivation is $\Sp$, a derivation of 
$\foc{\Psi}{\tackon{\Theta}{[t/a]\Sp}}{U}$. Furthermore,
we know that $\Delta$ matches $\frameoff{\Theta}{[\forall a{:}\tau.A]}$.
We need a lemma that tells us that $\sigma\Delta$ matches
$\frameoff{\sigma\Theta}{[\forall a{:}\tau.(\sigma, a/a)A^-]}$; then,
by rule $\forall_L$, it suffices to show 
$\foc{\Psi'}{\tackon{\sigma\Theta}{[[t/a](\sigma, a/a)A^-]}}
  {\sigma{U}}$, which is the same thing as
$\foc{\Psi'}{\tackon{\sigma\Theta}{[\sigma([t/a]A^-)]}}
  {\sigma{U}}$. This follows by the induction hypothesis on $\sigma$ and
$\Sp$.

\end{itemize}

Note that, in the case for $\forall_R$, the substitution $\sigma$ was
applied to the first-order type $\tau$ as well as to as at the fact
that our first-order terms are dependently typed.
\end{proof}

Given we write the constructive content of the variable substitution
theorem as $\sigma{E}$, where $E$ is an expression,
we can also write Theorem~\ref{thm:varsubst} as an admissible
rule in one of two ways, both with and without proof terms:
\[
\infer-[{\it varsubst}]
{\foct{\Psi'}{\sigma{\Delta}}{\sigma{E}}{\sigma{U}}}
{\Psi' \vdash \sigma : \Psi
 &
 \foct{\Psi}{\Delta}{E}{U}}
\qquad
\infer-[{\it varsubst}]
{\foc{\Psi'}{\sigma{\Delta}}{\sigma{U}}}
{\Psi' \vdash \sigma : \Psi
 &
 \foc{\Psi}{\Delta}{U}}
\]
We will tend towards the expression-annotated presentations, such as
the one on the left, in this chapter.

\subsection{Focal substitution}

Both cut admissibility and identity expansion depend on the same
focal substation theorem that was considered for linear logic in 
Section~\ref{sec:lin-suspended}. Both of these theorems use the
compound matching construct $\frameoff{\Theta}{\restrictto{\Delta}{\mlvl}}$,
a pattern that will also be used in the proof of cut admissibility: 
$\Delta'$ matches $\frameoff{\Theta}{\restrictto{\Delta}{\mlvl}}$
if $\Delta'$ matches $\frameoff{\Theta}{\Delta''}$
and $\Delta''$ matches $\restrictto{\Delta}{\mlvl}$.

\bigskip
\begin{theorem}[Focal substitution]~
\begin{itemize}
\item If $\foc{\Psi}{\Delta}{[A^+]}$, ~
      $\foc{\Psi}{\tackon{\Theta}{z{:}\islvl{\susp{A^+}}}}{U}$,\\
      and $\Xi$ matches $\frameoff{\Theta}{\restrictto{\Delta}{\mlvl}}$, ~
      then $\foc{\Psi}{\Xi}{U}$
\item If $\foc{\Psi}{\Delta}{\islvl{\susp{A^-}}}$, ~
      $\foc{\Psi}{\tackon{\Theta}{[A^-]}}{U}$, \\
      $\Xi$ matches $\frameoff{\Theta}{\Delta}$, ~
      and $U'$ matches $\restrictfrom{U}{\mlvl}$, ~
      then $\foc{\Psi}{\Xi}{U'}$
\end{itemize}
\end{theorem}

\begin{proof}
Both proofs proceed by induction over the derivation $E$ containing
the suspended proposition; this is the second derivation in positive 
focal substitution and the first derivation in negative focal substitution. 
\end{proof}

Pay attention to the way this compound matching construct is being
used. In unfocused linear logic, using the compound notation
effectively means that a single statement:
\begin{center}
If $\Delta \Rightarrow \istrue{A}$, ~
$\tackon{\Theta}{x{:}\islvl{A}} \Rightarrow \istrue{C}$, ~
and $\Delta'$ matches $\frameoff{\Theta}{\restrictto{\Delta}{\mlvl}}$, ~
then $\Delta' \Rightarrow C$.
\end{center}
can simultaneously express 
three cut
principles: \smallskip
\begin{itemize}
\item If $\oseq{\Gamma}{\cdot}{\cdot}{A}$ 
      and $\oseq{\Gamma, A}{\Delta'}{\Omega'}{C}$,
      then $\oseq{\Gamma}{\Delta'}{\Omega'}{C}$.
\item If $\oseq{\Gamma}{\Delta}{\cdot}{A}$ 
      and $\oseq{\Gamma}{\Delta', A}{\Omega'}{C}$,
      then $\oseq{\Gamma}{\Delta', \Delta}{\Omega'}{C}$.
\item If $\oseq{\Gamma}{\Delta}{\Omega}{A}$ 
      and $\oseq{\Gamma}{\Delta'}{\Omega_L, A, \Omega_R}{C}$,
      then $\oseq{\Gamma}{\Delta',\Delta}{\Omega_L, \Omega, \Omega_R}{C}$.
\end{itemize}
\smallskip
As an admissible rule, we would write this
as follows:
\[
\infer-[{\it unfocused\mbox{-}cut}]
{\frameoff{\Theta}{\restrictto{\Delta}{\mlvl}} \Rightarrow \istrue{C}}
{\Delta \Rightarrow \istrue{A}
 &
 \tackon{\Theta}{x{:}\islvl{A}} \Rightarrow \istrue{C}}
\]

In the negative focal substation (as the leftist substitutions of
cut admissibility), there will be a corresponding use of
$\restrictfrom{U}{\mlvl}$ to capture that we can use a proof of
$\istrue{A}$ to discharge a hypothesis of $\istrue{A}$ in a proof of
$\istrue{C}$ or a proof of $\islax{C}$, but that a proof of
$\islax{A}$ can only discharge a hypothesis of $\istrue{A}$ in a proof
of $\islax{C}$. 


The computational content of positive focal substitution is the
substitution of a value $V$ for a variable $z$ in an expression $E$, 
written $[V/z]E$. The computational
content of negative focal substitution is the substitution of a spine $\Sp$,
which represents a continuation, into an expression $E$ waiting on that 
continuation, written $[E]\Sp$. As admissible rules, focal substitution is
represented as follows:
\[
\infer-[{\it subst}^+]
{\foct{\Psi}{\frameoff{\Theta}{\restrictto{\Delta}{\mlvl}}}{[V/z]E}{U}}
{\foct{\Psi}{\Delta}{V}{[A^+]}
 &
 \foct{\Psi}{\tackon{\Theta}{z{:}\islvl{\susp{A^+}}}}{E}{U}}
\]
\[
\infer-[{\it subst}^-]
{\foct{\Psi}{\frameoff{\Theta}{\Delta}}{[ E ] \Sp}{\restrictfrom{U}{\mlvl}}}
{\foct{\Psi}{\Delta}{E}{\islvl{\langle A^- \rangle}}
 &
 \foct{\Psi}{\tackon{\Theta}{[A^-]}}{\Sp}{U}}
\]

\section{Cut admissibility}
\label{sec:ord-cut}

It is a little wordy to say that, in a context or succeedent, the only
judgments involving suspensions are $(\ispers{\susp{p^+_\mpers}})$,
$(\iseph{\susp{p^+_\meph}})$, $(\istrue{\susp{p^+}})$,
$(\istrue{\susp{p^-}})$, and $(\islax{\susp{p^-_\mlax}})$, but this is a
critical precondition of cut admissibility property for focused
\ollll. We'll call contexts and succeedants with this property {\it
  suspension-normal}.

\begin{theorem}[Cut admissibility]\label{thm:ord-cut}
For suspension-normal $\Psi$, $A^+$, $A^-$, $\Delta$, $\Theta$, $\Xi$, $U$, and $U'$,
\begin{enumerate}
\item If $\foc{\Psi}{\Delta}{[ A^+ ]}$, ~
         $\foc{\Psi}{\tackon{\Theta}{A^+}}{U}$,\\
      and $\Xi$ matches $\frameoff{\Theta}{{\Delta}}$, ~
      then $\foc{\Psi}{\Xi}{U}$.
\item If $\foc{\Psi}{\Delta}{A^-}$, ~
         $\foc{\Psi}{\tackon{\Theta}{[A^-]}}{U}$, ~
         $\Delta$ is stable, \\
      and $\Xi$ matches $\frameoff{\Theta}{\Delta}$, ~
      then $\foc{\Psi}{\Xi}{U}$.
\item If $\foc{\Psi}{\Delta}{\islvl{A^+}}$, ~
         $\foc{\Psi}{\tackon{\Theta}{A^+}}{U}$, ~
         $\Theta$ and $U$ are stable, \\ 
      $\Xi$ matches $\frameoff{\Theta}{\Delta}$, ~
      and $U'$ matches $\restrictfrom{U}{\mlvl}$, ~
      then $\foc{\Psi}{\Xi}{U'}$.
\item If $\foc{\Psi}{\Delta}{\istrue{A^-}}$, ~
         $\foc{\Psi}{\tackon{\Theta}{x{:}\islvl{A^-}}}{U}$, ~
         $\Delta$ is stable, \\ 
      and $\Xi$ matches $\frameoff{\Theta}{\restrictto{\Delta}{\mlvl}}$,
      then $\foc{\Psi}{\Xi}{U}$
\end{enumerate}
\end{theorem}
\medskip

\noindent
The four cases of cut admissibility (and their proof below) neatly
codify an observation about the structure of cut admissibility proofs
made by Pfenning in his work on structural cut elimination
\cite{pfenning00structural}.  The first two parts of
Theorem~\ref{thm:ord-cut} are the home of the {\it principal cases}
that decompose both derivations simultaneously, the third part
contains all the {\it left commutative cases} that perform case
analysis and induction only on the first given derivation, and the
fourth part contains all the {\it right commutative cases} that
perform case analysis and induction only on the second given
derivation.

In Pfenning's work on structural cut eliminiation, this classification
of cases was informal, but the structure of our cut admissibility
proofs actually isolates the principal, left commutative, and right
commutative cases into different parts of the theorem
\cite{pfenning00structural}. This separation of cases is the reason
why cut admissibility in a focused sequent calculus can use a more
refined induction metric than cut admissibility in an unfocused
sequent calculus. As noted previously \cite{simmons11structural}, the
refined induction metric can make the theorem easier to prove.  For
instance, the refined induction metric means that we do not need to
rely on the fact that weakening and variable substitution preserve
the size of proofs.

Before discussing the proof, it is worth noting that this theorem
statement is already a sort of victory. It is an extremely simple
statement of cut admissibility for a rather complex logic.

\subsection{Optimizing the statement of cut admissibility}

We will pick the cut admissibility proof
from Chaudhiri's thesis \cite{chaudhuri06focused} as a represenative
example of existing work on cut admissibility in focused logics.  His
statement of cut admissibility linear logic has 10 parts, 
which are sorted into
five groups. In order to extend his
proof structure to handle the extra lax and mobile modalities in
\ollll, we would need a dramatically larger number of
cases. Furthermore, at a computational level, Chaudhuri's proof
requires a lot of code duplication -- that is, the proof of two
different parts may both require a case that looks essentially the
same.

A great deal of simplification is due to the use of
$\restrictto{\Delta}{\mlax}$. Without that notation, part 3 would
split into two parts for $\mtrue$ and $\mlax$ and part 4 would split
into three parts for $\mtrue$, $\meph$, and $\mpers$. The fifth part
of Theorem~\ref{thm:lincut} in Chapter 2, which is computationally a
near-duplicate of the fourth part of the same theorem, is due to the
lack of this simplification.

Further simplification is due to defining right-focused, inverting,
and left-focused sequents as refinements of general sequents
$\foc{\Psi}{\Delta}{U}$. This strategy was not used in the structural
focalization development, and while no code duplication resulted, the
statement of part 3 was split into two parts (for substituing into
terms and spines) and the statement of part 4 was split in three (for
substituting into values, terms, spines) \cite{simmons11structural}.
Without either of the aforementioned simplfications, we would have 15
parts in the statement of Theorem~\ref{thm:ord-cut} instead of four
and about a 2x blowup in the total number of cases that needed to be
written down and checked.

A final improvement in our theorem statement is very subtle: the {\it
  particular} fixed inversion strategy we chose matters.  The use of a
fixed inversion strategy is critical to avoid two problems with the
development in Chapter~2. The more serious problem, dicussed in
Section~\ref{sec:confluent-v-fixed}, is the need to prove a tedious,
quadratically large confluence theorem in order to define a strong
notion of canonical forms. This first problem is solved by the choice
of {\it any} fixed inversion strategy. The less serious problem is
that the proof of Theorem~\ref{thm:lincut} duplicates many right
commutative cases in both part 1 and part 4 (which map directly onto
parts 1 and 4 of Theorem~\ref{thm:ord-cut} above). Our system
prioritizes the inversion of positive formulas on the left over the
inversion of negative formulas on the right. If we made the opposite
choice, as Chaudhri's system does, then this (smaller) problem 
would remain. We get a lot of mileage out of the
fact that if $\Xi = \tackon{\Theta}{A^+}$ then $A^+$ unambiguously
refers to the left-most proposition in $\Xi$, and this invariant would
no longer be possible to maintain in the statement of cut
admissibility if we prioritized inversion of negative propositions on
the right.


\subsection{Proof of cut admissibility, Theorem~\ref{thm:ord-cut}}

The proof proceeds by lexicographic induction.  In parts 1 and 2, the
type gets smaller in every call to the induction hypothesis. In part
3, the induction hypothesis is only ever invoked on the same type
$A^+$, and every invocation of the induction hypothesis is either to
part 1 (smaller part number) or to part 3 (same part number, first
derivation is smaller). Similarly, in part 4, the induction hypothesis
is only invoked at the same type $A^-$, and every invocation of the
induction hypothesis is either to part 2 (smaller part number) or to
part 4 (same part number, second derivation is smaller).

The remainder of this section will cover each of the four parts of
this proof in turn. Most of the theorem will be presented at the level
of proof terms, but for representative cases we will discuss what the
manipulation of proof terms means in terms of sequents and matching
constructs. In many cases, we discuss the necessity of constructing
certain contexts or frames; 

\subsubsection{Positive principal substitution}
Positive principal substitution encompasses half the {\it principal
  cuts} from Pfenning's structural cut admissibility proof -- the
principal cuts where the principal cut formula is positive. The
constructive content of this part is a function
$(\subst{V}{N})^{A^+}$ that normalizes a value against a
term. Induction is on the structure of the positive type. The
admissible rule associated with principal positive substitution is
${\it cut}^+$.
\[
\infer-[{\it cut}^+]
{\foct{\Psi}{\frameoff{\Theta}{\Delta}}{(\subst{V}{N})^{A^+}}{U}}
{\foct{\Psi}{\Delta}{V}{[A^+]}
 &
 \foct{\Psi}{\tackon{\Theta}{A^+}}{N}{U}}
\]
We have to be careful, in the positive principal substitution
associated with the type $A^+ \fuse B^+$, to maintain the
invariant that, in an unstable context, we only ever consider the {\it
  leftmost} inverting positive proposition.

\begin{itemize}
\item[--] $(\subst{z}{\tetap{z'}{N}})^{p^+_{\mlvl}} = [z/z']N$\smallskip\\
  We must show $\foc{\Psi}{\Xi}{U}$, where 
  \begin{itemize}
   \item $\Delta$ matches $z{:}\susp{p^+_\mlvl}$,
   \item $\tackon{\Theta}{p^+}$ matches $\frameoff{\Theta'}{p^+}$,
      $N$ is a derivation of 
      $\foc{\Psi}{\tackon{\Theta'}{z'{:}\islvl{\susp{p^+_\mlvl}}}}{U}$,
   \item and $\Xi$ matches $\frameoff{\Theta}{\Delta}$.
  \end{itemize}
  Because $\Delta$ is suspension-normal, there exists $\Delta'$ where
  $\Delta$ such that $\restrictto{\Delta'}{\mlvl}$ and
  $z{:}\susp{\Delta'}$ matches $\Delta'$.\footnote{Strictly
    speaking, we have that $\Delta$ matches
    $\restrictto{\Delta}{\mlvl}$; we introduce a new $\Delta'$ with
    affine logic in mind.} Therefore, we can derive
  $\foc{\Psi}{\Delta'}{[p^+_\mlvl]}$ by ${\it id}^+$.

  $\Xi$ matches
  $\frameoff{\Theta'}{\restrictto{\Delta'}{\mlvl}}$,
  so the result follows by focal substitution on $z$ and $N$. 

\futurework{I'm relying on a property I haven't explicitly proven:
 which feels familiar from the Agda developments:
 if $\Xi$   matches $\frameoff{\Theta}{\Delta}$
 and $\tackon{\Theta}{x:\islvl{T}}$ 
            matches $\frameoff{\Theta'}{x{:}\islvl{T}}$
 then $\Xi$ matches $\frameoff{\Theta}{\Delta}$.  }
\smallskip
 
\item[--] $(\subst{\tdownr{M}}{\tdownl{x}{N}})^{{\downarrow}A^-} 
           = \rsubsta{M}{x}{N}{A^-}$ %\\
%   We must show $\foc{\Psi}{\Xi}{U}$, where
%   \begin{itemize}
%   \item $M$ is derivation of $\foc{\Psi}{\Delta}{A^-}$, $\Delta$ is stable,
%   \item $\tackon{\Theta}{{\downarrow}A^-}$ matches 
%         $\frameoff{\Theta'}{{\downarrow}A^-}$,
%         $N$ is a derivation of 
%         $\foc{\Psi}{\tackon{\Theta'}{x{:}\istrue{A^-}}}{U}$, 
%   \item and $\Xi$ matches $\frameoff{\Theta}{\Delta}$.
%   \end{itemize}

%   $\Xi$ matches $\frameoff{\Theta'}{\Delta}$, so the result follows by
%   part 4 of cut admissibility on $N$ and $M$. 


\item[--] $(\subst{\tgnabr{M}}{\tgnabl{x}{N}})^{{\scriptgnab}A^-}
           = \rsubsta{M}{x}{N}{A^-}$

\item[--] $(\subst{\tbangr{M}}{\tbangl{x}{N}})^{{!}A^-}
           = \rsubsta{M}{x}{N}{A^-}$ \smallskip\\
  We must show $\foc{\Psi}{\Xi}{U}$, where
  \begin{itemize}
  \item $\Delta$ matches $\restrictto{\Delta'}{\mpers}$,
        $M$ is derivation of $\foc{\Psi}{\Delta'}{A^-}$, 
        $\Delta'$ is stable,
  \item $\tackon{\Theta}{{\downarrow}A^-}$ matches 
        $\frameoff{\Theta'}{{\downarrow}A^-}$,
        $N$ is a derivation of 
        $\foc{\Psi}{\tackon{\Theta'}{x{:}\istrue{A^-}}}{U}$,
  \item and $\Xi$ matches $\frameoff{\Theta}{\Delta}$.
  \end{itemize}

  $\Xi$ matches $\frameoff{\Theta'}{\restrictto{\Delta'}{\mpers}}$,
  so the result follows by
  part 4 of cut admissibility on $N$ and $M$. 

\futurework{Oh, stability has to be preserved my mapping. So many theorems!
 I'm glad I'm not proving this in Agda (this time).}
\smallskip

\item[--] $(\subst{\toner}{\tonel{N}})^{\one} = N$

\item[--] $(\subst{(\tfuser{V_1}{V_2})}{(\tfusel{N}}))^{A^+ \fuse B^+}
           = (\subst{V_2}{(\subst{V_1}{N})^{A^+}})^{B^+}$ \smallskip\\
  We must show $\foc{\Psi}{\Xi}{U}$, where
  \begin{itemize}
  \item $\Delta$ matches $\Delta_1, \Delta_2$,\\
        $V_1$ is a derivation of $\foc{\Psi}{\Delta_1}{[A^+]}$,
        $V_2$ is a derivation of $\foc{\Psi}{\Delta_2}{[B^+]}$,
  \item $\tackon{\Theta}{A^+ \fuse B^+}$ matches
        $\frameoff{\Theta'}{A^+ \fuse B^+}$, 
        $N$ is a derivation of 
        $\foc{\Psi}{\tackon{\Theta'}{A^+, B^+}}{U}$, 
  \item and $\Xi$ matches $\frameoff{\Theta}{\Delta}$.
  \end{itemize}

  It is clear we can to consturct a frame $\Theta_B$ such that
  $\tackon{\Theta'}{A^+, B^+} = \tackon{\Theta_B}{A^+}$; we're just exchanging
  the part in the frame with the part not in the frame. We also 
  must be able to construct a second frame, $\Theta_A$, such that
  1) $\Xi$ matches $\frameoff{\Theta_A}{\Delta_B}$ and 
  2) $\tackon{\Theta_A}{B^+}$ matches $\frameoff{\Theta_B}{\Delta_A}$.

\futurework{It's probably (I might say hopefully) not clear to the reader
 what a doozy that is, but I've done it in Agda once in that proof I lost, 
 I guess.}

  Because $\tackon{\Theta_A}{B^+}$ matches $\frameoff{\Theta_B}{\Delta_A}$,
  by the induction hypothesis on $V_1$ and $N$ we have
  $(\subst{V_1}{N})^{A^+}$, a derivation of 
  $\foc{\Psi}{\tackon{\Theta_A}{B^+}}{U}$.

  Because $\Xi$ matches $\frameoff{\Theta_A}{\Delta_B}$, by the induction
  hypothesis on $V_2$ and $(\subst{V_1}{N})^{A^+}$, we have a derivation
  of $\foc{\Psi}{\Xi}{U}$ as required. \smallskip

\item[--] $(\subst{\tinl{V}}{\toplusl{N_1}{N_2}})^{A^+ \oplus B^+} 
           = (\subst{V}{N_1})^{A^+}$

\item[--] $(\subst{\tinr{V}}{\toplusl{N_1}{N_2}})^{A^+ \oplus B^+} 
           = (\subst{V}{N_2})^{B^+}$

\item[--] $(\subst{\texistsr{t}{V}}{\texistsl{a}{N}})^{\exists a{:}\tau.A^+}
           = (\subst{V}{[t/a]N})^{[t/a]A^+}$ \smallskip\\
  We must show $\foc{\Psi}{\Xi}{U}$, where
  \begin{itemize}
  \item $\Psi \vdash t : \tau$, $V$ is a derivation of 
     $\foc{\Psi}{\Delta}{[[t/a]A^+]}$,
  \item $\tackon{\Theta}{\exists a{:}\tau. A^+}$ 
     matches $\frameoff{\Theta'}{\exists a{:}\tau. A^+}$, 
     $N$ is a derivation of 
     $\foc{\Psi, a{:}\tau}{\tackon{\Theta}{A^+}}{U}$,
  \item and $\Xi$ matches $\frameoff{\Theta}{\Delta}$.
  \end{itemize}
  $\Xi$ matches $\frameoff{\Theta'}{\Delta}$, and by the variable
  substitution on $[t/a]$ and $V$, we have a derivation $[t/a]N$ of
  $\foc{\Psi}{\tackon{\Theta}{[t/a]A^+}}{U}$.  We count $[t/a]A^+$ as
  being a smaller formula than $\exists x{:}\tau.A^+$, so by the
  induction hypothesis on $V$ and $[t/a]N$, we get a derivation of
  $\foc{\Psi}{\Xi}{U}$ as required. \smallskip

\item[--] $(\subst{\tunifr}{\phi})^{t \doteq t} = \phi({\sf id})$\smallskip\\
  We must show $\foc{\Psi}{\Xi}{U}$, where
  \begin{itemize}
  \item $\Delta$ matches $\cdot$,
  \item $\tackon{\Theta}{t \doteq t}$ matches $\frameoff{\Theta'}{t \doteq t}$,
     $\phi$ is a function from substitutions $\Psi' \vdash \sigma : \Psi$
     that unify $t$ and $t$ to derivations of 
     $\foc{\Psi}{\tackon{\Theta'}{\cdot}}{U}$,
  \item and $\Xi$ matches $\frameoff{\Theta}{t \doteq t}$.
  \end{itemize}
  We simply apply the identity substitution to $\phi$
  to obtain a derivation of $\foc{\Psi}{\tackon{\Theta'}{\cdot}}{U}$.
  Note that this is not the derivation of 
  $\foc{\Psi}{\Xi}{U}$ that we need; we need a simple lemma that, 
  given a derivation of $\foc{\Psi}{\tackon{\Theta'}{\cdot}}{U}$
  and the fact that $\Xi$ matches $\frameoff{\Theta'}{\cdot}$,
  we can get a proof of $\foc{\Psi}{\Xi}{U}$ as we require.

\end{itemize}

\subsubsection{Negative principal substitution}
Negative principal substitution encompass all the {\it principal cuts}
from Pfenning's structural cut admissibility proof for which the
principal formula is negative. The constructive content of this part
is a function $(\subst{N}{\Sp})^{A^-}$ that normalizes a term against
a spine; a similar function appears as {\it hereditary reduction} 
in presentations of hereditary
substitution for LF \cite{watkins02concurrent}. Induction is on the
structure of the negative type. The admissible rule associated with
negative principal substitution is ${\it cut}^-$:
\[
\infer-[{\it cut}^-]
{\foct{\Psi}{\frameoff{\Theta}{\Delta}}{(\subst{N}{\Sp})^{A^-}}{U}}
{\foct{\Psi}{\Delta}{N}{A^-}
 &
 \foct{\Psi}{\tackon{\Theta}{[A^-]}}{\Sp}{U}
 &
 \stableL{\Delta}}
\]

\begin{itemize}
\item[--] $(\subst{\tetan{N}}{\tnil})^{p^-_\mlvl} = N$\smallskip\\
   We must show $\foc{\Psi}{\Xi}{U}$, where
   \begin{itemize}
   \item $N$ is a derivation of $\foc{\Psi}{\Delta}{\islvl{\susp{p^-_\mlvl}}}$
   \item $\tackon{\Theta}{[p^-_\mlvl]}$ matches $[p^-_\mlvl]$,
      $U = \islvl{\susp{p^-_\mlvl}}'$,
   \item and $\Xi$ matches $\frameoff{\Theta}{\Delta}$.
   \end{itemize}
   Because $U$ is suspension-normal, $\mlvl = \mlvl'$.
   A derivation of $\foc{\Psi}{\Delta}{\islvl{\susp{p^-_\mlvl}}}$ is not
   quite a proof of $\foc{\Psi}{\Xi}{U}$, so we need a lemma that 
   we can get one from the other. \smallskip

\item[--] $(\subst{\tupr{N}}{\tupl{M}})^{{\uparrow}A^+} = \lsubsta{N}{M}{A^+}$

\item[--] $(\subst{\tlaxr{N}}{\tlaxl{M}})^{{\ocircle}A^+}
           = \lsubsta{N}{M}{A^+}$\smallskip\\
  We must show $\foc{\Psi}{\Xi}{U}$, where
  \begin{itemize}
  \item $N$ is a derivation of 
     $\foc{\Psi}{\Delta}{\islax{A^+}}$, 
  \item $\tackon{\Theta}{{\ocircle}A^+}$ matches 
     $\frameoff{\Theta'}{{\ocircle}A^+}$, 
     $U$ matches $\restrictfrom{U'}{\mlax}$, 
     $\Theta'$ and $U'$ are stable, 
     $M$ is a derivation of $\foc{\Psi}{\tackon{\Theta'}{A^+}}{U'}$,
  \item and $\Xi$ matches $\frameoff{\Theta}{\Delta}$.
  \end{itemize}
  $\Xi$ matches $\frameoff{\Theta'}{\Delta}$, so the result follows
  by part 3 of cut admissibility on $N$ and $M$. \smallskip

\item[--] $(\subst{(\tlaml{N})}{(\tappl{V}{\Sp})})^{A^+ \lefti B^-}
           = (\subst{(\subst{V}{N})^{A^+}}{\Sp})^{B^-}$

\item[--] $(\subst{(\tlamr{N})}{(\tappr{V}{\Sp})})^{A^+ \righti B^-} 
           = (\subst{(\subst{V}{N})^{A^+}}{\Sp})^{B^-}$\smallskip\\
  We must show $\foc{\Psi}{\Xi}{U}$, where
  \begin{itemize}
  \item $N$ is a derivation of 
     $\foc{\Psi}{\Delta, A^+}{B^-}$, $\Delta$ is stable (by the fixed
     inversion invariant -- we only invert on the right when there is 
     no further inversion to do on the left), 
  \item $\tackon{\Theta}{[A^+ \righti B^-]}$ matches 
     $\frameoff{\Theta'}{[A^+ \righti B^-], \Delta_A}$, 
     $V$ is a derivation of $\foc{\Psi}{\Delta_A}{[A^+]}$,
     $\Sp$ is a derivation of $\foc{\Psi}{\tackon{\Theta'}{[B^-]}}{U}$,
  \item and $\Xi$ matches $\frameoff{\Theta}{\Delta}$.
  \end{itemize}
  We can simultaneously view the construction $\Delta,A^+$ as a frame
  $\Theta_\Delta$ such that $\tackon{\Theta_\Delta}{A^+} =
  \Delta,A^+$.  Note that this is only possible to do because $\Delta$
  is stable; if there were a non-stable proposition in $\Delta$, the
  fixed inversion invariant would not permitted us to frame off the
  right-most proposition $A^+$.

  We next consturct a context $\Delta'_A$ that matches
  $\frameoff{\Theta'_A}{\Delta_A}$ (and also $\Delta, \Delta_A$ viewed
  as a matching construct), while simutaneously
  $\Xi$ matches $\frameoff{\Theta'}{\Delta'_A}$. 

  By the part 1 of cut admissibility
  on $V$ and $N$, we have $(\subst{V}{N})^{A^+}$, a derivation of 
  $\foc{\Psi}{\Delta'_A}{B^-}$, and 
  the result then follows by the induction hypothesis on 
  $(\subst{V}{N})^{A^+}$ and $\Sp$.  \smallskip

\item[--] $(\subst{(\twithr{N_1}{N_2})}{(\tpione{\Sp})})^{A^- \with B^-}
           = (\subst{N_1}{\Sp})^{A^-}$

\item[--] $(\subst{(\twithr{N_1}{N_2})}{(\tpitwo{\Sp})})^{A^- \with B^-}
           = (\subst{N_2}{\Sp})^{A^-}$

\item[--] $(\subst{(\tforallr{a}{N})}{(\tforalll{t}{\Sp})})^{\forall x.A^-}
           = (\subst{[t/a]N}{\Sp})^{[t/a]A^-}$
\end{itemize}

\subsubsection{Leftist substitution}
In focal substitution, the positive case
corresponds to our usual intuitions about cut admissibility and the
negative case is strange.  In cut admissibility, the situation is
reversed: rightist substitutions (considered in
Section~\ref{sec:rsubst} below), associated with negative principal
cut formals, look like normal substitutions, and the leftist 
substitutions, considered here, are strange, as they break
apart the expression that proves $A^+$ rather than the term
we are substituting into.

Leftist substitutions encompass all the {\it left commutative cuts}
from Pfenning's structural cut admissibility proof.
The constructive content of leftist substitution is a function
$\lsubst{E}{N}$. Induction is on the first subterm, as we crawl 
through $E$ looking for places where focus takes place on the 
right. The admissible rule associated with leftist substitution is
${\it lcut}$:
\[
\infer-[{\it lcut}]
{\foct{\Psi}{\frameoff{\Theta}{\Delta}}{\lsubsta{E}{M}{A^+}}{\restrictfrom{U}{\mlvl}}}
{\foct{\Psi}{\Delta}{E}{\islvl{A^+}}
 &
 \foct{\Psi}{\tackon{\Theta}{A^+}}{M}{U}
 &
 \stableL{\Theta}
 &
 \stableR{U}}
\]

Except for the case where the first given derivation ends in the rule
${\it focus}_R$, every case of this theorem involves a left rule in
which $\Delta$ matches $\frameoff{\Theta_B}{x{:}T}$ and it is
necessary to be able to construct a frame ${\Theta}{\circ}{\Theta_B}$
that is effectively the composition of $\Theta$ and $\Theta_B$. In
cases that we discuss in detail, necessary properties of this
composition frame are stated but not proven.

\paragraph{\it Substitution into terms}

\begin{itemize}
\item[--] $\lsubsta{\tfocusr{V}}{M}{A^+} = (\subst{V}{M})^{A^+}$\smallskip\\
  We must show $\foc{\Psi}{\Xi}{U'}$, where
  \begin{itemize}
  \item $V$ is a derivation of $\foc{\Psi}{\Delta}{[A^+]}$,
  \item $M$ is a derivation of $\foc{\Psi}{\tackon{\Theta}{A^+}}{U}$,
  \item $\Xi$ matches $\frameoff{\Theta}{\Delta}$, and $U'$ matches 
    $\restrictfrom{U}{\mlvl}$
  \end{itemize}
  $M$ is also a derivation of $\foc{\Psi}{\tackon{\Theta}{A^+}}{U}$,
  so the result follows from part 1 of cut admissibility
  on $V$ and $M$. \smallskip

\item[--] $\lsubsta{\tfocusl{x}{\Sp}}{M}{A^+}
           = \tfocusl{x}{(\lsubsta{\Sp}{M}{A^+})}$\smallskip\\
  We must show $\foc{\Psi}{\Xi}{U'}$, where
  \begin{itemize}
  \item $\Delta$ matches $\frameoff{\Theta_B}{x{:}B^-}$, 
    $\Sp$ is a derivation of
    $\foc{\Psi}{\tackon{\Theta_B}{[B^-]}}{\islvl{A^+}}$,
  \item $M$ is a derivation of $\foc{\Psi}{\tackon{\Theta}{A^+}}{U}$,
  \item $\Xi$ matches $\frameoff{\Theta}{\Delta}$, and $U'$ matches 
    $\restrictfrom{U}{\mlvl}$.
  \end{itemize}
  $\Xi$ matches $\frameoff{(\Theta {\circ} \Theta_B)}{x{:}B^-}$
  and $\tackon{(\Theta {\circ} \Theta_B)}{[B^-]}$ matches
  $\frameoff{\Theta'}{\tackon{\Theta_B}{[B^-]}}$. By the induction
  hypothesis on $\Sp$ and $M$ we have 
  $\foc{\Psi}{\tackon{(\Theta {\circ} \Theta_B)}{[B^-]}}{U'}$, and
  the required result then follows from rule ${\it focus}_L$. \smallskip
  
\item[--] $\lsubsta{\tetap{z}{N}}{M}{A^+} 
           = \tetap{z}{(\lsubsta{N}{M}{A^+})}$
\item[--] $\lsubsta{\tdownl{x}{N}}{M}{A^+} 
           = \tdownl{x}{(\lsubsta{N}{M}{A^+})}$
\item[--] $\lsubsta{\tgnabl{x}{N}}{M}{A^+} 
           = \tgnabl{x}{(\lsubsta{N}{M}{A^+})}$
\item[--] $\lsubsta{\tbangl{x}{N}}{M}{A^+} 
           = \tbangl{x}{(\lsubsta{N}{M}{A^+})}$\smallskip\\
  We must show $\foc{\Psi}{\Xi}{U'}$, where
  \begin{itemize}
  \item $\Delta$ matches $\frameoff{\Theta_B}{{!}B^-}$, 
    $N$ is a derivation of
    $\foc{\Psi}{\tackon{\Theta_B}{\ispers{x{:}B^-}}}{\islvl{A^+}}$,
  \item $M$ is a derivation of $\foc{\Psi}{\tackon{\Theta}{A^+}}{U}$,
  \item $\Xi$ matches $\frameoff{\Theta}{\Delta}$, and $U'$ matches 
    $\restrictfrom{U}{\mlvl}$.
  \end{itemize}
  We can construct a $\Theta'$ such that 
  $\tackon{\Theta'}{A^+} = (\tackon{\Theta}{A^+},x{:}\ispers{B^-})$. By
  admissible weakening, $M$ is a derivation of 
  $\foc{\Psi}{\tackon{\Theta'}{A^+}}{U}$, too.

  $\Xi$ matches $\frameoff{(\Theta {\circ} \Theta_B)}{{!}B^-}$ and
  $\tackon{(\Theta {\circ} \Theta_B)}{x{:}\ispers{B^-}}$ matches
  $\frameoff{\Theta'}{\tackon{\Theta_B}{x{:}\ispers{B^-}}}$.
  By the induction hypothesis on $N$ and $M$ we have 
  $\foc{\Psi}{\tackon{(\Theta {\circ}
      \Theta_B)}{x{:}\ispers{B^-}}}{U'}$, and the required
  result then follows from rule ${!}_L$. \smallskip

\item[--] $\lsubsta{\tfusel{N}}{M}{A^+} = \tfusel{(\lsubsta{N}{M}{A^+})}$
\item[--] $\lsubsta{\tabort}{M}{A^+} = \tabort$
\item[--] $\lsubsta{\toplusl{N_1}{N_2}}{M}{A^+} = \toplusl{(\lsubsta{N_1}{M}{A^+})}{(\lsubsta{N_2}{M}{A^+})}$
\item[--] $\lsubsta{\texistsl{a}{N}}{M}{A^+} = \texistsl{a}{(\lsubsta{N}{M}{A^+})}$\smallskip\\
  We must show $\foc{\Psi}{\Xi}{U'}$, where
  \begin{itemize}
  \item $\Delta$ matches $\frameoff{\Theta_B}{\exists a{:}\tau.B^+}$, 
     $N$ is a derivation of $\foc{\Psi,a{:}\tau}{\tackon{\Theta_B}{B^+}}{A^+}$,
  \item $M$ is a derivation of $\foc{\Psi}{\tackon{\Theta}{A^+}}{U}$,
  \item $\Xi$ matches $\frameoff{\Theta}{\Delta}$, and $U'$ matches 
     $\restrictfrom{U}{\mlvl}$.
  \end{itemize}
  $\Xi$ matches $\frameoff{(\Theta{\circ}\Theta_B)}{\exists a{:}\tau.B^+}$
  and $\tackon{(\Theta{\circ}\Theta_B)}{B^+}$ matches 
  $\frameoff{\Theta}{\tackon{\Theta_B}{B^+}}$. By variable weakening,
  $M$ is also a derivation of $\foc{\Psi, a{:}\tau}{\tackon{\Theta}{A^+}}{U}$,
  so by the induction hypothesis on $N$ and $M$ we have
  $\foc{\Psi, a{:}\tau}{\tackon{(\Theta{\circ}\Theta_B)}{B^+}}{U'}$, 
  and the required result then follows from rule $\exists_L$. \smallskip

\item[--] $\lsubsta{\phi}{M}{A^+} 
           = (\mathsf{fn}~\sigma \Rightarrow \lsubsta{\phi(\sigma)}{(\sigma{M})}{A^+})$\smallskip\\
  We must show $\foc{\Psi}{\Xi}{U'}$, where
  \begin{itemize}
  \item $\Delta$ matches $\frameoff{\Theta_B}{t \doteq s}$,
    $\phi$ is a function from substitutions $\Psi' \vdash \sigma : \Psi$
    to derivations of 
    $\foc{\Psi'}{\tackon{\sigma{\Theta_B}}{\cdot}}{\sigma{A^+}}$,
  \item $M$ is a derivation of $\foc{\Psi}{\tackon{\Theta}{A^+}}{U}$,
  \item $\Xi$ matches $\frameoff{\Theta}{\Delta}$, and $U'$ matches 
     $\restrictfrom{U}{\mlvl}$.
  \end{itemize}
  $\Xi$ matches $\frameoff{(\Theta{\circ}\Theta_B)}{t \doteq s}$, and 
  $\tackon{(\Theta{\circ}\Theta_B)}{\cdot}$ matches 
  $\frameoff{\Theta}{\tackon{\Theta_B}{\cdot}}$.
  By rule $\doteq_L$, it suffices to show that, 
  given an arbitrary substitution $\Psi' \vdash \sigma : \Psi$, 
  there is a derivation of 
  $\foc{\Psi'}{\tackon{\sigma(\Theta{\circ}\Theta_B)}{\cdot}}{\sigma{U'}}$.
  

  By applying $\sigma$ to $\phi$, we get $\phi(\sigma)$, a derivation 
  of $\foc{\Psi'}{\frameoff{\sigma\Theta_B}{\cdot}}{\sigma{A^+}}$;
  the usual interpretation of higher-order derivations is that 
  $\phi(\sigma)$ is a subderivation of $\phi$, so $\phi(\sigma)$ can be
  used to invoke the induction hypothesis.
  From variable substitution, we get
  $\sigma{M}$, a derivation of 
  $\foc{\Psi'}{\tackon{\sigma\Theta}{\sigma{A^+}}}{\sigma{U}}$.
  We need to know that 
  $\tackon{\sigma(\Theta{\circ}\Theta_B)}{\cdot}$ matches 
  $\frameoff{\sigma\Theta}{\tackon{\sigma\Theta_B}{\cdot}}$ and that
  $\sigma{U'}$ matches $\restrictfrom{\sigma{U}}{\mlvl}$, and then
  the result follows by the induction hypothesis on 
  $\phi(\sigma)$ and $\sigma{M}$. 

\end{itemize}

\paragraph{\it Substitution into spines}

\begin{itemize}
\item[--] $\lsubsta{\tupl{N}}{M}{A^+} = \tupl{(\lsubsta{N}{M}{A^+})}$
\item[--] $\lsubsta{\tlaxl{N}}{M}{A^+} = \tlaxl{\lsubsta{N}{M}{A^+}}$\smallskip\\
  We must show $\foc{\Psi}{\Xi}{U'}$, where
  \begin{itemize}
  \item $\Delta$ matches $\frameoff{\Theta_B}{{\ocircle}A^+}$, 
     $\islvl{A^+}$ matches $\restrictfrom{U_A}{\mlax}$,\\
     $N$ is a derivation of $\foc{\Psi}{\tackon{\Theta_B}{A^+}}{U_A}$
  \item $M$ is a derivation of $\foc{\Psi}{\tackon{\Theta}{A^+}}{U}$,
  \item $\Xi$ matches $\frameoff{\Theta}{\Delta}$, and $U'$ matches 
     $\restrictfrom{U}{\mlvl}$.
  \end{itemize}
  Because $\islvl{A^+}$ matches $\restrictfrom{U_A}{\mlax}$
  and $U'$ matches $\restrictfrom{U}{\mlvl}$, 
  we can conclude that $U_A = \islvl{A^+}$ and also that $U'$ matches 
  $\restrictfrom{U'}{\mlax}$. 

  $\Xi$ matches $\frameoff{(\Theta{\circ}\Theta_B)}{{\ocircle}B^+}$
  and $\tackon{(\Theta{\circ}\Theta_B)}{B^+}$ matches 
  $\frameoff{\Theta}{\tackon{\Theta_B}{B^+}}$.
  By the induction hypothesis on $N$ and $M$ we have
  $\foc{\Psi, a{:}\tau}{\tackon{(\Theta{\circ}\Theta_B)}{B^+}}{U'}$,
  and the result follows by rule ${\ocircle}_L$.  \smallskip

\item[--] $\lsubsta{\tappl{V}{\Sp}}{M}{A^+} 
           = \tappl{V}{(\lsubsta{\Sp}{M}{A^+})}$
\item[--] $\lsubsta{\tappr{V}{\Sp}}{M}{A^+} 
           = \tappr{V}{(\lsubsta{\Sp}{M}{A^+})}$\smallskip\\
  We must show $\foc{\Psi}{\Xi}{U'}$, where
  \begin{itemize}
  \item $\Delta$ matches 
     $\frameoff{\Theta_B}{\matchconj{[B_1^+ \righti B_2^-]}{\Delta_B}}$,
     $V$ is a derivation of $\foc{\Psi}{\Delta_B}{[B_1^+]}$, \\
     $\Sp$ is a derivation of 
     $\foc{\Psi}{\tackon{\Theta_B}{[B_2^-]}}{\islvl{A^+}}$,
  \item $M$ is a derivation of $\foc{\Psi}{\tackon{\Theta}{A^+}}{U}$,
  \item $\Xi$ matches $\frameoff{\Theta}{\Delta}$, and $U'$ matches 
     $\restrictfrom{U}{\mlvl}$.
  \end{itemize}
  $\Xi$ matches 
  $\frameoff{(\Theta{\circ}\Theta_B)}
    {\matchconj{[B_1^+ \righti B_2^-]}{\Delta_B}}$
  and $\tackon{(\Theta{\circ}\Theta_B)}{[B_2^-]}$ matches 
  $\frameoff{\Theta}{\tackon{\Theta_B}{[B_2^-]}}$.
  By the induction hypothesis on $\Sp$ and $M$ we have
  $\lsubsta{\Sp}{M}{A^+}$, a derivation of
  $\foc{\Psi}{\tackon{(\Theta{\circ}\Theta_B)}{[B_2^-]}}{U'}$.
  The required result follows by rule ${\righti}_L$ on $V$ and 
  $\lsubsta{\Sp}{M}{A^+}$.  \smallskip

\item[--] $\lsubsta{\tpione{\Sp}}{M}{A^+} = \tpione{(\lsubsta{\Sp}{M}{A^+})}$
\item[--] $\lsubsta{\tpitwo{\Sp}}{M}{A^+} = \tpitwo{(\lsubsta{\Sp}{M}{A^+})}$
\item[--] $\lsubsta{\tforalll{t}{\Sp}}{M}{A^+} 
           = \tforalll{t}{(\lsubsta{\Sp}{M}{A^+})}$
\end{itemize}


\subsubsection{Right commutative cuts}\label{sec:rsubst}
Rightist substitutions encompass all the {\it right commutative cuts}
from Pfenning's structural cut admissibility proof.  The constructive
content of this part is a function $\rsubsta{N}{x}{E}{A^-}$. Induction
is on the second subterm, as we crawl through $E$ looking for places
where $x$ is mentioned while focusing on the left.
The admissible rule associated with rightist substitution is
${\it rcut}$:
\[
\infer-[{\it rcut}]
{\foct{\Psi}{\frameoff{\Theta}{\restrictto{\Delta}{\mlvl}}}{\rsubsta{M}{x}{E}{A^-}}{U}}
{\foct{\Psi}{\Delta}{M}{A^-}
 &
 \foct{\Psi}{\tackon{\Theta}{x{:}\islvl{A^-}}}{E}{U}
 & 
 \stableL{\Delta}}
\]

A unique aspect of the right commutative cuts is that the previously implicit 
bookkeeping on contexts matters: when we have deal with
multiplicative rules and with left focus, 
we actually must consider that the variable $x$ that we're substituting
for can end up in only one specific branch of the proof 
(if $x$ is associated with a judgment $\istrue{A^-}$ or 
 $\iseph{A^-}$) or in both
branches of the proof (if
$x$ is associated with a judgment $x{:}\ispers{A^-}$). The computational
representation of these cases looks nondeterminstic, but it is actually
determined by the annotations and bookkeeping that is 

\paragraph{\it Substitution into values}

\begin{itemize}
\item[--] $\rsubsta{M}{x}{z}{A^-} = z$
\item[--] $\rsubsta{M}{x}{(\tdownr{N})}{A^-}
           = \tdownr{(\rsubsta{M}{x}{N}{A^-})}$
\item[--] $\rsubsta{M}{x}{(\tgnabr{N})}{A^-}
           = \tgnabr{(\rsubsta{M}{x}{N}{A^-})}$
\item[--] $\rsubsta{M}{x}{(\tbangr{N})}{A^-}
           = \tbangr{(\rsubsta{M}{x}{N}{A^-})}$\smallskip\\
  We must show $\foc{\Psi}{\Xi}{[{!}B^-]}$, where 
  \begin{itemize}
  \item $M$ is a derivation of $\foc{\Psi}{\Delta}{A^-}$, 
  \item $\tackon{\Theta}{x{:}\islvl{A^-}}$ matches 
     $\restrictto{\Delta'}{\mpers}$,
     $N$ is a derivation of $\foc{\Psi}{\Delta'}{B^-}$, 
  \item and $\Xi$ matches $\frameoff{\Theta}{\restrictto{\Delta}{\mlvl}}$.
  \end{itemize}
  Because $\tackon{\Theta}{x{:}\islvl{A^-}}$ matches
  $\restrictto{\Delta'}{\mpers}$ and $\Xi$ matches 
  $\frameoff{\Theta}{\restrictto{\Delta}{\mlvl}}$, we can conclude that
  there exists a $\Theta'$ such that
  $\Delta' = \tackon{\Theta'}{x{:}\islvl{A^-}}$ and also that
  $\Xi$ matches $\restrictto{\Xi}{\mpers}$.

  By the induction hypothesis on $M$ and $N$, 
  we have a derivation of $\foc{\Psi}{\Xi}{B^-}$, 
  and the result follows by rule ${!}_R$. 

\smallskip

\item[--] $\rsubsta{M}{x}{\toner}{A^-} = \toner$ \smallskip\\
  We must show $\foc{\Psi}{\Xi}{[\one]}$, where
  \begin{itemize}
  \item $M$ is a derivation of $\foc{\Psi}{\Delta}{A^-}$, 
  \item $\tackon{\Theta}{x{:}\islvl{A^-}}$ matches $\cdot$,
  \item and $\Xi$ matches $\frameoff{\Theta}{\restrictto{\Delta}{\mlvl}}$.
  \end{itemize}
  Because $\tackon{\Theta}{x{:}\islvl{A^-}}$ matches $\cdot$, it is
  effectively the case that we must have $\mlvl = \mpers$, and so 
  $\Xi$ matches $\cdot$ as well. The result follows by rule ${\one}_R$. 

\smallskip

\item[--] $\rsubsta{M}{x}{(\tfuser{V_1}{V_2})}{A^-} = $ \smallskip\\
    $\begin{array}{ll}
    \qquad \tfuser{(\rsubsta{M}{x}{V_1}{A^-})}{V_2}
     & \mbox{\it (if $x$ is in $V_1$'s context but not $V_2$'s)}\\
    \qquad \tfuser{V_1}{(\rsubsta{M}{x}{V_2}{A^-})}
     & \mbox{\it (if $x$ is in $V_2$'s context but not $V_1$'s)}\\
    \qquad \tfuser{(\rsubsta{M}{x}{V_1}{A^-})}{(\rsubsta{M}{x}{V_2}{A^-})}
     \qquad & \mbox{\it (if $x$ is in both $V_1$ and $V_2$'s contexts)}
    \end{array}$\smallskip\\
  We must show $\foc{\Psi}{\Xi}{[B_1^+ \fuse B_2^+]}$, where 
  \begin{itemize}
  \item $M$ is a derivation of $\foc{\Psi}{\Delta}{A^-}$, 
  \item $\tackon{\Theta}{x{:}\islvl{A^-}}$ matches $\Delta_1, \Delta_2$,
     $V_1$ is a derivation of $\foc{\Psi}{\Delta_1}{B_1^+}$, 
     $V_2$ is a derivation of $\foc{\Psi}{\Delta_2}{B_2^+}$, 
  \item and $\Xi$ matches $\frameoff{\Theta}{\restrictto{\Delta}{\mlvl}}$.
  \end{itemize}
  There are three possibilities: either $x$ is a mapping in $\Delta_1$
  or $\Delta_2$ but not both (if $\mlvl$ is $\meph$ or $\mtrue$) 
  or $x$ is a mapping in both $\Delta_1$ and $\Delta_2$ (if $\mlvl$ 
  is $\mpers$). 

  In the first case, assume without loss of generality that $x$ is a mapping
  in $\Delta_1$ but not $\Delta_2$; we can construct a 
  $\Theta_1$ and $\Delta_1'$ 
  such that $\tackon{\Theta_1}{x{:}\islvl{A^-}} = \Delta_1$,
  $\Delta_1'$ matches $\frameoff{\Theta_1}{\restrictto{\Delta_1}{\mlvl}}$,
  and $\Xi$ matches $\matchconj{\Delta_1'}{\Delta_2}$
  By the induction hypothesis on $M$ and $V_1$, we have 
  $\rsubsta{M}{x}{V_1}{A^-}$, a derivation of 
  $\foc{\Psi}{\Delta_1'}{[B_1^+]}$, and the result follows by
  rule ${\fuse}_R$ on $\rsubsta{M}{x}{V_1}{A^-}$ and $V_2$.

  The second case is similar; we construct a 
  $\Theta_1$, $\Delta_1'$, $\Theta_2$, and $\Delta_2'$ such that 
  $\tackon{\Theta_1}{x{:}\islvl{A^-}} = \Delta_1$,
  $\tackon{\Theta_2}{x{:}\islvl{A^-}} = \Delta_2$,
  $\Delta_1'$ matches $\frameoff{\Theta_1}{\restrictto{\Delta_1}{\mlvl}}$,
  $\Delta_2'$ matches $\frameoff{\Theta_1}{\restrictto{\Delta_2}{\mlvl}}$,
  and $\Xi$ matches $\matchconj{\Delta_1'}{\Delta_2'}$, which is only 
  possible because $\mlvl = \mpers$; we then invoke the induction 
  hypothesis twice. 

\smallskip

\item[--] $\rsubsta{M}{x}{(\tinl{V})}{A^-} 
           = \tinl{(\rsubsta{M}{x}{V}{A^-})}$
\item[--] $\rsubsta{M}{x}{(\tinr{V})}{A^-} 
           = \tinr{(\rsubsta{M}{x}{V}{A^-})}$
\item[--] $\rsubsta{M}{x}{(\texistsr{t}{V})}{A^-} 
           = \texistsr{t}{(\rsubsta{M}{x}{V}{A^-})}$
\item[--] $\rsubsta{M}{x}{\tunifr}{A^-} = \tunifr$
\end{itemize}

\paragraph{\it Substitution into terms}

\begin{itemize}
\item[--] $\rsubsta{M}{x}{\tfocusr{V}}{A^-} 
           = \tfocusr{\rsubsta{M}{x}{V}{A^-}}$
\item[--] $\rsubsta{M}{x}{(\tfocusl{y}{\Sp})}{A^-} 
           = \tfocusl{y}{(\rsubsta{M}{x}{\Sp}{A^-})} \qquad$ {\it ($x \# y$)}
\item[--] $\rsubsta{M}{x}{(\tfocusl{x}{\Sp})}{A^-} =$ \smallskip\\
    $\begin{array}{ll}
    \qquad (\subst{M}{(\rsubsta{M}{x}{\Sp}{A^-})})^{A^-}
     & \mbox{\it (if $x$ is in $\Sp$'s context)}\\
    \qquad (\subst{M}{\Sp})^{A^-}
     & \mbox{\it (if $x$ is not in $\Sp$'s context)}
    \end{array}$\smallskip\\
  We must show

\smallskip

\item[--] $\rsubsta{M}{x}{N}{A^-} 
           = \rsubsta{M}{x}{N}{A^-}$
\end{itemize}

\paragraph{\it Substitution into spines}

\begin{itemize}
\item[--] $\rsubsta{M}{x}{N}{A^-} 
           = \rsubsta{M}{x}{N}{A^-}$
\end{itemize}

\section{Identity expansion}
\label{sec:ord-identity}

\section{Correctness of focusing}
\label{sec:ord-correctness}

\subsection{Erasure}

\subsection{De-focalization}

\subsection{Unfocused admissibility}

\subsection{Focalization}


