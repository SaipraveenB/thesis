\chapter{Substructural logic}

Linear logic is the most famous of the {\it substructural logics}.
Persistent logic admits the three so-called {\it structural rules} of
weakening (premises need not be used), contraction (premeses may be
used multiple times) and exchange (the ordering of premises are 
irrelevant). Substructural logics, then, are logics that do not admit
these structural rules -- linear logic has only exchange, 
{\it affine} logic (which is frequently conflated with linear logic
by programming language designers) has exchange and weakening, and
{\it ordered} logic, first investigated as a proof theory by Lambek
\cite{lambek58mathematics}, lacks all three. 

Calling logics like linear, affine, and ordered logic
\underline{sub}structural relative to persistent logic (which is
structural) is arguably unfair to the substructural logics. Girard's
linear logic can express persistent provability using the exponential
connective ${!}A$, and this idea is generally applicable in
substructural logics -- for instance, it was applied by Polakow and
Pfenning to Lambek's ordered logic
\cite{polakow99natural}.\robnote{Check citation.} 

This section defines a first-order ordered logic that includes a lax
modality; a fragment of this system will form the basis of our logical
framework in the next section. The full focusing system for ordered
lax logic is presented in this chapter in keeping with Andreoli's
insistance that we avoid the confusion that results from conflating
``defining a foundational paradigm or a [logic] programming language
(two objectives that should clearly be kept separate)''
\cite{andreoli01focussing}. This concern is applicable for logical
frameworks as well as for logic programming langauges. This is, in a
sense, obviously true -- as Twelf, Lollimon, and Celf show, logical
frameworks can frequently be given straightforward life as logic
programming languages
\cite{pfenning99system,lopez05monadic,schacknielsen08celf}. In this
case, I believe that a full exposition of this logic is useful because
of the possibility that many other interesting and tractable logic
programming languages can be defined as fragments of this logic.

\section{Ordered linear lax logic}

Ordered linear logic was the subject of Polakow's thesis
\cite{polakow01ordered}, and the adaptation of Fairtlough and
Mendler's lax logic \cite{fairtlough95propositional} (as reconstructed
by Pfenning and Davies \cite{pfenning01judgmental}) to linear logic is
the basis of the CLF logical framework
\cite{watkins02concurrent}. Putting the pieces together 
is a straightforward proof-theoretic exercise. There are three contexts
relevant to the propositional presentation of ordered linear lax logic.
The persistent context $\Gamma$ and the linear context $\Delta$
are multisets as before (so we think of $\Delta_1, \Delta_2$ as being
equal to $\Delta_2, \Delta_1$, for instance). The ordered context
$\Omega$ is a sequence of propositions (as in Gentzen's original
presentation of sequent calculi) and not a multiset, so the two ordered
contexts $\Omega_1, \Omega_2$ and $\Omega_2, \Omega_1$ are, in general,
not the same.

\begin{figure}
\small
{\it Atomic propositions}
\[
\infer[{\it init}^+]
{\oiseq{\Gamma}{p}}
{}
\]

\medskip
{\it Modalities}
\[
\infer[{\gnab}_R]
{\orseq{\Gamma}{\Delta}{\cdot}{{\gnab}A}}
{\otseq{\Gamma}{\Delta}{\cdot}{A}}
\quad
\infer[{\gnab}_L]
{\olseq{\Gamma}{\Delta}{\Omega_L}{{\gnab}A}{\Omega_R}}
{\opseq{\Gamma}{\Delta, A}{\Omega_L}{\Omega_R}}
\]
\vspace{-5pt}
\[
\infer[{!}_R]
{\orseq{\Gamma}{\cdot}{\cdot}{{!}A}}
{\otseq{\Gamma}{\cdot}{\cdot}{A}}
\quad
\infer[{!}_L]
{\olseq{\Gamma}{\Delta}{\Omega_L}{{!}A}{\Omega_R}}
{\opseq{\Gamma,A}{\Delta}{\Omega_L}{\Omega_R}}
\]
\vspace{-5pt}
\[
\infer[{\ocircle}_R]
{\orseq{\Gamma}{\Delta}{\Omega}{{\ocircle}A}}
{\oseq{\Gamma}{\Delta}{\Omega}{\islax{A}}}
\quad
\infer[{\ocircle}_L]
{\oseq{\Gamma}{\Delta}{\Omega_L /{\ocircle}A/ \Omega_R}{\islax{C}}}
{\oseq{\Gamma}{\Delta}{\Omega_L, A, \Omega_R}{\islax{C}}}
\]

\medskip
{\it Multiplicative connectives}
\[
\infer[{\one}_R]
{\orseq{\Gamma}{\cdot}{\cdot}{\one}}
{}
\quad
\infer[{\one}_L]
{\olseq{\Gamma}{\Delta}{\Omega_L}{\one}{\Omega_R}}
{\opseq{\Gamma}{\Delta}{\Omega_L}{\Omega_R}}
\]
\[
\infer[{\fuse}_R]
{\orseq{\Gamma}{\Delta_1,\Delta_2}{\Omega_L,\Omega_R}{A \fuse B}}
{\otseq{\Gamma}{\Delta}{\Omega_L}{A}
 &
 \otseq{\Gamma}{\Delta}{\Omega_R}{B}}
\quad
\infer[{\fuse}_L]
{\olseq{\Gamma}{\Delta}{\Omega_L}{A \fuse B}{\Omega_R}}
{\opseq{\Gamma}{\Delta}{\Omega_L}{A,B,\Omega_R}}
\]
\vspace{-5pt}
\[
\infer[{\lefti}_R]
{\orseq{\Gamma}{\Delta}{\Omega}{A \lefti B}}
{\otseq{\Gamma}{\Delta}{A, \Omega}{B}}
\quad
\infer[{\lefti}_L]
{\olseq{\Gamma}{\Delta_A, \Delta}{\Omega_L, \Omega_A}{A \lefti B}{\Omega_R}}
{\otseq{\Gamma}{\Delta_A}{\Omega_A}{A}
 &
 \opseq{\Gamma}{\Delta}{\Omega_L}{B,\Omega_R}}
\]
\vspace{-5pt}
\[
\infer[{\righti}_R]
{\orseq{\Gamma}{\Delta}{\Omega}{A \righti B}}
{\otseq{\Gamma}{\Delta}{\Omega, A}{B}}
\quad
\infer[{\righti}_L]
{\olseq{\Gamma}{\Delta_A, \Delta}{\Omega_L}{A \righti B}{\Omega_A, \Omega_R}}
{\otseq{\Gamma}{\Delta_A}{\Omega_A}{A}
 &
 \opseq{\Gamma}{\Delta}{\Omega_L}{B, \Omega_R}}
\]

\medskip
{\it Additive connectives}
\[
\infer[{\zero}_L]
{\olseq{\Gamma}{\Delta}{\Omega_L}{\zero}{\Omega_R}}
{}
\quad
\infer[{\oplus}_{R1}]
{\orseq{\Gamma}{\Delta}{\Omega}{A \oplus B}}
{\otseq{\Gamma}{\Delta}{\Omega}{A}}
\quad
\infer[{\oplus}_{R2}]
{\orseq{\Gamma}{\Delta}{\Omega}{A \oplus B}}
{\otseq{\Gamma}{\Delta}{\Omega}{B}}
\]
\vspace{-5pt}
\[
\infer[{\oplus}_{L}]
{\olseq{\Gamma}{\Delta}{\Omega_L}{A \oplus B}{\Omega_R}}
{\opseq{\Gamma}{\Delta}{\Omega_L}{A,\Omega_R}
 &
 \opseq{\Gamma}{\Delta}{\Omega_L}{B,\Omega_R}}
\]
\vspace{-5pt}
\[
\infer[{\top}_R]
{\orseq{\Gamma}{\Delta}{\Omega}{\top}}
{}
\quad
\infer[{\with}_{L1}]
{\olseq{\Gamma}{\Delta}{\Omega_L}{A \with B}{\Omega_R}}
{\opseq{\Gamma}{\Delta}{\Omega_L}{A, \Omega_R}}
\quad
\infer[{\with}_{L2}]
{\olseq{\Gamma}{\Delta}{\Omega_L}{A \with B}{\Omega_R}}
{\opseq{\Gamma}{\Delta}{\Omega_L}{B, \Omega_R}}
\]
\vspace{-5pt}
\[
\infer[{\with}_R]
{\orseq{\Gamma}{\Delta}{\Omega}{A \with B}}
{\otseq{\Gamma}{\Delta}{\Omega}{A}
 &
 \otseq{\Gamma}{\Delta}{\Omega}{B}}
\]


\caption{Propositional ordered linear lax logic.}
\label{fig:ordered-prop}
\end{figure}


There are two judgments in ordered linear lax logic. The primary
judgment is $\oseq{\Gamma}{\Delta}{\Omega}{\istrue{A}}$, which says
that $A$ is an (ephemeral, ordered) resource derivable from the
persistent resources in $\Gamma$, the ephemeral resources in $\Delta$,
and the ephemeral, ordered resources in $\Omega$. There is also a
second judgment, $\oseq{\Gamma}{\Delta}{\Omega}{\islax{A}}$. The
judgment $\islax{A}$ is usually interpreted as truth under an
unspecified constraint; one defining characteristic of the lax
modality is that it if $\istrue{A}$ is derivable with 
some resources then $\islax{A}$ is derivable with the same resources.

Compare this to the relationship between persistent and lax truth in
linear logic, where the defining characteristic is that a persistent
resource (associated with judgments of the form 
$\ispers{A}$ in $\Gamma$) can
always satisfy the need for an ephemeral resource (associated with
judgments of the form $\iseph{A}$ in $\Delta$). In the previous
chapter, we first encoded this relationship as an explicit rule
${\it copy}$ rule:
\[
\infer[{\it copy}]
{\seq{\Gamma, \ispers{A}}{\Delta}{\iseph{C}}}
{\seq{\Gamma, \ispers{A}}{\Delta, \iseph{A}}{\iseph{C}}}
\]
In Section~\ref{sec:linnote}, based on a discussion of synthetic
connectives under the atom optimization, we considered a revision
to this notaiton 
\[
\infer[{\oplus}_L]
{\altseq{\Gamma}{\Delta/A \otimes B}{\iseph{C}}}
{\altseq{\Gamma}{\Delta, \iseph{A}}{\iseph{C}}
 &
 \altseq{\Gamma}{\Delta, \iseph{B}}{\iseph{C}}}
\]

Lax truth can be considered along the same lines, accoutning for the
fact that we're now using the much briefer $\istrue{A}$ instead of
$\iseph{A}$ to capture the judgment assocated with ordered, ephemeral
resources.  To follow existing judgmental presentations of lax logic,
we would include a distinct rule ${\it lax}$ that derives lax truth
from regular truth.
\[
\infer[{\it lax}]
{\Gamma; \Delta; \Omega \longrightarrow \islax{A}}
{\Gamma; \Delta; \Omega \longrightarrow \istrue{A}}
\]
The alternative is to rewrite all the right rules with a notation
$\orseq{\Gamma}{\Delta}{\Omega}{A}$ that matches both 
judgments of the form $\otseq{\Gamma}{\Delta}{\Omega}{A}$
and judgments of the form $\oseq{\Gamma}{\Delta}{\Omega}{\islax{A}}$.
\[
\infer[{\oplus}_{R1}]
{\orseq{\Gamma}{\Delta}{\Omega}{A \oplus B}}
{\otseq{\Gamma}{\Delta}{\Omega}{A}}
\qquad
\infer[{\oplus}_{R2}]
{\orseq{\Gamma}{\Delta}{\Omega}{A \oplus B}}
{\otseq{\Gamma}{\Delta}{\Omega}{B}}
\]

The related notation on the left-hand side is 
$\olseq{\Gamma}{\Delta}{\Omega_L}{A}{\Omega_R}$, which matches
the $\oseq{\Gamma}{\Delta'}{\Omega'}{U}$ if
\begin{itemize}
\item $\Omega' = \Omega_L, A, \Omega_R$ and $\Delta' = \Delta$;
\item $\Omega' = \Omega_L, \Omega_R$ and $\Delta' = \Delta, A$;
\item $\Omega' = \Omega_L, \Omega_R$, $\Delta' = \Delta$, and $A \in \Gamma$.
\end{itemize}
This style of presentation, expressed in Figure~\ref{fig:ordered-prop},
leaves both the ${\it copy}$ and ${\it place}$ rules of Polakow's 
presentation as admissible.

\subsection{First-order logic}

The presentation in Figure~\ref{fig:ordered-prop} is 

\section{Substructural contexts}

We will treat {\it substructural contexts} $\Delta$ are maps from {\it
  variables} to {\it judgments}. There are two kinds of variables: 
those that map {\it extent}.

${\mbox{\it [}}$


Two of the fundamental propeties of $\Delta$ are 
$x{:}\langle A^+ \rangle_l \sqsubseteq \Delta$. In linear logic, this
is the statement that $x{:}A \in \Gamma$, whereas in 

\[\small
\begin{array}{|c|c|c|c|}
%
\begin{array}{c}
\makebox[1.8in]{\it Dual intuitionstic linear logic}\medskip
\\
\infer
{\Gamma; p^+ \vdash p^+ \mathstrut}
{\mathstrut}
\end{array}
%
&
%
\begin{array}{c}
\makebox[1.8in]{\it Pfenning-Davies S4}\medskip
\\
\infer
{\Delta; \Gamma, p^+ \mathstrut \vdash p^+}
{\mathstrut}
\end{array}
%
&
%
\begin{array}{c}
\makebox[1.8in]{\it Unified}\medskip
\\
\infer
{\Delta \vdash [ p^+ ]_l \mathstrut}
{x{:}\langle p^+ \rangle_l \subseteq \Delta \mathstrut}
\end{array}
%
\end{array}
\]

\[\small
\begin{array}{|c|c|c|c|}
%
\begin{array}{c}
\makebox[1.8in]{\it Dual intuitionstic linear logic}\medskip
\\
\infer
{\Gamma; \Delta, A \with B \vdash C \mathstrut}
{\Gamma; \Delta, A \vdash C \mathstrut}
\end{array}
%
&
%
\begin{array}{c}
\makebox[1.8in]{\it Pfenning-Davies S4}\medskip
\\
\infer
{\Delta; \Gamma, A \wedge^- B \mathstrut \vdash C}
{\Delta; \Gamma, A \wedge^- B, A \vdash C \mathstrut}
\end{array}
%
&
%
\begin{array}{c}
\makebox[1.8in]{\it Unified}\medskip
\\
\infer
{\Theta\{ x{:} A \with B \} \vdash U \mathstrut}
{\Theta\{ x_1{:} A \} \vdash U \mathstrut}
\end{array}
%
\end{array}
\]


\[\small
\begin{array}{|c|c|c|c|}
%
\begin{array}{c}
\makebox[1.8in]{\it Dual intuitionstic linear logic}\medskip
\\
\infer
{\Gamma; \Delta_1, \Delta_2 \vdash A \otimes B \mathstrut}
{\Gamma; \Delta_1 \vdash A & \Gamma; \Delta_2 \vdash B \mathstrut}
\end{array}
%
&
%
\begin{array}{c}
\makebox[1.8in]{\it Pfenning-Davies S4}\medskip
\\
\infer
{\Delta; \Gamma \vdash A \wedge^+ B \mathstrut}
{\Delta; \Gamma \vdash A & \Delta; \Gamma \vdash B \mathstrut}
\end{array}
%
&
%
\begin{array}{c}
\makebox[1.8in]{\it Unified}\medskip
\\
\infer
{\Delta_1 \bowtie \Delta_2 \vdash A \otimes B \mathstrut}
{\Delta \vdash A & \Delta_2 \vdash B \mathstrut}
\end{array}
%
\end{array}
\]



\section{Explicit shifts in focusing}

\section{Polarization and erasure}

\section{Cut admissibility}

\section{Identity expansion}

\section{Unfocused admissibility}

\section{Soundness and completeness}


