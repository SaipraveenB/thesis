\chapter{Conclusion}
\label{chapter-conclusion}

This document has endeavored to support the following thesis:
\smallskip
\begin{quote} {\bf Thesis Statement:} {\it Logical frameworks based on
    a rewriting interpretation of substructural logics are suitable
    for modular specification of programming languages and formal
    reasoning about their properties.}
\end{quote}
\bigskip


In the service of this thesis, we first developed a logical framework
of substructural logical specifications (\sls) based on a rewriting
interpretation of ordered linear lax logic (\ollll). Part I of the
thesis discussed the design of this logical framework, and in the
process firmly established the elegant connection between two sets of
techniques:

\smallskip
\begin{enumerate}
\item Canonical forms and hereditary
substitution in a logical framework, on one hand, and 
\item Focused derivations and cut
admissibility in logic, on the other.
\end{enumerate}

\smallskip
\noindent 
The broad outlines of this connection have been known for a decade,
but this dissertation gives the first account of the connection that
generalizes to all logical connectives. This connection allowed the
\sls~framework to be presented as a syntactic refinement of focused
ordered linear lax logic; the steps and traces of \sls, which provide
its rewriting interpretation, are justified as partial proofs in
focused ordered linear lax logic. \sls~does move beyond the connection
with focused logic due to the introduction of concurrent equality,
which allows logically independent steps in a trace to be reordered;
we conjecture that the resulting equivalence relation imposed on our
logical framework is analogous to the one given by multifocusing in
logic, but a full exposition of this connection is left for future
work.

The \sls~framework acts as a bridge between the world of logical
frameworks, where deductive derivations are the principal objects of
study, and the world of rewriting logic, where rewriting sequences
that are similar to \sls~traces are the principal objects of
study. Part II of this thesis discusses a number of ways of describing
operational semantics specifications in \sls, using ordered resources
to encode control structures, using mobile/linear resources to encode
mutable state and concurrent communication, and using persistent
resources to represent memoization and binding. Different styles of
specification were connected to each other through systematic
transformations on \sls~specifications that we proved to be generally
sound, a methodology named the {\it logical correspondence}, following
Danvy et al.'s functional correspondence. Most of the systematic
transformations discussed in Chapter~\ref{chapter-absmachine} and
Chapter~\ref{chapter-destinations} -- operationalization,
defunctionalization, and destination-adding -- were implemented in the
\sls~prototype implementation. Utilizing this implementation, we show
in Appendix~\ref{appendix-hybrid} that it is possible to fuse together
a single coherent \sls~specification of a MiniML language with
concurrency, state, and communication using various different styles
of specification, including natural semantics where appropriate.

% \sls~still hews more
% closely to the tradition of logical frameworks, but future work will
% hopefully reduce the remaining distance between operational semantics
% specifications in \sls~and rewriting logic-based approaches to
% operational semantics specifications.

This thesis also discussed two different methodologies for formally
reasoning about properties of operational semantics specifications in
\sls. The program analysis methodology considered in
Chapter~\ref{chapter-approx} allows us to derive effectively
executable abstractions of an operational semantics directly from an
operational semantics specification in \sls. The methodology of
progress, preservation, and type safety considered in
Chapter~\ref{chapter-gen} and Chapter~\ref{chapter-safety} was
presented as a natural extension of traditional ``safety = progress +
preservation'' reasoning. In a sense, this thesis has pushed our
ability to reason {\it formally} about properties of
\sls~specifications (and substructural operational semantics
specifications in particular) some distance beyond our ability to {\it
  informally} reason about these specifications. An important
direction for future work will be to move beyond the
misleadingly-sequential language of \sls~traces and develop a more
user-friendly language for writing, talking, and thinking about traces
in \sls, especially generative traces.
