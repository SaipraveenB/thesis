\chapter{Generative specifications as invariants}
\label{chapter-genspec}

The relevant building blocks in our $\lambda$-calculus encoding are
just LF variables of type $\lf{a}{:}{\sf exp}$. It is sometimes
desirable to consider closed expressions, 



but in other examples they
involve multiple context elements that must appear in tandem. In
proving adequacy for typing derivations, for instance, we must
frequently prove that a variable $\lf{a}{:}{\sf exp}$ always appears
in tandem with a variable $\lf{d}{:}({\sf of}\,\lf{a}\,\lf{tp})$ that
associates the variable with some closed LF term $\lf{tp}$ of type
${\sf typ}$ that encodes the type of that variable.

These {\it regular worlds} are not just important for proving adequacy
theorems. They are a critical part of reasoning about LF terms: when
we quantify over the set of LF terms of a particular type, it is
always necessary to specify what world we are working with. In the
case of $\lambda$-calculus terms encoded in the type ${\sf exp}$, we
sometimes want to consider open expressions in the world of contexts
that take the form $\lf{a_1}{:}{\sf exp},\ldots,\lf{a_n}{:}{\sf exp}$
(for instance, when we are describing typing rules for open terms or
defining $\beta$-reduction under binders). We also sometimes want to
consider closed terms in the empty context (for instance, when we are
describing call-by-value evaluation of closed expressions).  This
choice affects the structure of case analysis: when reasoning
about closed expressions, we do not need to provide a base case for
when the expression is a variable, but when reasoning about open
expressions, we obviously do.

The {\it closed world assumption} -- considering terms in the absence
of any LF variables -- is the simplest one. 


The world affects the reasoning we have to do when we perform case
analysis: when we reason about terms of type ${\sf exp}$ defined in
the world that allows variables of type ${\sf exp}$, we have to
consider an extra case (the term is a variable $\lf{a_i}$ from the
context) that is not possible in the closed world. 

\section{Types}

\section{State}

\subsection{Store typing}


\subsection{Environment semantics}

\section{Destinations}
\label{sec:gen-destinations}


\section{Persistent continuations}
\label{sec:gen-letcc}

\section{Counting invariants}

\subsection{Parity}

\subsection{}

\subsection{Pointer inequality}
\label{sec:pointer-inequality}
