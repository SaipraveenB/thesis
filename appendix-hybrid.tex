\chapter{A hybrid specification of Mini-ML}
\label{appendix-hybrid}

In this section, we present the specification that was illustrated in
Figure~\ref{fig:modularcompose} as a full \sls~specification. This
specification is a hybrid or chimera: it has individual features that
are presented using big-step natural semantics, nested ordered abstract
machine semantics, flat ordered abstract machine semantics, and
destination-passing semantics.

Illustrating the logical correspondence methodology that we introduced
in Chapter~\ref{chapter-correspondence} and expounded upon in
Chapters~\ref{chapter-absmachine}~and~\ref{chapter-destinations}, all
these specifications can transformed into a common flat
destination-passing semantics. With the exception of Concurrent ML
primitives, which were only alluded to in
Section~\ref{sec:dest-synch}, all the pieces of this specification (or
very similar variants) have been presented elsewhere in the
dissertation. The specification in this section is careful to present
the entire \sls~specification, as opposed to other examples in which
the relevant LF declarations were almost always omitted.

The lowest common denominator of destination-passing semantics can be
represented in CLF, and the \sls~implementation is able to output CLF
code readable in Schack-Nielsen's Celf implementation
\cite{schacknielsen08celf}. The implemented logic programming engine
of Celf is therefore able to {\it execute} Mini-ML programs encoded as
terms of type ${\sf exp}$ in our hybrid specification.

\section{Pure Mini-ML}

There are various toy languages calling themselves ``Mini-ML'' in the
literature. All Mini-MLs reflect some of the flavor of functional
programming while avoiding features such as complex pattern-matching
and datatype declarations that make the core language of Standard ML
\cite{milner97definition} a bit more complicated.  Of course, Mini-MLs
universally avoid the sophisticated ML module language as well.

Like the PCF language \cite{plotkin77lcf}, Mini-ML variants usually
have at least a fixed-point operator, unary natural numbers, and
functions. We add Boolean and pair values to this mix, as well as the
arbitrary choice operator $\interp{e_1 \arb e_2} = \lf{{\sf
  arb}\,\interp{e_1}\,\interp{e_2}}$ from
Section~\ref{sec:choicefail}. The specification in
Section~\ref{sec:compose-pure} is an encoding of the natural semantics
judgment $\interp{e \Downarrow v} = {\sf ev}\,\interp{e}\,\interp{v}$
presented throughout Chapter~\ref{chapter-absmachine}. The language is
pure -- the only effect is nontermination -- so we can fully specify
the language as a big-step operational semantics.

\subsection{Syntax}

\fvset{fontsize=\footnotesize}
\VerbatimInput{sls/compose/pure-exp.sls}

\subsection{Natural semantics}
\label{sec:compose-pure}

\fvset{fontsize=\footnotesize}
\VerbatimInput{sls/compose/pure-natsem.sls}

\section{State}

The strength of an ordered abstract machine semantics specification is
its ability to handle modular addition of {\it stateful}
features. While Section~\ref{sec:richer-ordered-abstract} discussed
the modular extension of {\it flat} ordered abstract machines, nested
ordered abstract machines are also perfectly capable of handling
stateful features such as mutable storage
(Section~\ref{sec:mutable-storage}) and call-by-need recursive
suspensions (Section~\ref{sec:call-by-need}).

\subsection{Syntax}

\fvset{fontsize=\footnotesize}
\VerbatimInput{sls/compose/imp-exp.sls}

\subsection{Nested ordered abstract machine semantics}
\label{sec:compose-imp}

In Section~\ref{sec:mutable-storage}, we discussed mutable storage as
a flat ordered abstract machine (Figure~\ref{fig:ssos-mutable}), but
it is equally straightforward to describe a nested ordered abstract
machine for mutable storage.

\bigskip
\fvset{fontsize=\footnotesize}
\VerbatimInput{sls/compose/imp-ordmachine.sls}

\subsection{Flat ordered abstract machine semantics}
\label{sec:compose-susp}

In Section~\ref{sec:call-by-need}, we gave both a semantics for
call-by-need recursive suspensions, both as a flat ordered abstract
machine (Figure~\ref{fig:ssos-cbneed}) and a nested ordered abstract
machine (Figure~\ref{fig:ssos-cbneed-refun}). However, the nested
ordered abstract machine from Figure~\ref{fig:ssos-cbneed-refun} uses
the with connective $A^- \with B^-$, and our implementation of
defunctionalization transformation doesn't handle this connective.
Therefore, we repeat the flat ordered abstract machine from
Figure~\ref{fig:ssos-cbneed}.  Note, however, that there is no
technical reason why $A^- \with B^-$ should be difficult to handle;
any actual difficulty is mostly in terms of making sure uncurrying
(Section~\ref{sec:defunc-uncurry}) does something sensible.

\bigskip
\fvset{fontsize=\footnotesize}
\VerbatimInput{sls/compose/susp-ordmachine.sls}

\section{Failure}

The reason we introduced frames in Section~\ref{sec:defunc-type} was
to allow the semantics of recoverable failure to talk generically
about all continuations. In Section~\ref{sec:compose-control}, we
generalize the semantics from Section~\ref{sec:failure}~by having exceptions
carry a value.

\subsection{Syntax}

\fvset{fontsize=\footnotesize}
\VerbatimInput{sls/compose/control-exp.sls}

\subsection{Flat ordered abstract machine semantics}
\label{sec:compose-control}

\fvset{fontsize=\footnotesize}
\VerbatimInput{sls/compose/control-ordmachine.sls}

\section{Parallelism}

While ordered abstract machines can represent parallel evaluation, and
the operationalization transformation can expose it, parallel ordered
abstract machines and the destination-adding transformation do not
interact a helpful way. Therefore, for our hybrid specification,
we will describe parallel evaluation at the destination-passing 
level, as in Section~\ref{sec:modular-parallelism}.

\subsection{Destination-passing semantics}

Instead of the parallel pairs shown in
Figure~\ref{fig:ssos-minml-core} and Figure~\ref{fig:dest-pair}, we
will use a parallel let construct $\interp{{\sf letpar}\, (x_1,x_2) =
  (e_1,e_2) \,{\sf in}\,e} = \lf{{\sf
    letpar}\,\interp{e_1}\,\interp{e_2}\,\lambda x_1.\,\lambda
  x_2.\,\interp{e}}$.

\bigskip
\fvset{fontsize=\footnotesize}
\VerbatimInput{sls/compose/par-dest1.sls}

\subsection{Integration of parallelism and exceptions}

We have discussed two semantics for parallel evaluation. The first
semantics, in Section~\ref{sec:failure}, only raised an error if both
parallel branches terminated and one raised an error. The second
semantics, in Section~\ref{sec:modular-parallelism}, raised an error
if {\it either} branch raised an error, and then allowed the other
branch to return a value.

We will demonstrate a third option here, the sequential exception
semantics used by Manticore~\cite{fluet08scheduling}. An error raised
by the second scrutinee $\obj{e_2}$ of $\obj{\sf letpar}$ will only be
passed up the stack if the first scrutinee $\obj{e_1}$ returns a
value. We also represent Manticore's {\it cancellation} -- if the
first branch of a parallel evaluation raises an exception, then rather
than passively waiting for the second branch to terminate we
proactively walk up its stack attempting to cancel the computation.

\bigskip
\fvset{fontsize=\footnotesize}
\VerbatimInput{sls/compose/par-dest2.sls}


\section{Concurrency}

Concurrent ML is an excellent example of the power of
destination-passing specifications. The Concurrent ML primitives allow
a computation to develop a rich interaction structure that does not
mesh well with the use of ordered logic, but the destination-passing
style allows for a clean specification that is fundamentally like the
one used for simple synchronization in
Section~\ref{sec:dest-synch}. This account directly follows Cervesato
et al.'s account \cite{cervesato02concurrent}, similarly neglecting
negative acknowledgements.

\subsection{Syntax}

\fvset{fontsize=\footnotesize}
\VerbatimInput{sls/compose/concur-exp.sls}

\subsection{Natural semantics}

Many of the pieces of Concurrent ML do not interact with concurrency
directly; instead, they build channels and event values that drive
synchronization. In our hybrid specification methodology, we can give
these pure parts of the Concurrent ML specification a big-step
natural semantics specification.

\bigskip
\fvset{fontsize=\footnotesize}
\VerbatimInput{sls/compose/concur-natsem.sls}

\subsection{Destination-passing semantics}

The destination-passing semantics of Concurrent ML has three main
parts.  The first part, a ${\sf spawn}$ primitive, creates a new
disconnected thread of computation -- the same kind of disconnected
thread that we used for the interaction of parallelism and failure in
Section~\ref{sec:modular-parallelism}. The $\obj{\sf newch}$ primitive 
creates a new channel for communication.

\bigskip
\fvset{fontsize=\footnotesize}
\VerbatimInput{sls/compose/concur-dest1.sls}
\bigskip

\noindent
The critical feature of Concurrent ML is synchronization, which is
much more complex than the simple synchronization described in
Section~\ref{sec:dest-synch}, and has something of the flavor of the
labels described in that section. An action can include many
alternatives, but if a send and a receive can simultaneously take
place along a single channel, then the ${\sf synch/communicate}$ rule
can enable both of the waiting ${\sf synch}\,\lf{v}\,\lf{d}$ 
propositions to proceed
evaluating as ${\sf eval}$ propositions.

Here as in Cervesato et al.'s specification, events are atomically
paired up using the backward-chaining ${\sf action}$ rules, which are
not transformed: the intent is for the ${\sf action}$ predicate to act 
like a backtracking, backward-chaining logic programs in the course
of evaluation.

\bigskip
\fvset{fontsize=\footnotesize}
\VerbatimInput{sls/compose/concur-dest2.sls}


\section{Composing the semantics}

Within Standard ML, we can read the various specifications described
in this appendix and use directives to successively operationalize,
defunctionalize, add destinations, and output CLF that is readable by
the Celf implementation.

\bigskip
\fvset{fontsize=\footnotesize}
\VerbatimInput{sls/testA.sml}
