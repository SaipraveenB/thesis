\chapter{Introduction}
\label{chapter-introduction}

\section{Substructural logical specifications}



\subsection{Substructural operational semantics}
\label{sec:intro-ssos}

Abstract machine rules

\subsection{Modular and non-modular specification}
\label{sec:modularnonmodular}

(This can mostly come out of the thesis proposal, but use 
natural semantics instead of SOS.)

Natural semantics

Not modular (state), underspecified (parallel or not).


Add state in a modular way.

\subsection{Process states}

The general idea of representing the
intermediate states of a computation as contexts in substructural
logic dates back to Miller \cite{miller92pi} and his Ph.D. student
Chirimar \cite{chirimar95proof}, who encoded the intermediate states
of a $\pi$-calculus and of a low-level RISC machine (respectively) as
contexts in focused classical linear logic.

\section{Designing substructural logical frameworks}

\section{Specification styles}

\section{Invariants in substructural logic}
