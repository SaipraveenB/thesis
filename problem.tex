\documentclass[12pt]{article}
\usepackage{fancyvrb}
\usepackage{times}
\usepackage{dashrule}
\usepackage{proof-dashed}
\usepackage[show]{ed}
\usepackage{fullpage}
\usepackage{graphicx}
\usepackage{xcolor}
\usepackage{tikz}
\usepackage{rotating}
\usetikzlibrary{arrows,decorations.pathmorphing,backgrounds,fit}
\usepackage{amsthm}
\usepackage{amsmath}
\usepackage{latexsym}
\usepackage{amssymb}            % for \multimap (-o)
\usepackage{stmaryrd}           % for \binampersand (&), \bindnasrepma (\paar)
\usepackage{wasysym}            % for \ocircle
\usepackage[numbers,sort]{natbib}
\usepackage[backref,pageanchor=true,plainpages=false, pdfpagelabels, bookmarks,bookmarksnumbered,
%pdfborder=0 0 0,  %removes outlines around hyper links in online display
]{hyperref}
\usepackage{subfigure}

% Approximately 1" margins, more space on binding side
%\usepackage[letterpaper,twoside,vscale=.8,hscale=.75,nomarginpar]{geometry}
%for general printing (not binding)
\usepackage[letterpaper,twoside,vscale=.8,hscale=.75,nomarginpar,hmarginratio=1:1]{geometry}

% Provides a draft mark at the top of the document. 

\definecolor{grayout}{rgb}{.8,.8,.8}
\definecolor{plesantgreen}{rgb}{.1,.5,0}
\newcommand{\gr}[1]{\textcolor{plesantgreen}{\ensuremath{{#1}}}}

\definecolor{lfblue}{rgb}{.5,.2,0}
\newcommand{\lf}[1]{\textcolor{lfblue}{\ensuremath{{#1}}}}



\definecolor{justblack}{rgb}{0,0,0}
\newcommand{\no}[1]{\textcolor{justblack}{\ensuremath{{#1}}}}

\definecolor{objgreen}{rgb}{0,.3,.5}
\newcommand{\obj}[1]{\textcolor{objgreen}{\ensuremath{{#1}}}}

\hypersetup{colorlinks=true,citecolor=blue,urlcolor=blue,linkcolor=black}

\newcommand{\robnote}[1]{\footnote{{\bf NOTE TO SELF:} ~~ {#1}}}
\newcommand{\futurework}[1]{}

\newcommand{\ollll}{OL$_3$}
\newcommand{\sls}{SLS}
\newcommand{\simplearrow}{\rightarrow}
\newcommand{\rowan}{${\lambda}^{\ocircle}$}

\newcommand{\mb}{Coq}

\renewcommand{\labelitemi}{$\ast$}



% Theorems
\newtheorem{theorem}{Theorem}[chapter]
\newtheorem*{lemma}{Lemma}
\newtheorem{proposition}[theorem]{Proposition}
\newtheorem{definition}[theorem]{Definition}

% symbols of linear logic
\newcommand{\lolli}{\multimap}
\newcommand{\tensor}{\otimes}
\newcommand{\with}{\mathbin{\binampersand}}
\newcommand{\paar}{\mathbin{\bindnasrepma}}
\newcommand{\one}{\mathbf{1}}
\newcommand{\zero}{\mathbf{0}}
\newcommand{\bang}{{!}}
\newcommand{\pbang}{\mbox{\hspace{2pt}$\mathbb !$\hspace{-4.7pt}$_\curlyvee$\hspace{-1pt}}}
\newcommand{\whynot}{{?}}
\newcommand{\bilolli}{\mathrel{\raisebox{1pt}{\ensuremath{\scriptstyle\circ}}{\lolli}}}
% \oplus, \top, \bot
\newcommand{\deupdown}{\mbox{${\uparrow}{\downarrow}\hspace{-11.4pt}\diagup$}}
\newcommand{\dedownup}{\mbox{${\downarrow}{\uparrow}\hspace{-11.4pt}\diagdown$}}

\newcommand{\restrictto}[2]{\ensuremath{{#1}{\upharpoonright}_{#2}}}
\newcommand{\restrictsig}[2]{\ensuremath{{#1}{\mbox{\lightning}}_{#2}}}
\newcommand{\restrictfrom}[2]{\ensuremath{{#1}{\downharpoonright}^{#2}}}

% symbols of ordered logic
\newcommand{\fuse}{\mathbin{\bullet}}
\newcommand{\righti}{\twoheadrightarrow}
\newcommand{\lefti}{\rightarrowtail}
\newcommand{\gnab}{\mbox{\textexclamdown}}
\newcommand{\scriptgnab}{\mbox{\scriptsize{\textexclamdown}}}

\newcommand{\mconc}{{\it true}}
\newcommand{\mvalid}{{\it valid}}

\newcommand{\mlax}{{\it lax}}
\newcommand{\mtrue}{{\it ord}}
\newcommand{\meph}{{\it eph}}
\newcommand{\mpers}{{\it pers}}
\newcommand{\mlvl}{{\it lvl}}

\newcommand{\isconc}[1]{{#1}\,{\mconc}}
\newcommand{\isvalid}[1]{{#1}\,{\mvalid}}

\newcommand{\islax}[1]{{#1}\,{\mlax}}
\newcommand{\istrue}[1]{{#1}\,{\mtrue}}
\newcommand{\iseph}[1]{{#1}\,{\meph}}
\newcommand{\ispers}[1]{{#1}\,{\mpers}}
\newcommand{\islvl}[1]{{#1}\,{\mlvl}}

% judgments of linear logic
\newcommand{\altv}{\Longrightarrow}
\newcommand{\seq}[3]{{#1};{#2} \longrightarrow {#3} \mathstrut}
\newcommand{\altseq}[3]{{#1};{#2} \altv {#3} \mathstrut}
\newcommand{\pseq}[2]{{#1} \longrightarrow {#2} \mathstrut}

\newcommand{\mildseq}[3]{{#1};{#2} \vdash {#3} \mathstrut}
\newcommand{\andseq}[3]{{#1};{#2} \Vdash {#3} \mathstrut}
\newcommand{\mildrfoc}[3]{{#1};{#2} \vdash [{#3}] \mathstrut}
\newcommand{\mildinv}[3]{{#1};{#2} \vdash {#3} \mathstrut}
\newcommand{\mildlfoc}[4]{{#1};{#2}, [{#3}] \vdash {#4} \mathstrut}

\newcommand{\foc}[3]{{#1};{#2} \vdash {#3} \mathstrut}
\newcommand{\focx}[3]{{#1};{#2} \vdash_{\Sigma,\subord} {#3} \mathstrut}
\newcommand{\rfoc}[3]{{#1};{#2} \vdash [{#3}] \mathstrut}
\newcommand{\ifoc}[4]{{#1};{#2} {\mid} {#3} \vdash {#4} \mathstrut}
\newcommand{\lfoc}[4]{{#1};{#2}[{#3}] \vdash {#4} \mathstrut}

\newcommand{\foct}[4]{{#1}; {#2} \vdash {#3} : {#4} \mathstrut}
\newcommand{\foctx}[4]{{#1}; {#2} \vdash_{\Sigma,\subord} {#3} : {#4}
 \mathstrut}
\newcommand{\rfoct}[4]{{#1};{#2} \vdash {#3} : [{#4}] \mathstrut}
\newcommand{\lfoct}[4]{{#1};{#2}[{#3}] \vdash {#3} : [{#4}] \mathstrut}

\newcommand{\focsx}[5]{{#1}; {#2} \vdash_{\Sigma,\subord} \{{#3}\}{#4} : {#5}
 \mathstrut}

\newcommand{\slss}[4]{{#2}; {#3} \vdash_{#1} {#4}}
\newcommand{\slst}[5]{{#2}; {#3} \vdash_{#1} {#4} : {#5}}


\newcommand{\tfocusr}[1]{{#1}}
\newcommand{\tfocusl}[2]{{#1} \cdot {#2}}
\newcommand{\tetan}[1]{\langle {#1} \rangle}
\newcommand{\tetapp}[2]{\langle {#1} \rangle_p . {#2} }
\newcommand{\tetapl}[2]{\langle {#1} \rangle_l . {#2} }
\newcommand{\tetap}[2]{\langle {#1} \rangle . {#2}}
\newcommand{\tupr}[1]{{\uparrow}{#1}}
\newcommand{\tupl}[1]{{\uparrow}{#1}}
\newcommand{\tlaxr}[1]{\{{#1}\}}
\newcommand{\tlaxl}[1]{\{{#1}\}}
\newcommand{\tdownr}[1]{{\downarrow}{#1}}
\newcommand{\tdownl}[2]{{\downarrow}{#1}.{#2}}
\newcommand{\tbangr}[1]{{!}{#1}}
\newcommand{\tbangl}[2]{{!}{#1}.{#2}}
\newcommand{\tgnabr}[1]{{\gnab}{#1}}
\newcommand{\tgnabl}[2]{{\gnab}{#1}.{#2}}
\newcommand{\tnil}{\textsc{nil}}
\newcommand{\tabort}{\textsc{abort}}
\newcommand{\tinl}[1]{\textsc{inl}({#1})}
\newcommand{\tinr}[1]{\textsc{inr}({#1})}
\newcommand{\toplusl}[2]{[{#1},{#2}]}
\newcommand{\toner}{()}
\newcommand{\ttopr}{\top}
\newcommand{\tfuser}[2]{{#1} \fuse {#2}}
\newcommand{\twithr}[2]{{#1} \with {#2}}
\newcommand{\tpione}[1]{\pi_1;{#1}}
\newcommand{\tpitwo}[1]{\pi_2;{#1}}
\newcommand{\tfusel}[1]{{\fuse}{#1}}
\newcommand{\tlamr}[1]{{\lambda}^{>}{#1}}
\newcommand{\tappr}[2]{{#1}{^{>}}{#2}}
\newcommand{\tlaml}[1]{{\lambda}^{<}{#1}}
\newcommand{\tappl}[2]{{#1}{^{<}}{#2}}
\newcommand{\tonel}[1]{().{#1}}
\newcommand{\Sp}{{\it Sp}}
\newcommand{\texistsr}[2]{{#1},{#2}}
\newcommand{\texistsl}[2]{{#1}.{#2}}
\newcommand{\tforallr}[2]{[{#1}].{#2}}
\newcommand{\tforalll}[2]{[{#1}]; {#2}}
\newcommand{\tunifr}{\textsc{refl}}

\newcommand{\spi}{{\it sp}}
\newcommand{\lfnil}{\lf{()}}
\newcommand{\lflam}[2]{\lf{\lambda {#1}.{#2}}}
\newcommand{\lfapp}[2]{\lf{{#1};{#2}}}
\newcommand{\lfroot}[2]{{{#1} \cdot \lf{#2}}}
\newcommand{\lfpi}[3]{\Pi{\lf{{#1}}{:}{#2}.{#3}}}

\newcommand{\susp}[1]{\langle {#1} \rangle}

\newcommand{\ofirstseq}[5]{{#1};{#2};{#3};{#4} \altv {#5} \mathstrut}
\newcommand{\oseq}[4]{{#1};{#2};{#3} \altv {#4} \mathstrut}
\newcommand{\oiseq}[2]{\oseq{#1}{\cdot}{/{#2}/}{\islvl{#2}}}
\newcommand{\orseq}[4]{\oseq{#1}{#2}{{#3}}{\islvl{#4}}}
\newcommand{\orfseq}[4]{\ofirstseq{\Psi}{#1}{#2}{{#3}}{\islvl{#4}}}
\newcommand{\otseq}[4]{\oseq{#1}{#2}{{#3}}{\isconc{#4}}}
\newcommand{\olseq}[5]{\oseq{#1}{#2}{{#3}/{#4}/{#5}}{U}}
\newcommand{\olfseq}[5]{\ofirstseq{\Psi}{#1}{#2}{{#3}/{#4}/{#5}}{U}}
\newcommand{\opseq}[4]{\oseq{#1}{#2}{{#3},{#4}}{U}}
\newcommand{\opfseq}[4]{\ofirstseq{\Psi}{#1}{#2}{{#3},{#4}}{U}}

\newcommand{\mkunit}{\cdot}
\newcommand{\matchunit}{\cdot}
\newcommand{\mkconj}[2]{{#1},{#2}}
\newcommand{\matchconj}[2]{{#1},{#2}}

\newcommand{\invoff}[2]{{#1}\{{#2}\mbox\}}
\newcommand{\tackon}[2]{{#1}\{{#2}\}}
\newcommand{\frameoff}[2]{{#1}\mbox{$\{\hspace{-4pt}\{$}{#2}\mbox{$\}\hspace{-4pt}\}$}}

\newcommand{\tackonstart}{\{}
\newcommand{\tackonstop}{\}}
\newcommand{\frameoffstart}{\mbox{$\{\hspace{-4pt}\{$}}
\newcommand{\frameoffstop}{\mbox{$\}\hspace{-4pt}\}$}}

\newcommand{\urfoc}[3]{{#1};{#2} \longrightarrow [{#3}] \mathstrut}
\newcommand{\ulfoc}[4]{{#1};{#2} \,[#3] \longrightarrow {#4} \mathstrut}
\newcommand{\uinv}[4]{{#1};{#2};{#3} \longrightarrow {#4} \mathstrut}

\newcommand{\stableR}[1]{{#1}\,\mathit{stable_R} \mathstrut}
\newcommand{\stableL}[1]{{#1}\,\mathit{stable_L} \mathstrut}

\newcommand{\etana}[2]{\eta_{#2}({#1})}
\newcommand{\etapa}[3]{\eta_{#3}({#1}.{#2})}

\newcommand{\subst}[2]{{#1} \circ {#2}}
\newcommand{\rsubsta}[4]{\llbracket {#1}/{#2} \rrbracket^{#4} {#3}}
\newcommand{\lsubsta}[3]{\llbracket {#1} \rrbracket^{#3} {#2}}
\newcommand{\rsubst}[3]{\rsubsta{#1}{#2}{#3}{}}
\newcommand{\lsubst}[2]{\lsubsta{#1}{#2}{}}

\newcommand{\arb}{\mathbin{\mbox{$\ocircle$\hspace{-7pt}{\footnotesize ?}\hspace{3pt}}}}
\newcommand{\softinterp}[1]{{{\ulcorner{{#1}}\urcorner}}}
\newcommand{\interp}[1]{{\no{\ulcorner\obj{{#1}}\urcorner}}}
\newcommand{\ctxinterp}[1]{\no{\mbox{$\ulcorner\!\!\ulcorner$}\obj{{#1}}\mbox{$\urcorner\!\!\urcorner$}}}

\newcommand{\transop}[1]{{\it Op}({#1})}

\newcommand{\opbasic}[3]{\llbracket {#1} \rrbracket({\sf a},\lf{#2},\lf{#3})}
\newcommand{\opsubst}[1]{{{#1}^\dag}}

\newcommand{\tlet}[2]{\{{\sf let}\,{#1}\,{\sf in}\, {#2} \}}
\newcommand{\tstep}[3]{\{{#1}\} \leftarrow \tfocusl{#2}{#3}}
\newcommand{\trstep}[2]{\{{#1}\} \leftarrow {#2}}
\newcommand{\mkpat}[1]{{\it pat}({#1})}
\newcommand{\emptytrace}{\diamond}

\newcommand{\subord}{\mathcal R}


\newcommand{\siggen}{\Sigma_{\it Gen}}
\newcommand{\siggenorder}{\Sigma_{\it Gen\ref{fig:gen-order}}}
\newcommand{\siggenordertp}{\Sigma_{\it Gen\ref{fig:gen-ordertp}}}
\newcommand{\siggenstate}{\Sigma_{\it Gen\ref{fig:gen-state}}}
\newcommand{\siggendests}{\Sigma_{\it Gen\ref{fig:gen-destinations}}}
\newcommand{\siggenletcc}{\Sigma_{\it Gen\ref{fig:gen-letcc2}}}


\begin{document}
I think I've sorted out the problems that I was having. However, in doing
so it's been very important to make some stability assumptions in the 
cut admissibility proof. In particular, in these two cases:
\[
\infer-
{\mildseq{\Gamma}{\Delta', \delta}{U}}
{\mildseq{\Gamma}{\delta}{A^+}
 &
 \mildseq{\Gamma}{\Delta', A^+}{U}}
\quad
\infer-
{\mildseq{\Gamma}{\delta', \Delta}{\gamma}}
{\mildseq{\Gamma}{\Delta}{A^-}
 &
 \mildseq{\Gamma}{\delta', A^-}{\gamma}}
\]
I believe that it is rather important, if a structural proof is our aim, to
force $\Gamma'$, $\Gamma$, and $U$ to be stable (contexts contain no positive
propositions and succeedents are not negative propositions).

To see why, try running through what a structural focalization proof
{\it ought} to do on these cases:
\[
\infer-
{\mildseq{\Gamma}{A \otimes B, \langle p^+ \rangle, p^+ \lolli q^+}
  {C \multimap D}}
{\infer
 {\mildseq{\Gamma}{\langle p^+ \rangle, p^+ \lolli q^+}{q^+}}
 {\infer
  {\mildseq{\Gamma}{\langle p^+ \rangle, [p^+ \lolli q^+]}{q^+}}
  {\infer
   {\mildseq{\Gamma}{\langle p^+ \rangle}{[p^+]}}
   {}
   &
   \infer
   {\mildseq{\Gamma}{[q^+]}{q^+}}
   {\infer
    {\mildseq{\Gamma}{q^+}{q^+}}
    {\infer
     {\mildseq{\Gamma}{\langle q^+ \rangle}{q^+}}
     {\infer
      {\mildseq{\Gamma}{\langle q^+ \rangle}{[q^+]}}
      {}}}}}}
 &
 \infer
 {\mildseq{\Gamma}{A \otimes B, q^+}{C \multimap D}}
 {\infer
  {\mildseq{\Gamma}{A \otimes B, \langle q^+ \rangle}{C \lolli D}}
  {\infer
   {\mildseq{\Gamma}{A, B, \langle q^+ \rangle}{C \lolli D}}
   {\deduce
    {\mildseq{\Gamma}{A, B, \langle q^+ \rangle, C}{D}}
    {\vdots}}}}}
\]
\[
\infer-
{\mildseq{\Gamma}{A \otimes B, a^- \lolli b^-}{\langle b^- \rangle}}
{\infer
 {\mildseq{\Gamma}{A \otimes B}{a^-}}
 {\infer
  {\mildseq{\Gamma}{A \otimes B}{\langle a^- \rangle}}
  {\deduce
   {\mildseq{\Gamma}{A, B}{\langle a^- \rangle}}
   {\vdots}}}
 &
 \infer
 {\mildseq{\Gamma}{a^-, a^- \lolli b^-}{\langle b^- \rangle}}
 {\infer
  {\mildseq{\Gamma}{a^-, [a^- \lolli b^-]}{\langle b^- \rangle}}
  {\infer
   {\mildseq{\Gamma}{a^-}{[a^-]}}
   {\infer
    {\mildseq{\Gamma}{a^-}{a^-}}
    {\infer
     {\mildseq{\Gamma}{a^-}{\langle a^- \rangle}}
     {\infer
      {\mildseq{\Gamma}{[a^-]}{\langle a^- \rangle}}
      {}}}}
   &
   \infer
   {\mildseq{\Gamma}{[b^-]}{\langle b^- \rangle}}
   {}}}}
\]
I claim that the answer is
``permute the leftmost proof {\it around} the blocking invertible step
until the principle formula is the only invertible formula, then
begin a principal cut.'' I also claim that formalizing this has quadratic
proof complexity. 

\paragraph{Solution 1} Restrict cut as described above. This is my
desired solution, assuming the rest of the proof goes through. However,
your shiftless unfocused admissibility lemmas seem to
depend on the non-stable cut principles, so we're back to a situation
where you need to use shifts (or Miller-style delay operators)
to prove unfocused admissibility with linear proof complexity.

\paragraph{Solution 2} You could factor out all that complexity
if you factored all your left rules into one forward-chaining monotonicity 
rule
\[
\infer
{\mildseq{\Gamma}{\Delta, A^+}{U}}
{A^+ \leadsto \Theta
 &
 \forall (\Delta' \in \Theta)
 &
 \longrightarrow 
 &
 \mildseq{\Gamma}{\Delta, \Delta'}{U}}
\]
but I don't plan to take that route right now.

\end{document}


There is a still-deeper problem with the structural proof of cut admissibility
in the confluent system. In this presentation, I use ``unstable'' to refer
to an inversion sequent with multiple 

\paragraph{Claim 1}
In the positive focused cut, we {\it must} allow $\Delta'$ to be unstable.
\[
\infer-[]
{\Gamma; \Delta', \Delta \vdash U}
{\Gamma; \Delta \vdash [A^+]
 &
 \Gamma; \Delta', A^+ \vdash U}
\]

\paragraph{Justification 1}
If we do not allow $\Delta'$ to be unstable, we can't invoke the
induction hypothesis in a way that allows us to prove the principal
cut for tensor.
\[
\infer-[]
{\Gamma; \Delta', \Delta_1, \Delta_2 \vdash U}
{\infer
 {\Gamma; \Delta_1, \Delta_2 \vdash [A^+ \otimes B^+]}
 {\Gamma; \Delta_1 \vdash [A^+] 
  &
  \Gamma; \Delta_2 \vdash [B^+]}
 &
 \infer
 {\Gamma; \Delta', A^+ \otimes B^+ \vdash U}
 {\Gamma; \Delta', A^+, B^+ \vdash U}}
\]
Reduces either as
\[
\infer-[]
{\Gamma; \Delta', \Delta_1, \Delta_2 \vdash U}
{\Gamma; \Delta_1 \vdash [A^+] 
 &
 \infer-[]
 {\Gamma; \Delta', A^+, \Delta_2 \vdash U}
 {\Gamma; \Delta_2 \vdash [B^+]
  &
  \Gamma; \Delta', A^+, B^+ \vdash U}}
\]
or as
\[
\infer-[]
{\Gamma; \Delta', \Delta_1, \Delta_2 \vdash U}
{\Gamma; \Delta_2 \vdash [B^+] 
 &
 \infer-[]
 {\Gamma; \Delta', \Delta_1, B^+ \vdash U}
 {\Gamma; \Delta_1 \vdash [A^+]
  &
  \Gamma; \Delta', A^+, B^+ \vdash U}}
\]
and the ``more upper'' cut has an independent positive in the context
in both cases ($A^+$ in the first case, and $B^-$ in the second).

\paragraph{Claim 2}
In the negative unfocused cut, we {\it must} allow $\Delta'$ to be unstable.
\[
\infer-[]
{\Gamma; \Delta', \Delta \vdash U}
{\Gamma; \Delta \vdash A^-
 &
 \Gamma; \Delta', A^- \vdash U}
\]

\paragraph{Justification 2}
This cut arises inevitably from the principal cut for ${\downarrow}A^-$
(or blur focus, if you don't have shifts), which is an instance of positive
focused cut, where we have already established that we must have
an unstable $\Delta'$ (Claim 1).
\[
\infer-[]
{\Gamma; \Delta', \Delta \vdash U}
{\infer[]
 {\Gamma; \Delta \vdash [{\downarrow}A^-]}
 {\Gamma; \Delta \vdash A^-}
 &
 \infer[]
 {\Gamma; \Delta', {\downarrow}A^- \vdash U}
 {\Gamma; \Delta', A^- \vdash U}}
\]

\paragraph{Claim 1}
In the positive un-focused cut, we
{\it must} allow $\Delta'$ to be unstable. Otherwise, we are 



