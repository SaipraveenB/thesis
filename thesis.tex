%for a more compact document, add the option openany to avoid
%starting all chapters on odd numbered pages
\documentclass[12pt]{cmuthesis}

% This is a template for a CMU thesis.  It is 18 pages without any content :-)
% The source for this is pulled from a variety of sources and people.
% Here's a partial list of people who may or may have not contributed:
%
%        bnoble   = Brian Noble
%        caruana  = Rich Caruana
%        colohan  = Chris Colohan
%        jab      = Justin Boyan
%        josullvn = Joseph O'Sullivan
%        jrs      = Jonathan Shewchuk
%        kosak    = Corey Kosak
%        mjz      = Matt Zekauskas (mattz@cs)
%        pdinda   = Peter Dinda
%        pfr      = Patrick Riley
%        dkoes = David Koes (me)

% My main contribution is putting everything into a single class files and small
% template since I prefer this to some complicated sprawling directory tree with
% makefiles.

\definecolor{ValidRed}{rgb}{0.5,0,0}
\definecolor{TrueRed}{rgb}{1,0.6,0.2}
\definecolor{ValidBlue}{rgb}{0,0,0.5}
\definecolor{TrueBlue}{rgb}{0.2,0.6,1}

% some useful packages
\usepackage{times}
\usepackage{proof}
\usepackage{fullpage}
\usepackage{graphicx}
\usepackage{amsmath}
\usepackage[numbers,sort]{natbib}
\usepackage[backref,pageanchor=true,plainpages=false, pdfpagelabels, bookmarks,bookmarksnumbered,
%pdfborder=0 0 0,  %removes outlines around hyper links in online display
]{hyperref}
\usepackage{subfigure}

% Approximately 1" margins, more space on binding side
%\usepackage[letterpaper,twoside,vscale=.8,hscale=.75,nomarginpar]{geometry}
%for general printing (not binding)
\usepackage[letterpaper,twoside,vscale=.8,hscale=.75,nomarginpar,hmarginratio=1:1]{geometry}

% Provides a draft mark at the top of the document. 
\draftstamp{\today}{DRAFT}

\hypersetup{colorlinks=true,citecolor=blue,urlcolor=blue,linkcolor=black}

\begin {document} 
\frontmatter

%initialize page style, so contents come out right (see bot) -mjz
\pagestyle{empty}

\title{ %% {\it \huge Thesis Proposal}\\
{\bf Awesome Work in Computer Science}}
\author{Robert J. Simmons}
\date{The Future}
\Year{Year In The Future}
\trnumber{CMU-CS-THE-FUTURE}

\committee{
Frank Pfenning, Chair \\
Robert Harper \\
Andr{\'e} Platzer \\
Iliano Cervesato, Carnegie Mellon Qatar \\
Dale Miller, INRIA-Saclay \& LIX/Ecole Polytechnique
}

\support{}
\disclaimer{}

% copyright notice generated automatically from Year and author.
% permission added if \permission{} given.

\keywords{Stuff, More Stuff}

\maketitle

\begin{dedication}
For my dog
\end{dedication}

\pagestyle{plain} % for toc, was empty

%% Obviously, it's probably a good idea to break the various sections of your thesis
%% into different files and input them into this file...

\begin{abstract}
A short summary.
\end{abstract}

\begin{acknowledgments}
My advisor is cool.
\end{acknowledgments}



\tableofcontents
\listoffigures
\listoftables

\mainmatter

%% Double space document for easy review:
%\renewcommand{\baselinestretch}{1.66}\normalsize

% The other requirements Catherine has:
%
%  - avoid large margins.  She wants the thesis to use fewer pages, 
%    especially if it requires colour printing.
%
%  - The thesis should be formatted for double-sided printing.  This
%    means that all chapters, acknowledgements, table of contents, etc.
%    should start on odd numbered (right facing) pages.
%
%  - You need to use the department standard tech report title page.  I
%    have tried to ensure that the title page here conforms to this
%    standard.
%
%  - Use a nice serif font, such as Times Roman.  Sans serif looks bad.
%
% Other than that, just make it look good...


\chapter{Introduction}

\part{Substructural logic}

\chapter{Substructural intuitionstic logics}

\chapter{Focusing substructural logics}

\section{Explicit shifts in focusing}

\section{Polarization and erasure}

\section{Cut admissibility}

\section{Identity expansion}

\section{Unfocused admissibility}

\section{Soundness and completeness}

\section{Ordered linear logic}

\chapter{Substructural logical specifications}

Logical framework time!

\section{The process interpetation}

\part{Logical correspondance and approximation}

\chapter{Implementing backwards-chaining with foward-chaining}

\section{Compilation}

\section{Tail-recursion}

\chapter{Transformations on logical specifications}

\section{Defunctionalization}

\section{Environment semantics}

\section{Destination-passing}

\chapter{Linear logical approximation}


\section{}

\section{Examples}

\subsection{Control flow analysis}

\subsection{Alias analysis}

\part{Reasoning about substructural logical specifications}

\chapter{Case studies}

% \chapter{Programming with canonical forms}

% \newcommand{\F}[1]{\ensuremath{F({#1})}}
% \newcommand{\G}[1]{\ensuremath{G(\textcolor{TrueBlue}{#1})}}
% \newcommand{\upX}[1]{\ensuremath{{\uparrow}\textcolor{ValidBlue}{#1}}}
% \newcommand{\downX}[1]{\ensuremath{{\downarrow}\textcolor{ValidRed}{#1}}}
% \newcommand{\upA}[1]{\ensuremath{{\uparrow}\textcolor{TrueBlue}{#1}}}
% \newcommand{\downA}[1]{\ensuremath{{\downarrow}\textcolor{TrueRed}{#1}}}

% \newcommand{\valid}[1]{\ensuremath{{\downarrow}\textcolor{ValidBlue}{{#1}\,\mathit{valid}}}}
% \newcommand{\true}[1]{\ensuremath{{\downarrow}\textcolor{TrueBlue}{{#1}\,\mathit{true}}}}

% \newcommand{\ajseq}[2]{\ensuremath{\mathstrut{#1} \vdash {#2}}}
% \newcommand{\ajinv}[3]{\ajseq{{#1}; {#2}}{\textcolor{ValidRed}{#3}}}
% \newcommand{\ajrfoc}[2]{\ajseq{{#1}}{[\textcolor{ValidBlue}{#2}]}}
% \newcommand{\ajlfoc}[3]{\ajseq{{#1} [{#2}]}{\textcolor{ValidRed}{#3}}}
% \newcommand{\ajAseq}[3]{\ensuremath{\mathstrut{#1} \vdash {#2}}}
% \newcommand{\ajAinv}[4]{\ajseq{{#1}; {#2}}{\textcolor{ValidRed}{#3}}}
% \newcommand{\ajArfoc}[3]{\ajseq{{#1}}{[\textcolor{ValidBlue}{#2}]}}
% \newcommand{\ajAlfoc}[4]{\ajseq{{#1} [{#2}]}{\textcolor{ValidRed}{#3}}}
% \newcommand{\ajXseq}[2]{\ensuremath{\mathstrut{#1} \vdash {#2}}}
% \newcommand{\ajXinv}[3]{\ajseq{{#1}; {#2}}{\textcolor{ValidRed}{#3}}}
% \newcommand{\ajXrfoc}[2]{\ajseq{{#1}}{[\textcolor{ValidBlue}{#2}]}}
% \newcommand{\ajXlfoc}[3]{\ajseq{{#1} [{#2}]}{\textcolor{ValidRed}{#3}}}

% \begin{figure}
% \fbox{\ajXrfoc{\Gamma}{X^+}}
% \[
% \infer[x^+_R]
% {\ajrfoc{\valid{x^+}}{x^+}}
% {}
% \qquad
% \infer[{\downarrow}_{XR}]
% {\ajXrfoc{\Gamma}{\downX{X^-}}}
% {\ajXinv{\Gamma}{\cdot}{X^-}}
% \qquad
% \infer[G_R]
% {\ajXrfoc{\Gamma}{\G{A}}}
% {\ajArfoc{\Gamma}{\cdot}{A}}
% \]

% \fbox{\ajArfoc{\Gamma}{\Delta}{A^+}}
% \[
% \infer[a^+_R]
% {\ajArfoc{\cdot}{a^+}{a^+}}
% {}
% \qquad
% \infer[{\downarrow}_{AR}]
% {\ajArfoc{\Gamma}{\Delta}{\downA{A^-}}}
% {\ajAinv{\Gamma}{\Delta}{\cdot}{A^-}}
% \]

% \caption{Focused adjoint logic}
% \end{figure}

\chapter{Conclusion}

%\appendix
%\include{appendix}

\backmatter

%\renewcommand{\baselinestretch}{1.0}\normalsize

% By default \bibsection is \chapter*, but we really want this to show
% up in the table of contents and pdf bookmarks.
\renewcommand{\bibsection}{\chapter{\bibname}}
%\newcommand{\bibpreamble}{This text goes between the ``Bibliography''
%  header and the actual list of references}
\bibliographystyle{plainnat}
\bibliography{register} %your bib file

\end{document}
