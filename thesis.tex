%for a more compact document, add the option openany to avoid
%starting all chapters on odd numbered pages
\documentclass[12pt,openany]{cmuthesis}

% This is a template for a CMU thesis.  It is 18 pages without any content :-)
% The source for this is pulled from a variety of sources and people.
% Here's a partial list of people who may or may have not contributed:
%
%        bnoble   = Brian Noble
%        caruana  = Rich Caruana
%        colohan  = Chris Colohan
%        jab      = Justin Boyan
%        josullvn = Joseph O'Sullivan
%        jrs      = Jonathan Shewchuk
%        kosak    = Corey Kosak
%        mjz      = Matt Zekauskas (mattz@cs)
%        pdinda   = Peter Dinda
%        pfr      = Patrick Riley
%        dkoes = David Koes (me)

% My main contribution is putting everything into a single class files and small
% template since I prefer this to some complicated sprawling directory tree with
% makefiles.

\definecolor{ValidRed}{rgb}{0.5,0,0}
\definecolor{TrueRed}{rgb}{1,0.6,0.2}
\definecolor{ValidBlue}{rgb}{0,0,0.5}
\definecolor{TrueBlue}{rgb}{0.2,0.6,1}

% some useful packages
\usepackage{fancyvrb}
\usepackage{times}
\usepackage{dashrule}
\usepackage{proof-dashed}
\usepackage[show]{ed}
\usepackage{fullpage}
\usepackage{graphicx}
\usepackage{xcolor}
\usepackage{tikz}
\usepackage{rotating}
\usetikzlibrary{arrows,decorations.pathmorphing,backgrounds,fit}
\usepackage{amsthm}
\usepackage{amsmath}
\usepackage{latexsym}
\usepackage{amssymb}            % for \multimap (-o)
\usepackage{stmaryrd}           % for \binampersand (&), \bindnasrepma (\paar)
\usepackage{wasysym}            % for \ocircle
\usepackage[numbers,sort]{natbib}
\usepackage[backref,pageanchor=true,plainpages=false, pdfpagelabels, bookmarks,bookmarksnumbered,
%pdfborder=0 0 0,  %removes outlines around hyper links in online display
]{hyperref}
\usepackage{subfigure}

% Approximately 1" margins, more space on binding side
%\usepackage[letterpaper,twoside,vscale=.8,hscale=.75,nomarginpar]{geometry}
%for general printing (not binding)
\usepackage[letterpaper,twoside,vscale=.8,hscale=.75,nomarginpar,hmarginratio=1:1]{geometry}

% Provides a draft mark at the top of the document. 

\definecolor{grayout}{rgb}{.8,.8,.8}
\definecolor{plesantgreen}{rgb}{.1,.5,0}
\newcommand{\gr}[1]{\textcolor{plesantgreen}{\ensuremath{{#1}}}}

\definecolor{lfblue}{rgb}{.5,.2,0}
\newcommand{\lf}[1]{\textcolor{lfblue}{\ensuremath{{#1}}}}



\definecolor{justblack}{rgb}{0,0,0}
\newcommand{\no}[1]{\textcolor{justblack}{\ensuremath{{#1}}}}

\definecolor{objgreen}{rgb}{0,.3,.5}
\newcommand{\obj}[1]{\textcolor{objgreen}{\ensuremath{{#1}}}}

\hypersetup{colorlinks=true,citecolor=blue,urlcolor=blue,linkcolor=black}

\newcommand{\robnote}[1]{\footnote{{\bf NOTE TO SELF:} ~~ {#1}}}
\newcommand{\futurework}[1]{}

\newcommand{\ollll}{OL$_3$}
\newcommand{\sls}{SLS}
\newcommand{\simplearrow}{\rightarrow}
\newcommand{\rowan}{${\lambda}^{\ocircle}$}

\newcommand{\mb}{Coq}

\renewcommand{\labelitemi}{$\ast$}



% Theorems
\newtheorem{theorem}{Theorem}[chapter]
\newtheorem*{lemma}{Lemma}
\newtheorem{proposition}[theorem]{Proposition}
\newtheorem{definition}[theorem]{Definition}

% symbols of linear logic
\newcommand{\lolli}{\multimap}
\newcommand{\tensor}{\otimes}
\newcommand{\with}{\mathbin{\binampersand}}
\newcommand{\paar}{\mathbin{\bindnasrepma}}
\newcommand{\one}{\mathbf{1}}
\newcommand{\zero}{\mathbf{0}}
\newcommand{\bang}{{!}}
\newcommand{\pbang}{\mbox{\hspace{2pt}$\mathbb !$\hspace{-4.7pt}$_\curlyvee$\hspace{-1pt}}}
\newcommand{\whynot}{{?}}
\newcommand{\bilolli}{\mathrel{\raisebox{1pt}{\ensuremath{\scriptstyle\circ}}{\lolli}}}
% \oplus, \top, \bot
\newcommand{\deupdown}{\mbox{${\uparrow}{\downarrow}\hspace{-11.4pt}\diagup$}}
\newcommand{\dedownup}{\mbox{${\downarrow}{\uparrow}\hspace{-11.4pt}\diagdown$}}

\newcommand{\restrictto}[2]{\ensuremath{{#1}{\upharpoonright}_{#2}}}
\newcommand{\restrictsig}[2]{\ensuremath{{#1}{\mbox{\lightning}}_{#2}}}
\newcommand{\restrictfrom}[2]{\ensuremath{{#1}{\downharpoonright}^{#2}}}

% symbols of ordered logic
\newcommand{\fuse}{\mathbin{\bullet}}
\newcommand{\righti}{\twoheadrightarrow}
\newcommand{\lefti}{\rightarrowtail}
\newcommand{\gnab}{\mbox{\textexclamdown}}
\newcommand{\scriptgnab}{\mbox{\scriptsize{\textexclamdown}}}

\newcommand{\mconc}{{\it true}}
\newcommand{\mvalid}{{\it valid}}

\newcommand{\mlax}{{\it lax}}
\newcommand{\mtrue}{{\it ord}}
\newcommand{\meph}{{\it eph}}
\newcommand{\mpers}{{\it pers}}
\newcommand{\mlvl}{{\it lvl}}

\newcommand{\isconc}[1]{{#1}\,{\mconc}}
\newcommand{\isvalid}[1]{{#1}\,{\mvalid}}

\newcommand{\islax}[1]{{#1}\,{\mlax}}
\newcommand{\istrue}[1]{{#1}\,{\mtrue}}
\newcommand{\iseph}[1]{{#1}\,{\meph}}
\newcommand{\ispers}[1]{{#1}\,{\mpers}}
\newcommand{\islvl}[1]{{#1}\,{\mlvl}}

% judgments of linear logic
\newcommand{\altv}{\Longrightarrow}
\newcommand{\seq}[3]{{#1};{#2} \longrightarrow {#3} \mathstrut}
\newcommand{\altseq}[3]{{#1};{#2} \altv {#3} \mathstrut}
\newcommand{\pseq}[2]{{#1} \longrightarrow {#2} \mathstrut}

\newcommand{\mildseq}[3]{{#1};{#2} \vdash {#3} \mathstrut}
\newcommand{\andseq}[3]{{#1};{#2} \Vdash {#3} \mathstrut}
\newcommand{\mildrfoc}[3]{{#1};{#2} \vdash [{#3}] \mathstrut}
\newcommand{\mildinv}[3]{{#1};{#2} \vdash {#3} \mathstrut}
\newcommand{\mildlfoc}[4]{{#1};{#2}, [{#3}] \vdash {#4} \mathstrut}

\newcommand{\foc}[3]{{#1};{#2} \vdash {#3} \mathstrut}
\newcommand{\focx}[3]{{#1};{#2} \vdash_{\Sigma,\subord} {#3} \mathstrut}
\newcommand{\rfoc}[3]{{#1};{#2} \vdash [{#3}] \mathstrut}
\newcommand{\ifoc}[4]{{#1};{#2} {\mid} {#3} \vdash {#4} \mathstrut}
\newcommand{\lfoc}[4]{{#1};{#2}[{#3}] \vdash {#4} \mathstrut}

\newcommand{\foct}[4]{{#1}; {#2} \vdash {#3} : {#4} \mathstrut}
\newcommand{\foctx}[4]{{#1}; {#2} \vdash_{\Sigma,\subord} {#3} : {#4}
 \mathstrut}
\newcommand{\rfoct}[4]{{#1};{#2} \vdash {#3} : [{#4}] \mathstrut}
\newcommand{\lfoct}[4]{{#1};{#2}[{#3}] \vdash {#3} : [{#4}] \mathstrut}

\newcommand{\focsx}[5]{{#1}; {#2} \vdash_{\Sigma,\subord} \{{#3}\}{#4} : {#5}
 \mathstrut}

\newcommand{\slss}[4]{{#2}; {#3} \vdash_{#1} {#4}}
\newcommand{\slst}[5]{{#2}; {#3} \vdash_{#1} {#4} : {#5}}


\newcommand{\tfocusr}[1]{{#1}}
\newcommand{\tfocusl}[2]{{#1} \cdot {#2}}
\newcommand{\tetan}[1]{\langle {#1} \rangle}
\newcommand{\tetapp}[2]{\langle {#1} \rangle_p . {#2} }
\newcommand{\tetapl}[2]{\langle {#1} \rangle_l . {#2} }
\newcommand{\tetap}[2]{\langle {#1} \rangle . {#2}}
\newcommand{\tupr}[1]{{\uparrow}{#1}}
\newcommand{\tupl}[1]{{\uparrow}{#1}}
\newcommand{\tlaxr}[1]{\{{#1}\}}
\newcommand{\tlaxl}[1]{\{{#1}\}}
\newcommand{\tdownr}[1]{{\downarrow}{#1}}
\newcommand{\tdownl}[2]{{\downarrow}{#1}.{#2}}
\newcommand{\tbangr}[1]{{!}{#1}}
\newcommand{\tbangl}[2]{{!}{#1}.{#2}}
\newcommand{\tgnabr}[1]{{\gnab}{#1}}
\newcommand{\tgnabl}[2]{{\gnab}{#1}.{#2}}
\newcommand{\tnil}{\textsc{nil}}
\newcommand{\tabort}{\textsc{abort}}
\newcommand{\tinl}[1]{\textsc{inl}({#1})}
\newcommand{\tinr}[1]{\textsc{inr}({#1})}
\newcommand{\toplusl}[2]{[{#1},{#2}]}
\newcommand{\toner}{()}
\newcommand{\ttopr}{\top}
\newcommand{\tfuser}[2]{{#1} \fuse {#2}}
\newcommand{\twithr}[2]{{#1} \with {#2}}
\newcommand{\tpione}[1]{\pi_1;{#1}}
\newcommand{\tpitwo}[1]{\pi_2;{#1}}
\newcommand{\tfusel}[1]{{\fuse}{#1}}
\newcommand{\tlamr}[1]{{\lambda}^{>}{#1}}
\newcommand{\tappr}[2]{{#1}{^{>}}{#2}}
\newcommand{\tlaml}[1]{{\lambda}^{<}{#1}}
\newcommand{\tappl}[2]{{#1}{^{<}}{#2}}
\newcommand{\tonel}[1]{().{#1}}
\newcommand{\Sp}{{\it Sp}}
\newcommand{\texistsr}[2]{{#1},{#2}}
\newcommand{\texistsl}[2]{{#1}.{#2}}
\newcommand{\tforallr}[2]{[{#1}].{#2}}
\newcommand{\tforalll}[2]{[{#1}]; {#2}}
\newcommand{\tunifr}{\textsc{refl}}

\newcommand{\spi}{{\it sp}}
\newcommand{\lfnil}{\lf{()}}
\newcommand{\lflam}[2]{\lf{\lambda {#1}.{#2}}}
\newcommand{\lfapp}[2]{\lf{{#1};{#2}}}
\newcommand{\lfroot}[2]{{{#1} \cdot \lf{#2}}}
\newcommand{\lfpi}[3]{\Pi{\lf{{#1}}{:}{#2}.{#3}}}

\newcommand{\susp}[1]{\langle {#1} \rangle}

\newcommand{\ofirstseq}[5]{{#1};{#2};{#3};{#4} \altv {#5} \mathstrut}
\newcommand{\oseq}[4]{{#1};{#2};{#3} \altv {#4} \mathstrut}
\newcommand{\oiseq}[2]{\oseq{#1}{\cdot}{/{#2}/}{\islvl{#2}}}
\newcommand{\orseq}[4]{\oseq{#1}{#2}{{#3}}{\islvl{#4}}}
\newcommand{\orfseq}[4]{\ofirstseq{\Psi}{#1}{#2}{{#3}}{\islvl{#4}}}
\newcommand{\otseq}[4]{\oseq{#1}{#2}{{#3}}{\isconc{#4}}}
\newcommand{\olseq}[5]{\oseq{#1}{#2}{{#3}/{#4}/{#5}}{U}}
\newcommand{\olfseq}[5]{\ofirstseq{\Psi}{#1}{#2}{{#3}/{#4}/{#5}}{U}}
\newcommand{\opseq}[4]{\oseq{#1}{#2}{{#3},{#4}}{U}}
\newcommand{\opfseq}[4]{\ofirstseq{\Psi}{#1}{#2}{{#3},{#4}}{U}}

\newcommand{\mkunit}{\cdot}
\newcommand{\matchunit}{\cdot}
\newcommand{\mkconj}[2]{{#1},{#2}}
\newcommand{\matchconj}[2]{{#1},{#2}}

\newcommand{\invoff}[2]{{#1}\{{#2}\mbox\}}
\newcommand{\tackon}[2]{{#1}\{{#2}\}}
\newcommand{\frameoff}[2]{{#1}\mbox{$\{\hspace{-4pt}\{$}{#2}\mbox{$\}\hspace{-4pt}\}$}}

\newcommand{\tackonstart}{\{}
\newcommand{\tackonstop}{\}}
\newcommand{\frameoffstart}{\mbox{$\{\hspace{-4pt}\{$}}
\newcommand{\frameoffstop}{\mbox{$\}\hspace{-4pt}\}$}}

\newcommand{\urfoc}[3]{{#1};{#2} \longrightarrow [{#3}] \mathstrut}
\newcommand{\ulfoc}[4]{{#1};{#2} \,[#3] \longrightarrow {#4} \mathstrut}
\newcommand{\uinv}[4]{{#1};{#2};{#3} \longrightarrow {#4} \mathstrut}

\newcommand{\stableR}[1]{{#1}\,\mathit{stable_R} \mathstrut}
\newcommand{\stableL}[1]{{#1}\,\mathit{stable_L} \mathstrut}

\newcommand{\etana}[2]{\eta_{#2}({#1})}
\newcommand{\etapa}[3]{\eta_{#3}({#1}.{#2})}

\newcommand{\subst}[2]{{#1} \circ {#2}}
\newcommand{\rsubsta}[4]{\llbracket {#1}/{#2} \rrbracket^{#4} {#3}}
\newcommand{\lsubsta}[3]{\llbracket {#1} \rrbracket^{#3} {#2}}
\newcommand{\rsubst}[3]{\rsubsta{#1}{#2}{#3}{}}
\newcommand{\lsubst}[2]{\lsubsta{#1}{#2}{}}

\newcommand{\arb}{\mathbin{\mbox{$\ocircle$\hspace{-7pt}{\footnotesize ?}\hspace{3pt}}}}
\newcommand{\softinterp}[1]{{{\ulcorner{{#1}}\urcorner}}}
\newcommand{\interp}[1]{{\no{\ulcorner\obj{{#1}}\urcorner}}}
\newcommand{\ctxinterp}[1]{\no{\mbox{$\ulcorner\!\!\ulcorner$}\obj{{#1}}\mbox{$\urcorner\!\!\urcorner$}}}

\newcommand{\transop}[1]{{\it Op}({#1})}

\newcommand{\opbasic}[3]{\llbracket {#1} \rrbracket({\sf a},\lf{#2},\lf{#3})}
\newcommand{\opsubst}[1]{{{#1}^\dag}}

\newcommand{\tlet}[2]{\{{\sf let}\,{#1}\,{\sf in}\, {#2} \}}
\newcommand{\tstep}[3]{\{{#1}\} \leftarrow \tfocusl{#2}{#3}}
\newcommand{\trstep}[2]{\{{#1}\} \leftarrow {#2}}
\newcommand{\mkpat}[1]{{\it pat}({#1})}
\newcommand{\emptytrace}{\diamond}

\newcommand{\subord}{\mathcal R}


\newcommand{\siggen}{\Sigma_{\it Gen}}
\newcommand{\siggenorder}{\Sigma_{\it Gen\ref{fig:gen-order}}}
\newcommand{\siggenordertp}{\Sigma_{\it Gen\ref{fig:gen-ordertp}}}
\newcommand{\siggenstate}{\Sigma_{\it Gen\ref{fig:gen-state}}}
\newcommand{\siggendests}{\Sigma_{\it Gen\ref{fig:gen-destinations}}}
\newcommand{\siggenletcc}{\Sigma_{\it Gen\ref{fig:gen-letcc2}}}

\draftstamp{\today}{DRAFT}

% \usepackage{titlesec,minitoc}  
% \titleclass{\part}{top}
% \titleformat{\part}
% {\centering\normalfont\Huge\bfseries} 
% {Foo} 
% {0pt} 
% {Bar}


\begin {document} 
\frontmatter

%initialize page style, so contents come out right (see bot) -mjz
\pagestyle{empty}

\title{ %% {\it \huge Thesis Proposal}\\
{\bf Substructural Logical Specifications}}
\author{Robert J. Simmons}
\date{The Future}
\Year{Year In The Future}
\trnumber{CMU-CS-THE-FUTURE}

\committee{
Frank Pfenning, Chair \\
Robert Harper \\
Andr{\'e} Platzer \\
Iliano Cervesato, Carnegie Mellon Qatar \\
Dale Miller, INRIA-Saclay \& LIX/Ecole Polytechnique
}

\support{}
\disclaimer{}

% copyright notice generated automatically from Year and author.
% permission added if \permission{} given.

\keywords{logical frameworks, linear logic, ordered logic, operational
  semantics}

\maketitle

% XXX MAKE DEDICATION
% \begin{dedication}
% For my dog
% \end{dedication}

\pagestyle{plain} % for toc, was empty

%% Obviously, it's probably a good idea to break the various sections of your thesis
%% into different files and input them into this file...

% XXX MAKE ABSTRACT
% \begin{abstract}
% A short summary.
% \end{abstract}

% XXX MAKE ACKNOLWEDGEMENTS
% \begin{acknowledgments}
% COMMITTEE: Frank Robert Andr\`e Iliano Dale
% MENTORS (who aren't also committee):
%   - Jessica Hunt
%   - Andrew Appel, Dave Walker
%   - Sriram Rajamani, Aditya Nori
% DISCUSSERS (who aren't also readers):
%   - Noam Zeilberger
%   - Dan Licata
%   - The Linear Logic class
%   - Bernardo Toninho
%   - Henry De Young
%   - POP group, generically
% READERS: 
%   - Jason Reed
%   - Rowan Davies
%   - Chris Martens 
%   - Ian Zerny
%   - Jonathan Aldrich, Roger Wolff, and the three anonymous HOSC reviewers
%   - Three anonymous reviewers of the ICFP submission
%   - Five anonymous reviewers of the LICS submission
% PEOPLE:
%   - Rachel, Jen, Jamie, Lauren, Pete, Laurie, Carsten, Nawshin, Mary
%   - Thomas, Emily, Nick, and the Yinzerstars
% FAMILY: Mom, Dad, Elizabeth
% WORDS WITH FRIENDS: Cyrus, Mom, Jonas, Chris, Ed
% \end{acknowledgments}



\tableofcontents
\listoffigures % XXX Do I want this?
% \listoftables XXX Probably don't need this at all.

\mainmatter

%% Double space document for easy review:
%\renewcommand{\baselinestretch}{1.66}\normalsize

% The other requirements Catherine has:
%
%  - avoid large margins.  She wants the thesis to use fewer pages, 
%    especially if it requires colour printing.
%
%  - The thesis should be formatted for double-sided printing.  This
%    means that all chapters, acknowledgements, table of contents, etc.
%    should start on odd numbered (right facing) pages.
%
%  - You need to use the department standard tech report title page.  I
%    have tried to ensure that the title page here conforms to this
%    standard.
%
%  - Use a nice serif font, such as Times Roman.  Sans serif looks bad.
%
% Other than that, just make it look good...


% Introduction
\chapter{Introduction}
\label{chapter-introduction}

% The topic of this thesis is the specification of {\it evolving systems}

% The {\it lingua franca} of research in programming languages and
% logics is the {\it inductive definition}. Type systems are defined in
% terms of inductive definitions like $\Gamma \vdash e : \tau$
% (within the typing context $\Gamma$, the term $e$ has type $\tau$).
% % \[
% % \infer
% % {\Gamma \vdash x : \tau \mathstrut}
% % {x{:}\tau \in \Gamma \mathstrut}
% % \quad
% % \infer
% % {\Gamma \vdash {\sf z} : {\sf nat} \mathstrut}
% % {}
% % \quad
% % \infer
% % {\Gamma \vdash {\sf s}\,e : {\sf nat} \mathstrut}
% % {\Gamma \vdash e_1 : {\sf nat}
% %  &
% %  \Gamma \vdash e_2 : {\sf nat}}
% % \]
% Operational semantics are also defined in terms of inductive
% definitions: a {\it small-step} semantics has the form $e \mapsto e'$
% (the expression or machine state $e$ can transition to $e'$), and a
% {\it big-step} semantics has the form $e \Downarrow v$ (the expression
% or machine state $e$ can ultimately produce the terminal machine state
% $v$). 

% Proof assistants -- computer programs that help programming language
% researchers specify systems, explore their behavior, and prove
% properties of their behavior -- obviously must therefore be able to
% talk about inductive definitions. The most common way proof assistants
% do this is by directly incorporating a notion of inductive definition
% into their framework. Coq \cite{}, Agda \cite{}, ATS \cite{},
% Isabelle/HOL \cite{}, Matita \cite{}, and Abella \cite{} all work this
% way; inductive types are introduced

%  which covers all the theorem provers used to prove the POPLMark
% challenge except for Twelf.

% used in these domains therefore universally
% include a notion of {\it inductive definition} as the primary form of

Suppose you find yourself in possession of
\smallskip
\begin{itemize}
\item a calculator of unfamiliar design, or 
\item a new board game, or
\item the control system for an army of robots, or
\item an implementation of a security protocol, or
\item the interface to a high-frequency trading system.
\end{itemize}
\smallskip The fundamental questions are the same: {\it What does it
  do? What are the rules of the game?} The answer to this question,
whether it comes in the form of an instruction manual, a legal
document, or an ISO standard, is a {\it specification}.

Specifications must be {\it formal}, because any room for
misinterpretation could (respectively) lead to incorrect calculations,
accusations of cheating, a robot uprising, a security breach, or
bankruptcy. At the same time, specifications must be {\it clear}:
while clarity is obviously in the eye of the beholder, a specification
that one finds hopelessly confusing or complex is no more useful than
one that is hopelessly vague.
%
Clarity is what allows us to
communicate with each other, to use specifications to gain a common
understanding of what some system does and to think about how that
system might be changed. Formality is what allows specifications to
interact with the world of computers, to say with confidence that the
{\it implementation} of the calculator or high-frequency trading
system obeys the specification. Formality also allows specifications
to interact with the world of mathematics, and this, in turn, enables
us to make precise and accurate statements about what may or may not
happen to a given system.

The specification of many (too many!)~critical systems still remains
in the realm of English text, and the inevitable lack of formality can
and does make formal reasoning about these specifications difficult or
impossible.
%
Notably, this is true about most of the programming languages used to
implement our calculators, program our robot army control systems,
enforce our security protocols, and interact with our high-frequency
trading systems. In recent years, however, we have finally begun to
seen the emergence of operational semantics specifications (the
``rules of the game'' for a programming language) that are truly
formal. A notable aspect of this recent work is that the formalization
effort is not done simply for formalization's sake. Ellison and Ro{\c
  s}u's formal semantics of C can be used to check individual programs
for undefined behavior, unsafe situations where the rules of the game
no longer apply and the compiler is free to do anything, including
unleashing the robot army \cite{ellison12executable}. Lee, Crary, and
Harper's formalization of Standard ML has been used to formally prove
-- using a computer to check all the proof's formal details -- a much
stronger safety property: that {\it every} program accepted by the
compiler is free of undefined behavior \cite{lee07towards}.

Mathematics, by contrast, has a century-long tradition of insisting on
absolute formality (at least in principle: practice often falls far
short).
%
Over time, this tradition
has become a collaboration between practicing mathematicians and
practicing computer scientists, because while humans are reasonable
judges of clarity, computers have an absolutely superhuman patience
for checking all the formal details of an argument.  One aspect of
this collaboration has been the development of {\it logical
  frameworks}. In a logical framework, the language of specifications
is derived from the language of logic, which gives specifications in a
logical framework an independent meaning based on the logic from which
the logical framework was derived. To be clear, the language of logic
is not a single, unified entity: logics are formal systems that
satisfy certain internal coherence properties, and we study many of
them. For example, the logical framework Coq is based on the Calculus
of Inductive Constructions \cite{coq10coq}, the logical framework Agda
is based on a variant of Martin-L\"of's type theory called ${\sf
  UTT}_\Sigma$ \cite{norell08towards}, and the logical Twelf is based
off of the dependent type theory $\lambda^\Pi$, also known as LF
\cite{pfenning99system}. Twelf was the basis of Lee, Crary, and
Harper's formalization of Standard ML.  % Specifications evolve
% gradually from half-baked ideas scrawled on coffee-stained napkins to
% formal specifications encoded in a logical framework. Another critical
% component of a logical framework is a methodology or philosophy that
% guides this process. LF and Twelf, in particular, have a formal theory
% of {\it adequacy} that addresses the relationship between the on-paper
% artifacts the people use to communicate with each other and the
% encoding of those artifacts in LF \cite{harper93framework}.

Why is there not a larger tradition of formally specifying the
programming languages that people actually use? Part of the answer is
that most languages that people actually use have lots of features --
like mutable state, or exception handling, or synchronization and
communication, or lazy evaluation -- that are not particularly
pleasant to specify using existing logical frameworks. Dealing with a
few unpleasant features at a time might not be much trouble, but the
combinations that appear in actual programming languages cause formal
programming language specifications to be both unclear for humans to
read and inconvenient for formal tools to manipulate. A more precise
statement is that the addition of the aforementioned features is {\it
  non-modular}, because handling a new feature requires reconsidering
and revising the rest of the specification.  Some headway on this
problem has been made by frameworks like the K semantic framework that
are formal but not logically derived; the K semantic framework is
based on a notion of rewriting rules \cite{rosu10overview}. Ellison
and Ro\c{s}u's formalization of C was done in the K semantic
framework.

This thesis considers the specification of systems, particularly
programming languages, in logical frameworks. In particular, we
consider a particular family of logics, called {\it substructural
  logics}, in which logical propositions can be given an
interpretation as rewriting rules as detailed by Cervesato and Scedrov
\cite{cervesato09relating}. % Deriving a framework from substructural
% logics allows us to combine the formality and generality of logical
% frameworks with the modular specification that is possible in
% rewriting frameworks. With this synthesis, 
We seek to support the
following:
\begin{quote} {\bf Thesis Statement:} {\it Logical frameworks based on
    a rewriting interpretation of substructural logics are suitable
    for modular specification of programming languages and formal
    reasoning about their properties}.\footnote{The original thesis
    proposal used the phrase ``forward reasoning in substructural
    logics'' instead of the phrase ``a rewriting interpretation of
    substructural logics,'' but these are synonymous, as discussed in
    Section~\ref{sec:framework-logicprog}.}
\end{quote}

\noindent
Part~1 of the thesis concerns the design of logical frameworks that
support this rewriting interpretation and the design of the logical
framework \sls~in particular. Part~2 concerns the modular
specification of programming language features in \sls~and the
methodology by which we organize and relate styles of
specification. Part~3 discusses formal reasoning about properties of
\sls~specifications, with an emphasis on establishing invariants.

\section{Logical frameworks}

Many interesting stateful systems have a natural notion of {\it
  ordering} that is fundamental to their behavior. A very simple
example is a push-down automaton (PDA) that reads a string of symbols
left-to-right while maintaining and manipulating a separate stack of
symbols. We can represent any configuration of the PDA as a sequence
with three regions:
\[
[~\mbox{the stack}~]
~
[~\mbox{the head}~]
~
[~\mbox{the string being read}~]
\]
where the symbols closest to the head are the top of the stack and the
symbol waiting to be read from the string. If we represent the head as
a token ${\sf hd}$, we can describe the behavior (the rules of the
game) for a push-down automaton for checking a string for correct
nesting of angle braces by using two rewriting rules:
\begin{align}
\tag{push} {\sf hd}~{<} ~&\rightsquigarrow~ {<}~{\sf hd}
\\
\tag{pop} {<}~{\sf hd}~{>} ~&\rightsquigarrow~ {\sf hd}
\end{align}
The distinguishing feature of these rewriting rules is that they are
{\it local} -- they do not mention the entire stack or the entire
string, just the relevant fragment at the beginning of the string and
the top of the stack. Execution of the PDA on a particular string of
tokens then consists of (1) appending the token ${\sf hd}$ to the
beginning of the string, (2) repeatedly performing rewritings until no
more rewrites are possible, and (3) checking to see if only a single
token ${\sf hd}$ remains. One possible series of transitions that this
rewriting system can take is shown in Figure~\ref{fig:pda-transitions}

\begin{figure}
\begin{align*}
{\sf hd}~~{<}~~{<}~~{>}~~{<}~~{<}~~{>}~~{>}~~{>}
& ~~~\rightsquigarrow~~\\
{<}~~{\sf hd}~~{<}~~{>}~~{<}~~{<}~~{>}~~{>}~~{>}
& ~~~\rightsquigarrow~~\\
{<}~~{<}~~{\sf hd}~~{>}~~{<}~~{<}~~{>}~~{>}~~{>}
& ~~~\rightsquigarrow~~\\
{<}~~{\sf hd}~~{<}~~{<}~~{>}~~{>}~~{>}
& ~~~\rightsquigarrow~~\\
{<}~~{<}~~{\sf hd}~~{<}~~{>}~~{>}~~{>}
& ~~~\rightsquigarrow~~\\
{<}~~{<}~~{<}~~{\sf hd}~~{>}~~{>}~~{>}
& ~~~\rightsquigarrow~~\\
{<}~~{<}~~{\sf hd}~~{>}~~{>}
& ~~~\rightsquigarrow~~\\
{<}~~{\sf hd}~~{>}
& ~~~\rightsquigarrow~~\\
{\sf hd} &
\end{align*}
\caption{Series of PDA transitions}
\label{fig:pda-transitions}
\end{figure}

Because our goal is to use a framework that is both simple and
logically motivated, we turn to a substructural logic called {\it
  ordered logic} (originally presented by
Lambek~\cite{lambek58mathematics}) where hypotheses have an
intrinsic notion of order. The rewriting rules we considered above can
be explained as propositions in ordered logic, where the tokens ${\sf
  hd}$, $>$, and $<$ are all treated as {\it atomic propositions}:
\begin{align*}
{\sf push} &: ~~ {\sf hd} \fuse {<} ~\lefti~ \{ {<} \fuse {\sf hd} \}
\\ 
{\sf pop} &: ~~ {<} \fuse {\sf hd} \fuse {>} ~\lefti~ \{ {\sf hd} \}
\end{align*}
The symbol $\fuse$ (pronounced ``fuse'') is the binary connective for
ordered conjunction (i.e. concatenation); it binds more tightly than
$\lefti$, a binary connective for ordered implication. The curly
braces $\{ \ldots \}$ can be ignored for now.

Our logic has first-order quantification, so we can generically
describe a more general push-down automaton that uses ${\sf left}(X)$
and ${\sf right}(X)$ to describe left and right angle braces ($X =
{\sf an}$), square braces ($X = {\sf sq}$), and parentheses ($X = {\sf
  pa}$). The string {\sf [ \textless~\textgreater~( [ ] ) ]} is then
represented by the following sequence of ordered atomic propositions:
\[
{\sf 
  left(sq) ~~
  left(an) ~~
  right(an) ~~
  left(pa) ~~
  left(sq) ~~
  right(sq) ~~
  right(pa) ~~
  right(sq)
}
\]
The following rules describe the more general push-down automaton:
\begin{align*}
{\sf push} &: ~~ \forall x.\, 
  {\sf hd} \fuse {\sf left}(x) ~\lefti~ \{ {\sf stack}(x) \fuse {\sf hd} \}
\\ 
{\sf pop} &: ~~ \forall x.\, 
  {\sf stack}(x) \fuse {\sf hd} \fuse {\sf right}(x) ~\lefti~ \{ {\sf hd} \}
\end{align*}
(This specification would still be possible in propositional ordered 
logic; we would just need one copy of the ${\sf push}$ rule and one copy
of the ${\sf pop}$ rule for each pair of braces.)
Note that while we use the fuse connective to indicate adjacent tokens
in the rules above, no fuses appear in
Figure~\ref{fig:pda-transitions}. That is because the intermediate
states are not propositions in the same way rules are
propositions. Rather, the intermediate states in
Figure~\ref{fig:pda-transitions} are {\it contexts} in ordered logic,
which we will refer to as {\it process states}. 

The most distinctive characteristic of these transition systems is
that the intermediate stages of computation are encoded in the
structure of a substructural context (a process state). This general
idea dates back to Miller \cite{miller92pi} and his Ph.D. student
Chirimar \cite{chirimar95proof}, who encoded the intermediate states
of a $\pi$-calculus and of a low-level RISC machine (respectively) as
contexts in focused classical linear logic.  Part~1 of this thesis is
concerned with the design of logical frameworks for specifying
transition systems.  In this respect, this part of the thesis follows
in the footsteps of Miller's Forum \cite{miller96forum}, Cervesato and
Scedrov's multiset rewriting language $\omega$
\cite{cervesato09relating}, and Watkins et al.'s CLF
\cite{watkins02concurrent}. 

\begin{figure}
\[
\begin{array}{ll}
\multicolumn{2}{l}
{{\sf bool}\,{\sf function}\,{\it foo}(l)}\\
{\sf f_0}{:} & {\sf if}\,l\,{\sf then}\\
{\sf f_1}{:} & ~~ {\sf return}\,{\sf ff}\\
       & {\sf else}\\
{\sf f_2}{:} & ~~ {\sf return}\,{\sf tt}\\
       & {\sf fi}\\[15pt]
\\
\multicolumn{2}{l}
{{\sf procedure}\,{\it main}()}\\
\multicolumn{2}{l}
{{\sf global}\,b}\\
{\sf m_0}{:} & {\sf while}\,b\,{\sf do}\\
{\sf m_1}{:} & ~~ b := {\it foo}(b)\\
       & {\sf od}\\
{\sf m_2}{:} & {\sf return}\\[15pt]
\\
\end{array}
~
\begin{array}{|rcl|}
~ \langle b \rangle\,\langle {\sf tt}, {\sf f_0} \rangle 
  & \rightarrow & \langle b \rangle\,\langle {\sf tt}, {\sf f_1} \rangle\\
\langle b \rangle\,\langle {\sf ff}, {\sf f_0} \rangle
  & \rightarrow & \langle b \rangle\,\langle {\sf ff},{\sf f_2} \rangle\\
\langle b \rangle\,\langle l, {\sf f_1} \rangle 
  & \rightarrow & \langle {\sf ff} \rangle\\
\langle b \rangle\,\langle l, {\sf f_2} \rangle 
  & \rightarrow & \langle {\sf tt} \rangle\\
  & & \\[15pt]
  & & \\
  & & \\
\langle {\sf tt} \rangle\,\langle {\sf m_0} \rangle
  & \rightarrow & \langle {\sf tt} \rangle\, \langle {\sf m_1} \rangle\\
\langle {\sf ff} \rangle\,\langle {\sf m_0} \rangle
  & \rightarrow & \langle {\sf ff} \rangle\,\langle {\sf m_2} \rangle\\
\langle b \rangle\,\langle {\sf m_1} \rangle
  & \rightarrow & \langle b \rangle\,\langle b, {\sf f_0} \rangle\,
\langle{\sf m_0}\rangle  ~\\
\langle b \rangle\,\langle {\sf m_2} \rangle
  & \rightarrow & \epsilon\\
  & &\\[15pt]
  & &\\
  & &\\
\end{array}
~
\begin{array}{l}
{\sf gl}(b) \fuse {\sf foo}({\sf tt}, {\sf f_0}) \lefti ~\\
\quad \{ {\sf gl}(b) \fuse {\sf foo}({\sf tt}, {\sf f_1}) \}\\[5pt]
{\sf gl}(b) \fuse {\sf foo}({\sf ff}, {\sf f_0}) \lefti ~\\
\quad \{ {\sf gl}(b) \fuse {\sf foo}({\sf ff}, {\sf f_1}) \}\\[5pt]
{\sf gl}(b) \fuse {\sf foo}(l, {\sf f_1}) \lefti \{ {\sf gl}({\sf ff}) \}\\[5pt]
{\sf gl}(b) \fuse {\sf foo}(l, {\sf f_2}) \lefti \{ {\sf gl}({\sf tt}) \}\\
\\
{\sf gl}({\sf tt}) \fuse {\sf main}({\sf m_0}) \lefti ~\\
\quad \{ {\sf gl}({\sf tt}) \fuse {\sf main}({\sf m_1}) \}\\[5pt]
{\sf gl}({\sf ff}) \fuse {\sf main}({\sf m_0}) \lefti ~\\
\quad \{ {\sf gl}({\sf tt}) \fuse {\sf main}({\sf m_2})\}\\[5pt]
{\sf gl}(b) \fuse {\sf main}({\sf m_1}) \lefti ~\\
\quad \{ {\sf gl}(b) \fuse {\sf foo}(b,{\sf f_0}) \fuse {\sf main}({\sf m_0})\}\\[5pt]
{\sf gl}(b) \fuse {\sf main}({\sf m_2}) \lefti \{ \one \}
\end{array}
\]
\caption{A Boolean program, encoded as a rewriting system and in \sls}
\label{fig:canonical}
\end{figure}

As an extension to CLF, the logical framework we develop is able to
specify systems like the $\pi$-calculus, security protocols, and Petri
nets that can be encoded in CLF \cite{cervesato02concurrent}. The
addition of ordered logic allows us easily incorporate systems that
are naturally understood as string rewriting systems. An example from
the verification domain, taken from Bouajjani and Esparsa
\cite{bouajjani06rewriting}, is shown in Figure~\ref{fig:canonical}.
The left-hand side of the figure is a simple Boolean program: the
procedure ${\it foo}$ has one local variable and the procedure ${\it
  main}$ has no local variables but mentions a global variable $b$.
Bouajjani and Esparsa represented Boolean programs like this one as
{\it canonical systems} like the one shown in the middle of
Figure~\ref{fig:canonical}. Canonical systems are rewriting systems
where only the left-most tokens are ever rewritten: the left-most
token in this canonical system always has the form $\langle b \rangle$,
where $b$ is either true (${\sf tt}$) or false (${\sf ff}$),
representing the valuation of the global variables -- there is only 
one, $b$.  The token to the
right of the global variables contains the current program counter and
the value of the current local variables. The token to the right of
{\it that} contains the program counter and local variables of the calling
procedure, and so on, forming a call stack that grows off to the right
(in contrast to the PDA's stack, which grew off to the left). This
canonical system can be directly represented in ordered logic, as
shown on the right-hand side of Figure~\ref{fig:canonical}. The atomic
proposition ${\sf gl}(b)$ contains the global variables (versus
$\langle b \rangle$ in the middle column), the atomic proposition
${\sf foo}(l, f)$ contains the local variables and program counter
within the procedure ${\sf foo}$ (versus $\langle l, f \rangle$ in the
middle column), and the atomic proposition ${\sf main}(m)$ contains
the program counter within the procedure ${\sf main}$ (versus $\langle
m \rangle$ in the middle column).

The development of \sls, a CLF-like framework of \underline{\bf
  s}ubstructural \underline{\bf l}ogical \underline{\bf
  s}pecifications that includes an intrinsic notion of order, is a
significant development of Part 1 of the thesis.  However, the
principal contribution of these three chapters is the development of
{\it structural focalization} \cite{simmons11structural}, which
unifies Andreoli's work on focused logics \cite{andreoli92logic} with
the {\it hereditary substitution} technique that Watkins developed in
the context of CLF \cite{watkins02concurrent}. Chapter~2 explains
structural focalization in the context of linear logic, Chapter~3
establishes focalization for a richer substructural logic \ollll, and
Chapter~4 takes focused \ollll~and carves out the \sls~framework as a
fragment of the focused logic.

\section{Substructural operational semantics}
\label{sec:intro-ssos}

We are not primarily interested in representing systems like PDAs, and
while application to the verification domain like the rewriting
semantics of Boolean programs are an interesting application of \sls,
they will not be a focus of this thesis. Instead, in Part~2 of the
thesis, we will concentrate on specifying the operational semantics of
programming languages in \sls.
%
We can represent operational semantics in
\sls~in many ways, but we are particularly interested in a broad
specification style called {\it substructural operational semantics},
or SSOS
\cite{pfenning04substructural,pfenning09substructural}.\footnote{The
  term {\it substructural operational semantics} merges structural
  operational semantics \cite{plotkin04structural}, which we seek to
  generalize, and substructural logic, which forms the basis of our
  specification framework.} SSOS is a synthesis of structural
operational semantics, abstract machines, and logical specifications.

One of our running
examples will be a call-by-value operational semantics for the untyped
lambda calculus, defined by the BNF grammar:
\[
\obj{e} ::= \obj{x} \mid \obj{\lambda x.e} \mid \obj{e_1\,e_2}
\]
Taking some liberties with our representation of terms, we can
describe call-by-value evaluation for this language with the same
rewriting rules we used to describe the PDA and the Boolean program's
semantics. Our specification uses three atomic propositions: one,
${\sf eval}(\obj{e})$, carries an unevaluated expression $\obj{e}$,
and another, ${\sf retn}(\obj{v})$, carries an evaluated value
$\obj{v}$.  The third atomic proposition, ${\sf cont}(\obj{f})$,
contains a {\it continuation frame} $\obj{f}$ that represents some
partially evaluated value: $\obj{f} = \obj{\Box\,e_2}$ is state of an
expression $\obj{e_1\,e_2}$ waiting on the evaluation of $\obj{e_1}$
to a value, and $\obj{f} = \obj{(\lambda x.e)\,\Box}$ is the state of
an expression $\obj{(\lambda x.e)\,e_2}$ waiting on the evaluation of
$\obj{e_2}$ to a value. A list of these continuation frames forms a
stack that grows off to the right (like the Boolean program's stack
and unlike the PDA's stack).


The evaluation of a function is simple, as a function is already a
fully evaluated value, so we replace ${\sf eval}(\obj{\lambda x.e})$
in-place with ${\sf retn}(\obj{\lambda x.e})$:
\begin{align*}
{\sf ev/lam}&: ~~ 
  {\sf eval}\,(\obj{\lambda x.e})
      \lefti \{ {\sf retn}\,(\obj{\lambda x.e}) \}
%
      \intertext{The evaluation of an application $\obj{e_1\,e_2}$, on
        the other hand, requires us to push a new element onto the
        stack. We evaluate $\obj{e_1\,e_2}$ by evaluating $\obj{e_1}$
        and leaving behind a frame $\obj{\Box\,e_2}$ that suspends the
        argument $\obj{e_2}$ while $\obj{e_1}$ is being evaluated to a
        value.}
%
{\sf ev/app}&: ~~ 
  {\sf eval}\,(\obj{e_1\,e_2}) \lefti \{ {\sf eval}\,(\obj{e_1}) 
     \fuse {\sf cont}\,(\obj{\Box\,e_2}) \}
%
     \intertext{When a function is returned to a waiting $\obj{\Box\,e_2}$
       frame, we switch to evaluating the function argument while
       storing the returned function in a frame $\obj{(\lambda
       x.e)\,\Box}$.}
%
{\sf ev/app1}&: ~~
  {\sf retn}\,(\obj{\lambda x.e}) \fuse {\sf cont}\,(\obj{\Box\,e_2})
    \lefti \{ {\sf eval}\,(\obj{e_2})
      \fuse {\sf cont}\,(\obj{(\lambda x.e)\,\Box}) \}
%
    \intertext{Finally, when an evaluated function argument is
      returned to the waiting $\obj{(\lambda x.e)\,\Box}$ frame, we
      substitute the value into the body of the function and evaluate
      the result.}
%
{\sf ev/app2}&: ~~
  {\sf retn}\,(\obj{v_2}) \fuse {\sf cont}\,(\obj{(\lambda x.e})\,\Box)
    \lefti \{ {\sf eval}\,([\obj{v_2}/\obj{x}]\obj{e}) \}
%
\end{align*}

\begin{figure}
\begin{align*}
{\sf eval}\,(\obj{(\lambda x.x)\,((\lambda y.y)\,(\lambda z.e))}) 
& ~~~\rightsquigarrow~~ \tag{by rule ${\sf ev/app}$}\\
{\sf eval}\,(\obj{\lambda x.x}) \quad
{\sf cont}\,(\obj{\Box\,((\lambda y.y)\,(\lambda z.e))})
& ~~~\rightsquigarrow~~ \tag{by rule ${\sf ev/lam}$}\\
{\sf retn}\,(\obj{\lambda x.x}) \quad
{\sf cont}\,(\obj{\Box\,((\lambda y.y)\,(\lambda z.e))})
& ~~~\rightsquigarrow~~ \tag{by rule ${\sf ev/app1}$}\\
{\sf eval}\,(\obj{(\lambda y.y)\,(\lambda z.e)}) \quad
{\sf cont}\,(\obj{(\lambda x.x)\,\Box})
& ~~~\rightsquigarrow~~ \tag{by rule ${\sf ev/app}$}\\
{\sf eval}\,(\obj{\lambda y.y}) \quad
{\sf cont}\,(\obj{\Box\,(\lambda z.e)}) \quad
{\sf cont}\,(\obj{(\lambda x.x)\,\Box})
& ~~~\rightsquigarrow~~ \tag{by rule ${\sf ev/lam}$}\\
{\sf retn}\,(\obj{\lambda y.y}) \quad
{\sf cont}\,(\obj{\Box\,(\lambda z.e)}) \quad
{\sf cont}\,(\obj{(\lambda x.x)\,\Box})
& ~~~\rightsquigarrow~~ \tag{by rule ${\sf ev/app1}$}\\
{\sf eval}\,(\obj{\lambda z.e}) \quad
{\sf cont}\,(\obj{(\lambda y.y)\,\Box}) \quad
{\sf cont}\,(\obj{(\lambda x.x)\,\Box})
& ~~~\rightsquigarrow~~ \tag{by rule ${\sf ev/lam}$}\\
{\sf retn}\,(\obj{\lambda z.e}) \quad
{\sf cont}\,(\obj{(\lambda y.y)\,\Box}) \quad
{\sf cont}\,(\obj{(\lambda x.x)\,\Box})
& ~~~\rightsquigarrow~~ \tag{by rule ${\sf ev/app2}$}\\
{\sf eval}\,(\obj{\lambda z.e}) \quad
{\sf cont}\,(\obj{(\lambda x.x)\,\Box})
& ~~~\rightsquigarrow~~ \tag{by rule ${\sf ev/lam}$}\\
{\sf retn}\,(\obj{\lambda z.e}) \quad
{\sf cont}\,(\obj{(\lambda x.x)\,\Box})
& ~~~\rightsquigarrow~~ \tag{by rule ${\sf ev/app2}$}\\
{\sf eval}\,(\obj{\lambda z.e}) 
& ~~~\rightsquigarrow~~ \tag{by rule ${\sf ev/lam}$}\\
{\sf retn}\,(\obj{\lambda z.e}) 
& ~~~\not\rightsquigarrow~~ 
\end{align*}
\caption{SSOS evaluation of an expression to a value}
\label{fig:ssos-example}
\end{figure}

These four rules constitute an SSOS specification of call-by-value
evaluation; an example of evaluating the expression $\obj{(\lambda
x.x)\,((\lambda y.y)\,(\lambda z.e))}$ to a value under this
specification is given in Figure~\ref{fig:ssos-example}.  Again, each
intermediate state is represented by a process state or ordered
context.

The \sls~framework admits many styles of specification. The SSOS
specification above resides in the {\it concurrent} fragment of
\sls. (This rewriting-like fragment is called concurrent in part
because we can just as easily seed the process state with two
propositions ${\sf eval}(e)$ and ${\sf eval}(e')$ that will evaluate
to values concurrently, side-by-side in the process state.)  Even
within purely-concurrent specifications, there is a large design space
of potential SSOS specifications, a point that will be explained
further in Chapter~5.

On the other end of the spectrum, the {\it deductive} fragment of
\sls~supports the specification of inductive definitions by the same
methodology used to represent inductive definitions in the LF logical
framework \cite{harper93framework}.  We can therefore use the
deductive fragment of \sls~to specify a big-step operational semantics
for call-by-value evaluation by inductively defining the judgment $\obj{e
\Downarrow v}$, which expresses that the expression $\obj{e}$ evaluates to
the value $\obj{v}$. On paper, this big-step operational semantics is
expressed with two inference rules:
\[
\infer
{\obj{\lambda x.e \Downarrow \lambda x.e} \mathstrut}
{}
\quad
\infer
{\obj{e_1\,e_2 \Downarrow v} \mathstrut}
{\obj{e_1 \Downarrow \lambda x.e}
 &
 \obj{e_2 \Downarrow v_2}
 &
 \obj{{[\obj{v_2}/\obj{x}]\obj{e_2}} \Downarrow v} \mathstrut}
\]
Big-step operational semantics specifications are compact and elegant,
but they is not particularly {\it modular}. As a (rather contrived)
example, consider the addition of a incrementing counter $\obj{\sf
  count}$ to the language of expressions $\obj{e}$. The language keeps
a numeral as a counter, and every time $\obj{\sf count}$ is evaluated,
it returns the value of the counter and then increments the
counter.\footnote{To keep the language small, we can represent
  numerals $\obj{\underline{\obj{n}}}$ as Church numerals:
  $\obj{\underline{\obj{0}}} = \obj{(\lambda f. \lambda x. x)}$,
  $\obj{\underline{\obj{1}}} = \obj{(\lambda f. \lambda x. f x)}$,
  $\obj{\underline{\obj{2}}} = \obj{(\lambda f. \lambda x. f (f x))}$,
  and so on.  Then, $\obj{\underline{\obj{n}} + \obj{1}} =
  \obj{\lambda f. \lambda x. f e}$ if $\underline{\obj{n}} =
  \obj{\lambda f. \lambda x. e}$.}  To extend the big-step operational
semantics with this new feature, we have to revise all the existing
rules so that they mention the running counter:
%
\[
\infer
{\obj{({\sf count}, \underline{n}) \Downarrow 
  (\underline{n}, \underline{n} +1)} \mathstrut}
{}
\quad 
\infer
{\obj{(\lambda x.e, \underline{n}) \Downarrow (\lambda x.e, \underline{n})}
 \mathstrut}
{}
\]
\[
\infer
{\obj{(e_1\,e_2, \underline{n}) \Downarrow (v, \underline{n'})} \mathstrut}
{\obj{(e_1, \underline{n}) \Downarrow (\lambda x.e, \underline n_1)}
 &
 \obj{(e_2, \underline n_1) \Downarrow (v_2, \underline n_2)}
 &
 \obj{({[\obj{v_2}/\obj{x}]\obj{e_2}}, \underline n_2 ) \Downarrow (v, \underline{n'})} \mathstrut}
\]

\begin{figure}
\begin{align*}
{\sf store}\,\obj{\underline 5} \quad
{\sf eval}\,(\obj{((\lambda x.\lambda y.y)\,{\sf count})\,{\sf count}})
& ~~~\rightsquigarrow~~ \tag{by rule ${\sf ev/app}$}\\
{\sf store}\,\obj{\underline 5} \quad
{\sf eval}\,(\obj{(\lambda x.\lambda y.y)\,{\sf count}}) \quad
{\sf cont}\,(\obj{\Box\,{\sf count}})
& ~~~\rightsquigarrow~~ \tag{by rule ${\sf ev/app}$}\\
{\sf store}\,\obj{\underline 5} \quad
{\sf eval}\,(\obj{\lambda x.\lambda y.y}) \quad
{\sf cont}\,(\obj{\Box\,{\sf count}}) \quad
{\sf cont}\,(\obj{\Box\,{\sf count}})
& ~~~\rightsquigarrow~~ \tag{by rule ${\sf ev/lam}$}\\
{\sf store}\,\obj{\underline 5} \quad
{\sf retn}\,(\obj{\lambda x.\lambda y.y}) \quad
{\sf cont}\,(\obj{\Box\,{\sf count}}) \quad
{\sf cont}\,(\obj{\Box\,{\sf count}})
& ~~~\rightsquigarrow~~ \tag{by rule ${\sf ev/app1}$}\\
{\sf store}\,\obj{\underline 5} \quad
{\sf eval}\,(\obj{\sf count}) \quad
{\sf cont}\,(\obj{(\lambda x.\lambda y.y)\,\Box}) \quad
{\sf cont}\,(\obj{\Box\,{\sf count}})
& ~~~\rightsquigarrow~~ \tag{by rule ${\sf ev/count}$}\\
{\sf store}\,\obj{\underline 6} \quad
{\sf retn}\,(\obj{\underline 5}) \quad
{\sf cont}\,(\obj{(\lambda x.\lambda y.y)\,\Box}) \quad
{\sf cont}\,(\obj{\Box\,{\sf count}})
& ~~~\rightsquigarrow~~ \tag{by rule ${\sf ev/app2}$}\\
{\sf store}\,\obj{\underline 6} \quad
{\sf eval}\,(\obj{\lambda y.y}) \quad
{\sf cont}\,(\obj{\Box\,{\sf count}})
& ~~~\rightsquigarrow~~ \tag{by rule ${\sf ev/lam}$}\\
{\sf store}\,\obj{\underline 6} \quad
{\sf retn}\,(\obj{\lambda y.y}) \quad
{\sf cont}\,(\obj{\Box\,{\sf count}})
& ~~~\rightsquigarrow~~ \tag{by rule ${\sf ev/app2}$}\\
{\sf store}\,\obj{\underline 6} \quad
{\sf eval}\,(\obj{{\sf count}}) \quad
{\sf cont}\,(\obj{(\lambda y.y)\,\Box})
& ~~~\rightsquigarrow~~ \tag{by rule ${\sf ev/count}$}\\
{\sf store}\,\obj{\underline 7} \quad
{\sf retn}\,(\obj{\underline 6}) \quad
{\sf cont}\,(\obj{(\lambda y.y)\,\Box})
& ~~~\rightsquigarrow~~ \tag{by rule ${\sf ev/app2}$}\\
{\sf store}\,\obj{\underline 7} \quad
{\sf eval}\,(\obj{\underline 6}) 
& ~~~\rightsquigarrow~~ \tag{by rule ${\sf ev/lam}$}\\
{\sf store}\,\obj{\underline 7} \quad
{\sf retn}\,(\obj{\underline 6}) 
& ~~~\not\rightsquigarrow~~ 
\end{align*}
\caption{Evaluation with an imperative counter}
\label{fig:eval-ssos-ctr}
\end{figure}


The simple elegance of our big-step operational semantics has been
tarnished by the need to deal with state, and each new stateful
feature requires a similar revision.  In contrast, our SSOS
specification can tolerate the addition of a counter without revision
to the existing rules; we just store the counter's value in an atomic
proposition ${\sf store}(\obj{\underline{n}})$ to the left of the ${\sf
  eval}(\obj{e})$ or ${\sf retn}(\obj{v})$ proposition in the ordered
context. Because the rules ${\sf ev/lam}$, ${\sf ev/app}$, ${\sf
  ev/app1}$, and ${\sf ev/app2}$ are local, they will ignore this
extra proposition, which only needs to be accessed by the rule ${\sf
  ev/count}$.
\begin{align*}
{\sf ev/count} &:~~
  {\sf store}\,\obj{\underline n} \fuse {\sf eval}\,\obj{{\sf count}}
    \lefti \{ {\sf store}\,(\obj{\underline n + 1}) 
      \fuse {\sf retn}\,\obj{\underline n} \}
\end{align*}
In Figure~\ref{fig:eval-ssos-ctr}, we give an example of evaluating
$\obj{(((\lambda x.\lambda y.y)\,{\sf count})\,{\sf count})}$ to a value
with a starting counter value of $\obj{\underline 5}$. This solution is
even more contrived than the problem was: if we wanted to introduce a
{\it second} counter, where would it go? Nevertheless, the example
does foreshadow how, in Part~2 of this thesis, we will show that SSOS
specifications in \sls~allow for the modular specification of many
programming language features.


An overarching theme of Part~2 is that we can have our cake and eat it
too by deploying the {\it logical correspondence}, an idea that was
developed jointly with Ian Zerny in unpublished work and is explained
in Chapter~5. As explained in Chapter~6, we can use logical
correspondence to directly connect the big-step semantics and SSOS
specifications above; in fact, we can automatically and mechanically
derive the latter from the former. Therefore, we can specify
call-by-value evaluation as a big-step semantics, and then transform
it for the purpose of modular extension. Further transformations,
developed in joint work with Pfenning \cite{simmons11logical}, create
new opportunities for modular extension; this is the topic of
Chapter~7. These transformations also allow for the derivation of
abstract analyses (such as control flow and alias analysis) from SSOS
specifications, which is the focus of Chapter~8.

\section{Invariants in substructural logic}

{\it Invariants} are fundamental for reasoning about any evolving
system. From a verification or model-checking viewpoint, invariants
are usually expressed in terms of temporal logics like LTL and are
usually verified by model checking or exhaustive search. Existing work
on verifying properties of rewriting specifications has generally
approached the problem with model checking. 

In Part~3 we offer a approach to invariants that is complementary to
the model checking approach.  From a programming languages
perspective, invariants are often associated with {\it types}. Type
invariants are well-formedness criteria on programs that are weak
enough to be preserved by state transitions (a property called {\it
  preservation}) but strong enough to allow us to express the property
we expect to hold of all well-formed program states. In systems free
of deadlock, a common property we want to hold is {\it progress} -- a
well-typed state is either final can evolve to some other state with a
state transition. (Even in systems where deadlock is a possibility,
progress can be handled by stipulating that a deadlocked state is
final.)

Chapter~9 discusses the use of {\it generative signatures} to
describe well-formedness invariants of specifications. Generative
signatures look a bit like context-free grammars, and they allow us to
characterize contexts by a describing rewriting rules that generate
process states in the same way that context-free grammars characterize
grammatical strings by describing rules that generate all grammatical
strings.

In our example SSOS specification, a process state that consists of
only a single ${\sf retn}(\obj{v})$ proposition is final, and our
well-formed state is any one that contains an atomic proposition ${\sf
  eval}(\obj{e})$ (where $\obj{e}$ is a closed expression) or ${\sf
  eval}(\obj{\lambda x. e})$ (where $\obj{\lambda x. e}$ is a closed
expression) containing a ${\sf eval}(\obj{e})$ to the left 
of a series of continuation frames ${\sf cont}(\obj{\Box\,e})$ 
or ${\sf cont}(\obj{(\lambda x.e)\,\Box})$. We can characterize 
all such states as being generated from an initial atomic 
proposition ${\sf gen\_state}$ under the following generative
signature:
\begin{align*}
{\sf gen/eval} &:~~
  {\sf gen\_state} \lefti \{ {\sf eval}(\obj{e}) \}
\\
{\sf gen/retn} &:~~
  {\sf gen\_state} \lefti \{ {\sf retn}(\obj{\lambda x.e}) \}
\\
{\sf gen/app1} &:~~
  {\sf gen\_state} 
    \lefti \{ {\sf gen\_state} \fuse {\sf cont}(\obj{\Box\,e_2}) \}
\\
{\sf gen/app2} &:~~
  {\sf gen\_state} 
    \lefti \{ {\sf gen\_state} \fuse {\sf cont}(\obj{(\lambda x.e)\,\Box}) \}
\end{align*}
An illustration of the derivation of one of the intermediate process
states from Figure~\ref{fig:ssos-example} is shown in
Figure~\ref{fig:ssos-gen}.

\begin{figure}
\begin{align*}
{\sf gen\_state}
& ~~~\rightsquigarrow~~ \tag{by rule ${\sf gen/app2}$}\\
{\sf gen\_state} \quad
{\sf cont}\,(\obj{(\lambda x.x)\,\Box})
& ~~~\rightsquigarrow~~ \tag{by rule ${\sf gen/app1}$}\\
{\sf gen\_state} \quad
{\sf cont}\,(\obj{\Box\,(\lambda z.e)}) \quad
{\sf cont}\,(\obj{(\lambda x.x)\,\Box})
& ~~~\rightsquigarrow~~ \tag{by rule ${\sf gen/retn}$}\\
{\sf retn}\,(\obj{\lambda y.y}) \quad
{\sf cont}\,(\obj{\Box\,(\lambda z.e)}) \quad
{\sf cont}\,(\obj{(\lambda x.x)\,\Box})
& ~~~\not\rightsquigarrow~~ 
\end{align*}
\caption{Proving well-formedness of one of the states from
  Figure~\ref{fig:ssos-example}}
\label{fig:ssos-gen}
\end{figure}

Well-formedness is a global property of specifications, so if we add
more state to the specification we have to change the description of
what counts as a final state and extend the grammar of well-formed
process states. In the case of our counter extension, final states
have a single ${\sf store}(\obj{e})$ proposition to the left of a
single ${\sf retn}(\obj{v})$ proposition, and well-formed states are
generated from an initial atomic proposition ${\sf gen}$ under the
following extension to the previous generative signature:
%
\begin{align*}
{\sf gen/all} &:~~ {\sf gen} \lefti \{ {\sf gen\_store} \fuse {\sf
  gen\_state} \}
\\
{\sf gen/store} &:~~ {\sf gen\_store} \lefti \{ {\sf
  store}(\obj{\underline n}) \}
\end{align*}

It is possible to prove more specific invariants of our SSOS
specification. For example, we can present a type system for the
simply-typed lambda calculus and show that the type of a computation
is always preserved. (Types will be considered in Chapter~9.) However,
the generative signature above is sufficient for proving the progress
theorem for our example SSOS specification, which means that our
specification is safe -- that is, free from undefined behavior.
Chapter~10 will consider the use of generative invariants for
proving safety properties of specifications.


\part{Focusing substructural logics}

% Linear logic

\chapter{Linear logic}

Logic as it has been traditionally understood and studied -- both in
its classical and intuitionistic varieties -- treats the truth of a
proposition as a {\it persistent resource}. That is, if we have
evidence for the truth of a proposition, we can ignore that evidence
if it is not needed and reuse the evidence as many times as we need
to. Throughout this thesis, ``logic as it has been traditionally
understood as studied'' will be referred to as {\it persistent} logic
to emphasize this treatment of evidence. 

Linear logic, which was studied and
popularized by Girard \cite{girard87linear},
treats evidence as an {\it ephemeral} resource; the use of an
ephemeral resource consumes it, at which point it is unavailable for
further use.  Linear logic, like persistent logic, comes in classical
and intuitionistic flavors. We will favor intuitionistic linear logic
in part because the propositions of intuitionistic linear logic
(written $A$, $B$, $C$, \ldots) have a more natural correspondence
with our physical intuitions about consumable resources. Linear
conjunction $A \tensor B$ represents the resource built from the
resources $A$ and $B$; if you have both a bowl of soup {\it and} a
sandwich, that resource can be represented by the proposition ${\sf
  soup} \otimes {\sf sandwich}$. Linear implication $A \lolli B$
represents a resource that can interact with another resource $A$ to
produce a resource $B$. One robot with batteries not included could be
represented as the linear resource $({\sf battery} \lolli {\sf
  robot})$, and the linear resource $({\sf 6bucks} \lolli {\sf soup}
\tensor {\sf sandwich})$ represents the ability to use \$6 to obtain
lunch -- but only once!\footnote{Conjunction will always bind more
  tightly than implication, so this is equivalent to the proposition
  ${\sf 6bucks} \lolli ({\sf soup} \tensor {\sf sandwich})$.} Linear
logic also has a modal connective ${!}A$ representing a persistent
resource that can be
used to generate any number of $A$ resources, including zero. The
Panera ``You Pick Two'' menu might be represented as
\[ {!}({\sf 6bucks} \lolli {\sf soup} \tensor {\sf sandwich}) \otimes
{!}({\sf 6bucks} \lolli {\sf soup} \tensor {\sf salad}) \otimes
{!}({\sf 6bucks} \lolli {\sf sandwich} \tensor {\sf salad}),\] as the
menu gives you the opportunity to exchange six dollars for two
distinct members of the set $\{ {\sf soup}, {\sf salad}, {\sf
  sandwich} \}$ any number of times.

\begin{figure}[t]
\begin{tabbing}
\quad $A$ \,\, \=  $::= p \mid {!}A \mid A \lolli B \mid \one \mid A \tensor B$\\
\quad $\Gamma$ \> $::= \cdot \mid \Gamma, A$ \qquad \= {\it (multiset)}\\
\quad $\Delta$ \> $::= \cdot \mid \Delta, A$ \> {\it (multiset)}\\
\end{tabbing}
%
%
\quad \fbox{$\seq{\Gamma}{\Delta}{A}$}
\[
\infer[{\it init}]
{\seq{\Gamma}{p}{p}}
{}
\qquad
\infer[{\it copy}]
{\seq{\Gamma, A}{\Delta}{C}}
{\seq{\Gamma, A}{\Delta, A}{C}}
%
\]

\[
%
\infer[{!}_R]
{\seq{\Gamma}{\cdot}{{!}A}}
{\seq{\Gamma}{\cdot}{A}}
\qquad
\infer[{!}_L]
{\seq{\Gamma}{\Delta, {!}A}{C}}
{\seq{\Gamma, A}{\Delta}{C}}
\qquad
\infer[\one_R]
{\seq{\Gamma}{\cdot}{\one}}
{}
\qquad
\infer[\one_L]
{\seq{\Gamma}{\Delta, \one}{C}}
{\seq{\Gamma}{\Delta}{C}}
\]

\[
%
\infer[{\tensor}_R]
{\seq{\Gamma}{\Delta_1,\Delta_2}{A \tensor B}}
{\seq{\Gamma}{\Delta_1}{A}
 &
 \seq{\Gamma}{\Delta_2}{B}}
\qquad
\infer[{\tensor}_L]
{\seq{\Gamma}{\Delta, A \tensor B}{C}}
{\seq{\Gamma}{\Delta, A, B}{C}}
\]

\[
%
\infer[{\lolli}_R]
{\seq{\Gamma}{\Delta}{A \lolli B}}
{\seq{\Gamma}{\Delta, A}{B}}
\qquad
\infer[{\lolli}_L]
{\seq{\Gamma}{\Delta_1,\Delta_2, A \lolli B}{C}}
{\seq{\Gamma}{\Delta_1}{A}
 &
 \seq{\Gamma}{\Delta_2, B}{C}}
%
\]
\caption{Intuitionstic linear logic}
\label{fig:linear}
\end{figure}


Figure~\ref{fig:linear} presents a standard sequent calculus for
linear logic, in particular the so-called {\it multiplicative,
  exponential} fragment of intuitionistic linear logic (or {\it
  MELL}). It corresponds most closely to Barber's dual intuitionistic
linear logic \cite{barber96dual}, but also to Andreoli's dyadic system
\cite{andreoli92logic} and Chang et al.'s judgmental analysis of
intuitionistic linear logic \cite{chang03judgmental}.

\subsection*{Transitions in linear logic}

The propositions of intuitionistic linear logic, and linear implication
in particular, capture a notion of state change: we can {\it
  transition} from a state where we have both a ${\sf battery}$ and
the battery-less robot (represented, as before, by the linear
implication ${\sf battery} \lolli {\sf robot}$) to a state where we
have the battery-endowed (and therefore presumably functional) robot
(represented by the proposition ${\sf robot}$). In other words, the
proposition
%
\[{\sf battery} \otimes ({\sf battery} \lolli {\sf robot}) \lolli
{\sf robot}\] 
%
is provable in linear logic. These transitions can be chained
together as well: if we start out with ${\sf
  6bucks}$ instead of ${\sf battery}$ but we also have the
persistent ability to turn ${\sf 6bucks}$ into a ${\sf battery}$ --
just like we turned \$6 into a bowl of soup and a salad at Panera --
then we can ultimately get our working robot as well.
Written as a series of transitions, the picture looks like this:
\[
\begin{array}{ccccc}
\begin{array}{c}
\mbox{\it \$6 (1)}\medskip\\ 
\mbox{\it battery-free robot (1)} \medskip\\ 
\mbox{\it turn \$6 into a battery}\\
\mbox{\it (all you want)}
\end{array}
& \leadsto &
\begin{array}{c}
\mbox{\it battery  (1)}\medskip\\ 
\mbox{\it battery-free robot (1)} \medskip\\ 
\mbox{\it turn \$6 into a battery}\\
\mbox{\it (all you want)}
\end{array}
& \leadsto &
\begin{array}{c}
\mbox{\it robot (1)} \medskip\\ 
\mbox{\it turn \$6 into a battery}\\
\mbox{\it (all you want)}\medskip\\~\\
\end{array}
\end{array}
\]
In linear logic, these transitions correspond to the provability
of the proposition
\[{!}({\sf 6bucks} \lolli {\sf battery}) \otimes {\sf 6bucks} \otimes
({\sf battery} \lolli {\sf robot}) \lolli {\sf robot}.\] 
A derivation of this proposition is given in
Figure~\ref{fig:unfocused-robot}.\footnote{In Chapter XXX, I will
  argue that this view isn't quite precise enough, and that the most
  natural representation of state change from the state $A$ to the
  state $B$ isn't really captured by derivations of the proposition $A
  \lolli B$ or by derivations of the hypothetical judgment
  $\seq{\cdot}{A}{B}$.  However, this view remains a simple and useful
  one; Cervesato and Scedrov cover it thoroughly in the context of
  intuitionistic linear logic \cite{cervesato09relating}.}  

\begin{figure}
\[
\infer[{\lolli}_R]
{\seq{\cdot}{\cdot}{{!}({\sf 6bucks} \lolli {\sf battery}) \otimes
                    {\sf 6bucks} \otimes 
                    ({\sf battery} \lolli {\sf robot}) \lolli {\sf robot}}}
{\infer[{\otimes}_L]
{\seq{\cdot}{{!}({\sf 6bucks} \lolli {\sf battery}) \otimes
                    {\sf 6bucks} \otimes 
                    ({\sf battery} \lolli {\sf robot})}{{\sf robot}}}
{\infer[{!}_L]
{\seq{\cdot}{{!}({\sf 6bucks} \lolli {\sf battery}),
                    {\sf 6bucks} \otimes 
                    ({\sf battery} \lolli {\sf robot})}{{\sf robot}}}
{\infer[{\otimes}_L]
{\seq{\Gamma}{{\sf 6bucks} \otimes 
                    ({\sf battery} \lolli {\sf robot})}{{\sf robot}}}
{\infer[{\lolli}_L]
{\seq{\Gamma}{{\sf 6bucks}, {\sf battery} \lolli {\sf robot}}{{\sf robot}}}
{\infer[{\it copy}]
 {\seq{\Gamma}{{\sf 6bucks}}{{\sf battery}}}
 {\infer[{\lolli}_L] 
  {\seq{\Gamma}{{\sf 6bucks}, {\sf 6bucks} \lolli {\sf battery}}{{\sf battery}}}
  {\infer[{\it init}]
   {\seq{\Gamma}{{\sf 6bucks}}{{\sf 6bucks}}}
   {}
   &
   \infer[{\it init}]
   {\seq{\Gamma}{{\sf battery}}{{\sf battery}}}
   {}}}
 &
 \infer[{\it init}]
 {\seq{\Gamma}{{\sf robot}}{{\sf robot}}}
 {}}}}}}
\] 
\caption{Proving that a transition is possible 
(where we let $\Gamma = {\sf 6bucks} \lolli {\sf battery}$)}
\label{fig:unfocused-robot}
\end{figure}


It is precisely because linear logic contains this natural notion of
state and state transition that a rich line of work, dating back to
Chirmar's 1995 Ph.D. thesis, has sought to use linear logic as a {\it
  logical framework} for describing stateful systems
\cite{chirimar95proof,cervesato02linear,
  cervesato02concurrent,pfenning04substructural,miller09formalizing,
  pfenning09substructural,cervesato09relating}.  

\subsection*{Logical frameworks}

Generally speaking, logical frameworks use the {\it structure} of
proofs in a logic (like linear logic) to describe the structures we're
really interested in (like the process of obtaining a robot).  There
are two related reasons why linear logic as described in
Figure~\ref{fig:linear} is not immediately useful as a logical
framework. First, the structure of the proof in
Figure~\ref{fig:unfocused-robot} doesn't really match the intuitive
two-step transition that we sketched out above. Second, there are {\it
  lots} of derivations of our example proposition according to the
rules in Figure~\ref{fig:linear}, even though there's only one
``real'' series of transitions that get us to a working robot. The use
of ${!}L$, for instance, could be permuted up past the ${\otimes}L$
and then past the ${\lolli}L$ into the left branch of the proof. These
differences represent inessential nondeterminism in proof construction
or in proof search -- they just get in the way of the structure that
we are trying to capture. 

This is a general problem in the construction of logical frameworks,
and we'll discuss two solutions in the context of LF, a logical
framework based on dependent type theory that has proved to be a
suitable means of encoding a wide variety of deductive systems, such
as logics and programming languages \cite{harper93framework}.  The
first solution is to define an appropriate equivalence class of
proofs, and the second solution is to define an appropriate fragment
of canonical proofs.

Using an appropriate equivalence class of proofs can be an effective
way of defining away the problem of inessential nondeterminism.  In
linear logic as presented above, if the permutability of rules like
${!}_L$ and ${\otimes}_L$ is problematic, we can instead reason about
{\it equivalence classes} of derivations where proofs that differ only
in the ordering of ${!}_L$ and ${\otimes}_L$ rules are treated as
equivalent (that is, as members of the same equivalence class):
\[
\infer[{!}_L]
{\seq{\Gamma}{\Delta, {!}A, B \otimes C}{D}}
{\infer[{\otimes}_L]
 {\seq{\Gamma,A}{\Delta, B \otimes C}{D}}
 {\deduce{\seq{\Gamma,A}{\Delta, B, C}{D}}{\mathcal D}}}
\quad
\deduce{\mathstrut}{\mathstrut{\equiv}}
\quad
\infer[{\otimes}_L]
{\seq{\Gamma}{\Delta, {!}A, B \otimes C}{D}}
{\infer[{!}_L]
 {\seq{\Gamma}{\Delta, {!}A, B, C}{D}}
 {\deduce{\seq{\Gamma,A}{\Delta, B, C}{D}}{\mathcal D}}}
\]

In LF, lambda calculus terms (which correspond to derivations by the
Curry-Howard) are considered modulo the least equivalence class that
includes
\begin{itemize}
\item $\alpha$-equivalence ($\lambda x.N \equiv \lambda y.N[y/x]$ if 
$y \not\in {\it FV}(N)$), 
\item $\beta$-equivalence 
($(\lambda x.\,M)N \equiv M[N/x]$ if $x \not\in {\it FV}(N)$), and 
\item $\eta$-equivalence ($N \equiv \lambda x.N\,x$).
\end{itemize}
The weak normalization property for LF establishes that given any
typed LF term, we can find an equivalent term that is $\beta$-normal
(no $\beta$-redexes of the form $(\lambda x.M) N$ exist) and
$\eta$-long (replacing $N$ with $\lambda x.N\,x$ anywhere would
introduce a $\beta$-redex or make the term ill-typed).  Furthermore,
in any given equivalence class of typed LF terms, all the
$\beta$-normal and $\eta$-long terms are $\alpha$-equivalent.
Therefore, because $\alpha$-equivalence is decidable, the equivalence
of typed LF terms is also decidable. 

The uniqueness of $\beta$-normal and $\eta$-long terms within an
equivalence class of lambda calculus terms (modulo
$\alpha$-equivalence, which we will henceforth take for granted) makes
these terms useful as canonical representatives of equivalence
classes. In Harper, Honsell, and Plotkin's original formulation
of LF, a deductive system was said to be {\it adequately encoded} as
an LF type family in the case that there is a compositional bijection
between the formal objects in the deductive system and these
$\beta$-normal, $\eta$-long representatives of equivalence classes
\cite{harper93framework}.

More modern presentations of LF, such as Harper and Licata's
\cite{harper07mechanizing}, follow the approach developed by Watkins
et al.~\cite{watkins02concurrent} and define the logical framework so
that it only contains these $\beta$-normal, $\eta$-long {\it canonical
  forms} of LF. This presentations of LF is called Canonical LF to
distinguish it from the original presentation of LF in which the
canonical forms are just a subset of the possible terms. A central
component in this approach is {\it hereditary substitution}.
Hereditary substitution also establishes a normalization property for
LF; using hereditary substitution we can easily take a regular LF
term and transform it into a Canonical LF term.\footnote{This process
  is the same as the way we use cut admissibility to prove cut
  elimination.} An oft-overlooked point, which we will return to in
Section~\ref{sec:warning}, is that the normalization theorem we prove
this way is a strictly weaker theorem than so-called weak
normalization.

One analogue to the canonical forms of LF will be the {\it focused
  derivations} of linear logic that are presented in the next
section. In Section~\ref{sec:foclinlog} below, we will present 
focused linear logic and see that there is exactly 
one focused derivation that derives the proposition
\[{!}({\sf 6bucks} \lolli {\sf battery}) \otimes {\sf 6bucks} \otimes
({\sf battery} \lolli {\sf robot}) \lolli {\sf robot}.\] 
%
We will furthermore see that the structure of this derivation matches
the intuitive transition interpretation of the proposition, a point
that is reinforced by the discussion of {\it synthetic inference
  rules} in Section~\ref{sec:linsynthetic}. In
Section~\ref{sec:linhack}, we argue that our focused system, while it
may be the most natural one for linear logic, does not precisely meet
the demands we will place upon it. This, in turn, motivates a
discussion of notation (Section~\ref{sec:linnote}) 
which we will continue in the next chapter.

\section{Focused linear logic}
\label{sec:foclinlog}

Andreoli's original motivation for introducing focusing was not to
describe a logical framework, it was to describe a paradigm of logic
programming based on proof search in classical linear logic
\cite{andreoli92logic}. The existence of multiple proofs that differ
in inessential ways is particularly problematic for proof search, as
inessential differences between derivations correspond to unnecessary
choice points that a proof search procedure will need to backtrack
over. The presentation of focusing for intuitionistic linear logic in
this section most closely resembles Chaudhuri's focused intuitionistic
linear logic \cite{chaudhuri06focused} and my presentation of
polarized intuitionistic persistent logic
\cite{simmons11structural}. The major exception is the treatment of
asynchronous rules as confluent rather than fixed and arbitrary
(discussed in Section~\ref{sec:confluent-v-fixed}).

\subsection{Polarization}
\label{sec:linpolar}

The first step in describing a focused sequent calculus is to classify
connectives into two groups.  Some connectives, such as linear
implication $A \lolli B$, are called {\it asynchronous} because their
right rules can always be applied eagerly, without backtracking,
during bottom-up proof search. Other connectives, such as disjunction
$A \tensor B$, are called {\it synchronous} because their right rules
cannot be applied eagerly. For instance, if we are trying to prove
$\seq{\Gamma}{A \tensor B}{B \tensor A}$, the ${\tensor}R$ rule cannot
be applied eagerly; we first have to decompose $A \tensor B$ on the
left using the ${\tensor}L$ rule.\footnote{Andreoli dealt with a
  one-sided classical sequent calculus; in intuitionistic logics, it
  is common to call asynchronous connectives {\it right}-asynchronous
  and {\it left}-synchronous. Similarly, it is common to call
  synchronous connectives {\it right}-synchronous and {\it
    left}-asynchronous.

  Synchronicity, a property of connectives, is closely connected to
  (and sometimes conflated with) a property of rules called {\it
    invertibility}; a rule is invertible if the conclusion of the rule
  implies the premises. So ${\lolli}R$ is invertible
  ($\seq{\Gamma}{\Delta}{A \lolli B}$ implies $\seq{\Gamma}{\Delta,
    A}{B}$) but ${\lolli}L$ is not ($\seq{\Gamma}{\Delta, A \lolli
    B}{C}$ does not imply that $\Delta = \Delta_1, \Delta_2$ such that
  $\seq{\Gamma}{\Delta_1}{A}$ and $\seq{\Gamma}{\Delta_2, B}{C}$).
  Rules that can be applied eagerly need to be invertible, so
  asynchronous connectives have invertible right rules and synchronous
  connectives have invertible left rules. Therefore, another synonym
  for asynchronous/negative is {\it right-invertible}, and another
  synonym for synchronous/positive is {\it left-invertible}.}  
The nontrivial result of focusing is that it is possible to separate a
proof into phases: inversion phases in which all asynchronous rules
are applied exhaustively, and focused phases where synchronous rules
are applied repeatedly and exhaustively to a single proposition (the
proposition {\it in focus}). 

We call the asynchronous connectives {\it negative} ($\lolli$, $\top$
and $\with$ in full propositional linear logic) and call the
synchronous connectives {\it positive} ($\zero$, $\oplus$, $\one$, and
$\otimes$ in full propositional linear logic). Each atomic proposition
must be assigned to be either positive or negative, though this
assignment can be arbitrary. At this point, there is an important
choice to make. One way forward is to treat positive and negative
propositions as a syntactic refinements of all propositions, in which
case we end up focusing a standard intuitionistic linear logic. The
other way forward is to treat positive and negative propositions as
distinct syntactic classes $A^+$ and $A^-$ with explicit inclusions
between them. In this second case, we end up focusing a {\it
  polarized} linear logic.  These inclusions are traditionally called
{\it shifts}. The positive proposition ${\downarrow}A^-$, pronounced
``downshift $A$,'' has a subterm that is a negative proposition; the
negative proposition ${\uparrow}A^+$, pronounced ``upshift $A$,'' has
a subterm that is a positive proposition.

The choice doesn't make a large difference for our purposes.
Polarized logics are interesting, and polarized linear logic is a bit
more expressive than regular linear logic, as heavily-shifted
propositions like ${\uparrow}{\downarrow}{\uparrow}{\downarrow}A^-$
can be expressed. This extra expressiveness won't help us in
the design of logical frameworks, but the use of shifts is helpful
when explaining identity expansion in Section~\ref{sec:linindentity}, 
so we will focus a polarized linear logic with shifts.

\begin{figure}
{\small \[
\begin{array}{rcl|rcl|rcl}
({\downarrow}A^-)^\circ & \!\!\!=\!\!\! & (A^-)^\circ & & & & & & 
\\
(p^+)^\circ & \!\!\!=\!\!\! & p^+ &
(p^+)^\oplus & \!\!\!=\!\!\! & p^+ &
(p^+)^\ominus & \!\!\!=\!\!\! & {\uparrow}p^+
\\
({!}A^-)^\circ & \!\!\!=\!\!\! & {!}(A^-)^\circ &
({!}A)^\oplus & \!\!\!=\!\!\! & {!}A^\ominus &
({!}A)^\ominus & \!\!\!=\!\!\! & {\uparrow}({!}A^\ominus)
\\
(\one)^\circ & \!\!\!=\!\!\! & \one &
(\one)^\oplus & \!\!\!=\!\!\! & \one &
(\one)^\ominus & \!\!\!=\!\!\! & {\uparrow}\one 
\\
(A^+ \otimes B^+)^\circ & \!\!\!=\!\!\! & (A^+)^\circ \otimes (B^+)^\circ &
(A \otimes B)^\oplus & \!\!\!=\!\!\! & A^\oplus \otimes B^\oplus &
(A \otimes B)^\ominus & \!\!\!=\!\!\! & {\uparrow}(A^\oplus \otimes B^\oplus)
\\
({\uparrow}A^+)^\circ & \!\!\!=\!\!\! & (A^+)^\circ & & & & & & 
\\
(p^-)^\circ & \!\!\!=\!\!\! & p^- &
(p^-)^\oplus & \!\!\!=\!\!\! & {\downarrow}p^- &
(p^-)^\ominus & \!\!\!=\!\!\! & p^- 
\\
(A^+ \lolli B^-)^\circ & \!\!\!=\!\!\! & (A^+)^\circ \lolli (B^-)^\circ &
(A \lolli B)^\oplus & \!\!\!=\!\!\! & {\downarrow}(A^\oplus \lolli B^\ominus) &
(A \lolli B)^\ominus & \!\!\!=\!\!\! & A^\oplus \lolli B^\ominus
\end{array}
\]}

\caption{De-polarizing and polarizing (with minimal shifts) propositions of MELL}
\label{fig:lin-shift}
\end{figure}


The relationship between unpolarized and polarized linear logic is
given by two erasure functions $(A^+)^\circ$ and $(A^-)^\circ$ that
wipe away all the shifts; this function is defined in
Figure~\ref{fig:lin-shift}. While shifts turn out to have a profound
impact on the structure of focused proofs, they are intended to have
no impact on provability. Therefore, the strongest statement of the
correctness of focusing is based on erasure: there is an unfocused
derivation of $(A^-)^\circ$ if and only if there is a focused
derivation of $A^-$.\footnote{I chose $A^-$ only to be brief; the
  condition that $(A^+)^\circ$ is derivable iff $A^+$ is could, of
  course, be added.}  However, most proofs of the correctness of
focusing prove a weaker property. Every proposition in linear logic
has an obvious polarized analogue with a minimal number of shifts;
this analogue is formalized as the two functions $A^\oplus$ and
$A^\ominus$ in Figure~\ref{fig:lin-shift}. Note that both of these
functions are partial inverses of erasure: $(A^\oplus)^\circ =
(A^\ominus)^\circ = A$. Almost all proofs of the correctness of
focusing work on the basis of these partial inverses, which we call
{\it polarization strategies}, establishing that there is an unfocused
proof of $A$ if and only if there is a focused proof of
$A^\ominus$.\footnote{The two exceptions are Zeilberger completeness
  proof in classical persistent logic \cite{zeilberger08unity} and my
  proof in intuitionistic persistent logic \cite{simmons11structural}.}
The weaker formulation is sufficient for our current purposes, so we
will discuss the weaker property of polarization-strategy-based
correctness, not erasure-based correctness.

\subsection{Sequent calculus}

Usually, focused logics are described as
having multiple judgments:
\begin{itemize}
\item $\mildrfoc{\Gamma}{\Delta}{A^+}$ (the {\it right focus} sequent, where
the proposition $A^+$ is in focus),
\item $\mildinv{\Gamma}{\Delta}{C}$ (the {\it inversion} sequent), and
\item $\mildlfoc{\Gamma}{\Delta}{A^-}{C}$ (the {\it left focus} sequent,
where the proposition $A^-$ is in focus).
\end{itemize}
Another reasonable presentation of linear logic uses only one sequent
$\mildseq{\Gamma}{\Delta}{U}$, but generalizes what is to allowed to
to appear in the linear context $\Delta$ or in the succeedant, which
we write $U$. We will use this interpretation to understand the logic
described in Figure~\ref{fig:kaustuv-focused}.

\begin{figure}[t]
\begin{tabbing}
\quad $A^+$ \= $::= p^+ 
              \mid {\downarrow}A^- 
              \mid {!}A^- 
              \mid \one
              \mid A \otimes B$\\
\quad $A^-$ \> $::= p^-
              \mid {\uparrow}A^+
              \mid A \lolli B$\\
\quad $\Gamma$ \> $::= \cdot \mid \Gamma, A^-$ \qquad\qquad\qquad\qquad\qquad\qquad\quad \= {\it (multiset)}\\
\quad $\Delta$ \> $::= \cdot \mid \Delta, A^+ \mid \Delta, A^- \mid \Delta, [A^-] \mid \Delta, \langle A^+ \rangle$ \> {\it (multiset)}\\
\quad $U$ \> $::= A^- \mid A^+ \mid [ A^+ ] \mid \langle A^- \rangle$\\
\end{tabbing}
%
%
\quad \fbox{$\mildseq{\Gamma}{\Delta}{U}$}
\[
\infer[{\it focus}^*_R]
{\mildseq{\Gamma}{\Delta}{A^+}}
{\mildseq{\Gamma}{\Delta}{[A^+]}}
\quad
\infer[{\it focus}^*_L]
{\mildseq{\Gamma}{\Delta,A^-}{U}}
{\mildseq{\Gamma}{\Delta,[A^-]}{U}}
\quad
\infer[{\it copy}^*]
{\mildseq{\Gamma, A^-}{\Delta}{U}}
{\mildseq{\Gamma, A^-}{\Delta, [A^-]}{U}}
\]

\[
\infer[\eta^+]
{\mildseq{\Gamma}{\Delta, p^+}{U}}
{\mildseq{\Gamma}{\Delta, \langle p^+ \rangle}{U}}
\quad
\infer[{\it id}^+]
{\mildseq{\Gamma}{\langle A^+ \rangle}{[A^+]}}
{}
\quad
\infer[\eta^-]
{\mildseq{\Gamma}{\Delta}{p^-}}
{\mildseq{\Gamma}{\Delta}{\langle p^- \rangle}}
\quad
\infer[{\it id}^-]
{\mildseq{\Gamma}{[A^-]}{\langle A^- \rangle}}
{}
\]

\[
\infer[{\uparrow}_R]
{\mildseq{\Gamma}{\Delta}{{\uparrow}A^+}}
{\mildseq{\Gamma}{\Delta}{A^+}}
\quad
\infer[{\uparrow}_L]
{\mildseq{\Gamma}{\Delta, [{\uparrow}A^+]}{U}}
{\mildseq{\Gamma}{\Delta, A^+}{U}}
\quad
\infer[{\downarrow}_R]
{\mildseq{\Gamma}{\Delta}{[{\downarrow}A^-]}}
{\mildseq{\Gamma}{\Delta}{A^-}}
\quad
\infer[{\downarrow}_L]
{\mildseq{\Gamma}{\Delta, {\downarrow}A^-}{U}}
{\mildseq{\Gamma}{\Delta, A^-}{U}}
\]

\[
%
\infer[{!}_R]
{\mildseq{\Gamma}{\cdot}{[{!}A^-]}}
{\mildseq{\Gamma}{\cdot}{A^-}}
\quad
\infer[{!}_L]
{\mildseq{\Gamma}{\Delta, {!}A^-}{U}}
{\mildseq{\Gamma, A^-}{\Delta}{U}}
\quad
\infer[\one_R]
{\mildseq{\Gamma}{\cdot}{[\one]}}
{}
\quad
\infer[\one_L]
{\mildseq{\Gamma}{\Delta, \one}{U}}
{\mildseq{\Gamma}{\Delta}{U}}
\]

\[
%
\infer[{\tensor}_R]
{\mildseq{\Gamma}{\Delta_1,\Delta_2}{[A^+ \tensor B^+]}}
{\mildseq{\Gamma}{\Delta_1}{[A^+]}
 &
 \mildseq{\Gamma}{\Delta_2}{[B^+]}}
\quad
\infer[{\tensor}_L]
{\mildseq{\Gamma}{\Delta, A^+ \tensor B^+}{U}}
{\mildseq{\Gamma}{\Delta, A^+, B^+}{U}}
\]

\[
%
\infer[{\lolli}_R]
{\mildseq{\Gamma}{\Delta}{A^+ \lolli B^-}}
{\mildseq{\Gamma}{\Delta, A^+}{B^-}}
\quad
\infer[{\lolli}_L]
{\mildseq{\Gamma}{\Delta_1,\Delta_2, [A^+ \lolli B^-]}{U}}
{\mildseq{\Gamma}{\Delta_1}{[A^+]}
 &
 \mildseq{\Gamma}{\Delta_2, [B^-]}{U}}
%
\]
\caption{Focused intuitionstic linear logic.}
\label{fig:kaustuv-focused}
\end{figure}


By adding a side condition to the three rules ${\it focus}_R$, ${\it
  focus}_L$, and ${\it copy}$ that neither the context $\Delta$ nor
the succeedant $U$ can contain an in-focus proposition $[A^+]$ or
$[A^-]$, derivations can maintain the invariant that there is always
at most one proposition in focus, effectively restoring the situation
in which there are three distinct judgments.  This restriction alone
gives us what Pfenning calls a {\it chaining} logic
\cite{pfenning02chaining} and which Laurent calls a {\it weakly
  focused} logic \cite{laurent04proof}.\footnote{This is not what I
  called a weakly focused logic \cite{simmons09weak}. That weakly
  focused system had an additional restriction that invertible rules
  could not be applied when any other proposition was in focus; this
  corresponded to what Laurent called a strongly $+$-focused logic.}
We obtain a fully focused logic by further restricting these three
rules so that they only apply when the sequent below the line is {\it
  stable}.  A sequent $\mildseq{\Gamma}{\Delta}{U}$ is stable if the
context $\Delta$ contains only negative propositions $A^-$ and
suspended positive propositions $\langle A^+ \rangle$ and the
succeedant $U$ is either a positive proposition $A^+$ or a suspended
negative proposition $\langle A^- \rangle$. 

We will now turn our attention to the meaning of these suspended
propositions and the four rules that interact with them: ${\it id}^+$,
${\it id}^-$, $\eta^+$, and $\eta^-$.

\subsection{Suspended propositions}

In unfocused sequent calculi, such as the one for linear logic in
Figure~\ref{fig:linear}, initial sequents are restricted to atomic
propositions. All sequent calculi, focused or unfocused, have the
subformula property: every rule breaks down a proposition, either on
the left or the right. Since the logical interpretation of atomic
propositions is that they are stand-ins for unknown propositions, we
are unable to break them down any further. We are therefore only able
to derive an atomic conclusion or use an atomic premise with the {\it
  init} rule that concludes $\seq{\Gamma}{p}{p}$ and has no premises.
This {\it init} rule is the only instance of the admissible identity
theorem $\seq{\Gamma}{A}{A}$ that must be explicitly included as a
proof rule. If we substitute in a proposition for an atomic
propositions, the structure of the proof stays exactly the same,
except that instances of initial sequents become admissible instances
of the general identity theorem.

To my knowledge, all published proof systems for focused logic have
attempted to replicate this initial rule {\it init}. This is a design
error, and it is one that has historically made it enormously (and
unnecessarily) difficult to prove the identity theorem for focused
systems. Our presentation uses {\it suspensions}: suspended positive
propositions $\langle A^+ \rangle$ only appear in the linear context
$\Delta$, and suspended negative propositions $\langle A^- \rangle$
only appear as succeedants. They treated as stable (we never break
down a suspended proposition) and are only used to immediately
prove a proposition in focus with one of the identity rules
${\it id}^+$ or ${\it id}^-$.

Suspended positive propositions act much like regular variables in a
natural deduction system. The positive identity rule ${\it id}^+$
allows us to prove any positive proposition given that positive
proposition appears suspended in the context.  There is a
corresponding substitution principle for focal substitutions that has
a natural-deduction-like flavor: we can substitute a derivation
right-focused on $A^+$ for a suspended positive proposition $\langle
A^+ \rangle$ in a context.

\bigskip
\begin{theorem}[Focal substitution (positive)]\label{thm:fsubst-pos}~\\
For stable $\Delta$,
if $\mildseq{\Gamma}{\Delta}{[A^+]}$ 
and $\mildseq{\Gamma}{\Delta', \langle A^+ \rangle}{U}$, 
then $\mildseq{\Gamma}{\Delta', \Delta}{U}$.
\end{theorem}

\begin{proof}
  Straightforward induction over the second given derivation, as in a
  proof of regular substitution in a natural deduction system. If the
  second derivation is the axiom ${\it id}^+$, the result follows
  immediately using the first given derivation.
\end{proof}

\noindent
Note that, in the statement of Theorem~\ref{thm:fsubst-pos}, the
second premise $\mildseq{\Gamma}{\Delta', \langle A^+ \rangle}{U}$ may
be a right-focused sequent $\mildseq{\Gamma}{\Delta', \langle A^+
  \rangle}{[B^+]}$, a left-focused sequent $\mildseq{\Gamma}{\Delta'',
  [B^-], \langle A^+ \rangle}{U}$, or an inverting sequent. 

Suspended negative propositions are a bit weirder. While a derivation
of $\mildseq{\Gamma}{\Delta', \langle A^+ \rangle}{U}$ is missing a
premise that can be satisfied by a derivation of
$\mildseq{\Gamma}{\Delta}{[A^+]}$, a derivation of 
$\mildseq{\Gamma}{\Delta}{\langle A^- \rangle}$ is missing a 
{\it continuation} that can be satisfied by a derivation of
$\mildseq{\Gamma}{\Delta', [A^-]}{U}$. The focal substitution principle,
however, still takes the basic form of a substitution principle.

\bigskip
\begin{theorem}[Focal substitution (negative)]\label{thm:fsubst-neg}~\\
For stable $\Delta'$ and $U$, 
if $\mildseq{\Gamma}{\Delta}{\langle A^- \rangle}$
and $\mildseq{\Gamma}{\Delta', [A^-]}{U}$, 
then $\mildseq{\Gamma}{\Delta', \Delta}{U}$. 
\end{theorem}

\begin{proof}
  Straightforward induction over the {\it first} given derivation; if
  the first derivation is the axiom ${\it id}^-$, the result follows
  immediately using the second given derivation.
\end{proof}

\noindent
As a regular substitution principle that is inductive over the structure
of the first given proposition, focal substitution is reminiscent of 
the {\it leftist substitutions} introduced by Pfenning and Davies in the 
context of the possibility modality \cite{pfenning01judgmental}.

Unlike cut admissibility, which we discuss in Section~\ref{sec:lincut}, both
of the focal substitution principles are straightforward inductions
over the structure of the derivation containing the suspended
proposition. In the development of structural focalization, I discuss
how, in a focused presentation of persistent intuitionistic logic that
is encoded in LF, a suspended positive premise can be encoded as a
hypothetical right focus. This encoding makes the ${\it id}^+$ rule an
instance of the hypothesis rule provided by LF and establishes
Theorem~\ref{thm:fsubst-pos} ``for free'' as an instance of LF
substitution. This is possible to do for negative focal substitution
as well, but it is somewhat counter-intuitive
\cite{simmons11structural}.

The two substitution
principles can be phrased as admissible rules for building derivations,
which we indicate using a dashed line:
\[
\infer-[{\it subst}^+]
{\mildseq{\Gamma}{\Delta', \Delta}{U}}
{\mildseq{\Gamma}{\Delta}{[A^+]}
 &
 \mildseq{\Gamma}{\Delta', \langle A^+ \rangle}{U}}
\qquad
\infer-[{\it subst}^-]
{\mildseq{\Gamma}{\Delta', \Delta}{U}}
{\mildseq{\Gamma}{\Delta}{\langle A^- \rangle}
 &
 \mildseq{\Gamma}{\Delta', [A^-]}{U}}
\]

\subsection{Identity expansions}
\label{sec:linindentity}

Suspended propositions appear in Figure~\ref{fig:kaustuv-focused} in
two places: first in the identity rules that we have just discussed
and connected with the focal substitution principles, and second in
the rules marked $\eta^+$ and $\eta^-$, which are also the only
mention of atomic propositions in the presentation. It is here that we
need to make an absolutely critical shift of perspective from
unfocused to focused logic. In an unfocused logic, the rules
nondeterministically break down propositions, and the initial rule {\it
  init} puts an end to this process when an atomic proposition is
reached. In a focused logic, the focus and inversion phases {\it must}
break down a proposition all the way until a shift is reached. The two
$\eta$ rules are what put an end to this when an atomic proposition is
reached, and they correspond to the two ${\it id}$ rules that allow
these necessarily suspended propositions to successfully conclude a
right or left focus.

\begin{figure}
{\small 
\[
\infer-[\eta^+]
{\mildseq{\Gamma}{\Delta, {\downarrow}A}{U}}
{\deduce
 {\mildseq{\Gamma}{\Delta, \langle {\downarrow}A \rangle}{U}}
 {\mathcal D}}
\quad
\deduce{\mathstrut}{\Longrightarrow}
\quad
\infer[{\downarrow}_L]
{\mildseq{\Gamma}{\Delta, {\downarrow}A}{U}}
{\infer-[{\it subst}^+]
 {\mildseq{\Gamma}{\Delta, A}{U}}
 {\infer[{\downarrow}_R]
  {\mildseq{\Gamma}{A}{[ {\downarrow}A ]}}
  {\infer-[{\eta}^-]
   {\mildseq{\Gamma}{A}{A}}
   {\infer-[{\it focus}_L]
    {\mildseq{\Gamma}{A}{\langle A \rangle}}
    {\infer[{\it id}^-]
     {\mildseq{\Gamma}{[ A ]}{\langle A \rangle}}
     {}}}}
  &
  \deduce
  {\mildseq{\Gamma}{\Delta, \langle {\downarrow}A \rangle}{U}}
  {\mathcal D}}}
\]

\[
\infer-[\eta^+]
{\mildseq{\Gamma}{\Delta,{!}A}{U}}
{\deduce
 {\mildseq{\Gamma}{\Delta, \langle {!}A \rangle}{U}}
 {\mathcal D}}
\quad
\deduce{\mathstrut}{\Longrightarrow}
\infer[{!}_L]
{\mildseq{\Gamma}{\Delta,{!}A}{U}}
{\infer-[{\it subst}^+]
 {\mildseq{\Gamma,A}{\Delta}{U}}
 {\infer[{!}_R]
  {\mildseq{\Gamma, A}{\cdot}{[{!}A]}}
  {\infer-[\eta^-]
   {\mildseq{\Gamma, A}{\cdot}{A}}
   {\infer[\it copy]
    {\mildseq{\Gamma, A}{\cdot}{\langle A \rangle}}
    {\infer[{\it id}^-]
     {\mildseq{\Gamma, A}{[ A ]}{\langle A \rangle}}
     {}}}}
  &
  \infer-[{\it weaken}]
  {\mildseq{\Gamma,A}{\Delta, \langle {!}A \rangle}{U}}
  {\deduce
   {\mildseq{\Gamma}{\Delta, \langle {!}A \rangle}{U}}
   {\mathcal D}}}}
\]

\[
\infer-[\eta^+]
{\mildseq{\Gamma}{\Delta, A \otimes B}{U}}
{\deduce{\mildseq{\Gamma}{\Delta, \langle A \otimes B \rangle}{U}}{\mathcal D}}
\quad
\deduce{\mathstrut}{\Longrightarrow}
\!\!\!\!\!\!\!\!\!
\infer[{\otimes}_L]
{\mildseq{\Gamma}{\Delta, A \otimes B}{U}}
{\infer-[\eta^+]
 {\mildseq{\Gamma}{\Delta, A, B}{U}}
 {\infer-[\eta^+]
 {\mildseq{\Gamma}{\Delta, \langle A \rangle, B}{U}}
 {\infer-[{\it subst}^+]
  {\mildseq{\Gamma}{\Delta, \langle A \rangle, \langle B \rangle}{U}}
  {\infer
   {\mildseq{\Gamma}{\langle A \rangle, \langle B \rangle}{[A \otimes B]}}
   {\infer[{\it id}^+]
    {\mildseq{\Gamma}{\langle A \rangle}{[A]}}
    {}
    & 
    \infer[{\it id}^+]
    {\mildseq{\Gamma}{\langle B \rangle}{[B]}}
    {}}
   & 
   \deduce
   {\mildseq{\Gamma}{\Delta, \langle A \otimes B \rangle}{U}}
   {\mathcal D}}}}}
\]

\[
\infer-[\eta^-]
{\mildseq{\Gamma}{\Delta}{{\uparrow}A}}
{\deduce
 {\mildseq{\Gamma}{\Delta}{\langle {\uparrow}A \rangle}}
 {\mathcal D}}
\quad
\deduce{\mathstrut}{\Longrightarrow}
\quad
\infer[{\uparrow}_R]
{\mildseq{\Gamma}{\Delta}{{\uparrow}A}}
{\infer-[{\it subst}^-]
 {\mildseq{\Gamma}{\Delta}{A}}
 {\deduce
  {\mildseq{\Gamma}{\Delta}{\langle {\uparrow}A \rangle}}
  {\mathcal D}
  &
  \infer[{\uparrow}_L]
  {\mildseq{\Gamma}{[{\uparrow}A]}{A}}
  {\infer-[\eta^+]
   {\mildseq{\Gamma}{A}{A}}
   {\infer[{\it focus}_R]
    {\mildseq{\Gamma}{\langle A \rangle}{A}} 
    {\infer[{\it id}^+]
     {\mildseq{\Gamma}{\langle A \rangle}{[ A ]}}
     {}}}}}}
\]

\[
\infer-[\eta^-]
{\mildseq{\Gamma}{\Delta}{A \lolli B}}
{\deduce
 {\mildseq{\Gamma}{\Delta}{\langle A \lolli B \rangle}}
 {\mathcal D}}
\quad
\deduce{\mathstrut}{\Longrightarrow}
\!\!\!\!\!\!
\infer[{\lolli}_R]
{\mildseq{\Gamma}{\Delta}{A \lolli B}}
{\infer-[\eta^+]
 {\mildseq{\Gamma}{\Delta, A}{B}}
 {\infer-[\eta^-]
  {\mildseq{\Gamma}{\Delta, \langle A \rangle}{B}}
  {\infer-[{\it subst}^-]
   {\mildseq{\Gamma}{\Delta, \langle A \rangle}{\langle B \rangle}}
   {\deduce
    {\mildseq{\Gamma}{\Delta}{\langle A \lolli B \rangle}}
    {\mathcal D}
    &
    \infer[{\lolli}_L]
    {\mildseq{\Gamma}{\langle A \rangle, [ A \lolli B ]}{\langle B \rangle}}
    {\infer[{\it id}^+]
     {\mildseq{\Gamma}{\langle A \rangle}{[ A ]}}
     {}
     &
     \infer[{\it id}^-]
     {\mildseq{\Gamma}{[ B ]}{\langle B \rangle}}
     {}}}}}}
\]}
\caption{Identity expansion -- restricting $\eta^+$ and $\eta^-$ to atomic 
 propositions}
\label{fig:lineta-1}
\end{figure}

\begin{figure}[t]
{\small

\[
\infer-[\eta^+]
{\mildseq{\Gamma}{\Delta, \one}{U}}
{\deduce
 {\mildseq{\Gamma}{\Delta, \langle \one \rangle}{U}}
 {\mathcal D}}
\quad
\Longrightarrow
\infer[{\one}_L]
{\mildseq{\Gamma}{\Delta, \one}{U}}
{\infer-[{\it subst}^+]
 {\mildseq{\Gamma}{\Delta}{U}}
 {\infer[{\one}_R]
  {\mildseq{\Gamma}{\cdot}{[ \one ]}}
  {}
  &
  \deduce
  {\mildseq{\Gamma}{\Delta, \langle \one \rangle}{U}}
  {\mathcal D}}}
\]

\[
\infer-[\eta^+]
{\mildseq{\Gamma}{\Delta, \zero}{U}}
{\deduce
 {\mildseq{\Gamma}{\Delta, \langle \zero \rangle}{U}}
 {\mathcal D}}
\quad
\Longrightarrow
\infer[\zero_L]
{\mildseq{\Gamma}{\Delta, \zero}{U}}
{}
\]

\[
\infer-[\eta^+]
{\mildseq{\Gamma}{\Delta, A \oplus B}{U}}
{\deduce
 {\mildseq{\Gamma}{\Delta, \langle A \oplus B \rangle}{U}}
 {\mathcal D}}
\quad
\Longrightarrow
\!\!\!\!\!\!\!\!\!\!\!\!\!\!\!\!
\infer[{\oplus}_L]
{\mildseq{\Gamma}{\Delta, A \oplus B}{U}}
{\infer-[\eta^+]
 {\mildseq{\Gamma}{\Delta, A}{U}}
 {\infer-[{\it subst}^+]
  {\mildseq{\Gamma}{\Delta, \langle A \rangle}{U}}
  {\infer[{\oplus}_{R1}]
   {\mildseq{\Gamma}{\Delta, \langle A \rangle}{[ A \oplus B ]}}
   {\infer[{\it id}^+]
    {\mildseq{\Gamma}{\langle A \rangle}{[ A ]}}
    {}}
   &
   \deduce
   {\mildseq{\Gamma}{\Delta, \langle A \oplus B \rangle}{U}}
   {\mathcal D}}}
 &
 \deduce
 {\mildseq{\Gamma}{\Delta, B}{U}}
 {\vdots}
 }
\]


\[
\infer-[\eta^-]
{\mildseq{\Gamma}{\Delta}{\top}}
{\deduce
 {\mildseq{\Gamma}{\Delta}{\langle \top \rangle}}
 {\mathcal D}}
\quad
\Longrightarrow
\quad
\infer[{\top}_R]
{\mildseq{\Gamma}{\Delta}{\top}}
{}
\]

\[
\infer-[\eta^-]
{\mildseq{\Gamma}{\Delta}{A \with B}}
{\deduce
 {\mildseq{\Gamma}{\Delta}{\langle A \with B \rangle}}
 {\mathcal D}}
\quad
\Longrightarrow
\!\!\!\!
\infer[{\with}_R]
{\mildseq{\Gamma}{\Delta}{A \with B}}
{\infer[\eta^-]
 {\mildseq{\Gamma}{\Delta}{A}}
 {\infer-[{\it subst}^-]
  {\mildseq{\Gamma}{\Delta}{\langle A \rangle}}
  {\deduce
   {\mildseq{\Gamma}{\Delta}{\langle A \with B \rangle}}
   {\mathcal D}
   &
   \infer[{\with}_{L1}]
   {\mildseq{\Gamma}{[A \with B]}{\langle A \rangle}}
   {\infer[{\it id}^-]
    {\mildseq{\Gamma}{[A]}{\langle A \rangle}}
    {}}}}
 & 
 \deduce
 {\mildseq{\Gamma}{\Delta}{B}}
 {\vdots}}
\]}

\caption{Identity expansion for units and additive connectives}
\label{fig:lineta-2}
\end{figure}


Just as the {\it init} rule is a particular instance of the admissible
identity sequent $\seq{\Gamma}{A}{A}$ in unfocused linear logic, the
atomic suspension rules $\eta^+$ and $\eta^-$ are instances of an admissible
identity expansion rule in focused linear logic:
\[
\infer-[\eta^+]
{\mildseq{\Gamma}{\Delta, A^+}{U}}
{\mildseq{\Gamma}{\Delta, \langle A^+ \rangle}{U}}
\qquad
\infer-[\eta^-]
{\mildseq{\Gamma}{\Delta}{A^-}}
{\mildseq{\Gamma}{\Delta}{\langle A^- \rangle}}
\]
This admissible rule must be established by mutual recursion; the
proof is structurally inductive on the structure of the proposition,
and uses focal substitution in a critical way. Most of the cases of
this proof are represented in Figure~\ref{fig:lineta-1}. (Note that in
this figure we omit polarity annotations from propositions as they are
always clear from the context.) The remaining case (for the
multiplicative unit $\one$) is presented in Figure~\ref{fig:lineta-2}
along with the cases for the additive connectives $\zero$, $\oplus$,
$\top$, and $\with$, which are neglected elsewhere in this chapter.

The admissible identity expansion rules fit with an interpretation of
positive atomic propositions as stand-ins for arbitrary positive
propositions and of negative atomic propositions as stand-ins for
negative atomic propositions: if we substitute a proposition in for
some atomic proposition, all the instances of atomic suspension
corresponding to that rule become admissible instances of identity
expansion. Furthermore, the usual identity principles are simple
corollaries of identity expansion:
\[
\infer-[\eta^+]
{\mildseq{\Gamma}{A^+}{A^+}}
{\infer[{\it focus}_R]
 {\mildseq{\Gamma}{\langle A^+ \rangle}{A^+}}
 {\infer[{\it id}^+]
  {\mildseq{\Gamma}{\langle A^+ \rangle}{[A^+]}}
  {}}}
\qquad
\infer-[\eta^-]
{\mildseq{\Gamma}{A^-}{A^-}}
{\infer[{\it focus}_L]
 {\mildseq{\Gamma}{A^-}{\langle A^- \rangle}}
 {\infer[{\it id}^-]
  {\mildseq{\Gamma}{[A^-]}{\langle A^- \rangle}}
  {}}}
\]

\subsection{Cut admissibility}
\label{sec:lincut}

Theorem~\ref{thm:lincut} mostly follows the well-worn contours of a
structural cut admissibility argument \cite{pfenning00structural}, so
we defer a full discussion of cut admissibility until the next
chapter, where the use of structural focalization will allow us to
give a tidier proof of the theorem.\footnote{The main ``untidy''
  aspect of Theorem~\ref{thm:lincut} is that the lack of a forced
  inversion order means that the right commutative cases dealing with
  invertible rules must be repeated in parts 1 and 3, and likewise for
  the left commutative cases dealing with invertible rules and parts 2
  and 4. The repetition of {\it all} cases between parts 3 and 5 will
  also be addressed in the next chapter.}
%
The only important caveat to emphasize about Theorem~\ref{thm:lincut}
is that cut admissibility is only applicable in the absence of any
non-atomic suspended propositions. If we did not make this
restriction, then in Theorem~\ref{thm:lincut}, part 1, we might encounter
a derivation of $\mildseq{\Gamma}{\langle A \tensor B \rangle}{[ A \tensor B ]}$
being cut into the derivation
\[
\infer[{\otimes}_R]
{\mildseq{\Gamma}{\Delta',A \tensor B}{U}}
{\deduce{\mildseq{\Gamma}{\Delta', A, B}{U}}{\mathcal E}}
\]
in which case there is no clear way to proceed and prove 
$\mildseq{\Gamma}{\Delta', \langle A \tensor B \rangle}{U}$. 

\bigskip
\begin{theorem}[Cut admissibility]\label{thm:lincut}
For all $\Gamma$, $A^+$, $A^-$, $\Delta$, $\Delta'$, and $U$ that
do not contain any non-atomic suspended propositions:
\begin{enumerate}
\item If $\mildseq{\Gamma}{\Delta}{[A^+]}$
      and $\mildseq{\Gamma}{\Delta',A^+}{U}$
      (where $\Delta$ is stable), 
      then $\mildseq{\Gamma}{\Delta',\Delta}{U}$.
\item If $\mildseq{\Gamma}{\Delta}{A^-}$
      and $\mildseq{\Gamma}{\Delta', [A^-]}{U}$
      (where $\Delta$, $\Delta'$, and $U$ are stable),
      then $\mildseq{\Gamma}{\Delta',\Delta}{U}$. 
\item If $\mildseq{\Gamma}{\Delta}{A^-}$
      and $\mildseq{\Gamma}{\Delta', A^-}{U}$
      (where $\Delta$ is stable), 
      then $\mildseq{\Gamma}{\Delta',\Delta}{U}$. 
\item If $\mildseq{\Gamma}{\Delta}{A^+}$
      and $\mildseq{\Gamma}{\Delta', A^+}{U}$
      (where $\Delta'$ and $U$ are stable),
      then $\mildseq{\Gamma}{\Delta',\Delta}{U}$. 
\item If $\mildseq{\Gamma}{\cdot}{A^-}$
      and $\mildseq{\Gamma, A^-}{\Delta}{U}$,
      then $\mildseq{\Gamma}{\Delta}{U}$. 
\end{enumerate}
\end{theorem}

\begin{proof}
By lexicographic induction. In each invocation of the induction
hypothesis, either the principal cut formula $A^+$ or $A^-$ gets 
smaller, or else it stays the same and the number of the part
gets smaller (as when we invoke part 2 when proving part 5). 

Within parts 3 and 5, the first two metrics stay the same while the
second given derivation gets smaller, and within part 4, the first two
metrics stay the same while the first given derivation gets smaller.
\end{proof}

\subsection{Correctness of focusing}

Now we will make precise the correctness for a focused, polarized logic
that was discussed in Section~\ref{sec:linpolar}: that there is a unfocused
proof of $A$ if and only if there is a focused proof of $A^\oplus$. The
proof require lifting of our erasure function and our ``obvious'' 
polarization strategy to contexts and succeedents, which is done
in Figure~\ref{fig:lin-shift-ctx}. Soundness is established on the 
basis of erasure as in \cite{simmons11structural}, but as discussed
we give the slightly simpler polarization-strategy-based proof of
completeness.

\bigskip
\begin{theorem}[Soundness of focusing]
If $\mildseq{\Gamma}{\Delta}{U}$, where $\Delta$ and $U$ contain only
atomic suspensions, then $\seq{\Gamma^\circ}{\Delta^\circ}{U^\circ}$.
\end{theorem}

\begin{proof}
  By straightforward induction on the given derivation; in each case,
  the result either follows directly by invoking the induction
  hypothesis or by invoking the induction hypothesis and applying one
  rule from Figure~\ref{fig:linear}.
\end{proof}

\begin{theorem}[Completeness of focusing]
If $\seq{\Gamma}{\Delta}{C}$,
then $\mildseq{\Gamma^\bullet}{\Delta^\bullet}{C^\bullet}$. 
\end{theorem}

\begin{proof}
  By induction on the first given derivation. Each rule in 
  Figure~\ref{fig:linear}

  Rule {\it copy}:
  \[
  \]

  Rule $\lolli_L$:
  \[
  \]

  Rule $\lolli_R$:
  {\small \[
  \infer-[{\it cut}(3)]
  {\mildseq{\Gamma^\bullet}{\Delta^\bullet}
    {{\downarrow}(A^\oplus \lolli B^\ominus)}}
  {\infer[{\lolli}_R]
   {\mildseq{\Gamma^\bullet}{\Delta^\bullet}
     {{\downarrow}{\uparrow}A^\oplus \lolli {\uparrow}{\downarrow}B^\ominus}}  
   {\infer[{\downarrow}_L]
    {\mildseq{\Gamma^\bullet}{\Delta^\bullet, {\downarrow}{\uparrow}A^\oplus}
      {{\uparrow}{\downarrow}B^\ominus}}
    {\infer[{\uparrow}_R]
     {\mildseq{\Gamma^\bullet}{\Delta^\bullet, {\uparrow}A^\oplus}
      {{\uparrow}{\downarrow}B^\ominus}}
     {\mildseq{\Gamma^\bullet}{\Delta^\bullet, {\uparrow}A^\oplus}
      {{\downarrow}B^\ominus}}}}
   &
   \infer[{\it focus}_R]
   {\mildseq{\Gamma}{{\downarrow}{\uparrow}A^\oplus \multimap B^\ominus}
      {{\downarrow}(A^\oplus \lolli B^\ominus)}}
   {\infer[{\downarrow}_R]
    {\mildseq{\Gamma^\bullet}
      {{\downarrow}{\uparrow}A^\oplus \multimap {\uparrow}{\downarrow}B^\ominus}
      {[ {\downarrow}(A^\oplus \lolli B^\ominus)} ]}
    {\infer[{\lolli}_R]
     {\mildseq{\Gamma^\bullet}
       {{\downarrow}{\uparrow}A^\oplus \multimap {\uparrow}{\downarrow}B^\ominus}
       {A^\oplus \lolli B^\ominus}}
     {\infer-[\eta^+]
      {\mildseq{\Gamma^\bullet}
        {{\downarrow}{\uparrow}A^\oplus 
           \multimap {\uparrow}{\downarrow}B^\ominus, 
         A^\oplus}
        {B^\ominus}}
      {\infer-[\eta^-]
       {\mildseq{\Gamma^\bullet}
         {{\downarrow}{\uparrow}A^\oplus
            \multimap {\uparrow}{\downarrow}B^\ominus, 
          \langle A^\oplus \rangle}
         {B^\ominus}}
       {\infer[{\it focus}_L]
        {\mildseq{\Gamma^\bullet}
          {{\downarrow}{\uparrow}A^\oplus 
             \multimap {\uparrow}{\downarrow}B^\ominus, 
           \langle A^\oplus \rangle}
          {\langle B^\ominus \rangle}}
        {\infer[{\lolli}_L]
         {\mildseq{\Gamma^\bullet}
           {[ {\downarrow}{\uparrow}A^\oplus 
              \multimap {\uparrow}{\downarrow}B^\ominus ], 
            \langle A^\oplus \rangle}
           {\langle B^\ominus \rangle}}
         {\infer[{\downarrow}_R]
          {\mildseq{\Gamma^\bullet}{\langle A^\oplus \rangle}
           {[{\downarrow}{\uparrow}A^\oplus]}}
          {\infer[{\uparrow}_R]
           {\mildseq{\Gamma^\bullet}{\langle A^+ \rangle}{{\uparrow}A^+}}
           {\infer[{\it focus}_R]
            {\mildseq{\Gamma^\bullet}{\langle A^+ \rangle}{A^+}} 
            {\infer[{\it id}^+]
             {\mildseq{\Gamma^\bullet}{\langle A^+ \rangle}{[A^+]}}
             {}}}}
          &
          \infer[{\uparrow}_L]
          {\mildseq{\Gamma^\bullet}{[{\uparrow}{\downarrow}B^\ominus]}
             {\langle B^\ominus \rangle}}
          {\infer[{\downarrow}_L]
           {\mildseq{\Gamma^\bullet}{{\downarrow}B^\ominus}
             {\langle B^\ominus \rangle}}
           {\infer[{\it focus}_L]
            {\mildseq{\Gamma^\bullet}{B^\ominus}{\langle B^\ominus \rangle}}
            {\infer[{\it id}^-]
             {\mildseq{\Gamma^\bullet}{[B^\ominus]}{\langle B^\ominus \rangle}}
             {}}}}}}}}}}}}
  \]}
\end{proof}

The correctness of focusing is established as in the structural focalization
methadology, except that we establish a weaker polarization-strategy-based
proof of completeness rather than an erasure-based proof as in 
\cite{simmons11structural}. Both the soundness and completeness of focusing


\begin{figure}
{\small \[
\begin{array}{rcl|rcl|rcl}
{(\Gamma)^\circ} & & &
{(\underline{\Delta})^\circ} & & &
{(\underline{U})^\circ} & & 
\\
(\cdot)^\circ & \!\!\!=\!\!\! & \cdot &
(\cdot)^\circ & \!\!\!=\!\!\! & \cdot &
(A^-)^\circ & \!\!\!=\!\!\! & (A^-)^\circ
\\
(\Gamma, A^-)^\circ & \!\!\!=\!\!\! & (\Gamma)^\circ, (A^-)^\circ &
(\Delta, A^+)^\circ & \!\!\!=\!\!\! & (\Delta)^\circ, (A^+)^\circ &
(A^+)^\circ & \!\!\!=\!\!\! & (A^+)^\circ
\\
& & & 
(\Delta, A^-)^\circ & \!\!\!=\!\!\! & (\Delta)^\circ, (A^-)^\circ &
([A^+])^\circ & \!\!\!=\!\!\! & (A^+)^\circ 
\\
& & &
(\Delta, [ A^- ])^\circ & \!\!\!=\!\!\! & (\Delta)^\circ, (A^-)^\circ & 
(\langle p^- \rangle)^\circ & \!\!\!=\!\!\! & p^-
\\
& & &
(\Delta, \langle p^+ \rangle)^\circ & \!\!\!=\!\!\! & (\Delta)^\circ, p^+ & 
& &
\end{array}\]}
\caption{Lifting erasure and polarization (Figure~\ref{fig:lin-shift}) to
contexts and succeedents}
\label{fig:lin-shift-ctx}
\end{figure}




\begin{proof}

The reverse direction is the soundness of focusing. It is completely
straightforward, 
\end{proof}

\subsection{Confluent versus fixed inversion}
\label{sec:confluent-v-fixed}

\subsection{Running example}

\begin{figure}
\[
\infer[{\lolli}_R]
{\mildseq{\cdot}{\cdot}{{!}({\sf 6bucks} \lolli {\sf battery}) \otimes
                    {\sf 6bucks} \otimes 
                    ({\sf battery} \lolli {\sf robot}) \lolli {\sf robot}}}
{\infer[{\otimes}_L]
{\mildseq{\cdot}{{!}({\sf 6bucks} \lolli {\sf battery}) \otimes
                    {\sf 6bucks} \otimes 
                    ({\sf battery} \lolli {\sf robot})}{{\sf robot}}}
{\infer[{!}_L]
{\mildseq{\cdot}{{!}({\sf 6bucks} \lolli {\sf battery}),
                    {\sf 6bucks} \otimes 
                    ({\sf battery} \lolli {\sf robot})}{{\sf robot}}}
{\infer[{\otimes}_L]
{\mildseq{\Gamma}{{\sf 6bucks} \otimes 
                    ({\sf battery} \lolli {\sf robot})}{{\sf robot}}}
{\infer[{\it copy}]
{\mildseq{\Gamma}{{\sf 6bucks}, {\sf battery} \lolli {\sf robot}}{{\sf robot}}}
{\infer[{\lolli}_L]
{\mildseq{\Gamma}{{\sf 6bucks}, {\sf battery} \lolli {\sf robot}, [{\sf 6bucks} \lolli {\sf battery}]}{{\sf robot}}}
{\infer[{\it init}^+]
 {\mildseq{\Gamma}{{\sf 6bucks}}{[{\sf 6bucks}]}}
 {}
 &
 \infer[{\it blur}_L]
 {\mildseq{\Gamma}{{\sf battery} \lolli {\sf robot}, [{\sf battery}]}{{\sf robot}}}
 {\infer[{\it focus}_L]
 {\mildseq{\Gamma}{{\sf battery} \lolli {\sf robot}, {\sf battery}}{{\sf robot}}}
 {\infer[{\lolli}_L]
 {\mildseq{\Gamma}{{\sf battery}, [{\sf battery} \lolli {\sf robot}]}{{\sf robot}}}
 {\infer[{\it init^+}]
  {\mildseq{\Gamma}{{\sf battery}}{[{\sf battery}]}}
  {}
  &
  \infer[{\it blur}_L]
  {\mildseq{\Gamma}{[{\sf robot}]}{{\sf robot}}}
  {\infer[{\it focus}_R]
  {\mildseq{\Gamma}{{\sf robot}}{{\sf robot}}}
  {\infer[{\it init}^+]
  {\mildseq{\Gamma}{{\sf robot}}{[{\sf robot}]}}
  {}}}}}}}}}}}}
\] 
\caption{The single focused transition is possible 
(where we let $\Gamma = {\sf 6bucks} \lolli {\sf battery}$).}
\label{fig:focused-robot}
\end{figure}

Figure

\section{Synthetic inference rules}
\label{sec:linsynthetic}

\section{Hacking the focusing system}
\label{sec:linhack}

\subsection{Atom optimization}

\subsection{Bang optimization}

\subsection{A more primitive logic?}

\paragraph{Adjoint logic}

\paragraph{Tensor logic}

\subsection{Concurrent equality}

\section{Revisiting our notation}
\label{sec:linnote}

\section{A warning about normalization}
\label{sec:warning}

Talk about equivalence, Chris's unpublished work, and where
the focalization theorem given by this approach is deficient -- 

% Substructural logic
\chapter{Substructural logic}
\label{chapter-order}

Linear logic is the most famous of the {\it substructural logics}.
Traditional intuitionistic logic, which we call persistent to emphasize
the treatment of truth as a persistent and reusable resource, admits
the three so-called {\it structural rules} of weakening (premises need
not be used), contraction (premises may be used multiple times) and
exchange (the ordering of premises are irrelevant). Substructural
logics, then, are logics that do not admit these structural rules --
linear logic has only exchange, {\it affine} logic (which is
frequently conflated with linear logic by programming language
designers) has exchange and weakening, and {\it ordered} logic, first
investigated as a proof theory by Lambek \cite{lambek58mathematics},
lacks all three.

Calling logics like
linear, affine, and ordered logic \underline{sub}structural relative
to persistent logic is greatly unfair to the
substructural logics. Girard's linear logic can express persistent
provability using the exponential connective ${!}A$, and this idea is
generally applicable in substructural logics -- for instance, it was
applied by Polakow and Pfenning to Lambek's ordered logic
\cite{polakow99natural}. It is certainly too
late to advocate for these logics to be understood as
\underline{super}structural logics, but that is undoubtedly what they
are: generalizations of persistent logic that introduce more 
expressive power. 

In this chapter, we will define a first-order ordered linear logic
with a lax connective ${\ocircle}A$ in both unfocused
(Section~\ref{sec:ord-unfocused}) and focused
(Section~\ref{sec:ord-focused}) flavors (this logic will henceforth be
called \ollll, for ordered linear lax logic). Then, following the
structural focalization methodology introduced in the previous
chapter, we establish cut admissibility (Section~\ref{sec:ord-cut}),
and identity expansion (Section~\ref{sec:ord-identity}) for
focused~\ollll; with these results, it is possible to prove the
soundness and completeness of focusing
(Section~\ref{sec:ord-correctness}) for \ollll.  A fragment of this
system will form the basis of the logical framework in
Chapter~\ref{chapter-framework}, and that framework will, in turn,
underpin the rest of this dissertation.

Why present the rich logic \ollll~here
if only the fragment detailed in Chapter~\ref{chapter-framework} 
is needed? There are two
main reasons.  First, while we will use only a fragment of this logic
in Chapter~\ref{chapter-framework}, 
other fragments of the logic may well be interesting and
useful for other purposes. Second, the presentation in this chapter,
and in particular the discussion of substructural contexts in
Section~\ref{sec:contexts}, introduces a presentation style and
infrastructure that I believe will generalize to focused presentations
of richer logics, such as the logic of bunched
implications~\cite{pym02semantics}, non-commutative linear logic (or
``rigid logic'') \cite{simmons09linear}, subexponential logics
\cite{nigam09algorithmic}, and so on.

Furthermore, the choice to present a full account of focusing in
\ollll~is in keeping with as Andreoli's insistence that we should
avoid ambiguity as to whether we are ``defining a foundational
paradigm or a [logic] programming language (two objectives that should
clearly be kept separate)'' \cite{andreoli01focussing}. Both the full
logic \ollll~and the general methodology followed in this chapter are
general, foundational paradigms within which it is possible to
instantiate families of logic programming languages and logical
frameworks, even though we will focus on a particular logical
framework starting in Chapter~\ref{chapter-framework}.

\section{Ordered linear lax logic}
\label{sec:ord-unfocused}

Ordered linear logic was the subject of Polakow's dissertation
\cite{polakow01ordered}. It extends linear logic with a notion of {\it
  ordered resources}.  In ordered logic the linear
multiplicative conjunction $A \otimes B$ of linear logic, which
represents that we have both the resources to make an $A$ and a $B$,
is replaced by a linear multiplicative conjunction $A \fuse B$, which
represents that we have the resources to make an $A$, and they're to
the left of the resources necessary to make a $B$. Linear implication
$A \lolli B$, which represents a resource that, given the resources
necessary to construct an $A$, can construct a $B$, splits into two
propositions in ordered logic. The proposition $A \lefti B$ represents
a resource that, given the resources necessary to construct an $A$
{\it to the left}, can construct a $B$, and the proposition $A \righti
B$ demands those resource to its right. 

Ordered propositions were used by Lambek to model language
\cite{lambek58mathematics}. The word ``clever'' is a adjective that,
given a noun to its right, constructs a noun phrase (``ideas''
is a noun, and ``clever ideas'' is a noun phrase). Therefore, the world
``clever'' can be seen as an ordered resource ${\sf Phrase \righti
  NounPhrase}$. Similarly, the word ``quietly'' is an adverb that,
given a verb to its left, constructs a verb phrase (``sleeps''
is a verb, and ``sleeps quietly'' is a verb phrase). Therefore, the word
``quietly'' can be seen as an ordered resource ${\sf Verb \lefti
  VerbPhrase}$. The key innovation made by Polakow and Pfenning was
integrating both persistent and linear logic into Lambek's system with
the persistent exponential ${!}A$ and the mobile exponential
${\gnab}A$. The latter proposition is pronounced ``$A$ mobile'' or,
whimsically, ``gnab $A$'' in reference to the pronunciation of ${!}A$
as ``bang $A$.''
 
The primary sequent of ordered logic is
$\oseq{\Gamma}{\Delta}{\Omega}{\isconc{A}}$, which expresses that $A$
is a resource derivable from the persistent resources in $\Gamma$,
the ephemeral resources in $\Delta$, and the ephemeral, ordered
resources in $\Omega$. The persistent context $\Gamma$ and the linear
context $\Delta$ are multisets as before (so we think of $\Delta_1,
\Delta_2$ as being equal to $\Delta_2, \Delta_1$, for instance). The
ordered context $\Omega$ is a sequence of propositions, as in
Gentzen's original presentation of sequent calculi, and {\it not} a
multiset.  This means that the two ordered contexts $\Omega_1,
\Omega_2$ and $\Omega_2, \Omega_1$ are, in general, not the same.

\begin{figure}
\small
{\it Atomic propositions}
\[
\infer[{\it init}^+]
{\oiseq{\Gamma}{p}}
{}
\]

\medskip
{\it Modalities}
\[
\infer[{\gnab}_R]
{\orseq{\Gamma}{\Delta}{\cdot}{{\gnab}A}}
{\otseq{\Gamma}{\Delta}{\cdot}{A}}
\quad
\infer[{\gnab}_L]
{\olseq{\Gamma}{\Delta}{\Omega_L}{{\gnab}A}{\Omega_R}}
{\opseq{\Gamma}{\Delta, A}{\Omega_L}{\Omega_R}}
\]
\vspace{-5pt}
\[
\infer[{!}_R]
{\orseq{\Gamma}{\cdot}{\cdot}{{!}A}}
{\otseq{\Gamma}{\cdot}{\cdot}{A}}
\quad
\infer[{!}_L]
{\olseq{\Gamma}{\Delta}{\Omega_L}{{!}A}{\Omega_R}}
{\opseq{\Gamma,A}{\Delta}{\Omega_L}{\Omega_R}}
\]
\vspace{-5pt}
\[
\infer[{\ocircle}_R]
{\orseq{\Gamma}{\Delta}{\Omega}{{\ocircle}A}}
{\oseq{\Gamma}{\Delta}{\Omega}{\islax{A}}}
\quad
\infer[{\ocircle}_L]
{\oseq{\Gamma}{\Delta}{\Omega_L /{\ocircle}A/ \Omega_R}{\islax{C}}}
{\oseq{\Gamma}{\Delta}{\Omega_L, A, \Omega_R}{\islax{C}}}
\]

\medskip
{\it Multiplicative connectives}
\[
\infer[{\one}_R]
{\orseq{\Gamma}{\cdot}{\cdot}{\one}}
{}
\quad
\infer[{\one}_L]
{\olseq{\Gamma}{\Delta}{\Omega_L}{\one}{\Omega_R}}
{\opseq{\Gamma}{\Delta}{\Omega_L}{\Omega_R}}
\]
\[
\infer[{\fuse}_R]
{\orseq{\Gamma}{\Delta_1,\Delta_2}{\Omega_L,\Omega_R}{A \fuse B}}
{\otseq{\Gamma}{\Delta}{\Omega_L}{A}
 &
 \otseq{\Gamma}{\Delta}{\Omega_R}{B}}
\quad
\infer[{\fuse}_L]
{\olseq{\Gamma}{\Delta}{\Omega_L}{A \fuse B}{\Omega_R}}
{\opseq{\Gamma}{\Delta}{\Omega_L}{A,B,\Omega_R}}
\]
\vspace{-5pt}
\[
\infer[{\lefti}_R]
{\orseq{\Gamma}{\Delta}{\Omega}{A \lefti B}}
{\otseq{\Gamma}{\Delta}{A, \Omega}{B}}
\quad
\infer[{\lefti}_L]
{\olseq{\Gamma}{\Delta_A, \Delta}{\Omega_L, \Omega_A}{A \lefti B}{\Omega_R}}
{\otseq{\Gamma}{\Delta_A}{\Omega_A}{A}
 &
 \opseq{\Gamma}{\Delta}{\Omega_L}{B,\Omega_R}}
\]
\vspace{-5pt}
\[
\infer[{\righti}_R]
{\orseq{\Gamma}{\Delta}{\Omega}{A \righti B}}
{\otseq{\Gamma}{\Delta}{\Omega, A}{B}}
\quad
\infer[{\righti}_L]
{\olseq{\Gamma}{\Delta_A, \Delta}{\Omega_L}{A \righti B}{\Omega_A, \Omega_R}}
{\otseq{\Gamma}{\Delta_A}{\Omega_A}{A}
 &
 \opseq{\Gamma}{\Delta}{\Omega_L}{B, \Omega_R}}
\]

\medskip
{\it Additive connectives}
\[
\infer[{\zero}_L]
{\olseq{\Gamma}{\Delta}{\Omega_L}{\zero}{\Omega_R}}
{}
\quad
\infer[{\oplus}_{R1}]
{\orseq{\Gamma}{\Delta}{\Omega}{A \oplus B}}
{\otseq{\Gamma}{\Delta}{\Omega}{A}}
\quad
\infer[{\oplus}_{R2}]
{\orseq{\Gamma}{\Delta}{\Omega}{A \oplus B}}
{\otseq{\Gamma}{\Delta}{\Omega}{B}}
\]
\vspace{-5pt}
\[
\infer[{\oplus}_{L}]
{\olseq{\Gamma}{\Delta}{\Omega_L}{A \oplus B}{\Omega_R}}
{\opseq{\Gamma}{\Delta}{\Omega_L}{A,\Omega_R}
 &
 \opseq{\Gamma}{\Delta}{\Omega_L}{B,\Omega_R}}
\]
\vspace{-5pt}
\[
\infer[{\top}_R]
{\orseq{\Gamma}{\Delta}{\Omega}{\top}}
{}
\quad
\infer[{\with}_{L1}]
{\olseq{\Gamma}{\Delta}{\Omega_L}{A \with B}{\Omega_R}}
{\opseq{\Gamma}{\Delta}{\Omega_L}{A, \Omega_R}}
\quad
\infer[{\with}_{L2}]
{\olseq{\Gamma}{\Delta}{\Omega_L}{A \with B}{\Omega_R}}
{\opseq{\Gamma}{\Delta}{\Omega_L}{B, \Omega_R}}
\]
\vspace{-5pt}
\[
\infer[{\with}_R]
{\orseq{\Gamma}{\Delta}{\Omega}{A \with B}}
{\otseq{\Gamma}{\Delta}{\Omega}{A}
 &
 \otseq{\Gamma}{\Delta}{\Omega}{B}}
\]


\caption{Propositional ordered linear lax logic.}
\label{fig:ordered-prop}
\end{figure}


The presentation of ordered linear lax logic in
Figure~\ref{fig:ordered-prop} uses an ordered logic adaptation of the
matching constructs introduced in Section~\ref{sec:linnote}; all the
left rules in that figure use the construct
$\olseq{\Gamma}{\Delta}{\Omega_L}{A}{\Omega_R}$, which matches the
sequent $\oseq{\Gamma}{\Delta'}{\Omega'}{U}$
%
\smallskip
\begin{itemize}
\item if $\Omega' = (\Omega_L, A, \Omega_R)$ and $\Delta' = \Delta$;
\item if $\Omega' = (\Omega_L, \Omega_R)$ and $\Delta' = (\Delta, A)$; 
\item or if $\Omega' = (\Omega_L, \Omega_R)$, $\Delta' = \Delta$, and $A \in \Gamma$.
\end{itemize}
\smallskip
As in the alternative presentation of linear logic where ${\it copy}$ was
admissible, both the ${\it copy}$ rule and a rule Polakow called ${\it
  place}$ are admissible in the logic described in
Figure~\ref{fig:ordered-prop}.
\[
\infer-[{\it copy}]
{\oseq{\Gamma, {A}}{\Delta}{\Omega_L, \Omega_R}{U}}
{\oseq{\Gamma, {A}}{\Delta}{\Omega_L, {A}, \Omega_R}{U}}
\qquad
\infer-[{\it place}]
{\oseq{\Gamma}{\Delta, {A}}{\Omega_L, \Omega_R}{U}}
{\oseq{\Gamma}{\Delta}{\Omega_L, {A}, \Omega_R}{U}}
\]
When this notation is used in the rule ${\it id}$, the meaning is that
the atomic proposition $p$ is either the only thing in the ordered
context alongside an empty linear context, or else it is the 
only thing in the linear context alongside an empty ordered context, or
else both the linear and ordered contexts are empty and $p$ is present in the
persistent context. This notion will be called {\it sole membership} 
in Section~\ref{sec:fundamental-operations-on-contexts}.

Ordered linear lax logic also encompasses Fairtlough and Mendler's lax
logic \cite{fairtlough95propositional} as reconstructed by Pfenning
and Davies \cite{pfenning01judgmental} and adapted to linear logic is
the basis of the CLF logical framework \cite{watkins02concurrent}.
The judgment $\islax{A}$ is the foundation of ordered logic, and is
usually interpreted as truth under some unspecified constraint
or as a weaker version of truth: if we know $\isconc{A}$
then we can conclude $\islax{A}$, but the opposite entailment
does not hold. 
\[
\infer-[{\it lax}]
{\oseq{\Gamma}{\Delta}{\Omega}{\islax{A}}}
{\oseq{\Gamma}{\Delta}{\Omega}{\isconc{A}}}
\]
If we know $\islax{A}$, on the other hand, we cannot prove $\isconc{A}$,
though we can prove
$\isconc{{\ocircle}A}$, where ${\ocircle}A$ is the propositional
internalization of the lax judgment (rule ${\ocircle}_R$ in
Figure~\ref{fig:ordered-prop}). % Furthermore, we can 

Lax truth is handled with the use of matching constructs, thereby
making the rule ${\it lax}$ rule above admissible just like ${\it
  copy}$ and ${\it place}$ are admissible.  We write all the right
rules in Figure~\ref{fig:ordered-prop} with a construct
$\orseq{\Gamma}{\Delta}{\Omega}{A}$ that matches both sequents of the
form $\otseq{\Gamma}{\Delta}{\Omega}{A}$ and sequents of the form
$\oseq{\Gamma}{\Delta}{\Omega}{\islax{A}}$ -- in other words, we treat
$\mlvl$ as a metavariable (``level'') that stands for judgments $\mconc$ or
$\mlax$ on the right.  The use of this construct gives us right rules
for $A \oplus B$ that look like this:
\[
\infer[{\oplus}_{R1}]
{\orseq{\Gamma}{\Delta}{\Omega}{A \oplus B}}
{\otseq{\Gamma}{\Delta}{\Omega}{A}}
\qquad
\infer[{\oplus}_{R2}]
{\orseq{\Gamma}{\Delta}{\Omega}{A \oplus B}}
{\otseq{\Gamma}{\Delta}{\Omega}{B}}
\]

The metavariable $U$ is even more generic, standing
for an arbitrary succedent $\isconc{A}$ or $\islax{A}$. 
Putting the pieces of ordered linear lax logic together into one 
sequent calculus in Figure~\ref{fig:ordered-prop}
is a relatively straightforward proof-theoretic
exercise; the language of propositions is as follows:
\begin{align*}
\mbox{Propositions} & &
A, B, C  & ::= p \mid {\gnab}A \mid {!}A \mid {\ocircle}A 
   \mid \\ & & & \qquad \one \mid A \fuse B \mid A \lefti B \mid A \righti B  
   \mid \zero \mid A \oplus B \mid \top \mid A \with B 
   \mid \\ & & & \qquad\exists \lf{a}{:}\tau. A \mid \forall \lf{a}{:}\tau 
   \mid \lf{t} \doteq_\tau \lf{s}
\end{align*}
The connectives and $\one$, $\zero$, $A \oplus B$, $\top$, and
$A \with B$
were not mentioned above but are closely analogous to their
counterparts in linear logic. The first-order connectives $\exists
\lf{a}{:}\tau. A$, $\forall \lf{a}{:}\tau$, and $\lf{t}
\doteq_\tau \lf{s}$ will be discussed presently.

\subsection{First-order logic}
\label{sec:firstorderlogic}

\begin{figure}
\[
\infer[{\exists}_R]
{\orfseq{\Gamma}{\Delta}{\Omega}{\exists \lf{a}{:}\tau.B}}
{\Psi \vdash \lf{t} : \tau 
 &
 \otseq{\Gamma}{\Delta}{\Omega}{\lf{[t/a]}B}}
\quad
\infer[{\exists}_L]
{\olfseq{\Gamma}{\Delta}{\Omega_L}{\exists \lf{a}{:}\tau.B}{\Omega_R}}
{\ofirstseq{\Psi,\lf{a}{:}\tau}{\Gamma}{\Delta}{\Omega_L,B,\Omega_R}{U}}
\]
\[
\infer[{\forall}_R]
{\orfseq{\Gamma}{\Delta}{\Omega}{\forall \lf{a}{:}\tau.B}}
{\ofirstseq{\Psi,\lf{a}{:}\tau}{\Gamma}{\Delta}{\Omega}{\isconc{B}}}
\quad
\infer[{\forall}_L]
{\olfseq{\Gamma}{\Delta}{\Omega_L}{\forall \lf{a}{:}\tau.B}{\Omega_R}}
{\Psi \vdash \lf{t} : \tau
 &
 \opfseq{\Gamma}{\Delta}{\Omega_L}{\lf{[t/a]}B,\Omega_R}}
\]
\[
\infer[{\doteq}_R]
{\orfseq{\Gamma}{\Delta}{\Omega}{\lf{t} \doteq_\tau \lf{t}}}
{}
\]
\[
\infer[{\doteq}_L]
{\olfseq{\Gamma}{\Delta}{\Omega_L}{\lf{t} \doteq_\tau \lf{s}}{\Omega_R}}
{\forall(\Psi' \vdash \lf{\sigma} : \Psi). 
 &
 \lf{\sigma t} = \lf{\sigma s}
 &
 \longrightarrow
 &
 \ofirstseq{\Psi'}{\lf{\sigma}\Gamma}{\lf{\sigma}\Delta}{\lf{\sigma}\Omega_L,\lf{\sigma}\Omega_R}{\lf{\sigma} U}}
\]

\caption{First-order ordered linear lax logic}
\label{fig:ordered-fo}
\end{figure}


The presentation in Figure~\ref{fig:ordered-prop} is propositional; by
uniformly adding a first-order context $\Psi$ to all sequents it can
be treated as first-order. We define quantification (existential and
universal), as well as first-order equality,\footnote{That is,
  equality of terms from the domain of first-order quantification.} in
Figure~\ref{fig:ordered-fo}.

The equality proposition $\lf{t} \doteq_\tau \lf{s}$ is an interesting
addition to our presentation of the logic, and will be present in the
framework \sls~defined in the next chapter, albeit in a highly
restricted form. Equality in \sls~will be used primarily in the
logical transformations presented in Chapter~\ref{chapter-destinations} 
and in the program
analysis methodology in Chapter~\ref{chapter-approx}. 
The left rule for equality $\lf{t}
\doteq_\tau \lf{s}$ has a higher-order premise, in the sense that it
reflects over the definition of simultaneous term substitutions $\Psi'
\vdash \lf{\sigma} : \Psi$ and over the syntactic equality judgment
for first-order terms $\lf{t} = \lf{s}$.  We used this exact style of
presentation previously in \cite{simmons11weak}, but the approach is
based on Schroeder-Heister's treatment of definitional reflection
\cite{schroeder93rules}.

In one sense, the left rule $\doteq_L$ is actually a rule schema:
there is one premise for each substitution $\lf\sigma$ that is a
unifier for $\lf{t}$ and $\lf{s}$ (a unifier is any substitution
$\lf\sigma$ that makes $\lf{\sigma t}$ and $\lf{\sigma s}$
syntactically identical). When we induct over the structure of proofs,
there is correspondingly one smaller subderivation for each unifying
substitution.  By this reading, $\doteq_L$ is a rule that, in general,
will have countably many premises; in the case of a trivially
satisfiable equality problem like $\lf{x} \doteq_\tau \lf{x}$ it will have
one premise for each well-formed substitution that substitutes a term
of the appropriate type for $\lf{x}$.  However, as suggested by
Zeilberger \cite{zeilberger08focusing}, it is more auspicious to take
the higher-order formulation at face value: the premise is actually a
(meta-level) mapping -- a function -- that takes a substitution
$\lf{\sigma}$, the codomain $\Psi'$ of that substitution, and any
evidence necessary to show that $\lf{\sigma}$ unifies $\lf{t}$ and
$\lf{s}$ and returns a derivation of
$\ofirstseq{\Psi'}{\lf{\sigma}\Gamma}{\lf{\sigma}\Delta}{\lf{\sigma}\Omega_L,\lf{\sigma}\Omega_R}{\lf{\sigma}
  U}$. When we induct over the structure of proofs, the result of
applying any unifying substitution to this function is a smaller
subderivation for the purpose of invoking the induction
hypothesis. This functional interpretation will be reflected in the
proof terms we assign to focused \ollll~in
Section~\ref{sec:ord-proof-terms}.


There are two important special cases. First, an unsatisfiable
equation on the left implies a contradiction, and makes the left rule
for equality equivalent (at the level of provability) to one with no
premises. For instance, this means that
\[
\infer[\doteq_{\it no}]
{\olfseq{\Gamma}{\Delta}{\Omega_L}{\lf{t} \doteq_\tau \lf{s}}{\Omega_R}}
{\mbox{\it no~unifier~for~} \lf{t} \mbox{\it ~and~} \lf{s}}
\]
is derivable -- a {\it unifier} is just a substitution $\lf{\sigma}$
such that $\lf{\sigma t}$ and $\lf{\sigma s}$ are syntactically
identical.  The other important special case is when $\lf{t}$ and
$\lf{s}$ have a {\it most general unifier} $\lf{\sigma_{\it mgu}}$,
which just means that for all $\Psi' \vdash \lf{\sigma} : \Psi$ such
that $\lf{\sigma t} = \lf{\sigma s}$, it is the case that $\lf{\sigma}
= \lf{\sigma' \circ \sigma_{\it mgu}}$ for some
$\lf{\sigma'}$.\footnote{Where $\lf\circ$ is composition --
  $\lf{(\sigma' \circ \sigma_{\it mgu})t} = \lf{\sigma'(\sigma_{\it
      mgu} t)}$.} In this case, the left rule for equality is
equivalent (again, at the level of determining which sequents are
provable) to the following rule:
\[
\infer[\doteq_{\it yes}]
{\olfseq{\Gamma}{\Delta}{\Omega_L}{\lf{t} \doteq_\tau \lf{s}}{\Omega_R}}
{\lf{\sigma} = {\it mgu}(\lf{t}, \lf{s})
 &
 \Psi' \vdash \lf{\sigma} : \Psi
 &
 \ofirstseq{\Psi'}{\lf{\sigma}\Gamma}{\lf{\sigma}\Delta}{\lf{\sigma}\Omega_L, \lf{\sigma}\Omega_R}{\lf{\sigma} U}}
\]
Therefore, given a first-order domain in which any two terms are
decidably either non-unifiable or unifiable with a most general
unifier, we can choose to define the logic with two rules $\doteq_{\it
  no}$ and $\doteq_{\it yes}$; the resulting logic will be equivalent, 
at the level of derivable sequents, to the logic defined with the $\doteq_L$
rule.

We have not yet thoroughly specified the type and term structure of
first-order individuals; in Section~\ref{sec:sls-termlanguage} we
clarify that these types and terms will actually be types and terms of
Spine Form LF. This does mean that we will have to pay attention, in
the proofs of this chapter, to the fact that the types of first-order
terms $\tau$ are {\it dependent types} that may include terms
$\lf{t}$. Particularly relevant in this chapter will be simultaneous
substitutions $\lf\sigma$: the judgment $\Psi' \vdash \lf\sigma :
\Psi$ expresses that $\lf\sigma$ can map terms and propositions
defined in the context $\Psi$ (the domain of the substitution) to
terms and propositions defined in the context $\Psi'$ (the range of
the substitution). Simultaneous substitutions are defined more
carefully in Section~\ref{sec:lf-simpletypesandhsubst} and in
\cite{nanevski08contextual}.

\section{Substructural contexts}
\label{sec:contexts}

First-ordered linear lax logic has a lot of contexts -- the persistent
context $\Gamma$, the linear context $\Delta$, the ordered context
$\Omega$, and the first-order context $\Psi$. In most presentations
of substructural logics, the many
contexts primarily serve to obscure the logic's presentation
and ensure that the {\LaTeX} code of figures and displays remains
permanently unreadable. And there are yet more contexts we might want to 
add, such as the affine contexts present in the Celf implementation
\cite{schacknielsen08celf}.

In this section, we will consider a more compact way of dealing with
the contexts that we interpret as containing resources (persistent,
affine, linear, or ordered resources), though we choose to maintain
the distinction between resource contexts and first-order variable
contexts $\Psi$.  The particular way we define substructural contexts
can be generalized substantially: it would be possible to extend this
presentation to the affine exponential ${@}A$, and we conjecture that
the subexponentials discussed by Nigam and Miller
\cite{nigam09algorithmic} as well as richer logics like the logic of
bunched implications \cite{pym02semantics} could be given a
straightforward treatment using this notation.

We write unified substructural contexts as either $\Delta$ or $\Xi$,
preferring the latter when there is a chance of confusing them with
linear contexts $\Delta$. For the purposes of encoding \ollll, we can
see these contexts as sequences of variable declarations, defined by
the grammar
\[
\Xi ::= \cdot 
  \mid \Xi, x{:}\istrue{T}
  \mid \Xi, x{:}\iseph{T}
  \mid \Xi, x{:}\ispers{T}
\]
where each of the {\em variables} $x$ are distinct, so that the
context also represents a finite map from variables $x$ to {\it
  judgments} $\islvl{T}$, where $\mlvl$ is either $\mtrue$, $\meph$,
or $\mpers$.  By separating out a substructural context into three
subsequences of persistent, linear, and ordered judgments, we can
recover the presentations of contexts for \ollll~given in
Figure~\ref{fig:ordered-prop}. We will use this observation in an
informal way throughout the chapter, writing $\Xi = \Gamma; \Delta;
\Omega$.

The domain represented by the metavariable $T$ is arbitrary: when
discussing the unfocused logic given in Figure~\ref{fig:ordered-prop},
$T$ varies over unpolarized propositions $A$, but when discussing a
focused logic in Section~\ref{sec:ord-focused} it will vary over
stable negative propositions $A^-$, positive suspended propositions
$\susp{A^+}$, focused negative propositions $[A^-]$, and inverting
positive propositions $A^+$.



% The first step towards this understanding 
% has already been take in Figure~\ref{fig:linear-alt} and
% Figure~\ref{fig:ordered-prop}, which make it quite obvious that the
% context-{\it matching} notation that we perform in the conclusion of
% an inference rule may not be the same as the context-{\it extending}
% notation we use in the premise of that rule.

The key innovation in this presentation was already present in the
unfocused logic shown in Figure~\ref{sec:ord-unfocused}: we need to
differentiate {\it constructions}, which appear in the premises of
rules, and {\it matching constructs}, which appear in the conclusions
of rules.  The notation $\Gamma; \Delta; \Omega_L/A\fuse B/\Omega_R$
that appears in the conclusion of ${\fuse}_L$ is a matching construct;
as discussed in Section~\ref{sec:ord-unfocused}, there are multiple
ways in which a context $\Gamma'; \Delta'; \Omega'$ could match this
context, because $A \fuse B$ could come from any of the three
contexts. However, $\Gamma; \Delta; \Omega_L,{A},{B}, \Omega_R$ in the
premise of ${\fuse}_L$ is a construction, and is unambiguously equal
to only one context $\Gamma'; \Delta'; \Omega'$ -- the one where
$\Gamma' = \Gamma$, $\Delta' = \Delta$, and 
$\Omega' = \Omega_L, {A}, {B}, \Omega_R$.

\subsection{Fundamental operations on contexts}
\label{sec:fundamental-operations-on-contexts}

The first fundamental idea we consider is {\it singleton} contexts.
We construct a single-element context by writing $x{:}\islvl{T}$.
The corresponding matching construct on contexts is 
$x{:}{T}$. In unfocused \ollll, we say that $\Xi$ matches 
$x{:}{A}$ if its decomposition into persistent, linear, and 
ordered contexts matches $\Gamma; \cdot; /A/$. Specifically,

\bigskip
\begin{definition}[Sole membership]
  $\Xi$ matches $x{:}T$ if
\begin{itemize}
\item $\Xi$ contains no linear judgments and contains exactly
one
ordered judgment $x{:}\istrue{T}$ (corresponding to the situation where
$\Xi = \Gamma; \cdot; T$), 
\item $\Xi$ contains no ordered judgments and contains exactly
one linear judgment $x{:}\iseph{T}$ (corresponding to the situation where
$\Xi = \Gamma; T; \cdot$), or 
\item $\Xi$ contains only persistent judgments, including
$x{:}\ispers{T}$ (corresponding to the situation where
$\Xi = \Gamma, T; \cdot; \cdot$). 
\end{itemize}
\end{definition}
\bigskip

Sole membership is related to the initial sequents and the 
matching construct $\Gamma; \cdot;/A/$ for contexts that was used
in Figure~\ref{fig:ordered-prop}.
We could rewrite the {\it id} rule
from that figure as follows:
\[
\infer[{\it id}]
{x{:}p \altv \islvl{p}}
{}
\]
As with all rules involving matching constructs in the conclusion,
it is fair to view the matching construct as an extra premise; thus,
the ${\it id}$ rule above is the same as the ${\it id}$ rule 
below:
\[
\infer[{\it id}]
{\Xi \altv \islvl{p}}
{\Xi \mbox{\it ~matches~} x{:}p}
\]

The second basic operation on contexts requires a new concept, {\it
  frames} $\Theta$. Intuitively, we can view a frame as a the
complete set of persistent, linear, and ordered contexts 
where the ordered context is
missing a particular piece.  We can write this missing
piece as a box: $\Gamma; \Delta; \Omega_L, \Box,
\Omega_R$. Alternatively, we can think of a frame as a one-hole context
or Huet-style zipper \cite{huet97zipper} over the structure of
substructural contexts. We will also think of them morally as linear
functions $(\lambda\Xi.\, \Xi_L, \Xi, \Xi_R)$ as in
\cite{simmons09linear}.

The construction associated with frames, $\tackon{\Theta}{\Xi}$,
is just a straightforward operation of filling in the hole or 
$\beta$-reducing the linear function; doing this requires that the 
variables in $\Theta$ and $\Xi$ be distinct. If we
think of $\Theta$ informally as $\Gamma; \Delta; \Omega_L, \Box,
\Omega_R$, then this is {\it almost} like the operation of filling in
the hole, as $\tackon{\Theta}{x{:}\istrue{A}} = \Gamma; \Delta;
\Omega_L, A, \Omega_R$. The main difference is that we can also use
the operation to insert linear propositions
($\tackon{\Theta}{x{:}\iseph{A}} = \Gamma; \Delta, A; \Omega_L,
\Omega_R$) and persistent propositions
($\tackon{\Theta}{x{:}\ispers{A}} = \Gamma, A; \Delta; \Omega_L,
\Omega_R$).

The construction associated with frames is straightforward, but the
matching construct associated with frames is a bit more complicated.
Informally, if we treat linear contexts as multisets and say that $\Xi
= \Gamma; \Delta, \Delta'; \Omega_L, \Omega', \Omega_R$, then we can
say $\Xi = \frameoff{\Theta}{\Xi'}$ in the case that $\Theta = \Gamma;
\Delta; \Omega_L, \Box, \Omega_R$ and $\Xi' = \Gamma; \Delta';
\Omega'$. The sub-context $\Xi'$, then, has been {\it framed off} from
$\Xi$, its frame is $\Theta$. If we only had ordered judgments
$\istrue{T}$, then the framing-off matching construct
$\frameoff{\Theta}{\Xi'}$ would be essentially the same as the
construction form $\tackon{\Theta}{\Xi'}$. However, persistent and
linear judgments can be reordered in the process of matching, and
persistent judgments always end up in both the frame and the 
framed-off context. 

\bigskip
\begin{definition}[Framing off]
$\Xi$ matches $\frameoff{\Theta}{\Xi'}$ if the union of the variables in 
$\Theta$ and $\Xi'$ is exactly the variables in $\Xi$ and
\begin{itemize}
\item if $x{:}\ispers{T} \in \Xi$, then the same variable declaration
  appears in $\Theta$ and $\Xi'$;
\item if $x{:}\iseph{T} \in \Xi$ or $x{:}\istrue{T} \in \Xi$, then the
  same variable declaration appears in $\Theta$ or $\Xi'$ (but not
  both);
\item in both $\Theta$ and $\Xi'$, 
  the sequence of variable declarations $x{:}\istrue{T}$ 
  is a subsequence of $\Xi$; and 
\item if $x{:}\istrue{T} \in \Theta$, then either
  \begin{itemize}
  \item for all $y{:}\istrue{T'} \in \Xi'$, the variable declaration
    for $x$ appeared before the variable declaration for $y$ in $\Xi$,
    or
  \item for all $y{:}\istrue{T'} \in \Xi'$, the variable declaration
    for $x$ appeared after the variable declaration for $y$ in $\Xi$.
  \end{itemize}
\end{itemize}
\end{definition}
\bigskip

We can use the framing-off notation to describe one of the
cut principles for ordered linear lax logic as follows:
\[
\infer-[{\it cut}]
{\frameoff{\Theta}{\Xi} \altv \isconc{C}}
{\Xi \altv \isconc{A}
 &
 \tackon{\Theta}{x{:}A\,{\it true}} \altv \isconc{C}}
\]
Especially for the eventual proof of this cut principle, it is
important to consider that the admissible rule above is equivalent to
the following admissible rule, which describes the matching as an
explicit extra premise:
\[
\infer-[{\it cut}]
{\Xi' \altv \isconc{C}}
{\Xi \altv \isconc{A}
 &
 \tackon{\Theta}{x{:}A\,{\it true}} \altv \isconc{C}
 &
 \Xi' \textit{~matches~} \frameoff{\Theta}{\Xi}}
\]

\futurework{
As an aside: it need not always be the case that the same operation
used to describe 

The idea that the operators $\frameoff{\Theta}{\Xi}$
and $\tackon{\Theta}{\Xi}$ are sufficient to describe the 
cut principle is related to the display property, which fails
for some reasonable logics, such as Reed's queue logic
\cite{reed09queue}.}

An important derived matching construct is $\frameoff{\Theta}{x{:}T}$,
which matches $\Xi$ if $\Xi$ matches $\frameoff{\Theta}{\Xi'}$ for
some $\Xi'$ such that $\Xi'$ matches $x{:}T$.  This notation is
equivalent to the matching construct
$\olseq{\Gamma}{\Delta}{\Omega_L}{A}{\Omega_R}$ from
Figure~\ref{fig:ordered-prop}, which is need to describe
almost every left rule for \ollll. Here are three rules given with
this matching construct:
\[
\infer[]
{\frameoff{\Theta}{x{:}A \oplus B} \altv U}
{\tackon{\Theta}{y{:}\istrue{A}} \altv U
 &
 \tackon{\Theta}{z{:}\istrue{B}} \altv U}
\quad
\infer[]
{\frameoff{\Theta}{x{:}A \with B} \altv U}
{\tackon{\Theta}{y{:}\istrue{A}} \altv U}
\quad
\infer[]
{\frameoff{\Theta}{x{:}A \with B} \altv U}
{\tackon{\Theta}{y{:}\istrue{B}} \altv U}
\]

To reemphasize, the reason we use the matching construct 
$\frameoff{\Theta}{x{:}A}$ in the conclusions of rules is the same reason
that we used the notation $\Gamma; \Delta; \Omega_L/A/\Omega_R$ in 
Figures~\ref{fig:ordered-prop}~and~\ref{fig:ordered-fo}: it allows us 
to generically talk about hypotheses associated with the judgments
$\mtrue$, $\meph$, and $\mpers$. The following rules are all derivable
using the last of the three rules above:
\[
\infer[]
{\tackon{\Theta}{x{:}\istrue{A \with B}} \altv U}
{\tackon{\Theta}{y{:}\istrue{B}} \altv U}
\quad
\infer[]
{\tackon{\Theta}{x{:}\iseph{A \with B}} \altv U}
{\tackon{\Theta}{y{:}\istrue{B}} \altv U}
\quad
\infer[]
{\tackon{\Theta}{x{:}\ispers{A \with B}} \altv U}
{\tackon{\Theta}{\mkconj{x{:}\ispers{A \with B}}{y{:}\istrue{B}}} \altv U}
\]

The consistent use of matching constructs like
$\frameoff{\Theta}{\Delta}$ in the conclusion of rules is also what
gives us the space to informally treat syntactically 
distinct sequences of variable
declarations as equivalent. As an example, we can think of $(x{:}\iseph{A},
y{:}\iseph{B})$ and $(y{:}\iseph{B}, x{:}\iseph{A})$ as equivalent by
virtue of the fact that they satisfy the same set of matching
constructs. Obviously, this means that none of the matching constructs
presented in the remainder of this section will observe
the ordering of ephemeral or persistent variable declarations.


\subsection{Multiplicative operations}

To describe the multiplicative connectives of \ollll, including the critical
implication connectives, we need to have multiplicative operations on 
contexts. As a construction, $\mkconj{\Xi_L}{\Xi_R}$ is just the
syntactic concatenation of two contexts with distinct variable domains, and
the unit $\mkunit$ is just the empty sequence. The matching constructs
are more complicated to define, but the intuition is, again, 
uncomplicated: if 
$\Xi = \Gamma; \Delta, \Delta'; \Omega_L, \Omega_R$, where linear contexts
are multisets and ordered contexts are sequences, then 
$\Xi = \mkconj{\Xi_L}{\Xi_R}$ if $\Xi_L = \Gamma; \Delta; \Omega_L$ and
$\Xi_R = \Gamma; \Delta'; \Omega_R$. Note that here we are using the same 
notation for constructions and matching constructs: 
$\matchconj{\Xi_L}{\Xi_R}$ is a matching construct when it appears
in the conclusion of a rule, $\mkconj{\Xi_L}{\Xi_R}$ is a construction
when it appears in the premise of a rule.

\bigskip
\begin{definition}[Conjunction]~\smallskip\\
$\Xi$ matches $\matchunit$ if $\Xi$ contains only
persistent judgments.

\smallskip
\noindent
$\Xi$ matches $\matchconj{\Xi_L}{\Xi_R}$ if the union 
of the variables in $\Xi_L$ and $\Xi_R$ is exactly the variables in $\Xi$
and 
\begin{itemize}
\item if $x{:}\ispers{T} \in \Xi$, then the same variable declaration appears in $\Xi_L$
  and $\Xi_R$;
\item if $x{:}\iseph{T} \in \Xi$ or $x{:}\istrue{T} \in \Xi$, then the
  same variable declaration appears in $\Xi_L$ or $\Xi_R$ (but not both); 
\item in both $\Xi_L$ and $\Xi_R$, 
  the sequence of variable declarations $x{:}\istrue{T}$ 
  is a subsequence of $\Xi$; and 
\item if $x{:}\istrue{T} \in \Xi_L$ and $y{:}\istrue{T'} \in \Xi_R$,
  then the variable declaration for $x$ appeared before the variable
  declaration for $y$ in $\Xi$.
\end{itemize}
\end{definition}
\bigskip

The constructs for context conjunction are put to obvious use
in the description of multiplicative conjunction, which is essentially
just the propositional internalization of context conjunction:
\[
\infer
{\matchconj{\Xi_L}{\Xi_R} \altv \islvl{A \fuse B}}
{\Xi_L \altv \isconc{A} & \Xi_R \altv \isconc{B}}
\quad
\infer
{\frameoff{\Theta}{x{:}A \fuse B} \altv U}
{\tackon{\Theta}{\mkconj{y{:}A}{z{:}B}} \altv U}
\quad
\infer
{\cdot \altv \islvl{\one}}
{}
\quad
\infer
{\frameoff{\Theta}{x{:}\one} \altv U}
{\tackon{\Theta}{\cdot} \altv U}
\]\[
\infer
{\Xi \altv \islvl{A \lefti B}}
{x{:}\istrue{A}, \Xi \altv \isconc{B}}
\quad
\infer
{\frameoff{\Theta}{\Xi_A, x{:}A \lefti B} \altv U}
{\Xi_A \altv \isconc{A} & \tackon{\Theta}{y{:}\istrue{B}} \altv U}
\]\[
\infer
{\Xi \altv \islvl{A \righti B}}
{\Xi, x{:}\istrue{A} \altv \isconc{B}}
\quad
\infer
{\frameoff{\Theta}{x{:}A \righti B, \Xi_A} \altv U}
{\Xi_A \altv \isconc{A} & \tackon{\Theta}{y{:}\istrue{B}} \altv U}
\]

Implication makes deeper use of context conjunction:
% we can look at 
$\Xi$ matches
$\frameoff{\Theta}{\matchconj{\Xi_A}{x{:}A \lefti B}}$ 
%in two ways.
%One way of reading this notation 
exactly when there exist $\Xi'$ and $\Xi''$ such that 
% is 
$\Xi$ matches $\frameoff{\Theta}{\Xi'}$, 
$\Xi'$ matches $\matchconj{\Xi_A}{\Xi''}$,
and $x{:}A \lefti B$ matches $\Xi''$. 

% The other
% way of reading the notation 
% is that $\Xi$ matches $\frameoff{\Theta'}{x{:}A \lefti B}$, and
% $\Theta'$ matches $\frameoff{\Theta}{\matchconj{\Xi_A}{\Box}}$, 
% It is critical that these two ways of 

% \bigskip
% \begin{definition}[Matching into frames]
% $\Theta$ matches $\frameoff{\Theta'}{\matchconj{\Xi_A}{\Box}}$ if, 
% for all $\Xi'$ and $\Xi_A$, $\Xi'$ matches $\matchconj{\Xi_A}{\Xi_B}$
% if and only if 

% For all $\Xi$, $\Xi_A$, $\Xi_B$ and $\Theta$, 
% \begin{itemize}
% \item there exists $\Xi'$ such that $\Xi$ matches $\frameoff{\Theta}{\Xi}$
%   and $\Xi$ matches $\matchconj{\Xi_A}{\Xi_B}$
%   if and only if there exists $\Delta'$ such that $\Xi$ matches 
%   $\frameoff{\Theta}{\Xi_A}$ and $\Theta$ matches 
%   $\frameoff{\Theta'}{}$
% \end{itemize}
% \end{definition}


% As a matching construct, $\Xi = \matchconj{\Xi_L}{\Xi_R}$ if every 
% $x{:}\ispers{T}$ in $\Xi$ appears in both $\Xi_L$ and $\Xi_R$, every 
% $x{:}\iseph{T}$ or $x{:}\istrue{T}$ 
% in $\Xi$ appears in exactly one of $\Xi_L$ or $\Xi_R$, and
% if furthermore every 
% $x{:}\iseph{T}$ $\Xi_L$


% As a construction form, 


\subsection{Exponential operations}

The exponentials ${!}$ and ${\gnab}$ do not have a construction form
associated with them, unless we view the singleton
construction forms $x{:}\ispers{T}$ and $x{:}\iseph{T}$ as
being associated with these exponentials. The matching
construct is quite simple: $\Xi$ matches $\restrictto{\Xi}{\mpers}$ if
$\Xi$ contains no ephemeral or ordered judgments -- in other words,
it says that $\Xi = \Gamma; \cdot; \cdot$. This form can then be used
to describe the right rule for ${!}A$ in unfocused \ollll:
\[
\infer
{\restrictto{\Xi}{\mpers} \altv \islvl{{!}A}}
{\Xi \altv \isconc{A}}
\]
Similarly, $\Xi$ matches $\restrictto{\Xi}{\meph}$ if $\Xi$ contains
no ordered judgments (that is, if $\Xi = \Gamma; \Delta; \cdot$).
$\Xi$ always matches $\restrictto{\Xi}{\mtrue}$; we don't ever
explicitly use this construct, but it allows us to generally refer to
$\restrictto{\Xi}{\mlvl}$ in the statement of theorems like cut admissibility.

The exponential matching constructs 
don't actually modify contexts in the way
other matching constructs do, but this is a consequence of the 
particular choice of logic we're considering. Given affine resources,
for instance, the matching construct associated with the 
affine connective ${@}A$ would
clear the context of affine facts: 
$\Xi$ matches $\restrictto{\Xi'}{\mpers}$ if 
$\Xi$ has only persistent and affine resources and $\Xi'$ 
contains the same persistent resources as $\Xi$ but none of the affine ones.

We can describe a mirror-image operation on succedents $U$.  $U$
matches $\restrictfrom{U}{\mlax}$ only if it has the form $\islax{T}$,
and $U$ always matches $\restrictfrom{U}{\mconc}$. The latter matching
construct is another degenerate form that similarly allows us to refer
to $\restrictfrom{U}{\mlvl}$ as a generic matching construct. We write
$\restrictto{\Delta}{\mlvl}$ as a judgment to mean that 
$\Delta$ matches $\restrictto{\Delta}{\mlvl}$, 
and write $\restrictfrom{U}{\mlvl}$
as a judgment to mean that $U$ matches $\restrictfrom{U}{\mlvl}$.

The context constructions and context matching constructs
that we have given are summarized as follows:
\begin{align*}
& \mbox{Constructions} & 
\Delta, \Xi & ::= x{:}\islvl{T} \mid \tackon{\Theta}{\Delta} \mid 
  \cdot \mid \Delta, \Xi
\\
\qquad\qquad & \mbox{Matching constructs} &
\Delta, \Xi & ::= x{:}T \mid \frameoff{\Theta}{\Delta} \mid \cdot \mid
  \Delta, \Xi \mid \restrictto{\Delta}{\mlvl} \qquad\qquad
\end{align*}

\section{Focused sequent calculus}
\label{sec:ord-focused}

\begin{figure}
\begin{align*}
&
 & & \mbox{In the context $\Delta$}
 & & \mbox{As the succedent $U$}
\\
& \mbox{\it stable propositions}
 & & x{:}\istrue{A^-}, \meph, \mpers
 & & \isconc{A^+}, \mlax
\\
& \mbox{\it suspended propositions (also stable)}
 & & x{:}\istrue{\susp{A^+}}, \meph, \mpers
 & & \isconc{\susp{A^-}}, \mlax
\\
& \mbox{\it focused propositions}
 & & \textcolor{grayout}{x{:}\istrue{\no{[A^-]}}}
 & & \textcolor{grayout}{\isconc{\no{[A^+]}}}
\\
& \mbox{\it inverting propositions}
 & & \textcolor{grayout}{x{:}\istrue{\no{A^+}}}
 & & \textcolor{grayout}{\isconc{\no{A^-}}}
\end{align*}
\caption{Summary of where propositions and judgments appear in \ollll~sequents}
\label{fig:where-do-they-go}
\end{figure}

A sequent in the focused sequent calculus presentation of \ollll~has
the form $\foc{\Psi}{\Delta}{U}$, where $\Psi$ is the first-order
variable context, $\Delta$ is a substructural context as described in
the previous section, and $U$ is a succedent. The domain $T$ of the
substructural context consists of stable negative propositions $A^-$,
positive suspended propositions $\susp{A^+}$, focused negative
propositions $[A^-]$, and inverting positive propositions $A^+$.

The form of the succedent $U$ is $\islvl{T}$, where $\mlvl$ is either
$\mconc$ or $\mlax$; in this way, $U$ is just a like a special
substructural context with exactly one element -- we don't need
to care about the name of the variable, because there's only one.  The
domain of $T$ for succedents is complementary to the domain of $T$
for contexts: stable positive propositions $A^+$, negative suspended
propositions $\susp{A^-}$, focused positive propositions $[A^+]$, and
inverting negative propositions $A^-$.

Figure~\ref{fig:where-do-they-go} summarizes the composition of
contexts and succedents, taking into account the restrictions
discussed below.

\subsection{Restrictions on the form of sequents}

A sequent $\foc{\Psi}{\Delta}{U}$ is {\it stable} when the context
$\Delta$ and succedent $U$ contain only stable propositions ($A^-$ in
the context, $A^+$ in the succedent) and suspended propositions
($\susp{A^+}$ in the context, $\susp{A^-}$ in the succedent). We adopt
the focusing constraint discussed in Chapter~\ref{chapter-foc}: 
there is only ever at
most one focused proposition in a sequent, and if there is focused
proposition in the sequent, then the sequent is otherwise stable. A
restriction on the rules ${\it focus}_L$ and ${\it focus}_R$
(presented below in Figure~\ref{fig:foc-mall}) is sufficient to enforce this
restriction: reading rules from top down, we can only use a rule ${\it
  focus}_L$ or ${\it focus}_R$ to prove a stable sequent, and reading
rules from bottom up, we can only apply ${\it focus}_L$ or ${\it
  focus}_R$ when we are searching for a proof of a stable sequent.

Because there is always a distinct focused proposition in a sequent,
we do not need a variable name to reference the focused proposition in
a context $\Delta$ any more than we need a variable name to reference
the unique member of the context-like succedent $U$. Therefore, we
can write $\istrue{[B^-]}$ instead of $x{:}\istrue{[B^-]}$. Furthermore,
for presentation of focusing that we want to give it suffices to 
restrict focused propositions and inverting propositions so that
they are always associated with the judgment $\mtrue$ (on the left)
or $\mconc$ (on the right). With this
restriction, we can write $[A^-]$ and $x{:}A^+$ instead of
$\istrue{[A^-]}$ and $x{:}\istrue{A^+}$ in $\Delta$, and we can write
$[A^+]$ and $A^-$ instead of $\isconc{[A^+]}$ and $\isconc{A^-}$ for
$U$.

In a confluent presentation of focused logic like the one given for
linear logic in Chapter~\ref{chapter-foc}, 
that would be as far as we could take our
simplifications. However, this presentation will use a fixed
presentation of logic from the structural focalization development
as described in Section~\ref{sec:confluent-v-fixed}. 
If there is more than one
invertible proposition in a sequent, {\it only} the leftmost one will
be treated as eligible to have a rule or matching applied to it. All the
propositions in $\Delta$ are treated as being to the left of the
succedent $U$, so we always prioritize inversion on positive
propositions in $\Delta$. With this additional restriction, it is
always unambiguous which proposition we are referring to in an
invertible rule, and we write $A^+$ instead of $x{:}A^+$ or
$x{:}\istrue{A^+}$.

We will maintain the notational convention (only) within this chapter that
first-order variables are written as $\lf{a}$, variables associated
with stable negative propositions are written as $x$, and variables
associated with suspended positive propositions are written as 
$z$. 

In summary,  the four forms of sequent in focused \ollll, which we define
the rules for in Section~\ref{sec:ord-proof-terms} below, are:
\smallskip
\begin{itemize}
\item Right focused sequents $\foc{\Psi}{\Delta}{[A^+]}$ (where
  $\Delta$ is stable, containing only variable declarations
  $x{:}\islvl{A^-}$ or $x{:}\islvl{\susp{A^+}}$),
\item Inversion sequents $\foc{\Psi}{\Delta}{U}$ (where $\Delta$ contains
  variable declarations $x{:}\islvl{A^-}$, $x{:}\islvl{\susp{A^+}}$ and
  inverting positive propositions $A^+$ and where $U$ is either 
  $\islvl{A^+}$, $\islvl{\susp{A^-}}$, or an inverting negative
  proposition $A^-$), 
\item Stable sequents, the special case of inversion sequents that
  contain no inverting positive propositions in $\Delta$ or inverting
  negative propositions in $U$.
\item Left focused sequents $\foc{\Psi}{\tackon{\Theta}{[A^+]}}{U}$
  (where $\Theta$ and $U$ are stable -- $\Theta$ contains only
  variable declarations $x{:}\islvl{A^-}$ or $x{:}\islvl{\susp{A^+}}$ and
  $U$ is either $\islvl{A^+}$ or $\islvl{\susp{A^-}}$).
\end{itemize}


\subsection{Polarized propositions}
\label{sec:ordpolarprop}

The propositions of ordered logic are fundamentally sorted into
positive propositions $A^+$ and negative propositions $A^-$; both
classes, and the inclusions between them, 
are shown in Figure~\ref{fig:ordered}. The
positive propositions have a refinement, {\it permeable} propositions
$A^+_\mpers$, that is analogous to the refinement discussed for linear
logic in Section~\ref{sec:permeable}. There is also a more generous
refinement, the {\it mobile} propositions, $A^+_\meph$, for positive
propositions that do not mention ${\downarrow}$ but that may mention
${\gnab}$. We introduce atomic propositions $p^+$ that stand for
arbitrary positive propositions, mobile
atomic propositions $p^+_\meph$ that stand for arbitrary
mobile propositions, and persistent $p^+_\mpers$ that stand for arbitrary
permeable propositions. We treat $A^+_\mtrue$ and $p^+_\mtrue$ as synonymous
with $A^+$ and $p^+$, respectively, which allows us to generically
refer to $A^+_\mlvl$ and $p^+_\mlvl$ in rules like $\eta^+$ and in the
statement of the identity expansion theorem.

\begin{figure}
{\footnotesize \begin{align*}
A^+ & ::= p^+ \mid p^+_\meph \mid p^+_\mpers
        \mid {\downarrow}A^- \mid {\gnab}A^- \mid {!}A^- 
        \mid \one \mid A^+ \fuse B^+ \hspace{17pt}
        \mid \zero \mid A^+ \oplus B^+  \hspace{17pt}
        \mid \exists \lf a{:}\tau. A^+ \hspace{9pt}
        \mid \lf{t} \doteq_\tau \lf{s}
\\
A^+_\meph & ::= \hspace{20pt} p^+_\meph \mid p^+_\mpers
        \hspace{27pt} \mid {\gnab}A^- \mid {!}A^- 
        \mid \one \mid A^+_\meph \fuse B^+_\meph \hspace{5pt}
        \mid \zero \mid A^+_\meph \oplus B^+_\meph \hspace{4pt}
        \mid \exists \lf a{:}\tau. A^+_\meph \hspace{2pt} 
        \mid \lf{t} \doteq_\tau \lf{s}
\\
A^+_\mpers & ::= \hspace{47pt} p^+_\mpers 
         \hspace{53pt} \mid {!}A^- 
        \mid \one \mid A^+_\mpers \fuse B^+_\mpers 
        \mid \zero \mid A^+_\mpers \oplus B^+_\mpers
        \mid \exists \lf a{:}\tau. A^+_\mpers \mid \lf{t} \doteq_\tau \lf{s}
\\
\\
A^- & ::= p^- \mid p^-_\mlax 
        \mid {\uparrow}A^+ \mid {\ocircle}A^+
        \mid A^+ \lefti B^- \hspace{5pt} \mid A^+ \righti B^- \hspace{4pt}
        \mid \top \mid A^- \with B^- \hspace{9pt}
        \mid \forall \lf a{:}\tau.A^-
\\
A^-_\mlax & ::= \hspace{20pt} p^-_\mlax \hspace{28pt}
        \mid {\ocircle}A^+
        \mid A^+ \lefti B^-_\mlax \mid A^+ \righti B^-_\mlax
        \mid \top \mid A^-_\mlax \with B^-_\mlax
        \mid \forall \lf a{:}\tau.A^-_\mlax
\end{align*}}\vspace{-12pt}
\caption{Propositions of polarized \ollll}
\label{fig:ordered}
\end{figure}

Negative propositions also have a refinement, $A^-_\mlax$, for
negative propositions that do not end in an upshift ${\uparrow}A^+$ or
in a negative atomic proposition $p^-$.  This is interesting as a
formal artifact and there is very little overhead involved in putting
it into our development, but the meaning of this 
syntactic class, as well as the
meaning of right-permeable atomic propositions $p^-_\mlax$, is
unclear.  Certainly we do {\it not} want to include such propositions
in our logical framework, as to do so would interfere with our
development of traces as a syntax for partial proofs in
Chapter~\ref{chapter-framework}.

The presentation of the exponentials, and the logic that we now
present, emphasizes the degree to which the shifts ${\uparrow}$ and
${\downarrow}$ have much of the character of exponentials in a focused
substructural logic. The upshift ${\uparrow}A^+$ is like an ordered
variant of the lax truth ${\ocircle}A^+$ that puts no constraints on
the form of the succedent, and the downshift ${\downarrow}A^-$ is
like an ordered variant of the persistent and linear exponentials
${!}A^-$ and ${\gnab}A^-$ that puts no constraints on the form of the
context. This point is implicit in Laurent's dissertation
\cite{laurent02etude}. In that dissertation, Laurent defines the polarized
LLP {\it without} the shifts ${\uparrow}$ and ${\downarrow}$, so that
the only connection points between the polarities are the
exponentials. Were it not for atomic propositions, the resulting logic
would be more persistent than linear,
a point we will return to in Section~\ref{sec:perm-fragments}.

\subsection{Derivations and proof terms}
\label{sec:ord-proof-terms}

\renewcommand{\foct}[4]{{#2} \vdash {#3} : {#4}}

\begin{figure}
\small
{\it Focus, identity, and atomic propositions}

\[
\infer[{\it focus}_R^*]
{\foct{\Psi}{\Delta}{\tfocusr{V}}{\islvl{A^+}}}
{\foct{\Psi}{\Delta}{V}{[A^+]}}
\quad
\infer[{\it focus}_L^*]
{\foct{\Psi}{\frameoff{\Theta}{x{:}{A^-}}}
  {\tfocusl{x}{\Sp}}{U}}
{\foct{\Psi}{\tackon{\Theta}{[A^-]}}{\Sp}{U}}
\]
\vspace{-5pt}
\[
\infer[\eta^+]
{\foct{\Psi}{\frameoff{\Theta}{p^+_\mlvl}}{\tetap{z}{N}}{U}}
{\foct{\Psi}{\tackon{\Theta}{z{:}\islvl{\susp{p^+_\mlvl}}}}{N}{U}}
\quad
\infer[{\it id}^+]
{\foct{\Psi}{z{:}\langle A^+ \rangle}{z}{[A^+]}}
{}
\]
\vspace{-5pt}
\[
\infer[\eta^-]
{\foct{\Psi}{\Delta}{\tetan{N}}{p^-_\mlvl}}
{\foct{\Psi}{\Delta}{N}{\islvl{\susp{p^-_\mlvl}}}}
\quad
\infer[{\it id}^-]
{\foct{\Psi}{[A^-]}{\tnil}{\islvl{\susp{A^-}}}}
{}
\]

\medskip
{\it Shifts and modalities}
\[
\infer[{\downarrow}_R]
{\foct{\Psi}{\Delta}{\tdownr{N}}{[{\downarrow}A^-}]}
{\foct{\Psi}{\Delta}{N}{A^-}}
\quad
\infer[{\downarrow}_L]
{\foct{\Psi}{\frameoff{\Theta}{[{\downarrow}A^-]}}{\tdownl{x}{N}}{U}}
{\foct{\Psi}{\tackon{\Theta}{x{:}\istrue{A^-}}}{N}{U}}
\]
\vspace{-5pt}
\[
\infer[{\gnab}_R]
{\foct{\Psi}{\restrictto{\Delta}{\meph}}{\tgnabr{N}}{[{\gnab}A^-}]}
{\foct{\Psi}{\Delta}{N}{A^-}}
\quad
\infer[{\gnab}_L]
{\foct{\Psi}{\frameoff{\Theta}{[{\gnab}A^-]}}{\tgnabl{x}{N}}{U}}
{\foct{\Psi}{\tackon{\Theta}{x{:}\iseph{A^-}}}{N}{U}}
\]
\vspace{-5pt}
\[
\infer[{!}_R]
{\foct{\Psi}{\restrictto{\Delta}{\mpers}}{\tbangr{N}}{[{!}A^-}]}
{\foct{\Psi}{\Delta}{N}{A^-}}
\quad
\infer[{!}_L]
{\foct{\Psi}{\frameoff{\Theta}{[{!}A^-]}}{\tbangl{x}{N}}{U}}
{\foct{\Psi}{\tackon{\Theta}{x{:}\ispers{A^-}}}{N}{U}}
\]
\vspace{-5pt}
\[
\infer[{\uparrow}_R]
{\foct{\Psi}{\Delta}{\tupr{N}}{{\uparrow}A^+}}
{\foct{\Psi}{\Delta}{N}{\istrue{A^+}}}
\quad
\infer[{\uparrow}_L]
{\foct{\Psi}{\frameoff{\Theta}{[{\uparrow}A^+]}}{\tupl{N}}{U}}
{\foct{\Psi}{\tackon{\Theta}{A^+}}{N}{U}}
\]
\vspace{-5pt}
\[
\infer[{\ocircle}_R]
{\foct{\Psi}{\Delta}{\tlaxr{N}}{{\ocircle}A^+}}
{\foct{\Psi}{\Delta}{N}{\islax{A^+}}}
\quad
\infer[{\ocircle}_L]
{\foct{\Psi}{\frameoff{\Theta}{[{\ocircle}A^+]}}
   {\tlaxl{N}}{\restrictfrom{U}{\mlax}}}
{\foct{\Psi}{\tackon{\Theta}{A^+}}{N}{U}}
\]


\medskip
{\it Multiplicative connectives}
\[
\infer[{\one}_R]
{\foct{\Psi}{\matchunit}{\toner}{[\one]}}
{}
\quad
\infer[{\one}_L]
{\foct{\Psi}{\frameoff{\Theta}{\one}}{\tonel{N}}{U}}
{\foct{\Psi}{\tackon{\Theta}{\mkunit}}{N}{U}}
\]
\vspace{-5pt}
\[
\infer[{\fuse}_R]
{\foct{\Psi}{\matchconj{\Delta_1}{\Delta_2}}
   {\tfuser{V_1}{V_2}}{[A^+ \fuse B^+]}}
{\foct{\Psi}{\Delta_1}{V_1}{[A^+]}
 &
 \foct{\Psi}{\Delta_2}{V_2}{[B^+]}}
\quad
\infer[{\fuse}_L]
{\foct{\Psi}{\frameoff{\Theta}{A^+ \fuse B^+}}{\tfusel{N}}{U}}
{\foct{\Psi}{\tackon{\Theta}{\mkconj{A^+}{B^+}}}{N}{U}}
\]
\vspace{-5pt}
\[
\infer[{\lefti}_R]
{\foct{\Psi}{\Delta}{\tlaml{N}}{A^+ \lefti B^-}}
{\foct{\Psi}{\mkconj{A^+}{\Delta}}{N}{B^-}}
\quad
\infer[{\lefti}_L]
{\foct{\Psi}{\frameoff{\Theta}{\matchconj{\Delta_A}{[A \lefti B]}}}
   {\tappl{V}{\Sp}}{U}}
{\foct{\Psi}{\tackon{\Theta}{[B]}}{\Sp}{U}}
\]
\vspace{-5pt}
\[
\infer[{\righti}_R]
{\foct{\Psi}{\Delta}{\tlamr{N}}{A^+ \righti B^-}}
{\foct{\Psi}{\mkconj{\Delta}{A^+}}{N}{B^-}}
\quad
\infer[{\righti}_L]
{\foct{\Psi}{\frameoff{\Theta}{\matchconj{[A \righti B]}{\Delta_A}}}
   {\tappr{V}{\Sp}}{U}}
{\foct{\Psi}{\Delta_A}{V}{[A^+]}
 &
 \foct{\Psi}{\tackon{\Theta}{[B]}}{\Sp}{U}}
\]
\caption{Multiplicative, exponential fragment of focused \ollll.
(Contexts $\Psi$ suppressed.)}
\label{fig:foc-mall}
\end{figure}

\renewcommand{\foct}[4]{{#1}; {#2} \vdash {#3} : {#4}}

\renewcommand{\foct}[4]{{#2} \vdash {#3} : {#4}}

\begin{figure}
\small

\[
\infer[{\zero}_L]
{\foct{\Psi}{\frameoff{\Theta}{\zero}}{\tabort}{U}}
{}
\quad
\infer[{\oplus}_{R1}]
{\foct{\Psi}{\Delta}{\tinl{V}}{[A^+ \oplus B^+]}}
{\foct{\Psi}{\Delta}{V}{[A^+]}}
\quad
\infer[{\oplus}_{R2}]
{\foct{\Psi}{\Delta}{\tinr{V}}{[A^+ \oplus B^+]}}
{\foct{\Psi}{\Delta}{V}{[B^+]}}
\]
\vspace{-5pt}
\[
\infer[{\oplus}_L]
{\foct{\Psi}{\frameoff{\Theta}{A^+ \oplus B^+}}{\toplusl{N_1}{N_2}}{U}}
{\foct{\Psi}{\tackon{\Theta}{A^+}}{N_1}{U}
 &
 \foct{\Psi}{\tackon{\Theta}{B^+}}{N_2}{U}}
\]
\vspace{-5pt}
\[
\infer[\top_R]
{\foct{\Psi}{\Delta}{\ttopr}{\top}}
{}
\quad
\infer[{\with}_R]
{\foct{\Psi}{\Delta}{\twithr{N_1}{N_2}}{A^- \with B^-}}
{\foct{\Psi}{\Delta}{N_1}{A^-}
 &
 \foct{\Psi}{\Delta}{N_2}{B^-}}
\]
\vspace{-5pt}
\[
\infer[{\with}_{L1}]
{\foct{\Psi}{\frameoff{\Theta}{[A^- \with B^-]}}{\tpione{\Sp}}{U}}
{\foct{\Psi}{\tackon{\Theta}{[A^-]}}{\Sp}{U}}
\quad
\infer[{\with}_{L2}]
{\foct{\Psi}{\frameoff{\Theta}{[A^- \with B^-]}}{\tpitwo{\Sp}}{U}}
{\foct{\Psi}{\tackon{\Theta}{[B^-]}}{\Sp}{U}}
\]


\caption{Additive connectives of focused \ollll.
(Contexts $\Psi$ suppressed.)}
\label{fig:foc-add}
\end{figure}

\renewcommand{\foct}[4]{{#1}; {#2} \vdash {#3} : {#4}}

\begin{figure}[t]
\small

\[
\infer[{\exists}_{R}]
{\foct{\Psi}{\Delta}{\texistsr{t}{V}}{[\exists a{:}\tau.A^+]}}
{\Psi \vdash t : \tau
 &
 \foct{\Psi}{\Delta}{V}{[A^+[t/a]]}}
\quad
\infer[{\exists}_L]
{\foct{\Psi}{\frameoff{\Theta}{\exists a{:}\tau.A^+}}{\texistsl{a}{N}}{U}}
{\foct{\Psi, a{:}\tau}{\tackon{\Theta}{A^+}}{N}{U}}
\]
\vspace{-5pt}
\[
\infer[{\doteq}_R]
{\foct{\Psi}{\cdot}{\phi}{t \doteq t}}
{}
\quad
\infer[{\doteq}_L]
{\foct{\Psi}{\frameoff{\Theta}{t \doteq s}}{\phi}{U}}
{\forall(\Psi' \vdash \sigma : \Psi).
 &
 \sigma t = \sigma s
 &
 \longrightarrow
 &
 \foct{\Psi'}{\tackon{\sigma\Theta}{\cdot}}{\phi(\sigma)}{\sigma U}
 }
\]
\vspace{-5pt}
\[
\infer[\forall_R]
{\foct{\Psi}{\Delta}{\tforallr{a}{N}}{\forall a{:}\tau.A^-}}
{\foct{\Psi, a{:}\tau}{\Delta}{N}{A^-}}
\quad
\infer[{\forall}_L]
{\foct{\Psi}{\frameoff{\Theta}{[\forall a{:}\tau.A^-]}}{\tforalll{t}{\Sp}}{U}}
{\Psi \vdash t : \tau
 &
 \foct{\Psi}{\tackon{\Theta}{[A^-[t/a]]}}{\Sp}{U}}
\]


\caption{First-order connectives of focused \ollll.}
\label{fig:foc-fo}
\end{figure}


The multiplicative and exponential fragment of focused \ollll~is given
in Figure~\ref{fig:foc-mall}, the additive fragment is given in
Figure~\ref{fig:foc-add}, and the first-order connectives are treated
in Figure~\ref{fig:foc-fo}. We follow the convention of using matching
constructs in the conclusions of rules and constructions in the premises
with the exception of rules that are at the leaves, such as 
${\it id}^+$ and ${\doteq}_R$, where we write out the matching
condition as a premise. 

These rules are all written with sequents
of the form $\foct{\Psi}{\Delta}{E}{U}$, where $E$ is a {\it proof
  term} that corresponds to a derivation of that sequent. Just
as sequent forms are divided into the right-focused, inverting, and
left-focused sequents, we divide expressions into {\it values} $V$,
derivations of right-focused sequents; {\it terms} $N$,
derivations of inverting sequents; and {\it spines} $\Sp$,
derivations of left-focused sequents. The structure of
values, terms, and spines is as follows:
\begin{align*}
V & ::= z                     %% z
   \mid \tbangr{N}            %% !N
   \mid \tgnabr{N}            %% $N
   \mid \tdownr{N}            %% N
%
   \mid \toner
   \mid \tfuser{V_1}{V_2}     %% V1 * V2
%  
   \mid \tinl{V}
   \mid \tinr{V}
   \mid \texistsr{t}{V}
   \mid \tunifr
 \\
N & ::= \tfocusr{V}           %% V
   \mid \tfocusl{x}{\Sp}      %% x @ Sp
   \mid \tetap{z}{N}          %% <z>.N
   \mid \tetan{N}             %% <N>
   \mid \tdownl{x}{N}         %% x.N
   \mid \tgnabl{x}{N}         %% $x.N
   \mid \tbangl{x}{N}         %% !x.N
   \mid \tupr{N}              %% N
   \mid \tlaxr{N}             %% {N}
\\ & ~~~~ %
   \mid \tfusel{N}            %% *N
   \mid \tlaml{N}             %% <N
   \mid \tlamr{N}             %% >N
%
   \mid \tabort
   \mid \toplusl{N_1}{N_2}    %% [N1,N2]
   \mid \ttopr 
   \mid \twithr{N_1}{N_2}     %% N1 & N2
   \mid \texistsl{a}{N}
   \mid \tforallr{a}{N}
   \mid \phi
\\
\Sp & ::= \tnil               %% nil
   \mid \tupl{N}              %% N
   \mid \tlaxl{N}             %% {N} 
   \mid \tappl{V}{\Sp}        %% V < Sp
   \mid \tappr{V}{\Sp}        %% V > Sp
   \mid \tpione{\Sp}          %% fst Sp
   \mid \tpitwo{\Sp}          %% snd Sp
   \mid \tforalll{t}{\Sp}
\end{align*}


%Expressions are treated as in the structural focalization development
%\cite{simmons11structural}. 
It is possible to take a ``Curry-style''
view of expressions as {\it extrinsically} typed, which means we
consider both well-typed and ill-typed expressions; the well-typed
expressions are then those for which the sequent
$\foct{\Psi}{\Delta}{E}{U}$ is derivable. However, we will take the
``Church-style'' view that expressions are intrinsically typed
representatives of derivations: that is, $\foct{\Psi}{\Delta}{E}{U}$
expresses that $E$ is a derivation of the sequent
$\foc{\Psi}{\Delta}{U}$. To justify this close correspondence, we
require the inductive structure of expressions to be faithful to the
inductive structure of proofs; this is one reason that we
don't introduce the patterns that are common in other proof term
assignments for focused logic
\cite{watkins02concurrent,licata08focusing,krishnaswami09focusing}.
(In Section~\ref{sec:framework-patterns}, a limited syntax for patterns
is introduced as part of the logical framework \sls.)

Proof terms for the left and right identity rules include angle
brackets that reflect the notation for suspended propositions:
$\tetan{N}$ for $\eta^-$ and $\tetap{z}{N}$ for $\eta^+$. We
distinguish proof terms dealing with existential quantifiers from those
dealing with universal quantifiers in a nonstandard way by using
square brackets for the latter: $\tforalll{\lf t}{\Sp}$ and
$\tforallr{\lf a}{N}$ represent the left and right rules for universal
quantification, whereas $\texistsl{\lf a}{N}$ and $\texistsr{\lf
  t}{V}$ represent the left and right rules for existential
quantification. Other than that, the main novelty in the proof term
language and in Figures~\ref{fig:foc-mall}-\ref{fig:foc-fo} is again
the treatment of equality.  We represent the proof term corresponding
to the left rule for equality as ${\textsc{unif}}\,({\sf
  fn}\,\lf{\sigma} \Rightarrow \phi(\lf\sigma))$, where $({\sf
  fn}\,\lf{\sigma} \Rightarrow \phi(\lf\sigma))$ is intended to be a
function from unifying substitutions $\lf\sigma$ to proof terms. This
corresponds to the view of the $\doteq_L$ rule that takes the
higher-order formulation seriously as a function, and we treat any
proof term $\phi(\lf\sigma))$ where $\lf\sigma$ is a unifying
substitution as a subterm of ${\textsc{unif}}\,({\sf fn}\,\lf{\sigma}
\Rightarrow \phi(\lf\sigma))$.

There are two caveats to the idea that expressions are representatives
of derivations. One caveat is that, in order for there to be an actual
correspondence between expressions and terms, we need to annotate all
variables with the judgment they are associated with, and we need to
annotate the proof terms 
$\tinr{V}$, $\tinl{V}$, $\tpione{\Sp}$, and $\tpitwo{\Sp}$ 
with the type of the branch not taken. Pfenning writes these as
superscripts \cite{pfenning08church}, but we will follow Girard
in leaving them implicit \cite{girard89proofs}. The second caveat is
that, because we do not explicitly represent the significant
bookkeeping associated with matching constructs in proof terms, if
$\foct{\Psi}{\Delta}{E}{U}$, then $\foct{\Psi, \lf{a}{:}\tau}{\Delta,
  x{:}\ispers{A^+}}{E}{U}$ as well. Therefore, even given appropriate
type annotations, when we say that some expression $E$ is a derivation
of $\foc{\Psi}{\Delta}{U}$, it is only {\it uniquely} a derivation of
that sequent if we account for the implicit bookkeeping on contexts.
It is likely that the first caveat can be largely dismissed by
treating Figures~\ref{fig:foc-mall}-\ref{fig:foc-fo} as bidirectional
type system for proof terms. Addressing the second caveat will
require a careful analysis of when the bookkeeping on contexts
can be reconstructed, which we leave for future work. 

The proof terms presented here mirror our formulation of a
logical framework in the next chapter.  Additionally, working on the
level of proof terms allows for a greatly compressed presentation of
cut admissibility and identity expansion that emphasizes the
computational nature of these proofs: cut admissibility clearly
generalizes the {\it hereditary substitution} operation in so-called {\it
  spine form} presentations of LF \cite{cervesato02linear}, and
identity expansion is, computationally, a novel $\eta$-expansion
property on proof terms \cite{simmons11structural}.  To be fair, much
of this compression is due to neglecting the implicit bookkeeping
associated with matching constructs, bookkeeping that must be made
explicit in proofs like the cut admissibility theorem.

One theorem that takes place entirely at the level of this implicit
bookkeeping is the admissible weakening lemma: if $\Delta'$ contains
only persistent propositions and $N$ is a derivation of
$\foc{\Psi}{\Delta}{U}$, then $N$ is also a derivation of
$\foc{\Psi}{\Delta, \Delta'}{U}$. As usual, this proof can be
established by straightforward induction on the structure of $N$.

\subsection{Variable substitution}

The first-order variables introduced by universal quantifiers (on the
right) and existential quantifiers (on the left) are proper {\it
  variables} in the sense that the meaning of first-order variables is
given by substitution \cite[Chapter 1]{harper12practical}. A sequent
with free variables is thus a {\it generic} representative of all the
sequents that can be obtained by plugging terms in for those free
variables through the operation of substitution. This intuition is
formalized by the variable substitution theorem,
Theorem~\ref{thm:varsubst}.

\bigskip
\begin{theorem}[Variable substitution]
\label{thm:varsubst}
If $\Psi' \vdash \lf\sigma : \Psi$ and $\foc{\Psi}{\Delta}{U}$, then 
$\foc{\Psi'}{\lf\sigma\Delta}{\lf\sigma{U}}$.
\end{theorem}

\begin{proof}
On the level of proof terms, 
we are given $E$, a expression corresponding to a derivation of
$\foc{\Psi}{\Delta}{U}$; we are defining the operation $\lf\sigma{E}$,
an expression corresponding to a derivation of 
$\foc{\Psi'}{\lf\sigma\Delta}{\lf\sigma{U}}$.

\paragraph{\it Propositional fragment}
For the exponential, multiplicative, and additive fragments, this
operation is simple to define at the level of proof terms, and we will
omit most of the cases: $\lf\sigma(\tfuser{V_1}{V_2}) =
\tfuser{\lf\sigma{V_1}}{\lf\sigma{V_2}}$, $\lf\sigma(\tdownl{x}{N}) =
\tdownl{x}{\lf\sigma N}$, and so on. (Note that first-order variables
$\lf{a}$ do not interact with variables $x$ and $z$ in the substructural
context.) However, this compact notation does
capture a great deal of complexity. In particular, it is important to
emphasize that we need lemmas saying that variable substitution is
compatible with all the context matching operations from
Section~\ref{sec:contexts}.  In full detail, these two simple cases
would be:

\begin{itemize}

\item[--]
$\lf\sigma(\tfuser{V_1}{V_2}) = \tfuser{\lf\sigma{V_1}}{\lf\sigma{V_2}}$\smallskip\\
We are given a proof of $\foc{\Psi}{\Delta}{[A^+ \fuse B^+]}$ that
ends with the ${\fuse}_R$ rule; the subderivations are
$V_1$, a derivation of $\foc{\Psi}{\Delta_1}{[A^+]}$, and
$V_2$, a derivation of $\foc{\Psi}{\Delta_2}{[B^+]}$. Furthermore, we know that
$\Delta$ matches $\matchconj{\Delta_1}{\Delta_2}$. We need a lemma that
tells us that $\lf\sigma\Delta$ matches $\lf\sigma\Delta_1, \lf\sigma\Delta_2$;
then, by rule ${\fuse}_R$, it suffices to show that
$\foc{\Psi'}{\lf\sigma\Delta_1}{\lf\sigma{A^+}}$ (which we have by the 
induction hypothesis on $\lf\sigma$ and $V_1$) and that
$\foc{\Psi'}{\lf\sigma\Delta_2}{\lf\sigma{B^+}}$ 
(which we have by the induction hypothesis
on $\lf\sigma$ and $V_2$). \smallskip

\item[--]
$\lf\sigma(\tdownl{x}{N}) = \tdownl{x}{\lf\sigma N}$ \smallskip\\ 
We are given a proof
of $\foc{\Psi}{\Delta}{U}$ that ends with ${\downarrow}_L$; 
the subderivation is $N$, a derivation of
$\foc{\Psi}{\tackon{\Theta}{x{:}\istrue{A^-}}}{U}$. Furthermore, we know that
$\Delta$ matches $\frameoff{\Theta}{{\downarrow}A^-}$. We need a lemma
that tells us that $\lf\sigma\Delta$ matches
$\frameoff{\lf\sigma\Theta}{{\downarrow}\lf\sigma A^-}$; then, by 
rule ${\downarrow}_L$, it suffices to show 
$\foc{\Psi'}{\tackon{\lf\sigma\Theta}{x{:}\istrue{\lf\sigma{A^-}}}}{\lf\sigma{U}}$ 
(which we have by the induction hypothesis on $\lf\sigma$ and $N$).

\end{itemize}

\paragraph{\it First-order fragment} We will present variable
substitution on the first-order fragment fully. Note the $\doteq_L$
rule in particular, which does 
{\it not} require an invocation of the induction hypothesis. The cases
for the $\exists$ quantifier mimic the ones we give for the $\forall$
quantifier, and so the discussion of these cases is omitted.

\begin{itemize}

\item[--]
$\lf\sigma(\texistsr{\lf{t}}{N}) = \left(\texistsr{\lf{\sigma t}}{\lf\sigma{N}}
\right)$ 

\item[--]
$\lf\sigma(\texistsl{\lf{a}}{\Sp}) = \texistsl{\lf{a}}{\lf{(\sigma, a/a)}\Sp}$ 

\item[--]
$\lf\sigma(\tunifr) = \tunifr$

\item[--]
$\lf\sigma(\textsc{unif}\,({\sf fn}\,{\lf{\sigma''}} \Rightarrow \phi(\lf{\sigma''}))) = \textsc{unif}\,\left( \mathsf{fn}~\lf{\sigma'} \Rightarrow \phi \lf{(\sigma'
  \circ \sigma)}\right)$ \smallskip\\ 
%
  We are given a proof of $\foc{\Psi}{\Delta}{U}$ that ends with
  ${\doteq}_L$; we know that $\Delta$ matches $\frameoff{\Theta}{\lf{t}
    \doteq \lf{s}}$, and the subderivation is $\phi$, a function from
  substitutions $\Psi'' \vdash \lf{\sigma''} : \Psi$ that unify $\lf{t}$ and $\lf{s}$
  to derivations of
  $\foc{\Psi''}{\tackon{\lf{\sigma''}\Theta}{\cdot}}{\lf{\sigma''}{U}}$. We need
  a lemma that tells us that $\lf\sigma\Delta$ matches
  $\frameoff{\lf\sigma\Theta}{\lf{\sigma t} \doteq \lf{\sigma s}}$; then, by rule
  ${\doteq}_L$, it suffices to show that for all $\Psi'' \vdash
  \lf{\sigma'} : \Psi'$ that unify $\lf{\sigma t}$ and $\lf{\sigma s}$, 
  there exists
  a derivation of $\foc{\Psi''}{\tackon{\lf{\sigma'}(\lf{\sigma}
      \Theta)}{\cdot}}{\lf{\sigma'}(\lf{\sigma} U)}$, which is the same thing as
  a derivation of $\foc{\Psi''}{\tackon{\lf{(\sigma' \circ \sigma)}
      \Theta}{\cdot}} {\lf{(\sigma' \circ \sigma)} U}$. We have that
  $\Psi'' \vdash \lf{\sigma' \circ \sigma} : \Psi$, and certainly $\lf{\sigma'
  \circ \sigma}$ unifies $\lf t$ and $\lf s$, so we can conclude by passing
  $\lf{\sigma' \circ \sigma}$ to $\phi$.\smallskip

\item[--]
$\lf\sigma(\tforallr{\lf{a}}{N}) = \tforallr{\lf{a}}{\lf{(\sigma, a/a)}N}$ \smallskip\\ We are
given a proof of $\foc{\Psi}{\Delta}{\forall \lf{a}{:}\tau.A^-}$ 
that ends with $\forall_R$; the subderivation
is $N$, a derivation of $\foc{\Psi,\lf{a}{:}\tau}{\Delta}{A^-}$. Because
$\lf \sigma(\forall{\lf{a}}{:}{\tau}.A^-) 
 = \forall{\lf{a}}{:}{\lf\sigma\tau}.\lf{(\sigma, a/a)}{A^-}$,
by rule $\forall_R$ it suffices to show 
$\foc{\Psi', \lf{a}{:}\lf\sigma\tau}{\lf\sigma\Delta}{\lf{(\sigma, a/a)}A^-}$, 
which is the same thing
as $\foc{\Psi', \lf{a}{:}\lf\sigma\tau}{\lf{(\sigma, a/a)}\Delta}{\lf{(\sigma, a/a)}A^-}$.
The result
follows by the induction hypothesis on $\lf{(\sigma, a/a)}$ and $N$. \smallskip

\item[--]
$\lf\sigma(\tforalll{\lf{t}}{\Sp}) = \tforalll{\lf{\sigma t}}{\lf\sigma\Sp}$ \smallskip\\
We are given a proof of $\foc{\Psi}{\Delta}{U}$ that ends with 
$\forall_L$; the subderivation is $\Sp$, a derivation of 
$\foc{\Psi}{\tackon{\Theta}{\lf{[t/a]}\Sp}}{U}$. Furthermore,
we know that $\Delta$ matches $\frameoff{\Theta}{[\forall \lf{a}{:}\tau.A]}$.
We need a lemma that tells us that $\lf\sigma\Delta$ matches
$\frameoff{\lf\sigma\Theta}{[\forall \lf{a}{:}\tau.\lf{(\sigma, a/a)}A^-]}$; then,
by rule $\forall_L$, it suffices to show 
$\foc{\Psi'}{\tackon{\lf\sigma\Theta}{[\lf{[\sigma{t}/a](\sigma, a/a)}A^-]}}
  {\lf\sigma{U}}$, which is the same thing as
$\foc{\Psi'}{\tackon{\lf\sigma\Theta}{[\lf{\sigma}(\lf{[t/a]}A^-)]}}
  {\lf\sigma{U}}$. This follows by the induction hypothesis on $\lf\sigma$ and
$\Sp$.

\end{itemize}

Note that, in the case for $\forall_R$, the substitution $\lf\sigma$
was applied to the first-order type $\tau$ as well as to the
proposition $A^-$.  This alludes to the fact that our first-order
terms are dependently typed (Section~\ref{sec:sls-termlanguage}).
\end{proof}

Given that we write the constructive content of the variable substitution
theorem as $\lf\sigma{E}$, where $E$ is an expression,
we can also write Theorem~\ref{thm:varsubst} as an admissible
rule in one of two ways, both with and without proof terms:
\[
\infer-[{\it varsubst}]
{\foct{\Psi'}{\lf\sigma{\Delta}}{\lf\sigma{E}}{\lf\sigma{U}}}
{\Psi' \vdash \lf\sigma : \Psi
 &
 \foct{\Psi}{\Delta}{E}{U}}
\qquad
\infer-[{\it varsubst}]
{\foc{\Psi'}{\lf\sigma{\Delta}}{\lf\sigma{U}}}
{\Psi' \vdash \lf\sigma : \Psi
 &
 \foc{\Psi}{\Delta}{U}}
\]
We will tend towards the expression-annotated presentations, such as
the one on the left, in this chapter.

\subsection{Focal substitution}

Both cut admissibility and identity expansion depend on the same
focal substitution theorem that was considered for linear logic in 
Section~\ref{sec:lin-suspended}. Both of these theorems use the
compound matching construct $\frameoff{\Theta}{\restrictto{\Delta}{\mlvl}}$,
a pattern that will also be used in the proof of cut admissibility: 
$\Delta'$ matches $\frameoff{\Theta}{\restrictto{\Delta}{\mlvl}}$
if $\restrictto{\Delta}{\mlvl}$ (which, again, is a shorthand way of 
saying $\Delta$ matches $\restrictto{\Delta}{\mlvl}$) and if
$\Delta'$ matches $\frameoff{\Theta}{\Delta}$.

\bigskip
\begin{theorem}[Focal substitution]~
\begin{itemize}
\item If $\foc{\Psi}{\Delta}{[A^+]}$, ~
      $\foc{\Psi}{\tackon{\Theta}{z{:}\islvl{\susp{A^+}}}}{U}$,\\
      and $\Xi$ matches $\frameoff{\Theta}{\restrictto{\Delta}{\mlvl}}$, ~
      then $\foc{\Psi}{\Xi}{U}$
\item If $\foc{\Psi}{\Delta}{\islvl{\susp{A^-}}}$, ~
      $\foc{\Psi}{\tackon{\Theta}{[A^-]}}{U}$, \\
      $\Xi$ matches $\frameoff{\Theta}{\Delta}$, ~
      and $\restrictfrom{U}{\mlvl}$, ~
      then $\foc{\Psi}{\Xi}{U}$
\end{itemize}
\end{theorem}

\begin{proof}
The computational content of positive focal substitution is the
substitution of a value $V$ for a variable $z$ in an expression $E$, 
written $[V/z]E$. As an admissible rule, positive focal substitution is
represented as follows:
\[
\infer-[{\it subst}^+]
{\foct{\Psi}{\frameoff{\Theta}{\restrictto{\Delta}{\mlvl}}}{[V/z]E}{U}}
{\foct{\Psi}{\Delta}{V}{[A^+]}
 &
 \foct{\Psi}{\tackon{\Theta}{z{:}\islvl{\susp{A^+}}}}{E}{U}}
\]
The proof of positive focal substitution proceeds by induction over
the derivation $E$ containing the suspended proposition. In the case
where $E = z$, the derivation $z$ concludes by right focusing on the
proposition that we have a focused proof $V$ of, so the result we are
looking for is $V$.

The computational
content of negative focal substitution is the substitution of a spine $\Sp$,
which represents a continuation, out of an expression $E$ waiting on that 
continuation, written $[E]\Sp$. As an admissible rule, negative focal
substitution is represented as follows:
\[
\infer-[{\it subst}^-]
{\foct{\Psi}{\frameoff{\Theta}{\Delta}}{[ E ] \Sp}{\restrictfrom{U}{\mlvl}}}
{\foct{\Psi}{\Delta}{E}{\islvl{\langle A^- \rangle}}
 &
 \foct{\Psi}{\tackon{\Theta}{[A^-]}}{\Sp}{U}}
\]
The proof of negative focal substitution proceeds by induction over
the derivation $E$ containing the suspended proposition. In the case
where $E = \tnil$, the derivation $\tnil$ concludes by left focus on
the proposition that we have a spine $\Sp$ for, so the result we are
looking for is $\Sp$.
\end{proof}

Pay attention to the way compound matching constructs are being
used. In unfocused \ollll, using the compound notation
effectively means that a single statement:
\begin{center}
If $\Delta \altv \isconc{A}$, ~
$\tackon{\Theta}{x{:}\islvl{A}} \altv \isconc{C}$, ~
and $\Delta'$ matches $\frameoff{\Theta}{\restrictto{\Delta}{\mlvl}}$, ~
then $\Delta' \altv C$.
\end{center}
can simultaneously express 
three cut
principles: \smallskip
\begin{itemize}
\item If $\oseq{\Gamma}{\cdot}{\cdot}{\isconc A}$ 
      and $\oseq{\Gamma, A}{\Delta'}{\Omega'}{\isconc{C}}$,
      then $\oseq{\Gamma}{\Delta'}{\Omega'}{\isconc C}$.
\item If $\oseq{\Gamma}{\Delta}{\cdot}{\isconc A}$ 
      and $\oseq{\Gamma}{\Delta', A}{\Omega'}{\isconc C}$,
      then $\oseq{\Gamma}{\Delta', \Delta}{\Omega'}{\isconc C}$.
\item If $\oseq{\Gamma}{\Delta}{\Omega}{\isconc A}$ 
      and $\oseq{\Gamma}{\Delta'}{\Omega_L, A, \Omega_R}{\isconc C}$,\\
      then $\oseq{\Gamma}{\Delta',\Delta}{\Omega_L, \Omega, \Omega_R}{\isconc C}$.
\end{itemize}
\smallskip
As an admissible rule, we write these three cut principles generically
as:
\[
\infer-[{\it unfocused\mbox{-}cut}]
{\frameoff{\Theta}{\restrictto{\Delta}{\mlvl}} \altv {\isconc C}}
{\Delta \altv \isconc{A}
 &
 \tackon{\Theta}{x{:}\islvl{A}} \altv \isconc{C}}
\]

In the negative focal substitution (as the leftist substitutions of
cut admissibility), there is a corresponding use of
$\restrictfrom{U}{\mlvl}$ to capture that we can use a proof of
$\isconc{A}$ to discharge a hypothesis of $\istrue{A}$ in a proof of
$\isconc{C}$ or a proof of $\islax{C}$, but that a proof of
$\islax{A}$ can only discharge a hypothesis of $\istrue{A}$ in a proof
of $\islax{C}$. 



\section{Cut admissibility}
\label{sec:ord-cut}

It is a little wordy to say that, in a context or succedent, the only
judgments involving suspensions are $(\ispers{\susp{p^+_\mpers}})$,
$(\iseph{\susp{p^+_\meph}})$, $(\istrue{\susp{p^+}})$,
$(\isconc{\susp{p^-}})$, and $(\islax{\susp{p^-_\mlax}})$, but this is a
critical precondition of cut admissibility property for focused
\ollll. We'll call contexts and succedents with this property {\it
  suspension-normal}.

\bigskip
\begin{theorem}[Cut admissibility]\label{thm:ord-cut}
For suspension-normal $\Psi$, $A^+$, $A^-$, $\Delta$, $\Theta$, $\Xi$, and $U$,
\begin{enumerate}
\item If $\foc{\Psi}{\Delta}{[ A^+ ]}$, ~
         $\foc{\Psi}{\tackon{\Theta}{A^+}}{U}$,\\
      and $\Xi$ matches $\frameoff{\Theta}{{\Delta}}$, ~
      then $\foc{\Psi}{\Xi}{U}$.
\item If $\foc{\Psi}{\Delta}{A^-}$, ~
         $\foc{\Psi}{\tackon{\Theta}{[A^-]}}{U}$, ~
         $\Delta$ is stable, \\
      and $\Xi$ matches $\frameoff{\Theta}{\Delta}$, ~
      then $\foc{\Psi}{\Xi}{U}$.
\item If $\foc{\Psi}{\Delta}{\islvl{A^+}}$, ~
         $\foc{\Psi}{\tackon{\Theta}{A^+}}{U}$, ~
         $\Theta$ and $U$ are stable, \\ 
      $\Xi$ matches $\frameoff{\Theta}{\Delta}$, ~
      and $\restrictfrom{U}{\mlvl}$, ~
      then $\foc{\Psi}{\Xi}{U}$.
\item If $\foc{\Psi}{\Delta}{{A^-}}$, ~
         $\foc{\Psi}{\tackon{\Theta}{x{:}\islvl{A^-}}}{U}$, ~
         $\Delta$ is stable, \\ 
      and $\Xi$ matches $\frameoff{\Theta}{\restrictto{\Delta}{\mlvl}}$,
      then $\foc{\Psi}{\Xi}{U}$
\end{enumerate}
\end{theorem}
\bigskip

\noindent
The four cases of cut admissibility (and their proof below) neatly
codify an observation about the structure of cut admissibility proofs
made by Pfenning in his work on structural cut elimination
\cite{pfenning00structural}.  The first two parts of
Theorem~\ref{thm:ord-cut} are the home of the {\it principal cases}
that decompose both derivations simultaneously -- part 1 is for
positive cut formulas and part 2 is for negative cut formulas. The third
part contains all the {\it left commutative cases} that perform case
analysis and induction only on the first given derivation, and the
fourth part contains all the {\it right commutative cases} that
perform case analysis and induction only on the second given
derivation.

In Pfenning's work on structural cut elimination, this classification
of cases was informal, but the structure of our cut admissibility
proofs actually isolates the principal, left commutative, and right
commutative cases into different parts of the theorem
\cite{pfenning00structural}. This separation of cases is the reason
why cut admissibility in a focused sequent calculus can use a more
refined induction metric than cut admissibility in an unfocused
sequent calculus. As noted previously in the proof of
Theorem~\ref{thm:lincut}, the refined induction metric does away with
the precondition, essential to structural cut admissibility, that
weakening and variable substitution preserve the size of derivations.

Before discussing the proof, it is worth noting that this theorem
statement is already a sort of victory. It is an extremely simple
statement of cut admissibility for a rather complex logic.

\subsection{Optimizing the statement of cut admissibility}

We will pick the cut admissibility proof
from Chaudhuri's dissertation \cite{chaudhuri06focused} as a representative
example of existing work on cut admissibility in focused logics.  His
statement of cut admissibility for linear logic has 10 parts, 
which are sorted into
five groups. In order to extend his
proof structure to handle the extra lax and mobile connectives in
\ollll, we would need a dramatically larger number of
cases. Furthermore, at a computational level, Chaudhuri's proof
requires a lot of code duplication -- that is, the proof of two
different parts may both require a case that looks essentially the
same.

% Observing this duplication, one of our goals in
% \cite{simmons11structural} was to give a short and mechanizable proof
% of the completeness of focusing that was entirely free of code
% duplication. 
The structural focalization development in this chapter gives a
compact proof of the completeness of focusing that is entirely free of
code duplication. 
% Furthermore, based on the experience mechanizing
% structural focalization developments for intuitionstic logic, we
% believe that mechanizing this development is a reasonable future goal
% \cite{simmons11structural}.  
A great deal of simplification is due to
the use of the matching constructs
$\frameoff{\Theta}{\restrictto{\Delta}{\mlvl}}$ and
$\restrictfrom{U}{\mlvl}$. Without that notation, part 3 would split
into two parts for $\mconc$ and $\mlax$ and part 4 would split into
three parts for $\mtrue$, $\meph$, and $\mpers$. The fifth part of the
cut admissibility theorem in Section~\ref{sec:lincut}
(Theorem~\ref{thm:lincut}), which is computationally a near-duplicate
of the fourth part of the same theorem, is due to the lack of this
device.

Further simplification is due to defining right-focused, inverting,
and left-focused sequents as refinements of general sequents
$\foc{\Psi}{\Delta}{U}$. Without this approach, the
statement of part 3 must be split into two parts (for substituting into
terms and spines) and the statement of part 4 must be split in parts (for
substituting into values, terms, spines). % \cite{simmons11structural}.
Without either of the aforementioned simplifications, we would have 15
parts in the statement of Theorem~\ref{thm:ord-cut} instead of four
and twice as many cases that needed to be
written down and checked.

Picking a fixed inversion strategy prevents us from having to prove
the tedious, quadratically large confluence theorem discussed for linear
logic in Section~\ref{sec:confluent-v-fixed}. % in order to characterize
%the canonical forms that we need as the basis of the logical framework
%\sls. 
This confluence theorem is certainly true, and we might want to
prove it for any number of reasons, but it is interesting that we can
avoid it altogether in our current development.  A final improvement
in our theorem statement is very subtle: insofar as our goal is to
give a short proof of the completeness of focusing that avoids
redundancy, the {\it particular} fixed inversion strategy we choose
matters. The proof of Theorem~\ref{thm:lincut} duplicates many right
commutative cases in both part 1 and part 4 (which map directly onto
parts 1 and 4 of Theorem~\ref{thm:ord-cut} above). Our system
prioritizes the inversion of positive formulas on the left over the
inversion of negative formulas on the right. If we made the opposite
choice, as Chaudhuri's system does, then this issue would remain,
resulting in code duplication.  We get a lot of mileage out of the
fact that if $\Xi = \tackon{\Theta}{A^+}$ then $A^+$ unambiguously
refers to the left-most proposition in $\Xi$, and this invariant would
no longer be possible to maintain in the proof of cut admissibility if
we prioritized inversion of negative propositions on the right.


\subsection{Proof of cut admissibility, Theorem~\ref{thm:ord-cut}}

The proof proceeds by lexicographic induction.  In parts 1 and 2, the
type gets smaller in every call to the induction hypothesis. In part
3, the induction hypothesis is only ever invoked on the same type
$A^+$, and every invocation of the induction hypothesis is either to
part 1 (smaller part number) or to part 3 (same part number, first
derivation is smaller). Similarly, in part 4, the induction hypothesis
is only invoked at the same type $A^-$, and every invocation of the
induction hypothesis is either to part 2 (smaller part number) or to
part 4 (same part number, second derivation is smaller).

The remainder of this section will cover each of the four parts of
this proof in turn. Most of the theorem will be presented at the level
of proof terms, but for representative cases we will discuss what the
manipulation of proof terms means in terms of sequents and matching
constructs. The computational content of parts 1 and 2 is {\it
  principal substitution}, written as $(\subst{V}{N})^{A^+}$ and
$(\subst{N}{\Sp})^{A^-}$ respectively, the computational content of
part 3 is {\it leftist substitution}, written as
$\lsubsta{E}{N}{A^+}$, and the computational content of part 4 is {\it
  rightist substitution}, written as $\rsubsta{M}{x}{E}{A^-}$.

In many cases, we discuss the necessity of constructing
certain contexts or frames; in general, we will state the necessary
properties of these constructions without detailing the relatively
straightforward process of constructing them.

\subsubsection{Positive principal substitution}
Positive principal substitution encompasses half the {\it principal
  cuts} from Pfenning's structural cut admissibility proof -- the
principal cuts where the principal cut formula is positive. The
constructive content of this part is a function
$(\subst{V}{N})^{A^+}$ that normalizes a value against a
term. Induction is on the structure of the positive type. The
admissible rule associated with principal positive substitution is
${\it cut}^+$.
\[
\infer-[{\it cut}^+]
{\foct{\Psi}{\frameoff{\Theta}{\Delta}}{(\subst{V}{N})^{A^+}}{U}}
{\foct{\Psi}{\Delta}{V}{[A^+]}
 &
 \foct{\Psi}{\tackon{\Theta}{A^+}}{N}{U}}
\]
We have to be careful, especially in the positive principal substitution
associated with the type $A^+ \fuse B^+$, to maintain the
invariant that, in an unstable context, we only ever consider the {\it
  leftmost} inverting positive proposition.

In most of these cases, one of the givens is that
$\tackon{\Theta}{A^+}$ matches $\frameoff{\Theta'}{A^+}$ for some
$\Theta'$. Because this implies that $\Theta = \Theta'$, we take
the equality for granted rather than mentioning and 
reasoning explicitly about the premise every time.

\begin{itemize}
\item[--] $(\subst{z}{\tetap{z'}{N_1}})^{p^+_{\mlvl}} = [z/z']N_1$\smallskip\\
  We must show $\foc{\Psi}{\Xi}{U}$, where 
  \begin{itemize}
   \item $\Delta$ matches $z{:}\susp{p^+_\mlvl}$,
   \item 
      $N_1$ is a derivation of 
      $\foc{\Psi}{\tackon{\Theta}{z'{:}\islvl{\susp{p^+_\mlvl}}}}{U}$,
   \item and $\Xi$ matches $\frameoff{\Theta}{\Delta}$.
  \end{itemize}
  Because $\Delta$ is suspension-normal, 
  we can derive $\foc{\Psi}{\Delta}{[p^+_\mlvl]}$ by ${\it
    id}^+$, and
  $\Xi$ matches 
  $\frameoff{\Theta}{\restrictto{\Delta}{\mlvl}}$.
  Therefore, the result follows by focal substitution on $z$ and $N_1$. 

\futurework{I'm relying on a property I haven't explicitly proven:
 which feels familiar from the Agda developments:
 if $\Xi$   matches $\frameoff{\Theta}{\Delta}$
 and $\tackon{\Theta}{x:\islvl{T}}$ 
            matches $\frameoff{\Theta'}{x{:}\islvl{T}}$
 then $\Xi$ matches $\frameoff{\Theta}{\Delta}$.  }
\smallskip
 
\item[--] $(\subst{\tdownr{M}}{\tdownl{x}{N_1}})^{{\downarrow}A^-} 
           = \rsubsta{M}{x}{N_1}{A^-}$ %\\
%   We must show $\foc{\Psi}{\Xi}{U}$, where
%   \begin{itemize}
%   \item $M$ is derivation of $\foc{\Psi}{\Delta}{A^-}$, $\Delta$ is stable,
%   \item $\tackon{\Theta}{{\downarrow}A^-}$ matches 
%         $\frameoff{\Theta'}{{\downarrow}A^-}$,
%         $N$ is a derivation of 
%         $\foc{\Psi}{\tackon{\Theta'}{x{:}\istrue{A^-}}}{U}$, 
%   \item and $\Xi$ matches $\frameoff{\Theta}{\Delta}$.
%   \end{itemize}

%   $\Xi$ matches $\frameoff{\Theta'}{\Delta}$, so the result follows by
%   part 4 of cut admissibility on $N$ and $M$. 


\item[--] $(\subst{\tgnabr{M}}{\tgnabl{x}{N_1}})^{{\scriptgnab}A^-}
           = \rsubsta{M}{x}{N_1}{A^-}$

\item[--] $(\subst{\tbangr{M}}{\tbangl{x}{N_1}})^{{!}A^-}
           = \rsubsta{M}{x}{N_1}{A^-}$ \smallskip\\
  We must show $\foc{\Psi}{\Xi}{U}$, where
  \begin{itemize}
  \item $\Delta$ matches $\restrictto{\Delta}{\mpers}$,
        $M$ is derivation of $\foc{\Psi}{\Delta}{A^-}$, 
  \item $N_1$ is a derivation of 
        $\foc{\Psi}{\tackon{\Theta}{x{:}\ispers{A^-}}}{U}$,
  \item and $\Xi$ matches $\frameoff{\Theta}{\Delta}$.
  \end{itemize}

  $\Xi$ matches $\frameoff{\Theta}{\restrictto{\Delta}{\mpers}}$
  and $\Delta$ is stable (it was in a focused
  sequent $\foct{\Psi}{\Delta}{\tbangr{M}}{[{!}A^-]}$),
  so the result follows by
  part 4 of cut admissibility on $N_1$ and $M$. 

\futurework{Oh, stability has to be preserved my mapping. So many theorems!
 I'm glad I'm not proving this in Agda (this time).}
\smallskip

\item[--] $(\subst{\toner}{\tonel{N}_1})^{\one} = N_1$

\item[--] $(\subst{(\tfuser{V_1}{V_2})}{(\tfusel{N_1}}))^{A^+ \fuse B^+}
           = (\subst{V_2}{(\subst{V_1}{N_1})^{A^+}})^{B^+}$ \smallskip\\
  We must show $\foc{\Psi}{\Xi}{U}$, where
  \begin{itemize}
  \item $\Delta$ matches $\matchconj{\Delta_1}{\Delta_2}$,\\
        $V_1$ is a derivation of $\foc{\Psi}{\Delta_1}{[A^+]}$,
        $V_2$ is a derivation of $\foc{\Psi}{\Delta_2}{[B^+]}$,
  \item $N_1$ is a derivation of 
        $\foc{\Psi}{\tackon{\Theta}{A^+, B^+}}{U}$, 
  \item and $\Xi$ matches $\frameoff{\Theta}{\Delta}$.
  \end{itemize}

  We can to construct a frame $\Theta_B$ such that
  $\tackon{\Theta}{\mkconj{A^+}{B^+}} = \tackon{\Theta_B}{A^+}$;
  we're just exchanging
  the part in the frame with the part not in the frame. We can also 
  construct a second frame, $\Theta_A$, such that
  1) $\Xi$ matches $\frameoff{\Theta_A}{\Delta_2}$ and 
  2) $\tackon{\Theta_A}{B^+}$ matches $\frameoff{\Theta_B}{\Delta_1}$.

\futurework{It's probably (I might say hopefully) not clear to the reader
 what a doozy that is, but I've done it in Agda once in that proof I lost, 
 I guess.}

  Because $\tackon{\Theta_A}{B^+}$ matches $\frameoff{\Theta_B}{\Delta_1}$,
  by the induction hypothesis on $V_1$ and $N_1$ we have
  $(\subst{V_1}{N_1})^{A^+}$, a derivation of 
  $\foc{\Psi}{\tackon{\Theta_A}{B^+}}{U}$.

  Because $\Xi$ matches $\frameoff{\Theta_A}{\Delta_2}$, by the induction
  hypothesis on $V_2$ and $(\subst{V_1}{N_1})^{A^+}$, we have a derivation
  of $\foc{\Psi}{\Xi}{U}$ as required. \smallskip

\item[--] $(\subst{\tinl{V_1}}{\toplusl{N_1}{N_2}})^{A^+ \oplus B^+} 
           = (\subst{V_1}{N_1})^{A^+}$

\item[--] $(\subst{\tinr{V_2}}{\toplusl{N_1}{N_2}})^{A^+ \oplus B^+} 
           = (\subst{V_2}{N_2})^{B^+}$

\item[--] $(\subst{\texistsr{\lf{t}}{V_1}}{\texistsl{a}{N_1}})^{\exists \lf{a}{:}\tau.A^+}
           = (\subst{V_1}{\lf{[t/a]}N_1})^{\lf{[t/a]}A^+}$ \smallskip\\
  We must show $\foc{\Psi}{\Xi}{U}$, where
  \begin{itemize}
  \item $\Psi \vdash \lf{t} : \tau$, $V_1$ is a derivation of 
     $\foc{\Psi}{\Delta}{[\lf{[t/a]}A^+]}$,
  \item 
     $N_1$ is a derivation of 
     $\foc{\Psi, \lf{a}{:}\tau}{\tackon{\Theta}{A^+}}{U}$,
  \item and $\Xi$ matches $\frameoff{\Theta}{\Delta}$.
  \end{itemize}
  By variable
  substitution on $\lf{[t/a]}$ and $N_1$, we have a derivation $\lf{[t/a]}N_1$ of
  $\foc{\Psi}{\tackon{\Theta}{\lf{[t/a]}A^+}}{U}$.  We count $\lf{[t/a]}A^+$ as
  being a smaller formula than $\exists \lf{a}{:}\tau.A^+$, so by the
  induction hypothesis on $V_1$ and $\lf{[t/a]}N_1$, we get a derivation of
  $\foc{\Psi}{\Xi}{U}$ as required. \smallskip

\item[--] $(\subst{\tunifr}{\textsc{unif}\,({\sf fn}\,\lf{\sigma} \Rightarrow \phi(\lf\sigma))})^{\lf t \doteq \lf t} = \phi(\lf{\sf id})$\smallskip\\
  We must show $\foc{\Psi}{\Xi}{U}$, where
  \begin{itemize}
  \item $\Delta$ matches $\cdot$,
  \item $\phi$ is a function from substitutions $\Psi' \vdash \lf\sigma : \Psi$
     that unify $\lf t$ and $\lf t$ to derivations of 
     $\foc{\Psi}{\tackon{\Theta}{\cdot}}{U}$,
  \item and $\Xi$ matches $\frameoff{\Theta}{\Delta}$.
  \end{itemize}
  We simply apply the identity substitution to $\phi$
  to obtain a derivation of $\foc{\Psi}{\tackon{\Theta}{\cdot}}{U}$.
  Note that this is not quite the derivation of 
  $\foc{\Psi}{\Xi}{U}$ that we need; we need an exchange-like lemma that, 
  given a derivation of $\foc{\Psi}{\tackon{\Theta}{\cdot}}{U}$
  and the fact that $\Xi$ matches $\frameoff{\Theta}{\cdot}$,
  we can get a proof of $\foc{\Psi}{\Xi}{U}$ as we require.

\end{itemize}

\subsubsection{Negative principal substitution}
Negative principal substitution encompass all the {\it principal cuts}
from Pfenning's structural cut admissibility proof for which the
principal formula is negative. The constructive content of this part
is a function $(\subst{N}{\Sp})^{A^-}$ that normalizes a term against
a spine; a similar function appears as {\it hereditary reduction} 
in presentations of hereditary
substitution for LF \cite{watkins02concurrent}. Induction is on the
structure of the negative type. The admissible rule associated with
negative principal substitution is ${\it cut}^-$:
\[
\infer-[{\it cut}^-]
{\foct{\Psi}{\frameoff{\Theta}{\Delta}}{(\subst{N}{\Sp})^{A^-}}{U}}
{\foct{\Psi}{\Delta}{N}{A^-}
 &
 \foct{\Psi}{\tackon{\Theta}{[A^-]}}{\Sp}{U}
 &
 \stableL{\Delta}}
\]

\begin{itemize}
\item[--] $(\subst{\tetan{N}}{\tnil})^{p^-_\mlvl} = N$\smallskip\\
   We must show $\foc{\Psi}{\Xi}{U}$, where
   \begin{itemize}
   \item $N$ is a derivation of $\foc{\Psi}{\Delta}{\islvl{\susp{p^-_\mlvl}}}$
   \item $\tackon{\Theta}{[p^-_\mlvl]}$ matches $[p^-_\mlvl]$,
      $U = \islvl{\susp{p^-_\mlvl}}'$,
   \item and $\Xi$ matches $\frameoff{\Theta}{\Delta}$.
   \end{itemize}
   Because $U$ is suspension-normal, $\mlvl = \mlvl'$.
   A derivation of $\foc{\Psi}{\Delta}{\islvl{\susp{p^-_\mlvl}}}$ is not
   quite a proof of $\foc{\Psi}{\Xi}{U}$, so we need an exchange-like 
   lemma that we can get one from the other. \smallskip

\item[--] $(\subst{\tupr{N}}{\tupl{M}})^{{\uparrow}A^+} = \lsubsta{N}{M}{A^+}$

\item[--] $(\subst{\tlaxr{N}}{\tlaxl{M}})^{{\ocircle}A^+}
           = \lsubsta{N}{M}{A^+}$\smallskip\\
  We must show $\foc{\Psi}{\Xi}{U}$, where
  \begin{itemize}
  \item $N$ is a derivation of 
     $\foc{\Psi}{\Delta}{\islax{A^+}}$, 
  \item $\tackon{\Theta}{{\ocircle}A^+}$ matches 
     $\frameoff{\Theta'}{{\ocircle}A^+}$, 
     $\restrictfrom{U}{\mlax}$, 
     $\Theta'$ and $U$ are stable, 
     $M$ is a derivation of $\foc{\Psi}{\tackon{\Theta'}{A^+}}{U}$,
  \item and $\Xi$ matches $\frameoff{\Theta}{\Delta}$.
  \end{itemize}
  $\Xi$ matches $\frameoff{\Theta'}{\Delta}$, so the result follows
  by part 3 of cut admissibility on $N$ and $M$. \smallskip

\item[--] $(\subst{(\tlaml{N})}{(\tappl{V}{\Sp})})^{A^+ \lefti B^-}
           = (\subst{(\subst{V}{N})^{A^+}}{\Sp})^{B^-}$

\item[--] $(\subst{(\tlamr{N})}{(\tappr{V}{\Sp})})^{A^+ \righti B^-} 
           = (\subst{(\subst{V}{N})^{A^+}}{\Sp})^{B^-}$\smallskip\\
  We must show $\foc{\Psi}{\Xi}{U}$, where
  \begin{itemize}
  \item $N$ is a derivation of 
     $\foc{\Psi}{\mkconj{\Delta}{A^+}}{B^-}$, $\Delta$ is stable (by the fixed
     inversion invariant -- we only invert on the right when there is 
     no further inversion to do on the left), 
  \item $\tackon{\Theta}{[A^+ \righti B^-]}$ matches 
     $\frameoff{\Theta'}{\matchconj{[A^+ \righti B^-]}{\Delta_A}}$, 
     $V$ is a derivation of $\foc{\Psi}{\Delta_A}{[A^+]}$,
     $\Sp$ is a derivation of $\foc{\Psi}{\tackon{\Theta'}{[B^-]}}{U}$,
  \item and $\Xi$ matches $\frameoff{\Theta}{\Delta}$.
  \end{itemize}
  We can simultaneously view the construction $\Delta,A^+$ as a frame
  $\Theta_\Delta$ such that $\tackon{\Theta_\Delta}{A^+} =
  \Delta,A^+$.  Note that this is only possible to do because $\Delta$
  is stable; if there were a non-stable proposition in $\Delta$, the
  fixed inversion invariant would not permit us to frame off the
  right-most proposition $A^+$.

  We next construct a context $\Delta'_A$ that matches
  $\frameoff{\Theta_\Delta}{\Delta_A}$ (and also $\Delta, \Delta_A$ viewed
  as a matching construct), while simultaneously
  $\Xi$ matches $\frameoff{\Theta'}{\Delta'_A}$. 

  By the part 1 of cut admissibility
  on $V$ and $N$, we have $(\subst{V}{N})^{A^+}$, a derivation of 
  $\foc{\Psi}{\Delta'_A}{B^-}$, and 
  the result then follows by the induction hypothesis on 
  $(\subst{V}{N})^{A^+}$ and $\Sp$.  \smallskip

\item[--] $(\subst{(\twithr{N_1}{N_2})}{(\tpione{\Sp})})^{A^- \with B^-}
           = (\subst{N_1}{\Sp})^{A^-}$

\item[--] $(\subst{(\twithr{N_1}{N_2})}{(\tpitwo{\Sp})})^{A^- \with B^-}
           = (\subst{N_2}{\Sp})^{A^-}$

\item[--] $(\subst{(\tforallr{\lf a}{N})}{(\tforalll{\lf t}{\Sp})})^{\forall \lf{a}{:}\tau.A^-}
           = (\subst{\lf{[t/a]}N}{\Sp})^{\lf{[t/a]}A^-}$
\end{itemize}

\subsubsection{Leftist substitution}
In focal substitution, the positive case
corresponds to our usual intuitions about substitution and the
negative case is strange.  In cut admissibility, the situation is
reversed: rightist substitutions (considered in
Section~\ref{sec:rsubst} below), associated with negative principal
cut formals, look like normal substitutions, and the leftist 
substitutions, considered here, are strange, as they break
apart the expression that proves $A^+$ rather than the term
where $A^+$ appears in the context.

Leftist substitutions encompass all the {\it left commutative cuts}
from Pfenning's structural cut admissibility proof.
The constructive content of leftist substitution is a function
$\lsubst{E}{M}$; we say we are {\it substituting} $M$ {\it out of} $E$. 
Induction is on the first subterm, as we crawl 
through $E$ looking for places where focus takes place on the 
right. The admissible rule associated with leftist substitution is
${\it lcut}$:
\[
\infer-[{\it lcut}]
{\foct{\Psi}{\frameoff{\Theta}{\Delta}}{\lsubsta{E}{M}{A^+}}{\restrictfrom{U}{\mlvl}}}
{\foct{\Psi}{\Delta}{E}{\islvl{A^+}}
 &
 \foct{\Psi}{\tackon{\Theta}{A^+}}{M}{U}
 &
 \stableL{\Theta}
 &
 \stableR{U}}
\]

Except for the case where the first given derivation ends in the rule
${\it focus}_R$, every case of this theorem involves a left rule.
The general pattern for these cases is that
$\Xi$ matches $\frameoff{\Theta}{\Delta}$ and
$\Delta$ matches $\frameoff{\Theta_B}{x{:}\istrue{T}}$.
$\Theta$ and $\Theta_B$ have the same persistent variables but
distinct ephemeral and ordered variables, and we must construct
a frame ${\Theta}{\circ}{\Theta_B}$
that is effectively the composition of $\Theta$ and $\Theta_B$. 
In
cases that we discuss in detail, necessary properties of this
composition frame are stated but not proven.

\paragraph{\it Substitution out of terms}

\begin{itemize}
\item[--] $\lsubsta{\tfocusr{V}}{M}{A^+} = (\subst{V}{M})^{A^+}$\smallskip\\
  We must show $\foc{\Psi}{\Xi}{U}$, where
  \begin{itemize}
  \item $V$ is a derivation of $\foc{\Psi}{\Delta}{[A^+]}$,
  \item $M$ is a derivation of $\foc{\Psi}{\tackon{\Theta}{A^+}}{U}$,
  \item $\Xi$ matches $\frameoff{\Theta}{\Delta}$, and 
    $\restrictfrom{U}{\mlvl}$
  \end{itemize}
  The result follows from part 1 of cut admissibility
  on $V$ and $M$. \smallskip

\item[--] $\lsubsta{\tfocusl{x}{\Sp}}{M}{A^+}
           = \tfocusl{x}{(\lsubsta{\Sp}{M}{A^+})}$\smallskip\\
  We must show $\foc{\Psi}{\Xi}{U}$, where
  \begin{itemize}
  \item $\Delta$ matches $\frameoff{\Theta_B}{x{:}B^-}$, 
    $\Sp$ is a derivation of
    $\foc{\Psi}{\tackon{\Theta_B}{[B^-]}}{\islvl{A^+}}$,
  \item $M$ is a derivation of $\foc{\Psi}{\tackon{\Theta}{A^+}}{U}$,
  \item $\Xi$ matches $\frameoff{\Theta}{\Delta}$, and
    $\restrictfrom{U}{\mlvl}$.
  \end{itemize}
  $\Xi$ matches $\frameoff{(\Theta {\circ} \Theta_B)}{x{:}B^-}$
  and $\tackon{(\Theta {\circ} \Theta_B)}{[B^-]}$ matches
  $\frameoff{\Theta'}{\tackon{\Theta_B}{[B^-]}}$. By the induction
  hypothesis on $\Sp$ and $M$ we have 
  $\foc{\Psi}{\tackon{(\Theta {\circ} \Theta_B)}{[B^-]}}{U}$, and
  the required result then follows from rule ${\it focus}_L$. \smallskip
  
\item[--] $\lsubsta{\tetap{z}{N}}{M}{A^+} 
           = \tetap{z}{(\lsubsta{N}{M}{A^+})}$
\item[--] $\lsubsta{\tdownl{x}{N}}{M}{A^+} 
           = \tdownl{x}{(\lsubsta{N}{M}{A^+})}$
\item[--] $\lsubsta{\tgnabl{x}{N}}{M}{A^+} 
           = \tgnabl{x}{(\lsubsta{N}{M}{A^+})}$
\item[--] $\lsubsta{\tbangl{x}{N}}{M}{A^+} 
           = \tbangl{x}{(\lsubsta{N}{M}{A^+})}$\smallskip\\
  We must show $\foc{\Psi}{\Xi}{U}$, where
  \begin{itemize}
  \item $\Delta$ matches $\frameoff{\Theta_B}{{!}B^-}$, 
    $N$ is a derivation of
    $\foc{\Psi}{\tackon{\Theta_B}{\ispers{x{:}B^-}}}{\islvl{A^+}}$,
  \item $M$ is a derivation of $\foc{\Psi}{\tackon{\Theta}{A^+}}{U}$,
  \item $\Xi$ matches $\frameoff{\Theta}{\Delta}$, and
    $\restrictfrom{U}{\mlvl}$.
  \end{itemize}
  We can construct a $\Theta'$ such that 
  $\tackon{\Theta'}{A^+} = (\tackon{\Theta}{A^+},x{:}\ispers{B^-})$. By
  admissible weakening, $M$ is a derivation of 
  $\foc{\Psi}{\tackon{\Theta'}{A^+}}{U}$, too.

  $\Xi$ matches $\frameoff{(\Theta {\circ} \Theta_B)}{{!}B^-}$ and
  $\tackon{(\Theta {\circ} \Theta_B)}{x{:}\ispers{B^-}}$ matches
  $\frameoff{\Theta'}{\tackon{\Theta_B}{x{:}\ispers{B^-}}}$.
  By the induction hypothesis on $N$ and $M$ we have 
  $\foc{\Psi}{\tackon{(\Theta {\circ}
      \Theta_B)}{x{:}\ispers{B^-}}}{U}$, and the required
  result then follows from rule ${!}_L$. \smallskip

\item[--] $\lsubsta{\tfusel{N}}{M}{A^+} = \tfusel{(\lsubsta{N}{M}{A^+})}$
\item[--] $\lsubsta{\tabort}{M}{A^+} = \tabort$
\item[--] $\lsubsta{\toplusl{N_1}{N_2}}{M}{A^+} = \toplusl{(\lsubsta{N_1}{M}{A^+})}{(\lsubsta{N_2}{M}{A^+})}$
\item[--] $\lsubsta{\texistsl{\lf{a}}{N}}{M}{A^+} = \texistsl{\lf{a}}{(\lsubsta{N}{M}{A^+})}$\smallskip\\
  We must show $\foc{\Psi}{\Xi}{U}$, where
  \begin{itemize}
  \item $\Delta$ matches $\frameoff{\Theta_B}{\exists \lf{a}{:}\tau.B^+}$, 
     $N$ is a derivation of $\foc{\Psi,\lf{a}{:}\tau}{\tackon{\Theta_B}{B^+}}{A^+}$,
  \item $M$ is a derivation of $\foc{\Psi}{\tackon{\Theta}{A^+}}{U}$,
  \item $\Xi$ matches $\frameoff{\Theta}{\Delta}$, and
     $\restrictfrom{U}{\mlvl}$.
  \end{itemize}
  $\Xi$ matches $\frameoff{(\Theta{\circ}\Theta_B)}{\exists \lf{a}{:}\tau.B^+}$
  and $\tackon{(\Theta{\circ}\Theta_B)}{B^+}$ matches 
  $\frameoff{\Theta}{\tackon{\Theta_B}{B^+}}$. By variable weakening (a special case of variable substitution),
  $M$ is also a derivation of $\foc{\Psi, \lf{a}{:}\tau}{\tackon{\Theta}{A^+}}{U}$,
  so by the induction hypothesis on $N$ and $M$ we have
  $\foc{\Psi, \lf{a}{:}\tau}{\tackon{(\Theta{\circ}\Theta_B)}{B^+}}{U}$, 
  and the required result then follows from rule $\exists_L$. \smallskip

\item[--] $\lsubsta{\textsc{unif}\,({\sf fn}\,\lf\sigma \Rightarrow \phi(\lf\sigma))}{M}{{A^+}}
           = \textsc{unif}\,(\mathsf{fn}~\lf\sigma \Rightarrow \lsubsta{\phi(\lf\sigma)}{(\lf\sigma{M})}{\lf\sigma A^+})$\smallskip\\
  We must show $\foc{\Psi}{\Xi}{U}$, where
  \begin{itemize}
  \item $\Delta$ matches $\frameoff{\Theta_B}{\lf t \doteq \lf s}$,
    $\phi$ is a function from substitutions $\Psi' \vdash \lf \sigma : \Psi$
    that unify $\lf t$ and $\lf s$ to derivations of 
    $\foc{\Psi'}{\tackon{\lf \sigma{\Theta_B}}{\cdot}}{\lf \sigma{A^+}}$,
  \item $M$ is a derivation of $\foc{\Psi}{\tackon{\Theta}{A^+}}{U}$,
  \item $\Xi$ matches $\frameoff{\Theta}{\Delta}$, and 
     $\restrictfrom{U}{\mlvl}$.
  \end{itemize}
  $\Xi$ matches $\frameoff{(\Theta{\circ}\Theta_B)}{\lf t \doteq \lf s}$, and 
  for any substitution $\lf \sigma$, 
  $\restrictfrom{\lf \sigma{U}}{\mlvl}$ and
  $\tackon{\lf \sigma(\Theta{\circ}\Theta_B)}{\cdot}$ matches 
  $\frameoff{\lf \sigma\Theta}{\tackon{\lf\sigma\Theta_B}{\cdot}}$.
  By rule $\doteq_L$, it suffices to show that, 
  given an arbitrary substitution $\Psi' \vdash \lf\sigma : \Psi$, 
  there is a derivation of 
  $\foc{\Psi'}{\tackon{\lf \sigma(\Theta{\circ}\Theta_B)}{\cdot}}{\sigma{U}}$.
  

  By applying $\lf \sigma$ to $\phi$, we get $\phi(\lf\sigma)$, a
  derivation of
  $\foc{\Psi'}{\tackon{\lf\sigma\Theta_B}{\cdot}}{\lf\sigma{A^+}}$.
  We treat $\lf\sigma A^+$ as having the same size as $A^+$, and the
  usual interpretation of higher-order derivations is that
  $\phi(\lf\sigma)$ is a subderivation of $\phi$, so $\phi(\lf\sigma)$
  can be used to invoke the induction hypothesis.  From variable
  substitution, we get $\lf\sigma{M}$, a derivation of
  $\foc{\Psi'}{\tackon{\lf\sigma\Theta}{\lf\sigma{A^+}}}{\lf\sigma{U}}$,
  and then the result follows by the induction hypothesis on
  $\phi(\lf\sigma)$ and $\lf\sigma{M}$.

\end{itemize}

\paragraph{\it Substitution out of spines}

\begin{itemize}
\item[--] $\lsubsta{\tupl{N}}{M}{A^+} = \tupl{(\lsubsta{N}{M}{A^+})}$
\item[--] $\lsubsta{\tlaxl{N}}{M}{A^+} = \tlaxl{\lsubsta{N}{M}{A^+}}$\smallskip\\
  We must show $\foc{\Psi}{\Xi}{U}$, where
  \begin{itemize}
  \item $\Delta$ matches $\frameoff{\Theta_B}{{\ocircle}B^+}$, 
     $\restrictfrom{(\islvl{A^+})}{\mlax}$,\\
     $N$ is a derivation of $\foc{\Psi}{\tackon{\Theta_B}{A^+}}{U_A}$
  \item $M$ is a derivation of $\foc{\Psi}{\tackon{\Theta}{A^+}}{U}$,
  \item $\Xi$ matches $\frameoff{\Theta}{\Delta}$, and $U'$ matches 
     $\restrictfrom{U}{\mlvl}$.
  \end{itemize}
  Because $\restrictfrom{(\islvl{A^+})}{\mlax}$ and $\restrictfrom{U}{\mlvl}$, 
  we can conclude that 
  $\restrictfrom{U}{\mlax}$. 

  $\Xi$ matches $\frameoff{(\Theta{\circ}\Theta_B)}{{\ocircle}B^+}$
  and $\tackon{(\Theta{\circ}\Theta_B)}{B^+}$ matches 
  $\frameoff{\Theta}{\tackon{\Theta_B}{B^+}}$.
  By the induction hypothesis on $N$ and $M$ we have
  $\foc{\Psi}{\tackon{(\Theta{\circ}\Theta_B)}{B^+}}{U}$,
  and the result follows by rule ${\ocircle}_L$.  \smallskip

\item[--] $\lsubsta{\tappl{V}{\Sp}}{M}{A^+} 
           = \tappl{V}{(\lsubsta{\Sp}{M}{A^+})}$
\item[--] $\lsubsta{\tappr{V}{\Sp}}{M}{A^+} 
           = \tappr{V}{(\lsubsta{\Sp}{M}{A^+})}$\smallskip\\
  We must show $\foc{\Psi}{\Xi}{U}$, where
  \begin{itemize}
  \item $\Delta$ matches 
     $\frameoff{\Theta_B}{\matchconj{[B_1^+ \righti B_2^-]}{\Delta_B}}$,
     $V$ is a derivation of $\foc{\Psi}{\Delta_B}{[B_1^+]}$, \\
     $\Sp$ is a derivation of 
     $\foc{\Psi}{\tackon{\Theta_B}{[B_2^-]}}{\islvl{A^+}}$,
  \item $M$ is a derivation of $\foc{\Psi}{\tackon{\Theta}{A^+}}{U}$,
  \item $\Xi$ matches $\frameoff{\Theta}{\Delta}$, and 
     $\restrictfrom{U}{\mlvl}$.
  \end{itemize}
  $\Xi$ matches 
  $\frameoff{(\Theta{\circ}\Theta_B)}
    {\matchconj{[B_1^+ \righti B_2^-]}{\Delta_B}}$
  and $\tackon{(\Theta{\circ}\Theta_B)}{[B_2^-]}$ matches 
  $\frameoff{\Theta}{\tackon{\Theta_B}{[B_2^-]}}$.
  By the induction hypothesis on $\Sp$ and $M$ we have
  $\lsubsta{\Sp}{M}{A^+}$, a derivation of
  $\foc{\Psi}{\tackon{(\Theta{\circ}\Theta_B)}{[B_2^-]}}{U}$.
  The required result follows by rule ${\righti}_L$ on $V$ and 
  $\lsubsta{\Sp}{M}{A^+}$.  \smallskip

\item[--] $\lsubsta{\tpione{\Sp}}{M}{A^+} = \tpione{(\lsubsta{\Sp}{M}{A^+})}$
\item[--] $\lsubsta{\tpitwo{\Sp}}{M}{A^+} = \tpitwo{(\lsubsta{\Sp}{M}{A^+})}$
\item[--] $\lsubsta{\tforalll{\lf t}{\Sp}}{M}{A^+} 
           = \tforalll{\lf t}{(\lsubsta{\Sp}{M}{A^+})}$
\end{itemize}


\subsubsection{Rightist substitution}\label{sec:rsubst}
Rightist substitutions encompass all the {\it right commutative cuts}
from Pfenning's structural cut admissibility proof.  The constructive
content of this part is a function $\rsubsta{M}{x}{E}{A^-}$; we 
say we are {\it substituting} $M$ {\it into} $E$. Induction
is on the second subterm, as we crawl through $E$ looking for places
where $x$ is mentioned.
The admissible rule associated with rightist substitution is
${\it rcut}$:
\[
\infer-[{\it rcut}]
{\foct{\Psi}{\frameoff{\Theta}{\restrictto{\Delta}{\mlvl}}}{\rsubsta{M}{x}{E}{A^-}}{U}}
{\foct{\Psi}{\Delta}{M}{A^-}
 &
 \foct{\Psi}{\tackon{\Theta}{x{:}\islvl{A^-}}}{E}{U}
 & 
 \stableL{\Delta}}
\]

A unique aspect of the right commutative cuts is that the implicit 
bookkeeping on contexts matters to the computational behavior of 
the proof: when we deal with
multiplicative connectives like $A^+ \fuse B^+$ and $A^+ \lefti B^+$ 
under focus, 
we actually must consider that the variable $x$ that we're substituting
for can end up in only one specific branch of the proof 
(if $x$ is associated with a judgment $\istrue{A^-}$ or 
 $\iseph{A^-}$) or in both
branches of the proof (if
$x$ is associated with a judgment $x{:}\ispers{A^-}$). The computational
representation of these cases looks nondeterministic, but it is actually
determined by the annotations and bookkeeping that we don't write
down as part of the proof term. This is a point that we return to in
Section~\ref{sec:intrinsic-extrinsic}.

For cases involving left rules, the general pattern is that
$\Xi$ matches $\frameoff{\Theta}{\restrictto{\Delta}{\mlvl}}$ and
the action of the left rule, when we read it bottom-up, is observe that  
$\tackon{\Theta}{x{:}\islvl{A^-}}$ matches 
$\frameoff{\Theta'}{y{:}\istrue{T}}$ in its conclusion and constructs 
$\frameoff{\Theta'}{y{:}\islvl{T'}'}$ in its premise(s). 
Effectively, we need to abstract
a {\it two}-hole function (call it $\Gamma$) from $\Xi$. 
One hole -- the place where $x$ is --
is defined by the frame $\Theta$: morally, 
$\Theta = \lambda \Delta_B. \Gamma(x{:}\islvl{A^-})(\Delta_B)$.
The other hole -- the place where $y$ is -- 
is defined by $\Theta'$: morally,
$\Theta' = \lambda \Delta_A. \Gamma(\Delta_A)(y{:}\istrue{T})$. 
However, we cannot directly represent these functions due to
the need to operate around matching constructs. Instead, we 
construct $\Theta_\Delta$ to represent the frame that is
morally $\lambda \Delta_B. \Gamma(\Delta)(\Delta_B)$, and 
$\Theta_{T'}$ to represent the frame that is morally
$\lambda \Delta_A. \Gamma(\Delta_A)(y{:}\islvl{T'}')$. As before, in
cases that we discuss in detail, necessary properties of these
two frames are stated but not proven.


\paragraph{\it Substitution into values}

\begin{itemize}
\item[--] $\rsubsta{M}{x}{z}{A^-} = z$
\item[--] $\rsubsta{M}{x}{(\tdownr{N})}{A^-}
           = \tdownr{(\rsubsta{M}{x}{N}{A^-})}$
\item[--] $\rsubsta{M}{x}{(\tgnabr{N})}{A^-}
           = \tgnabr{(\rsubsta{M}{x}{N}{A^-})}$
\item[--] $\rsubsta{M}{x}{(\tbangr{N})}{A^-}
           = \tbangr{(\rsubsta{M}{x}{N}{A^-})}$\smallskip\\
  We must show $\foc{\Psi}{\Xi}{[{!}B^-]}$, where 
  \begin{itemize}
  \item $M$ is a derivation of $\foc{\Psi}{\Delta}{A^-}$, 
  \item $\tackon{\Theta}{x{:}\islvl{A^-}}$ matches 
     $\restrictto{\Delta'}{\mpers}$,
     $N$ is a derivation of $\foc{\Psi}{\Delta'}{B^-}$, 
  \item and $\Xi$ matches $\frameoff{\Theta}{\restrictto{\Delta}{\mlvl}}$.
  \end{itemize}
  Because $\tackon{\Theta}{x{:}\islvl{A^-}}$ matches
  $\restrictto{\Delta'}{\mpers}$ and $\Xi$ matches 
  $\frameoff{\Theta}{\restrictto{\Delta}{\mlvl}}$, we can conclude that
  there exists a $\Theta'$ such that
  $\Delta' = \tackon{\Theta'}{x{:}\islvl{A^-}}$ and also that
  $\Xi$ matches $\restrictto{\Xi}{\mpers}$.

  By the induction hypothesis on $M$ and $N$, 
  we have a derivation of $\foc{\Psi}{\Xi}{B^-}$, 
  and the result follows by rule ${!}_R$. 

\smallskip

\item[--] $\rsubsta{M}{x}{\toner}{A^-} = \toner$ \smallskip\\
  We must show $\foc{\Psi}{\Xi}{[\one]}$, where
  \begin{itemize}
  \item $M$ is a derivation of $\foc{\Psi}{\Delta}{A^-}$, 
  \item $\tackon{\Theta}{x{:}\islvl{A^-}}$ matches $\cdot$,
  \item and $\Xi$ matches $\frameoff{\Theta}{\restrictto{\Delta}{\mlvl}}$.
  \end{itemize}
  Because $\tackon{\Theta}{x{:}\islvl{A^-}}$ matches $\cdot$, it must
  be the case that $\mlvl = \mpers$, and so 
  $\Xi$ matches $\cdot$ as well. The result follows by rule ${\one}_R$. 

\smallskip

\item[--] $\rsubsta{M}{x}{(\tfuser{V_1}{V_2})}{A^-} = $ \smallskip\\
    $\begin{array}{ll}
    \qquad \tfuser{(\rsubsta{M}{x}{V_1}{A^-})}{V_2}
     & \mbox{\it (if $x$ is in $V_1$'s context but not $V_2$'s)}\\
    \qquad \tfuser{V_1}{(\rsubsta{M}{x}{V_2}{A^-})}
     & \mbox{\it (if $x$ is in $V_2$'s context but not $V_1$'s)}\\
    \qquad \tfuser{(\rsubsta{M}{x}{V_1}{A^-})}{(\rsubsta{M}{x}{V_2}{A^-})}
     \qquad & \mbox{\it (if $x$ is in both $V_1$ and $V_2$'s contexts)}
    \end{array}$\smallskip\\
  We must show $\foc{\Psi}{\Xi}{[B_1^+ \fuse B_2^+]}$, where 
  \begin{itemize}
  \item $M$ is a derivation of $\foc{\Psi}{\Delta}{A^-}$, 
  \item $\tackon{\Theta}{x{:}\islvl{A^-}}$ matches $\Delta_1, \Delta_2$,
     $V_1$ is a derivation of $\foc{\Psi}{\Delta_1}{B_1^+}$, \\
     $V_2$ is a derivation of $\foc{\Psi}{\Delta_2}{B_2^+}$, 
  \item and $\Xi$ matches $\frameoff{\Theta}{\restrictto{\Delta}{\mlvl}}$.
  \end{itemize}
  There are three possibilities: either $x$ is a variable declaration
  in $\Delta_1$ or $\Delta_2$ but not both (if $\mlvl$ is $\meph$ or
  $\mtrue$) or $x$ is a variable declaration in both $\Delta_1$ and
  $\Delta_2$ (if $\mlvl$ is $\mpers$).

  The first two cases are symmetric; 
  assume without loss of generality that $x$ is a variable declaration
  in $\Delta_1$ but not $\Delta_2$; we can construct a 
  $\Theta_1$ and $\Delta_1'$ 
  such that $\tackon{\Theta_1}{x{:}\islvl{A^-}} = \Delta_1$,
  $\Delta_1'$ matches $\frameoff{\Theta_1}{\restrictto{\Delta_1}{\mlvl}}$,
  and $\Xi$ matches $\matchconj{\Delta_1'}{\Delta_2}$
  By the induction hypothesis on $M$ and $V_1$, we have 
  $\rsubsta{M}{x}{V_1}{A^-}$, a derivation of 
  $\foc{\Psi}{\Delta_1'}{[B_1^+]}$, and the result follows by
  rule ${\fuse}_R$ on $\rsubsta{M}{x}{V_1}{A^-}$ and $V_2$.

  The third case is similar; we construct a 
  $\Theta_1$, $\Delta_1'$, $\Theta_2$, and $\Delta_2'$ such that 
  $\tackon{\Theta_1}{x{:}\islvl{A^-}} = \Delta_1$,
  $\tackon{\Theta_2}{x{:}\islvl{A^-}} = \Delta_2$,
  $\Delta_1'$ matches $\frameoff{\Theta_1}{\restrictto{\Delta_1}{\mlvl}}$,
  $\Delta_2'$ matches $\frameoff{\Theta_1}{\restrictto{\Delta_2}{\mlvl}}$,
  and $\Xi$ matches $\matchconj{\Delta_1'}{\Delta_2'}$, which is only 
  possible because $\mlvl = \mpers$; we then invoke the induction 
  hypothesis twice. 

\smallskip

\item[--] $\rsubsta{M}{x}{(\tinl{V})}{A^-} 
           = \tinl{\rsubsta{M}{x}{V}{A^-}}$
\item[--] $\rsubsta{M}{x}{(\tinr{V})}{A^-} 
           = \tinr{\rsubsta{M}{x}{V}{A^-}}$
\item[--] $\rsubsta{M}{x}{(\texistsr{\lf t}{V})}{A^-} 
           = \texistsr{\lf t}{(\rsubsta{M}{x}{V}{A^-})}$
\item[--] $\rsubsta{M}{x}{\tunifr}{A^-} = \tunifr$
\end{itemize}

\paragraph{\it Substitution into terms}

\begin{itemize}
\item[--] $\rsubsta{M}{x}{\tfocusr{V}}{A^-} 
           = \tfocusr{\rsubsta{M}{x}{V}{A^-}}$
\item[--] $\rsubsta{M}{x}{(\tfocusl{y}{\Sp})}{A^-} 
           = \tfocusl{y}{(\rsubsta{M}{x}{\Sp}{A^-})} \qquad$ {\it ($x \# y$)}
\item[--] $\rsubsta{M}{x}{(\tfocusl{x}{\Sp})}{A^-} =$\\
    $\begin{array}{ll}
    \qquad (\subst{M}{\Sp})^{A^-}
     & \mbox{\it (if $x$ is not in $\Sp$'s context)}\\
    \qquad (\subst{M}{(\rsubsta{M}{x}{\Sp}{A^-})})^{A^-}
     & \mbox{\it (if $x$ is in $\Sp$'s context)}
    \end{array}$\smallskip\\
  We must show $\foc{\Psi}{\Xi}{U}$, where
  \begin{itemize}
  \item $M$ is a derivation of $\foc{\Psi}{\Delta}{A^-}$, 
  \item $\tackon{\Theta}{x{:}\islvl{A^-}}$ matches
     $\frameoff{\Theta'}{x{:}A^-}$, 
     $\Sp$ is a derivation of 
     $\foc{\Psi}{\tackon{\Theta'}{[A^-]}}{U}$,
  \item and $\Xi$ matches $\frameoff{\Theta}{\restrictto{\Delta}{\mlvl}}$.
  \end{itemize}
  If $\mlvl$ is $\meph$ or $\mtrue$, then 
  $\Xi$ matches $\frameoff{\Theta'}{\Delta}$, and the result follows by
  part 1 of cut admissibility on $M$ and $\Sp$. 

  If $\mlvl$ is $\mpers$, $\Xi$ doesn't match $\frameoff{\Theta'}{\Delta}$,
  since $\Theta'$ has an extra variable declaration $x{:}\ispers{A^-}$. Instead,
  we have that 
    $\tackon{\Theta}{[A^-]}$ matches 
   $\frameoff{\Theta_{[A^-]}}{\restrictto{\Delta}{\mpers}}$
  and $\tackon{\Theta_{[A^-]}}{x{:}\ispers{A^-}} = \tackon{\Theta'}{[A^-]}$,
  so $\Sp$ is also a derivation of 
  $\foc{\Psi}{\tackon{\Theta_{[A^-]}}{x{:}\ispers{A^-}}}{U}$. 
  By the induction hypothesis on $M$ and $\Sp$, we have 
  $\rsubsta{M}{x}{\Sp}{A^-}$, a derivation of 
  $\foc{\Psi}{\tackon{\Theta}{[A^-]}}{U}$. Then, because
  $\Xi$ matches $\frameoff{\Theta}{\Delta}$, the result follows
  from part 1 of cut admissibility on $M$ and $\rsubsta{M}{x}{\Sp}{A^-}$.
  
 
\futurework{For affine logic: extra step of weakening to get from 
 $M$ to $M$ weakened with more affine stuff.}

\smallskip

\item[--] $\rsubsta{M}{x}{(\tetap{z}{N})}{A^-} 
           = \tetap{z}{(\rsubsta{M}{x}{N}{A^-})}$
\item[--] $\rsubsta{M}{x}{\tetan{N}}{A^-} 
           = \tetan{\rsubsta{M}{x}{N}{A^-}}$

\item[--] $\rsubsta{M}{x}{(\tdownl{y}{N})}{A^-} 
           = \tdownl{y}{(\rsubsta{M}{x}{N}{A^-})}$

\item[--] $\rsubsta{M}{x}{(\tgnabl{y}{N})}{A^-} 
           = \tgnabl{y}{(\rsubsta{M}{x}{N}{A^-})}$ %\smallskip\\
%   We must show $\foc{\Psi}{\Xi}{U}$, where
%   \begin{itemize}
%   \item $M$ is a derivation of $\foc{\Psi}{\Delta}{A^-}$, 
%   \item $\tackon{\Theta}{x{:}\islvl{A^-}}$ matches 
%      $\frameoff{\Theta'}{{\gnab}B^-}$, $N$ is a derivation of 
%      $\foc{\Psi}{\tackon{\Theta'}{y{:}\iseph{B^-}}}{U}$,
%   \item and $\Xi$ matches $\frameoff{\Theta}{\restrictto{\Delta}{\mlvl}}$.
%   \end{itemize}
%   $\Xi$ matches $\frameoff{\Theta_\Delta}{{!}B^-}$, 
%   $\tackon{\Theta_\Delta}{y{:}\iseph{B^-}}$ matches
%   $\frameoff{\Theta}{\restrictto{\Delta}{\mlvl}}$, and
%   $\tackon{\Theta_B}{x{:}\islvl{A}} = \tackon{\Theta'}{y{:}\iseph{B^-}}$.
%   By the induction hypothesis on $M$ and $N$ we have
%   $\foc{\Psi}{\tackon{\Theta_\Delta}{y{:}\iseph{B^-}}}{U}$, and 
%   the result follows by rule ${\gnab}_L$. 

\item[--] $\rsubsta{M}{x}{(\tbangl{y}{N})}{A^-} 
           = \tbangl{y}{(\rsubsta{M}{x}{N}{A^-})}$\smallskip\\
  We must show $\foc{\Psi}{\Xi}{U}$, where
  \begin{itemize}
  \item $M$ is a derivation of $\foc{\Psi}{\Delta}{A^-}$, 
  \item $\tackon{\Theta}{x{:}\islvl{A^-}}$ matches 
     $\frameoff{\Theta'}{{!}B^-}$, $N$ is a derivation of 
     $\foc{\Psi}{\tackon{\Theta'}{y{:}\ispers{B^-}}}{U}$,
  \item and $\Xi$ matches $\frameoff{\Theta}{\restrictto{\Delta}{\mlvl}}$.
  \end{itemize}
 
  Let $\Delta' = \mkconj{\Delta}{y{:}\ispers{B^-}}$.
  By admissible weakening, $M$ is derivation of $\foc{\Psi}{\Delta'}{A^-}$ too.

  $\Xi$ matches $\frameoff{\Theta_{\Delta}}{{!}B^-}$,
  $\tackon{\Theta_{\Delta}}{y{:}\ispers{B^-}}$ matches 
  $\frameoff{\Theta_{B^-}}{\restrictto{\Delta'}{\mlvl}}$,
  and 
  $\tackon{\Theta_{B^-}}{x{:}\islvl{A^-}} = \tackon{\Theta'}{y{:}\ispers{B^-}}$.
  By the induction hypothesis on $M$ and $N$ we have
  $\foc{\Psi}{\tackon{\Theta_{\Delta}}{y{:}\ispers{B^-}}}{U}$, and the result
  follows by rule ${!}_L$.

  \smallskip

\item[--] $\rsubsta{M}{x}{(\tupr{N})}{A^-} 
           = \tupr{(\rsubsta{M}{x}{N}{A^-})}$
\item[--] $\rsubsta{M}{x}{\tlaxr{N}}{A^-} 
           = \tlaxr{\rsubsta{M}{x}{N}{A^-}}$

\item[--] $\rsubsta{M}{x}{\tfusel{N}}{A^-} 
           = \tfusel{(\rsubsta{M}{x}{N}{A^-})}$
\item[--] $\rsubsta{M}{x}{\tlaml{N}}{A^-} 
           = \tlaml{(\rsubsta{M}{x}{N}{A^-})}$
\item[--] $\rsubsta{M}{x}{\tlamr{N}}{A^-} 
           = \tlamr{(\rsubsta{M}{x}{N}{A^-})}$
\item[--] $\rsubsta{M}{x}{\tabort}{A^-} 
           = \tabort$
\item[--] $\rsubsta{M}{x}{\toplusl{N_1}{N_2}}{A^-} 
           = \toplusl{\rsubsta{M}{x}{N_1}{A^-}}{\rsubsta{M}{x}{N_2}{A^-}}$
  \smallskip\\
  We must show $\foc{\Psi}{\Xi}{U}$, where 
  \begin{itemize}
  \item $M$ is a derivation of $\foc{\Psi}{\Delta}{A^-}$, 
  \item $\tackon{\Theta}{x{:}\islvl{A^-}}$ matches 
     $\frameoff{\Theta'}{B_1^+ \oplus B_2^+}$, $N_1$ is a derivation of 
     $\foc{\Psi}{\tackon{\Theta'}{B_1^+}}{U}$,\\
     $N_2$ is a derivation of $\foc{\Psi}{\tackon{\Theta'}{B_2^+}}{U}$,
  \item and $\Xi$ matches $\frameoff{\Theta}{\restrictto{\Delta}{\mlvl}}$.
  \end{itemize}

  $\Xi$ matches $\frameoff{\Theta_{\Delta}}{B_1^+ \oplus B_2^+}$,
  and for $i \in \{1,2\}$,
  $\tackon{\Theta_{\Delta}}{B_i^+}$ matches
  $\frameoff{\Theta_{B_i^+}}{\restrictto{\Delta}{\mpers}}$ 
  and
  $\tackon{\Theta_{B_i^+}}{x{:}\islvl{A^-}} =
   \tackon{\Theta'}{B_i^+}$.

  By the induction hypothesis on $M$ and $N_1$, we have
  $\foc{\Psi}{\tackon{\Theta_{\Delta}}{B_1^+}}{U}$, by the induction 
  hypothesis on $M$ and $N_2$, we have 
  $\foc{\Psi}{\tackon{\Theta_{\Delta}}{B_2^+}}{U}$, and the
  result follows by rule $\oplus_L$. 


\smallskip

\item[--] $\rsubsta{M}{x}{\ttopr}{A^-} 
           = \ttopr$
\item[--] $\rsubsta{M}{x}{(\twithr{N_1}{N_2})}{A^-} 
           = \twithr{(\rsubsta{M}{x}{N_1}{A^-})}{(\rsubsta{M}{x}{N_2}{A^-})}$

\item[--] $\rsubsta{M}{x}{\texistsl{\lf a}{N}}{A^-} 
           = \texistsl{\lf a}{(\rsubsta{M}{x}{N}{A^-})}$
\item[--] $\rsubsta{M}{x}{\tforallr{\lf a}{N}}{A^-} 
           = \tforallr{\lf a}{(\rsubsta{M}{x}{N}{A^-})}$
\item[--] $\rsubsta{M}{x}{\textsc{unif}\,({\sf fn}\,\lf\sigma \Rightarrow \phi(\lf\sigma))}{A^-} 
           = \textsc{unif}\,({\sf fn}~\lf\sigma \Rightarrow 
              \rsubsta{\lf\sigma{M}}{x}{\phi(\lf\sigma)}{A^-})$\smallskip\\
  We must show $\foc{\Psi}{\Xi}{U}$, where
  \begin{itemize}
  \item $M$ is a derivation of $\foc{\Psi}{\Delta}{A^-}$, 
  \item $\tackon{\Theta}{x{:}\islvl{A^-}}$ matches 
     $\frameoff{\Theta'}{\lf t \doteq \lf s}$, $\phi$ 
     is a function from substitutions $\Psi' \vdash \lf \sigma : \Psi$
     that unify $\lf t$ and $\lf s$ to derivations of 
     $\foc{\Psi'}{\tackon{\lf \sigma\Theta'}{\cdot}}{\lf \sigma{U}}$.
  \item and $\Xi$ matches $\frameoff{\Theta}{\restrictto{\Delta}{\mlvl}}$.
  \end{itemize}

  $\Xi$ matches $\frameoff{\Theta_\Delta}{\lf t \doteq \lf s}$, and 
  for any substitution $\lf \sigma$, $\tackon{\lf \sigma\Theta_\Delta}{\cdot}$
  matches $\frameoff{\lf\sigma\Theta}{\restrictto{\Delta}{\mlvl}}$.
  By rule $\doteq_L$, it suffices to show that, given an arbitrary
  substitution $\Psi' \vdash \lf \sigma : \Psi$, there
  is a derivation of 
  $\foc{\Psi'}{\tackon{\lf\sigma\Theta_\Delta}{\cdot}}{\lf\sigma{U}}$.

  By applying $\lf\sigma$ to $\phi$, we get $\phi(\lf\sigma)$, a derivation 
  of $\foc{\Psi'}{\tackon{\lf\sigma\Theta_B}{\cdot}}{\lf\sigma{A^+}}$;
  the usual interpretation of higher-order derivations is that 
  $\phi(\lf\sigma)$ is a subderivation of $\phi$, so $\phi(\lf\sigma)$ can be
  used to invoke the induction hypothesis.
  From variable substitution, we get $\lf\sigma{M}$, a derivation
  of 
  $\foc{\Psi'}{\lf \sigma\Delta}{\islvl{\lf \sigma{A^-}}}$,
  and the result follows
  by the induction hypothesis on $\lf \sigma{M}$ and 
  $\phi(\lf \sigma)$.

\end{itemize}

\paragraph{\it Substitution into spines}

\begin{itemize}
\item[--] $\rsubsta{M}{x}{\tnil}{A^-} 
           = \tnil$
\item[--] $\rsubsta{M}{x}{(\tupl{N})}{A^-} 
           = \tupl{(\rsubsta{M}{x}{N}{A^-})}$
\item[--] $\rsubsta{M}{x}{\tlaxl{N}}{A^-} 
           = \tlaxl{\rsubsta{M}{x}{N}{A^-}}$
\item[--] $\rsubsta{M}{x}{\tappl{V}{\Sp}}{A^-} =$\\
    $\begin{array}{ll}
    \qquad \tappl{(\rsubsta{M}{x}{V}{A^-})}{\Sp}
     & \mbox{\it (if $x$ is in $V$'s context but not $\Sp$'s)}\\
    \qquad \tappl{V}{(\rsubsta{M}{x}{\Sp}{A^-})}
     & \mbox{\it (if $x$ is in $\Sp$'s context but not $V$'s)}\\
    \qquad \tappl{(\rsubsta{M}{x}{V}{A^-})}{(\rsubsta{M}{x}{\Sp}{A^-})}
     \qquad & \mbox{\it (if $x$ is in both $V$ and $\Sp$'s contexts)}
    \end{array}$
\item[--] $\rsubsta{M}{x}{\tappr{V}{\Sp}}{A^-} =$\\
    $\begin{array}{ll}
    \qquad \tappr{(\rsubsta{M}{x}{V}{A^-})}{\Sp}
     & \mbox{\it (if $x$ is in $V$'s context but not $\Sp$'s)}\\
    \qquad \tappr{V}{(\rsubsta{M}{x}{\Sp}{A^-})}
     & \mbox{\it (if $x$ is in $\Sp$'s context but not $V$'s)}\\
    \qquad \tappr{(\rsubsta{M}{x}{V}{A^-})}{(\rsubsta{M}{x}{\Sp}{A^-})}
     \qquad & \mbox{\it (if $x$ is in both $V$ and $\Sp$'s contexts)}
    \end{array}$ \smallskip\\
  We must show $\foc{\Psi}{\Xi}{\forall x{:}\tau. B^-}$, where
  \begin{itemize}
  \item $M$ is a derivation of $\foc{\Psi}{\Delta}{A^-}$, 
  \item $\tackon{\Theta}{x{:}\islvl{A^-}}$ matches 
     $\frameoff{\Theta'}{[B_1^+ \righti B_2^-], \Delta_A}$, 
     $V$ is a derivation of 
     $\foc{\Psi}{\Delta_A}{[B_1^+]}$,\\
     $\Sp$ is a derivation of 
     $\foc{\Psi}{\tackon{\Theta'}{[B_2^-]}}{U}$,
  \item and $\Xi$ matches $\frameoff{\Theta}{\restrictto{\Delta}{\mlvl}}$.
  \end{itemize}
  
  There are three possibilities: either $x$ is a variable declaration in 
  $\Theta'$ or $\Delta_A$ but not both (if $\mlvl$ is $\meph$
  or $\mtrue$) or $x$ is a variable declaration in both $\Theta'$ and $\Delta_A$
  (if $\mlvl$ is $\mpers$). 

  In the first case ($x$ is a variable declaration in $\Delta_A$ only), 
  $\Xi$ matches 
  $\frameoff{\Theta'}{\matchconj{[B_1^+ \righti B_2^-]}{\Delta_A'}}$,
  $\Delta_A'$ matches $\frameoff{\Theta_A}{\restrictto{\Delta}{\mlvl}}$, and 
  $\Delta_A = \tackon{\Theta_A}{x{:}\islvl{A^-}}$. 
  By the induction hypothesis on $M$ and $V$ we have
  $\rsubsta{M}{x}{V}{A^-}$, a derivation of $\foc{\Psi}{\Delta'_A}{[B_1^+]}$,
  and the result follows by rule ${\righti}_L$ on 
  $\rsubsta{M}{x}{V}{A^-}$ and $\Sp$.

  In the second case ($x$ is a variable declaration in $\Theta'$ only),
  $\Xi$ matches 
  $\frameoff{\Theta_\Delta}{\matchconj{[B_1^+ \righti B_2^-]}{\Delta_A}}$,
  $\tackon{\Theta_\Delta}{[B_2^-]}$ matches
  $\frameoff{\Theta_{[B_2^-]}}{\restrictto{\Delta}{\mlvl}}$,
  and  
  $\tackon{\Theta_{[B_2^-]}}{x{:}\mlvl} = \tackon{\Theta'}{[B_2^-]}$.
  By the induction hypothesis on $M$ and $\Sp$, we have 
  $\rsubsta{M}{x}{\Sp}{A^-}$, a derivation of 
  $\foc{\Psi}{\tackon{\Theta_\Delta}{[B_2^-]}}{U}$, and the result follows
  by rule ${\righti}_L$ on $V$ and $\rsubsta{M}{x}{\Sp}{A^-}$.

  In the third case ($x$ is a variable declaration in $\Theta'$ and $\Delta_A$), 
  $\Xi$ matches 
  $\frameoff{\Theta_\Delta}{\matchconj{[B_1^+ \righti B_2^-]}{\Delta_A'}}$,
  where $\Theta_\Delta$ and $\Delta_A'$ have the same properties as before,
  and we proceed invoking the induction hypothesis twice.

\smallskip
  
\item[--] $\rsubsta{M}{x}{\tpione{\Sp}}{A^-} 
           = \tpione{(\rsubsta{M}{x}{\Sp}{A^-})}$
\item[--] $\rsubsta{M}{x}{\tpitwo{\Sp}}{A^-} 
           = \tpitwo{(\rsubsta{M}{x}{\Sp}{A^-})}$
\item[--] $\rsubsta{M}{x}{\tforalll{\lf t}{\Sp}}{A^-} 
           = \tforalll{\lf t}{(\rsubsta{M}{x}{\Sp}{A^-})}$
\end{itemize}

\section{Identity expansion}
\label{sec:ord-identity}

The form of the identity expansion theorems is already available to
us: the admissible rules $\eta_{A^+_\mlvl}$ and $\eta_{A^-_\mlvl}$ are
straightforward generalizations of the explicit rules $\eta^+$ and
$\eta^-$ in Figure~\ref{fig:foc-mall} from ordered atomic propositions
$p^+$ and $p^-$ to arbitrary propositions and from permeable atomic
propositions $p^+_\meph$, $p^+_\mpers$, and $p^-_\mlax$ to arbitrary
permeable propositions $A^+_\meph$, $A^+_\mpers$ and $A^-_\mlax$. The
content of Theorem~\ref{thm:ordidexpand} below is captured by the two
admissible rules $\eta_{A^+_\mlvl}$ and $\eta_{A^-_\mlvl}$ and also by
the two functions and $\etapa{z}{N}{A^+_\mlvl}$ and
$\etana{N}{A^-_\mlvl}$ that operate on proof terms.
\[
\infer-[\eta_{A^+_\mlvl}]
{\foct{\Psi}{\frameoff{\Theta}{A^+_\mlvl}}{\etapa{z}{N}{A^+_\mlvl}}{U}}
{\foct{\Psi}{\tackon{\Theta}{z{:}\islvl{\langle A^+_\mlvl \rangle}}}{N}{U}}
\quad
\infer-[\eta_{A^-_\mlvl}]
{\foct{\Psi}{\Delta}{\etana{N}{A^-_\mlvl}}{A^-_\mlvl}}
{\foct{\Psi}{\Delta}{N}{\islvl{\langle A^-_\mlvl \rangle}}
 &
 \stableL{\Delta}}
\]

Identity expansion is not the
perhaps not the best name for the property; the name comes from the
fact that the usual identity properties are a corollary of identity
expansion. Specifically, $\etapa{z}{\tfocusr{z}}{A^+}$ is a derivation
of $\foc{\Psi}{A^+}{\isconc{A^+}}$
and $\etana{\tfocusl{x}{\tnil}}{A^-}$ is a derivation of
$\foc{\Psi}{x{:}\istrue{A^-}}{A^-}$.

In the proof of identity expansion, we do pay some price in return for
including permeable propositions, as we perform slightly different
bookkeeping depending on whether or not it is necessary to apply
admissible weakening to the subderivation $N$. However, this cost is
mostly borne by the part of the context we leave implicit.

\bigskip
\begin{theorem}[Identity expansion]~\label{thm:ordidexpand}
\begin{itemize}
\item If 
  $\foc{\Psi}{\tackon{\Theta}{z{:}\islvl{\langle A^+_\mlvl \rangle}}}{U}$
  and $\Delta$ matches $\frameoff{\Theta}{A^+_\mlvl}$, 
  then $\foc{\Psi}{\Delta}{U}$.
\item If
  $\foc{\Psi}{\Delta}{\islvl{\langle A^-_\mlvl \rangle}}$
  and $\Delta$ is stable,
  then $\foc{\Psi}{\Delta}{A^-_\mlvl}$.
\end{itemize}
\end{theorem}

\begin{proof} By mutual induction over the structure of types. We
  provide the full definition at the level of proof terms and include
  an extra explanatory derivation for a few of the positive cases.

\subsubsection{Positive cases}

\begin{itemize}
\item[--] $\etapa{z}{N}{p^+_\mlvl} = \tetap{z}{N}$

  \smallskip
  $N$ is a derivation of 
  $\foc{\Psi}{\tackon{\Theta}{z{:}\islvl{p^+_\mlvl}}}{U}$; the 
  result follows immediately by the rule $\eta^+$: 
  \[\footnotesize
  \infer[\eta^+]
   {\foct{\Psi}{\frameoff{\Theta}{p^+_\mlvl}}{\tetap{z}{N}}{U}}
   {\foct{\Psi}{\tackon{\Theta}{z{:}\islvl{p^+_\mlvl}}}{N}{U}}
  \]
 
\medskip

\item[--] $\etapa{z}{N}{{\downarrow}A^-} 
           = \tdownl{x}{([(\tdownr{(\etana{\tfocusl{x}{\tnil}}{A^-})})/z]N)}$

  \smallskip
  $N$ is a derivation of 
  $\foc{\Psi}{\tackon{\Theta}{z{:}\istrue{\susp{{\downarrow}A^-}}}}{U}$. 
  We construct a context $\Xi$ that contains only the persistent
  propositions from $\Delta$. This means that 
  $\frameoff{\Theta}{\Xi, x{:}\istrue{A^-}}$ matches 
  $\tackon{\Theta}{x{:}\istrue{A^-}}$. We can then derive:
  \[\footnotesize \infer[{\downarrow}_L]
  {\foct{\Psi}{\frameoff{\Theta}{{\downarrow}A^-}}
     {\tdownl{x}{([(\tdownr{(\etana{\tfocusl{x}{\tnil}}{A^-})})/z]N)}}
     {U}}
  {\infer-[{\it subst}^+]
   {\foct{\Psi}{\tackon{\Theta}{x{:}\istrue{A^-}}}
      {[(\tdownr{(\etana{\tfocusl{x}{\tnil}}{A^-})})/z]N}
      {U}}
   {\infer[{\downarrow}_R]
    {\foct{\Psi}{{\Xi}, {x{:}\istrue{A^-}}}{\tdownr{(\etana{\tfocusl{x}{\tnil}}{A^-})}}{[{\downarrow}A^-]}}
    {\infer-[{\eta}_{A^-}]
     {\foct{\Psi}{{\Xi},{x{:}\istrue{A^-}}}{\etana{\tfocusl{x}{\tnil}}{A^-}}{\isconc{A^-}}}
     {\infer[{\it focus}_L]
      {\foct{\Psi}{{\Xi}, {x{:}\istrue{A^-}}}{\tfocusl{x}{\tnil}}{\isconc{\susp{A^-}}}}
      {\infer[{\it id}^-]
       {\foct{\Psi}{{\Xi}, {[A^-]}}{\tnil}{\isconc{\susp{A^-}}}}
       {}}}}
    &
    \foct{\Psi}{\tackon{\Theta}{z{:}\istrue{\susp{{\downarrow}A^-}}}}{N}{U}}}\]

\item[--] $\etapa{z}{N}{{\scriptgnab}A^-}
           = \tgnabl{x}{([(\tgnabr{(\etana{\tfocusl{x}{\tnil}}{A^-})})/z]N)}$ 
%   \smallskip

%   $N$ is a derivation of 
%   $\foc{\Psi}{\tackon{\Theta}{z{:}\islvl{\susp{{\gnab}A^-}}}}{U}$, where
%   $\mlvl$ is $\mtrue$ or $\meph$.
%   By ${\gnab}_L$, it suffices to show 
%   $\foc{\Psi}{\tackon{\Theta}{x{:}\iseph{A^-}}}{U}$.
%   We construct a context $\Xi$ that matches $x{:}A^-$ such that
%   $\tackon{\Theta}{x{:}\iseph{A^-}}$ matches $\frameoff{\Theta}{\Xi}$
%   and $\Xi$ matches $\restrictto{\Xi}{\meph}$.
%   \smallskip

%   $\tfocusl{x}{\tnil}$ (that is, ${\it focus}_L$ followed by
%   ${\it id}^-$) is a derivation of $\foc{\Psi}{\Xi}{\istrue{\susp{A^-}}}$. 
%   By the induction hypothesis, we have $\foc{\Psi}{\Xi}{A^-}$, 
%   and by ${\gnab}_R$ and the fact that $\Xi$ matches $\restrictto{\Xi}{\meph}$
%   we have $\foc{\Psi}{\Xi}{[{\gnab}A^-]}$. The result follows
%   by focal substitution on this derivation into $N$.
%   \smallskip

\item[--] $\etapa{z}{N}{{!}A^-}
           = \tbangl{x}{([(\tbangr{(\etana{\tfocusl{x}{\tnil}}{A^-})})/z]N)}$ 
  \smallskip

  $N$ is a derivation of 
  $\foc{\Psi}{\tackon{\Theta}{z{:}\islvl{\susp{{!}A^-}}}}{U}$, where
  $\mlvl$ can be anything ($\mtrue$, $\meph$, or $\mpers$). We construct
  a context $\Xi$ that contains only the persistent propositions
  from $\Delta$ and a frame $\Theta^+$ that is $\Theta$ plus an extra
  variable declaration $x{:}\ispers{A^-}$. This means that
  $\tackon{\Theta}{x{:}\ispers{A^-}}$ matches
  $\frameoff{\Theta^+}{\restrictto{(\Xi, x{:}\ispers{A^-})}{\mlvl}}$. 
  We can then derive:
  \[\footnotesize \infer[{!}_L]
  {\foct{\Psi}{\frameoff{\Theta}{{!}A^-}}
     {\tbangl{x}{([(\tbangr{(\etana{\tfocusl{x}{\tnil}}{A^-})})/z]N)}}
     {U}}
  {\infer-[{\it subst}^+]
   {\foct{\Psi}{\tackon{\Theta}{x{:}\ispers{A^-}}}
      {[(\tbangr{(\etana{\tfocusl{x}{\tnil}}{A^-})})/z]N}
      {U}}
   {\infer[{!}_R]
    {\foct{\Psi}{{\Xi},{x{:}\ispers{A^-}}}{\tbangr{(\etana{\tfocusl{x}{\tnil}}{A^-})}}{[{!}A^-]}}
    {\infer-[{\eta}_{A^-}]
     {\foct{\Psi}{{\Xi},{x{:}\ispers{A^-}}}{\etana{\tfocusl{x}{\tnil}}{A^-}}{\isconc{A^-}}}
     {\infer[{\it focus}_L]
      {\foct{\Psi}{{\Xi},{x{:}\ispers{A^-}}}{\tfocusl{x}{\tnil}}{\isconc{\susp{A^-}}}}
      {\infer[{\it id}^-]
       {\foct{\Psi}{{\Xi},{x{:}\ispers{A^-}}, {[A^-]}}{\tnil}{\isconc{\susp{A^-}}}}
       {}}}}
    &
    \infer-[{\it weaken}]
    {\foct{\Psi}{\tackon{\Theta^+}{z{:}\islvl{\susp{{\downarrow}A^-}}}}{N}{U}}
    {\foct{\Psi}{\tackon{\Theta}{z{:}\islvl{\susp{{\downarrow}A^-}}}}{N}{U}}}}\]


\item[--] $\etapa{z}{N}{\one} = \tonel{([\toner/z]N)}$ 
\item[--] $\etapa{z}{N}{A^+_\mlvl \fuse B^+_\mlvl} =
            \tfusel{\big(\etapa{z_1}
             {~\etapa{z_2}
              {~[(\tfuser{z_1}{z_2})/z]N}
              {B^+_\mlvl}}
             {A^+_\mlvl}\big)}$

  \smallskip
  $N$ is a derivation of 
  $\foc{\Psi}{\tackon{\Theta}
   {z{:}\islvl{\susp{A^+_\mlvl \fuse B^+_\mlvl}}}}{U}$, where $\mlvl$
  can be anything ($\mtrue$, $\meph$, or $\mpers$). We construct a context
  $\Xi$ that contains only the persistent propositions from $\Delta$
  and a frame $\Theta^+$ that is either 
  $\Theta$ (if $\mlvl$ is $\mtrue$ or $\meph$) 
  or it is $\Theta$ plus additional variable declarations
  $z_1{:}\islvl{\susp{A^+_\mlvl}}$ and 
  $z_2{:}\islvl{\susp{B^+_\mlvl}}$ (if $\mlvl$ is $\mpers$). This means that
  $\tackon{\Theta}
                  {x_1{:}\islvl{\susp{A^+_\mlvl}}, 
                   x_2{:}\islvl{\susp{B^+_\mlvl}}}$ matches
  $\frameoff{\Theta^+}{\restrictto{(\Xi, x_1{:}\islvl{\susp{A^+_\mlvl}}, 
                   x_2{:}\islvl{\susp{B^+_\mlvl}})}{\mlvl}}$. 
   We can then derive:
  \[\footnotesize 
  \infer[{\fuse}_L]
  {\foct{\Psi}{\frameoff{\Theta}{A^+_\mlvl \fuse B^+_\mlvl}}{\tfusel{(\etapa{z_1}
             {~\etapa{z_2}
              {~[(\tfuser{z_1}{z_2})/z]N}
              {B^+_\mlvl}}
             {A^+_\mlvl})}}{U}}
  {\infer-[\eta_{A^+_\mlvl}]
   {\foct{\Psi}{\tackon{\Theta}{A^+_\mlvl, B^+_\mlvl}}{{\etapa{z_1}
             {~\etapa{z_2}
              {~[(\tfuser{z_1}{z_2})/z]N}
              {B^+_\mlvl}}
             {A^+_\mlvl}}}{U}}
   {\infer-[\eta_{B^+_\mlvl}]
    {\foct{\Psi}{\tackon{\Theta}{z_1{:}\islvl{\susp{A^+_\mlvl}}, B^+_\mlvl}}{{
             {\etapa{z_2}
              {~[(\tfuser{z_1}{z_2})/z]N}
              {B^+_\mlvl}}}}{U}}
    {\infer-[{\it subst}^+]
     {\foct{\Psi}{\tackon{\Theta}
                  {z_1{:}\islvl{\susp{A^+_\mlvl}}, 
                   z_2{:}\islvl{\susp{B^+_\mlvl}}}}{{{
              {[(\tfuser{z_1}{z_2})/z]N}}}}{U}}
     {\infer[{\fuse}_R]
      {\foct{\Psi}{\Xi, z_1{:}\islvl{\susp{A^+_\mlvl}}, 
                   z_2{:}\islvl{\susp{B^+_\mlvl}}}
             {\tfuser{z_1}{z_2}}{[A^+_\mlvl \fuse B^+_\mlvl]}}
      {\infer[{\it id}^+]
       {\foct{\Psi}{\Xi_1}{{z_1}}{[A^+_\mlvl]}}
       {}
       &
       \infer[{\it id}^+]
       {\foct{\Psi}{\Xi_2}{{z_2}}{[B^+_\mlvl]}}
       {}}
      &
      \infer-[{\it weaken}]
      {\foct{\Psi}{\tackon{\Theta^+}
       {z{:}\islvl{\susp{A^+_\mlvl \fuse B^+_\mlvl}}}}{N}{U}}
      {\foct{\Psi}{\tackon{\Theta}
       {z{:}\islvl{\susp{A^+_\mlvl \fuse B^+_\mlvl}}}}{N}{U}}}}}}
  \]
  Either $\Xi_1$ and $\Xi_2$ are both $\Xi, z_1{:}\islvl{\susp{A^+_\mlvl}}, 
                   z_2{:}\islvl{\susp{B^+_\mlvl}}$ (if $\mlvl$ is $\mpers$),
  or $\Xi_1$ is $\Xi, z_1{:}\islvl{\susp{A^+_\mlvl}}$ and
  $\Xi_2$ is $\Xi, z_2{:}\islvl{\susp{B^+_\mlvl}}$ (if $\mlvl$ is $\mtrue$
  or $\meph$). 
  \smallskip
 

\item[--] $\etapa{z}{N}{\zero} = \tabort$ 

\item[--] $\etapa{z}{N}{A^+_\mlvl \oplus B^+_\mlvl} = 
           \toplusl
            {\etapa{z_1}{~[\tinl{z_1}/z]N}{A^+_\mlvl}}
            {\etapa{z_2}{~[\tinr{z_2}/z]N}{B^+_\mlvl}}$
% \smallskip

% $N$ is a derivation 
% $\foc{\Psi}{\tackon{\Theta}{z{:}\susp{A^+_\mlvl \oplus B^+_\mlvl}}}{U}$.
% By $\oplus_L$, it suffices to show 
% $\foc{\Psi}{\tackon{\Theta}{A^+_\mlvl}}{U}$
% and
% $\foc{\Psi}{\tackon{\Theta}{B^+_\mlvl}}{U}$; we will show the first, as the
% two cases are symmetric. 
% \smallskip

% We construct a context $\Xi_1$ that matches $z_1{:}\susp{A^+_\mlvl}$
% such that $\tackon{\Theta}{z_1{:}\islvl{\susp{A^+_\mlvl}}}$ matches
% $\frameoff{\Theta_1}{\Xi}$. The frame $\Theta_1$ is either 
% $\Theta$ (if $\mlvl$ is $\mtrue$ or $\meph$) or it is $\Theta$
% plus an additional variable declaration $z_1{:}\islvl{\susp{A^+_\mlvl}}$ (if
% $\mlvl$ is $\mpers$). In either case, $N$ is also a derivation
% of $\foc{\Psi}{\tackon{\Theta_1}{z{:}\susp{A^+_\mlvl \oplus B^+_\mlvl}}}{U}$,
% either immediately or by by admissible weakening.
% \smallskip

% $\tinl{z_1}$ (that is, $\oplus_{R1}$ followed by ${\it id}^+$) is a
% derivation of $\foc{\Psi}{\Xi_1}{[A^+_\mlvl \oplus B^+_\mlvl]}$.  By
% focal substitution into $N$, we have 
% $\foc{\Psi}{\tackon{\Theta}{z_1{:}\islvl{\susp{A^+_\mlvl}}}}{U}$, and by the
% induction hypothesis on $A^+_\mlvl$ we have 
% $\foc{\Psi}{\tackon{\Theta}{A^+_\mlvl}}{U}$ as required.
% \smallskip

\item[--] $\etapa{z}{N}{\exists \lf a{:}\tau. A^+_\mlvl} 
           = \texistsl{\lf a}
              {\etapa{z'}
                {~[(\texistsr{\lf a}{z'})/z]N}
                {A^+}}$ 
% \smallskip

% $N$ is a derivation of 
% $\foc{\Psi}{\tackon{\Theta}{z{:}\islvl{\susp{\exists a{:}\tau. A^+_\mlvl}}}}
%   {U}$.
% By $\exists_L$, it suffices to show 
% $\foc{\Psi, a{:}\tau}{\tackon{\Theta}{A^+}}{U}$.
% We construct a context $\Xi$ that matches
% $z'{:}\susp{A^+_\mlvl}$ such that
% $\tackon{\Theta}{z'{:}\islvl{\susp{A^+_\mlvl}}}$ matches
% $\frameoff{\Theta'}{\Xi}$. The frame $\Theta'$ is either $\Theta$
% (if $\mlvl$ is $\mtrue$ or $\meph$) or it is $\Theta$ plus 
% an additional variable declaration $z'{:}\islvl{\susp{A^+_\mlvl}}$
% (if $\mlvl$ is $\mpers$). In either case,
% $N$ is a derivation of 
% $\foc{\Psi, a{:}\tau}
%   {\tackon{\Theta'}{z{:}\islvl{\susp{\exists a{:}\tau. A^+_\mlvl}}}}
%   {U}$ by variable weakening and (possibly) admissible weakening.

% \smallskip
% $\texistsr{a}{z'}$ (that is, $\exists_R$ followed by 
% an instance of ${\it id}^+$) is a derivation of 
% $\foc{\Psi, a{:}\tau}{\Xi}{[A^+_\mlvl]}$. By focal 
% substitution into $N$, we have a derivation of 
% $\foc{\Psi, a{:}\tau}{\tackon{\Theta}{z'{:}\islvl{\susp{A^+_\mlvl}}}}{U}$,
% and by the induction hypothesis on $A^+_\mlvl$ we have
% $\foc{\Psi, a{:}\tau}{\tackon{\Theta}{A^+}}{U}$ as required.

% \smallskip

\item[--] $\etapa{z}{N}{\lf{t} \doteq_\tau \lf{s}} 
           = \textsc{unif}\,({\sf fn}~ \lf\sigma \Rightarrow [\tunifr/z](\lf\sigma{N}))$
% \smallskip

% $N$ is a derivation of $\foc{\Psi}{\tackon{\Theta}{z{:}\islvl{\susp{t
%         \doteq s}}}}{U}$, and by $\doteq_L$, it suffices to show, for
% an arbitrary $\Psi' \vdash \sigma : \Psi$ such that $\sigma{t} =
% \sigma{s}$, that
% $\foc{\Psi'}{\tackon{\sigma\Theta}{\cdot}}{\sigma{U}}$.  We construct
% a context $\Xi$ containing just the persistent variable declarations in
% $\tackon{\sigma\Theta}{\cdot}$ ($\Xi$ therefore matches $\cdot$), and
% have that $\tackon{\sigma\Theta}{\cdot}$ matches
% $\frameoff{\sigma\Theta}{\Xi}$.

% Because $\sigma{t} = \sigma{s}$, $\tunifr$ (that is, the derivation
% consisting of ${\doteq}_R$) is a derivation of
% $\foc{\Psi'}{\Xi}{[\sigma{t} \doteq \sigma{s}]}$.  The result then
% follows by focal substitution into $\sigma{N}$, the derivation of
% $\foc{\Psi}{\tackon{\sigma\Theta}{z{:}\islvl{\susp{\sigma{t} \doteq
%         \sigma{s}}}}}{U}$ obtained from $N$ by variable substitution.

\end{itemize}

\subsubsection{Negative cases}

\begin{itemize}
\item[--] $\etana{N}{p^-_\mlvl} = \tetan{N}$
\item[--] $\etana{N}{{\uparrow}A^+} 
           = \tupr{([N](\tupl{(\etapa{z}{\tfocusr{z}}{A^+})}))}$
% \smallskip

% $N$ is a derivation of $\foc{\Psi}{\Delta}{\istrue{\susp{{\uparrow}A^+}}}$.
% By ${\uparrow}_R$, it suffices to show $\foc{\Psi}{\Delta}{\istrue{A^+}}$.
% We construct the empty (except for persistent resources) 
% frame $\Theta$ such that 
% $\tackon{\Theta}{z{:}\istrue{\susp{A^+}}}$ matches $z{:}\susp{A^+}$
% and $\Delta$ matches $\frameoff{\Theta}{\Delta}$. 
% \smallskip

% $\tfocusr{z}$ (that is, ${\it focus}_R$ followed by ${\it id}^+$) is
% a derivation of 
% $\foc{\Psi}{\tackon{\Theta}{z{:}\istrue{\susp{A^+}}}}{\istrue{A^+}}$. 
% By the induction hypothesis on $A^+$, we have
% $\foc{\Psi}{\tackon{\Theta}{A^+}}{\istrue{A^+}}$, and 
% by rule ${\uparrow}_L$ we then have
% $\foc{\Psi}{\tackon{\Theta}{[{\uparrow}A^+]}}{\istrue{A^+}}$, 
% and the result follows by focal substitution out of $N$.
% \smallskip

\item[--] $\etana{N}{{\ocircle}A^+} 
           = \tlaxr{[N](\tlaxl{\etapa{z}{\tfocusr{z}}{A^+}})}$
% \smallskip

% $N$ is a derivation of 
% $\foc{\Psi}{\Delta}{\islvl{\susp{{\ocircle}A^+}}}$, where
% $\mlvl$ is $\mtrue$ or $\mlax$.
% By ${\ocircle}_R$, it suffices to show $\foc{\Psi}{\Delta}{\islax{A^+}}$.
% We construct the empty (except for persistent resources) 
% frame $\Theta$ such that 
% $\tackon{\Theta}{z{:}\istrue{\susp{A^+}}}$ matches $z{:}\susp{A^+}$
% and $\Delta$ matches $\frameoff{\Theta}{\Delta}$. 
% \smallskip

% $\tfocusr{z}$ (that is, ${\it focus}_R$ followed by ${\it id}^+$) is
% a derivation of 
% $\foc{\Psi}{\tackon{\Theta}{z{:}\istrue{\susp{A^+}}}}{\islax{A^+}}$. 
% By the induction hypothesis on $A^+$, we have
% $\foc{\Psi}{\tackon{\Theta}{A^+}}{\islax{A^+}}$, and 
% by rule ${\ocircle}_L$ we then have
% $\foc{\Psi}{\tackon{\Theta}{[{\ocircle}A^+]}}{\islax{A^+}}$.
% The result follows by focal substitution out of $N$.
% \smallskip

\item[--] $\etana{N}{A^+ \lefti B^-_\mlvl}
           = \tlaml{(\etapa{z}{~\etana{[N](\tappl{z}{\tnil})}{B^-_\mlvl}}{A^+})}$
\item[--] $\etana{N}{A^+ \righti B^-_\mlvl}
           = \tlamr{(\etapa{z}{~\etana{[N](\tappr{z}{\tnil})}{B^-_\mlvl}}{A^+})}$
% \smallskip

% $N$ is a derivation of 
% $\foc{\Psi}{\Delta}{\islvl{\susp{A^+ \righti B^-_\mlvl}}}$. By 
% ${\righti}_R$, it suffices to show $\foc{\Psi}{\Delta, A^+}{B^-_\mlvl}$.
% We construct a frame $\Theta$ with $\Delta$'s persistent resources and 
% a hole to the left of the variable declaration $z{:}\istrue{\susp{A^+}}$; therefore,
% $\mkconj{\Delta}{z{:}\istrue{\susp{A^+}}}$ matches 
% $\frameoff{\Theta}{\Delta}$.

% % empty (except for persistent resources)
% %frame $\Theta$ such that $\tackon{}{}$
% % \smallskip

% % $\tappr{z}{\tnil}$ (that is, ${\righti}_R$ followed by ${\it id}^+$ and
% % ${\it id}^-$) is a derivation of 
% % $\foc{\Psi}{[A^+ \righti B^-_\mlvl],\Xi}{\islvl{\susp{B^-_\mlvl}}}$.
% % By focal substitution out of $N$, we have
% % $\foc{\Psi}{\tackon{\Theta}{x{:}\istrue{\susp{A^+}}}}
% % {\islvl{\susp{B^-_\mlvl}}}$

% $\tappr{z}{\tnil}$ (that is, ${\righti}_R$ followed by ${\it id}^+$ and
% ${\it id}^-$) is a derivation of 
% $\foc{\Psi}{\tackon{\Theta}{[A^+ \righti B^-_\mlvl]}}{\islvl{\susp{B^-_\mlvl}}}$.
% By focal substitution out of $N$, we have a derivation of 
% $\foc{\Psi}{\Delta, z{:}\istrue{\susp{A^+}}}{\islvl{\susp{B^-_\mlvl}}}$,
% by the induction hypothesis on $B^-_\mlvl$ we have
% $\foc{\Psi}{\Delta, z{:}\istrue{\susp{A^+}}}{B^-_\mlvl}$, and
% by the induction hypothesis on $A^+$ we have
% $\foc{\Psi}{\Delta, A^+}{B^-_\mlvl}$ as required.

% \smallskip

\item[--] $\etana{N}{\top} = \top$ 
\item[--] $\etana{N}{A^-_\mlvl \with B^-_\mlvl}
           = \twithr
              {(\etana{[N](\tpione{\tnil})}{A^-_\mlvl})}
              {(\etana{[N](\tpitwo{\tnil})}{B^-_\mlvl})}$
% \smallskip

% $N$ is a derivation of 
% $\foc{\Psi}{\Delta}{\islvl{\susp{A^-_\mlvl \with B^-_\mlvl}}}$. By 
% $\with_R$, it suffices to show $\foc{\Psi}{\Delta}{A^-_\mlvl}$
% and $\foc{\Psi}{\Delta}{B^-_\mlvl}$; we will show the first, as the two
% cases are symmetric.
% \smallskip

% We construct the empty (except for persistent resources) frame
% $\Theta$ such that $\Delta$ matches $\frameoff{\Theta}{\Delta}$. 
% $\tpione{\tnil}$ (that is, ${\with}_{R1}$ followed by ${\it id}^-$) is
% a derivation of 
% $\foc{\Psi}{\tackon{\Theta}{[A^-_\mlvl \with B^-_\mlvl]}}
%   {\islvl{\susp{A^-_\mlvl}}}$.
% By focal substitution out of $N$, we have 
% a derivation of $\foc{\Psi}{\Delta}{\islvl{\susp{A^-_\mlvl}}}$, and
% by the induction 
% hypothesis on $A^-_\mlvl$ we have 
%  $\foc{\Psi}{\Delta}{A^-_\mlvl}$ as required.
% \smallskip

\item[--] $\etana{N}{\forall \lf{a}{:}\tau.A^-_\mlvl}
           = \tforallr{\lf a}{(\etana{[N](\tforalll{\lf a}{\tnil})}{A^-_\mlvl})}$

\end{itemize}

\end{proof}

\section{Correctness of focusing}
\label{sec:ord-correctness}

Our proof of the correctness of focusing is based on erasure as
described in Section~\ref{sec:lincorrectness}. 
The argument follows the one from the
structural focalization development, and the key component is the
set of {\it
  unfocused admissibility lemmas}, lemmas that establish that each of
the reasoning steps that can be made in unfocused \ollll~are
admissible inferences made on stable sequents in focused \ollll.

% In fact, once we have established the unfocused admissibility lemmas,
% the proof of the completeness of focusing is arguably a bit simpler
% than the mechanized structural focalization result for persistent
% logic in \cite{simmons11structural}. In the soundness proof of that
% paper, we had to work around a separate inversion context that
% enforced our fixed inversion strategy, but for us that structure is
% already present in the stucutre of the ordered context. In the
% completeness proof of that paper, on the other hand, we had to account
% for the fact that our 


\subsection{Erasure}

As in Section~\ref{sec:lincorrectness}, 
we define erasure only on stable, suspension-normal
sequents. Erasure for propositions is defined as in
Figure~\ref{fig:ord-erasure}. As discussed in
Section~\ref{sec:permable-atomic}, even though we have not
incorporated a notion of permeable and mobile atomic propositions into
the unfocused presentation of \ollll, it is possible to erase a
permeable atomic proposition $p^+_\mpers$ as
${!}p^+_\mpers$.\footnote{The polarity and level annotations are
  meaningless in the unfocused logic. We keep them only to emphasize
  that $p^+_\mpers$ and $p^-_\mlax$ do $\it not$ erase to the same
  unpolarized atomic proposition $p$ but two distinct unpolarized
  atomic propositions.}  In this way, we can see the separation
criteria from our previous work
\cite{simmons08linear,pfenning09substructural} arising as an emergent
property of erasure.

We have to define erasure on non-stable sequents in order for the
soundness of focusing to go through, though we will only define
erasure on suspension-normal sequents.  The erasure of sequents,
$U^\circ$, maps polarized succedents $\islvl{A^+}$,
$\islvl{\susp{p^-_\mlvl}}$, $[A^+]$, and $A^-$ in the obvious way to
unpolarized succedents $\islvl{(A^+)^\circ}$, $\islvl{p^-_\mlvl}$,
$\istrue{(A^+)^\circ}$, and $\istrue{(A^-)^\circ}$, respectively.  To
describe the erasure of contexts more simply, we will assume that we
can give a presentation of unfocused \ollll~that uses unified
substructural contexts, as we outlined in
Section~\ref{sec:contexts}; the judgments of this presentation
have the form $\Psi; \Delta \altv U$. 
In this presentation, we can define $\Delta^\circ$ that
takes every variable declaration $x{:}\islvl{A^-}$, $x{:}\islvl{\susp{p^+_\mlvl}}$,
$[A^-]$, or $A^+$ to a variable declaration $x{:}\islvl{(A^-)^\circ}$,
$x{:}\islvl{p^+_\mlvl}$, $x{:}\istrue{(A^-)^\circ}$, or
$x{:}\istrue{(A^+)^\circ}$ (and in the process, either comes up with
or reveals the suppressed variable names associated with focused
negative propositions and inverting positive propositions).
Erasure of succedents $U^\circ$ is similar: 
$(\islvl{A^+})^\circ = \islvl{(A^+)^\circ}$,
$(\islvl{\susp{p^-_\mlvl}})^\circ = \islvl{p^-_\mlvl}$, 
$([A^-])^\circ = \isconc{(A^-)^\circ}$, and
$(A^+)^\circ = \isconc{(A^+)^\circ}$. 

\begin{figure}
{\small \[
\begin{array}{rcl|rcl}
\fbox{$(A^+)^\circ$} & & &
\fbox{$(A^-)^\circ$} & & 
\\
(p^+)^\circ & \!\!\!=\!\!\! & p^+ &
(p^-)^\circ & \!\!\!=\!\!\! & p^- \\
(p^+_\meph)^\circ & \!\!\!=\!\!\! & {\gnab}p^+_\meph &
(p^-_\mlax)^\circ & \!\!\!=\!\!\! & {\ocircle}p^-_\mlax 
\\
(p^+_\mpers)^\circ & \!\!\!=\!\!\! & {!}p^+_\mpers &
& & 
\\
({\downarrow}A^-)^\circ & \!\!\!=\!\!\! & (A^-)^\circ &
({\uparrow}A^+)^\circ & \!\!\!=\!\!\! & (A^+)^\circ 
\\
({\gnab}A^-)^\circ & \!\!\!=\!\!\! & {\gnab}(A^-)^\circ &
({\ocircle}A^+)^\circ & \!\!\!=\!\!\! & {\ocircle}(A^+)^\circ 
\\
({!}A^-)^\circ & \!\!\!=\!\!\! & {!}(A^-)^\circ &
& & 
\\
(\one)^\circ & \!\!\!=\!\!\! & \one &
(A^+ \lefti B^-)^\circ & \!\!\!=\!\!\! & (A^+)^\circ \lefti (B^+)^\circ 
\\
(A^+ \fuse B^+)^\circ & \!\!\!=\!\!\! & (A^+)^\circ \fuse (B^+)^\circ &
(A^+ \righti B^-)^\circ & \!\!\!=\!\!\! & (A^+)^\circ \righti (B^-)^\circ 
\\
(\zero)^\circ & \!\!\!=\!\!\! & \zero &
(\top)^\circ & \!\!\!=\!\!\! & \top 
\\
(A^+ \oplus B^+)^\circ & \!\!\!=\!\!\! & (A^+)^\circ \oplus (B^+)^\circ &
(A^- \with B^-)^\circ & \!\!\!=\!\!\! & (A^-)^\circ \with (B^-)^\circ 
\\
(\exists\lf a{:}\tau.A^+)^\circ & \!\!\!=\!\!\! & \exists\lf a{:}\tau. (A^+)^\circ &
(\forall\lf a{:}\tau.A^-)^\circ & \!\!\!=\!\!\! & \forall\lf a{:}\tau. (A^-)^\circ 
\\
(\lf t \doteq \lf s)^\circ & \!\!\!=\!\!\! & \lf t \doteq \lf s &
& &
\end{array}
\]}

\caption{Erasure in \ollll}
\label{fig:ord-erasure}
\end{figure}


\subsection{De-focalization}

The act of taking a focused proof of a sequent and getting an unfocused
proof of the corresponding erased sequent is {\it de-focalization}.
If we run the constructive content of 
the proof of the soundness of focusing (the \ollll~analogue of
Theorem~\ref{thm:linfocsound} from Section~\ref{sec:lincorrectness}),
the proof performs de-focalization.

\bigskip
\begin{theorem}[Soundness of focusing/de-focalization]
If $\foc{\Psi}{\Delta}{U}$, then $\Psi; \Delta^\circ \altv U^\circ$.
\end{theorem}

\begin{proof}
  By induction over the structure of focused proofs. 
  Most rules (${\fuse}_L$, $\righti_R$, etc.) in the
  focused derivations have an obviously analogous rule in the unfocused
  logic, and for the four rules dealing with shifts,
  the necessary result follows directly from the induction hypothesis. 
  The ${\it focus}_L$ rule potentially requires an instance of the
  admissible ${\it copy}$ or ${\it place}$ rules in unfocused \ollll, and
  the ${\it focus}_R$ rule potentially requires an instance of the
  admissible ${\it lax}$ rule in unfocused \ollll.
\end{proof}

\begin{figure}[tp]
\small

{\it Atomic propositions}
\[
\infer-
{\foct{\Psi}{z{:}\susp{p^+_\mlvl}}
  {\tfocusr{z}}
  {\islvl{p^+_\mlvl}'}}
{}
\quad
\infer-
{\foct{\Psi}{x{:}{\uparrow}p^+_\mlvl}
  {\tfocusl{x}{\tupl{\tetap{z}{~\tfocusr{z}}}}}
  {\islvl{p^+_\mlvl}'}}
{}
\]
\[
\infer-
{\foct{\Psi}{x{:}{p^-_\mlvl}}
  {\tfocusl{x}{\tnil}}
  {\islvl{\susp{p^-_\mlvl}}}}
{}
\quad
\infer-
{\foct{\Psi}{x{:}p^-_\mlvl}
  {\tfocusr{\tdownr{\tetan{\tfocusl{x}{\tnil}}}}}
  {\islvl{{\downarrow}p^-_\mlvl}}}
{}
\]

\medskip
{\it Exponentials}
\[
\infer-
{\foct{\Psi}{\Delta}
  {\tfocusr{\tdownr{\tupr{N}}}}
  {\islvl{{\downarrow}{\uparrow}A^+}}}
{\foct{\Psi}{\Delta}{N}{\istrue{A^+}}}
\qquad
\infer-
{\foct{\Psi}{\frameoff{\Theta}{x{:}{\uparrow}{\downarrow}A^-}}
  {\tfocusl{x}{\tupl{\tdownl{x'}{N}}}}
  {U}}
{\foct{\Psi}{\tackon{\Theta}{x'{:}\istrue{A^-}}}{N}{U}}
\]
\[
\infer-
{\foct{\Psi}{\restrictto{\Delta}{\meph}}
  {\rsubsta{\tupr{N}}{x}
    {\tfocusr
      {\tgnabr
        {\etana
          {\tfocusl{x}{\tupl{\tdownl{x'}{~\tfocusl{x'}{\tnil}}}}}
          {A^-}}}}
    {{\uparrow}{\downarrow}A^-}}{\islvl{{\gnab}A^-}}}
{\foct{\Psi}{\Delta}{N}{\istrue{{\downarrow}A^-}}}
\quad
\infer-
{\foct{\Psi}{\frameoff{\Theta}{x{:}{\uparrow}{\gnab}A^-}}
  {\tfocusl{x}{\tupl{\tgnabl{x'}{N}}}}
  {U}}
{\foct{\Psi}{\tackon{\Theta}{x'{:}\iseph{A^-}}}{N}{U}}
\]
\[
\infer-
{\foct{\Psi}{\restrictto{\Delta}{\mpers}}
  {\rsubsta{\tupr{N}}{x}
    {\tfocusr
      {\tbangr
        {\etana
          {\tfocusl{x}{\tupl{\tdownl{x'}{~\tfocusl{x'}{\tnil}}}}}
          {A^-}}}}
    {{\uparrow}{\downarrow}A^-}}{\islvl{{!}A^-}}}
{\foct{\Psi}{\Delta}{N}{\istrue{{\downarrow}A^-}}}
\quad
\infer-
{\foct{\Psi}{\frameoff{\Theta}{x{:}{\uparrow}{!}A^-}}
  {\tfocusl{x}{\tupl{\tbangl{x'}{N}}}}
  {U}}
{\foct{\Psi}{\tackon{\Theta}{x'{:}\ispers{A^-}}}{N}{U}}
\]
\[
\infer-
{\foct{\Psi}{\Delta}
  {\tfocusr{\tdownr{\tlaxr{N}}}}
  {\islvl{{\downarrow}{\ocircle}A^+}}}
{\foct{\Psi}{\Delta}{N}{\islax{A^+}}}
\quad
\infer-
{\foct{\Psi}{\frameoff{\Theta}{x{:}{\ocircle}A^+}}
  {\lsubsta
    {\tfocusl{x}{\etapa{z}{~\tfocusr{\tdownr{\tupr{\tfocusr{z}}}}}{A^+}}}
    {\tdownl{x'}{N}}{{\downarrow}{\uparrow}A^+}}
  {\restrictfrom{U}{\mlax}}}
{\foct{\Psi}{\tackon{\Theta}{x'{:}\istrue{{\uparrow}A^+}}}{N}{U}}
\]

\medskip
{\it Multiplicative connectives ($\lefti$ and $\righti$ are symmetric)}
\[
\infer-
{\foct{\Psi}{\cdot}{\tfocusr{\toner}}{\islvl{\one}}}
{}
\quad
\infer-
{\foct{\Psi}{\frameoff{\Theta}{x{:}{\uparrow}\one}}
  {\tfocusl{x}{\tupl{\tonel{N}}}}{U}}
{\foct{\Psi}{\tackon{\Theta}{\cdot}}{N}{U}}
\]
\[
\infer-
{\foct{\Psi}{\matchconj{\Delta_1}{\Delta_2}}
  {\rsubsta{\tupr{N_1}}{x_1}
    {(\lsubsta{N_2}
      {\etapa{z_2}
        {~\tfocusl{x_1}
          {\tupl
            {\etapa{z_1}
              {~\tfocusr{\tfuser{z_1}{z_2}}} 
              {A^+}}}}
        {B^+}}
      {B^+})}
    {{\uparrow}A^+}}
  {\islvl{A^+ \fuse B^+}}}
{\foct{\Psi}{\Delta_1}{N_1}{\istrue{A^+}}
 &
 \foct{\Psi}{\Delta_2}{N_2}{\istrue{B^+}}}
\]
\[
\infer-
{\foct{\Psi}{\frameoff{\Theta}{x{:}{\uparrow}(A^+ \fuse B^+)}}
  {\tfocusl{x}
    {\tupl
      {\lsubsta
        {\etapa{z_1}
          {~\etapa{z_2}
            {~\tfocusr
              {\tfuser
                {\tdownr{\tupr{\tfocusr{z_1}}}}
                {\tdownr{\tupr{\tfocusr{z_2}}}}}}
            {B^+}}
          {A^+}}
        {\tfusel{(\tdownl{x_1}{\tdownl{x_2}{N}})}}
        {{\downarrow}{\uparrow}A^+ \fuse 
         {\downarrow}{\uparrow}B^+}}}}
  {U}}
{\foct{\Psi}
  {\tackon{\Theta}{x_1{:}{\uparrow}A^+,x_2{:}{\uparrow}B^+}}{N}{U}}
\]
\[
\infer-
{\foct{\Psi}{\Delta}
  {\rsubsta{(\tlaml{(\tdownl{x}{\tupr{N}})})}{x'}
    {(\tfocusr
      {\tdownr
       {\tlaml
        {\etapa{z}
          {~\etana
            {\tfocusl{x'}
              {\tappl
                {(\tdownr{\tupr{\tfocusr{z}}})}
                {(\tupl{\tdownl{x''}{\tfocusl{x''}{\tnil}}})}}}
            {B^-}}
          {A^+}}}})}
    {{\downarrow}{\uparrow}A^+ \lefti {\uparrow}{\downarrow}B^-}}
  {\istrue{{\downarrow}(A^+ \lefti B^-)}}}
{\foct{\Psi}{x{:}\istrue{{\uparrow}A^+},\Delta}{N}{\istrue{{\downarrow}B^-}}}
\]
\[
\infer-
{\foct{\Psi}{\frameoff{}{\matchconj{\Delta_A}{x{:}{A^+ \lefti B^-}}}}
  {\lsubst
    {\lsubst
      {N}
      {\etapa{z}
        {~\tfocusr
          {\tdownr
            {\etana
              {\tfocusl{x}
                {\tappl{z}{\tnil}}}
              {B^-}}}}
        {A^+}}}
    {\tdownl{x'}{N_2}}}
  {U}}
{\foct{\Psi}{\Delta_A}{N_1}{\istrue{{A^+}}}
 &
 \foct{\Psi}{\tackon{\Theta}{x'{:}\istrue{B^-}}}{N_2}{U}}
\]

\caption{Unfocused admissibility for the 
multiplicative, exponential fragment of \ollll}
\label{fig:admit-mell}
\end{figure}

\begin{figure}[tp]
\[
\infer-
{\foct{\Psi}{\frameoff{\Theta}{x{:}{\uparrow}{\zero}}}
  {\tfocusl{x}{\tupl{\tabort}}}
  {U}}
{}
\qquad
\infer-
{\foct{\Psi}{\Delta}
  {\lsubst{N_1}{\etapa{z}{\tfocusr{\tinl{z}}}{A^+}}{A^+}}
  {\islvl{A^+ \oplus B^+}}}
{\foct{\Psi}{\Delta}{N_1}{\istrue{A^+}}}
\]
\[
\infer-
{\foct{\Psi}{\frameoff{\Theta}{x{:}{\uparrow}(A^+ \oplus B^+)}}
  {\tfocusl{x}{\tupl{\lsubst
      {\toplusl{\etapa{y}{\tfocusr{\tdownr{\tupr{\tfocusr{y}}}}}{A^+}}
               {\etapa{y}{\tfocusr{\tdownr{\tupr{\tfocusr{y}}}}}{B^+}}}
      {\toplusl{\tdownl{x_1}{N_1}}{\tdownl{x_2}{N_2}}}}}}
  {U}}
{\foct{\Psi}{\tackon{\Theta}{x_1{:}\istrue{{\uparrow}A^+}}}{N_1}{U}
 &
 \foct{\Psi}{\tackon{\Theta}{x_1{:}\istrue{{\uparrow}B^+}}}{N_2}{U}}
\]
\[
\infer-
{\foct{\Psi}{\Delta}{\tfocusr{\tdownr{\ttopr}}}{\islvl{{\downarrow}\top}}}
{}
\qquad
\infer-
{\foct{\Psi}{\frameoff{\Theta}{x{:}A^- \with B^-}}
  {\rsubst{\etana{\tfocusl{x}{\tpione{\tnil}}}{A^-}}{x_1}{N_1}}
  {U}}
{\foct{\Psi}{\tackon{\Theta}{x_1{:}\istrue{A^-}}}{N_1}{U}}
\]
\[
\infer-
{\foct{\Psi}{\Delta}
  {\rsubst{(\twithr{\tupr{N_1}}{\tupr{N_2}})}{x}
      {\tfocusr{\tdownr{(\twithr
         {\etana{\tfocusl{x}{\tpione{\tupl{(\tdownl{y}{\,\tfocusl{y}{\tnil}})}}}}{A^-}}
         {\etana{\ldots}{B^-}})}}}}
  {\islvl{{\downarrow}(A^- \with B^-)}}}
{\foct{\Psi}{\Delta}{N_1}{\istrue{{\downarrow}A^-}}
 &
 \foct{\Psi}{\Delta}{N_2}{\istrue{{\downarrow}B^-}}}
\]

\caption{Unfocused admissibility for the additive connectives of 
\ollll~(omits ${\oplus}_{R2}$, ${\with}_{L2}$)}
\label{fig:admit-additive}
\end{figure}

\begin{figure}[t]
\[
\infer-
{\foct{\Psi}{\Delta}
  {\lsubst{N}{(\etapa{z}{\,(\tfocusr{\texistsr{\lf t}{z}})}{\lf{[t/a]}A^+})}}
  {\islvl{\exists\lf{a}{:}\tau.A^+}}}
{\Psi \vdash \lf{t} : \tau
 & 
 \foct{\Psi}{\Delta}{N}{\istrue{\lf{[t/a]}A^+}}}
\]
\[
\infer-
{\foct{\Psi}{\frameoff{\Theta}{x{:}{\uparrow}(\exists \lf{a}{:}\tau. A^+)}}
  {\tfocusl{x}{\tupl{
      \lsubsta
        {\texistsl{\lf a}{\etapa{z}{\,(\tfocusr{\texistsr{\lf a}{\tdownr{\tupr{\tfocusr{z}}}}})}{A^+}}}
        {(\texistsl{\lf a}{\tdownl{x'}{N}})}
        {\exists \lf{a}{:}\tau.{\downarrow}{\uparrow}A^+}}}} 
  {U}}
{\foct{\Psi,\lf{a}{:}\tau}{\tackon{\Theta}{x'{:}\istrue{{\uparrow}A^+}}}
  {N} 
  {U}}
\]
\[
\infer-
{\foct{\Psi}{\Delta}
  {\rsubsta{\tforallr{\lf a}{\tupr{N}}}{x}
    {\tfocusr{\tupr{(\tforallr{\lf a}{\etana{\tfocusl{x}{\tforalll{\lf a}{\tupl{(\tdownl{y}{\tfocusl{y}{\tnil}})}}}}{A^-}})}}}
    {\forall\lf{a}{:}\tau.{\uparrow}{\downarrow}A^-}}
  {\islvl{{\downarrow}(\forall \lf{a}{:}\tau. A^-)}}}
{\foct{\Psi, \lf{a}{:}\tau}{\Delta}
  {N}
  {\istrue{{\downarrow}A^-}}}
\]
\[
\infer-
{\foct{\Psi}{\frameoff{\Theta}{x{:}\forall \lf{a}{:}\tau. A^-}}
  {\rsubst{\etana{\tfocusl{x}{\tforalll{\lf a}{\tnil}}}{\lf{[t/a]}A^-}}{x'}{N'}}
  {U}}
{\Psi \vdash \lf{t} : \tau
 &
 \foct{\Psi}{\tackon{\Theta}{x'{:}\istrue{\lf{[t/a]}A^-}}}
  {N}
  {U}}
\quad
\infer-
{\foct{\Psi}{\cdot}
  {\tfocusr{\tunifr}}
  {\islvl{\lf t \doteq \lf t}}}
{}
\]
\[
\infer-
{\foct{\Psi}{\frameoff{\Theta}{x{:}{\uparrow}({\lf t \doteq \lf s})}}
  {\tfocusl{x}{\tupl{(\textsc{unif}\,({\sf fn}~ \lf \sigma \Rightarrow \phi(\lf \sigma)))}}}
  {U}}
{\forall(\Psi' \vdash \lf\sigma : \Psi).
 &
 \lf{\sigma t} = \lf{\sigma s}
 &
 \rightarrow
 &
 \foct{\Psi'}{\tackon{\lf\sigma\Theta}{\cdot}}
  {\phi(\lf\sigma)}
  {\lf\sigma U}}
\]


\caption{Unfocused admissibility for the first-order connectives of \ollll}
\label{fig:admit-fo}
\end{figure}

\subsection{Unfocused admissibility}
\label{sec:ord-unfocused-admissibility}

Unfocused admissibility has a structure that is unchanged from the
previous discussion in the proof of the completeness of focusing for
linear logic (Theorem~\ref{thm:linfoccomplete} in
Section~\ref{sec:lincorrectness}). In this presentation, as in the
structural focalization development, we present unfocused
admissibility primarily on the level of proof terms. The resulting
presentation is quite dense; proofs of this variety really ought to be
mechanized, though we leave that for future work.

For the most part, there is exactly one unfocused admissibility rule
for each rule of unfocused \ollll. 
The justifications for the 
unfocused admissibility lemmas for the multiplicative, exponential
fragment of \ollll~are given in Figure~\ref{fig:admit-mell}; the
additive fragment is given in Figure~\ref{fig:admit-additive}, and
the first-order connectives are treated in Figure~\ref{fig:admit-fo}.
There are two additional
rules that account for the fact that different polarized
propositions, like ${\downarrow}{\uparrow}{\downarrow}{\uparrow}A^+$
and $A^+$ erase to the same unpolarized proposition
$(A^+)^\circ$. 
For the same reason, Figure~\ref{fig:admit-mell} contains four
${\it init}$-like rules, since atomic propositions can come in positive
and negative varieties and can appear in the context either suspended or not.

We can view unfocused admissibility as creating an abstraction layer
of admissible rules that can be used to build focused proofs of stable
sequents.  The proof of the completeness of focusing below constructs
focused proofs entirely by working through the interface layer of
unfocused admissibility.


\subsection{Focalization}

The act of taking an unfocused proof of an erased sequent and getting
a focused proof of the un-erased sequent is {\it focalization}. If 
we run the constructive content of the proof of the completeness of
focusing (the \ollll~analogue of Theorem~\ref{thm:linfoccomplete} from
Section~\ref{sec:lincorrectness}), which takes any stable, 
suspension-normal sequent as input, the proof performs focalization.

\bigskip
\begin{theorem}[Complteness of focusing/focalization]~\\
If $\Psi; \Delta^\circ \altv U^\circ$, where $\Delta$ and $U$ are 
stable and suspension-normal, then $\foc{\Psi}{\Delta}{U}$. 
\end{theorem}

\begin{proof}
By an outer induction on the structure of unfocused proofs and
an inner induction over the structure of polarized formulas $A^+$ and
$A^-$ in order to remove series of shifts 
${\uparrow}{\downarrow}\ldots{\uparrow}{\downarrow}A^-$ from formulas until
an unfocused admissibility lemma can be applied.
\end{proof}

\section{Properties of syntactic fragments}
\label{sec:perm-fragments}

In the structural focalization methodology, once 
cut admissibility and identity expansion are established the only
interesting part of the proof of the completeness of focusing is 
the definition of an erasure function and the presentation of a 
series of unfocused admissibility lemmas. The unfocused admissibility
lemmas for non-invertible rules, like ${\fuse}_R$ and 
${\lefti}_L$, look straightforward:
\[
\infer-
{\foc{\Psi}{\matchconj{\Delta_1}{\Delta_2}}{\islvl{A^+ \fuse B^+}}}
{\foc{\Psi}{\Delta_1}{\isconc{A^+}}
 &
 \foc{\Psi}{\Delta_2}{\isconc{B^+}}}
\quad
\infer-
{\foc{\Psi}{\frameoff{\Theta}{\matchconj{\Delta_A}{x{:}A^+ \lefti B^-}}}{U}}
{\foc{\Psi}{\Delta_A}{\istrue{A^+}}
 &
 \foc{\Psi}{\tackon{\Theta}{x'{:}\istrue{B^-}}}{U}}
\]
Because unfocused admissibility is defined only on
stable sequents in our methodology, the invertible rules,
like ${\fuse}_L$ and ${\lefti}_R$, require the presence of shifts:
\[
\infer-
{\foc{\Psi}{\frameoff{\Theta}{x{:}{\uparrow}(A^+ \fuse B^+)}}{U}}
{\foc{\Psi}{\tackon{\Theta}{\mkconj{x_1{:}\istrue{{\uparrow}A^+}}{x_2{:}\istrue{{\uparrow}B^+}}}}{U}}
\quad
\infer-
{\foc{\Psi}{\Delta}{\islvl{{\downarrow}(A^+ \lefti B^-)}}}
{\foc{\Psi}{\mkconj{x{:}\istrue{{\uparrow}A^+}}{\Delta}}{\isconc{{\downarrow}B^-}}}
\]
The presence of shifts is curious, due to our observation in
Section~\ref{sec:ordpolarprop} that the shifts have much of the character
of exponentials; they are 
exponentials that do not place any restrictions on the form of the
context.

As a thought experiment, imagine the removal of shifts ${\uparrow}$
and ${\downarrow}$ from the language of propositions in \ollll. Were
it not for the presence of atomic propositions $p^+$ and $p^-$, this
change would make every proposition $A^+$ a mobile proposition
$A^+_\meph$ and would make every proposition $A^-$ a right-permeable
proposition $A^-_\mlax$. But arbitrary atomic propositions
are intended to be stand-ins for arbitrary propositions! If arbitrary
propositions lack shifts, then non-mobile atomic propositions would
appear to no longer stand for anything. Therefore, let's
remove them too, leaving only the permeable, mobile,
and right-permeable atomic propositions $p^+_\mpers$, $p^+_\meph$, and
$p^-_\mlax$. Having done so, every positive proposition is mobile, and
every negative proposition is right-permeable.

Now we have a logical fragment where every positive proposition is 
mobile and every negative proposition is observed to be
right-permeable. Consider a derivation
$\foc{\Psi}{\Delta}{\islax{A^+}}$ where $\Delta$ is stable and
includes only linear and persistent judgments (that is, 
$\restrictto{\Delta}{\meph}$). It is simple to observe that, for every
subderivation $\foc{\Psi'}{\Delta'}{U'}$, if $\Delta'$ is stable
then $\Delta' = \restrictto{\Delta'}{\meph}$, and if $U$ is stable
then $U = \restrictfrom{U}{\mlax}$. Given that this is the case, the
restrictions that the focused ${\gnab}_R$ and ${\ocircle}_L$ rules
make are {\it always satisfiable}, the same property that we previously
observed of focused shift rules ${\downarrow}_R$ and ${\uparrow}_L$.
In our syntactic fragment, in other words, the exponentials 
${\gnab}$ and ${\ocircle}$ have become effective {\it replacements} for
${\downarrow}$ and ${\uparrow}$. 

The cut and identity theorems survive our restriction of the logic
entirely intact: these theorems handle each of the connectives
separately and are stable to the addition or removal of individual
connectives.  That is not true for the unfocused admissibility lemmas,
which critically and heavily use shifts. However, while we no longer
have our original shifts, we have replacement shifts in the form of
${\gnab}$ and ${\ocircle}$, and can replay the logic of the unfocused
admissibility lemmas in order to gain new ones that look like this:
\[
\infer-
{\foc{\Psi}{\matchconj{\Delta_1}{\Delta_2}}{\islax{A^+ \fuse B^+}}}
{\foc{\Psi}{\Delta_1}{\islax{A^+}}
 &
 \foc{\Psi}{\Delta_2}{\islax{B^+}}}
\quad
\infer-
{\foc{\Psi}{\frameoff{\Theta}{\matchconj{\Delta_A}{x{:}A^+ \lefti B^-}}}{U}}
{\foc{\Psi}{\Delta_A}{\islax{A^+}}
 &
 \foc{\Psi}{\tackon{\Theta}{x'{:}\iseph{B^-}}}{U}}
\]
\[
\infer-
{\foc{\Psi}{\frameoff{\Theta}{x{:}{\gnab}(A^+ \fuse B^+)}}{U}}
{\foc{\Psi}{\tackon{\Theta}{\mkconj{x_1{:}\iseph{{\gnab}A^+}}{x_2{:}\iseph{{\gnab}B^+}}}}{U}}
\quad
\infer-
{\foc{\Psi}{\Delta}{\islax{{\ocircle}(A^+ \lefti B^-)}}}
{\foc{\Psi}{\mkconj{x{:}\iseph{{\gnab}A^+}}{\Delta}}{\islax{{\ocircle}B^-}}}
\]
(To be clear, just as all the unfocused admissibility lemmas only applied
to stable sequents, the unfocused admissibility lemmas above only apply when
contexts and succedents are both stable and free of judgments 
$\istrue{T}$ and ${\isconc{T}}$.)

The point of this exercise is that, given the definition and
metatheory of \ollll, there is a reasonably large family of related
systems, including ordered linear logic, lax logic, linear lax logic, and
linear logic, that can be given erasure-based focalization proofs
relative to \ollll; at most, the erasure function and the unfocused
admissibility lemmas need to be adapted. The fragment we have defined
here corresponds to regular linear logic. In the erasure of polarized
\ollll~propositions to linear logic propositions, the
``pseudo-shifts'' ${\ocircle}$ and ${\gnab}$ are wiped away:
$({\ocircle}A^+)^\circ = (A^+)^\circ$ and $({\gnab}A^-)^\circ =
(A^-)^\circ$.  Additionally, the two implications are conflated: $(A^+
\lefti B^-)^\circ = (A^+ \righti B^-)^\circ = (A^+)^\circ \lolli
(B^+)^\circ$. Beyond that, and the renaming of fuse to tensor -- $(A^+
\fuse B^+)^\circ = (A^+)^\circ \otimes (B^+)^\circ$ -- the structure
of erasure remains intact, and we can 
meaningfully focalize unfocused linear logic derivations into focused 
\ollll~derivations.

\section{The design space of proof terms}
\label{sec:intrinsic-extrinsic}

In the design space of logical frameworks, our decision to view
proof terms $E$ as being fully intrinsically typed representatives
of focused derivations is somewhat unusual. This is because, in 
a dependently typed logical framework, the variable substitution
theorem (which we had to establish very early on) and the cut 
admissibility theorem (which we established much later) are effectively
the same theorem; handling everything at once is difficult at best,
and dependent types seem to force everything to be handled at once in
an intrinsically typed presentation.

Since the advent of Watkins' observations about the existence of
hereditary substitution and its application to logical frameworks
\cite{watkins02concurrent}, the dominant approach to the metatheory of
logical frameworks has to define proof terms $E$ that have little, if
any, implicit type structure: just enough so that
it is possible to define the hereditary substitution function
$\rsubst{M}{x}{E}$. The work by Martens and Crary goes further,
treating hereditary substitution as a relation, not a function, so
that absolutely no intrinsic type system is necessary, and the proof
terms are merely untyped abstract binding trees
\cite{martens12lf}.

If we were to take such an approach, we would need to treat the
judgment $\foct{\Psi}{\Delta}{E}{U}$ as a genuine four-place relation,
rather than the three-place relation $\foc{\Psi}{\Delta}{U}$ annotated
with a derivation $E$ of that sequent.  Then, the analogue of cut
admissibility (part 4) would show that if
$\foct{\Psi}{\Delta}{M}{A^-}$ and
$\foct{\Psi}{\tackon{\Theta}{x{:}\islvl{A^-}}}{E}{U}$, and $\Xi$
matches $\frameoff{\Theta}{\restrictto{\Delta}{\mlvl}}$, then
$\foct{\Psi}{\Xi}{\rsubst{M}{x}{E}}{U}$, where {\it $\rsubst{M}{x}{E}$
  is some function on proof terms that has already been defined},
rather than just an expression of the computational content of the
theorem. Being able to comfortably conflate the computational content
of a theorem with its operation on proof terms is the primary
advantage of the approach taken in this chapter; it avoids a great
deal of duplicated effort. The cost to this approach is that we cannot
apply the modern Canonical LF methodology in which we define a proof
term language that is intrinsically only simply well-typed and then
overlay a dependent type system on top of it (this is discussed in
Section~\ref{sec:lf-simpletypesandhsubst} in the context of LF). As we
discuss further in Section~\ref{sec:why-not-fully-dependent}, this
turns out not to be a severe limitation given the way we want to use
\ollll.

It is not obvious how the substitution 
$\rsubst{M}{x}{E}$ could be defined without accounting for the 
full structure of derivations. The rightist substitution function,
in particular, is computationally dependent
on the implicit bookkeeping associated with the matching constructs, and
that bookkeeping is far more of a difficulty in our setting than the
implicit type annotations. 
The problem, if we wish to see it as a problem, is that we cannot
substitute a derivation $M$ of $\foc{\Psi}{\Delta}{A^-}$
into a derivation $E$ of $\foc{\Psi}{\tackon{\Theta}{x{:}\istrue{A^-}}}{U}$
unless $x$ is {\it actually free} in $E$. Therefore, when we try to
substitute the same $M$ into $\tfuser{V_1}{V_2}$, we are forced to determine
what judgment $x$ is associated with; if $x$ is associated with a
linear or ephemeral judgment, we must track which subderivation 
$x$ is assigned to in order to determine what is to be done next. 

Fine-grained tracking of variables during substitution is both very
inefficient when type theories are implemented as logical frameworks
and unnatural to represent for proof assistants like Twelf that
implement a persistent notion of bound variables.  Therefore other
developments have addressed this problem (see, for example, Cervesato
et al.~\cite{cervesato99explicit}, Schack-Nielsen and
Sch\"urmann~\cite{schacknielsen10curry}, and
Crary~\cite{crary10higher}). It might be possible to bring our
development more in line with these other developments by introducing
a new matching construct of substitution into contexts, the
substitution construct $[\Delta/(x{:}\islvl{A^-})]\Xi$.  If $\Xi =
\tackon{\Theta}{x{:}\islvl{A^-}}$, then this would be the same as
$\frameoff{\Theta}{\Delta}$, but if $x$ is not in the variable domain
of $\Xi$, then $\Xi$ matches $[\Delta/(x{:}\islvl{A^-})]\Xi$.
\[
\infer-[{\it rcut}]
{\foct{\Psi}{[\Delta/(x{:}\islvl{A^-})]\Xi}{E'}{U}}
{\restrictto{\Delta}{\mlvl}
 &
 \foct{\Psi}{\Delta}{M}{A^-}
 &
 \foct{\Psi}{\Xi}{E}{U}
 &
 \stableL{\Delta}
 &
 \rsubst{M}{x}{E} = E'}
\]
Using this formulation of {\it rcut}, it becomes unproblematic to
define $\rsubst{M}{x}{(\tfuser{V_1}{V_2})}$ to be
$\tfuser{(\rsubst{M}{x}{V_1})}{(\rsubst{M}{x}{V_2})}$, as we are
allowed to substitute for $x$ even in terms where the variable cannot
appear. Using this strategy, it should be possible to describe and
formalize the development in this chapter with proof terms that do
nothing more than capture the binding structure of derivations.


The above argument suggests that the framing-off operation is {\it
  inconvenient} to use for specifying the {\it rcut} part of cut
admissibility, because it forces us to track where the variable ends
up and direct the computational content of cut admissibility
accordingly. However, the development in this chapter shows that it is
clearly possible to define cut admissibility in terms of the
framing-off operation $\frameoff{\Theta}{\Delta}$. That is not
necessarily the case for every logic. For instance, to give a focused
presentation of Reed's queue logic \cite{reed09queue}, we would need a
matching construct $[\Delta/x]\Xi$ that is quite different from the
framing-off operation $\frameoff{\Delta}{x{:}A^-}$ used to describe
the logic's left rules.  I conjecture that logics where the
framing-off operation is adequate for the presentation of cut
admissibility are the same as those logics which can be treated in
Belnap's display logic \cite{belnap82display}.



% Substructural logical specifications
\chapter{Substructural logical specifications}
\label{chapter-framework}

In this chapter, we design a logical framework of substructural
logical specifications (\sls), a framework heavily inspired by the
Concurrent Logical Framework (CLF) \cite{watkins02concurrent}. The
framework is justified as a fragment of the logic \ollll~from
Chapter~\ref{chapter-order}. 
There are a number of reasons why we do not just use the
already-specified \ollll~outright as a logical framework.
%
\smallskip
\begin{itemize}
\item{\it Formality.} The specifics of the domain of first-order
  quantification in \ollll~were omitted in Chapter~\ref{chapter-order}, so in
  Section~\ref{sec:sls-termlanguage} we give a careful presentation of
  the term language for \sls, Spine Form LF.

\item{\it Clarity.} The syntax constructions that we presented for
  \ollll~proof terms had a 1-to-1 correspondence with the sequent
  calculus rules; the drawback of this presentation was that large
  proof terms are notationally heavy and difficult to read. The proof
  terms we present for \sls~will leave implicit some of the
  information present in the diacritical marks of \ollll~proof
  terms. 

  An implementation based on these proof terms would need to consider
  type reconstruction and/or bidirectional typechecking to recover the
  omitted information, but we will not consider those issues in this
  thesis.

\item{\it Separating concurrent and deductive reasoning.} Comparing
  CLF to the focused logics from previous chapters leads us to
  conclude that the single most critical design feature of CLF is its
  omission of the proposition ${\uparrow}A^+$. This single
  omission\footnote{In our development, the omission of
    right-permeable propositions $p^-_\mlax$ from \ollll~is equally
    important, but permeable propositions as we have presented them in
    Section~\ref{sec:permeable} were not a relevant consideration in
    the design of CLF.} means that stable sequents in CLF or \sls~are
  effectively restricted to have the succedent $\istrue{\susp{p^-}}$
  or the succedent $\islax{A^+}$.

  Furthermore, any left focus when the succedent is
  $\istrue{\susp{p^-}}$ must conclude with the rule ${\it id}^-$, and
  any left focus when the succedent is $\islax{A^+}$ must conclude
  with ${\ocircle}_L$ -- without the elimination of ${\uparrow}A^+$,
  left focus in both cases could additionally conclude with the rule
  ${\uparrow}_L$. This allows derivations that prove
  $\istrue{\susp{p^-}}$ -- the {\it deductive fragment} of CLF or SLS
  -- to adequately represent deductive systems, conservatively
  extending deductive logical frameworks like LF and LLF. Derivations
  that prove $\islax{A^+}$, on the other hand, fall into the {\it
    concurrent fragment} of CLF and SLS and can encode evolving
  systems.
 
\item{\it Partial proofs.} The design of CLF makes it difficult to
  reason about and manipulate the proof terms corresponding to partial
  evaluations of evolving systems in the concurrent fragment: the proof
  terms in CLF correspond to complete proofs and partial evaluations
  naturally correspond to partial proofs.

  The syntax of \sls~is designed to support the explicit
  representation of partial \ollll~proofs. The omission of the
  propositions $\zero$, $A^+ \oplus B^+$, and the restrictions we
  place on $\lf{t} \doteq_\tau \lf{s}$ are made in the service of
  presenting a convenient and simple syntax for partial proofs. The
  three syntactic objects representing partial proofs, {\it patterns}
  (Section~\ref{sec:framework-patterns}), {\it steps}, and {\it
    traces} (Section~\ref{sec:framework-concurrent}), allow us to
  treat proof terms for evolving systems as first-class members of
  \sls.

  The removal of $\zero$ and $A^+ \oplus B^+$, and the restrictions we
  place on $\lf{t} \doteq_\tau \lf{s}$, also assist in imposing
  equivalence relation, {\it concurrent equality}, on
  \sls~terms. Concurrent equality is a coarser equivalence relation
  than the $\alpha$-equivalence of \ollll~terms.

\item{\it Removal of $\top$.} The presence of $\top$ causes pervasive
  problems in the design of substructural logical frameworks. May of
  these problems arise at the level of implementation and type
  reconstruction, which motivated Schack-Nielson to remove $\top$ from
  the Celf implementation of CLF
  \cite{schacknielsen11implementing}. Even though those considerations
  are outside the scope of this thesis, the presence of $\top$ causes
  other pervasive difficulties: for instance, the presence of $\top$
  complicates the discussion of concurrent equality in CLF. We therefore
  follow Schack-Nielson in removing $\top$ from \sls.

\end{itemize}
\smallskip
\noindent
In summary, with \sls~we {\it simply} the presentation of \ollll~for
convenience and readability, {\it restrict} the propositions of
\ollll~to separate concurrent and deductive reasoning and to make the
syntax for partial proofs feasible, and {\it extend} \ollll~with a
syntax for partial proofs and a coarser equivalence relation.

In Section~\ref{sec:sls-termlanguage} we give a brief presentation of
the term language for \sls, Spine Form LF.
%
In Section~\ref{sec:slsframework} we present \sls~as a fragment of
\ollll, and in Section~\ref{sec:framework-concurrenteq} we discuss
concurrent equality. 
%
In Section~\ref{sec:sls-adequate} we adopt the methodology of adequate
encoding from LF to \sls, in the process introducing {\it generative
  signatures}, which play a critical role in Part 3 of this thesis.
%
In Section~\ref{sec:prototype} we cover the \sls~prototype
implementation, and in Section~\ref{sec:framework-logicprog} we review
some intuitions about logic programming in the framework. 
%
Finally, in Section~\ref{sec:designdecisions}, we discuss some of the
decisions reflected in the design of \sls~and how some decisions could
have been potentially been made differently.

\section{Spine Form LF as a term language}
\label{sec:sls-termlanguage}

Other substructural logical frameworks, like Cervesato and Pfenning's
LLF \cite{cervesato02linear}, Polakow's OLF \cite{polakow01ordered},
and Watkins et al.'s CLF \cite{watkins02concurrent} are {\it
  fully-dependent type theories}: the language of terms (that is, the
domain of first-order quantification) is the same as the language of
proof terms, the representatives of logical derivations (we will call
the domain of quantification the {\it object terms} when ``terms''
would be ambiguous). The logical framework \sls~presented in this
chapter breaks from this tradition -- a choice we discuss further in
Section~\ref{sec:why-not-fully-dependent}. The domain of first-order
quantification, which was left unspecified in Chapter~\ref{chapter-order}, 
will be
presently described as Spine Form LF, a well-understood logical
framework derived from the normal forms of the purely persistent type
theory LF \cite{harper93framework}.

All the information in this section is standard and adapted from
various sources, especially Harper, Honsell, and Plotkin's original
presentation of LF \cite{harper93framework}, Cervesato and Pfenning's
discussion of spine form terms \cite{cervesato02linear}, Watkins et
al.'s presentation of the canonical forms of CLF
\cite{watkins02concurrent}, Nanevski et al.'s dependent contextual
modal type theory \cite{nanevski08contextual}, Harper and Licata's
discussion of Canonical LF \cite{harper07mechanizing}, and Reed's
spine form presentation of HLF \cite{reed09hybrid}.

It would be entirely consistent for us to appropriate Harper and
Licata's Canonical LF presentation instead of presenting Spine Form
LF. Nevertheless, a spine-form presentation of canonical LF serves to
make our presentation more uniform, as spines are used in the proof
term language of \sls. Canonical term languages like Canonical LF
correspond to normal natural deduction presentations of logic, whereas
spine form term languages correspond to focused sequent calculus
presentations like the ones we have considered thus far.

\subsection{Core syntax}

The syntax of Spine Form LF is extended in two places to handle \sls:
rules ${\sf r} : A^-$ in the signature contain negative \sls~types
$A^-$ (it would be possible to separate out the LF portion of
signatures from the \sls~rules), and several new base kinds
are introduced for the sake of \sls~-- ${\sf prop}$, ${\sf prop}\,{\sf
  ord}$, ${\sf prop}\,{\sf lin}$, and ${\sf prop}\,{\sf pers}$.
% We also add four additional kinds, ${\sf prop}$, which classifies
% negative ordered atomic types $p^-$, ${\sf prop}\,{\sf ord}$, which
% classifies positive ordered atomic types $p^+$, ${\sf prop}\,{\sf
%   lin}$, which classifies positive linear/mobile/ephemeral atomic
% types $p^+_\meph$, and ${\sf prop}\,{\sf ord}$, which classifies
% positive persistent atomic types $p^+_\mpers$.  Other than the extra
% kinds classifying atomic \sls~propositions, kinds $\kappa$ are
% otherwise exactly as they are in other presentations of LF; kinds
% classify types $\tau$, and types $\tau$ classify normal terms
% $\lf{t}$ and spines $\lf{\spi}$. Kinds $\kappa$ and types $\tau$ are
% both treated as syntactic refinements of {\it classifiers} $\nu$.
\begin{align*}
& \mbox{Signatures} & \Sigma & ::= \cdot 
  \mid \Sigma, \lf{\sf c} : \tau
  \mid \Sigma, {\sf a} : \kappa
  \mid \Sigma, {\sf r} : A^-
\\
& \mbox{Variables} & \lf{a}, \lf{b} & ::= \ldots
\\
& \mbox{Variable contexts} & \Psi & ::= \cdot
  \mid \Psi, \lf{a} {:} \tau 
\\
& \mbox{Kinds} & \kappa & ::=  \lfpi{a}{\tau}{\kappa} 
  \mid {\sf type}
  \mid {\sf prop}
  \mid {\sf prop}\,{\sf ord}
  \mid {\sf prop}\,{\sf lin}
  \mid {\sf prop}\,{\sf pers}
\\
& \mbox{Types} & \tau & ::= \lfpi{a}{\tau}{\tau'} 
  \mid \lfroot{\sf a}{\spi}
\\
& \mbox{Heads} & \lf{h} & ::= \lf{a} \mid \lf{\sf c}
\\
& \mbox{Normal terms} & \lf{t}, \lf{s} & ::= \lf{\lambda a.t}
  \mid \lf{\lfroot{h}{\spi}}
\\
& \mbox{Spines} & \lf{\spi} & ::= \lf{t; \spi} \mid \lf{\lfnil}
\\
& \mbox{Substitutions} & \lf{\sigma} & ::= \lf{\cdot}
  \mid \lf{t/a, \sigma}
  \mid \lf{b/\!\!/a, \sigma}
\end{align*}
\noindent
Types $\tau$ and kinds $\kappa$ overlap, and will be referred to
generically as {\it classifiers} $\nu$ when it is convenient to do so;
types and kinds can be seen as refinements of classifiers. Another
important refinement are {\it atomic classifiers} $\lfroot{\sf
  a}{\spi}$, which we abbreviate as $p$.

LF spines $\lf\spi$ are just sequences of terms $\lf{(t_1; (\ldots;
  (t_n;())\ldots))}$; we will follow common convention and write
$\lf{h\,t_1\ldots t_n}$ as a convenient shorthand for the atomic term
$\lf{\lfroot{h}{(t_1; \ldots; (t_n;())\ldots)}}$; similarly, we will
write ${\sf a}\,\lf{t_1\ldots t_n}$ as a shorthand for atomic
classifiers $\lfroot{\sf a}{(t_1;
  (\ldots; (t_n;())\ldots))}$. This shorthand is given a formal justification
in \cite{cervesato02linear}; we will use the same shorthand for 
\sls~proof terms in Section~\ref{sec:framework-deductive}.

% This shorthand evokes a canonical-forms
% presentation, as an atomic term, type, or proposition is a head
% $\lf{h}$ or ${\sf a}$ with the terms $\lf{t_1\ldots t_n}$ applied to
% it.

\begin{figure}[t]
\begin{align*}
\fbox{$\subst{\lf{t}}{\lf{\spi}}$}
\\
\subst{(\lf{\lambda a. t'})}{(\lf{t; \spi})}
 & = \subst{\rsubst{\lf{t}}{\lf{a}}{\lf{t'}}}{\lf{\spi}}
\\
\subst{\lfroot{\lf h}{\spi}}{\lfnil}
 & = \lfroot{\lf h}{\spi}
\end{align*}\begin{align*}
\fbox{$\rsubst{\lf{t}}{\lf{a}}{\lf{\spi}}$}&
&
\fbox{$\rsubst{\lf{t}}{\lf{a}}{\lf{t'}}$}&
&
\\
\rsubst{\lf t}{\lf a}{(\lf{t'; \spi})}
 & = \lf{\no{\rsubst{\lf t}{\lf a}{\lf{t'}}}; 
         \no{\rsubst{\lf t}{\lf a}{\lf{\spi}}}} &
\rsubst{\lf t}{\lf a}(\lf{\lambda y. t'})
 & = \lf{\lambda b.\, \no{\rsubst{\lf t}{\lf a}{\lf{t'}}}} 
      & (\lf a \neq \lf b) 
\\
\rsubst{\lf t}{\lf a}{\lfnil} 
 & = \lfnil &
\rsubst{\lf t}{\lf a}{(\lf{\lfroot{a}{\spi}})}
 & = \subst{\lf t}{\rsubst{\lf t}{\lf a}{\lf{\spi}}}
\\
& & 
\rsubst{\lf t}{\lf a}{(\lf{\lfroot{h}{\spi}})}
 & = \lfroot{\lf h}{\no{\rsubst{\lf t}{\lf a}{\lf{\spi}}}}
      & ({\it if}~ \lf{h} \neq \lf{a})
\end{align*}
\caption{Hereditary substitution on terms, spines, and classifiers}
\label{fig:lf-hsubst}
\end{figure}

\subsection{Simple types and hereditary substitution}
\label{sec:lf-simpletypesandhsubst}

In addition to LF types like $\lfpi{a}{(\lfpi{z}{(\lfroot{\sf
      a1}{\spi_1})}{\,(\lfroot{\sf
      a2}{\spi_2})})}{\,\lfpi{y}{(\lfroot{\sf
      a3}{\spi_3})}{\,(\lfroot{\sf a4}{\spi_4})}}$, both Canonical LF
and Spine Form LF take {\it simple types} into consideration. The
simple type corresponding to the type above is $({\sf a1} \simplearrow {\sf
  a2}) \simplearrow {\sf a3} \simplearrow {\sf a4}$, where ${\simplearrow}$
associates to the right. The simple type associated with the 
LF type
$\tau$ is given by the function ${\mid}\tau{\mid}^- = \tau_s$, where
${\mid}\lfroot{\sf a}{\spi}{\mid}^- = {\sf a}$ and
${\mid}\lfpi{a}{\tau}{\tau'}{\mid}^- = {\mid}\tau{\mid}^- \simplearrow
{\mid}\tau'{\mid}^-$. 


\begin{figure}
\begin{align*}
\fbox{$\lf{\sigma}(\lf{\spi})$}&
&
\fbox{$\lf{\sigma}(\lf{t'})$}
\\
\lf{\sigma}(\lf{t'; \spi}) 
 & = \lf{\no{\lf{\sigma}(\lf{t'})}; \no{\lf{\sigma}(\lf{\spi})}} &
\lf{\sigma}(\lf{\lambda a.t'}) 
 & = \lf{\lambda a.\,\no{\lf{(\sigma, a/\!\!/a)}(\lf{t'})}}
 & (\lf{a} \# \lf{\sigma})
\\
\lf{\sigma}\lfnil 
 & = \lfnil &
\lf{\sigma}(\lf{\lfroot{a}{\spi}}) 
 & = \subst{\lf t}{\no{\lf{\sigma}(\lf{\spi})}}
      & \lf{t/a} \in \lf{\sigma} 
\\
& &
\lf{\sigma}(\lf{\lfroot{a}{\spi}}) 
 & = \lf{\lfroot{b}{\no{\lf{\sigma}(\lf{\spi})}}} 
      & \lf{b/\!\!/a} \in \lf{\sigma} 
\\
& &
\lf{\sigma}(\lf{\lfroot{\sf c}{\spi}}) 
 & = \lf{\lfroot{\sf c}{\no{\lf{\sigma}(\lf{\spi})}}} 
\end{align*}\begin{align*}
\fbox{$\lf{\sigma}{\nu}$} &
\\
\lf{\sigma}(\lfpi{b}{\nu}{\nu'})
 & = \lfpi{b}{\lf{\sigma}\nu}{\,\lf{(\sigma, b/\!\!/b)}\nu'}
     \qquad (\lf a \neq \lf b) 
\\
\lf{\sigma}({\sf type})
  & = {\sf type}
\\ 
\lf{\sigma}({\sf prop}) 
 & = {\sf prop} 
\\
\lf{\sigma}({\sf prop}\,{\sf ord}) 
 & = {\sf prop}\,{\sf ord} 
\\
\lf{\sigma}({\sf prop}\,{\sf lin}) 
 & = {\sf prop}\,{\sf lin} 
\\
\lf{\sigma}({\sf prop}\,{\sf pers}) 
 & = {\sf prop}\,{\sf pers} 
\\
\lf{\sigma}(\lfroot{\sf a}{\spi}) 
 & = \lfroot{\sf a}{{\lf{\sigma\spi}}}
\end{align*}
\caption{Simultaneous substitution on terms, spines, and classifiers}
\label{fig:simsubst}
\end{figure}

Variables and constants are treated as having an intrinsic simple
type; these intrinsic simple types are sometimes written explicitly as
annotations $\lf{a}^{\tau_s}$ or $\lf{\sf c}^{\tau_s}$ (see
\cite{pfenning08church} for an example), but we will leave them
implicit.  An atomic term $\lf{h\,t_1\ldots t_n}$ must have a
simple atomic type ${\sf a}$. This means that the head $\lf h$ must
have simple type $\tau_{s1} \simplearrow \ldots \simplearrow \tau_{sn} \simplearrow
{\sf a}$ and each $\lf{t_i}$ much have simple type
$\tau_{si}$. Similarly, a lambda term $\lf{\lambda a. t}$ must have
simple type $\tau_s \simplearrow \tau_s'$ where $\lf a$ is a variable with
simple type $\tau_s$ and $\lf t$ has simple type $\tau_s'$.

Simple types, which are treated in full detail elsewhere
\cite{harper07mechanizing,reed09hybrid}, are critical because they
allow us to define hereditary substitution and hereditary reduction as
total functions in Figure~\ref{fig:lf-hsubst}. Intrinsically-typed
Spine Form LF terms correspond to the proof terms for a focused
presentation of (non-dependent) minimal logic. Hereditary reduction
$\subst{\lf{t}}{\lf{\spi}}$ and hereditary substitution $\rsubst{\lf
  t}{\lf a}{\lf{t'}}$, which are both implicitly indexed by the simple
type $\tau_s$ of $\lf t$, capture the computational content of
structural cut admissibility on these proof terms. Informally, the
action of hereditary substitution is to perform a substitution into a
term and then continue to reduce any $\beta$-redexes that would
introduced by a traditional substitution operation.  Therefore,
$\rsubst{\lf{\lambda x. x}}{\lf f}{\lf{({\sf a}\,(f\,{\sf
      b})\,(f\,{\sf c}))}}$ is not $\lf{{\sf a}\,((\lambda x.x)\,{\sf
    b})\,((\lambda x.x)\,{\sf c})}$ -- that's not even a syntactically
well-formed term according to the grammar for Spine Form LF. Rather,
the result of that substitution is $\lf{{\sf a}\,{\sf b}\,{\sf c}}$.

\subsection{Judgments}

Hereditary substitution is necessary to define simultaneous
substitution into types and terms in Figure~\ref{fig:simsubst}.  We
will treat simultaneous substitutions in a mostly informal way,
relying on the more careful treatment by Nanevski et
al.~\cite{nanevski08contextual}. A substitution takes every variable
in the context and either substitutes a term for it (the form
$\lf{t/a,\sigma}$) or substitutes another variable for it (the form
$\lf{b/\!\!/a,\sigma}$). The latter form is helpful for defining
identity substitutions, which we write as $\lf{\sf id}$ or $\lf{\sf
  id}_\Psi$, as well as generic substitutions $\lf{[t/a]}$ that act
like the identity on all variables except for $\lf{a}$; the latter
notation is used in the definition of LF typing in in
Figure~\ref{fig:lf-form}, which is adapted to Spine Form LF from
Harper and Licata's Canonical LF presentation
\cite{harper07mechanizing}. The judgments $\lf{a}\#\lf{\sigma}$,
$\lf{a}\#\Psi$, $\lf{\sf c}\#\Sigma$, ${\sf a}\#\Sigma$, and ${\sf
  r}\#\Sigma$ assert that the relevant variable or constant does not
already appear in the context $\Psi$ (as a binding $\lf{a}{:}\tau$),
the signature $\Sigma$ (as a declaration $\lf{\sf c} : \tau$, ${\sf a}
: \nu$, or ${\sf r} : A^-$), or the substitution $\lf{\sigma}$ (as a
binding $\lf{t/a}$ or \mbox{$\lf{b/\!\!/a}$}).

\begin{figure}
\fbox{$\vdash_\subord \Sigma\,{\sf sig}$}\vspace{-10pt}
\[
\infer
{\vdash_\subord \cdot\,{\sf sig} \mathstrut}
{}
\quad
\infer
{\vdash_\subord (\Sigma, \lf{\sf c} : \tau)\,{\sf sig} \mathstrut}
{\vdash_\subord \Sigma\,{\sf sig} 
 &
 \cdot \vdash_{\Sigma,\subord} \tau\,{\sf type}
 &
 \tau \prec_\subord \tau
 &
 \lf{\sf c} \# \Sigma \mathstrut}
\]
\[
\infer
{\vdash_\subord (\Sigma, {\sf a} : \kappa)\,{\sf sig} \mathstrut}
{\vdash_\subord \Sigma\,{\sf sig}
 &
 \vdash_{\Sigma, \subord} \kappa \,{\sf kind}
 &
 \kappa  \sqsubset_\subord {\sf a} 
 &
 {\sf a} \# \Sigma\mathstrut}
\quad
\infer
{\vdash_\subord (\Sigma, {\sf r} : A^-)\,{\sf sig} \mathstrut}
{\vdash_\subord \Sigma\,{\sf sig}
 &
 \vdash_{\Sigma, \subord} A^- \,{\sf prop}^-
 &
 {\sf r} \# \Sigma \mathstrut}
\]

\medskip
\fbox{$\vdash_{\Sigma,\subord} \Psi\,{\sf ctx}$} -- presumes
  $\vdash_{\subord} \Sigma\,{\sf sig}$\vspace{-10pt}
\[
\infer
{\vdash_{\Sigma,\subord} \cdot\,{\sf ctx} \mathstrut}
{}
\quad
\infer
{\vdash_{\Sigma,\subord} (\Psi, \lf{a}{:}\tau)\,{\sf ctx} \mathstrut}
{\vdash_{\Sigma,\subord} \Psi\,{\sf ctx}
 &
 \Psi \vdash_{\Sigma, \subord} \tau\,{\sf type}
 &
 \lf a \# \Psi}
\]

\medskip
\fbox{$\Psi \vdash_{\Sigma,\subord} \kappa\,{\sf kind}$} -- presumes
  $\vdash_{\Sigma, \subord} \Psi\,{\sf ctx}$
\[
\infer
{\Psi \vdash_{\Sigma,\subord} (\lfpi{a}{\tau}{\kappa})\,{\sf kind} \mathstrut}
{\Psi \vdash_{\Sigma,\subord} \tau\,{\sf type}
 &
 \Psi, \lf{a}{:}\tau \vdash_{\Sigma,\subord} \kappa\,{\sf kind}}
\quad
\infer{\Psi \vdash_{\Sigma,\subord} {\sf type}\,{\sf kind} \mathstrut}{}
\quad
\infer{\Psi \vdash_{\Sigma,\subord} {\sf prop}\,{\sf kind} \mathstrut}{}
\]
\[
\infer
{\Psi \vdash_{\Sigma,\subord} ({\sf prop}\,{\sf ord})\,{\sf kind}\mathstrut}{}
\quad
\infer
{\Psi \vdash_{\Sigma,\subord} ({\sf prop}\,{\sf lin})\,{\sf kind}\mathstrut}{}
\quad
\infer
{\Psi \vdash_{\Sigma,\subord} ({\sf prop}\,{\sf pers})\,{\sf kind}\mathstrut}{}
\]

\medskip
\fbox{$\Psi \vdash_{\Sigma,\subord} \tau\,{\sf type}$} -- presumes
  $\vdash_{\Sigma, \subord} \Psi\,{\sf ctx}$
\[
\infer
{\Psi \vdash_{\Sigma,\subord}(\lfpi{a}{\tau}{\tau'})\,{\sf type} \mathstrut}
{\Psi \vdash_{\Sigma,\subord} \tau\,{\sf type}
 &
 \Psi, \lf{a}{:}\tau \vdash_{\Sigma,\subord} \tau'\,{\sf type}
 &
 \tau \preceq_\subord \tau' \mathstrut}
\quad
\infer
{\Psi \vdash_{\Sigma,\subord}(\lfroot{\sf a}{\spi})\,{\sf type} \mathstrut}
{a{:}\kappa \in \Sigma
 &
 \Psi, [\kappa] \vdash_{\Sigma,\subord} \lf{\spi} : {\sf type}
 \mathstrut}
\]

\medskip
\fbox{$\Psi \vdash_{\Sigma,\subord} \lf t : \tau$} -- presumes 
  $\Psi \vdash_{\Sigma,\subord} \tau\,{\sf type}$
\[
\infer
{\Psi \vdash_{\Sigma,\subord} \lf{\lambda a.t} : \lfpi{x}{\tau}{\tau'}\mathstrut}
{\Psi, \lf{a}{:}\tau \vdash_{\Sigma,\subord} \lf{t} : \tau'\mathstrut}
\quad
\infer
{\Psi \vdash_{\Sigma,\subord} \lf{\lfroot{\sf c}{\spi}} : p
 \mathstrut}
{\lf{\sf c} : \tau \in {\Sigma}
 &
 \Psi, [\tau] \vdash_{\Sigma,\subord} \lf{\spi} : \tau'
 &
 \tau' = p\mathstrut}
\]
\[
\infer
{\Psi \vdash_{\Sigma,\subord} \lf{\lfroot{a}{\spi}} : p
 \mathstrut}
{\lf{a} {:} \tau \in {\Psi}
 &
 \Psi, [\tau] \vdash_{\Sigma,\subord} \lf{\spi} : \tau'
 &
 \tau' = p}
\]

\medskip
\fbox{$\Psi, [\nu] \vdash_{\Sigma,\subord} \lf{\spi} : \nu_0$} --
presumes that either $\Psi \vdash_{\Sigma,\subord} \nu\, {\sf type}$
or that $\Psi \vdash_{\Sigma,\subord} \nu\, {\sf kind}$
\[
\infer
{\Psi, [\nu] \vdash_{\Sigma,\subord} \lfnil : \nu \mathstrut}
{}
\quad
\infer
{\Psi, [\lfpi{a}{\tau}{\nu}] \vdash_{\Sigma,\subord} \lf{t; \spi} : \nu_0
 \mathstrut}
{\Psi \vdash_{\Sigma,\subord} \lf{t} : \tau
 &
 \lf{[t/a]}\nu = \nu'
 &
 \Psi, [\nu'] \vdash_{\Sigma,\subord} \lf{\spi} : \nu_0 \mathstrut}
\]

\medskip
\fbox{$\Psi \vdash \lf{\sigma} : \Psi'$} -- presumes
 $\vdash_{\Sigma,\subord} \Psi\,{\sf ctx}$
 and
 $\vdash_{\Sigma,\subord} \Psi'\,{\sf ctx}$
\[
\infer
{\Psi \vdash_{\Sigma,\subord} \cdot : \cdot \mathstrut}
{}
\quad
\infer
{\Psi \vdash_{\Sigma,\subord} (\lf{\sigma, t/a}) : \Psi', \lf{a}{:}\tau
  \mathstrut}
{\Psi \vdash_{\Sigma,\subord} \lf{\sigma} : \Psi' 
 &
 \Psi \vdash_{\Sigma,\subord} \lf{t} : \lf{\sigma}\tau 
  \mathstrut}
\quad
\infer
{\Psi \vdash_{\Sigma,\subord} (\lf{\sigma, b/\!\!/a}) : \Psi', \lf{a}{:}\tau
  \mathstrut}
{\Psi \vdash_{\Sigma,\subord} \lf{\sigma} : \Psi'
 &
 \lf{b}{:}\lf{\sigma}\tau \in \Psi
  \mathstrut}
\]

\caption{LF formation judgments ($\tau' = p$ refers to $\alpha$-equivalence)}
\label{fig:lf-form}
\end{figure}

All the judgments in Figure~\ref{fig:lf-form} are indexed by a
transitive {\it subordination relation} $\subord$, similar to the one
introduced by Virga in \cite{virga99higherorder}. The subordination
relation is used to determine a term or variable of type $\tau_1$ can
be a (proper) subterm of a term of type $\tau_2$. Uses of
subordination appear in the definition of well-formed equality
propositions $\lf t \doteq_\tau \lf s$ in
Section~\ref{sec:slsframework}, in the preservation proofs in
Section~\ref{sec:gen-destinations}, and in adequacy arguments (as
discussed in \cite{harper07mechanizing}). We treat $\subord$ as a
binary relation on type family constants.  Let ${\sf head}(\tau) =
{\sf a}$ if $\tau =
\lfpi{a_1}{\tau_{1}}{\,.\,.\lfpi{a_{m}}{\tau_{m}}{\,\lfroot{\sf
      a}{\spi}}}$. The signature formation operations depend on three
judgments. The index subordination judgment, $\kappa \sqsubset_\subord
{\sf a}$, relates type family constants to types.
%
It is always the case that $\kappa =
\lfpi{a_1}{\tau_1}{\ldots\lfpi{a_n}{\tau_n}{\sf type}}$, and the
judgment $\kappa \sqsubset_\subord {\sf a}$ holds if $({\sf
  head}(\tau_i), {\sf a}) \in \subord$ for $1 \leq i \leq n$.
%
The type subordination judgment $\tau \prec_\subord \tau'$ holds if
$({\sf head}(\tau), {\sf head}(\tau')) \in \subord$, and the judgment
$\tau \preceq_\subord \tau'$ is the symmetric extension of this
relation.

In Figure~\ref{fig:lf-form}, we define the judgments $\vdash_\subord
\Sigma\,{\sf sig}$, which takes a context $\Sigma$ and determines
whether it is well-formed.  The premise $\tau \prec_\subord \tau$ is
used in the definition of term constants to enforce that only
self-subordinate types can have constructors. This, conversely, means
that types that are not self-subordinate can only be inhabited by
variables $\lf{a}$, which is important for one of the two types of
equality $\lf{t} \doteq_\tau \lf{s}$ that \sls~supports. The judgments
$\vdash_{\Sigma,\subord} \Psi\,{\sf ctx}$, $\Psi
\vdash_{\Sigma,\subord} \kappa\,{\sf kind}$, and $\Psi
\vdash_{\Sigma,\subord} \tau\,{\sf type}$ similarly take contexts
$\Psi$, kinds $\kappa$, and types $\tau$ and ensure that they are
well-formed in the current signature or (if applicable) context.  The
judgment $\Psi \vdash_{\Sigma,\subord} \lf{t}{:} \tau$ takes a term
and a type and typechecks the term against the type, and the judgment
$\Psi \vdash_{\Sigma,\subord} \lf{\sigma} : \Psi'$ checks that a
substitution $\lf{\sigma}$ can transport objects (terms, types, etc.)
defined in the context $\Psi'$ to objects defined in $\Psi$.

The judgment $\Psi, [\nu] \vdash_{\Sigma,\subord} \lf\spi : \nu_0$ is
read a bit differently than these other judgments. The notation, first
of all, is meant to evoke the (exactly analogous) left-focus judgments
from Chapters~\ref{chapter-foc}~and~\ref{chapter-order}. 
In most other sources (for example, in
\cite{cervesato02linear}) this judgment is instead written as $\Psi
\vdash_{\Sigma,\subord} \lf\spi : \nu > \nu_0$. In either case, we
read this judgment as checking a spine $\lf\spi$ against a classifier
$\nu$ (actually either a type $\tau$ or a kind $\kappa$) and {\it
  synthesizing} a return classifier $\nu_0$. In other words, $\nu_0$
is an output of the judgment $\Psi, [\nu] \vdash_{\Sigma,\subord}
\lf\spi : \nu_0$, and given that this judgment presumes that either
$\Psi \vdash_{\Sigma,\subord} \nu\,{\sf type}$ or $\Psi
\vdash_{\Sigma,\subord} \nu\,{\sf kind}$, it {\it ensures} that either
$\Psi \vdash_{\Sigma,\subord} \nu_0\,{\sf type}$ or $\Psi
\vdash_{\Sigma,\subord} \nu_0\,{\sf kind}$, where the classifiers of
$\nu$ and $\nu_0$ (${\sf type}$ or ${\sf kind}$) always match.  It is
because $\nu_0$ is an output that we add an explicit premise to check
that $\tau' = p$ in the typechecking rule for $\lf{\lfroot{\sf
    c}{\spi}}$; this equality refers to the $\alpha$-equality of Spine
Form LF terms.

There are a number of well-formedness theorems that we need to
consider, such as the fact that substitutions compose in a
well-behaved way and that hereditary substitution is always
well-typed.  However, as these theorems are adequately covered in the
aforementioned literature on LF, we will proceed with using LF as a
term language and will treat term-level operations like substitution
somewhat informally.

We will include annotations for the signature $\Sigma$ and the
subordination relation $\subord$ in the definitions of this section
and the next one. In the following sections and chapters, however, we
will often leave the signature $\Sigma$ implicit when it is
unambiguous or unimportant. We will almost always leave the
subordination relation implicit; we can assume where applicable that
we are working with the {\it strongest} (that is, the smallest)
subordination relation for the given signature
\cite{harper07mechanizing}.

\subsection{Adequacy}
\label{sec:lf-adequacy}

{\it Adequacy} was the name given by Harper, Honsell, and Plotkin to
the methodology of connecting inductive definitions to the canonical
forms of a particular type family in LF. Consider, as a standard
example, the untyped lambda calculus, which is generally specified by
a BNF grammar such as the following:
\[
\obj{e} ::= \obj{x} \mid \obj{\lambda x.e} \mid \obj{e_1\,e_2}
\]
We can adequately encode this language of terms into LF (with a
subordination relation $\subord$ such that $({\sf exp}, {\sf
  exp}) \in \subord$) by giving the following signature:
\begin{align*}
\Sigma & = \cdot, 
\\
 & ~\quad {\sf exp} : {\sf type}, 
\\
 & ~\quad \lf{\sf app} : 
     \lfpi{a}{{\sf exp}}{\,\lfpi{b}{\sf exp}{\,\sf exp}},
\\
 & ~\quad \lf{\sf lam} : 
     \lfpi{a}{(\lfpi{b}{\sf exp}{\,\sf exp})}{\,\sf exp}
\end{align*}
Note that the variables $\lf{a}$ and $\lf{b}$ are bound by
$\Pi$-quantifiers in the declaration of $\lf{\sf app}$ and $\lf{\sf lam}$ but
never used. The usual convention is to abbreviate
$\lfpi{a}{\tau}{\tau'}$ as $\tau \rightarrow \tau'$ when $\lf{a}$ is
not free in $\tau'$, which would give $\lf{\sf app}$ type ${\sf exp}
\rightarrow {\sf exp} \rightarrow {\sf exp}$ and $\lf{\sf lam}$ type
$({\sf exp} \rightarrow {\sf exp}) \rightarrow {\sf exp}$.

\bigskip
\begin{theorem}[Adequacy for terms]\label{thm:expadequacy}
  Up to standard $\alpha$-equivalence, there is a bijection between
  expressions $\obj{e}$ (with free variables in the set
  $\{\obj{x_1},\ldots,\obj{x_n}\}$) and Spine Form LF terms $\lf{t}$ such
  that $\lf{x_1}{:}\mathsf{exp}, \ldots, \lf{x_n}{:}\mathsf{exp} \vdash
  \lf{t} : \mathsf{exp}$. 
\end{theorem}

\begin{proof}
By induction on the structure of the inductive definition of $\obj{e}$
in the forward direction and by induction on the structure of 
terms $\lf{t}$ with type ${\sf exp}$ in the reverse direction.
\end{proof}

We express the constructive content of this theorem as a bijective
function $\interp{e} = \lf{t}$ from object language terms $\obj{e}$ to
representations LF terms $\lf{t}$ of type ${\sf exp}$:
\smallskip
\begin{itemize}
\item $\interp{x} = \lf{x}$, 
\item $\interp{e_1\,e_2} = \lf{{\sf
    app}\,\interp{e_1}\,\interp{e_2}}$, and
\item  $\interp{\lambda x.e} =
\lf{{\sf lam}\,\lambda x.\,\interp{e}}$.
\end{itemize}
\smallskip If we had also defined
substitution $\obj{[e/x]e'}$ on terms, it would be necessary to show
that the bijection is compositional: that is, that
$[\interp{e}/\lf{x}]\interp{e'} = \interp{[e/x]e'}$.  Note that
adequacy critically depends on the context having the form
$\lf{x_1}{:}{\sf exp},\ldots,\lf{x_n}{:}{\sf exp}$. If we had a
context with a variable $\lf{y}{:}({\sf exp} \rightarrow {\sf exp})$,
then we could form a term $\lf{y\,({\sf lam}\,\lambda x. x)}$ with
type ${\sf exp}$ that does {\it not} adequately encode any term
$\obj{e}$ in the untyped lambda calculus.

One of the reasons subordination is important in practice is that it
allows us to consider the adequate encoding of expressions in contexts
$\Psi$ that have other variables $\lf{x}{:}\tau$ as long as $({\sf
  head}(\tau),{\sf exp}) \notin \subord$. If $\Psi,\lf{x}{:}\tau
\vdash_{\Sigma,\subord} \lf{t} : {\sf exp}$ and $\tau
\not\preceq_\subord {\sf exp}$, then $\lf{x}$ cannot be free in
$\lf{t}$, so $\Psi \vdash_{\Sigma,\subord} \lf{t} : {\sf exp}$ holds as
well. By iterating this procedure, it may be possible to strengthen a
context $\Psi$ into one of the form $\lf{x_1}{:}{\sf
  exp},\ldots,\lf{x_n}{:}{\sf exp}$, in which case we can conclude
that $\lf t = \interp{e}$ for some untyped lambda calculus term $\obj
e$.



\section{The logical framework \sls}
\label{sec:slsframework}

In this section, we will describe a restricted set of polarized
\ollll~propositions and focused \ollll~proof terms that make up the
logical framework \sls. For the remainder of the thesis, we will work
exclusively with the following positive and negative
\sls~propositions, which are a syntactic refinement of the positive
and negative propositions of polarized \ollll:
\begin{align*}
A^+, B^+, C^+ & ::= p^+ \mid p^+_\meph \mid p^+_\mpers \mid {\downarrow}A^-
  \mid {\gnab}A^- \mid {!}A^- \mid \one \mid A^+ \fuse B^+
  \mid \exists \lf{a}{:}\tau.A^+ \mid \lf{t} \doteq_\tau \lf{s}
\\
A^-, B^-, C^- & ::= p^- \mid {\ocircle}A^+ \mid A^+ \lefti B^- 
  \mid A^+ \righti B^- \mid A^- \with B^-
  \mid \forall \lf{a}{:}\tau.A^-
\end{align*}
%Aside from the type annotation $\tau$ on unification $\lf{t}
%\doteq_\tau \lf{s}$ and on the quantifiers $\forall \lf{x}{:}\tau. A^-$
%and $\exists \lf{x}{:}\tau. A^+$, which we will in general leave implicit,
%this is exactly a refinement of the  
%Notably missing from this refinement are
%upshifts ${\uparrow}A^+$ and right-permeable atomic propositions
%$p^-_\mlax$.
% We will also continue to
% avoid using variable names $x$ with inverting positive propostiions
% and focused negative propositions. In the case of focused
% propositions, there is a straightforward justification: the form of
% context ensures there is at most one of them. In the case of positive
% propositions, we are justified by the convention discussed in the last
% chapter that we only ever frame off the leftmost positive proposition
% in the context.
We now have to deal with a point of notational dissonance: all
existing work on CLF, all existing implementations of CLF, and the
prototype implementation of \sls~(Section~\ref{sec:prototype}) use the
notation $\{ A^+ \}$ for the connective internalizing the judgment
$\islax{A^+}$, which we have written as ${\ocircle}A^+$, following
Fairtlough and Mendler \cite{fairtlough95propositional}. The
traditional notation overloads curly braces, which we also use for
the context-framing notation 
$\tackon{\Theta}{\Delta}$ introduced in Section~\ref{sec:contexts}. We
will treat ${\ocircle}A^+$ and $\{ A^+ \}$ as synonyms in \sls, preferring the
former in this chapter and the latter afterwards.

Positive ordered atomic propositions $p^+$ are atomic classifiers
${\sf a}\,\lf{t_1}\ldots\lf{t_n}$ with kind ${\sf prop}\,{\sf ord}$,
positive linear and persistent atomic propositions $p^+_\meph$ and
$p^+_\mpers$ are (respectively) atomic classifiers with kind ${\sf
  prop}\,{\sf lin}$ and ${\sf prop}\,{\sf pers}$, and negative ordered
atomic propositions $p^-$ are atomic classifiers with kind ${\sf
  prop}$.  From this point on, we will unambiguously refer to atomic
propositions $p^-$ as negative atomic propositions, omitting
``ordered.'' Similarly, we will refer to atomic propositions $p^+$,
$p^+_\meph$, and $p^+_\mpers$ collectively as positive atomic
propositions but individually as ordered, linear, and persistent
propositions, respectively, omitting ``positive.''  (``Mobile'' and
``ephemeral'' will continue to be used as synonyms for ``linear.'')

\subsection{Propositions}

\begin{figure}
\fbox{$\Psi; \mathcal S \vdash_{\Sigma,\subord} A^+\,{\sf prop}^+$} -- presumes
  $\vdash_{\Sigma,\subord} \Psi\,{\sf ctx}$ and $\mathcal S \subseteq \Psi$
\[
\infer
{\Psi; \mathcal S
   \vdash_{\Sigma,\subord} \lfroot{\sf a}{\spi}\,{\sf prop}^+ \mathstrut}
{{\sf a}{:}\kappa \in \Sigma
 &
 \Psi, [\kappa] \vdash_{\Sigma,\subord} \lf{\spi} : {\sf prop}\,{\sf ord} \mathstrut}
\quad
\infer
{\Psi; \mathcal S
   \vdash_{\Sigma,\subord} \lfroot{\sf a}{\spi}\,{\sf prop}^+ \mathstrut}
{{\sf a}{:}\kappa \in \Sigma
 &
 \Psi, [\kappa] \vdash_{\Sigma,\subord} \lf{\spi} : {\sf prop}\,{\sf lin} \mathstrut}
\]
\[
\infer
{\Psi; \mathcal S
   \vdash_{\Sigma,\subord} \lfroot{\sf a}{\spi}\,{\sf prop}^+ \mathstrut}
{{\sf a}{:}\kappa \in \Sigma
 &
 \Psi, [\kappa] \vdash_{\Sigma,\subord} \lf{\spi} : {\sf prop}\,{\sf pers} \mathstrut}
\]
\[
\infer
{\Psi; \mathcal S \vdash_{\Sigma,\subord} {\downarrow}A^-\,{\sf prop}^+ \mathstrut}
{\Psi; \cdot \vdash_{\Sigma,\subord} A^-\,{\sf prop}^- \mathstrut}
\quad
\infer
{\Psi; \mathcal S \vdash_{\Sigma,\subord} {\gnab}A^-\,{\sf prop}^+ \mathstrut}
{\Psi; \cdot \vdash_{\Sigma,\subord} A^-\,{\sf prop}^- \mathstrut}
\quad
\infer
{\Psi; \mathcal S \vdash_{\Sigma,\subord} {!}A^-\,{\sf prop}^+ \mathstrut}
{\Psi; \cdot \vdash_{\Sigma,\subord} A^-\,{\sf prop}^- \mathstrut}
\quad
\infer
{\Psi; \mathcal S \vdash_{\Sigma,\subord} \one\,{\sf prop}^+ \mathstrut}
{}
\] 
\[
\infer
{\Psi; \mathcal S_1, \mathcal S_2 \vdash_{\Sigma,\subord} A^+ \fuse B^+\,{\sf prop}^+ \mathstrut}
{\Psi; \mathcal S_1 \vdash_{\Sigma,\subord} A^+\,{\sf prop}^+ 
 &
 \Psi; \mathcal S_2 \vdash_{\Sigma,\subord} B^+\,{\sf prop}^+  \mathstrut}
\quad
\infer
{\Psi; \mathcal S \vdash_{\Sigma,\subord} \exists \lf{a}{:}\tau. A^+\,{\sf prop}^+ \mathstrut}
{\Psi \vdash_{\Sigma,\subord} \tau\,{\sf type}
 &
 \Psi, \lf{a}{:}\tau; \mathcal S, \lf{a}{:}\tau \vdash_{\Sigma,\subord} A^+\,{\sf prop}^+ \mathstrut}
\] 
\[
\infer
{\Psi; \mathcal S \vdash_{\Sigma,\subord} \lf{t} \doteq_p \lf{s}\,{\sf prop}^+}
{\Psi \vdash_{\Sigma,\subord} p\,{\sf type}
 &
 \Psi \vdash_{\Sigma,\subord} \lf{t} : p
 &
 \Psi \vdash_{\Sigma,\subord} \lf{s} : p
 & 
 p \not\prec_\subord p}
\]
\[
\infer
{\Psi; \mathcal S \vdash_{\Sigma,\subord} \lf{a} \doteq_p \lf{s}\,{\sf prop}^+}
{\Psi \vdash_{\Sigma,\subord} p\,{\sf type}
 &
 \Psi \vdash_{\Sigma,\subord} \lf{a} : p
 &
 \Psi \vdash_{\Sigma,\subord} \lf{s} : p
 & 
 \lf{a}{:}p \in \mathcal S}
\]


\medskip
\fbox{$\Psi \vdash_{\Sigma,\subord} A^-\,{\sf prop}^-$} -- presumes
  $\vdash_{\Sigma,\subord} \Psi\,{\sf ctx}$ and
  $\mathcal S \subseteq \Psi$
\[
\infer
{\Psi; \mathcal S
   \vdash_{\Sigma,\subord} \lfroot{\sf a}{\spi}\,{\sf prop}^- \mathstrut}
{{\sf a}{:}\kappa \in \Sigma
 &
 \Psi, [\kappa] \vdash_{\Sigma,\subord} \lf{\spi} : {\sf prop} \mathstrut}
\quad
\infer
{\Psi; \mathcal S \vdash_{\Sigma,\subord} {\ocircle}A^+\,{\sf prop}^- \mathstrut}
{\Psi; \cdot \vdash_{\Sigma,\subord} A^+ : {\sf prop}^+ \mathstrut}
\]
\[
\infer
{\Psi; \mathcal S_1, \mathcal S_2 \vdash_{\Sigma,\subord} A^+ \lefti B^-\,{\sf prop}^- \mathstrut}
{\Psi; \mathcal S_1 \vdash_{\Sigma,\subord} A^+\,{\sf prop}^+ 
 &
 \Psi; \mathcal S_2 \vdash_{\Sigma,\subord} B^-\,{\sf prop}^-  \mathstrut}
\quad
\infer
{\Psi; \mathcal S_1, \mathcal S_2 \vdash_{\Sigma,\subord} A^+ \righti B^-\,{\sf prop}^- \mathstrut}
{\Psi; \mathcal S_1 \vdash_{\Sigma,\subord} A^+\,{\sf prop}^+ 
 &
 \Psi; \mathcal S_2 \vdash_{\Sigma,\subord} B^-\,{\sf prop}^-  \mathstrut}
\] 
\[
\infer
{\Psi; \mathcal S \vdash_{\Sigma,\subord} A^- \with B^-\,{\sf prop}^- \mathstrut}
{\Psi; \mathcal S \vdash_{\Sigma,\subord} A^-\,{\sf prop}^- 
 &
 \Psi; \mathcal S \vdash_{\Sigma,\subord} B^-\,{\sf prop}^-  \mathstrut}
\quad
\infer
{\Psi; \mathcal S \vdash_{\Sigma,\subord} \forall \lf{a}{:}\tau. A^-\,{\sf prop}^- \mathstrut}
{\Psi; \mathcal S \vdash_{\Sigma,\subord} \tau\,{\sf type}
 &
 \Psi, \lf{a}{:}\tau; \mathcal S, \lf{a}{:}\tau \vdash_{\Sigma,\subord} A^-\,{\sf prop}^- \mathstrut}
\] 
\caption{\sls~proposition formation judgments}
\label{fig:sls-propform}
\end{figure}

The formation judgments for \sls~types are given in
Figure~\ref{fig:sls-propform}.  As discussed in the introduction to
this chapter, the removal of ${\uparrow}A^+$ and $p^-_\mlax$ is
fundamental to the separation of the deductive and concurrent
fragments of \sls; all the other restrictions made to the language are
for the purpose of giving partial proofs a linear structure.  In
particular, all positive propositions whose left rules have more or
less than one premise are restricted. The propositions $\zero$ and
$A^+ \oplus B^+$ are excluded from \sls~to this end, and we must place
rather draconian restrictions on the use of equality in order to ensure
that $\doteq_L$ can always be treated as having exactly one premise.

The formation rules for propositions are given in
Figure~\ref{fig:sls-propform}. Much of the complexity of this
presentation, such as the existence of an additional context $\mathcal
S$ that is treated like a multiset, is needed to support the inclusion
of equality \sls.  The restrictions ensure that, whenever we decompose
a positive proposition $\lf{s} \doteq \lf{t}$ on the left, we have
that $\lf{s} = \lf{a}$ for some variable $\lf{a}$ in the context. When
this is the case, $\lf{[t/a]}$, is always a most general unifier of
$\lf{s} = \lf{a}$ and $\lf{t}$, which in turn means that the left rule
for equality in \ollll
\[
\infer[{\doteq}_L]
{\foc{\Psi}{\frameoff{\Theta}{\lf{t} \doteq_\tau \lf{s}}}{U}}
{\forall(\Psi' \vdash \lf{\sigma} : \Psi).
 &
 \lf{\sigma t} = \lf{\sigma s}
 &
 \longrightarrow
 &
 \foc{\Psi'}{\tackon{\lf{\sigma}\Theta}{\cdot}}{\lf{\sigma} U}
 }
\]
is equivalent to a much simpler rule:
\[
\infer[{\doteq}_{\it yes}]
{\foc{\Psi, \lf{a}{:}\tau, \Psi'}{\frameoff{\Theta}{\lf{a} \doteq_\tau \lf{t}}}{U}}
{\foc{\Psi, \lf{[t/a]}\Psi'}{\tackon{\lf{[t/a]}\Theta}{\cdot}}{\lf{[t/a]} U}}
\]
Usually, when we require the ``existence of most general unifiers,''
that signals that a most general unifier must exist if any unifier
exists. The condition we are requiring is much stronger: for the
unification problems we will encounter due to the $\doteq_L$ rule, a
most general unifier {\it must} exist. Allowing unification problems
that could fail would require us to consider positive inversion rules
with zero premises, and the proposition $\zero$ was excluded from
\sls~precisely to prevent us from needing to deal with positive
inversion rules with zero premises.\footnote{The other side of this
  observation is that, if we allow the proposition $\zero$ and adapt
  the logical framework accordingly, it might be possible to reduce or
  eliminate the restrictions we have placed on equality.}

There are two distinct conditions under which we can be sure that
unification problems always have a most general solution -- when
equality is performed over {\it pure variables} and when equality is
used as a {\it notational definition}
\cite{pfenning99algorithms}. Equality of pure variable types is
necessary for the destination-adding transformation in
Chapter~\ref{chapter-destinations}, and notational definitions are
used extensively Chapter~\ref{chapter-approx}.

\paragraph{Pure variables} Equality at an atomic type $p$ that is {\it
  not subordinate to itself} ($p \not\prec_\subord p$) is always
allowed.  This is reflected in the first formation rule for $\lf{t}
\doteq \lf{s}$ in Figure~\ref{fig:sls-propform}.

Types that are not self-subordinate can only be inhabited by
variables: that is, if $p \not\prec_\subord p$ and $\Psi
\vdash_{\Sigma,\subord} \lf{t} : p$, then $\lf{t} = \lf{{a}}$ where
$\lf{a}:p \in \Psi$. For any unification problem $\lf{{a}} \doteq
\lf{{b}}$, both $\lf{[{a}/b]}$ and $\lf{[{b}/a]}$ are most general
unifiers.


\paragraph{Notational definitions} Using equality as a notational
definition allows us manipulate propositions in ways that have no
effect on the structure of synthetic inference rules. The subset
$\mathcal S$ of the LF context, which is treated as an a multiset,
enforces that each variable bound by an existential quantifier
$\exists \lf{a}{:}p.\,A^+$ or a universal quantifier $\forall
\lf{a}{:}p.\,A^-$ can be associated with at most one proposition
$\lf{a} \doteq \lf{t}$, where $\lf{t}$ is arbitrary. This condition is
handled by the second formation rule for $\lf{t} \doteq \lf{s}$ in
Figure~\ref{fig:sls-propform}. The rule ${\ocircle}(\exists \lf{a}.\,
\lf{a} \doteq \lf{t} \fuse \lf{a} \doteq \lf{s})$ does not satisfy
this condition because the introduced variable $\lf{a}$ is associated
with the left-hand side of two different equalities, $\lf{a} \doteq
\lf{t}$ and $\lf{a} \doteq \lf{s}$, that together encode an arbitrary
unification problem $\lf{t} \doteq \lf{s}$.


The proposition $\lf{a} \doteq \lf{t}$ must be reachable from its
associated quantifier without crossing a shift or an exponential -- in
Andreoli's terms, it must be in the same {\it monopole}
(Section~\ref{sec:linsynthetic}).  This is enforced by the formation
rules for shifts and exponentials, which clear the subset $\mathcal S$
in their premise.  The proposition $\forall \lf{a}.\,
{\downarrow}({\sf p}\,\lf{a}) \lefti \lf{a} \doteq \lf{t} \lefti {\sf
  p}\,\lf{t}$ satisfies this condition but $\forall \lf{a}.\,\forall
\lf{b}.\,{\ocircle}(\lf{a} \doteq \lf{b})$ does not ($\ocircle$ breaks
focus), and the proposition ${\ocircle}(\exists \lf{a}. \lf{a} \doteq
\lf{t})$ satisfies this condition but ${\ocircle}(\exists \lf{a}.
{\uparrow}(\lf{a} \doteq \lf{t} \lefti {\sf p}\,\lf{a}))$ does not
(${\uparrow}$ breaks focus).

%   This restriction ensures that, if the unification appears on the
%   left, the left-hand side of the unification will always be a
%   variable $\lf{a}$, meaning that $\lf{[t/a]}$ is always a most
%   general unifier.  This usage of unification is essentially just a
%   notational definition \cite{pfenning99algorithms}.


\subsection{Substructural contexts}

\begin{figure}[t]
\fbox{$\Psi \vdash_{\Sigma,\subord} T\,{\sf left} \mathstrut$} -- presumes
  $\vdash_{\Sigma,\subord} \Psi\,{\sf ctx}$
\[
\infer
{\Psi \vdash_{\Sigma,\subord} (\islvl{A^-})\,{\sf left} \mathstrut}
{\Psi; \cdot \vdash_{\Sigma,\subord} A^-\,{\sf prop}^- \mathstrut}
\quad
\infer
{\Psi \vdash_{\Sigma,\subord} 
   (\istrue{\susp{\lfroot{\sf a}{\spi}}})\,{\sf left} \mathstrut}
{{\sf a} : \kappa \in \Sigma
 & 
 \Psi, [\kappa] \vdash_{\Sigma,\subord} \lf{\spi} : {\sf prop}\,{\sf ord}
 \mathstrut}
\]
\[
\quad
\infer
{\Psi \vdash_{\Sigma,\subord} 
   (\iseph{\susp{\lfroot{\sf a}{\spi}}})\,{\sf left} \mathstrut}
{{\sf a} : \kappa \in \Sigma
 & 
 \Psi, [\kappa] \vdash_{\Sigma,\subord} \lf{\spi} : {\sf prop}\,{\sf lin}
 \mathstrut}
\quad
\infer
{\Psi \vdash_{\Sigma,\subord} 
   (\ispers{\susp{\lfroot{\sf a}{\spi}}})\,{\sf left} \mathstrut}
{{\sf a} : \kappa \in \Sigma
 & 
 \Psi, [\kappa] \vdash_{\Sigma,\subord} \lf{\spi} : {\sf prop}\,{\sf pers}
 \mathstrut}
\]

\medskip
\fbox{$\Psi \vdash_{\Sigma,\subord} \Delta\,{\sf stable} \mathstrut$} -- presumes
  $\vdash_{\Sigma,\subord} \Psi\,{\sf ctx}$
\[
\infer
{\Psi \vdash_{\Sigma,\subord} \cdot\,{\sf stable} \mathstrut}
{}
\quad
\infer
{\Psi \vdash_{\Sigma,\subord} (\Delta, x{:}T)\,{\sf stable} \mathstrut}
{\Psi \vdash_{\Sigma,\subord} \Delta\,{\sf stable}
 &
 \Psi \vdash_{\Sigma,\subord} T\,{\sf left}}
\]

\medskip
\fbox{$\Psi \vdash_{\Sigma,\subord} \Delta\,{\sf inv} \mathstrut$} -- presumes
  $\vdash_{\Sigma,\subord} \Psi\,{\sf ctx}$
\[
\infer
{\Psi \vdash_{\Sigma,\subord} \cdot\,{\sf inv} \mathstrut}
{}
\quad
\infer
{\Psi \vdash_{\Sigma,\subord} (\Delta, x{:}T)\,{\sf inv} \mathstrut}
{\Psi \vdash_{\Sigma,\subord} \Delta\,{\sf inv}
 &
 \Psi \vdash_{\Sigma,\subord} T\,{\sf left}}
\quad
\infer
{\Psi \vdash_{\Sigma,\subord} (\Delta, x{:}\istrue{A^+})\,{\sf inv} \mathstrut}
{\Psi \vdash_{\Sigma,\subord} \Delta\,{\sf inv}
 &
 \Psi; \mathcal C \vdash_{\Sigma,\subord} A^+\,{\sf prop}^+}
\]

\medskip
\fbox{$\Psi \vdash_{\Sigma,\subord} \Delta\,{\sf infoc} \mathstrut$} -- presumes
  $\vdash_{\Sigma,\subord} \Psi\,{\sf ctx}$
\[
\infer
{\Psi \vdash_{\Sigma,\subord} (\Delta, x{:}T)\,{\sf infoc} \mathstrut}
{\Psi \vdash_{\Sigma,\subord} \Delta\,{\sf infoc}
 &
 \Psi \vdash_{\Sigma,\subord} T\,{\sf left}}
\quad
\infer
{\Psi \vdash_{\Sigma,\subord} (\Delta, x{:}\istrue{[A^-]})\,{\sf infoc} 
 \mathstrut}
{\Psi \vdash_{\Sigma,\subord} \Delta\,{\sf stable}
 &
 \Psi; \mathcal C \vdash_{\Sigma,\subord} A^-\,{\sf prop}^-
 %&
 %\Psi \vdash_{\Sigma,\subord} \lf{\sigma} : \Psi, \Psi'
 }
\]
\caption{\sls~contexts}
\label{fig:sls-ctxform}
\end{figure}


Figure~\ref{fig:sls-ctxform} describes the well-formed substructural
contexts in \sls. The judgment $\Psi \vdash_{\Sigma,\subord} T\,{\sf
  left}$ is used to check stable bindings $x{:}\islvl{A^-}$ and
$z{:}\islvl{p^+_\mlvl}$ that can appear as a part of stable,
inverting, or focused sequents; the judgment $\Psi
\vdash_{\Sigma,\subord} \Delta\,{\sf stable}$ just maps this judgment
over the context. The judgment $\Psi \vdash_{\Sigma,\subord}
\Delta\,{\sf inv}$ describes contexts during the inversion phase,
which can also contain inverting positive propositions $A^+$. The
judgment $\Psi \vdash_{\Sigma,\subord} \Delta\,{\sf infoc}$ describes
a context that is stable aside from the one negative proposition in
focus. 

The last rule in Figure~\ref{fig:sls-ctxform} is strange because it
appears to make up a substitution $\lf\sigma$ out of thin air.  This
is necessary because the property of being a well-formed proposition
is not stable under substitution. Therefore, even though
$\forall\lf{a}{:}p.\,(\lf{a} \doteq_p \lf{\sf c}) \lefti A^-$ is a
well-formed negative type according to Figure~\ref{fig:sls-propform},
$(\lf{\sf c} \doteq_p \lf{\sf c}) \lefti \lf{[{\sf c}/a]}A^-$ is not.
Therefore, to explain why the intermediate states of a focusing phase
are well-formed, we need to keep track of the substitutions that we
have applied to the negative proposition; the proposition only needs
to be well-formed prior to applying that substitution for the logic to
behave correctly. The same trick will need to be played with the rules
for right-focus in Figure~\ref{fig:sls-values}.

The restrictions we make to contexts justify our continued practice of
omitting the $\mtrue$ annotation when talking about inverting positive
propositions $A^+$ or focused negative propositions $[A^-]$ in the
context, since these context constituents only appear in conjunction
with the $\mtrue$ judgment. 

This discussion of well-formed propositions and contexts takes care of
any issues dealing with variables that were swept under the rug in
Chapter~\ref{chapter-order}.  We could stop here and use the
refinement of \ollll~proof terms that corresponds to our refinement of
propositions as the language of \sls~proof terms. This is not
desirable for two main reasons. First, the proof terms of
focused \ollll~make it inconvenient (though not impossible) to talk
about concurrent equality
(Section~\ref{sec:framework-concurrenteq}). Second, one of our primary
uses of \sls~in this thesis will be to talk about {\it traces}, which
correspond roughly to partial proofs
\[
\deduce
{\foc{\Psi}{\Delta}{\islax{A^+}}\mathstrut}
{\deduce{\vdots\mathstrut}
  {\vspace{-4pt}\foc{\Psi'}{\Delta'}{\islax{A^+}}}}
\]
in \ollll, where both the top and bottom sequents are stable and where
$A^+$ is some unspecified, parametric positive proposition. Using
\ollll-derived proof terms makes it difficult to talk about about and
manipulate proofs of this form.

In the remainder of this section, we will present a proof term
assignment for \sls~that facilitates discussing concurrent equality
and partial proofs. \sls~proof terms are in bijective correspondence
with a refinement of \ollll~proof terms when we consider complete
(deductive) proofs, but the introduction of patterns and traces
reconfigures the structure of derivations and proof terms.

\subsection{Process states}

A {\it process state} is a disembodied left-hand side of a sequent that
we use to describe the intermediate states of concurrent systems. Traces,
which we will type in terms of process states in Section~\ref{sec:framework-concurrent}, are intended to capture the structure of partial proofs:
\[
\deduce
{\foc{\Psi}{\Delta}{\islax{A^+}}\mathstrut}
{\deduce{\vdots\mathstrut}
  {\vspace{-4pt}\foc{\Psi'}{\Delta'}{\islax{A^+}}}}
\]
as a relation between two 
process states. As a first cut, we can represent the initial state as 
$(\Psi; \Delta)$ and the final state as $(\Psi'; \Delta')$, and we can
omit $\Psi$ and just write $\Delta$ when that is sufficiently clear.

Representing a process state as merely an LF context $\Psi$ and a
substructural context $\Delta$ is insufficient because of the way
equality -- pure variable equality in particular -- can unify distinct
variables. Consider the following partial proof:
\[
\deduce
{\foc{\lf{a}{:}p, \lf{b}{:}p}
  {~~x{:}\iseph{{\ocircle}(\lf{a} \doteq_\tau \lf{b})}, ~~
   z{:}\iseph{\susp{{\sf foo}\,\lf{a}\,\lf{a}}}}
  {\islax{({\sf foo}\,\lf{a}\,\lf{b})}}\mathstrut}
{\deduce{\vdots\mathstrut}
  {\vspace{-4pt}\foc{\lf{b}{:}p}
   {~~ z{:}\iseph{\susp{{\sf foo}\,\lf{b}\,\lf{b}}}}
   {\islax{({\sf foo}\,\lf{b}\,\lf{b})}}}}
\]
This partial proof can be constructed in one focusing stage by a left 
focus on $x$. It is insufficient to capture the first process
state as 
$\left({\lf{a}{:}p, \lf{b}{:}p}; ~~
 {x{:}{\ocircle}(\lf{a} \doteq_\tau \lf{b}), ~~
  z{:}\iseph{\susp{{\sf foo}\,\lf{a}\,\lf{a}}}}\right)$
and the second process state as
$\left({\lf{b}{:}p};~~
 {z{:}\iseph{\susp{{\sf foo}\,\lf{b}\,\lf{b}}}}\right)$, as this would fail to 
capture that the succedent $\islax{({\sf foo}\,\lf{b}\,\lf{b})}$
is a substitution instance of
the succedent
$\islax{({\sf foo}\,\lf{a}\,\lf{b})}$. 
%
In general, if the derivation above proved some arbitrary
succedent $\islax{A^+}$ instead of $\islax{({\sf
    foo}\,\lf{a}\,\lf{b})}$, then the missing subproof has the succedent
$\islax{\lf{[b/a]}A^+}$.

A process state is therefore written as $(\Psi; \Delta)_{\lf{\sigma}}$
and is well-formed under signature $\Sigma$ and subordination relation
$\subord$ if $\Psi \vdash_{\Sigma,\subord} \Delta\,{\sf inv}$
(which presumes that $\vdash_{\Sigma,\subord} \Psi\,{\sf ctx}$, as
defined in Figure~\ref{fig:lf-form}) and if $\Psi \vdash \lf{\sigma} :
\Psi_0$, where $\Psi_0$ is some other context that represents the
starting point, the context in which the disconnected succedent
$\islax{A^+}$ is well-formed.
\[
\infer
{\vdash_{\Sigma,\subord} (\Psi; \Delta)_{\lf{\sigma}}\,{\sf state}
 \mathstrut}
{%\vdash_{\Sigma,\subord} \Psi : {\sf ctx}
 %&
 \Psi \vdash_{\Sigma,\subord} \Delta\,{\sf inv}
 &
 \vdash_{\Sigma,\subord} \Psi_0 : {\sf ctx}
 &
 \Psi \vdash \lf{\sigma} : \Psi_0
 \mathstrut}
\]
Taking $\Psi_0 = \lf{a}{:}p, \lf{b}{:}p$, the partial proof above can
thus be represented as a step (Section~\ref{sec:framework-concurrent})
between these two process states:
\[
\left({\lf{a}{:}p, \lf{b}{:}p}; ~~
 {x{:}{\ocircle}(\lf{a} \doteq_\tau \lf{b}),  ~~
  z{:}\iseph{\susp{{\sf foo}\,\lf{a}\,\lf{a}}}}\right)_{\lf{(a/a,\,b/b)}}
\leadsto_{\Sigma,\subord}
\left({\lf{b}{:}p}; ~~
 {z{:}\iseph{\susp{{\sf foo}\,\lf{b}\,\lf{b}}}}\right)_{\lf{(b/a,\,b/b)}}
\]

Substitutions are just one of several ways that we could cope with
free variables in succedents; another option, discussed in
Section~\ref{sec:constraint-store-solution}, is to track the set of
constraints $\lf{a} = \lf{b}$ that have been encountered by
unification.  When we consider traces in isolation, we will generally
let $\Psi_0 = \cdot$ and $\lf{\sigma} = \lf \cdot$, which corresponds
to the case where the parametric conclusion $A^+$ is a closed
proposition. When the substitution is not mentioned, it can therefore
be presumed to be empty. Additionally, when the LF context $\Psi$ is
empty or clear from the context, we will omit it as well. One further
simplification is that we will occasionally omit the judgment $\mlvl$
associated with a suspended positive atomic proposition
$\islvl{\susp{p^+_\mlvl}}$, but only when it is unambiguous from the
current signature that $p^+_\mlvl$ is an ordered, linear, or
persistent positive atomic proposition. In the examples above, we
tacitly assumed that ${\sf foo}$ was given kind $p \rightarrow p
\rightarrow {\sf prop}\,{\sf lin}$ in the signature $\Sigma$ when we
tagged the suspended atomic propositions with the judgment $\meph$. If
it had been clear that ${\sf foo}$ was linear, then this judgment
could have been omitted.

%In almost all cases, the substitution $\lf\sigma$ associated with a process
%state is the identity and will be omitted, and the LF context $\Psi$
%will frequently be omitted as well. 

\begin{figure}
\fbox{$P :: (\Psi; \Delta)_{\lf{\sigma}}
         \Longrightarrow_{\Sigma,\subord}
           (\Psi'; \Delta')_{\lf{\sigma'}}$}
 -- presumes
 $\vdash_{\Sigma,\subord} (\Psi; \Delta)_{\lf{\sigma}}\,{\sf state}$
\[
\infer[()]
{() :: (\Psi; \Delta)_{\lf{\sigma}}
           \Longrightarrow_{\Sigma,\subord} 
       (\Psi; \Delta)_{\lf{\sigma}}}
{\Psi \vdash_{\Sigma,\subord} \Delta\,{\sf stable}}
\]
\[
\infer[\eta^+]
{z,P :: (\Psi; \frameoff{\Theta}{p^+_\mlvl})_{\lf{\sigma}}
           \Longrightarrow_{\Sigma,\subord} 
        (\Psi'; \Delta')_{\lf{\sigma'}}}
{P :: (\Psi; \tackon{\Theta}{z{:}\islvl{\susp{p^+_\mlvl}}})_{\lf{\sigma}}
           \Longrightarrow_{\Sigma,\subord}
        (\Psi'; \Delta')_{\lf{\sigma'}}}
\quad
\infer[{\downarrow}_L]
{x,P :: (\Psi; \frameoff{\Theta}{{\downarrow}A^-})_{\lf{\sigma}}
           \Longrightarrow_{\Sigma,\subord}
        (\Psi'; \Delta')_{\lf{\sigma'}}}
{P :: (\Psi; \tackon{\Theta}{x{:}\istrue{A^-}})_{\lf{\sigma}}
           \Longrightarrow_{\Sigma,\subord}
        (\Psi'; \Delta')_{\lf{\sigma'}}}
\]
\[
\infer[{\gnab}_L]
{x,P :: (\Psi; \frameoff{\Theta}{{\gnab}A^-})_{\lf{\sigma}}
           \Longrightarrow_{\Sigma,\subord}
        (\Psi'; \Delta')_{\lf{\sigma'}}}
{P :: (\Psi; \tackon{\Theta}{x{:}\iseph{A^-}})_{\lf{\sigma}}
           \Longrightarrow_{\Sigma,\subord}
        (\Psi'; \Delta')_{\lf{\sigma'}}}
\quad
\infer[{\bang}_L]
{x,P :: (\Psi; \frameoff{\Theta}{{!}A^-})_{\lf{\sigma}}
           \Longrightarrow_{\Sigma,\subord}
        (\Psi'; \Delta')_{\lf{\sigma'}}}
{P :: (\Psi; \tackon{\Theta}{x{:}\ispers{A^-}})_{\lf{\sigma}}
           \Longrightarrow_{\Sigma,\subord}
        (\Psi'; \Delta')_{\lf{\sigma'}}}
\]
\[
\infer[{\one}_L]
{P :: (\Psi; \frameoff{\Theta}{\one})_{\lf{\sigma}}
           \Longrightarrow_{\Sigma,\subord}
        (\Psi'; \Delta')_{\lf{\sigma'}}}
{P :: (\Psi; \tackon{\Theta}{\cdot})_{\lf{\sigma}}
           \Longrightarrow_{\Sigma,\subord}
        (\Psi'; \Delta')_{\lf{\sigma'}}}
\quad 
\infer[{\fuse}_L]
{P :: (\Psi; \frameoff{\Theta}{A^+ \fuse B^+})_{\lf{\sigma}}
           \Longrightarrow_{\Sigma,\subord}
        (\Psi'; \Delta')_{\lf{\sigma'}}}
{P :: (\Psi; \tackon{\Theta}{\mkconj{A^+}{B^+}})_{\lf{\sigma}}
           \Longrightarrow_{\Sigma,\subord}
        (\Psi'; \Delta')_{\lf{\sigma'}}}
\]
\[
\infer[{\exists}_L]
{\lf{a},P :: (\Psi; \frameoff{\Theta}{\exists \lf{a}{:}\tau. A^+})_{\lf{\sigma}}
           \Longrightarrow_{\Sigma,\subord}
        (\Psi'; \Delta')_{\lf{\sigma'}}}
{P :: (\Psi, \lf{a}{:}\tau; \tackon{\Theta}{A^+})_{\lf{\sigma}}
           \Longrightarrow_{\Sigma,\subord}
        (\Psi'; \Delta')_{\lf{\sigma'}}}
\]
\[
\infer[{\doteq}_L]
{\lf{t/a},P :: 
  (\Psi, \lf{a}{:}\tau, \Psi'; \frameoff{\Theta}{\lf{a} \doteq_\tau \lf{t}}
   )_{\lf{\sigma}}
           \Longrightarrow_{\Sigma,\subord}
        (\Psi'; \Delta')_{\lf{\sigma'}}}
{P :: (\Psi, \lf{[t/a]}\Psi'; {\tackon{\lf{[t/a]}\Theta}{\cdot}}
       )_{\lf{[t/a]\sigma}}
           \Longrightarrow_{\Sigma,\subord}
        (\Psi'; \Delta')_{\lf{\sigma'}}}
\]
\caption{\sls~patterns}
\label{fig:sls-patterns}
\end{figure}


\subsection{Patterns}
\label{sec:framework-patterns}

 % With the restrictions we have made to unificaiton, we can 
% treat $\doteq_L$ as always having one premise, so all remaining
% left rules for positive atomic propositions have exactly one 
% premise. 

%
A {\it pattern} is a syntactic entity that captures the linear
structure of left inversion on positive propositions. The \ollll~proof
term for the proposition
%
$(\exists \lf{a}.\,{\sf p}\,\lf{a} 
             \fuse {\gnab}A^-
             \fuse {\downarrow}B^-) \lefti C^-$,
%
is somewhat inscrutable:
${\tlaml{\texistsl{\lf{a}}
    {\tfusel{\tfusel{\tetap{x}{\tgnabl{y}{\tdownl{z}{N}}}}}}}}$. The
\sls~proof of this proposition, which uses patterns, is
$(\lambda \lf{a},x,y,z.\, N)$. The pattern $P = \lf{a}, x,y,z$ captures
the structure of left inversion on the positive proposition 
$\exists \lf{a}.\,{\sf p}\,\lf{a} 
             \fuse {\gnab}A^-
             \fuse {\downarrow}B^-$.

The grammar of patterns is straightforward.
% 
Inversion on positive propositions
can only have the effect of introducing new bindings (either LF
variables $\lf{a}$ or \sls~variables $x$) or handling a unification
$\lf{a} \doteq_p \lf{t}$, which by our discussion above can always be
resolved by the most general unifier $\lf{[t/a]}$, so the pattern associated
with a proposition $\lf{a} \doteq_p \lf{t}$ is $\lf{t/a}$. 
\[
P ::= () \mid x, P \mid \lf{a}, P \mid \lf{t/a}, P
\] 
For sequences with one or more elements, we omit the trailing
comma and $()$, writing $x, \ldots, z$ 
instead of $x, \ldots, z, ()$. 

\sls~patterns have a linear structure (the comma is right associative)
because they capture the linear structure of proofs. The associated
decomposition judgment which takes two process states: $P :: (\Psi;
\Delta)_{\lf{\sigma}} \Longrightarrow_{\Sigma,\subord} (\Psi';
\Delta')_{\lf{\sigma'}}$, and operates a bit like the spine typing
judgment from Figure~\ref{fig:lf-form} in that the process state
$(\Psi; \Delta)_{\lf{\sigma}}$ (and the pattern $P$) are treated as an
input and the process state $(\Psi'; \Delta')_{\lf{\sigma'}}$ is
treated as an output.  The typing rules for \sls~patterns are given in
Figure~\ref{fig:sls-patterns}. We preserve the side conditions from
the previous chapter: when we frame off a inverting positive
proposition in the process state, it is required to be the left-most
one. As in focused \ollll, this justifies our omission of the
variables associated with positive propositions: the positive
proposition we frame off is always uniquely identified not by its
associated variable but by its position in the context.

Note that there no longer appears to be a one-to-one
correspondence between proof terms and rules: ${\downarrow}_L$,
${\gnab}_L$, and ${!}_L$ appear to have the same proof term, and
${\one}_L$ and ${\fuse}_L$ appear to have no proof term at all. To
view patterns as being intrinsically typed -- that is, to view them as
actual representatives of (incomplete) derivations -- we must think of
patterns as carrying extra annotations that allow them to continue
matching the structure of proof rules.

\subsection{Values, terms, and spines}
\label{sec:framework-deductive}

Notably missing from the \sls~types are the upshifts ${\uparrow}A^+$
and right-permeable negative atomic propositions $p^-_\mlax$. The
removal of these two propositions effectively means that the succedent
of a stable \sls~sequent can only be $\istrue{\susp{p^-}}$ or
$\islax{A^+}$. The \sls~framework only considers {\it complete} proofs
of judgments $\istrue{\susp{p^-}}$, whereas traces, associated with
proofs of $\islax{A^+}$ and introduced below in
Section~\ref{sec:framework-concurrent}, are a proof term assignment
for partial proofs. Excepting the proof term $\tlet{T}{V}$, which we
present as part of the {\it concurrent} fragment of \sls~in
Section~\ref{sec:framework-concurrent} below, the values, terms, and
spines that stand for complete proofs will be referred to as the {\it
  deductive fragment} of \sls.
\begin{align*}
&\mbox{\sls~values (Figure~\ref{fig:sls-values})}& 
V & ::= z
   \mid N
   \mid \tgnabr{N}
   \mid \tbangr{N}
   \mid \toner
   \mid \tfuser{V_1}{V_2}
   \mid \texistsr{\lf{t}}{V}
   \mid \tunifr
\\
&\mbox{\sls~atomic terms (Figure~\ref{fig:sls-atomic-terms})}&
R & ::= \tfocusl{x}{\Sp} 
   \mid \tfocusl{\sf r}{\Sp} 
\\
&\mbox{\sls~terms (Figure~\ref{fig:sls-terms})}&
N & ::= R
   \mid \lambda P.N 
   \mid N_1 \with N_2
   \mid \tforallr{\lf{a}}{N}
   \mid \tlet{T}{V}
\\
&\mbox{\sls~spines (Figure~\ref{fig:sls-spines})}&
\Sp & ::= \tnil 
   \mid V; \Sp
   \mid \pi_1; \Sp 
   \mid \pi_2; \Sp
   \mid \lf{t}; \Sp
\end{align*}

In contrast to \ollll, we distinguish the syntactic category $R$ of
atomic terms that correspond to stable sequents. As with patterns, we
appear to conflate the proof terms associated with different proof
rules -- we have a single $\lambda P.N$ constructor and a single
$V;\Sp$ spine rather than one term $\tlamr{N}$ and spine
$\tappr{V}{\Sp}$ associated with propositions $A^+ \righti B^-$ and
another term $\tlaml{N}$ and spine $\tappl{V}{\Sp}$ associated with
propositions $A^+ \lefti B^-$.  As with patterns, it is possible to
think of these terms as just having extra annotations ($\lambda^>$ or
$\lambda^<$) that we have omitted.  Without these annotations, proof
terms carry less information than derivations, and the rules for
values, terms, and spines in
Figures~\ref{fig:sls-values}--\ref{fig:sls-spines} must be seen as
typing rules. With these extra implicit annotations (or, possibly,
with some of the technology of bidirectional typechecking), values,
terms, and spines can continue to be seen as representatives of
derivations.

\begin{figure}
\fbox{$\foctx{\Psi}{\Delta}{V}{[A^+]}$} -- presumes
  $\Psi \vdash_{\Sigma,\subord} \Delta\,{\sf stable}$, 
  and  $\Psi; \mathcal C \vdash_{\Sigma,\subord} A^+\,{\sf prop}^+$
\[
\infer[{\it id}^+]
{\foctx{\Psi}{\Delta}{z}{[A^+]}}
{\Delta \mbox{\it ~matches~} z{:}\susp{A^+}}
\]
\[
\infer[{\downarrow}_R]
{\foctx{\Psi}{\Delta}{\tdownr{N}}{[{\downarrow}A^-]}}
{\foctx{\Psi}{\Delta}{N}{A^-}}
\quad
\infer[{\gnab}_R]
{\foctx{\Psi}{\restrictto{\Delta}{\meph}}{\tgnabr{N}}{[{\gnab}A^-]}}
{\foctx{\Psi}{\Delta}{N}{A^-}}
\quad
\infer[{!}_R]
{\foctx{\Psi}{\restrictto{\Delta}{\mpers}}{\tbangr{N}}{[{!}A^-]}}
{\foctx{\Psi}{\Delta}{N}{A^-}}
\]
\[
\infer[{\one}_R]
{\foctx{\Psi}{\Delta}{\toner}{[\one]}}
{\Delta \mbox{\it ~matches~} \cdot}
\quad
\infer[{\fuse}_R]
{\foctx{\Psi}
  {\matchconj{\Delta_1}{\Delta_2}}{\tfuser{V_1}{V_2}}{[A^+_1 \fuse A^+_2]}}
{\foctx{\Psi}{\Delta_1}{V_1}{[A^+_1]}
 &
 \foctx{\Psi}{\Delta_2}{V_2}{[A^+_2]}}
\]
\[
\infer[{\exists}_R]
{\foctx{\Psi}{\Delta}{\texistsr{\lf{t}}{V}}{[\exists \lf{a}{:}\tau. A^+]}}
{\Psi; \Delta \vdash_{\Sigma,\subord} \lf{t} : \tau
 &
 \foctx{\Psi}{\Delta}{V}{[\lf{[t/a]}A^+]}}
\quad
\infer[{\doteq}_R]
{\foctx{\Psi}{\cdot}{\tunifr}{\lf{t} \doteq \lf{t}}}
{\Delta \mbox{\it ~matches~} \cdot}
\]
\caption{\sls~values}
\label{fig:sls-values}
\end{figure}

\begin{figure}
\fbox{$\foctx{\Psi}{\Delta}{R}{U}$} -- presumes
  $\Psi \vdash_{\Sigma,\subord} \Delta\,{\sf stable}$ and
  $U = \istrue{\susp{C^-}}$
\[
\infer[{\it focus}_L]
{\foctx{\Psi}{\frameoff{\Theta}{x{:}A^-}}{\tfocusl{x}{\Sp}}{U}}
{\foctx{\Psi}{\tackon{\Theta}{[A^-]}}{\Sp}{U}}
\quad
\infer[{\it rule}]
{\foctx{\Psi}{\frameoff{\Theta}{\cdot}}{\tfocusl{\sf r}{\Sp}}{U}}
{{\sf r} : A^- \in \Sigma 
 &
 \foctx{\Psi}{\tackon{\Theta}{[A^-]}}{\Sp}{U}}
\]
\caption{\sls~atomic terms}
\label{fig:sls-atomic-terms}
\end{figure}

\begin{figure}
\fbox{$\foctx{\Psi}{\Delta}{N}{\istrue{A^-}}$} -- presumes
  $\Psi \vdash_{\Sigma,\subord} \Delta\,{\sf stable}$ and
  $\Psi; \mathcal C \vdash_{\Sigma,\subord} A^-\,{\sf prop}^-$
\[
\infer[\eta^-]
{\foctx{\Psi}{\Delta}{R}{\istrue{p^-}}}
{\foctx{\Psi}{\Delta}{R}{\istrue{\susp{p^-}}}}
\]
\[
\infer[{\lefti}_R]
{\foctx{\Psi}{\Delta}{\lambda P.N}{\istrue{A^+ \lefti B^-}}}
{P :: (\Psi; \mkconj{A^+}{\Delta})_{\lf{\sf id}_{\Psi}} 
  \Longrightarrow_{\Sigma,\subord}
 (\Psi';\Delta')_{\lf{\sigma}}
 &
 \foctx{\Psi'}{\Delta'}{N}{\istrue{\lf{\sigma}B^-}}}
\]
\[
\infer[{\righti}_R]
{\foctx{\Psi}{\Delta}{\lambda P.N}{\istrue{A^+ \righti B^-}}}
{P :: (\Psi; \mkconj{\Delta}{A^+})_{\lf{\sf id}_{\Psi}} 
  \Longrightarrow_{\Sigma,\subord}
 (\Psi';\Delta')_{\lf{\sigma}}
 &
 \foctx{\Psi'}{\Delta'}{N}{\istrue{\lf{\sigma}B^-}}}
\]
\[
\infer[{\with}_R]
{\foctx{\Psi}{\Delta}{\twithr{N_1}{N_2}}{\istrue{A_1^- \with A_2^-}}}
{\foctx{\Psi}{\Delta}{N_1}{\istrue{A_1^-}}
 &
 \foctx{\Psi}{\Delta}{N_2}{\istrue{A_2^-}}}
\]
\[
\infer[{\forall}_R]
{\foctx{\Psi}{\Delta}{\tforallr{\lf{a}}{N}}
    {\istrue{\forall \lf{a}{:}\tau. A^-}}}
{\foctx{\Psi, \lf{a}{:}\tau}{\Delta}{N}{\istrue{A^-}}}
\]
\[
\infer[{\ocircle}_R]
{\foctx{\Psi}{\Delta}{\tlet{T}{V}}{\istrue{{\ocircle}A^+}}}
{T :: (\Psi; \Delta)_{\lf{\sf id}_{\Psi}}
  \leadsto^*_{\Sigma,\subord}
 (\Psi';\Delta')_{\lf{\sigma'}}
 &
 \foctx{\Psi'}{\Delta'}{V}{[\lf{\sigma}A^+]}}
\]
\caption{\sls~terms}
\label{fig:sls-terms}
\end{figure}

%Spines
\begin{figure}
\fbox{$\foctx{\Psi}{\Delta}{\Sp}{U}$} --
  presumes
  $\Psi \vdash_{\Sigma,\subord} \Delta\,{\sf infoc}$
\[
\infer[{\it id}^-]
{\foctx{\Psi}{{\Delta}}{\tnil}{\istrue{\susp{A^-}}}}
{\Delta \mbox{\it ~matches~} [A^-] \mathstrut}
\]
\[
\infer[{\lefti}_L]
{\foctx{\Psi}{\frameoff{\Theta}{\matchconj{\Delta}{[A^+ \lefti B^-]}}}
  {V; \Sp}{U}}
{\foctx{\Psi}{\Delta}{V}{[A^+]}
 &
 \foctx{\Psi}{\tackon{\Theta}{[B^-]}}{\Sp}{U}}
\]
\[
\infer[{\righti}_L]
{\foctx{\Psi}{\frameoff{\Theta}{\matchconj{[A^+ \righti B^-]}{\Delta}}}
  {V; \Sp}{U}}
{\foctx{\Psi}{\Delta}{V}{[A^+]}
 &
 \foctx{\Psi}{\tackon{\Theta}{[B^-]}}{\Sp}{U}}
\]
\[
\infer[{\with}_{L1}]
{\foctx{\Psi}{\frameoff{\Theta}{[A_1^- \with A_2^-]}}
  {\pi_1; \Sp}{U}}
{\foctx{\Psi}{\tackon{\Theta}{[A_1^-]}}{\Sp}{U}}
\quad
\infer[{\with}_{L2}]
{\foctx{\Psi}{\frameoff{\Theta}{[A_1^- \with A_2^-]}}
  {\pi_2; \Sp}{U}}
{\foctx{\Psi}{\tackon{\Theta}{[A_2^-]}}{\Sp}{U}}
\]
\[
\infer[{\forall}_L]
{\foctx{\Psi}{\frameoff{\Theta}{[\forall \lf{a}{:}\tau. B^-]}}
  {\lf{t}; \Sp}{U}}
{\Psi \vdash_{\Sigma,\subord} \lf{t} : \tau
 &
 \lf{[t/a]}B^- = B'^-
 &
 \foctx{\Psi}{\tackon{\Theta}{[B'^-]}}{\Sp}{U}}
\]
\caption{\sls~spines}
\label{fig:sls-spines}
\end{figure}


Aside from ${\ocircle}_R$ and it's associated term $\tlet{T}{V}$,
which belong to the concurrent fragment of \sls, there is one rules in
Figures~\ref{fig:sls-values}--\ref{fig:sls-spines} that does not have
an exact analogues as rules in \ollll, the rule ${\it rule}$ in
Figure~\ref{fig:sls-atomic-terms}. This rule corresponds to an atomic
term $\tfocusl{\sf r}{\Sp}$ and accounts for the fact that there is an
additional source of persistent facts in \sls, the signature $\Sigma$,
that is not present in \ollll.  To preserve the bijective
correspondence between \ollll~and \sls~proof terms, we need to place
every rule ${\sf r} : A^-$ in the \sls~signature $\Sigma$ into the
corresponding \ollll~context as a persistent proposition.

As with LF terms, we will use a shorthand for atomic terms
$\tfocusl{x}{\Sp}$ and $\tfocusl{\sf r}{\Sp}$, writing $({\sf
  foo}\,\lf{t}\,\lf{s}\,V\,V')$ instead of $\tfocusl{\sf
  foo}{(\lf{t};\lf{s};V;V';\tnil)}$ when we are not concerned with the
fact that atomic terms consist of a variable and a spine.

\subsection{Steps and traces}
\label{sec:framework-concurrent}

The deductive fragment of \sls~presented in
Figures~\ref{fig:sls-patterns}--\ref{fig:sls-spines} covers every
\sls~proposition except for the lax modality ${\ocircle}A^+$. It is in
the context of the lax modality that we will present proof terms
corresponding to partial proofs; we call this fragment the {\it
  concurrent} fragment of \sls~because of its relationship with
concurrent equality, described in
Section~\ref{sec:framework-concurrenteq}.
\begin{align*}
\qquad\qquad\qquad\qquad
&\mbox{Steps}&
S & ::= \trstep{P}{R}
\qquad\qquad\qquad\qquad
\\
&\mbox{Traces}&
T & ::= \emptytrace \mid T_1; T_2 \mid S
\end{align*}

A step $S = \tstep{P}{x}{\Sp}$ corresponds precisely to the notion of
a {\it synthetic inference rule} as discussed in
Section~\ref{sec:linsynthetic}. A
step in \sls~corresponds to a use of left focus, a use of the
left rule for the lax modality, and a use of the 
admissible focal substitution lemma in \ollll:
\[
\infer-[{\it subst}^-]
{\foc{\Psi}{\frameoff{\Theta}{\tackon{\Theta'}{x{:}\istrue{A^-}}}}{\islax{\lf{\sigma}A^+}}}
{\infer[{\it focus}_L]
 {\foc{\Psi}{\tackon{\Theta'}{x{:}\istrue{A^-}}}{\istrue{\susp{{\ocircle}B^+}}}}
 {\deduce
  {\foc{\Psi}{\tackon{\Theta'}{[A^-]}}{\istrue{\susp{{\ocircle}B^+}}}}
  {\vdots\mathstrut}}
 &
 \infer[{\ocircle}_L]
 {\foc{\Psi}{\tackon{\Theta}{[{\ocircle}B^+]}}{\islax{\lf{\sigma}A^+}}}
 {\deduce
  {\foc{\Psi}
    {\tackon{\Theta}{B^+}}
    {\islax{\lf{\sigma}A^+}}\mathstrut} 
  {\deduce{\vdots\mathstrut}
    {\vspace{-4pt}\foc{\Psi'}
     {\Delta'}
     {\islax{\lf{\sigma'}A^+}}}}}}
\]
The spine $\Sp$ corresponds to the complete proof of
$\foc{\Psi}{\tackon{\Theta'}{[A^-]}}{\istrue{\susp{{\ocircle}B^+}}}$,
and the pattern $P$ corresponds to the partial proof from $\foc{\Psi'}
{\Delta'} {\islax{\lf{\sigma'}A^+}}$ to $\foc{\Psi}
{\tackon{\Theta}{B^+}} {\islax{\lf{\sigma}A^+}}$. The typing rules for
steps are given in Figure~\ref{fig:sls-steps}.  Because we understand
these synthetic inference rules as relations between process states,
we call the type of a step a {\it synthetic transition}. Traces $T$
are monoids over steps -- $\emptytrace$ is an empty trace, $S$ is a
trace consisting of a single step, and $T_1; T_2$ is the sequential
composition of traces. The typing rules for traces in
Figure~\ref{fig:sls-traces} straightforwardly reflect this monoid
structure. Both of the judgments $S :: (\Psi; \Delta)_{\lf{\sigma}} 
   \leadsto_{\Sigma,\subord}
  (\Psi'; \Delta')_{\lf{\sigma'}}$ and $T :: (\Psi; \Delta)_{\lf{\sigma}} 
   \leadsto^*_{\Sigma,\subord}
  (\Psi'; \Delta')_{\lf{\sigma'}}$ work like the rules for patterns,
in that the step $S$ or trace $T$ is treated as an input along with
the initial process state $(\Psi; \Delta)_{\lf{\sigma}}$, whereas the final
process state $(\Psi'; \Delta')_{\lf{\sigma'}}$ is treated as an output.

\begin{figure}
\fbox{$S :: (\Psi; \Delta)_{\lf{\sigma}} 
   \leadsto_{\Sigma,\subord}
  (\Psi'; \Delta')_{\lf{\sigma'}}$} -- presumes
  $\vdash_{\Sigma,\subord} (\Psi; \Delta)_{\lf{\sigma}}\,{\sf state}$
  and $\Psi \vdash_{\Sigma,\subord} \Delta\,{\sf stable}$
\[
\infer
{\trstep{P}{R} :: 
  (\Psi; \frameoff{\Theta}{\Delta})_{\lf{\sigma}} 
   \leadsto_{\Sigma,\subord}
  (\Psi'; \Delta')_{\lf{\sigma'}}}
{\foctx{\Psi}{\Delta}{R}{\istrue{\susp{{\ocircle}{B^+}}}}
 &
 P :: (\Psi,\tackon{\Theta}{B^+})_{\lf{\sigma}}
   \Longrightarrow_{\Sigma,\subord}
      (\Psi'; \Delta')_{\lf{\sigma'}}}
\]
\caption{\sls~steps}
\label{fig:sls-steps}
\end{figure}

\begin{figure}
\fbox{$T :: (\Psi; \Delta)_{\lf{\sigma}} 
   \leadsto^*_{\Sigma,\subord}
  (\Psi'; \Delta')_{\lf{\sigma'}}$} -- presumes
  $\vdash_{\Sigma,\subord} (\Psi; \Delta)_{\lf{\sigma}}\,{\sf state}$
  and $\Psi \vdash_{\Sigma,\subord} \Delta\,{\sf stable}$
\[
\infer
{\emptytrace :: (\Psi; \Delta)_{\lf{\sigma}} 
               \leadsto^*_{\Sigma,\subord}
             (\Psi; \Delta)_{\lf{\sigma}}}
{}
\quad
\infer
{S :: (\Psi; \Delta)_{\lf{\sigma}}
               \leadsto^*_{\Sigma,\subord}
             (\Psi'; \Delta')_{\lf{\sigma'}}}
{S :: (\Psi; \Delta)_{\lf{\sigma}}
               \leadsto_{\Sigma,\subord}
             (\Psi'; \Delta')_{\lf{\sigma'}}}
\]
\[
\infer
{T; T' :: (\Psi_1; \Delta_1)_{\lf{\sigma_1}}
               \leadsto^*_{\Sigma,\subord}
             (\Psi_3; \Delta_3)_{\lf{\sigma_3}}}
{T :: (\Psi_1; \Delta_1)_{\lf{\sigma_1}}
               \leadsto^*_{\Sigma,\subord}
             (\Psi_2; \Delta_2)_{\lf{\sigma_2}}
&
T' :: (\Psi_2; \Delta_2)_{\lf{\sigma_2}}
               \leadsto^*_{\Sigma,\subord}
             (\Psi_3; \Delta_3)_{\lf{\sigma_3}}}
\]
\caption{\sls~traces}
\label{fig:sls-traces}
\end{figure}

Steps incorporate left focus and the left rule for ${\ocircle}$, and
{\it let-expressions} $\tlet{T}{V}$, which include traces in deductive
terms, incorporate right focus and the right rule for
the lax modality in \ollll:

\[
\infer[{\ocircle}_R]
{\foc{\Psi}{\Delta}{{\ocircle}A^-}}
{\deduce
  {\foc{\Psi}
    {\Delta}
    {\islax{A^+}}\mathstrut} 
  {\deduce{\vdots\mathstrut}
    {\vspace{-4pt}\infer[]
     {\foc{\Psi'}
      {\Delta'}
      {\islax{\lf{\sigma'}A^+}}}
     {\deduce{\foc{\Psi'}{\Delta'}{[\lf{\sigma'}A^+]}}
      {\vdots}}}}}
\]
The trace $T$ represents the entirety of the partial
proof from $\foc{\Psi}
    {\Delta}
    {\islax{A^+}}$ to $\foc{\Psi'}
      {\Delta'}
      {\islax{\lf{\sigma'}A^+}}$ that proceeds by repeated use of steps
or synthetic transitions, and the eventual conclusion  $V$ represents the 
complete proof of $\foc{\Psi'}
      {\Delta'}
      {\islax{[\lf{\sigma'}A^+]}}$ that follows the series of synthetic
transitions.

Both of the endpoints of a trace are stable sequents, but it will
occasionally be useful to talk about steps and traces that start from
unstable sequents by decomposition positive propositions. We will use
the usual trace notation $(\Psi; \Delta)_{\lf\sigma}
\leadsto^*_{\Sigma,\subord} (\Psi'; \Delta')_{\lf{\sigma'}}$ to
describe the type of these partial proofs. The proof term
associated with this type will be written as $\lambda P.T$, where $P :: (\Psi;
\Delta)_{\lf\sigma} \Longrightarrow_{\Sigma,\subord} (\Psi'';
\Delta'')_{\lf{\sigma''}}$ and $T :: (\Psi'';
\Delta'')_{\lf{\sigma''}} \leadsto^*_{\Sigma,\subord} (\Psi';
\Delta')_{\lf{\sigma'}}$. 


\subsection{Presenting traces}

To present traces in a readable way, we can use a notation that
interleaves process states among the steps of a trace, a common
practice in Hoare-style reasoning \cite{hoare71proof}.  As an example,
recall the series of transitions that our money-store-battery-robot
system took Section~\ref{sec:linlogtrans}: 
%
\[
\begin{array}{ccccc}
\begin{array}{c}
\mbox{\it \$6 (1)}\medskip\\ 
\mbox{\it battery-less robot (1)} \medskip\\ 
\mbox{\it turn \$6 into a battery}\\
\mbox{\it (all you want)}
\end{array}
& \leadsto &
\begin{array}{c}
\mbox{\it battery  (1)}\medskip\\ 
\mbox{\it battery-less robot (1)} \medskip\\ 
\mbox{\it turn \$6 into a battery}\\
\mbox{\it (all you want)}
\end{array}
& \leadsto &
\begin{array}{c}
\mbox{\it robot (1)} \medskip\\ 
\mbox{\it turn \$6 into a battery}\\
\mbox{\it (all you want)}\medskip\\~\\
\end{array}
\end{array}
\]
%
This evolution can now be precisely captured as a trace in \sls:
\begin{align*}
&\qquad\qquad
\left(
 x{:}\iseph{\susp{\sf 6bucks}}, ~~
 f{:}\iseph{({\sf battery} \lefti {\ocircle}{\sf robot})}, ~~
 g{:}\ispers{({\sf 6bucks} \lefti {\ocircle}{\sf battery})}
\right)
\\
&\trstep{y}{g\,x};
\\
&\qquad\qquad
\left(
 y{:}\iseph{\susp{\sf battery}}, ~~
 f{:}\iseph{({\sf battery} \lefti {\ocircle}{\sf robot})}, ~~
 g{:}\ispers{({\sf 6bucks} \lefti {\ocircle}{\sf battery})}
\right)
\\
&\trstep{z}{f\,y}
\\
&\qquad\qquad
\left(
 z{:}\iseph{\susp{\sf robot}}, ~~
 g{:}\ispers{({\sf 6bucks} \lefti {\ocircle}{\sf battery})}
\right)
\end{align*}

\subsection{Frame properties}

The {\it frame rule} is a concept from separation logic
\cite{reynolds02separation}.  It states that if a property holds of
some program, then the property holds under any extension of the
mutable state. The frame rule increases the modularity of separation
logic proofs, because two program fragments that reason about
different parts of the state can be reasoned about independently.

Similar frame properties hold for \sls~traces. The direct analogue of
the frame rule is the observation that a trace can always have some
extra state framed on to the outside. This is a generalization of
weakening to \sls~traces.

\bigskip
\begin{theorem}[Frame weakening]\label{thm:frameweak}~\\
If $T :: (\Psi; \Delta) \leadsto^*_{\Sigma, \subord} (\Psi'; \Delta')$, then
$T :: (\Psi, \Psi''; \tackon{\Theta}{\Delta})
       \leadsto^*_{\Sigma, \subord} (\Psi', \Psi''; \tackon{\Theta}{\Delta})$.
\end{theorem}
\begin{proof}
Induction on $T$ and case analysis on the first steps of $T$, using admissible
weakening and the properties of matching constructs at each step.
\end{proof}

The frame rule is a weakening property which ensures that new,
irrelevant state can can always be added to a state. Conversely, any
state that is never accessed or modified by a trace can be always be 
removed without making the trace ill-typed. This property
is a generalization of strengthening to \sls~traces.

\bigskip
\begin{theorem}[Frame strengthening]\label{thm:framestrong}~\\
  If $T :: (\Psi; \tackon{\Theta}{x{:}\islvl{Y}}) \leadsto^*_{\Sigma,
    \subord} (\Psi'; \tackon{\Theta'}{x{:}\islvl{Y}})$ and $x$ is not
  free in any of the steps of $T$, then $T :: (\Psi;
  \tackon{\Theta}{\cdot}) \leadsto^*_{\Sigma, \subord} (\Psi';
  \tackon{\Theta}{\cdot})$.
\end{theorem}
\begin{proof}
  Induction on $T$ and case analysis on the first steps of $T$, using
  a lemma to enforce that, if $x$ is not free in an individual step,
  it is either not present in the context of the subderivation 
  (if $\mlvl = \mtrue$ or $\meph$) or else
  it can be strengthened away (if $\mlvl = \mpers$).
\end{proof}

\section{Concurrent equality}
\label{sec:linconcurrenteq}
\label{sec:framework-concurrenteq}

Concurrent equality is a notion of equivalence on traces that is
coarser than the equivalence relation we would derive from partial
\ollll~proofs. Consider the following \sls~signature:
\begin{align*}
 \Sigma = \cdot, 
~&{\sf a} : {\sf prop}\,{\sf lin},
~ {\sf b} : {\sf prop}\,{\sf lin},
~ {\sf c} : {\sf prop}\,{\sf lin},
~ {\sf d} : {\sf prop}\,{\sf lin},
~ {\sf e} : {\sf prop}\,{\sf lin},
~ {\sf f} : {\sf prop}\,{\sf lin},
\\ & 
  {\sf first}  : {\sf a} \lefti {\ocircle}({\sf b} \fuse {\sf c}), 
\\ &
  {\sf left}  : {\sf b} \lefti {\ocircle}{\sf d}, ~
\\ &
  {\sf right} : {\sf c} \lefti {\ocircle}{\sf e}, ~
\\ &
  {\sf last} : {\sf d} \fuse {\sf e} \lefti {\ocircle}{\sf f}
\end{align*}
Under the signature $\Sigma$,
we can create two traces with the type
$x_a{:}\susp{\sf a} \leadsto^*_{\Sigma,\subord} x_f{:}\susp{\sf f}$:
\[
\begin{array}{rcl}
T_1 & = 
 & \trstep{x_b, x_c}{\sf first}\,x_a;\\
&& \trstep{x_d}{\sf left}\,x_b;\\
&& \trstep{x_e}{\sf right}\,x_c;\\
&& \trstep{x_f}{\sf last}\,(\tfuser{x_d}{x_e})
\end{array}
\quad
\mbox{versus}
\quad
\begin{array}{rcl}
T_2 & = 
 & \trstep{x_b, x_c}{\sf first}\,x_a;\\
&& \trstep{x_e}{\sf right}\,x_c;\\
&& \trstep{x_d}{\sf left}\,x_b;\\
&& \trstep{x_f}{\sf last}\,(\tfuser{x_d}{x_e})
\end{array}
\]
In both cases, there is an $x_a{:}\susp{\sf a}$ resource
that transitions to a $x_b{:}\susp{\sf b}$ resource and a
$x_c{:}\susp{\sf c}$ resource, and then $x_b{:}\susp{\sf b}$
transitions to $x_d{:}\susp{\sf d}$ while, independently,
$x_c{}\susp{\sf c}$ transitions to $x_d{:}\susp{\sf d}$. Then,
finally, the $x_d{:}\susp{\sf d}$ and $x_e{:}\susp{\sf e}$ combine to
transition to $x_f{:}\susp{\sf f}$, which completes the trace. 

The independence here is key: if two steps consume different
resources, then we want to treat them as independent concurrent steps
that could have equivalently happened in the other order. However, if
we define equivalence only in terms of the $\alpha$-equivalence of
partial \ollll~derivations, the two traces above are distinct. In this
section, we introduce a coarser equivalence relation, {\it concurrent
  equality}, that allows us to treat traces that differ only in the
interleaving of independent and concurrent steps as being equal.  The
previous section considered the proof terms of \sls~as a fragment of
\ollll~better able to talk about partial proofs.  The introduction of
concurrent equality takes a step beyond \ollll, because it breaks the
bijective correspondence between \ollll~proofs and \sls~proofs. As the
example above indicates, there are simply more \ollll~proofs than
\sls~proofs when we quotient the latter modulo concurrent equality and
declare $T_1$ and $T_2$ to be (concurrently) equal.

Concurrent equality was first was introduced and explored in the
context of CLF \cite{watkins02concurrent}, but our presentation
follows the reformulation in \cite{cervesato12trace}, which defines
concurrent equivalence based on an analysis of the variables that are
used (inputs) and introduced (outputs) by a given step.  Specifically,
our strategy will be to take a particular well-typed trace $T$ and
define a set $I$ of pairs of states $(S_1, S_2)$ with the property
that, if $S_1; S_2$ is a well-typed trace, then $S_2; S_1$ is a
concurrently equivalent and well-typed trace.  This {\it independency
  relation} allows us to treat the trace $T$ as a {\it trace
  monoid}. Concurrent equality, in turn, is just $\alpha$-equality of
\sls~proof terms combined with treating $\tlet{T}{V}$ and
$\tlet{T'}{V}$ as equivalent if $T$ and $T'$ are equivalent
according to the equivalence relation imposed by treating $T$ and $T'$
as trace monoids.

This formulation of concurrent equality facilitates applying the rich
theory developed around trace monoids to \sls~traces.  For example, it
is decidable whether two traces $T$ and $T'$ are equivalent as trace
monoids, and there are algorithms for determining whether $T'$ is a
subtrace of $T$ (that is, whether there exist $T_{\it pre}$ and
$T_{\it post}$ such that $T$ is equivalent $T_{\it pre}; T'; T_{\it
  post}$) \cite{diekert90combinatorics}. A different sort of matching
problem, in which we are given $T$, $T_{\it pre}$, and $T_{\it post}$
and must determine whether there exists a $T'$ such that $T$ is
equivalent $T_{\it pre}; T'; T_{\it post}$, was considered in
\cite{cervesato12trace}. However, this treatment is complicated by 

Unfortunately, the presence of equality in \sls~complicates our
treatment of independency. The {\it interface} of a step is used to
define independency on steps $S = (\trstep{P}{R})$. Two components of
the interface, the input variables ${^\bullet}S$ and the output
variables $S{^\bullet}$ are standard in the literature on Petri nets
-- see, for example, \cite[p.~553]{murata89petri}. The third component
(unified variables ${^\circledast}S$) is unique to our presentation.
\bigskip
\begin{definition}[Interface of a step]~
\begin{itemize}
\item The {\em input variables} of a step, denoted ${^\bullet}S$, are
  all the LF variables $\lf{a}$ and \sls~variables $x$ free in the
  normal term $R$.

\item The {\em output variables} of a step $S = (\trstep{P}{R})$,
  denoted by $S{^\bullet}$, are all the LF variables $\lf{a}$ and
  \sls~variables $x$ bound by the pattern $P$ that are not
  subsequently consumed by a substitution $\lf{t/a}$ in the same
  pattern.

\item The {\em unified variables} of a step, denoted by ${^\circledast}S$, are
  the free variables of a step that are modified by unification. If
  $\lf{t/a}$ appears in a pattern and $\lf{a}$ is free in the pattern,
  then $\lf{t} = \lf{b}$ for some other variable $\lf{b}$; both
  $\lf{a}$ and $\lf{b}$ (if the latter is free in the pattern) are
  included in the step's unified variables.
\end{itemize}
\end{definition}
\bigskip

Consider a well-typed trace $S_1; S_2$ with two steps. It is possible,
by renaming variables bound in patterns, to ensure that $\emptyset =
{^\bullet}S_1 \cap S_2{^\bullet} = {^\bullet}S_1 \cap S_1{^\bullet} =
{^\bullet}S_2 \cap S_2{^\bullet}$. We will generally assume that, in the
traces we consider, the variables introduced in each step are renamed
to be distinct from variables bound or free in all previous steps.

If $S_1; S_2$ is a well-typed trace, then the order of $S_1$ and $S_2$
is fixed if $S_1$ introduces variables that are used by $S_2$ -- that
is, if $\emptyset \neq S_1{^\bullet} \cap {^\bullet}S_2$. For example,
if $S_1 = (\trstep{x_b, x_c}{\sf first}\,x_a)$ and $S_2 =
(\trstep{x_d}{\sf left}\,x_b)$, then $\{x_b\} = S_1{^\bullet} \cap
{^\bullet}S_2$, and the two steps cannot be reordered relative to one
another.  Conversely, the condition that $\emptyset = S_1{^\bullet}
\cap {^\bullet}S_2$ is sufficient to allow reordering in a CLF-like
framework \cite{cervesato12trace}, and is also sufficient to allow
reordering in \sls~when neither step contain no unified variables
(that is, when $\emptyset = {^\circledast}S_1 = {^\circledast}S_2$).
The unification driven by equality, however, can have subtle
effects. Consider the following two-step trace:
\begin{align*}
& \qquad\qquad
(\lf{a}{:}p, \lf{b}{:}p; ~~ x{:}\iseph{\left({\ocircle}(\lf{b} \doteq_p \lf{a})\right)}, ~~
 y{:}\iseph{\left({\sf foo}\,\lf{b} 
                 \lefti {\ocircle}({\sf bar}\, \lf{a})\right)}, ~~
 z{:}\iseph{\susp{{\sf foo}\,\lf{a}}})
\\
& \trstep{\lf{b/a}}{x};
\\
& \qquad\qquad
(\lf{b}{:}p; ~~ 
 y{:}\iseph{\left({\sf foo}\,\lf{b} 
                 \lefti {\ocircle}({\sf bar}\, \lf{b})\right)}, ~~
 z{:}\iseph{\susp{{\sf foo}\,\lf{b}}})
\\
& \trstep{w}{y\,z} 
\\
& \qquad\qquad
(\lf{b}{:}p; ~~ w{:}\iseph{\susp{{\sf bar}\,\lf{b}}})
\end{align*}
This trace cannot be reordered even though $\emptyset = \emptyset \cap
\{ y, z \} = (\trstep{\lf{b/a}}{x}){^\bullet} \cap
{^\bullet}(\trstep{w}{y\,z})$, because the atomic term $y\,z$ is only
well typed after the LF variables $\lf{a}$ and $\lf{b}$ are unified.
It is not even sufficient to compare the free and unified variables
(requiring that $\emptyset = {^\circledast}S_1 \cap {^\bullet}S_2$),
as in the example above ${^\circledast}(\trstep{\lf{b/a}}{x}) = \{
\lf{a},\lf{b} \}$ and ${^\bullet}(\trstep{w}{y\,z}) = \{ y, z \}$ --
and obviously $\emptyset = \{ \lf{a},\lf{b} \} \cap \{ y, z \}$.


The simplest solution is to forbid steps with unified variables from
being reordered at all: we can say that $(S_1,S_2) \in I$ if
$\emptyset = S_1{^\bullet} \cap {^\bullet}S_2 = {^\circledast}S_1 =
{^\circledast}S_2$. It is unlikely that this condition is satisfying
in general, but it is sufficient for all the examples in this
thesis. In the remainder of this thesis, we will define concurrent
equality on the basis of this simple solution.  Nevertheless, three
other possibilities are worth considering; all three are
equivalent to this simple solution as far as the examples in this
thesis are concerned.

\paragraph{Restricting open propositions} Part of the problem with the
example above was that there were variables free in the {\it type} of
a transition that were not free in the {\it term}.  A solution is to
restrict propositions so that negative propositions in the context are
always {\it closed} relative to the LF context (or at least relative
to the part of the LF context that mentions types subject to pure
variable equality, which would be simple enough to determine with a
subordination-based analysis). This restriction means that a step $S =
\trstep{P}{R}$ can only have the parameter $\lf{a}$ free in $R$'s type
if $\lf{a}$ is free in $R$, allowing us to declare that $S_1$ and
$S_2$ are reorderable -- meaning $(S_1,S_2)$ and $(S_2,S_1)$ are in the
independency relation $I$ -- whenever
$\emptyset = S_1{^\bullet} \cap {^\bullet}S_2 = {^\circledast}S_1 \cap
{^\bullet}S_2 = {^\bullet}S_1 \cap {^\circledast}S_2$. 

While this
restriction would be sufficient for the examples in this thesis, it
would preclude a conjectured extension of the destination-adding
transformation given in Chapter~\ref{chapter-destinations} to nested
specifications (nested versus flat specifications are discussed in
Section~\ref{section-introtologicalcorrespondence}).

\paragraph{Re-typing}
Another alternate solution would be to follow CLF and allow
any reordering permitted by the input and output interfaces, but then
forbid those that cannot be re-typed. (This was necessary in CLF to
deal with the presence of $\top$.) This is very undesirable, however, 
because it leads to strange asymmetries. The following trace would be 
reorderable by this definition, for example, but a symmetric case where
the equality was $\lf{b} \doteq_p \lf{a}$ instead of  
$\lf{a} \doteq_p \lf{b}$, that would no longer be the case.
\begin{align*}
& \qquad\qquad
(\lf{a}{:}p, \lf{b}{:}p; ~~ x{:}\iseph{\left({\ocircle}(\lf{a} \doteq_p \lf{b})\right)}, ~~
 y{:}\iseph{\left(\forall \lf{w}.\,{\sf foo}\,\lf{w} 
                 \lefti {\ocircle}({\sf bar}\, \lf{w})\right)}, ~~
 z{:}\iseph{\susp{{\sf foo}\,\lf{b}}})
\\
& \trstep{w}{y\,\lf{b}\,z};
\\
& \qquad\qquad
(\lf{a}{:}p, \lf{b}{:}p; ~~ x{:}\iseph{\left({\ocircle}(\lf{a} \doteq_p \lf{b})\right)}, ~~
 w{:}\iseph{\susp{{\sf bar}\,\lf{a}}})
\\
& \trstep{\lf{b/a}}{x}
\\
& \qquad\qquad
(\lf{b}{:}p; ~~ w{:}\iseph{\susp{{\sf bar}\,\lf{b}}})
\end{align*}

\paragraph{Process states with equality constraints}
A third possible solution is be to change the way we handle the
interaction of process states and unification. In this formulation of
\sls, the process state $(\Psi; \Delta)_{\lf\sigma}$ uses $\lf\sigma$
to capture the constraints that have been introduced by equality. As
an alternative, we could have process states mention an explicit
constraint store of equality propositions that have been encountered,
as in Jagadeesan et al.'s formulation of concurrent constraint
programming \cite{jagadeesan05testing}. Process states with equality
constraints might facilitate talking about the interaction of equality
and typing, which in our current formulation is left rather implicit. 

\subsection{Multifocusing}

Concurrent equality is related to the equivalence relation induced by
{\it multifocusing} \cite{chaudhuri08canonical}. Like concurrent
equality, multifocusing imposes a coarser equivalence relation on
focused proofs. The coarser equivalence relation is enabled by a
somewhat different mechanism: we are allowed to begin focus on
multiple propositions simultaneously.


\begin{figure}
\begin{center}
\begin{tikzpicture}
\draw[->] (0,20) -- node[above]{$u_1{:}\susp{\sf a}$} (1,20);
\draw (1,20) node[right]
{\footnotesize \gr{\trstep{v_1,w_1}{{\sf first}\,u_1}}};
%
\draw[->] (4,20.3) -- node[above]{$v_1{:}\susp{\sf b}$\qquad~} (5,21);
\draw (5,21) node[right]
{\footnotesize \gr{\trstep{x_1}{{\sf left}\,v_1}}};
\draw[->] (4,19.7) -- node[below]{$w_1{:}\susp{\sf c}$\qquad~} (5,19);
\draw (5,19) node[right]
{\footnotesize \gr{\trstep{y_1}{{\sf right}\,w_1}}};
%
\draw[->] (7.4,21) -- node[above]{~\qquad$x_1{:}\susp{\sf d}$} (8.7,20.3) ;
\draw[->] (7.7,19) -- node[below]{~\qquad$y_1{:}\susp{\sf e}$} (8.7,19.7) ;
\draw (8.7,20) node[right]
{\footnotesize \gr{\trstep{z_1}{{\sf last}\,(\tfuser{x_1}{y_1})}}};
%
\draw[->] (12.1,20) -- node[above]{$z_1{:}\susp{\sf f}$} (13.1,20);
%%%
%%%
%%%
\draw[->] (0,17) -- node[above]{$u_2{:}\susp{\sf a}$} (1,17);
\draw (1,17) node[right]
{\footnotesize \gr{\trstep{v_2,w_2}{{\sf first}\,u_2}}};
%
\draw[->] (4,17.3) -- node[above]{$v_2{:}\susp{\sf b}$\qquad~} (5,18);
\draw (5,18) node[right]
{\footnotesize \gr{\trstep{x_2}{{\sf left}\,v_2}}};
\draw[->] (4,16.7) -- node[below]{$w_2{:}\susp{\sf c}$\qquad~} (5,16);
\draw (5,16) node[right]
{\footnotesize \gr{\trstep{y_2}{{\sf right}\,w_2}}};
%
\draw[->] (7.4,18) -- % node[above]{\quad\qquad$x_2{:}\susp{\sf d}$}
 (8.7,14.3) ;
\draw[->] (7.7,16) -- %node[above, near end]{$y_2{:}\susp{\sf e}$} 
(8.7,17) ;
\draw (8.7,17) node[right]
{\footnotesize \gr{\trstep{z_2}{{\sf last}\,(\tfuser{x_3}{y_2})}}};
%
\draw[->] (12.1,17) -- node[above]{$z_2{:}\susp{\sf f}$} (13.1,17);
%%%
%%%
%%%
\draw[->] (0,14) -- node[above]{$u_3{:}\susp{\sf a}$} (1,14);
\draw (1,14) node[right]
{\footnotesize \gr{\trstep{v_3,w_3}{{\sf first}\,u_3}}};
%
\draw[->] (4,14.3) -- node[above]{$v_3{:}\susp{\sf b}$\qquad~} (5,15);
\draw (5,15) node[right]
{\footnotesize \gr{\trstep{x_3}{{\sf left}\,v_3}}};
\draw[->] (4,13.7) -- node[below]{$w_3{:}\susp{\sf c}$\qquad~} (5,13);
\draw (5,13) node[right]
{\footnotesize \gr{\trstep{y_3}{{\sf right}\,w_3}}};
%
\draw[->] (7.4,15) -- %node[below, near end]{\quad\qquad$x_3{:}\susp{\sf d}$} 
(8.7,16.7) ;
\draw (8.5,17.4) node{$y_2{:}\susp{\sf e}$};
\draw (8.5,16.4) node[right]{$x_3{:}\susp{\sf d}$};
\draw (8.5,15) node[right]{$x_2{:}\susp{\sf d}$};
\draw[->] (7.7,13) -- node[below]{~\qquad$y_3{:}\susp{\sf e}$} (8.7,13.7) ;
\draw (8.7,14) node[right]
{\footnotesize \gr{\trstep{z_3}{{\sf last}\,(\tfuser{x_2}{y_3})}}};
%
\draw[->] (12.1,14) -- node[above]{$z_3{:}\susp{\sf f}$} (13.1,14);
\end{tikzpicture}
\end{center}
\caption{Interaction diagram for a trace $(u_1{:}\susp{\sf a}, u_2{:}\susp{\sf a},u_3{:}\susp{\sf a}) \leadsto^*_{\Sigma} (z_1{:}\susp{\sf f}, z_2{:}\susp{\sf f},z_3{:}\susp{\sf f})$}
\label{fig:trace-dag}
\end{figure}

Both multifocusing and concurrent equality seek to address the
sequential structure of focused proofs. The sequential structure of
a computation obscures the fact that the interaction between resources
has the structure of a directed acyclic graph (DAG), not a list. We
sketch a radically different way of presenting traces in
Figure~\ref{fig:trace-dag}, where resources are the edges in the DAG
and steps or synthetic inference rules are the vertexes. (The crossed
edges that exchange $x_2$ and $x_3$ are only well-typed because, in
our example trace, ${\sf e}$ and ${\sf d}$ were both declared to be
ephemeral propositions.) Multifocusing gives a unique normal form to
proofs by gathering all the focusing steps that can be rotated all the
way to the beginning, then all the focusing steps that can happen as
soon as those first steps have been rotated all the way to the
beginning, etc. In \sls, by contrast, we are content to represent the
DAG structure as a list combined with the equivalence relation given
by concurrent equality.

Multifocusing has only been explored carefully in the context of
classical linear logic. We conjecture that a suitably-defined notion
of multifocusing for \ollll~would be in bijective correspondence with
\sls~terms modulo concurrent equivalence, at least if we omit
equality. Of course, without a formal notion of multifocusing for
intuitionistic logic, this conjecture is impossible to state
explicitly. The analogy with multifocusing may be able shed light on
our difficulties in integrating concurrent equality and unification
of pure variable types, because multifocusing has an independent
notion of correctness: the equivalence relation given by multifocusing
coincides with the the least equivalence relation that includes all
permutations of independent rules in an unfocused sequent calculus
proof \cite{chaudhuri08canonical}.

\section{Adequate encoding}
\label{sec:sls-adequate}

In Section~\ref{sec:lf-adequacy} we discussed encoding untyped
$\lambda$-calculus terms as LF terms of type ${\sf exp}$, captured by
the invertible function $\interp{e}$. Adequacy was extended to Linear
LF (LLF) by Cervesato and Pfenning \cite{cervesato02linear} and was
extended to Ordered LF (OLF) by Polakow \cite{polakow01ordered}. The
deductive fragment of \sls~approximately extends both LLF and OLF, and
the adequacy arguments made by Cervesato and Polakow extend
straightforwardly to the deductive fragment of \sls. These adequacy
arguments do not extend to the systems we want to encode in the
non-deductive fragment of \sls, however. The more general techniques
we consider in this section will be explored further in 
Chapter~\ref{chapter-gen} as a
general technique for capturing invariants of \sls~specifications.

The example that we will give to illustrate adequate encoding is the
following signature, the \sls~encoding of the push-down automata 
for parentheses matching from the introduction; we replace the atomic
proposition ${\sf <}$ with ${\sf L}$ and the proposition ${\sf >}$ with
${\sf R}$: 
\begin{align*}
 \Sigma_{\it PDA} = \cdot, 
~&{\sf L} : {\sf prop}\,{\sf ord},\\
~&{\sf R} : {\sf prop}\,{\sf ord},\\
~&{\sf hd} : {\sf prop}\,{\sf ord},\\
~&{\sf push} : 
     {\sf hd} \fuse {\sf L}
       \lefti {\ocircle}({\sf L} \fuse {\sf hd}),\\
~&{\sf pop} : 
     {\sf L} \fuse {\sf hd} \fuse {\sf R}
       \lefti {\ocircle}({\sf hd})
\end{align*}
We will relate this specification to a push-down automata defined in
terms of stacks $\obj{k}$ and strings $\obj{s}$, which we define inductively:
\begin{align*}
\obj{k} & ::= \obj{\cdot} \mid \obj{k{\sf <}}\\
\obj{s} & ::= \obj{\cdot} \mid \obj{{\sf <}s} \mid \obj{{\sf >}s}
\end{align*}
The transition system defined in terms of the stacks and strings
has two transitions:
\begin{align*}
\obj{(k
  \rhd ({\sf <}s))} & \mapsto \obj{((k{\sf <}) \rhd s)} \\
\obj{((k{\sf <}) \rhd ({\sf >}s))} & \mapsto \obj{(k \rhd s)}
\end{align*}

Existing adequacy arguments for CLF specifications by Cervesato et
al.~\cite{cervesato02concurrent} and by
Schack-Nielsen~\cite{schacknielsen07induction} have a three-part
structure structure.  The first step is to define an encoding
function $\ctxinterp{k \rhd s} = \Delta$ from PDA states $\obj{k \rhd
  s}$ to process states $\Delta$, so that, for example, the PDA state
$\obj{(\cdot{\sf <<}) \rhd ({\sf >>><}\cdot)}$ is encoded as the
process state
\[
x_2{:}\istrue{\susp{\sf L}}, ~~
x_1{:}\istrue{\susp{\sf L}}, ~~
h{:}\istrue{\susp{\sf hd}}, ~~
y_1{:}\istrue{\susp{\sf R}}, ~~
y_2{:}\istrue{\susp{\sf R}}, ~~
y_3{:}\istrue{\susp{\sf R}}, ~~
y_4{:}\istrue{\susp{\sf L}}
\]
The second step is to prove a preservation-like property: if
$\ctxinterp{\obj{k \rhd s}} \leadsto_{\Sigma_{\it PDA}}
\Delta'$, then $\Delta' = \ctxinterp{{k' \rhd s'}}$ for some
$\obj{k'}$ and $\obj{s'}$. The third step is the main adequacy result:
that $\ctxinterp{k \rhd s} \leadsto_{\Sigma_{\it PDA}} \ctxinterp{k'
  \rhd s'}$ if and only if $\obj{k \rhd s} \mapsto \obj{k' \rhd
  s'}$. 

The second step is crucial in general: without it, we might transition
in \sls~from the encoding of some $\obj{k \rhd s}$ to a state
$\Delta'$ that is not in the image of encoding. % With our
%deterministic example, this concern is less obvious: we could avoid
%the second proof by instead proving that if $\ctxinterp{k \rhd s}
%\leadsto_{\Sigma_{\it PDA}} \Delta$ and $\ctxinterp{k \rhd s}
%\leadsto_{\Sigma_{\it PDA}} \Delta'$ then $\Delta = \Delta'$. 
We will 
take the opportunity to re-factor Cervesato et al.'s approach, 
replacing the second step with a general statement about transitions
in $\Sigma_{\it PDA}$ preserving a well-formedness invariant. The 
invariant we discuss is a simple instance of
the well-formedness invariants that we will explore
further in Chapter~\ref{chapter-gen}. 

The first step in our revised methodology is to describe a generative
signature $\Sigma_{\it Gen}$ that precisely captures the set of
process states that encode machine states
(Theorem~\ref{thm:pda-encoding} below).  The second step is showing
that the generative signature $\Sigma_{\it Gen}$ describes an
invariant of the signature $\Sigma_{\it PDA}$
(Theorem~\ref{thm:pda-preservation}).  The third step, showing that
$\ctxinterp{k \rhd s} \leadsto_{\Sigma_{\it PDA}} \ctxinterp{k' \rhd
  s'}$ if and only if $\obj{k \rhd s} \mapsto \obj{k' \rhd s'}$, is
straightforward and follows other developments.

\subsection{Regular and generated worlds}

A critical aspect of any adequacy argument is an understanding of the
structure of the relevant context(s) (the LF context in LF encodings,
the substructural context in CLF encodings, both in \sls~encodings).
In the statement of adequacy for untyped $\lambda$-calculus terms
(Section~\ref{sec:lf-adequacy}), for instance, it was necessary to
require that the LF context $\Psi$ take the form $\lf{a_1}{:}{\sf
  exp},\ldots,\lf{a_n}{:}{\sf exp}$. In the adequacy theorems that
have been presented for deductive logical frameworks, the structure of
the context is always {\it regular} -- we can describe a set of
building blocks that build small pieces of the context, and then
define the set of valid contexts (the {\it world}) to be any context
that can be built from a particular set of building blocks.  The
simplest {\it regular worlds} are those that forbid the presence of
any LF variables -- when we adequately encode unary natural numbers
$\obj{n}$ as LF terms of type ${\sf nat}$, letting $\interp{\sf z} =
\lf{\sf z}$ and letting $\interp{{\sf s} n} = \lf{s}\,\interp{n}$, we
assume there aren't any free LF variables of type ${\sf nat}$! This is
the {\it closed world assumption}, and regular worlds can be presented
as a generalization of closed worlds.  Twelf contains both a syntax
for declaring regular worlds and facilities for checking that a
signature respects a particular regular world structure
\cite{schurmann00automating}.

The relevant building blocks in our $\lambda$-calculus encoding are
just LF variables of type $\lf{a}{:}{\sf exp}$. In other LF examples
regular worlds involve multiple context elements that must appear in
tandem. In proving adequacy for typing derivations, for instance, we
must generally assume that a variable $\lf{a}{:}{\sf exp}$ always
appears in tandem with a variable $\lf{d}{:}({\sf
  of}\,\lf{a}\,\lf{tp})$ that associates the variable with some closed
LF term $\lf{tp}$ of type ${\sf typ}$ that encodes the type of that
variable.

Regular worlds remain sufficient for the encoding of stores in Linear
LF \cite{cervesato02linear} and stacks in Ordered LF
\cite{polakow01ordered}. Our PDA states, on the other hand, do not
have the structure that can be described by Sch\"urmann's regular
worlds language, because the process state is organized into three
distinct zones:
\[
[~\mbox{the stack}~]
~
[~\mbox{the head}~]
~
[~\mbox{the string being read}~]
\]
This structure {\it can} be described with a regular expression, but
the structures that can be described by regular expressions are richer
than those that can be described by the language of regular worlds
given by Sch\"urmann and implemented in Twelf
\cite{schurmann00automating}.

The generalization we propose is a move from a description of worlds
based on regular worlds descriptions to a description of worlds based
on (something like) context-free grammars.  Conveniently, the
something-like-context-free grammars we are interested in can be
characterized within the machinery of \sls~itself by describing {\it
  generative signatures} that can generate the set of process states
we are interested in from a single seed context.  The signature
$\Sigma_{\it Gen}$ in Figure~\ref{fig:pda-gen} treats all the atomic
propositions of $\Sigma_{\it PDA}$ -- the atomic propositions ${\sf
  L}$, ${\sf R}$ and ${\sf hd}$ -- as {\it terminals}, and introduces
three {\it nonterminals} ${\sf gen}$, ${\sf gen\_stack}$, and ${\sf
  gen\_string}$.

\begin{figure}
\begin{align*}
 \Sigma_{\it Gen} = \cdot, 
~&{\sf L} : {\sf prop}\,{\sf ord},\\
~&{\sf R} : {\sf prop}\,{\sf ord},\\
~&{\sf hd} : {\sf prop}\,{\sf ord},\\
~&{\sf gen} : {\sf prop}\,{\sf ord},\\
~&{\sf gen\_stack} : {\sf prop}\,{\sf ord},\\
~&{\sf gen\_string} : {\sf prop}\,{\sf ord},\\
~&{\sf state} : {\sf gen} 
       \lefti {\ocircle}({\sf gen\_stack} \fuse {\sf hd} 
                         \fuse {\sf gen\_string})
&& G \rightarrow G_k\,{\sf hd}\,G_s\\
~&{\sf stack/left} : {\sf gen\_stack} 
       \lefti {\ocircle}({\sf L} \fuse {\sf gen\_stack})
&& G_k \rightarrow {\sf <}\,G_k \\
~&{\sf stack/done} : {\sf gen\_stack} \lefti {\ocircle}(\one)
&& G_k \rightarrow \epsilon\\
~&{\sf string/left} : 
     {\sf gen\_string} 
       \lefti {\ocircle}({\sf gen\_string} \fuse {\sf L})
&& G_s \rightarrow G_s\,{\sf <}\\
~&{\sf string/right} : 
     {\sf gen\_string} 
       \lefti {\ocircle}({\sf gen\_string} \fuse {\sf R})
&& G_s \rightarrow G_s\,{\sf >}\\
~&{\sf string/done} : 
     {\sf gen\_string} 
       \lefti {\ocircle}(\one)
&& G_s \rightarrow \epsilon
\end{align*}
\caption{Generative specification of PDA states and an analogous
  context-free grammar}
\label{fig:pda-gen}
\end{figure}

An informal translation of the signature $\Sigma_{\it Gen}$ as a
context-free grammar is given on the right-hand side of
Figure~\ref{fig:pda-gen}. Observe that the sentences in the language
$G$ encode the states of our PDA as a string. 

\subsection{Restriction}

The operation of {\it restriction} adapts the concept of ``terminal''
and ``non-terminal'' to \sls. Note that process states $\Delta$
such that $(x{:}\istrue{\susp{\sf gen}}) \leadsto^*_{\Sigma_{\it Gen}}
\Delta$ are only well-formed under the signature $\Sigma_{\it PDA}$ if
they are free of nonterminals; we can define an operation of {\it
  restriction} that filters out the non-terminal process states by
checking whether they are well-formed in a signature that only
declares the terminals.

\bigskip
\begin{definition}[Restriction]~
\begin{itemize}
\item
  $\restrictsig{\Psi}{\Sigma}$ is a total function that returns the
  largest context $\Psi' \subseteq \Psi$ such that $\vdash_\Sigma
  \Psi\,{\sf ctx}$ (defined in Figure~\ref{fig:lf-form}) by removing
  all the LF variables in $\Psi$ whose types are not-well-formed in
  the context $\Sigma$.

\item
  $\restrictsig{(\Psi; \Delta)}{\Sigma}$ is a partial function that is
  defined exactly when, for every variable declaration
  $x{:}\istrue{T}$ or $x{:}\iseph{T}$ in $\Delta$, we have that
  $(\restrictsig{\Psi}{\Sigma}) \vdash_\Sigma T\,{\sf left}$ (defined
  in Figure~\ref{fig:sls-ctxform}). When it is defined,
  $\restrictsig{(\Psi; \Delta)}{\Sigma} = ((\restrictsig{}{\Sigma});
  \Delta')$, where $\Delta'$ is $\Delta$ except for the variable
  declarations $x{:}\ispers{T}$ in $\Delta$ for which it was not the
  case that $(\restrictsig{\Psi}{\Sigma}) \vdash_\Sigma T\,{\sf
    left}$.

\item
  We will also use $\restrictsig{(\Psi; \Delta)}{\Sigma}$ as a
  judgment which expresses that the function as defined.

\end{itemize}
\end{definition}
\bigskip

Because 
restriction is only defined if
all the ordered and linear propositions in $\Delta$ are well-formed
in $\Sigma$; this means that $\restrictsig{(x{:}\istrue{\susp{\sf
      gen}})}{\Sigma_{\it PDA}}$ is not defined.
Restriction acts as a semi-permeable membrane on process states: some
process states cannot pass through at all, and others pass through
with some of their LF variables and persistent propositions
removed. We can represent context restriction $\restrictsig{(\Psi;
  \Delta)}{\Sigma} = (\Psi'; \Delta')$ in a two-dimensional notation
as a dashed line annotated with the restricting signature:
\begin{center}
\begin{tikzpicture} 
\draw (.8,.5) node{$(\Psi; \Delta)$};
\draw (-.3, .1) node{$\Sigma$};
\draw [thick,dash pattern = on 2.82842842712mm off 2mm,decorate,decoration={saw,amplitude=2mm,segment length=2mm}] 
(0,0) -- (1.6,0); 
\draw (.8,-.3) node{$(\Psi'; \Delta')$};
\end{tikzpicture} 
\end{center}

For all process states that evolve from the initial state
$(x{:}\istrue{\susp{\sf gen}})$ under the signature $\Sigma_{\it
  Gen}$, restriction to $\Sigma_{\it PDA}$ is the identity function
whenever it is defined. Therefore, in the statement of
Theorem~\ref{thm:pda-encoding}, we use restriction as a judgment
$\restrictsig{\Delta}{\Sigma_{\it PDA}}$ that holds whenever the partial 
function is defined.

\bigskip
\begin{theorem}[Encoding]\label{thm:pda-encoding}
  Up to variable renaming, there is a bijective correspondence between
  PDA states $\obj{k \rhd s}$ and process states $\Delta$ such that
  $T :: (x{:}\istrue{\susp{\sf gen}}) \leadsto^*_{\Sigma_{\it Gen}}
  \Delta$ and $\restrictsig{\Delta}{\Sigma_{\it
      PDA}}$.
\end{theorem}

\begin{proof}To establish the bijective correspondence, we first define
an encoding function from PDA states to process states:
\smallskip
\begin{itemize}
\item $\ctxinterp{k \rhd s} = 
  \ctxinterp{k}, ~~
  h{:}\istrue{\susp{\sf hd}}, ~~
  \ctxinterp{s}$
\item $\ctxinterp{\cdot} = \cdot$
\item $\ctxinterp{k{\sf <}} = \ctxinterp{k}, ~~ x{:}\istrue{\susp{\sf L}}$
\item $\ctxinterp{{\sf <}s} = y{:}\istrue{\susp{\sf L}}, ~~ \ctxinterp{s}$
\item $\ctxinterp{{\sf >}s} = y{:}\istrue{\susp{\sf R}}, ~~ \ctxinterp{s}$
\end{itemize}
\smallskip 
It is always the case that $\restrictsig{\ctxinterp{k \rhd
    s}}{\Sigma_{\it PDA}}$ -- the encoding only includes terminals.

It is straightforward to observe that if $\ctxinterp{k \rhd s} =
\ctxinterp{k' \rhd s'}$ if an only if $\obj{k} = \obj{k'}$ and
$\obj{s} = \obj{s'}$. The interesting part of showing that context
interpretation is an injective function is mostly just showing that it
is a function: that is, showing that, for any $\obj{k \rhd s}$, there
exists a trace $T :: (x{:}\istrue{\susp{\sf gen}}) \leadsto^*_{\it
  Gen} \ctxinterp{k \rhd s}$. To show that the encoding function is
surjective, we must show that if $T :: (x{:}\istrue{\susp{\sf gen}})
\leadsto^*_{\Sigma_{\it Gen}} \Delta$ and
$\restrictsig{\Delta}{\Sigma_{\it PDA}}$ then $\Delta = \ctxinterp{k
  \rhd s}$ for some $\obj{k}$ and $\obj{s}$. This will complete the
proof: an injective and surjective function is bijective.

\subsubsection{Encoding is injective}

We prove that for
any $\obj{k \rhd s}$, there exists a trace $T ::
(x{:}\istrue{\susp{\sf gen}}) \leadsto^*_{\it Gen} \ctxinterp{k \rhd
  s}$ with a series of three lemmas.

\begin{lemma} For all $\obj{k}$, there exists
$T :: ({x{:}\istrue{\susp{\sf gen\_stack}}}) \leadsto^*_{\Sigma_{\it Gen}} 
({\ctxinterp{k}, x'{:}\istrue{\susp{\sf gen\_stack}}})$.
\end{lemma}
\noindent
By induction on $\obj{k}$. 
\begin{itemize}
\item If $\obj{k} = \cdot$, $T = \emptytrace ::
({x{:}\istrue{\susp{\sf gen\_stack}}}) \leadsto^*_{\Sigma_{\it Gen}} 
(x{:}\istrue{\susp{\sf gen\_stack}})$
\item If $\obj{k} = \obj{k' {\sf <}}$, we have 
$T' :: ({x{:}\istrue{\susp{\sf gen\_stack}}}) \leadsto^*_{\Sigma_{\it Gen}} 
({\ctxinterp{k'}, x''{:}\istrue{\susp{\sf gen\_stack}}})$ by the induction
hypothesis, so $T = (T'; \trstep{x_1, x_2}{{\sf stack/left}\,x'}) :: 
({x{:}\istrue{\susp{\sf gen\_stack}}}) \leadsto^*_{\Sigma_{\it Gen}} 
({\ctxinterp{k'}, x_1{:}\istrue{\susp{\sf L}}, x_2{:}\istrue{\susp{\sf gen\_stack}}})$
\end{itemize}

\begin{lemma} For all $\obj{s}$, there exists
$T :: ({y{:}\istrue{\susp{\sf gen\_string}}}) \leadsto^*_{\Sigma_{\it Gen}} 
({y'{:}\istrue{\susp{\sf gen\_string}}, \ctxinterp{s}})$.
\end{lemma}
\noindent
By induction on $\obj{s}$.
\begin{itemize}
\item If $\obj{s} = \cdot$, $T = \emptytrace ::
({y{:}\istrue{\susp{\sf gen\_string}}}) \leadsto^*_{\Sigma_{\it Gen}} 
(y{:}\istrue{\susp{\sf gen\_string}})$
\item If $\obj{s} = \obj{{\sf <}s' }$, we have 
$T' :: ({y{:}\istrue{\susp{\sf gen\_string}}}) \leadsto^*_{\Sigma_{\it Gen}} 
(y'{:}\istrue{\susp{\sf gen\_string}}, \ctxinterp{s'})$ by the induction
hypothesis, so $T = (T'; \trstep{y_1, y_2}{{\sf string/left}\,y'}) :: 
({y{:}\istrue{\susp{\sf gen\_stack}}}) \leadsto^*_{\Sigma_{\it Gen}} 
(y_1{:}\istrue{\susp{\sf gen\_stack}},
 y_2{:}\istrue{\susp{\sf L}},
\ctxinterp{s'})$
\item If $\obj{s} = \obj{{\sf >}s' }$, we have 
$T' :: ({y{:}\istrue{\susp{\sf gen\_string}}}) \leadsto^*_{\Sigma_{\it Gen}} 
(y'{:}\istrue{\susp{\sf gen\_string}}, \ctxinterp{s'})$ by the induction
hypothesis, so $T = (T'; \trstep{y_1, y_2}{{\sf string/right}\,y'}) :: 
({y{:}\istrue{\susp{\sf gen\_stack}}}) \leadsto^*_{\Sigma_{\it Gen}} 
(y_1{:}\istrue{\susp{\sf gen\_stack}},
 y_2{:}\istrue{\susp{\sf R}},
\ctxinterp{s'})$
\end{itemize}

\begin{lemma} For all $\obj{k}$ and $\obj{s}$, there exists
$T :: ({g{:}\istrue{\susp{\sf gen}}}) \leadsto^*_{\Sigma_{\it Gen}} 
(\ctxinterp{k \rhd s})$. 
\end{lemma}
\noindent
By straightforward construction using the first two lemmas
and frame weakening (Theorem~\ref{thm:frameweak}): 
\begin{align*}
& \qquad\qquad (g{:}\istrue{\susp{\sf gen}})\\
& \trstep{x,h,y}{{\sf state}\,g};\\
& \qquad\qquad
       (x{:}\istrue{\susp{\sf gen\_stack}},  ~~
        h{:}\istrue{\susp{\sf hd}},~~ 
        y{:}\istrue{\susp{\sf gen\_string}})\\
& T_k; ~~ \mbox{\it (given by the first lemma and frame weakening)}\\
& \qquad\qquad
       (\ctxinterp{k},  ~~
        x'{:}\istrue{\susp{\sf gen\_stack}},  ~~
        h{:}\istrue{\susp{\sf hd}}, ~~
        y{:}\istrue{\susp{\sf gen\_string}})\\
& \trstep{()}{{\sf stack/done}\,x'}\\
& \qquad\qquad
       (\ctxinterp{k},  ~~
        h{:}\istrue{\susp{\sf hd}}, ~~
        y{:}\istrue{\susp{\sf gen\_string}})\\
& T_s; ~~ \mbox{\it (given by the second lemma and frame weakening)}\\
& \qquad\qquad
       (\ctxinterp{k}, ~~
        h{:}\istrue{\susp{\sf hd}}, ~~
        y'{:}\istrue{\susp{\sf gen\_string}}, ~~
        \ctxinterp{s})\\
& \trstep{()}{{\sf string/done}\,y'}\\
& \qquad\qquad 
       (\ctxinterp{k}, ~~
        h{:}\istrue{\susp{\sf hd}}, ~~
        \ctxinterp{s}) \\
& \qquad\qquad = \ctxinterp{k \rhd s}
\end{align*}


\subsubsection{Encoding is surjective}

We prove that if $T :: (x{:}\istrue{\susp{\sf gen}}) \leadsto^*_{\Sigma_{\it
    Gen}} \Delta$ and $\restrictsig{\Delta}{\Sigma_{\it PDA}}$ then
$\Delta = \ctxinterp{k \rhd s}$ for some $\obj{k}$ and $\obj{s}$
any $\obj{k \rhd s}$, there exists a trace $T ::
(x{:}\istrue{\susp{\sf gen}}) \leadsto^*_{\it Gen} \ctxinterp{k \rhd
  s}$ with a series of two lemmas.

\begin{lemma} If 
$T :: (\ctxinterp{k}, 
       x{:}\istrue{\susp{\sf gen\_stack}}, 
       h{:}\istrue{\susp{\sf hd}},
       y{:}\istrue{\susp{\sf gen\_store}},
       \ctxinterp{s}) 
  \leadsto^*_{\Sigma_{\it Gen}} \Delta$ and
$\restrictsig{\Delta}{\Sigma_{\it
      PDA}}$, then $\Delta = \ctxinterp{k' \rhd s'}$ for some $\obj{k'}$
and $\obj{s'}$.
\end{lemma}
\noindent
By induction on the structure of $T$ and case analysis on the 
first steps in $T$. Up to concurrent equality, there
are four possibilities:
\smallskip
\begin{itemize}
\item $T = (\trstep{()}{{\sf stack/done}\,x}; \trstep{()}{{\sf
      string/done}\,y})$ -- this is a base case, and we can finish by
  letting $\obj{k'} = \obj{k}$ and $\obj{s'} = \obj{s}$.
\item $T = (\trstep{x_1,x_2}{{\sf stack/left}\,x}; T')$ -- apply 
  the ind.~hyp.~(letting $x = x_2$, $\obj{k} =
  \obj{k{\sf <}}$).
\item $T = (\trstep{y_1,y_2}{{\sf string/left}\,y}; T')$ -- apply
  the ind.~hyp.~(letting $y = y_1$, $\obj{s} =
  \obj{{\sf <}s}$).
\item $T = (\trstep{y_1,y_2}{{\sf string/right}\,y}; T')$ -- apply
  the ind.~hyp.~(letting $y = y_1$, $\obj{s} =
  \obj{{\sf >}s}$).
\end{itemize}
\smallskip 
The proof above takes
a number of facts about concurrent equality for granted. 
%
For example, the trace $T = (\trstep{()}{{\sf stack/done}\,x};
\trstep{y_1,y_2}{{\sf string/right}\,y}; T')$ does not syntactically
match any of the traces above if we do not account for concurrent
equality. Modulo concurrent equality, on the other hand, $T =
(\trstep{y_1,y_2}{{\sf string/right}\,y}; \trstep{()}{{\sf
    stack/done}\,x}; T')$, matching the last branch of the case
analysis.  If we didn't implicitly rely on concurrent equality in this
way, the resulting proof would have twice as many cases.  We will take
these finite uses of concurrent equality for granted when we specify
that a proof proceeds by case analysis on the first steps of $T$ (or,
conversely, by case analysis on the last steps of $T$).

\begin{lemma} If 
$T :: (g{:}\istrue{\susp{\sf gen}}) 
  \leadsto^*_{\Sigma_{\it Gen}} \Delta$ and
$\restrictsig{\Delta}{\Sigma_{\it
      PDA}}$, then $\Delta = \ctxinterp{k' \rhd s'}$ for some $\obj{k'}$
and $\obj{s'}$.
\end{lemma}
\noindent
This is a corollary of the previous lemma, as it can only
be the case that $T = \trstep{x,h,y}{{\sf state}\,g}; T'$. We can apply the previous
lemma to $T'$, letting $\obj{k} = \obj{s} = \obj{\cdot}$.
This establishes that encoding is a surjective function, which in turn
completes the proof. 
\end{proof}

Theorem~\ref{thm:pda-encoding} establishes that the generative
signature $\Sigma_{\it Gen}$ describes a world -- a set of
\sls~process states -- that precisely corresponds to the states of a
push-down automata.  We can (imperfectly) illustrate the content of
this theorem in our two-dimensional notation as follows, where 
$\Delta \Leftrightarrow \obj{k \rhd s}$ indicates the presence of a
bijection:
\begin{center}
\begin{tikzpicture} 
\draw (.8,2.3) node{$(x{:}\istrue{\susp{\sf gen}})$};
\draw [->,decorate, 
decoration={snake,amplitude=.3mm,segment length=3mm,post length=1mm}] 
(0.8,2) -- (.8,.8); 
\draw (.95,.8) node{$_*$};
\draw (.3,1.4) node{$\Sigma_{\it Gen}$};
\draw (.8,.5) node{$\Delta$};
\draw (-.2, .1) node{$\Sigma_{\it PDA}$};
\draw [thick,dash pattern = on 2.82842842712mm off 2mm,decorate,decoration={saw,amplitude=2mm,segment length=2mm}] 
(.4,0) -- (1.2,0); 
\draw (.8,-.3) node{$\Delta$};
\draw (1.1,-.7) node{\begin{turn}{-45}$\Leftrightarrow$\end{turn}};
\draw (1.8,-1) node{$\obj{k \rhd s}$};
\end{tikzpicture} 
\end{center}

It is interesting to note how the proof of
Theorem~\ref{thm:pda-encoding} takes advantage of the associative
structure of traces: the inductive process that constructed traces in
the first two lemmas treated trace composition as left-associative,
but the induction we performed on traces in the next-to-last lemma
treated trace composition as right-associative.

\subsection{Generated world preservation}
\label{sec:sls-pda-preservation}

The generative signature $\Sigma_{\it Gen}$ precisely captures the
world of \sls~process states that are in the image of the encoding
$\ctxinterp{k \rhd s}$ of PDA states as process states. In order for
the signature $\Sigma_{\it PDA}$ to encode a reasonable notion of
transition between PDA states, we need to show that steps in this
signature only take encoded PDA states to encoded PDA states. Because 
the generative signature $\Sigma_{\it Gen}$ precisely captures the 
process states that represent encoded PDA states, we can describe
and prove this property without reference to the actual encoding function:
 
\bigskip
\begin{theorem}[Preservation]\label{thm:pda-preservation}
If $T_1 :: (x{:}\istrue{\susp{\sf gen}}) \leadsto^*_{\Sigma_{\it Gen}} \Delta_1$,
$\restrictsig{\Delta_1}{\Sigma_{\it PDA}}$, and 
$S :: \Delta_1 \leadsto_{\Sigma_{\it PDA}} \Delta_2$, then 
$T_2 :: (x{:}\istrue{\susp{\sf gen}}) \leadsto^*_{\Sigma_{\it Gen}} \Delta_2$.
\end{theorem}
\bigskip

\noindent
If we illustrate the given elements as solid lines and elements that we have
to prove as dashed lines, the big picture of the encoding and preservation
theorems is the following:

\begin{center}
\begin{tikzpicture} 
\draw (.8,2.3) node{$(x{:}\istrue{\susp{\sf gen}})$};
\draw [->,decorate, 
decoration={snake,amplitude=.3mm,segment length=3mm,post length=1mm}] 
(0.8,2) -- (.8,.8); 
\draw (.95,.8) node{$_*$};
\draw (.3,1.4) node{$\Sigma_{\it Gen}$};
\draw (.8,.5) node{$\Delta$};
\draw [thick,dash pattern = on 2.82842842712mm off 2mm,decorate,decoration={saw,amplitude=2mm,segment length=2mm}] 
(.4,0) -- (1.2,0); 
\draw (.8,-.3) node{$\Delta$};
\draw (1.1,-.7) node{\begin{turn}{-45}$\Leftrightarrow$\end{turn}};
\draw (1.8,-1) node{$\obj{k \rhd s}$};
%
\draw (4.8,2.3) node{$(x{:}\istrue{\susp{\sf gen}})$};
\draw [->,densely dotted,decorate, 
decoration={snake,amplitude=.3mm,segment length=3mm,post length=1mm}] 
(4.8,2) -- (4.8,.8); 
\draw (4.95,.8) node{$_*$};
\draw (4.3,1.4) node{$\Sigma_{\it Gen}$};
\draw (4.8,.5) node{$\Delta'$};
\draw (2.8, 0) node{$\Sigma_{\it PDA}$};
\draw [thick,dash pattern = on 0.677mm off .4mm on 0.676142375mm off .4mm on 0.676142375mm off 2mm,decorate,decoration={saw,amplitude=2mm,segment length=2mm}] 
(4.4,0) -- (5.2,0); 
\draw (4.8,-.3) node{$\Delta'$};
\draw (5.1,-.7) node{\begin{turn}{-45}$\Leftrightarrow$\end{turn}};
\draw (5.8,-1) node{$\obj{k' \rhd s'}$};
%
\draw [->,decorate, 
decoration={snake,amplitude=.3mm,segment length=3mm,post length=1mm}] 
(1.2,-.3) -- (4.4,-.3); 
\end{tikzpicture} 
\end{center}

\noindent
The proof of Theorem~\ref{thm:pda-preservation} relies on two lemmas, 
which we will consider before the proof itself. They are both
{\it inversion lemmas}: they help uncover the structure of the
trace based on the type of that trace. Treating
traces modulo concurrent equality is critical in both cases. 

\bigskip
\begin{lemma}
  Let $\Delta = \tackon{\Theta}{x{:}\istrue{\susp{\sf gen\_stack}},
    h{:}\istrue{\susp{\sf hd}}, y{:}\istrue{\susp{\sf gen\_string}}}$.
  If $T :: \Delta \leadsto^*_{\Sigma_{\it Gen}} \Delta'$ and
  $\restrictsig{\Delta'}{\Sigma_{\it PDA}}$, then $T = (T';
  \trstep{()}{{\sf stack/done}\,x'}; \trstep{()}{{\sf
      string/done}\,y'})$, where $T' ::
  \Delta \leadsto^*_{\Sigma_{\it Gen}}
  \tackon{\Theta'}{x'{:}\istrue{\susp{\sf gen\_stack}},
    h{:}\istrue{\susp{\sf hd}}, y'{:}\istrue{\susp{\sf gen\_string}}}$
  and $\Delta' = \tackon{\Theta'}{h{:}\istrue{\susp{\sf hd}}}$.
Or, as a picture: 

\begin{center}
\begin{tikzpicture} 
\draw (0,2.3) 
  node{$\Delta = \tackon{\Theta}{x{:}\istrue{\susp{\sf gen\_stack}},
    h{:}\istrue{\susp{\sf hd}}, y{:}\istrue{\susp{\sf gen\_string}}}$};
\draw [->,decorate, 
decoration={snake,amplitude=.3mm,segment length=3mm,post length=1mm}] 
(-2.7,2) -- (-2.7,-1.3); 
\draw (-2.55,-1.3) node{$_*$};
\draw (-2.7,1.65) node[left]{$T$};
\draw (-2.7,-1.6) node{$\Delta' \mathstrut$};
%
\draw (-2.2,1.65) node{$=$};
%
\draw [->,densely dotted,decorate, 
decoration={snake,amplitude=.3mm,segment length=3mm,post length=1mm}] 
(2.2,2) -- (2.2,1.3); 
\draw (2.2,1.65) node[left]{$T'$};
\draw [->,densely dotted,decorate, 
decoration={snake,amplitude=.3mm,segment length=3mm,post length=1mm}] 
(2.2,.7) -- (2.2,0); 
\draw (2.2,.35) node[left]{$\trstep{()}{{\sf stack/done}\,x'}$};
\draw [->,densely dotted,decorate, 
decoration={snake,amplitude=.3mm,segment length=3mm,post length=1mm}] 
(2.2,-.6) -- (2.2,-1.3); 
\draw (2.2,-.95) node[left]{$\trstep{()}{{\sf string/done}\,y'}$};
\draw (2.35,1.3) node{$_*$};
%\draw (2.2,2.3) node{$\Delta$};
\draw (1.9,1) node[right]{$\tackon{\Theta'}{x'{:}\istrue{\susp{\sf gen\_stack}},
    h{:}\istrue{\susp{\sf hd}}, y'{:}\istrue{\susp{\sf gen\_string}}}$};
\draw (1.9,-.3) node[right]{$\tackon{\Theta'}{
    h{:}\istrue{\susp{\sf hd}}, y'{:}\istrue{\susp{\sf gen\_string}}}$};
\draw (1.9,-1.6) node[right]{$\Delta' = \tackon{\Theta'}{h{:}\istrue{\susp{\sf hd}}} \mathstrut$};
\end{tikzpicture} 
\end{center}
\end{lemma}

\begin{proof}
By induction on the structure of $T$ and case analysis on the first
steps in $T$. Up to concurrent equality, there are five possibilities:
\begin{itemize}
\item $T = (\trstep{()}{{\sf stack/done}\,x}; \trstep{()}{{\sf
      string/done}\,y})$. Immediate, letting $T' = \emptytrace$.
\item $T = (\trstep{x_1,x_2}{{\sf stack/left}\,x}; T'')$. By the
  induction hypothesis (where the new frame incorporates
  $x_1{:}\istrue{\susp{\sf L}}$), we have $T'' = (T''';
  \trstep{()}{{\sf stack/done}\,x}; \trstep{()}{{\sf
      string/done}\,y})$. Let $T' = (\trstep{x_1,x_2}{{\sf
      stack/left}\,x}; T''')$.
\item $T = (\trstep{y_1,y_2}{{\sf string/left}\,y}; T'')$. By the
  induction hypothesis (where the new frame incorporates
  $y_1{:}\istrue{\susp{\sf L}}$), we have $T'' = (T''';
  \trstep{()}{{\sf stack/done}\,x}; \trstep{()}{{\sf
      string/done}\,y})$. Let $T' = (\trstep{x_1,x_2}{{\sf
      string/left}\,y}; T''')$.
\item $T = (\trstep{y_1,y_2}{{\sf string/right}\,y}; T'')$. By the
  induction hypothesis (where the new frame incorporates
  $y_2{:}\istrue{\susp{\sf R}}$), we have $T'' = (T''';
  \trstep{()}{{\sf stack/done}\,x}; \trstep{()}{{\sf
      string/done}\,y})$. Let $T' = (\trstep{x_1,x_2}{{\sf
      string/right}\,y}; T''')$.
\item $T = (S; T'')$, where $x$ and $y$ are not free in $S$. 
      By the induction hypothesis, we have
      $T'' = (T''';
       \trstep{()}{{\sf stack/done}\,x}; \trstep{()}{{\sf
       string/done}\,y})$. Let $T' = (S; T''')$. (This case will not arise
      in the way we use this lemma, but the statement of the theorem 
      leaves open the possibility that there are other nonterminals
      in $\Theta$.)
\end{itemize}
This completes the proof. 
\end{proof}

A corollary of this lemma is that if 
$T :: (g{:}\istrue{\susp{\sf gen}}) \leadsto^*_{\Sigma_{\it Gen}} \Delta$ and
  $\restrictsig{\Delta}{\Sigma_{\it PDA}}$, then $T = (T';
  \trstep{()}{{\sf stack/done}\,x}; \trstep{()}{{\sf
      string/done}\,y})$ -- modulo concurrent equality, naturally -- 
where $T' ::
  (g{:}\istrue{\susp{\sf gen}}) \leadsto^*_{\Sigma_{\it Gen}}
  \tackon{\Theta}{x'{:}\istrue{\susp{\sf gen\_stack}},
    h{:}\istrue{\susp{\sf hd}}, y'{:}\istrue{\susp{\sf gen\_string}}}$
  and $\Delta = \tackon{\Theta}{h{:}\istrue{\susp{\sf hd}}}$. To prove
the corollary, we observe
that $T = (\trstep{x,h,r}{{\sf state}\,g\,}; T'')$ and apply the lemma
to $T''$. 

\bigskip
\begin{lemma} The following all hold:
\begin{itemize}
\item If $T :: (g{:}\istrue{\susp{\sf gen}}) \leadsto^*_{\Sigma_{\it Gen}} 
       \tackon{\Theta}{x_1{:}\istrue{\susp{\sf L}}, 
          x_2{:}\istrue{\susp{\sf gen\_stack}}}$, \\then 
$T = (T'; \trstep{x_1,x_2}{{\sf stack/left}\,x'})$ for some $x'$.
\item If $T :: (g{:}\istrue{\susp{\sf gen}}) \leadsto^*_{\Sigma_{\it Gen}} 
       \tackon{\Theta}{y_1{:}\istrue{\susp{\sf gen\_string}},
           y_2{:}\istrue{\susp{\sf L}}}$, \\then 
$T = (T'; \trstep{y_1,y_2}{{\sf string/left}\,y'})$ for some $y'$.
\item If $T :: (g{:}\istrue{\susp{\sf gen}}) \leadsto^*_{\Sigma_{\it Gen}} 
       \tackon{\Theta}{y_1{:}\istrue{\susp{\sf gen\_string}},
           y_2{:}\istrue{\susp{\sf R}}}$, \\then 
$T = (T'; \trstep{y_1,y_2}{{\sf string/right}\,y'})$ for some $y'$.
\end{itemize}
To give the last of the three statements as a picture:
\begin{center}
\begin{tikzpicture} 
\draw [->,decorate, 
decoration={snake,amplitude=.3mm,segment length=3mm,post length=1mm}] 
(-4.7,2) -- (-4.7,0); 
\draw (-4.55,0) node{$_*$};
\draw (-4.7,1.65) node[left]{$T$};
\draw (-4.7,2.3) node{$g{:}\istrue{\susp{\sf gen}}$};
\draw (-4.7,-.3) node{$\tackon{\Theta'}{y_1{:}\istrue{\susp{\sf gen\_string}},
           y_2{:}\istrue{\susp{\sf R}}}$};
%
\draw (-4.2,1.65) node{$=$};
%
\draw [->,densely dotted,decorate, 
decoration={snake,amplitude=.3mm,segment length=3mm,post length=1mm}] 
(2.2,2) -- (2.2,1.3); 
\draw (2.35,1.3) node{$_*$};
\draw (2.2,1.65) node[left]{$T'$};
\draw [->,densely dotted,decorate, 
decoration={snake,amplitude=.3mm,segment length=3mm,post length=1mm}] 
(2.2,.7) -- (2.2,0); 
\draw (2.2,2.3) node{$g{:}\istrue{\susp{\sf gen}}$};
\draw (2.2,.35) node[left]{$\trstep{y_1,y_2}{{\sf string/right}\,y'}$};
\draw (2.2,1) node{$\tackon{\Theta'}{y'{:}\istrue{\susp{\sf gen\_string}}}$};
\draw (2.2,-.3) node{$\tackon{\Theta'}{y_1{:}\istrue{\susp{\sf gen\_string}},
           y_2{:}\istrue{\susp{\sf R}}}$};
\end{tikzpicture} 
\end{center}
\end{lemma}
\begin{proof}
The proofs are all by induction on the structure of $T$ and case
analysis on the last steps in $T$; we will prove the last statement, as
the other two are similar. Up to concurrent equality, there are two
possibilities:
\begin{itemize}
\item $T = (T'; \trstep{y_1, y_2}{{\sf string/right}\,y'})$ -- Immediate.
\item $T = (T''; S)$, where $y_1$ and $y_2$ are not among the output variables $S{^\bullet}$. By the 
induction hypothesis, $T'' = (T'''; \trstep{y_1,y_2}{{\sf string/right}\,y'})$.
Let $T' = (T''; S)$. 
\end{itemize}
This completes the proof. 
\end{proof}

Note that we do not consider any cases where 
$T = (T'; \trstep{y_1',y_2}{{\sf string/right}\,y'})$ (for $y_1 \neq y_1'$),
$T = (T'; \trstep{y_1',y_2}{{\sf string/right}\,y'})$ (for $y_2 \neq y_2'$),
or (critically) where 
$T = (T'; \trstep{y_1,y_2'}{{\sf string/left}\,y'})$. There is no 
way for any of these traces to have the correct type, which makes
the resulting case analysis quite simple. 

\begin{proof}[Proof of Theorem~\ref{thm:pda-preservation} (Preservation)]
By case analysis on the structure of $S$. 

\bigskip
\noindent
{\bf Case 1:} $S = \trstep{x',h'}{{\sf push}\,(\tfuser{h}{y})}$,
which means that we are given the following 
generative trace in $\Sigma_{\it Gen}$:
\begin{align*}
& \qquad ({g{:}\istrue{\susp{\sf gen}}})\\
& T\\
& \qquad \tackon{\Theta}{h{:}\istrue{\susp{\sf hd}}, ~~
                   y{:}\istrue{\susp{\sf L}}}
\intertext{and we must construct a trace 
$({g{:}\istrue{\susp{\sf gen}}}) \leadsto^*_{\Sigma_{\it Gen}} 
\tackon{\Theta}{x'{:}\istrue{\susp{\sf L}},
                   h'{:}\istrue{\susp{\sf hd}}}$. Changing
$h$ to $h'$ is just renaming a bound variable, so we have}
& \qquad 
({g{:}\istrue{\susp{\sf gen}}})\\
& T'\\
& \qquad \tackon{\Theta}{h'{:}\istrue{\susp{\sf hd}}, ~~
                   y{:}\istrue{\susp{\sf L}}}
\intertext{
The corollary 
to the first inversion lemma above on $T'$ gives us}
T' = & \qquad 
({g{:}\istrue{\susp{\sf gen}}})\\
& T'';\\
& \qquad \tackon{\Theta}{
                   x_g{:}\istrue{\susp{\sf gen\_stack}}, ~~
                   h'{:}\istrue{\susp{\sf hd}}, ~~
                   y_g{:}\istrue{\susp{\sf gen\_string}}, ~~
                   y{:}\istrue{\susp{\sf L}}}\\
& \trstep{()}{{\sf stack/done}\,x_g};\\
& \trstep{()}{{\sf string/done}\,y_g}\\
& \qquad \tackon{\Theta}{h'{:}\istrue{\susp{\sf hd}}, ~~
                   y{:}\istrue{\susp{\sf L}}}
\intertext{The second inversion lemma (second part) on $T''$ gives us}
T'' = & \qquad 
({g{:}\istrue{\susp{\sf gen}}})\\
& T''';\\
& \qquad \tackon{\Theta}{
                   x_g{:}\istrue{\susp{\sf gen\_stack}}, ~~
                   h'{:}\istrue{\susp{\sf hd}}, ~~
                   y_g'{:}\istrue{\susp{\sf gen\_string}}}\\
& \trstep{y_g,y}{{\sf string/left}\,y_g'};\\
& \qquad \tackon{\Theta}{
                   x_g{:}\istrue{\susp{\sf gen\_stack}}, ~~
                   h'{:}\istrue{\susp{\sf hd}}, ~~
                   y_g{:}\istrue{\susp{\sf gen\_string}}, ~~
                   y{:}\istrue{\susp{\sf L}}}\\
\intertext{Now, we can construct the trace we need using $T'''$:}
& \qquad 
({g{:}\istrue{\susp{\sf gen}}})\\
& T''';\\
& \qquad \tackon{\Theta}{
                   x_g{:}\istrue{\susp{\sf gen\_stack}}, ~~
                   h'{:}\istrue{\susp{\sf hd}}, ~~
                   y_g'{:}\istrue{\susp{\sf gen\_string}}}\\
& \trstep{x', x_g'}{{\sf stack/left}\,x_g};\\
& \qquad \tackon{\Theta}{
                   x'{:}\istrue{\susp{\sf L}}, ~~
                   x_g'{:}\istrue{\susp{\sf gen\_stack}}, ~~
                   h'{:}\istrue{\susp{\sf hd}}, ~~
                   y_g'{:}\istrue{\susp{\sf gen\_string}}}\\
& \trstep{()}{{\sf stack/done}\,x_g'};\\
& \trstep{()}{{\sf string/done}\,y_g'}\\
& \qquad \tackon{\Theta}{
                   x'{:}\istrue{\susp{\sf L}}, ~~
                   h'{:}\istrue{\susp{\sf hd}}}
\end{align*}

\bigskip
\noindent
{\bf Case 2:} $S = \trstep{h'}{{\sf pop}\,(\tfuser{x}{\tfuser{h}{y}})}$,
which means that we are given the following 
generative trace in $\Sigma_{\it Gen}$:
\begin{align*}
& \qquad ({g{:}\istrue{\susp{\sf gen}}})\\
& T\\
& \qquad \tackon{\Theta}{x{:}\istrue{\susp{\sf L}}, ~~
                   h{:}\istrue{\susp{\sf hd}}, ~~
                   y{:}\istrue{\susp{\sf R}}}
\intertext{and we must construct a trace 
$({g{:}\istrue{\susp{\sf gen}}}) \leadsto^*_{\Sigma_{\it Gen}} 
\tackon{\Theta}{h'{:}\istrue{\susp{\sf hd}}}$. Changing
$h$ to $h'$ is just renaming a bound variable, so we have}
& \qquad ({g{:}\istrue{\susp{\sf gen}}})\\
& T'\\
& \qquad \tackon{\Theta}{x{:}\istrue{\susp{\sf L}}, ~~
                   h'{:}\istrue{\susp{\sf hd}}, ~~
                   y{:}\istrue{\susp{\sf R}}}
\intertext{The corollary to the first inversion lemma above on $T'$ gives us}
T' = & \qquad ({g{:}\istrue{\susp{\sf gen}}})\\
& T'';\\
& \qquad \tackon{\Theta}{x{:}\istrue{\susp{\sf L}}, ~~
                   x_g{:}\istrue{\susp{\sf gen\_stack}}, ~~
                   h'{:}\istrue{\susp{\sf hd}}, ~~
                   y_g{:}\istrue{\susp{\sf gen\_string}}, ~~
                   y{:}\istrue{\susp{\sf R}}}\\
& \trstep{()}{{\sf stack/done}\,x_g};\\
& \trstep{()}{{\sf string/done}\,y_g}\\
& \qquad \tackon{\Theta}{x{:}\istrue{\susp{\sf L}}, ~~
                   h'{:}\istrue{\susp{\sf hd}}, ~~
                   y{:}\istrue{\susp{\sf R}}}
\intertext{The second inversion lemma (first part) on $T''$ gives us}
T'' = & \qquad ({g{:}\istrue{\susp{\sf gen}}})\\
& T''';\\
& \qquad \tackon{\Theta}{x_g'{:}\istrue{\susp{\sf gen\_stack}}, ~~
                   h'{:}\istrue{\susp{\sf hd}}, ~~
                   y_g{:}\istrue{\susp{\sf gen\_string}}, ~~
                   y{:}\istrue{\susp{\sf R}}}\\
& \trstep{x, x_g}{{\sf stack/left}\,x_g'};\\
& \qquad \tackon{\Theta}{x{:}\istrue{\susp{\sf L}}, ~~
                   x_g{:}\istrue{\susp{\sf gen\_stack}}, ~~
                   h'{:}\istrue{\susp{\sf hd}}, ~~
                   y_g{:}\istrue{\susp{\sf gen\_string}}, ~~
                   y{:}\istrue{\susp{\sf R}}}\\
\intertext{The second inversion lemma (third part) on $T'''$ gives us}
T''' = & \qquad ({g{:}\istrue{\susp{\sf gen}}})\\
& T'''';\\
& \qquad \tackon{\Theta}{x_g'{:}\istrue{\susp{\sf gen\_stack}}, ~~
                   h'{:}\istrue{\susp{\sf hd}}, ~~
                   y_g'{:}\istrue{\susp{\sf gen\_string}}}\\
& \trstep{y_g, y}{{\sf string/right}\,y_g'};\\
& \qquad \tackon{\Theta}{x_g'{:}\istrue{\susp{\sf gen\_stack}}, ~~
                   h'{:}\istrue{\susp{\sf hd}}, ~~
                   y_g{:}\istrue{\susp{\sf gen\_string}}, ~~
                   y{:}\istrue{\susp{\sf R}}}\\
\intertext{Now, we can construct the trace we need using $T''''$:}
& \qquad ({g{:}\istrue{\susp{\sf gen}}})\\
& T'''';\\
& \qquad \tackon{\Theta}{x_g'{:}\istrue{\susp{\sf gen\_stack}}, ~~
                   h'{:}\istrue{\susp{\sf hd}}, ~~
                   y_g'{:}\istrue{\susp{\sf gen\_string}}}\\
& \trstep{()}{{\sf stack/done}\,x_g'};\\
& \trstep{()}{{\sf string/done}\,y_g'}\\
& \qquad \tackon{\Theta}{h'{:}\istrue{\susp{\sf hd}}}
\end{align*}

\noindent
These two cases represent the only two synthetic transitions that are possible
under the signature $\Sigma_{\it PDA}$, so we are done.
\end{proof}

Proving that generation under a generative signature like $\Sigma_{\it
  Gen}$ is invariant under transitions in a signature like
$\Sigma_{\it PDA}$ is something we will consider further
in Chapter~\ref{chapter-gen}.  
All such proofs essentially follow the structure of
Theorem~\ref{thm:pda-preservation}. First, we enumerate the synthetic
transitions associated with a given signature. Second, in each of those
cases, we use the type of the synthetic transition to perform
inversion on the structure of the given generative trace.
Third, we construct a generative
trace that establishes the fact that the invariant was preserved.

\subsection{Adequacy of the transition system}
\label{sec:pda-adequacy}

The hard work of adequacy is established by the preservation theorem; 
the actual adequacy theorem is just an enumeration in both directions.

\bigskip
\begin{theorem}[Adequacy]\label{thm:pda-adequacy}
  $\ctxinterp{k \rhd s} \leadsto_{\Sigma_{\it PDA}} \ctxinterp{k'
    \rhd s'}$ if and only if $\obj{k \rhd s} \mapsto \obj{k' \rhd s'}$.
\end{theorem}

\begin{proof}
  Both directions can be established by case analysis on the structure
  of $\obj{k}$ and $\obj{s}$.
\end{proof}

As an immediate corollary of this theorem and preservation
(Theorem~\ref{thm:pda-preservation}), we have the stronger adequacy
property that $\ctxinterp{k \rhd s} \leadsto_{\Sigma_{\it PDA}}
\Delta'$, then $\Delta' = \ctxinterp{k' \rhd s'}$ for some $\obj{k}$
and $\obj{s'}$ such that $\obj{k \rhd s} \mapsto \obj{k' \rhd s'}$.
In our two-dimensional notation, the complete discussion of adequacy
for \sls~is captured by the following picture:

\begin{center}
\begin{tikzpicture} 
\draw (.8,2.3) node{$(x{:}\istrue{\susp{\sf gen}})$};
\draw [->,decorate, 
decoration={snake,amplitude=.3mm,segment length=3mm,post length=1mm}] 
(0.8,2) -- (.8,.8); 
\draw (.9,.8) node{$_*$};
\draw (.3,1.4) node{$\Sigma_{\it Gen}$};
\draw (.8,.5) node{$\Delta$};
\draw [thick,dash pattern = on 2.82842842712mm off 2mm,decorate,decoration={saw,amplitude=2mm,segment length=2mm}] 
(.4,0) -- (1.2,0); 
\draw (.8,-.3) node{$\Delta$};
\draw (1.1,-.7) node{\begin{turn}{-45}$\Leftrightarrow$\end{turn}};
\draw (1.8,-1) node{$\obj{k \rhd s}$};
%
\draw (4.8,2.3) node{$(x{:}\istrue{\susp{\sf gen}})$};
\draw [->,densely dotted,decorate, 
decoration={snake,amplitude=.3mm,segment length=3mm,post length=1mm}] 
(4.8,2) -- (4.8,.8); 
\draw (4.9,.8) node{$_*$};
\draw (4.3,1.4) node{$\Sigma_{\it Gen}$};
\draw (4.8,.5) node{$\Delta'$};
\draw (2.8, 0) node{$\Sigma_{\it PDA}$};
\draw [thick,dash pattern = on 0.677mm off .4mm on 0.676142375mm off .4mm on 0.676142375mm off 2mm,decorate,decoration={saw,amplitude=2mm,segment length=2mm}] 
(4.4,0) -- (5.2,0); 
\draw (4.8,-.3) node{$\Delta'$};
\draw (5.1,-.7) node{\begin{turn}{-45}$\Leftrightarrow$\end{turn}};
\draw (5.8,-1) node{$\obj{k' \rhd s'}$};
%
\draw [->,decorate, 
decoration={snake,amplitude=.3mm,segment length=3mm,post length=1mm}] 
(1.2,-.3) -- (4.4,-.3); 
\draw [|->] (2.5,-1.04) -- (5,-1.04);;
\end{tikzpicture} 
\end{center}

\section{The \sls~implementation}
\label{sec:prototype}

The prototype implementation of \sls~contains a parser and typechecker
for the SLS language, and is available from
\url{https://github.com/robsimmons/sls}. Code that is checked by this
prototype implementation will appear frequently in the rest of this
thesis, always in a \verb|fixed-width font|.

\begin{figure}
\newcommand{\thingamajig}{=}
\begin{align*}
{\downarrow}A^- & \thingamajig \mbox{\Verb|A|} 
 & {\ocircle}A^+ & \thingamajig \mbox{\Verb|\{A\}|}
 & \lf{\lambda a.t} & \thingamajig \mbox{{\texttt{\char`\\}}\Verb|a.t|}
\\
{\gnab}A^- & \thingamajig \mbox{\Verb|\$A|}
 & A^+ \lefti B^- & \thingamajig \mbox{\Verb|A >-> B|}
 & \lf{{\sf foo}\,t_1\ldots t_n} & \thingamajig \mbox{\Verb|foo t1...tn|}
\\
{!}A^- & \thingamajig \mbox{\Verb|!A|}
 & A^+ \righti B^- & \thingamajig \mbox{\Verb|A ->> B|}
\\
\one & \thingamajig \mbox{\Verb|one|}
 & A^- \with B^- & \thingamajig \mbox{\Verb|A \& B|}
 & \Pi\lf{a}{:}\tau.\nu & \thingamajig \mbox{\Verb|Pi x.nu|}
\\
A^+ \fuse B^+ & \thingamajig \mbox{\Verb|A * B|}
 & \forall \lf{a}.\tau. A^- & \thingamajig \mbox{\Verb|All x.A|}
 & \tau \rightarrow \nu & \thingamajig{\mbox{\Verb|tau -> nu|}}
\\
\exists \lf{a}.\tau. A^+ & \thingamajig \mbox{\Verb|Exists x.A|}
 & {\gnab}A^- \lefti B^- & \thingamajig \mbox{\Verb|A -o B|}
 & {\sf bar}\,\lf{t_1 \ldots t_n} & \thingamajig{\mbox{\Verb|bar t1...tn|}}
\\
\lf{t} \doteq \lf{s} & \thingamajig \mbox{\Verb|t == s|}
 & {!}A^- \lefti B^- & \thingamajig \mbox{\Verb|A -> B|}
\end{align*}
\caption{Mathematical and ASCII representations of propositions,
  terms, and classifiers}
\label{fig:translate-types}
\end{figure}


The checked \sls~code differs slightly from mathematical
\sls~specifications in a few ways -- the translation between the
mathematical notation we use for \sls~propositions and the ASCII
representation used in the implementation is outlined in
Figure~\ref{fig:translate-types}.  Following CLF and the Celf
implementation, we write the lax modality ${\ocircle}A$ in ASCII as
\verb|{A}| -- recall that in Section~\ref{sec:slsframework} we
introduced the $\{ A^+ \}$ notation from CLF as a synonym for
Fairtlough and Mendler's ${\ocircle}A^+$.  The exponential ${\gnab}A$
doesn't have an ASCII representation, so we write \verb|$A| when $A$
is mobile. Upshifts and downshifts are always inferred: this means
that we can't write down ${\uparrow}{\downarrow}A$ or
${\downarrow}{\uparrow}A$, but neither of these \ollll~propositions
are part of the \sls~fragment anyway.

The \sls~implementation also supports conventional abbreviations for
arrows that we won't use in mathematical notation: ${\gnab}A^- \lefti
B^-$ can be written as \verb|A -o B| or \verb|$A >-> B| in the
\sls~implementation, and ${!}A^- \lefti B^-$ can be written as
\verb|A -> B| or \verb|!A >-> B|.  This final proposition is
ambiguous, because \verb|X -> Y| can be an abbreviation for ${!}X
\lefti Y$ or $\Pi \lf{a}{:}X. Y$, but \sls~can figure out whether the
proposition or classifier was intended by analyzing the structure of
\verb|Y|. Also note that we could have just as easily made
\verb|A -o B| an abbreviation for \verb|$A -o B|, but we had to pick
one and the choice absolutely doesn't matter.  All arrows can also be
written backwards: \verb|B <-< A| is equivalent to \verb|A >-> B|,
\verb|B o- A| is equivalent to \verb|A -o B|, and so on. Also
following traditional conventions, upper-case variables that are free
in a rule will be treated as implicitly quantified. Therefore, the
line \bigskip
\begin{verbatim}
rule: foo X <- (bar Y -> baz Z).
\end{verbatim}
\bigskip
will be reconstructed as the \sls~declaration 
\[{\sf rule} : \forall\lf{Y}{:}\tau_1.\,\forall\lf{Z}{:}\tau_2.\,\forall\lf{X}{:}\tau_3.\,{!}({!}{\sf bar}\,\lf{Y} \lefti {\sf baz}\,\lf{Z}) \lefti {\sf foo}\,\lf{X}\] 
%
where the implementation infers the types $\tau_1$, $\tau_2$, and
$\tau_3$ appropriately from the declarations of the negative
predicates ${\sf foo}$, ${\sf bar}$, and ${\sf baz}$.

Another significant piece of syntactic sugar introduced for 
the sake of readability is less conventional, if only because
positive atomic propositions are not conventional. If \verb|P| is a
persistent atomic proposition, we can optionally write \verb|!P|
wherever \verb|P| is expected, and if \verb|P| is a linear atomic
proposition, we can write \verb|$P| wherever \verb|P| is
expected. This means that if ${\sf a}$, ${\sf b}$, and ${\sf c}$ are
(respectively) ordered, linear, and persistent positive atomic
propositions, we can write the positive proposition ${\sf a} \fuse
{\sf b} \fuse {\sf c}$ in the \sls~implementation as
\verb|(a * b * c)|, \verb|(a * $b * c)|, \verb|(a * b * !c)|, or
\verb|(a * $b * !c)|. Without these annotations, it is difficult to
tell at a glance which propositions are ordered, linear, or persistent
when a signature uses more than one proposition. When all of these
optional annotations are included, the rules in a signature that uses
positive atomic propositions look the same as rules in a signature
that uses the pseudo-positive negative atomic propositions described
in Section~\ref{sec:pseudopositive}. 


In the code examples given in the remainder of this thesis, we will
use these optional annotations in a consistent way.  We will omit the
optional \verb|$A| annotations only in specifications with no ordered
atomic propositions, and we will omit the optional \verb|!A|
annotations in specifications with no ordered or linear atomic
propositions. This makes the mixture of different exponentials
explicit while avoiding the need for rules like
\verb|($a * $b * $c >-> {$d * $e})| when specifications are entirely
linear (and likewise when specifications are entirely persistent).

\section{Logic programming}
\label{sec:framework-logicprog}

One logic programming interpretation of CLF was explored by the
Lollimon implementation \cite{lopez05monadic} and adapted by the Celf
implementation
\cite{schacknielsen08celf,schacknielsen11implementing}. Logic
programming interpretations of \sls~are not a focus this thesis, but
we will touch on a few points in this section.

Logic programming is important because it provides us with operational
intuitions about the intended behavior of the systems we specify in
\sls. One specific set of these intuitions will form the basis of the
operationalization transformations on \sls~specifications considered
in Chapter~\ref{chapter-absmachine}. 
Additionally, logic programming intuitions are relevant
because they motivated the design of \sls, in particular the
presentation of the concurrent fragment in terms of partial, rather
than complete, proofs. We discuss this point in
Section~\ref{sec:framework-logicprog-trace}.

\subsection{Deductive computation and backward chaining}
\label{sec:framework-logicprog-deductive}
\label{sec:framework-modes}

Deductive computation in \sls~is the search for {\it complete} proofs
of sequents of the form $\foc{\Psi}{\Delta}{\istrue{\susp{p^-}}}$.  A
common form of deductive computation is {\it goal-directed search}, or
what Andreoli calls the {\it proof construction paradigm}
\cite{andreoli01focussing}.
In \sls, goal-directed search for the proof of a sequent
$\foc{\Psi}{\Delta}{\istrue{\susp{p^-}}}$ can only proceed by focusing
on a proposition like ${\downarrow}p_n^- \lefti \ldots \lefti
{\downarrow}p_1^- \lefti p^-$ which has a head $p^-$ that matches the
succedent. This replaces the goal sequent
$\foc{\Psi}{\Delta}{\istrue{\susp{p^-}}}$ with $n$ subgoals:
$\foc{\Psi}{\Delta_1}{\istrue{\susp{p_1^-}}}$ \ldots
$\foc{\Psi}{\Delta_n}{\istrue{\susp{p_n^-}}}$, where $\Delta$ matches
$\Delta_1,\ldots,\Delta_n$.

When goal-directed search only deals with the unproved subgoals of a
single coherent derivation at a time, it is called {\it backward
  chaining}, because we're working backwards from the goal we want to
prove.\footnote{The alternative is to try and derive the same sequent
  in multiple ways simultaneously, succeeding whenever some way of
  proving the sequent is discovered. Unlike backward chaining, this
  strategy of breadth-first search is complete: if a proof exists, it
  will be found.  Backward chaining as we define it is only
  nondeterministically or partially complete, because it can fail to
  terminate when a proof exists. We will call this alternative to
  backtracking {\it breadth-first theorem proving}, as it amounts to
  taking a breadth-first, instead of depth-first, view of the
  so-called {\it failure continuation} \cite{pfenning06backtracking}.}
The term {\it top-down logic programming} is also used, and refers to
the fact that, in the concrete syntax of Prolog, the rule
${\downarrow}p_n^- \lefti \ldots \lefti {\downarrow}p_1^- \lefti p^-$
would be written with $p^-$ on the first line, $p_1^-$ on the second,
etc. This is exactly backwards from a proof-construction perspective,
as we think of backward chaining as building partial proofs from the
bottom up, the root towards the leaves, so we will avoid this
terminology.

The backward-chaining interpretation of intuitionistic logics dates
back to the work by Miller et al.~on uniform proofs
\cite{miller91uniform}.  An even older concept, Clark's {\it
  negation-as-failure} \cite{clark87negation}, is based on a {\it
  partial completeness} criteria for logic programming interpreters.
Partial correctness demands that if the
interpreter reports that it has found a proof of a goal-directed
sequent, such a proof should exist. Partial completeness, on the other
hand, demands that if the interpreter gives up up on finding a proof,
no proof should exist. (The interpreter is also allowed to run forever
without succeeding or giving up.)  Partial completeness requires {\it
  backtracking} in backward-chaining search: if we we try to prove
$\foc{\Psi}{\Delta}{\istrue{\susp{p^-}}}$ by focusing on a particular
proposition and one of the resulting subgoals fails to be provable, we
have to consider any other propositions that could have been used to
prove the sequent before giving up. Backtracking can be extremely
powerful in certain cases and incredibly expensive in others, and so
most logic programming languages have an escape hatch that modifies or
limits backtracking at the user's discretion, such as the Prolog cut
(no relation to the admissible rule ${\it cut}$) or Twelf's
deterministic declarations. Non-backtracking goal-oriented deductive
computation is called {\it flat resolution} \cite{aitkaci99warrens}.

One feature of backward chaining and goal directed search is that it
usually allows for terms that are not completely specified -- these
unspecified pieces are are traditionally called {\it logic
  variables}. Because LF variables are also ``logic variables,''
the literature on $\lambda$Prolog and Twelf calls unspecified
pieces of terms {\it existential variables}, 
but as they bear no relation to the variables introduced
by the left rule for $\exists \lf{a}{:}\tau. A^+$, 
that terminology is also unhelpful
here. Consider the following \sls~signature:
\begin{align*}
 \Sigma_{\it Add} = \cdot, 
~&{\sf nat} : {\sf type}, 
~~\lf{\sf z} : {\sf nat}, 
~~\lf{\sf s} : {\sf nat} \rightarrow {\sf nat},\\
~&{\sf plus} : {\sf nat} \rightarrow {\sf nat} \rightarrow {\sf nat} 
                 \rightarrow {\sf prop},\\
~&{\sf plus/z} : \forall \lf{N}{:}{\sf nat}.\,
({\sf plus}\,\lf{\sf z}\,\lf{N}\,\lf{N}),\\
~&{\sf plus/s} : \forall \lf{N}{:}{\sf nat}.\, 
                 \forall \lf{M}{:}{\sf nat}.\, 
                 \forall \lf{P}{:}{\sf nat}.\,
{!}({\sf plus}\,\lf{N}\,\lf{M}\,\lf{P})
\lefti ({\sf plus}\,({\sf s}\,\lf{N})\,\lf{M}\,({\sf s}\,\lf{P}))
\end{align*}
In addition to searching for a proof of ${\sf plus}\,\lf{\sf
  (s\,z)}\,\lf{\sf (s\,z)}\,\lf{\sf (s\,(s\,z))}$ (which will succeed,
as $1 + 1 = 2$) or searching for a proof of ${\sf plus}\,\lf{\sf
  (s\,z)}\,\lf{\sf (s\,z)}\,\lf{\sf (s\,(s\,(s\,z)))}$ (which will
fail, as $1 + 1 \neq 3$), we can use goal-oriented deductive
computation to search for ${\sf plus}\,\lf{\sf (s\,z)}\,\lf{\sf
  (s\,z)}\,X$, where $X$ represents an initially unspecified term.
This search will succeed, reporting that $X = \lf{\sf
  (s\,(s\,z))}$. Unification is generally used in backward-chaining
logic programming languages as a technique for implementing partially
unspecified terms, but this implementation technique should not be
confused with our use of unification-based equality $\lf{t} \doteq
\lf{s}$ as a proposition in \sls.

We say that ${\sf plus}$ in the signature above is a {\it well-moded}
predicate with {\it mode} $({\sf plus}\,{+}\,{+}\,{-})$, because
whenever we perform deductive computation to derive $({\sf
  plus}\,\lf{n}\,\lf{m}\,\lf{p})$ where $\lf{n}$ and $\lf{m}$ are
fully specified, any unspecified portion of $\lf{p}$ must be fully
specified in any completed derivation. Well-moded predicates can be
treated as nondeterministic partial functions from their inputs (the
indices marked ``${+}$'' in the mode) to their outputs (the indices
marked ``${-}$'' in the mode). A predicate can sometimes be given more
than one mode: $({\sf plus}\,{+}\,{-}\,{+})$ is a valid mode for ${\sf
  plus}$, but $({\sf plus}\,{+}\,{-}\,{-})$ is not.

The implementation of backward chaining in substructural logic has
been explored by Hodas \cite{hodas94logic}, Polakow
\cite{polakow00linear,polakow01ordered}, Armel\'in and Pym
\cite{armelin01bunched}, and others. Efficient implementation of these
languages is complicated by the problem of {\it resource
  management}. In linear logic proof search, it would be technically
correct but highly inefficient to perform proof search by enumerating
the ways that a context can be split and then backtracking over each
possible split. Resource management allows the interpreter to avoid
this potentially exponential backtracking, but describing resource
management and proving it correct, especially for richer substructural
logics, can be complex and subtle \cite{cervesato00efficient}.

The term {\it deductive computation} is meant to be interpreted very
broadly, and goal-directed search is not the only form of deductive
computation. Another paradigm for deductive computation is the {\it
  inverse method}, where the interpreter attempts to prove a sequent
$\foc{\Psi}{\Delta}{\istrue{\susp{p^-}}}$ by creating and growing
database of sequents that are derivable, attempting to build the
appropriate derivation from the leaves down. The inverse method is
generally associated with theorem proving and not logic
programming. However, Chaudhuri, Pfenning, and Price have shown that
that deductive computation with the inverse method in a focused linear
logic can simulate both backward chaining and forward chaining
(considered below) for persistent Horn-clause logic programs
\cite{chaudhuri10logical}. 

\begin{figure}
\begin{tikzpicture}
\draw (0,10) node{~};
\draw (8,10.45) node{\bf Deductive computation};
\draw (8,10) node{Search for complete derivations $\foc{\Psi}{\Delta}{U}$};
\draw [->] (7.5,9.5) -- (6.5,8.5); 
\draw (7.2,9.3) node[left]{\it maintains sets of};
\draw (6.8,8.9) node[left]{\it subgoal sequents};
\draw [->] (8.5,9.5) -- (9.5,8.5); 
\draw (8.9,9.3) node[right]{\it maintains sets of};
\draw (9.2,8.9) node[right]{\it derivable sequents};
%
\draw (6,8) node{\bf goal-directed search};
\draw [->] (5.5,7.5) -- (4.5,6.5); 
\draw (5,7.1) node[left]{\it depth-first};
\draw [->] (6.5,7.5) -- (7.5,6.5); 
\draw (7,7.1) node[right]{\it breadth-first};
%
\draw (4,6) node{\bf backward chaining};
\draw [->] (3.5,5.5) -- (2.5,4.5); 
\draw (3,5.1) node[left]{\it backtracking};
\draw [->] (4.5,5.5) -- (5.5,4.5); 
\draw (5,5.1) node[right]{\it committed-choice};
%
\draw (2,4) node{\bf backward chaining};
%
\draw (6,4) node{\bf flat resolution};
%
\draw (9.6,6) node{\bf breadth-first theorem proving};
%
\draw (11.6,8) node{\bf inverse method theorem proving};
%
% \draw (13,10.45) node{\bf Trace computation};
% \draw (13,10) node{Search for partial proofs 
%    $(\Psi; \Delta) \leadsto^* (\Psi'; \Delta')$};
% \draw [->] (12.5,9.5) -- (11.5,8.5); 
% \draw [->] (13.5,9.5) -- (14.5,8.5); 
%
\end{tikzpicture}
\caption{A rough taxonomy of deductive computation}
\label{fig:computation-taxonomy}
\end{figure}

Figure~\ref{fig:computation-taxonomy} gives an taxonomy (incomplete
and imperfect) of the forms of deductive computation mentioned in this
section. Note that, while we will generally use {\it backward
  chaining} to describe backtracking search, backward chaining does
not always imply full backtracking and partial completeness. This
illustration, and the preceding discussion, leaves out many important
categories, especially tabled logic programming, and many potentially
relevant implementation choices, such as breath-first versus
depth-first or parallel exploration of the success continuation.


\subsection{Concurrent computation}
\label{sec:framework-logicprog-trace}


Concurrent computation is the search for {\it partial} proofs of
sequents. As the name suggests, in \sls~concurrent computation is
associated with the search for partial proofs of the judgment
$\islax{A^+}$, which correspond to traces $(\Psi;
\Delta) \leadsto^* (\Psi'; \Delta')$. 

The paradigm we will primarily associate with concurrent computation
is {\it forward chaining}, which implies that we take an initial
process state $(\Psi;\Delta)$ and allow it to evolve freely by the
application of synthetic transitions. Additional conditions can be
imposed on forward chaining: for instance, synthetic transitions like
$(\Delta, x{:}\ispers{\susp{p^+_\mpers}}) \leadsto (\Delta,
x{:}\ispers{\susp{p^+_\mpers}}, y{:}\ispers{\susp{p^+_\mpers}})$ that
do not meaningfully change the state can be excluded (if a persistent
proposition already exists, two copies of that proposition don't add
anything).\footnote{Incidentally, Lollimon implements this restriction
  and Celf does not.} Forward chaining with this restriction in a
purely-persistent logic is strongly associated with the Datalog
language and its implementations; we will refer to forward chaining in
persistent logics as {\it saturating logic programming} in 
Chapter~\ref{chapter-approx}.
Forward chaining does not always deal with partially-unspecified
terms; when persistent logic programming languages support forward
chaining with partially-unspecified terms variables, it is called {\it
  hyperresolution} \cite{fermuller01resolution}.

The presence of ephemeral or ordered resources in substructural logic
means that a process state may evolve in multiple
mutually-incompatible ways. {\it Committed choice} is a version of
forward chaining that never goes back and reconsiders alternative
evolutions from the initial state. Just as the default interpretation
of backward chaining includes backtracking, we will consider the
default interpretation of forward chaining to be committed choice,
following \cite{lopez05monadic}.  An alternate interpretation of
forward chaining would consider multiple evolutionary paths, which is
a version of {\it exhaustive search}.  Trace computation that works backwards
from a final state instead of forward from an initial state can also
be considered, and {\it planning} can be seen as specifying both the
initial and final process states and trying to extrapolate a trace
between them by working in both directions.

Outside of this work and Saurin's work on Ludics programming
\cite{saurin08towards}, there is not much work on explicitly
characterizing and searching for partial proofs in substructural
logics.\footnote{As such, ``concurrent computation,'' while
  appropriate for \sls, may or may not prove to be a good name for the
  general paradigm.}  Other forms of computation can be characterized
as trace computation, however.  Multiset rewriting and languages like
GAMMA can be partially or completely understood in terms of forward
chaining in linear logic \cite{cervesato09relating,paola96linear}, and
the ordered aspects of \sls~allow it to capture fragments of rewriting
logic. Rewriting logic, and in particular the Maude implementation of
rewriting logic \cite{clavel11ltl}, implements the committed choice
interpretation and exhaustive search interpretations, as well as a
{\it model checking} interpretation that characterize sets of process
states or traces using logical formulas. Constraint handling
rules \cite{betz10complete} and concurrent constraint programming
\cite{jagadeesan05testing} are other logic programming models can be 
characterized as forms of concurrent computation.


% not consider unspecified variables and is largely free of the resource
% management problems that appear in backward-chaining deductive
% computation for substructural logics. In previous work, we designed a
% forward chaining interpreter for linear logic that admits abstract
% reasoning about the asymptotic complexity of logical specifications
% \cite{simmons08linear}, but this is outside the scope of this thesis.



\subsection{Integrating deductive and trace computation}

In the logic programming interpretation of CLF used by Lollimon and
Celf, backtracking backward chaining is associated with the deductive
fragment, and committed-choice forward chaining is associated with the
lax modality. We will refer to an adaptation of the Lollimon/Celf
semantics to \sls~as LCI (``Lollimon/Celf Interpreter'') for
brevity in this section.

Forward chaining and backward chaining have an uneasy relationship in
LCI. To see why, consider the following \sls~signature:
\begin{align*}
 \Sigma_{\it Demo} = \cdot, 
~&{\sf posA} : {\sf prop}\,{\sf ord}, 
~~{\sf posB} : {\sf prop}\,{\sf ord}, 
~~{\sf posC} : {\sf prop}\,{\sf ord}, 
~~{\sf negD} : {\sf prop},\\
~&{\sf fwdruleAB} : {\sf posA} \lefti {\ocircle}{\sf posB},\\
~&{\sf fwdruleAC} : {\sf posA} \lefti {\ocircle}{\sf posC},\\
~&{\sf bwdrule} : ({\sf posA} \lefti {\ocircle}{\sf posB}) \lefti {\sf negD}
\end{align*}

In an empty context, there is only one derivation of ${\sf negD}$
under this signature: it is represented by the proof term ${\sf
  bwdrule}\,(\lambda x.\,\tlet{\trstep{y}{{\sf
      fwdruleAB}\,x}}{y})$. The partially complete interpretation of
backward chaining stipulates that an interpreter tasked with finding a
proof of ${\sf negD}$ should either find this proof or never
terminate, but LCI only admits this interpretation for purely
deductive proofs. To see why, consider backward-chaining search
attempting to prove ${\sf negD}$ in a closed context.  This can only
be done with the rule ${\sf bwdrule}$, generating the subgoal ${\sf
  posA} \lefti {\ocircle}{\sf posB}$.  At this point, LCI will switch
from backward chaining to forward chaining and attempt to satisfy this
subgoal by constructing a trace $(x{:}\istrue{\susp{\sf posA}})
\leadsto (y{:}\istrue{\susp{\sf posB}})$.

There are {\it two} nontrivial traces in this signature starting from
the process state $(x{:}\istrue{\susp{\sf posA}})$ -- the first is
$(\trstep{y}{{\sf fwdruleAB}\,x}) :: (x{:}\istrue{\susp{\sf posA}})
\leadsto (y{:}\istrue{\susp{\sf posB}})$, and the second is
$(\trstep{y}{{\sf fwdruleAC}\,x})::(x{:}\istrue{\susp{\sf posA}})
\leadsto (y{:}\istrue{\susp{\sf posC}})$.  Forward chaining can
plausibly come up with either one, and if it happens to derive the
second one, the subgoal fails. LCI then tries to backtrack to find
other rules that can prove the conclusion ${\sf negD}$, but there are
none, so LCI will report that it failed to prove ${\sf negD}$.

This example indicates that it is difficult to make backward chaining
(in its default backtracking form) reliant on committed-choice forward
chaining (in its default committed-choice form) in the style of
Lollimon or Celf. Either we can restrict forward chaining to confluent
systems (excluding $\Sigma_{\it Demo}$) or else we can give up on the
usual partially complete interpretation of backward chaining.  In the
other direction, however, it is entirely natural to make forward
chaining dependent upon backward chaining. The fragment of CLF that
encodes this kind of computation was labeled the {\it semantic
  effects} fragment by DeYoung \cite{deyoung09reasoning}. At the
logical level, the semantic effects fragment of \sls~removes the right
rule for ${\ocircle}A^+$, which corresponds to the proof term
$\tlet{T}{V}$.  As discussed in
Section~\ref{sec:framework-concurrent}, let-expressions are the only point
where traces are included into the language of
deductive terms.

\section{Design decisions}
\label{sec:designdecisions}

Aside from ordered propositions, there are several significant
differences between the framework \sls~presented in this chapter and
the existing logical framework CLF, including the presence of positive
atomic propositions, the introduction of traces as an explicit
notation for partial proofs, the restriction of the term language to
LF, and the presence of equality $\lf{t} \doteq \lf{s}$ as a
proposition. In this section, we will discuss design choices that were
made in terms of each of these features, their effects, and what
choices could have been made differently.

\subsection{Pseudo-positive atoms}
\label{sec:pseudopositive}

Unlike \sls, the CLF framework does not include positive atomic
propositions. Positive atomic propositions make it easy to
characterize the synthetic transitions associated with a particular
rule. For example, if ${\sf foo}$, ${\sf bar}$, and ${\sf baz}$ are
all linear atomic propositions, then the presence of a rule ${\sf
  somerule} : \left({\sf foo} \fuse {\sf bar} \lefti {\ocircle}{\sf
    baz}\right)$ in the signature is associated with synthetic
transitions of the form
%
$(\Psi; \matchconj{\Delta}{\matchconj{x{:}\iseph{\susp{\sf
        foo}}}{y{:}\iseph{\susp{\sf bar}}}})
 \leadsto
 (\Psi; \mkconj{\Delta}{z{:}\iseph{\susp{\sf baz}}})$.
%
The presence of the
rule $\sf somerule$ enables steps of this form, and every step made by
focusing on the rule has this form.

CLF has no positive propositions, so the closest analogue that we can
consider is where ${\sf foo}$, ${\sf bar}$, and ${\sf baz}$ are
negative propositions, and the rule ${\gnab}{\sf foo} \fuse
{\gnab}{\sf bar} \lefti \ocircle({\gnab}{\sf baz})$ appears in the
signature. Such a rule is associated with synthetic transitions of the
form
%
$(\Psi; \matchconj{\Delta}{\matchconj{\Delta_1}{\Delta_2}}) \leadsto
(\Psi; \mkconj{\Delta}{z{:}\istrue{{\sf baz}}})$ such that
$\foc{\Psi}{\restrictto{\Delta_1}{\meph}}{\istrue{\susp{\sf foo}}}$
and $\foc{\Psi}{\restrictto{\Delta_2}{\meph}}{\istrue{\susp{\sf
      bar}}}$. In \sls, it is a relatively simple syntactic criterion
to enforce that a sequent like $\foc{\Psi}{\Delta_1}{\istrue{\susp{\sf
      foo}}}$ can only be derived if $\Delta_1$ matches
$x{:}\sf{foo}$; we must simply ensure that there are no propositions
of the form $\ldots \lefti {\sf foo}$ or $\ldots \righti {\sf foo}$ in
the signature or context. (In fact, this is essentially the
\sls~version of the subordination criteria that allows us to conclude
that an LF type was only inhabited by variables in 
Section~\ref{sec:slsframework}.)  Note that, in full \ollll, this task
would not be so easy: we might prove $\istrue{\susp{\sf foo}}$
indirectly by forward chaining. This is one reason why association of
traces with the lax modality is so important!

When it is the case that  $\foc{\Psi}{\Delta_1}{\istrue{\susp{\sf
      foo}}}$ can only be derived if $\Delta_1$ matches
$x{:}\sf{foo}$, we can
associate the rule ${\gnab}{\sf foo} \fuse
{\gnab}{\sf bar} \lefti \ocircle({\downarrow}({\gnab}{\sf baz}))$
with the synthetic transition $(\Psi;
\matchconj{\Delta}{\matchconj{x{:}{\islvl{\sf foo}}}{y{:}{\islvl{\sf
        bar}'}}}) \leadsto (\Psi; \mkconj{\Delta}{z{:}\iseph{\susp{\sf
      baz}}})$ under the condition that neither $\mlvl$ or $\mlvl'$
are $\mtrue$.
Negative atomic propositions that can only be concluded when they are
the sole member of the context, like ${\sf foo}$ and ${\sf bar}$ in
this example, can be called {\it
  pseudo-positive}. Pseudo-positive atoms can actually be used a bit
more generally than \sls's positive atomic propositions. A positive
atomic proposition is necessarily associated with one of the three
judgments $\mtrue$, $\meph$, or $\mpers$, but pseudo-positive
propositions can associate with any of the contexts. This,
incidentally, gives pseudo-positive atoms in CLF or \sls~the flavor of
positive atomic propositions under Andreoli's atom optimization
(Section~\ref{sec:atomopt}).

It is, of course, possible to consistently associate particular
pseudo-positive propositions with particular modalities, which means
that pseudo-positive propositions can subsume the positive
propositions of \sls. The tradeoff between positive and
pseudo-positive propositions could be resolved either way. By
including positive atomic propositions, we made \sls~more complicated,
but in a local way -- we needed a few more kinds and a few more
rules. On the other hand, if we used pseudo-positive propositions, the
notion of synthetic transitions would be intertwined with the
subordination-like analysis that enforces their correct usage.

\subsection{The need for traces}
\label{sec:whytraces}

One of the most important differences between \sls~and its
predecessors, especially CLF, is that traces are treated as
first-class syntactic objects. This allows us to talk about 
partial proofs and thereby encode our earlier 
money-store-battery-robot example as a trace with this type:
\begin{align*}
& \left(
 x{:}\iseph{\susp{\sf 6bucks}}, ~~
 f{:}\iseph{({\sf battery} \lefti {\ocircle}{\sf robot})}, ~~
 g{:}\ispers{({\sf 6bucks} \lefti {\ocircle}{\sf battery})}
\right)
\\
\leadsto^* &
\left(
 z{:}\iseph{\susp{\sf robot}}, ~~
 g{:}\ispers{({\sf 6bucks} \lefti {\ocircle}{\sf battery})}
\right)
\end{align*}
It is also possible to translate the example from Chapter~\ref{chapter-foc}
as a {\it complete} proof of the following proposition:
\[
  {\sf 6bucks} 
      \fuse {\gnab}({\sf battery} \lefti {\ocircle}{\sf robot})
      \fuse {!}({\sf 6bucks} \lefti {\ocircle}{\sf battery})
     \lefti
     {\ocircle}{\sf robot}
\]

Generally speaking, we can try to represent a trace $T :: (\Psi;
\Delta) \leadsto^* (\Psi'; \Delta')$ as a closed deductive proof
$\lambda P.\,\tlet{T}{V}$ of the proposition $(\exists
\Psi.\,{\fuse}\Delta) \lefti {\ocircle}(\exists
\Psi'.\,{\fuse}\Delta)$,\footnote{The notation ${\fuse}{\Delta}$ fuses
  together all the propositions in the context. For example, if
  $\Delta = w{:}\iseph{\susp{p^+_\meph}} \fuse x{:}\istrue{A^-},
  y{:}\iseph{B^-}, z{:}\ispers{C^-}$, then ${\fuse}{\Delta} = p^+_\meph
  \fuse {\downarrow}A^- \fuse {\gnab}B^- \fuse {!}C^-$. The notation
  $\exists \Psi. A^+$ turns all the bindings in the context $\Psi =
  \lf{a_1}{:}\tau_1,\ldots,\lf{a_n}{:}\tau_n$ into existential
  bindings $\exists \lf{a_1}{:}\tau_1\ldots\exists
  \lf{a_n}{:}\tau_n.A^+$.}  where the pattern $P$ re-creates the initial
process state $(\Psi; \Delta)$ and all the components of the final
state are captured in the value $V$.  The problem with this approach
is that the final proposition is under no particular obligation to
faithfully
capture the structure of the final process state. This can be seen in
the example above: to actually capture the structure of the final
process state, we should have concluded ${\sf robot} \fuse {!}({\sf
  6bucks} \lefti {\ocircle}{\sf battery})$ instead of simply ${\sf
  robot}$. It is also possible to conclude any of the following:
\smallskip
\begin{enumerate}
\item ${\sf robot} \fuse {!}({\sf 6bucks} \lefti {\ocircle}{\sf
  battery}) \fuse {!}({\sf 6bucks} \lefti {\ocircle}{\sf
  battery})$, or 
\item ${\sf robot} \fuse {\downarrow}({\sf 6bucks} \lefti {\ocircle}{\sf
  battery}) \fuse {\gnab}({\sf 6bucks} \lefti {\ocircle}{\sf
  battery})$, or even
\item ${\sf robot}  \fuse {\gnab}({\sf 6bucks} \fuse {!}({\sf battery} \lefti {\ocircle}{\sf robot} \}) \lefti {\ocircle}{\sf robot})
  \fuse {\downarrow}({\sf robot}
\lefti {\ocircle}{\sf robot})$.
\end{enumerate}
\smallskip 
%
The problem with encoding traces as complete proofs, then, is that
values cannot be forced to 
precisely capture the structure of contexts, especially when there are
no variables or persistent propositions. Cervesato and Scedrov
approach this problem by severely restricting the logic and changing
the interpretation of the existential quantifier so that it acts like
a nominal quantifier on the right \cite{cervesato09relating}. The
introduction of traces allows us to avoid similar restrictions in
\sls.

Despite traces being proper syntactic objects, they are not
first-class concepts in the theory: they are derived from focused
\ollll~terms and interpreted as partial proofs. Because hereditary
substitution, identity expansion, and focalization are only defined on
complete \ollll~proofs, these theorems and operations only apply by
analogy to the deductive fragment of \sls; they do not apply to
traces.  In joint work with Deng and Cervesato, we considered a
presentation of logic that treats process states and traces as
first-class concepts and reformulates the usual properties of cut and
identity in terms of coinductive simulation relations on process
states \cite{deng12relating}. We hope that this work will eventually
lead to a better understanding of traces, but the gap remains quite
large.

% In \cite{deng12relating}, we presented the {\it logical preorder} as a
% relation $\Delta_1 \preceq \Delta_2$ between propositional process states
% that holds whenever, for all $\Theta$ and $U$, we have that
% $\tackon{\Theta}{\Delta_1} \vdash U$ implies $\tackon{\Theta}{\Delta_2}
% \vdash U$. An elegant property, {\it harmony}, relates the logical 
% preorder to cut admissibility and identity expansion. 


% \subsection{A logic of traces}

% Traces in \sls~are syntactic objects. They are not, however,
% first-class objects in the theory: they are derived from focused
% \ollll~terms and explained as partial proofs. Because hereditary
% substitution, identity expansion, and focalization are only defined on
% complete \ollll~proofs, these theorems and operations only apply by
% analogy to the deductive fragment \sls; they do not apply to traces.

% In \cite{deng12relating}, we presented the {\it logical preorder} as a
% relation $\Delta_1 \preceq \Delta_2$ between propositional process states
% that holds whenever, for all $\Theta$ and $U$, we have that
% $\tackon{\Theta}{\Delta_1} \vdash U$ implies $\tackon{\Theta}{\Delta_2}
% \vdash U$. An elegant property, {\it harmony}, relates the logical 
% preorder to cut admissibility and identity expansion. 

\subsection{LF as a term language}
\label{sec:why-not-fully-dependent}

The decision to use LF as a first-order domain of quantification
rather than using a fully-dependent system is based on several
considerations. First and foremost, this choice was sufficient for the
purposes of this thesis. In fact, for the purposes of this thesis, we
could have used an even simpler term language of simply-typed LF
\cite{pfenning08church}. Two other logic programming interpretations
of \sls-like frameworks, Lollimon \cite{lopez05monadic} and Ollibot
\cite{pfenning09substructural}, are in fact based on simply-typed term
languages. Canonical LF and Spine Form LF are, at this point,
sufficiently well understood that the additional overhead of fully
dependently-typed terms is not a significant burden, and there are
many examples beyond the scope of this thesis where dependent types are
useful.

On a theoretical level, it is a significant simplification when we
restrict ourselves to {\it any} typed term language with a reasonable
notion of equality and simultaneous substitution. The conceptual
priority in this chapter is clear: Section~\ref{sec:sls-termlanguage}
describes object terms, Section~\ref{sec:slsframework} describes proof
terms as a fragment of focused \ollll, and
Section~\ref{sec:framework-concurrenteq} describes a coarser
equivalence on proof terms, concurrent equality. If the domain of
first-order of quantification was \sls~terms, these three
considerations would be mutually dependent -- we would need to
characterize concurrent equality before presenting the logic
itself. For the purposes of showing that a logical framework can be
carved out from a focused logic -- the central thesis of this and the
previous two chapters -- it is easiest to break this circular
dependency. We conjecture that this complication is no great obstacle,
but this thesis avoids the issue.

On a practical level, there are advantages to using a well-understood
term language. The \sls~prototype implementation
(Section~\ref{sec:prototype}) uses the mature type reconstruction
engine of Twelf to reconstruct LF terms. Schack-Nielsen's
implementation of type reconstruction for Celf is complicated by the
requirements of dealing with type reconstruction for a substructural
term language, a consideration that is orthogonal to this
thesis \cite{schacknielsen08celf}. 

Finally, it is not clear that the addition of full CLF-like dependency
comes with great expressive benefit. 
% Even in LF and Twelf, many
% interesting specifications could be encoded in a two-level version of
% the language: a simply-typed object term language and a
% dependently-typed proof term language with first-order quantification
% over object terms. This restriction is sufficient for settings such as
% Harper's comprehensive survey of programming language design
% \cite{harper12practical},\footnote{Harper's metatheory also extends LF
%   by drawing a distinction between standard variables and nominal
%   parameters, but this is an orthogonal point.} and it is built in to
% the educational proof assistant SASyLF \cite{aldrich08sasylf}. 
In LF and Twelf, the ability to use full dependent types is critical
in part because it allows us to express {\it metatheorems} -- theorems
about the programming languages and logics we have encoded, like
progress and preservation for a programming language or cut
admissibility for a logic. Substructural logical frameworks like LLF
and CLF, in contrast, have not been successful in capturing
metatheorems with dependent types. Instead, metatheorems about
substructural logics have thus far generally been performed in logical
frameworks based on persistent logics. Crary proved theorems about linear
logics and languages in LF using the technique of explicit contexts
\cite{crary10higher}. Reed was able to prove cut admissibility for
linear logic and preservation for the LLF encoding of Mini-ML in HLF,
a persistent extension to LF that uses an equational theory to capture
the structure of substructural contexts \cite{reed09hybrid}.

% But in substructural logical frameworks like Linear LF, full
% dependency has been found to be {\it insufficient} for expressing
% metatheorems, which motivated the development of Hybrid LF as a
% framework for writing metatheorems about LF \cite{reed09hybrid}. The
% implementation of Hybrid LF effectively creates a stratification like
% \sls's -- full LF as an object term language, a linear logical
% framework with first-order quantification over object language terms,
% and a hybrid language that can inspect both LF object terms and linear
% proof terms.



% \subsection{Variations on concurrent equality}


% The interactions between unification and concurrent equality are
% delicate, and we do not claim that the answers we give here are
% final. In Section~\ref{sec:independency}, we motivated both the
% independency requirement that $\emptyset = {\bullet}S_1 \cap
% {\ast}S_2$ and the requirement that $\emptyset = {\ast}S_1 \cap
% {\bullet}S_2$ by giving ill-typed counterexamples. The violation of
% either condition does not, in general, imply that $S_2; S_1$ will be
% ill-typed, however. This indicates that it might be possible to give a
% more precise condition that admits a coarser notion of concurrent
% equality.

% For both
% conditions, however, there are steps $S_1$ and $S_2$ where the
% $S_1; S_2$ and $S_2; S_1$ are both well-tyled traces even though
% one of these independency requirements is not satisfied.


% coincide with the equivalence  \sls~

% ${a^+} \simplearrow (a^+ \simplearrow {\uparrow}b^+) \simplearrow 
%   ({\downarrow}{\uparrow}b^+ \simplearrow c^-) \simplearrow c^-$. 

% It is not obvious that our treatment of the interaction between 
% unification and concurrent equivalence is the right one. 

% \subsection{Concurrent equality and multifocusing}

% Concurrent equality is related to the equivalence relation induced by
% {\it multifocusing} \cite{chaudhuri08canonical}. Multifocusing is a
% concept, 

% One reason multifocusing is 

%  that has only been carefully explored in classical linear
% logic; the central change is that the rules which begins a focusing
% phase (in our presentation of MELL there were three: ${\it focus_L}$,
% ${\it focus_R}$, and ${\it copy}$) are allowed to simultaneously pull
% other propositions into focus.  As an illustration, if we reuse our
% notation from Section~\ref{sec:linnote} we can present the following
% plausible candidates for the multifocus rules in an intuitionistic
% system:
% \[
% \infer[{\it focus}_L]
% {\mildseq{\Gamma}{\Delta / A_1^-, \ldots, A_n^- }{U}}
% {n > 1
%  &
%  \mildseq{\Gamma}{\Delta, [A_1^-], \ldots, [A_n^-]}{U}}
% \quad
% \infer[{\it focus}_R]
% {\mildseq{\Gamma}{\Delta / A_1^-, \ldots, A_n^-}{C^+}}
% {n \geq 1
%  &
%  \mildseq{\Gamma}{\Delta, [A_1^-], \ldots, [A_n^-]}{[C^+]}}
% \]
% Multifocusing, however,
% appears to provide an even coarser notion of equivalence on focused
% proofs than concurrent equality does. In particular, the two
% distinct focusing proofs below are not concurrently equal: the proof
% on the right succeeds at proving $\langle c^- \rangle$ in one step,
% but leaves a subgoal in which $b^+$ is proved indirectly, whereas the
% proof at the right first transitions from having $\langle a^+ \rangle$
% and $a^+ \lolli {\uparrow} b^+$ resources to having a $\langle b^+
% \rangle$ resource, and only then proves $\langle c^- \rangle$, leaving
% a subgoal in which $b^+$ is proved directly.
% \[
% \infer
% {\mildseq{\cdot}
%   {~~
%    \langle a^+ \rangle, ~
%    a^+ \lolli {\uparrow}b^+, ~
%    {\downarrow}{\uparrow}b^+ \lolli c^-
%    ~~}
%   {~~\langle c^- \rangle}}
% {\infer
% {\mildseq{\cdot}
%   {~~
%    \langle a^+ \rangle, ~
%    a^+ \lolli {\uparrow}b^+
%    ~~}
%   {b^+}}
% {\infer
% {\mildseq{\cdot}
%   {~~
%    \langle b^+ \rangle
%    ~~}
%   {b^+}}
% {}}}
% \deduce{\mathstrut}
% {\deduce{\mathstrut}
% {\mbox{\it vs.}\mathstrut}}
% \infer
% {\mildseq{\cdot}
%   {~~
%    \langle a^+ \rangle, ~
%    a^+ \lolli {\uparrow}b^+, ~
%    {\downarrow}{\uparrow}b^+ \lolli c^-
%    ~~}
%   {~~\langle c^- \rangle}}
% {\infer
% {\mildseq{\cdot}
%   {~~
%    \langle b^+ \rangle, ~
%    {\downarrow}{\uparrow}b^+ \lolli c^-
%    ~~}
%   {~~\langle c^- \rangle}}
% {\infer
% {\mildseq{\cdot}
%   {~~
%    \langle b^+ \rangle
%    ~~}
%   {~~b^+}}
% {}}}
% \]
% Despite the lack of a full account of intuitionistic multifocusing, we
% can observe that the analogue of this sequent in classical linear
% logic has only one multifocused proof, and it is reasonable to
% conjecture that an account of multifocusing for intuitionistic logic
% would also relate these proofs. In classical linear logic,
% multifocusing offers a very fundamental normal form: any two proofs
% that can be made equal by locally permuting inference rules have the
% same multifocused proof.

% CLF's restricted form of concurrent equality will be sufficient for
% the logical framework in Chapter~\ref{chapter-framework}. 
% In fact, for the fragment of the
% the logic in Chapter~\ref{chapter-order} 
% that comprises our logical framework in Chapter~\ref{chapter-framework},
% I conjecture that concurrent equality and the equality given by
% multifocusing coincide.\footnote{This obviously means that the example
%   above will be outside the logical fragment that comprises the logical
%   framework.}  This conjecture is obviously difficult to make precise,
% much less prove, without a general theory of multifocusing in
% intuitionistic logic.


% \subsection{A warning about normalization}
% \label{sec:warning}

% In our earlier discussion of hereditary substitution and canonical
% forms in Section~\ref{sec:linlogicalframeworks}, we mentioned that the
% normalization theorem provided by hereditary substitution was weaker
% than the so-called weak normalization theorem for LF. That is because
% the weak normalization theorem says that any well-typed term can be
% converted into a canonical ($\beta$-normal and $\eta$-long) term by a
% particular series of $\beta$ and $\eta$ conversions. It is
% self-evident, by this statement of the theorem, that the resulting
% canonical term is equivalent to the original term.

% On the other hand, when we use hereditary substitution in the obvious
% way to obtain a Canonical LF term from an arbitrary non-canonical LF
% term, we gain {\it no guarantees} about the relationship between the
% non-canonical LF term and the Canonical LF term. The statement of the
% theorem does not preclude taking a $\beta$-normal, $\eta$-long LF term
% (like $\lambda x. \lambda y. x$ of type $p \rightarrow p \rightarrow
% p$ for some atomic type $p$) into a structurally different Canonical
% LF term (like $\lambda x. \lambda y. y$, which also has type $p
% \rightarrow p \rightarrow p$). It is possible to gain such a guarantee
% for LF, as Martens and Crary have shown in unpublished work
% \cite{martens11mechanizing}, but this result is a non-trivial statement
% about the constructive content of the normalization theorem. 

% In our setting, we should be concerned that we might take a focused
% proof, turn it into an unfocused proof by the obvious de-focalization
% procedure (the constructive content of
% Theorem~\ref{thm:linfocsound}), and then turn it back into a focused
% proof by focalization (the constructive content of
% Theorem~\ref{thm:linfoccomplete}) only to obtain a proof that was not
% identical or even related. This is not at all a merely hypothetical
% concern. We can run the mechanized structural focalization result from
% \cite{simmons11structural} on a persistent proposition,
% %
%    $a^+ \simplearrow 
%    {\downarrow}(a^+ \simplearrow {\uparrow}b^+) \simplearrow
%    {\downarrow}({\downarrow}{\uparrow}b^+ \simplearrow c^-) \simplearrow
%    c^-$, 
% %
% which is similar to the example from
% Section~\ref{sec:linconcurrenteq}.  In persistent logic (as in
% linear logic) that proposition has two focused propositions that
% are probably multifocusing equivalent (given a reasonable intuitionistic
% notion of multifocusing) but that are not concurrently equivalent
% under the proposed definition of concurrent equality. 
% However, if we take the focused proof that focuses 
% first on $a^+ \simplearrow {\uparrow}b^+$, transform it into an unfocused 
% proof, and then re-focus it, we will get the proof that focuses 
% first on ${\downarrow}{\uparrow}b^+ \simplearrow c^-$. Focalization,
% in other words, is not a partial inverse of de-focalization in the structural
% focalization development, except maybe modulo the (as yet undefined)
% equivalence relation established by multifocusing. 

% This example illustrates why we must be careful, but it is not a fatal
% flaw for two reasons. The first reason is the aforementioned
% conjecture that, for the restricted logical fragment defined in
% Chapter~\ref{chapter-framework} 
% as the basis of our logical framework, the focalizations of
% two proofs are concurrently equal if and only if the original proofs
% are convertible by local permutations of rules, the same condition
% that multifocusing satisfies. If this conjecture holds, it ought to be
% the case that, modulo this coarser equivalence, focalization {\it is}
% a partial inverse of de-focalization. Second, what is really at stake
% here is our ability to write down non-normal proofs in a logical
% framework that then normalizes them -- which is what the Twelf
% implementation of LF and the Celf implementation of CLF do -- with the
% confidence that we can look at a non-normal proof and know its
% corresponding canonical form. In this thesis, we will be content to
% work throughout with focused proofs and their analogues, so we can
% afford to leave questions about convertability and weak normalization
% to future work.



\part{Substructural operational semantics}

% On logical correspondence
\chapter{On logical correspondence}

In Part 1, we defined \sls, the logical framework of substructural
logical specifications.

\section{Logical transformation: compilation}

\subsection{Tail-recursion}

\subsection{Parallelism}

\section{Logical transformation: defunctionalization}

\section{Logical transformation: factoring}

Example: exceptions

\section{Exploring the richer fragment}

\subsection{Mutable storage}
\label{sec:mutable-storage}

No check for pointer inequality! This is a fundamental restriction of
the fact that we're using existential quantificaiton rather than some
form of nominal quantification. (Hack due to Favonia and Bob, personal
communication.)

\subsection{Call-by-need}

\subsection{Environment semantics}

\subsection{Looking back at natural semantics}
\label{sec:enriching-natsem}

\section{Partial transformation}

\subsection{Evaluation contexts}

\subsection{Temporal logic}

The natural semantics of \rowan~are not, on a superficial level,
significantly more complex than other natural semantics. However, it
turns out that the usual set of techniques for adding state to a
natural semantics break down, and discussing a \rowan-like logic with
state remained a challenge for many years.\robnote{Figure out from
  Rowan what the recent work he told you about was.} Through the
logical correspondance, it is easy to see why: the natural SSOS
specification of \rowan~integrates both concurrent and deductive
reasoning in an arbitrarily nested way. In fact, Figure XXX is the
only SLS specification in this thesis that exhibits this form of
recursive dependency between concurrent and deductive reasoning.  In
particular, the \rowan~specification is way out of the image of the
extended natural semantics we considered in
Section~\ref{sec:enriching-natsem}. The natural encoding in state lies
in the ambient substructural context of a concurrent computation, but
that ambient computation cannot properly enter into a deductive
sub-computation. If we tried to add state to \rowan~the same way we
added it in Section~\ref{sec:mutable-storage}, the entire store
would effectively leave scope whenever computation considered
the subterm $e$ of ${\sf next}(e)$. That consideration happens
as deductive reasoning, not as concurrent reasoning!

 it is the only we
will consider in this thesis that has with property.

It's hard to include state in temporal logic! But the logical correspondence
helps us understand why: the natural SSOS specification of 

% Ordered abstract machines
\chapter{Ordered abstract machines}
\label{chapter-absmachine}

This chapter centers around two transformations on logical
specifications.  Taken together, the operationalization transformation
(Section~\ref{sec:operationalization}), and the defunctionalization
transformation (Section~\ref{sec:defunctionalization}) allow us to
establish the logical correspondence between the deductive SLS
specification of a natural semantics and the concurrent SLS
specification of an abstract machine.

Natural semantics specifications are common in the literature,
and are also easy to encode in either the deductive fragment of
\sls~or in a purely deductive logical framework like LF.  We will
continue to use the natural semantics specification of call-by-value
evaluation for the lambda calculus as our running example:
\[
\infer[{\sf ev/lam}]
{\lambda x. e \Downarrow \lambda x. e \mathstrut}
{}
\quad
\infer[{\sf ev/app}]
{e_1\,e_2 \Downarrow v \mathstrut}
{e_1 \Downarrow \lambda x.e
 &
 e_2 \Downarrow v_2
 &
 [v_2/x]e \Downarrow v \mathstrut}
\]

Abstract machine semantics are less prevalent than natural
semantics. The most well-known is almost certainly Landin's SECD
machine \cite{landin64mechanical}, though the abstract machine is much
more similar to Danvy's SC machine from \cite{danvy03rational} and
Harper's $\mathcal K\{{\sf nat}{\rightharpoonup}\}$ system from
\cite[Chapter 27]{harper12practical}.  The abstract machine semantics
that we will show to be has two states $s$. The state $s = k \rhd e$
represents the expression $e$ being evaluated on top of the stack $k$,
and the state $s = k \lhd v$ represents the value $v$ being returned
to the stack $k$. Stacks $k$ are sequences of frames $f$
with the form $((\ldots({\sf halt}; f_1); \ldots); f_n)$, and each
frame $f$ either has the form $\Box\,e_2$ (an application frame
waiting for an evaluated function to be returned to it) or the form
$(\lambda x.e)\,\Box$ (an application frame with an evaluated function
waiting for an evaluated value to be returned to it). Given states,
stacks, and frames, we can define a ``classical'' abstract machine for
call-by-value evaluation of the lambda calculus as a transition system
with four transition rules:
\begin{align*}
{\sf absmachine/lam}{:} & ~~ k \rhd \lambda x.e ~ \mapsto ~ k \lhd \lambda x.e
\\
{\sf absmachine/app}{:} & ~~ k \rhd e_1\,e_2 ~ \mapsto ~ (k; \Box\,e_2) \rhd e_1
\\
{\sf absmachine/app1}{:} & ~~ 
  (k; \Box\,e_2) \lhd \lambda x.e ~ \mapsto ~ (k; (\lambda x.e)\,\Box) \rhd e_2
\\
{\sf absmachine/app2}{:} & ~~
  (k; (\lambda x.e)\,\Box) \lhd v_2 ~ \mapsto ~ k \rhd [v_2/x]e
\end{align*}

The operational intuition for these rules is precisely the same as the
operational intuition for the rewriting rules given in
Section~\ref{sec:intro-ssos}. This is not coincidental: the
\sls~specification from the introduction adequately encodes the
transition system $s \mapsto s'$ defined above, a point that we will
make precise in Section~\ref{sec:nat-ssos-adequacy}. The
\sls~specification from the introduction is {\it also} the result of
applying the operationalization and defunctionalization
transformations to the \sls~encoding of the natural semantics given
above, so the these two transformations combined with the adequacy
arguments at either end constitute a logical correspondence between
natural semantics and abstract machines. 

As discussed in Section~\ref{sec:the-point-is-modular-extension}, it
is interesting to put existing specification styles into logical
correspondence, but that is not our main reason for being interested
in the logical correspondence. Instead, we are primarily interested in
exploring the set of programming language features that can be
modularly integrated into a transformed \sls~specification that could
not be integrated into a natural semantics specification.  In
Section~\ref{sec:richer-ordered-abstract} we explore a selection of
these features, including mutable storage, call-by-need evaluation,
and recoverable failure.

\section{Logical transformation: operationalization}
\label{sec:operationalization}

The intuition behind operationalization is rather simple: we examine
the behavior of a deductive computation and then encode that
operational intuition as a concurrent computation.  Before presenting
the general transformation, we will motivate this transformation using
our natural semantics specification of call-by-value-evaluation. 

The definition of $e \Downarrow v$ is moded with $e$ as an input and
$v$ as an output, so it is meaningful to talk about being given a
particular expression $e$ and using deductive computation to search
for a $v$ such that $e \Downarrow v$ is derivable.  Consider a
recursive search procedure implementing this particular deductive
computation:
\begin{itemize}
\item
      If $e = \lambda x. e'$, 
      it is possible to derive 
      $\lambda x. e' \Downarrow \lambda x. e'$
      with the rule ${\sf ev/lam}$.
\item
       If $e = e_1\,e_2$,
       attempt to derive 
       $e_1\,e_2 \Downarrow v$
       using the rule ${\sf ev/app}$ by doing the following:
    \begin{enumerate}
    \item Search for a $v_1$ such that 
          $e_1 \Downarrow v_1$ is derivable.
    \item Assert that $v_1 = \lambda x.e'$ for some
          $e'$; fail if it is not.
    \item Search for a $v_2$ such that 
          $e_2 \Downarrow v_2$ is derivable.
    \item Search for a $v$ such that 
          $[v_2/x]e \Downarrow v$ is derivable.
    \end{enumerate}
% \item
%       If $e = e_1 \arb e_2$,
%       attempt to derive $e_1 \arb e_2 \Downarrow v$ using
%       the rule ${\sf ev/choose1}$ by searching for a 
%       $v$ such that $e_1 \Downarrow v$ is derivable.
% \item
%       If $e = e_1 \arb e_2$,
%       attempt to derive $e_1 \arb e_2 \Downarrow v$ using 
%       the rule ${\sf ev/choose2}$ by searching for a 
%       $v$ such that $e_2 \Downarrow v$ is derivable.  
\end{itemize}
%
The goal of the operationalization transformation is to implement this
deductive computation as a concurrent computation. The first step in
doing so is to introduce two new ordered atomic propositions.  The
proposition ${\sf eval}\,\interp{e}$ is the starting point, indicating
that we want to search for a $v$ such that $e \Downarrow v$, and the
proposition ${\sf retn}\,\interp{v}$ indicates the successful
completion of this procedure. Therefore, searching for a $v$ such that
$e \Downarrow v$ is derivable will be analogous to building a trace $T
:: x_e{:}\susp{{\sf eval}\,\interp{e}} \leadsto^* x_v{:}\susp{{\sf
    retn}\,\interp{e}}$ with concurrent computation.

Representing the first case is straightforward: if we are evaluating
$\lambda x.e$, then we have succeeded and can return $\lambda x.e$. 
This is encoded in the rule ${\sf ev/lam}$. 
\[
{\sf ev/lam} : {\sf eval}\,({\sf lam}\,\lambda x.\,E\,x)
   \lefti \{ {\sf retn}\,({\sf lam}\,\lambda x.\,E\,x) \}
\]
Because the second rule involves both recursion and multiple subgoals,
we will generalize our picture of the process state to allow us to store a
stack of unfinished work in the ordered context, growing out to the
right. Our new understanding, then, is that contexts either have the
form $x{:}\susp{{\sf eval}\,\interp{e}}, \Delta$ or the form $x{:}\susp{{\sf
  retn}\,\interp{v}}, \Delta$. In the process of concurrently
computing a trace $x_e{:}\susp{{\sf eval}\,\interp{e_1\,e_2}}, \Delta
\leadsto^* x_r{:}\susp{{\sf retn}\,\interp{v}}, \Delta$, each of the
recursive calls to the search procedure will involve a sub-trace of the
form
%
\[x_e{:}\susp{{\sf eval}\,\interp{e'}}, y{:}\istrue{A^-}, \Delta
  \leadsto^*
  x_r{:}\susp{{\sf retn}\,\interp{v'}}, y{:}\istrue{A^-}, \Delta\]
%
where $A^-$ is a negative proposition that is prepared to interact
with the final ${\sf retn}\,\interp{v'}$ proposition to kickstart the
rest of the computation.

It's helpful to work backwards: in the fourth step, we have found
$E\,x = \interp{e}$ (where $e$ potentially has $x$ free) and $V_2 =
\interp{v_2}$, and the recursive call is to ${\sf
  eval}\,\interp{[v_2/x]e}$, which is the same thing as ${\sf
  eval}\,(E\,V_2)$. If the recursive call successfully returns, the
context will contain a suspended atomic proposition of the form ${\sf
  retn}\,V$ where $V = \interp{v}$, and the search procedure as a
whole is complete: the answer is $v$.  Thus, the negative proposition
that implements the continuation can be written as $(\forall V. {\sf
  retn}\,V \lefti \{ {\sf retn}\,V \})$. The positive proposition that
will create this sub-computation can be written as follows:
\begin{align*}
{\it Step_4}(E,V_2) & \equiv {\sf eval}\,(E\,V_2) 
\fuse {\downarrow}(\forall V.\, {\sf retn}\,V \lefti \{ {\sf retn}\,V \})
%
\intertext{Moving backwards, in the third step we have a $E_2 =
  \interp{e_2}$ that we were given and $E\,x = \interp{e}$ that we
  have computed. The recursive call is to ${\sf
    eval}\,\interp{e_2}$, and assuming that it completes, we need
  to begin the fourth step. The positive proposition that will 
  create this sub-computation can be written as follows:}
%
{\it Step_3}(E_2,E) & \equiv {\sf eval}\,E_2 
\fuse {\downarrow}(\forall V_2.\,
  {\sf retn}\,V_2 \lefti \{ {\it Step_4}(E,V_2) \})
%
\intertext{Finally, the first two steps can be handled together. We have
$E_1 = \interp{e_1}$ and $E_2 = \interp{e_2}$; the recursive
call is to ${\sf eval}\,\interp{e_1}$. Once the
recursive call completes, we can enforce that the returned value has
the form $\interp{\lambda x.e}$ before proceeding
to the continuation.}
{\it Step_{1,2}}(E_1, E_2) & \equiv {\sf eval}\,E_1
\fuse {\downarrow}(\forall E.\, {\sf retn}\,({\sf lam}\,\lambda x.\,E\,x)
\lefti \{ {\it Step_3}(E_2, E)\})
\end{align*}
Thus, the rule implementing this entire portion of the search
procedure is 
\[
\forall E_1.\,\forall E_2.\,
{\sf eval}\,({\sf app}\,E_1\,E_2) \righti \{ {\it
  Step_{1,2}}(E_1, E_2) \}
\]
The \sls~encoding of our example natural semantics is shown in
Figure~\ref{fig:example-transform-cbv} alongside the transformed
specification, which has the form of an ordered abstract machine
semantics, though it is different than the ordered abstract machine
semantics presented in the introduction. We say the specification
above is {\it higher-order}, as ${\sf ev/app}$ is a rule that, when it
participates in a transition, produces a new rule $(\forall E.\,{\sf
  retn}\,({\sf lam}\,\lambda x.\,E\,x) \lefti \{ \ldots \})$ that
lives in the context. The ordered abstract machine semantics from the
introduction was {\it first-order}, because the head $\{ A^+ \}$ of
every concurrent rule contains only positive atomic propositions.  We
discuss the defunctionalization transformation, which allows us to
derive first-order specifications from specifications that are
higher-order in this way, in Section~\ref{sec:defunctionalization}
below.

\begin{figure}
\begin{minipage}[b]{0.36\linewidth}
\fvset{fontsize=\small,boxwidth=auto}
\VerbatimInput{sls/cbv-ev.sls}
\end{minipage}
\hspace{0.5cm}
\begin{minipage}[b]{0.64\linewidth}
\fvset{fontsize=\small,boxwidth=auto}
\VerbatimInput{sls/cbv-ev-ssos.sls}
\end{minipage}
\caption{A natural semantics for CBV (left) and the corresponding (higher-order)
  ordered abstract machine (right).}
\label{fig:example-transform-cbv}
\end{figure}

The intuitive connection between natural semantics specifications and
concurrent specifications has been explored previously and
independently by Schack-Nielsen \cite{schacknielsen07induction} and by
Cruz and Favonia \cite{cruz12parallel}; Schack-Nielsen proves the
equivalence of the two specifications, whereas Cruz and Favonia used
the connection informally. The contribution of this section is to
describe a general transformation (of which
Figure~\ref{fig:example-transform-cbv} is one instance) and to prove
the transformation correct in general. 

In Section~\ref{sec:trans-subset} we will present the subset of
specifications that our operationalization transformation handles, and
in Section~\ref{sec:trans-basic} we present the most basic form of the
transformation.  In
Sections~\ref{sec:trans-tail}~and~\ref{sec:trans-par} we extend the
basic transformation to be both tail-recursion optimizing and
parallelism-enabling. Finally, in
Section~\ref{sec:operationalization-correct}, we establish the
correctness of the overall transformation.

\subsection{Transformable signatures}
\label{sec:trans-subset}

The starting point for the operationalization transformation is a
deductive signature that is well-moded in the sense described in
Section~\ref{sec:framework-modes}. Every declared negative predicate
will either remain defined by deductive proofs (we write the atomic
propositions built with these predicates as $p_d^-$, $d$ for
deductive) or will be transformed so that it is concurrently defined
(we write the atomic propositions built with these predicates as
$p_c^-$, $c$ for concurrent).

For the purposes of describing and proving the correctness of the
operationalization transformation, we will assume that all transformed
atomic propositions $p_d^-$ have two arguments where the first
argument is moded as an input and the second is an output. That is,
their predicates are declared as follows:
\begin{align*}
& {\sf a} : \tau_1 \rightarrow \tau_2 \rightarrow {\sf prop}.\\
& {\sf \#mode~a~{+}~{-}}.
\end{align*}
Without dependency, two-place relations are sufficient for describing
$n$-place relations.\footnote{As an example, to handle addition on
  natural numbers, defined as a three-place relation ${\sf add} : {\sf
    nat} \rightarrow {\sf nat} \rightarrow {\sf nat} \rightarrow {\sf
    type}$ with its usual mode (${\sf add}~{+}~{+}~{-}$), we define a
  unique type ${\sf add\_in}$ with one binary constructor ${\sf
    add\_c} : {\sf nat} \rightarrow {\sf nat} \rightarrow {\sf
    add\_in}$. Then we can declare (${\sf add'} : {\sf add\_in}
  \rightarrow {\sf nat} \rightarrow {\sf type}$) with mode (${\sf
    add'}~{+}~{-}$).}  It should be possible to handle dependent
predicates (that is, those with declarations of the form ${\sf a} :
\Pi x{:}\tau_1.\,\tau_2(x) \rightarrow {\sf type}$), but we will not do
so here.

The restriction on signatures furthermore enforces that all rules must
be of the form ${\sf r} : C$ or ${\sf r} : D$, where $C$ and $E$ are
refinements of the negative propositions of \sls~that are defined as
follows:
\begin{align*}
C & ::= p^-_{c} 
    \mid \forall x{:}\tau.\, C
    \mid p^+_\mpers \lefti C
    \mid {!}p^-_c \lefti C
    \mid {!}G \lefti C \\
D & ::= p^-_{d}
    \mid \forall x{:}\tau.\, D
    \mid p^+_\mpers \lefti D
    \mid {!}p^-_c \lefti D
    \mid {!}G \lefti C \\
G & ::= p^-_d 
    \mid \forall x{:}\tau.\, G
    \mid p^+_\mpers \lefti G
    \mid {!}D \lefti G
\end{align*}
If {\it all} propositions are to remain deductive, then the
propositions $p^-_c$ and $C$ are irrelevant, and this restriction
describes all persistent, deductive specifications -- essentially, any
signature that could be executed by the standard logic programming
interpretation of LF \cite{pfenning98elf}. On the other hand, if all
propositions are to be transformed, then the propositions $p^-_d$ and
$D$ are irrelevant and this restriction amounts to restricting
rules to the Horn fragment.

All propositions $C$ are equivalent (at the level of synthetic
inference rules) to propositions of the form $\forall
\overline{x_0}\ldots \forall \overline{x_n}.\,A^+_n \lefti \ldots
\lefti A^+_1 \lefti {\sf a}\,t_{0}\,t_{n-1}$, where the $\forall
\overline{x_i}$ are shorthand for a series of universal quantifiers
$\forall {x_{i1}}{:}{\tau_{i1}} \ldots \forall {x_{\it
    ik}}{:}{\tau_{\it ik}}$ and where each variable in
$\overline{x_i}$ does not appear in $t_0$ (unless $i = 0$) nor in any
$A^+_j$ with $j < i$ but does appear in $A^+_i$ (or $t_0$ if $i =
0$). Therefore, when we consider moded proof search, the variables
bound in $\underline{x_0}$ are all fixed by the query and those bound
in the other $\underline{x_i}$ are all fixed by the output position of
the $i^{\rm th}$ premise.

\subsection{Basic transformation}
\label{sec:trans-basic}

The operationalization transformation $\transop{\Sigma}$
operates on SLS signatures $\Sigma$ that have the form described in the
previous section. We
will first give the transformation on signatures; the transformation
of rule declarations ${\sf r} : C$ is the key case.

Each two-place predicate ${\sf a}$ that we plan to operationalize gets
turned into two one-place predicates ${\sf eval\_a}$ and ${\sf
  retn\_a}$.  We will write $\opsubst{X}$ for the operation of
substituting all occurrences of $p^-_c = {\sf a}\,t_1\,t_2$ with
$({\sf eval\_a}\,t_1 \lefti \{ {\sf retn\_a}\,t_2 \})$ in $X$. This
substitution operation is used on propositions, contexts, and frames;
it appears in the transformation of rules ${\sf r} : D$ below.

\begin{itemize}
\item $\transop{\cdot} = \cdot$
\item $\transop{\Sigma, {\sf a} : \tau_1 \rightarrow \tau_2
    \rightarrow {\sf prop}} = \transop{\Sigma}, ~ {\sf eval\_a} :
  \tau_1 \rightarrow {\sf prop\,ord}, ~ {\sf retn\_a} : \tau_2
  \rightarrow {\sf prop\,ord}$ \\ {\it (if $\sf a$ is one of the
    predicates that we are translating)}
\item $\transop{\Sigma, {\sf a} : K} = \transop{\Sigma}, ~ {\sf a}
  : K$ {\it (otherwise)}
\item $\transop{\Sigma, {\sf c} : \tau} = \transop{\Sigma}, ~ {\sf
    c} : \tau$ 
\item $\transop{\Sigma, {\sf r} : C} = \transop{\Sigma}, ~ {\sf r}
  : \forall \overline{x_0}.\, {\sf eval\_a}\,t_0 \lefti \llbracket A^+_1,
  \ldots, A^+_n \rrbracket (t_{n+1}, {\sf id})$ \\ {\it (where $C$ is
    equivalent to $\forall \overline{x_0}\ldots \forall
    \overline{x_n}.\, A^+_n \lefti \ldots \lefti A^+_1 \lefti {\sf
      a}\,t_{0}\,t_{n+1}$)}
\item $\transop{\Sigma, {\sf r} : D} = \transop{\Sigma}, ~ {\sf r}
  : \opsubst{D}$
\end{itemize}

The transformation of a proposition $C$ of the form $\forall
\overline{x_0}\ldots \forall \overline{x_n}.\,A^+_n \lefti \ldots
\lefti A^+_1 \lefti {\sf a}\,t_{0}\,t_{n+1}$ involves the definition
$\opbasic{A^+_i,\ldots,A^+_n}{t_{n+1}}{\sigma}$, where $\sigma$
substitutes only for variables in $\overline{x_j}$ where $j < i$. The
function is defined inductively on the length of the sequence
$A^+_i,\ldots,A^+_n$.

\begin{itemize}
\item $\opbasic{}{t_{n+1}}{\sigma} = \{ {\sf retn\_a}\,(\sigma{t_{n+1}}) \}$
\item $\opbasic{p^+_\mpers,A^+_{i+1},\ldots,A^+_n}{t_{n+1}}{\sigma} 
  = \forall \overline{x_i}.\, (\sigma{p^+_\mpers}) \lefti \opbasic{A^+_{i+1},\ldots,A^+_n}{t_{n+1}}{\sigma}$
\item $\opbasic{{!}p^-_c,A^+_{i+1},\ldots,A^+_n}{t_{n+1}}{\sigma}$
  \\
  $~ \qquad = \{ {\sf eval\_b}\,({\sigma}t^{\it in}_i) \fuse
  (\forall\overline{x_i}.\, {\sf retn\_b}\,(\sigma{t^{\it out}_i})
  \lefti \opbasic{A^+_{i+1},\ldots,A^+_n}{t_{n+1}}{\sigma}) \}$\\
  {\it (where $p^-_c$ is ${\sf b}\,t^{\it in}_i\,t^{\it out}_i$)}
\item $\opbasic{{!}G,A^+_{i+1},\ldots,A^+_n}{t_{n+1}}{\sigma} = \forall
  \overline{x_i}.\, {!}(\sigma\opsubst{G}) \lefti
  \opbasic{A^+_{i+1},\ldots,A^+_n}{t_{n+1}}{\sigma}$
\end{itemize}

\noindent
This operation is slightly more general than it needs to be to
describe the transformation on signatures, because the substitution
$\sigma$ will always just be the identity substitution ${\sf id}$.
Non-identity substitutions arise during the proof of correctness, which
is why we introduced them here.

We have already given an example of the this basic operationalization
transformation, as Figure~\ref{fig:example-transform-cbv} is an
instance of this transformation.

\subsection{Tail-recursion}
\label{sec:trans-tail}

Consider again our motivating example, the procedure for that takes
expressions $e$ and searches for expressions $v$ such that $e
\Downarrow v$ is derivable. If we were to implement that procedure as
a functional program, the procedure would be {\it tail-recursive}. In
the procedure that handles the case when $e = e_1\,e_2$, the last step
is that the search procedure is invoked recursively. If and when that
callee returns $v$, then the caller will also return $v$.

Tail-recursion is significant in functional programming because
tail-recursive calls can be implemented without allocating a stack
frame: when a compiler makes this more efficient choice, we say it is
performing {\it tail-recursion optimization}.\footnote{Or {\it tail-call optimization}, as a tail-recursive function call is just a
  specific instance of a tail call.} An analogous opportunity for
tail-recursion optimization also arises in our logical compilation
procedure. In our motivating example, the last step in the $e_1\,e_2$
case was operationlized as a positive proposition of the form ${\sf
  eval}\,(E\, V) \fuse (\forall v.\,{\sf retn}\,v \lefti \{ {\sf
  retn}\,v \})$. In a successful search, the process state 
\[ x{:}{\sf
  eval}\,(E\, V), y{:}\istrue{(\forall v.\,{\sf retn}\,v \lefti \{
  {\sf retn}\,v \})}, \Delta\]
will concurrently compute until the
state 
\[ x'{:}{\sf retn}\,V, y{:}\istrue{(\forall v.\,{\sf retn}\,v \lefti
  \{ {\sf retn}\,v \})}, \Delta\] is reached, at which point the next
step \[y'{:}{\sf retn}\,V, \Delta\] is reached in one step by focusing
on $y$. 

If we operationalize the last step in the $e_1\,e_2$ case as ${\sf
  eval}\,(E\,V)$ instead of as ${\sf eval}\,(E\, V) \fuse (\forall
v.\,{\sf retn}\,v \lefti \{ {\sf retn}\,v \})$, we will reach the same
final state with one less transition. The tail-recursion optimizing
version of the operationalization transformation creates concurrent
computations that avoid these useless steps.

We cannot perform tail recursion in general. The obvious reason for
this to be the case is when the output of the last subgoal is
different from the output of the goal. For example, the rule ${\sf r}
: \forall{x}.\,\forall{y}.\,{!}{\sf a}\,x\,y \lefti {\sf a}\,({\sf
  c}\,x)\,({\sf c}\,y)$, will translate to
\[ {\sf r} : \forall{x}.\,{\sf eval\_a}\,({\sf c}\,x) \lefti \{ {\sf
  eval\_a}\,x \fuse (\forall y.\, {\sf retn\_a}\,y \lefti \{ {\sf
  retn\_a}\,({\sf c}\,y) \} ) \} \] There is no opportunity for
tail-recursion optimization, because the output of the last search
procedure, $t^{\it out}_n = y$, is different than the value returned
down the stack, $t_{n+1} = {\sf c}\,y$. This case corresponds to
functional programs that cannot be tail-call optimized.

More subtly, we cannot even eliminate all cases where $t^{\it out}_n =
t_{n+1}$ unless these terms are {\it fully general}. We say that
$t_{n+1}$ with type $\tau$ is fully general if all of its free
variables are in $\overline{x_n}$ (and therefore not fixed by the
input of any other premise) and if, for any variable-free term $t'$ of
type $\tau$, there exists a substitution $\sigma$ such that $t =
{\sigma}t_{n+1}$. The simplest example way to do this is to force
$t_{n+1} = t^{\it out}_n = y$ where $y = \overline{x_n}$.\footnote{It
  is also possible to have a fully general $t_{n+1} = {\sf
    c}\,y_1\,y_2$ if, for instance, ${\sf c}$ has type $\tau_1
  \rightarrow \tau_2 \rightarrow {\sf foo}$ and there are no other
  constructors of type ${\sf foo}$. However, we also have to check
  that there are no other first-order variables in $\Psi$ with types
  like $\tau_3 \rightarrow {\sf foo}$ that could be used to make other
  terms of type ${\sf foo}$. The technology to handle this, worlds
  checking and subordination analysis, is well-understood and surveyed
  elsewhere \cite{harper07mechanizing}, but this is tangential to the
  current discussion.} This condition doesn't have an analogue in
functional programming, because it corresponds to the possibility that
moded deduction computation can perform pattern matching on {\it
  outputs} and fail if the pattern match fails.

The tail-recursive procedure can be described by adding a new 
case to the definition of 
$\opbasic{A^+_i,\ldots,A^+_n}{t_{n+1}}{\sigma}$:

\begin{itemize}
\item $\opbasic{{!}{\sf a}\,t^{\it in}_n\,t_{n+1}}{t_{n+1}}{\sigma} 
  = \{{\sf eval\_a}\,({\sigma}{t^{\it in}_n})\}$
\\
  {\it (where $t_{n+1}$ is fully general)}
\end{itemize}
This case overlaps with the third case of the definition given
in Section~\ref{sec:trans-basic}, which indicates that tail-recursion
optimization can be applied or not in a nondeterministic manner.

\begin{figure}
\fvset{fontsize=\small,boxwidth=auto}
\VerbatimInput{sls/cbv-ev-ssos-tail.sls}
\caption{A higher-order ordered abstract machine semantics for the CBV
  evaluation.}
\label{fig:cbv-ev-ssos-tail}
\end{figure}

\subsubsection{Example}

Operationalizing the natural semantics from
\ref{fig:example-transform-cbv} results in the ordered abstract
machine in Figure~\ref{fig:cbv-ev-ssos-tail}.  A more dramatic
illustration of tail-call optimization can be given if we consider a
big-step evaluation function that is based on a small-step structural
operational semantics (SOS) specification. In SOS specifications,
single-step evaluation is defined as the two-place relation ${\sf
  step} : {\sf exp} \rightarrow {\sf exp} \rightarrow {\sf prop}$
(moded ${\sf exp}\,{+}\,{-}$) that makes use of the helper judgment
${\sf value} : {\sf exp} \rightarrow {\sf prop}$ (moded ${\sf
  value}\,{+}$). We will not define these propositions here, but we do
so later on in Section~\ref{sec:evaluationcontexts}.

Given the definition of ${\sf step}\,\interp{e}\,\interp{e'}$, it is
easy to define big-step evaluation ${\sf ev}\,\interp{e}\,\interp{v}$
as a series of small steps:

\smallskip
\fvset{fontsize=\small,boxwidth=auto}
\VerbatimInput{sls/cbv-sos-steps.sls}
\smallskip

\begin{figure}
\begin{minipage}[b]{0.55\linewidth}
\fvset{fontsize=\small,boxwidth=auto}
\VerbatimInput{sls/cbv-sos-proc2.sls}
\end{minipage}
\hspace{0.5cm}
\begin{minipage}[b]{0.45\linewidth}
\fvset{fontsize=\small,boxwidth=auto}
\VerbatimInput{sls/cbv-sos-proc.sls}
\end{minipage}
\caption{The transformation of a trivial big-step semantics, both
  without (left) and with (right) tail-recursion optimization.}
\label{fig:sos-tailrecursion}
\end{figure}

Running this specification through the operationalization
transformation and only operationalizing the ${\sf ev}$ predicate
results in what I consider to be the most boring substructural
operational semantics specification, shown in
Figure~\ref{fig:sos-tailrecursion} both without the tail-recursion
optimization (left) and with the tail-recursion optimization (right).

\begin{figure}
\begin{align*}
& x_1{:}{\sf eval}\,\interp{e_1} \\
\leadsto ~ & x_2{:}{\sf eval}\,\interp{e_2}, 
  y_1{:}\istrue{(\forall v. {\sf retn}\,v \lefti \{ {\sf retn}\,v \})} \\
\leadsto ~ & x_3{:}{\sf eval}\,\interp{e_3}, 
  y_2{:}\istrue{(\forall v. {\sf retn}\,v \lefti \{ {\sf retn}\,v \})}, 
  y_1{:}\istrue{(\forall v. {\sf retn}\,v \lefti \{ {\sf retn}\,v \})} \\
\leadsto ~ & \cdots\\
\leadsto ~ & x_n{:}{\sf eval}\,\interp{v}, 
  y_{n-1}{:}\istrue{(\forall v. {\sf retn}\,v \lefti \{ {\sf retn}\,v \})}, 
  \cdots,
  y_1{:}\istrue{(\forall v. {\sf retn}\,v \lefti \{ {\sf retn}\,v \})} \\
\leadsto ~ & z_n{:}{\sf retn}\,\interp{v}, 
  y_{n-1}{:}\istrue{(\forall v. {\sf retn}\,v \lefti \{ {\sf retn}\,v \})}, 
  \cdots,
  y_1{:}\istrue{(\forall v. {\sf retn}\,v \lefti \{ {\sf retn}\,v \})} \\
\leadsto ~ & \cdots \\
\leadsto ~ & z_3{:}{\sf retn}\,\interp{v}, 
  y_2{:}\istrue{(\forall v. {\sf retn}\,v \lefti \{ {\sf retn}\,v \})}, 
  y_1{:}\istrue{(\forall v. {\sf retn}\,v \lefti \{ {\sf retn}\,v \})} \\
\leadsto ~ & z_2{:}{\sf retn}\,\interp{v}, 
  y_1{:}\istrue{(\forall v. {\sf retn}\,v \lefti \{ {\sf retn}\,v \})} \\
\leadsto ~ & z_1{:}{\sf retn}\,\interp{v}
\end{align*}
\caption{Example trace with the non-tail-recursion-optimized
  semantics in Figure~\ref{fig:sos-tailrecursion}}
\label{fig:example-proc-non-tail-recursive-trace}
\end{figure}

The tail-recursion optimized translation is definitely superior for
this example. Concurrent proofs for the non-tail-recursion-optimized
specification build up an enormous stack of useless copies of the
proposition $(\forall v. {\sf retn}\,v \lefti \{ {\sf retn}\,v \})$,
as shown in Figure~\ref{fig:example-proc-non-tail-recursive-trace}.
In contrast, the tail-recursion optimized version on the right hand
side of Figure~\ref{fig:sos-tailrecursion} takes half as many steps,
and each step is smaller, simpler, and the overall trace does a better
job of actually capturing the linear computation that is actually
involved in describing a big-step semantics using a small-step
structural operational semantics:
\[
x_1{:}{\sf eval}\,\interp{e_1} 
 ~\leadsto~
x_2{:}{\sf eval}\,\interp{e_2}
 ~\leadsto~
x_3{:}{\sf eval}\,\interp{e_3}
 ~\leadsto~ \cdots ~\leadsto~
x_n{:}{\sf eval}\,\interp{v}
 ~\leadsto~ 
z{:}{\sf retn}\,\interp{v}
\]

\subsection{Parallelism}
\label{sec:trans-par}

Both the basic transformation and the tail-recursive transformation
are sequential: if $x{:}{\sf eval}\,\interp{e} \leadsto^* \Delta$,
then the process state $\Delta$ contains at most one proposition ${\sf
  eval}\,\interp{e'}$ or ${\sf retn}\,\interp{v}$ that can potentially
be a part of any further transition. Put differently, the first two
operationalization transformations express deductive computation as a
concurrent computation that does not exhibit concurrency (sequential
computation being a special case of concurrent computation).

Sometimes, this is what we want: in
Section~\ref{sec:nat-ssos-adequacy} we will see that the sequential
tail-recursion-optimized abstract machine is what we want to
adequately represent the traditional on-paper abstract machines for
the call-by-value lambda calculus. In general, however, when distinct
subgoals do not have input-output dependencies (that is, when none of
subgoal $i$'s outputs are inputs to subgoal $i+1$), deductive computation
can search for subgoal $i$ and $i+1$ simultaneously, and this can 
be represented in the operationalization transformation.

In the previous transformations, our process states were structured
such that every negative proposition $A^-$ was waiting on a single
${\sf retn}$ to be computed to its left; at that point, the negative
proposition could be focused on, which effectively invokes the
continuation stored in that negative proposition. If we ignore the
first-order structure of the concurrent computation, the intermediate
states of look like this:
\[
  (\mbox{subgoal 1}), y{:}\istrue{({\sf retn} \lefti {\it cont})}
\]
Note that $(\mbox{subgoal 1})$ is intended to represent some nonempty
sequence of ordered propositions, not a single proposition. With the
parallelism-enabling transformation, subgoal 1 can even be performing
parallel search for its own subgoals:
\[
 (\mbox{subgoal 1.1}), (\mbox{subgoal 1.2}), 
   y_1{:}\istrue{({\sf retn}_{1.1} \fuse {\sf retn}_{1.2} \lefti {\it cont}_1)}, 
   y{:}\istrue{({\sf retn} \lefti {\it cont})}
\]
The two subcomputations $(\mbox{subgoal 1.1})$ and $(\mbox{subgoal
  1.2})$ are next to one another in the ordered context, but the
structure of transformed specifications ensures that the only way they
can interact is if they both finish (becoming $z_{1.1}{:}\susp{{\sf
    retn}_{1.1}}$ and $z_{1.2}{:}\susp{{\sf retn}_{1.2}}$), which will
allow us to focus on $y_1$ and begin working on the continuation ${\it
  cont}_1$. The principle at work is the same one that says that postfix
notations like Reverse Polish notation are unambiguous: there's always
only one way to reconstruct the tree of subgoals. 

To allow for the transformed programs to have parallelism, we again
add a new case to the function that transforms propositions $C$ in the
signature.  In this case, the new case will actually subsume the old
case that dealt with sequences of the form ${!}p_c^-,
A^+_{i+1},\ldots,A^+_n$; that old case is now an instance of the
general case where $i = j$. 

\begin{itemize}
\item $\opbasic{{!}p^-_{ci},\ldots,{!}p^-_{cj},A^+_{j+1},\ldots,A^+_n}{t_{n+1}}{\sigma}$
  \\
  $~ \qquad = \{ {\sf eval\_bi}\,({\sigma}t^{\it in}_i) 
                    \fuse \ldots \fuse
                 {\sf eval\_bj}\,({\sigma}t^{\it in}_j)$
  \\
  $~ \qquad \qquad (\forall\overline{x_i}\ldots\forall\overline{x_j}.\, 
     {\sf retn\_bi}\,(\sigma{t^{\it out}_i})
     \fuse \ldots \fuse 
     {\sf retn\_bj}\,(\sigma{t^{\it out}_j})$
  \\
  $~ \qquad \qquad \quad
   \lefti \opbasic{A^+_{j+1},\ldots,A^+_n}{t_{n+1}}{\sigma}) \}$\\
  {\it (where
   $p^-_{ck}$ is ${\sf bk}\,t^{\it in}_k\,t^{\it out}_k$ 
   and $FV(t_k^{\it in}) \notin (\overline{x_i} \cup \ldots \cup \overline{x_j})$ 
   for $i \leq k \leq j$)}
\end{itemize}

\noindent
Note that the second side condition on the free variables of inputs is
necessary if the resulting term is to be well-scoped, and is trivially 
satisfied in the sequential case where $i = j$. 

\begin{figure}
\fvset{fontsize=\small,boxwidth=auto}
\VerbatimInput{sls/cbv-ev-ssos-par.sls}
\caption{The parallel, tail-recursion optimized ordered abstract machine for
 call-by-value evaluation.}
\label{fig:cbv-ev-ssos-par}
\end{figure}

The result of running the natural semantics from
Figure~\ref{fig:example-transform-cbv} through the parallel and
tail-recursion optimizing ordered abstract machine is shown in
Figure~\ref{fig:cbv-ev-ssos-par}; it represents that we can
search for the subgoals $e_1 \Downarrow \lambda x.e$ and
$e_2 \Downarrow v_2$ in parallel. We cannot, of course, run either
of these subgoals in parallel with the third subgoal 
$[v_2/x]e \Downarrow v$ because the input $[v_2/x]e$ mentions the outputs
of both of the previous subgoals. 

\subsection{Correctness}
\label{sec:operationalization-correct}

The correctness of the basic, tail-recursion-optimizing, and parallel
transformations all follow from the correctness of the parallel
transformation; because the transformation is nondeterministic, the
previously presented transformations are just instances of this most
general one.

\bigskip
\begin{theorem}[No effect on the LF fragment]
  $\Psi \vdash_\Sigma t : \tau$ if and only if $\Psi
  \vdash_{\transop{\Sigma}} t : \tau$.
\end{theorem}

\begin{proof}
Straightforward induction in both directions; the transformation 
leaves the LF-relevant part of the signature unchanged.
\end{proof}

\begin{theorem}[Soundness of operationalization]
XXX Soundness
\end{theorem}

\begin{proof}
XXX Proof
\end{proof}

\begin{theorem}[Completeness of operationalization]
If all propositions in $\Gamma$ have the form 
$x{:}D$ or $z{:}\susp{p^+_{\mpers}}$, then
\begin{enumerate}
\item  
If $\slss{\Sigma}{\Psi}{\Gamma}{\susp{p^-_d}}$,
then $\slss{\transop{\Sigma}}{\Psi}{\opsubst{\Gamma}}{\susp{p^-_d}}$.
\item  
If $\slss{\Sigma}{\Psi}{\Gamma, [D]}{\susp{p^-_d}}$,
then $\slss{\transop{\Sigma}}{\Psi}{\opsubst{\Gamma}, [\opsubst{D}]}{\susp{p^-_d}}$.
\item  
If $\slss{\Sigma}{\Psi}{\Gamma}{G}$,
then $\slss{\transop{\Sigma}}{\Psi}{\opsubst{\Gamma}}{\opsubst{G}}$.
\item
If $\Delta$ matches $\frameoff{\Theta}{\Gamma}$ 
and $\slss{\Sigma}{\Psi}{\Gamma}{\susp{p^-_c}}$
(where $p^-_c = {\sf a}\,t\,s$),\\
then
$(\Psi; \tackon{\opsubst{\Theta}}{x{:}\susp{{\sf eval\_a}\,t}}) 
  \leadsto^*_{\transop{\Sigma}}
 (\Psi; \tackon{\opsubst{\Theta}}{y{:}\susp{{\sf retn\_a}\,s}})$.
\end{enumerate}
\end{theorem}

\begin{proof}
Mutual induction on the size 
of the input derivation.

The first three parts are straightforward. In part 1, we have
$\slst{\Sigma}{\Psi}{\Gamma}{\tfocusl{h}{\Sp}}{\susp{p^-_d}}$ where
either $h = x$ and $x{:}D \in \Gamma$ or else $h = {\sf r}$ and ${\sf
  c}{:}D \in \Sigma$. In either case the necessary result is
$\tfocusl{h}{\Sp'}$, where we get $\Sp'$ from the induction hypothesis
(part 2) on $\Sp$.

In part 2, we proceed by case analysis on the proposition $D$ in focus. 
The only interesting case is where $D = {!}p^-_c \lefti D'$
\begin{itemize}
\item If $D = p_d^-$, $\Sp = \tnil$ and $\tnil$ gives the desired result.

\item If $D = \forall x{:}\tau.\,D'$ or $D = p^+_{\sf
    pers} \lefti D'$, then $\Sp = (\tforalll{t}{\Sp'})$ 
  or $\Sp = (\tappl{z}{\Sp'})$ (respectively). The necessary result is
  $(\tforalll{t}{\Sp''})$ 
  or $(\tappl{z}{\Sp''})$ (respectively) where we get $\Sp''$ from the
  induction hypothesis (part 2) on $\Sp'$. 

\item If $D = {!}p^-_c \lefti D'$ and $p^-_c = {\sf a}\,t_1\,t_2$, then 
  $\Sp = (\tappl{\tbangr{\tetan{N}}}{\Sp'})$
  and $\opsubst{D} = {!}({\sf eval}\,t_1 \lefti \{ {\sf retn}\,t_2 \}) \lefti \opsubst{D'}$.

  \begin{tabbing}
  $\slst{\Sigma}{\Psi}{\Gamma}{N}{\susp{p^-_c}}$
  \` (given)
  \\
  $\slst{\Sigma}{\Psi}{\Gamma, [D]}{\Sp'}{\susp{p^-_d}}$
  \` (given)
  \\
  $T :: (\Psi; \opsubst{\Gamma}, x{:}{\sf eval_a}\,t_1)
    \leadsto^*_{\transop{E}} (\Psi; \opsubst{\Gamma}, y{:}{\sf eval_a}\,t_2)$
  \` (ind. hyp. (part 2) on $N$)
  \\
  $\slst{\transop{\Sigma}}{\Psi}{\Gamma, [\opsubst{D'}]}{\Sp''}{\susp{p_d^-}}$
  \` (ind. hyp. (part 2) on $\Sp'$)
  \\
  $\slst{\transop{\Sigma}}{\Psi}{\opsubst{\Gamma}}
    {\tlaml{\tetap{x}{\,\tlet{T}{y}}}}
    {{\sf eval\_a}\,t_1 \lefti \{ {\sf retn\_a}\,t_2 \}}$
  \` (construction)
  \\
  $\slst{\transop{\Sigma}}{\Psi}{\opsubst{\Gamma}, [\opsubst{D}]}
    {\tappl{\tbangr{(\tlaml{\tetap{x}{\,\tlet{T}{y}}})}}{\Sp'}}
    {\susp{p^-_d}}$
  \` (construction)
  \end{tabbing}
\item If $D = {!}G \lefti D'$, then $\Sp =
  (\tappl{\tbangr{\tetan{N}}}{\Sp'})$. The necessary result is
  $(\tappl{\tbangr{\tetan{N'}}}{\Sp''})$; we get $N'$ from the
  induction hypothesis (part 3) on $N$ and get $\Sp''$ from the induction
  hypothesis (part 2) on $\Sp'$.
\end{itemize}

The cases of part 3 follow the same pattern as the ones from part 2, but
without the interesting case (which is excluded from the refinement $G$);
the result follows by the induction hypothesis (part 3 or part 1). 

In part 4, we have $\slst{\Sigma}{\Psi}{\Gamma}{\tfocusl{\sf
    c}{\Sp}}{\susp{p^-_d}}$, where ${\sf r}{:}C \in \Sigma$ and the
proposition $C$ is equivalent to
$\forall{\overline{x_0}}\ldots\forall{\overline{x_n}}.\, A^+_n \lefti
\ldots \lefti A^+_1 \lefti {\sf a}\,t_0\,t_{n+1}$ as described in
Section~\ref{sec:trans-basic}. This means we can decompose $\Sp$ to
get $\sigma_i = (\overline{s_0}/\overline{x_0},\ldots,
\overline{s_i}/\overline{x_i})$ (for some terms $\overline{s_0} \ldots
\overline{s_i}$ that have the correct type in $\Psi$) and a value
$\slst{\Sigma}{\Psi}{\Gamma}{V_i}{[\sigma_i{A^+_i}]}$ for each $0 \leq
i \leq n$. We also have $t = \sigma_0{t_0}$ and $s = \sigma_n{t_{n+1}}$.

Because 
${\sf r}{:}\forall\overline{x_0}.\,{\sf eval\_a}\,t_0 \lefti \opbasic{A_1, \ldots, A_n}{t_{n+1}}{{\sf id}} \in \transop{\Sigma}$, by left-focusing
on that constant
it suffices to show that there is a $\Sp'$ such that 
$\slst{\transop\Sigma}{\Psi}{\Gamma,[
\opbasic{A_1^+, \ldots, A_n^+}{t_{n+1}}{\sigma_0}
]}{\Sp'}{\susp{\{C^+\}}}$ and a trace of the form
$T :: (\Psi; \tackon{\opsubst{\Theta}}{C^+}) \leadsto^*_{\transop{\Sigma}}
 (\Psi; \tackon{\opsubst{\Theta}}{y{:}{\sf retn\_a}\,})$. We will prove this
by induction on the length of the trace; the general statement is that
there is a $\Sp'$ such that 
$\slst{\transop\Sigma}{\Psi}{\Gamma,[
\opbasic{A_i^+, \ldots, A_n^+}{t_{n+1}}{\sigma_{i-1}}
]}{\Sp'}{\susp{\{C^+\}}}$ and a trace of the form
$T :: (\Psi; \tackon{\opsubst{\Theta}}{C^+}) \leadsto^*_{\transop{\Sigma}}
 (\Psi; \tackon{\opsubst{\Theta}}{y{:}{\sf retn\_a}\,s})$. We proceed
by case analysis on the definition of the operationalization transformation:
\begin{itemize}
\item $\opbasic{}{t_{n+1}}{\sigma_n} = \{ {\sf retn\_a}\,(\sigma_{n}{t_{n+1}}) \}$

  \bigskip
  This is a base case: $\Sp' = \tnil$ and because $\sigma_n{t_{n+1}} = s$, 
  $(\Psi; \tackon{\opsubst{\Theta}}{{\sf retn\_a}\,(\sigma_{n}{t_{n+1}})})$ decomposes
  to $(\Psi; 
  \tackon{\opsubst{\Theta}}{y{:}{\susp{{\sf retn\_a}\,(\sigma_{n}{t_{n+1}})}}})$
  in an inversion phase.
  \bigskip

\item $\opbasic{{!}{\sf a}\,t^{\it in}_n\,t_{n+1}}{t_{n+1}}{\sigma_{n-1}} 
  = \{{\sf eval\_a}\,(\sigma_{n-1}{t^{\it in}_n})\}$

  \bigskip
  We are given a value 
  $\slst{\Sigma}{\Psi}{\Gamma}{\tbangr{N}}
   {[{\bang}{\sf a}\,\sigma_n{t^{\it in}_n}\,\sigma_n{t_{n+1}}]}$;
  observe that $\sigma_{n-1}{t_n^{\it in}} = \sigma_n{t_n^{\it in}}$

  \smallskip
  This is also a base case of the inner induction: $\Sp' = \tnil$ and
  $(\Psi; 
  \tackon{\opsubst{\Theta}}
  {{\sf eval\_a}\,(\sigma_{n}{t^{\it in}_n})})$
  decomposes to 
  $(\Psi; 
  \tackon{\opsubst{\Theta}}
  {x_{n}{:}{\susp{{\sf eval\_a}\,(\sigma_{n}{t^{\it in}_n})}}})$.
  We must demonstrate a trace the rest of the way to
  $(\Psi; \tackon{\opsubst{\Theta}}{y{:}{\sf retn\_a}\,s})$. Because 
  $s = \sigma_n{t_{n+1}}$, this is established by the 
  outer induction hypothesis (part 4) on $N$. 
  \bigskip

\item $\opbasic{p^+_\mpers,A^+_{i+1},\ldots,A^+_n}{t_{n+1}}{\sigma_{i-1}} 
  = \forall \overline{x_i}.\, \sigma_{i-1}{p^+}_\mpers \lefti \opbasic{A^+_i,\ldots,A^+_n}{t_{n+1}}{\sigma_{i-1}}$

  \begin{tabbing}
  $\slst{\Sigma}{\Psi}{\Gamma}{z}{[\sigma_i{p^+_\mpers}]}$
  \` (given) 
  \\
  $\sigma_i = (\sigma_{i-1}, \overline{s_i}/\overline{x_i})$.
  \` (definition of $\sigma_i$)
  \\
  $\slst{\transop{\Sigma}}{\Psi}{\Gamma, [\opbasic{A^+_i,\ldots,A^+_n}{t_{n+1}}{\sigma_{i}}]}{\Sp'}{\susp{\{ C^+ \} }}$
  \` (by inner ind. hyp.)
  \\
  $T :: (\Psi; \tackon{\opsubst{\Theta}}{C^+}) \leadsto^*_{\transop{\Sigma}}
   (\Psi; \tackon{\opsubst{\Theta}}{y{:}{\sf retn\_a}\,s})$
  \` (by inner ind. hyp.)
  \\
  $\slst{\transop{\Sigma}}{\Psi}
    {\Gamma, [\forall \overline{x_i}.\, \sigma_{i-1}{p^+}_\mpers 
                \lefti \opbasic{A^+_i,\ldots,A^+_n}{t_{n+1}}{\sigma_{i-1}}]}
    {\left(\tforalll{\overline{s_i}}{\tappl{z}{\Sp'}}\right)}{\susp{\{ C^+ \}}}$
  \\ 
  \` (construction)
  \end{tabbing}

\item $\opbasic{{!}p^-_{ci},\ldots,{!}p^-_{cj},A^+_{j+1},\ldots,A^+_n}{t_{n+1}}{\sigma_{i-1}}$
  \\
  $~ \qquad = \{ {\sf eval\_bi}\,({\sigma_{i-1}}t^{\it in}_i) 
                    \fuse  \ldots \fuse
                 {\sf eval\_bj}\,({\sigma_{i-1}}t^{\it in}_j)$
  \\
  $~ \qquad \qquad (\forall\overline{x_i}\ldots\forall\overline{x_j}.\, 
     {\sf retn\_bi}\,(\sigma_{i-1}{t^{\it out}_i})
     \fuse \ldots \fuse 
     {\sf retn\_bj}\,(\sigma_{i-1}{t^{\it out}_j})$
  \\
  $~ \qquad \qquad \quad
   \lefti \opbasic{A^+_{j+1},\ldots,A^+_n}{t_{n+1}}{\sigma_{i-1}}) \}$\\
  {\it (where
   $p^-_{ck}$ is ${\sf bk}\,t^{\it in}_k\,t^{\it out}_k$ 
   and $FV(t_k^{\it in}) \notin (\overline{x_i} \cup \ldots \cup \overline{x_j})$ 
   for $i \leq k \leq j$)}

  \bigskip
  Let $\Sp = \tnil$. It then suffices to show that there is a trace
\begin{align*}
    &(\Psi, \opsubst{\Theta} \tackonstart
        x_i{:}\susp{{\sf eval\_bi}\,({\sigma_{i-1}}t^{\it in}_i)}, \ldots,
    x_j{:}\susp{{\sf eval\_bj}\,({\sigma_{i-1}}t^{\it in}_j)},
  \\
  & \qquad\qquad y_{ij}{:}(\forall\overline{x_i}\ldots\forall\overline{x_j}.\, 
     {\sf retn\_bi}\,(\sigma_{i-1}{t^{\it out}_i})
     \fuse \ldots \fuse 
     {\sf retn\_bj}\,(\sigma_{i-1}{t^{\it out}_j})\\
  & \qquad\qquad\qquad
      \lefti \opbasic{A^+_{j+1},\ldots,A^+_n}{t_{n+1}}{\sigma_{i-1}})\,\mtrue
       \tackonstop)
  \\
  & \quad \leadsto^*_{\transop{\Sigma}} 
     (\Psi; \tackon{\opsubst{\Theta}}{y{:}\susp{{\sf retn\_a}\,s}})
\end{align*}

  \begin{tabbing}
  $\slst{\Sigma}{\Psi}{\Gamma}{\tbangr{N_k}}{[{!}{{\sf bk}\,({\sigma_k}t_k^{\it in})\,({\sigma_k}t_k^{\it out})}]}$ \quad $(i \leq k \leq j)$
  \` (given) 
  \\
  $\slst{\Sigma}{\Psi}{\Gamma}{\tbangr{N_k}}{[{!}{{\sf bk}\,({\sigma_{i-1}}t_k^{\it in})\,({\sigma_j}t_k^{\it out})}]}$ \quad $(i \leq k \leq j)$
  \` (condition on translation, defn. of $\sigma_k$)
  \\
  $T :: (\Psi$\=$, \opsubst{\Theta} \tackonstart
        x_i{:}\susp{{\sf eval\_bi}\,({\sigma_{i-1}}t^{\it in}_i)}, \ldots,
    x_j{:}\susp{{\sf eval\_bj}\,({\sigma_{i-1}}t^{\it in}_j)},$\\
  \>$~ \qquad y_{ij}{:}(\forall\overline{x_i}\ldots\forall\overline{x_j}.\, 
     {\sf retn\_bi}\,(\sigma_{i-1}{t^{\it out}_i})
     \fuse \ldots \fuse 
     {\sf retn\_bj}\,(\sigma_{i-1}{t^{\it out}_j})$\\
  \>$~ \qquad\qquad
      \lefti \opbasic{A^+_{j+1},\ldots,A^+_n}{t_{n+1}}{\sigma_{i-1}})\,\mtrue
       \tackonstop)$\\
  $~ \qquad \leadsto^*_{\transop{\Sigma}} 
        (\Psi$\=$, \opsubst{\Theta} \tackonstart
        y_i{:}\susp{{\sf retn\_bi}\,({\sigma_{j}}t^{\it in}_i)}, \ldots,
    y_j{:}\susp{{\sf eval\_bj}\,({\sigma_{j}}t^{\it in}_j)},$\\
  \>$~ \qquad y_{ij}{:}(\forall\overline{x_i}\ldots\forall\overline{x_j}.\, 
     {\sf retn\_bi}\,(\sigma_{i-1}{t^{\it out}_i})
     \fuse \ldots \fuse 
     {\sf retn\_bj}\,(\sigma_{i-1}{t^{\it out}_j})$\\
  \>$~ \qquad\qquad
      \lefti \opbasic{A^+_{j+1},\ldots,A^+_n}{t_{n+1}}{\sigma_{i-1}})\,\mtrue
       \tackonstop)$\\
  \` (by outer ind. hyp. (part 4) on each of the $N_k$ in turn)
  \\
  $\slst{\transop{\Sigma}}{\Psi}
     {\Gamma,[\opbasic{A^+_{j+1},\ldots,A^+_n}{t_{n+1}}{\sigma_{j}}]}
     {\Sp'}{\susp{\{ C^+ \}}}$  \` (by inner ind. hyp.)
  \\
  $T' :: (\Psi, \tackon{\opsubst{\Theta}}{C^+}) \leadsto^*_{\transop{\Sigma}}
        (\Psi, \tackon{\opsubst{\Theta}}{y{:}\susp{{\sf retn\_a}\,s}})$ 
   \` (by inner ind. hyp.)
  \end{tabbing}

  The construction 
  $\left(T; \tstep{\mkpat{C^+}}{y_{ij}}{(\tforalll{\overline{s_i}\ldots\overline{s_j}}{\tappl{(\tfuser{y_i}{\tfuser{\ldots}{y_j}})}{\Sp'}})}; T'\right) $
  is then a trace of the correct type.
  \bigskip

\item $\opbasic{{!}G,A^+_{i+1},\ldots,A^+_n}{t_{n+1}}{\sigma} = \forall
  \overline{x_i}.\, {!}\sigma\opsubst{G} \lefti
  \opbasic{A^+_i,\ldots,A^+_n}{t_{n+1}}{\sigma}$

  \begin{tabbing}
  $\slst{\Sigma}{\Psi}{\Gamma}{\tbangr{N}}{[{!}\sigma_i{G}]}$
  \` (given) 
  \\
  $\slst{\transop{\Sigma}}{\Psi}{\opsubst{\Gamma}}{N'}{\sigma_i{\opsubst{G}}}$
  \` (by outer ind. hyp. (part 3)  on $N$) 
  \\
  $\sigma_i = (\sigma_{i-1}, \overline{s_i}/\overline{x_i})$.
  \` (definition of $\sigma_i$)
  \\
  $\slst{\transop{\Sigma}}{\Psi}{\Gamma, [\opbasic{A^+_i,\ldots,A^+_n}{t_{n+1}}{\sigma_{i}}]}{\Sp'}{\susp{\{ C^+ \} }}$
  \` (by inner ind. hyp.)
  \\
  $T :: (\Psi; \tackon{\opsubst{\Theta}}{C^+}) \leadsto^*_{\transop{\Sigma}}
   (\Psi; \tackon{\opsubst{\Theta}}{y{:}{\sf retn\_a}\,s})$
  \` (by inner ind. hyp.)
  \\
  $\slst{\transop{\Sigma}}{\Psi}
    {\Gamma, [\forall \overline{x_i}.\, {!}(\sigma_{i-1}{G})
                \lefti \opbasic{A^+_i,\ldots,A^+_n}{t_{n+1}}{\sigma_{i-1}}]}
    {\left(\tforalll{\overline{s_i}}{\tappl{\tbangr{N}}{\Sp'}}\right)}{\susp{\{ C^+ \}}}$
  \\ 
  \` (construction)
  \end{tabbing}
\end{itemize}

\noindent
This completes the inner induction in the fourth part, and hence
the proof.
\end{proof}

\section{Logical transformation: defunctionalization}
\label{sec:defunctionalization}

Defunctionalization is a procedure for turning higher-order
concurrent \sls~specifications into first-order concurrent
\sls~specifications. It is based on the following intuitions:
if $A^-$ is a closed negative proposition
of the form $\forall \overline{x}.\,A^+_1 \lefti \{ A^+_2 \}$
and we have a single-step transition 
$(\Psi; \tackon{\Theta}{y{:}\istrue{A^-}}) 
 \leadsto_{\Sigma} 
 (\Psi; \Delta')$
in an \sls~specification (witnessed by the step 
$(\tstep{\mkpat{A^+_2}}{y}{(\tforalll{\overline{t}}{\tappl{V}{\tnil}})})$), 
then we can define an augmented signature
\begin{align*}
\Sigma' = ~ & \Sigma, 
\\    ~~ & {\sf cont} : {\sf prop\,ord}, 
\\    ~~ & {\sf run\_cont} : \forall{\overline x}.\,p^+_\mtrue \fuse {\sf cont} \lefti \{ A^+ \}
\end{align*}
and it is the case that 
$(\Psi; \tackon{\Theta}{y{:}\susp{\sf cont}}) 
 \leadsto_{\Sigma'} 
 (\Psi; \Delta')$
as well; this new transition is witnessed by the step
$(\tstep{\mkpat{A^+}}{\sf run\_cont}{(\tappl{(\tfuser{V}{y})}{\tnil})})$.

More generally, if we are allowed to extend the signature and $A^-$
falls into the very specific form we have
specified,\footnote{Obviously, the restriction to propositions $A^-$
  of the form $\forall \overline{x}.\,p^+_\mtrue \lefti \{ A^+ \}$ is
  overly specific and designed to apply specifically to the output of
  operationalization, but we will not consider a generalization here.
  Conceptually, it is not complicated to consider a similar operation
  on other propositions, but it is difficult to elegantly describe the
  more general transformation due our use of ordered logic, and we do
  not currently need the more general transformation.}  we can create
a new ordered atomic proposition to do a negative proposition's
job. As long as $\Delta = \tackon{\Theta}{x{:}\susp{\sf cont}}$ and
${\sf cont}$ does not appear in $\Theta$, then $[{\downarrow}A^- /{\sf
  cont}]\Delta \leadsto_\Sigma [{\downarrow}A^- /{\sf cont}]\Delta'$
if and only if $\Delta \leadsto_{\Sigma'} \Delta'$. \footnote{Recall
  from Section~\ref{sec:framework-substprop} that we treat
  %
  $[{\downarrow}A^-/{\sf cont}](\tackon{\Theta}{z{:}\istrue{\susp{\sf cont}}})$
  %
  as being equal to the context in which we {\it first} perform the
  straightforward substitution, giving us
  $(\tackon{\Theta}{z{:}\istrue{{\downarrow}A^-}})$, and then {\it
    second} apply invertible rules, giving us
  $(\tackon{\Theta}{z'{:}\istrue{A^-}})$.} 

We need not restrict ${\sf cont}$ to just a single appearance
suspended in the process state, however. Multiple instances of ${\sf
  cont}$ can appear in the process state without a problem.  It is
similarly unproblematic for ${\sf cont}$ to appear in the monadic head
of some other rule in the process state, as the appearance of an
ordered atomic proposition in a monadic head will not effect the
existence of any transition, but may cause the ordered atomic
proposition to become a suspended ordered proposition in the process
state after the transition. 

By the same reasoning, it is similarly
unproblematic for ${\sf cont}$ to appear in the head of a rule in the
signature.  Therefore, we can replace propositions in the monadic heads
of rules in the signature, like this one:
\begin{align*}
& \Sigma, \\
& {\sf r} : {\sf a} \lefti \{ {\sf b} \fuse {\uparrow}({\sf c} \lefti \{ {\sf d} \fuse {\uparrow}({\sf e} \lefti \{ {\sf f} \}) \}) \}
\intertext{to produce a signature that looks like this:}
& \Sigma, \\
& {\sf cont1} : {\sf prop\,ord}, \\
& {\sf r1} : {\sf c} \fuse {\sf cont1} \lefti \{ {\sf d} \fuse {\uparrow}({\sf e} \lefti \{ {\sf f} \}) \}, \\
& {\sf r} : {\sf a} \lefti \{ {\sf b} \fuse {\sf cont1} \}
\intertext{and the process can be iterated to obtain 
a fully first-order signature:}
& \Sigma, \\
& {\sf cont2} : {\sf prop\,ord}, \\
& {\sf r2} : {\sf e} \fuse {\sf cont2} \lefti \{ {\sf f} \}, \\
& {\sf cont1} : {\sf prop\,ord}, \\
& {\sf r1} : {\sf c} \fuse {\sf cont1} \lefti \{ {\sf d} \fuse {\sf cont2} \}, \\
& {\sf r} : {\sf a} \lefti \{ {\sf b} \fuse {\sf cont1} \}
\end{align*}
This propositional transformation is similar to the one proposed by
Miller in~\cite{miller02higherorder}, where the new propositions were
introduced to hide the internal states of processes.

We can go further and allow $A^-$ to contain free variables if
$A^- = [t_1/y_1]\ldots[t_m/x_m]B^-$ where $B^- = \forall
\overline{x}.\,B_1^+ \lefti \{ B_2^+ \}$ has only the variables
$\overline{y} = y_1\ldots y_m$ free.  In this more general case, we
can revise the signature as follows:
\begin{align*}
\Sigma'' = ~ & \Sigma,
\\    ~~ & {\sf cont} : 
       \Pi x_y{:}\tau_1\ldots \Pi y_m{:}\tau_m.\, {\sf prop\,ord},
\\    ~~ & {\sf run\_cont} : \forall \overline{x}.\,\forall \overline{y}.\,
       p^+_\mtrue \fuse {\sf cont}\,\overline{y} \lefti \{ B^+ \}
\end{align*}
With this revision, we maintain that
%
$[{\downarrow}B^-/{\sf cont}\,\overline{x}]\Delta \leadsto_{\Sigma}
[{\downarrow}B^-/{\sf cont}\,\overline{x}]\Delta'$ if and only if
$\Delta \leadsto_{\Sigma''} \Delta'$ (as long as propositions of the
form ${\sf cont}\,\overline{t}$ only appear suspended in the process
state or in the monadic heads of rules that appear in the process
state).\robnote{The process of proving this is mostly an issue of
  stating it precisely, which is a pain. I'd appreciate feedback as to
  whether this seems clear or whether I need to write out the detailed
  proof.}

The one twist we make to the defunctionalization transformation is
that, instead of introducing a new ordered atomic proposition ${\sf
  cont}\,\overline{t}$ for each iteration of the defunctionalization
procedure, we introduce a single type $({\sf frame} : {\sf type})$ and a
single atomic proposition $({\sf cont} : {\sf frame} \rightarrow {\sf
  prop\,ord})$. Then, each iteration of the defunctionalization
procedure produces a new constant with type $\Pi x_y{:}\tau_1\ldots
\Pi y_m{:}\tau_m.\, {\sf frame}$ instead of a new atomic proposition
with kind $\Pi x_y{:}\tau_1\ldots \Pi y_m{:}\tau_m.\, {\sf
  prop\,ord}$.  Operationally, these two approaches are equivalent,
though the approach using frames requires us to disallow variables
that can construct new terms of type ${\sf frame}$ from appearing in
the variable context $\Psi$.

\begin{figure}
\fvset{fontsize=\small,boxwidth=229pt}
\VerbatimInput{sls/cbv-ev-ssos-fun.sls}
\caption{A first-order ordered abstract machine semantics for CBV
  evaluation.}
\label{fig:cbv-ev-ssos-fun}
\end{figure}

Using defunctionalization procedure outlined above, we obtain the
first-order specification in Figure~\ref{fig:cbv-ev-ssos-fun} from the
higher-order specification in Figure~\ref{fig:cbv-ev-ssos-tail}, which
was in turn derived from the natural semantics for CBV evaluation by
operationalization with tail-recursion optimization.

% As long as 
% $({\sf cont}\,t_1\ldots t_n)$ only appears in $\Delta$ as a 
% suspended atomic proposition, then it is the case that
% $[{\downarrow}(B^-\,x_1\ldots x_n)
%     /{\sf cont}\,x_1\ldots x_n]\Delta 
%  \leadsto_\Sigma
%  [{\downarrow}(B^-\,x_1\ldots x_n)
%     /{\sf cont}\,x_1\ldots\,x_n]\Delta'$ 
% if and only if 
% %
% $\Delta \leadsto_{\Sigma''} \Delta'$.\footnote{Recall from
%   Section~\ref{sec:framework-substprop} that
%   $[{\downarrow}(B^-\,x_1\ldots x_n)/({\sf cont}\,x_1\ldots
%   x_n)](z{:}\susp{{\sf cont}\,t_1\ldots t_n})$ as being equal to the
%   context in which we substitue and then apply invertible rules, i.e.
%   $z{:}\istrue{B^-\,t_1\ldots t_n}$}

% In addition to allowing these newly introduced 
% atomic propositions to appear suspended in the context, it is not 
% a problem to allow them to appear in the heads of monadic clauses. 
% This means that we can 

%  monadic
% clauses, there is no 

% The defunctionalization transformation then applies the same reasoning
% to signatures: if a proposition ${\downarrow}A^-$ appears in the monadic
% head of some rule, 

%  $A^- = B^-\,t_1\,t_2\,t_3$

% This is even
% true if $A^-$ has free variables: we can always define a closed
% $B^- : $




\section{Adequacy with abstract machines}
\label{sec:nat-ssos-adequacy}

I claim that the four-rule abstract machine specification given at the
beginning of this chapter is adequately represented by the derived
\sls~specification in Figure~\ref{fig:cbv-ev-ssos-fun}. For terms and
for deductive computations, adequacy is a well-understood concept: we
know what it means to define an adequate encoding function $\interp{e}
= t$ from ``on-paper'' terms $e$ with (potentially) variables
$x_1,\ldots,x_n$ free to LF terms $t$ where $x_1{:}{\sf
  exp},\ldots,x_n{:}{\sf exp} \vdash t : {\sf exp}$, and we know what
it means to adequately encode the judgment $e \Downarrow v$ as a
negative atomic \sls~proposition ${\sf ev}\,\interp{e}\,\interp{v}$
and to encode derivations of this judgment to \sls~terms $N$ where
$\slst{\Sigma}{\cdot}{\cdot}{N}{\susp{{\sf
      ev}\,\interp{e}\,\interp{v}}}$
\cite{harper93framework,harper07mechanizing}. What does it mean to
adequately represent machine states as process states (that is,
substructural contexts) and to encode a transition system as a 
concurrent \sls~specification? 

The answer given in the literature by Cervesato et
al.~\cite{cervesato02concurrent} and by
Schack-Nielsen~\cite{schacknielsen07induction} has three steps. The
first step is to, define an interpretation function from states $s$
and stacks $k$ to process states $\Delta$, so that, for example, the
state
\[
((\ldots({\sf halt}; \Box\,e_1)\ldots); (\lambda x.e_n)\,\Box) \lhd v
\]
is interpreted as the process state
\[
y{:}\susp{{\sf retn}\,\interp{v}}, ~~
x_n{:}\susp{{\sf cont}\,({\sf app2}\,\lambda x.\interp{e_n})}, ~~
\ldots, ~~
x_1{:}\susp{{\sf cont}\,({\sf app1}\,\interp{e_1})}, ~~
\]
The second step is to, prove a preservation-like adequacy theorem. Let
$\Sigma\ref{fig:cbv-ev-ssos-fun}$ be the signature from
Figure~\ref{fig:cbv-ev-ssos-fun}: we show that if state $s$ is
interpreted and $\Delta$ and $\Delta
\leadsto_{\Sigma\ref{fig:cbv-ev-ssos-fun}} \Delta'$, then there is a
state $s'$ such that $s'$ is interpreted as $\Delta'$. Then we can
prove the main adequacy result: that the interpretation of state $s$
steps to the interpretation of state $s'$ if and only if $s \mapsto
s'$.

I believe that the approach to adequacy given in previous work is
unsatisfactory because the interpretation of process states into
contexts is 1-to-1 but not onto (and therefore not invertible).  This
means that there is no {\it internal} notion of what it means for a
process state to encode a state $s$ or a stack $k$. By analogy,
``having type ${\sf exp}$'' captures what it means for an LF term
encode an expression and ``having type ${\sf
  ev}\,\interp{e}\,\interp{v}$'' captures what it means for an
\sls~term to encode a derivation of $e \Downarrow v$.

In this section, we will present a different three-part approach that
addresses this perceived deficiency. First, we create a signature
$\Sigma\sf gen$ that encodes well-formed states: the $\Delta$ such
that $x{:}\susp{\sf gen} \leadsto^*_{\Sigma\sf gen} \Delta$ and ${\sf
  gen} \notin \Delta$ are in a bijection with the states $s$
(Section~\ref{sec:nat-ssos-adequacy-gen}). This gives us the internal
notion of what it means to encode a process state, which is what we
were previously lacking. Second, we prove the preservation-like
property from before. The difference is that this can now be stated
formally as a property of \sls~specifications: if $x{:}\susp{\sf gen}
\leadsto^*_{\Sigma\sf gen} \Delta$ and $\Delta
\leadsto_{\Sigma\ref{fig:cbv-ev-ssos-fun}} \Delta'$, then $x{:}
\susp{\sf gen} \leadsto^*_{\Sigma\sf gen} \Delta'$
(Section~\ref{sec:nat-ssos-adequacy-pres}). The structure of this
theorem is critical, a point that we will consider in greater depth in
Part III of this thesis. Finally, the third step is the same as it was
in other approaches: we prove that the interpretation of state $s$
steps to the interpretation of state $s'$ if and only if $s \mapsto s'$.

\subsection{Adequacy of states}
\label{sec:nat-ssos-adequacy-gen}

Our first goal is to describe a signature $\Sigma\sf gen$ with the
property that if $x{:}\susp{\sf gen} \leadsto^*_{\Sigma\sf gen}
\Delta$ and ${\sf gen} \notin \Delta$ then $\Delta$. A well-formed
process state represting a state $k \rhd e$ has the form
\[
y{:}\susp{{\sf eval}\,\interp{e}}, ~~
x_n{:}\susp{{\sf cont}\,\interp{f_n}}, ~~
\ldots, ~~
x_1{:}\susp{{\sf cont}\,\interp{f_1}}
\]
where $\interp{\Box\,e_2} = {\sf app1}\,\interp{e_2}$ and
$\interp{(\lambda x.e)\,\Box} = {\sf app2}\,(\lambda.\interp{e})$. 
A well-formed process state representing a state $k \lhd e$ has 
the same form, but with a suspended ${\sf retn}\,\interp{v}$ instead
of ${\sf eval}\,\interp{e}$. 

The simplest \sls~signature that encodes this structure essentially
has the structure of a Chomsky normal form describing well-formed
contexts, with two unary productions and one unary production.

\smallskip
\fvset{fontsize=\small,boxwidth=229pt}
\VerbatimInput{sls/cbv-ev-ssos-gen.sls}
\smallskip

\noindent In addition to the four declarations above, the full
signature $\Sigma\sf gen$ includes all the type, proposition, and
constant declarations from Figure~\ref{fig:cbv-ev-ssos-fun}, but none
of the rules.

Note that this specification most definitely is {\it not} well-moded.
{\it Generative signatures} such as this one are not generally moded,
and we don't think about traces under these signatures as 
necessarily being concurrent computations in the same way we think
about ordered abstract machines being concurrent computations. That is,
rather than thinking of traces in these signatures being produced by
different computations, such as the computational content of the adequacy
theorem:

\bigskip
\begin{theorem}[Adequacy of states]~
\label{thm:adequacy-states}
\begin{itemize}
\item There is a bijection (up to the renaming of variables in the context) 
  between states $s$ and contexts $\Delta$ such that
  $x{:}\susp{\sf gen} \leadsto^*_{\Sigma\sf gen} \Delta$ where 
  ${\sf gen} \notin \Delta$.
\item There is a bijection (up to the renaming of variables in the context) 
  between stacks $k$ and frames $\Theta$ such that $x{:}\susp{\sf
    gen} \leadsto^*_{\Sigma\sf gen} \tackon{\Theta}{x'{:}{\sf gen}}$.
\end{itemize}
\end{theorem}

\begin{proof}
We will give only the two translation from the ``on paper''
semantic artifacts (states $s$ and stacks $k$) to traces:
\begin{itemize}
\item $\interp{s},$ which outputs
a trace $T$ with type $x{:}\susp{\sf gen} \leadsto_{\Sigma\sf gen} \Delta$
where ${\sf gen} \not\in \Delta$, and 
\item $\interp{k}$, which outputs
both a trace $T$ with type 
$x{:}\susp{\sf gen}
  \leadsto_{\Sigma\sf gen} \tackon{\Theta}{x'{:}\susp{\sf gen}}$ where
${\sf gen} \not\in \Delta$ and the variable name $x'$ of the resulting
${\sf gen}$ proposition (which may be the same as $x$). Rather than 
representing this output explicitly, we just assume it is always
named $x'$ in the definition below.  
\end{itemize}
Note that both functions builds contexts only indirectly by building 
traces; similarly, the inverses of these functions are defined by induction
on the structure of traces, not on the structure of contexts.
\begin{tabbing}
~~ \= \qquad\quad\qquad \= $~ :: ~$ \=\kill
\> $\interp{k \rhd e} = \interp{k}; 
     \tstep{z}{\sf gen/eval}{(\tforalll{\interp{e}}{(\tappl{x'}{\tnil})})}$
\\ \>\> $~ :: ~$ 
  \> $x{:}\susp{\sf gen} 
       \leadsto^*_{\Sigma\sf gen} \tackon{\Theta}{x'{:}\susp{\sf gen}} 
       \leadsto_{\Sigma\sf gen} 
          \tackon{\Theta}{z{:}\susp{{\sf eval}\,\interp{e}}}$
\\[4pt]
\> $\interp{k \lhd v} = \interp{k}; 
     \tstep{z}{\sf gen/retn}{(\tforalll{\interp{v}}{(\tappl{x'}{\tnil})})}$
\\ \>\> $~ :: ~$ 
  \> $x{:}{\sf gen} 
       \leadsto^*_{\Sigma\sf gen} \tackon{\Theta}{x'{:}\susp{\sf gen}} 
       \leadsto_{\Sigma\sf gen} 
          \tackon{\Theta}{z{:}\susp{{\sf retn}\,\interp{v}}}$
\\[4pt]
\> $\interp{{\sf halt}} = \emptytrace$
\> $~ :: ~$
  \> $x{:}\susp{\sf gen} \leadsto^* x{:}\susp{\sf gen}$
\\[4pt]
\> $\interp{k; \Box\,e_2} = \interp{k}; \tstep{z, x''}{\sf gen/cont}
     {(\tforalll{{\sf app1}\,\interp{e_2}}{(\tappl{x'}{\tnil})})}$
\\ \>\> $~ :: ~$
 \> $x{:}\susp{\sf gen}
       \leadsto^*_{\Sigma\sf gen} \tackon{\Theta}{x'{:}\susp{\sf gen}}
       \leadsto_{\Sigma\sf gen} \tackon{\Theta}
            {\mkconj
               {z{:}\susp{{\sf cont}\,({\sf app1}\,\interp{e_2})}}
               {x''{:}\susp{\sf gen}}}$
\\[4pt]
\> $\interp{k; (\lambda x.e)\,\Box} = \interp{k}; \tstep{z, x''}{\sf gen/cont}
     {(\tforalll{{\sf app1}\,\interp{e_2}}{(\tappl{x'}{\tnil})})}$
\\ \>\> $~ :: ~$
 \> $x{:}\susp{\sf gen}
       \leadsto^*_{\Sigma\sf gen} \tackon{\Theta}{x'{:}\susp{\sf gen}}
       \leadsto_{\Sigma\sf gen} \tackon{\Theta}
            {\mkconj
               {z{:}\susp{{\sf cont}\,({\sf app2}\,\lambda x.\interp{e})}}
               {x''{:}\susp{\sf gen}}}$
\end{tabbing}
To complete the theorem, it is necessary to show that the two encoding
functions are one-to-one and onto. This can be done by demonstrating
the existence of a function $\interp{T}^{-1}_s = s$ from traces $T$
with type $x{:}\susp{\sf gen} \leadsto_{\Sigma\sf gen} \Delta$ where
${\sf gen} \notin \Delta$ to states $s$ and a function
$\interp{T}^{-1}_k = k$ from traces $T$ with type $x{:}\susp{\sf gen}
\leadsto_{\Sigma\sf gen} \tackon{\Theta}{x'{:}{\sf gen}}$ (where $x$
and $x'$ may be the same) to stacks $k$ and then showing that
the functions compose to the identity in both directions. In this case,
that proof is tedious but straightforward.
\end{proof}

Note that two traces $T :: x{:}\susp{\sf gen} \leadsto^*_{\Sigma\sf
  gen} \Delta$ and $T' :: x{:}\susp{\sf gen} \leadsto^*_{\Sigma\sf
  gen} \Delta'$ are distinct if and only if the contexts $\Delta$ and
$\Delta'$ are distinct. Therefore, we can equivalently see adequacy as
a bijection between traces and abstract machine states $s$ or as a
bijection between contexts and abstract machine states $s$. In a
situation where this 1-to-1 correspondence between states and traces
did not exist (because two traces generated the same context), it is
not entirely clear whether it would be preferable to define adquecy in
terms of contexts or in terms of traces.

\subsection{Preservation}
\label{sec:nat-ssos-adequacy-pres}

Before we prove that the concurrent system from
Figure~\ref{fig:cbv-ev-ssos-fun} adequately represents the transition
system from the beginning of the chapter, we must show that our
criteria for context well-formedness is actually preserved by the
concurrent computations in Figure~\ref{fig:cbv-ev-ssos-fun}. This is
part of the adequacy argument, but because we state it in terms of the
generative signature $\Sigma\sf gen$, it is also a reasonable
standalone theorem entirely about of \sls~specifications. We will
return to theorems of this form in Part III of this thesis.

\bigskip
\begin{theorem}[Generation by $\Sigma\sf gen$ is invariant under
 $\Sigma\ref{fig:cbv-ev-ssos-fun}$]\label{thm:adequate-pres}~\\
  If $x{:}\susp{\sf gen} \leadsto^*_{\Sigma\sf gen} \Delta$ and
  $\Delta \leadsto_{\Sigma\ref{fig:cbv-ev-ssos-fun}} \Delta'$, then
  $x{:} \susp{\sf gen} \leadsto^*_{\Sigma\sf gen} \Delta'$
\end{theorem}

\begin{proof}
  Primarily by enumeration of the possible synthetic transitions of
  $\Sigma\ref{fig:cbv-ev-ssos-fun}$ and secondarily by case analysis
  on the structure of the trace $T :: x{:}\susp{\sf gen}
  \leadsto^*_{\Sigma\sf gen} \Delta$.

  \begin{itemize}
  \item $\tstep{z}{\sf ev/lam}{(\tforalll{\lambda x.e\,x}
                                {(\tappl{y}{\tnil})})}$

    \qquad $:: \frameoff{\Theta}
                 {y{:}\susp{{\sf eval}\,({\sf lam}\,\lambda x.e\,x)}}
               \leadsto
               \tackon{\Theta}
                 {z{:}\susp{{\sf retn}\,({\sf lam}\,\lambda x.e\,x)}} $

    \medskip

    $T = T'; \tstep{y}{\sf gen/eval}{(\tforalll{{\sf lam}\,\lambda x.e\,x}{\tappl{x'}{\tnil}})}$,\\
    so we construct\\
    $T'; \tstep{z}{\sf gen/retn}{(\tforalll{{\sf lam}\,\lambda x.e\,x}{\tappl{x'}{\tnil}})}$

    \medskip

  \item $\tstep{z_1, z_2}{\sf ev/app}{(\tforalll{e_1}
                                {\tforalll{e_2}{(\tappl{y}{\tnil})}})}$

    \qquad $:: \frameoff{\Theta}
                 {y{:}\susp{{\sf eval}\,({\sf app}\,e_1\,e_2)}}
               \leadsto
               \tackon{\Theta}
                 {\mkconj
                  {z_1{:}\susp{{\sf eval}\,e_1}}
                  {z_2{:}\susp{{\sf cont}\,({\sf app1}\,e_2)}}} $

    \medskip

    $T = T'; \tstep{y}{\sf gen/eval}{(\tforalll{{\sf app}\,e_1\,e_2}{\tappl{x'}{\tnil}})}$,\\
    so we construct\\
    $T'; 
     \tstep{z', z_2}{\sf gen/cont}{(\tforalll{{\sf app1}\,e_2}{(\tappl{x'}{\tnil})})};
     \tstep{z_1}{\sf gen/eval}{(\tforalll{e_1}{\tappl{z'}{\tnil}})}$

    \medskip


  \item $\tstep{z_1, z_2}{\sf ev/app1}{(\tforalll{\lambda x.e\,x}
                       {\tforalll{e_2}{(\tappl{\tfuser{y_1}{y_2}}{\tnil})}})}$

    \qquad $:: \frameoff{\Theta}
                 {\matchconj
                  {y_1{:}\susp{{\sf retn}\,({\sf lam}\,\lambda x.e\,x)}}
                  {y_2{:}\susp{{\sf cont}\,({\sf app1}\,e_2)}}}$

    \qquad\qquad
               $\leadsto
               \tackon{\Theta}
                 {\mkconj
                  {z_1{:}\susp{{\sf eval}\,e_2}}
                  {z_2{:}\susp{{\sf cont}\,({\sf app2}\,(\lambda x.e\,x))}}} $

    \medskip

    $T =
     T'; 
     \tstep{y', y_2}{\sf gen/cont}{(\tforalll{{\sf app1}\,e_2}{\tappl{x'}{\tnil}})};
     \tstep{y_1}{\sf gen/retn}{(\tforalll{{\sf lam}\,\lambda x.e\,x}{(\tappl{y'}{\tnil})})}$,\\
    so we construct\\
    $T'; 
     \tstep{z', z_2}{\sf gen/cont}{(\tforalll{{\sf app2}\,\lambda x.e\,x}{(\tappl{x'}{\tnil})})};
     \tstep{z_1}{\sf gen/eval}{(\tforalll{e_2}{\tappl{z'}{\tnil}})}$

    \medskip

  \item $\tstep{z}{\sf ev/app2}{(\tforalll{v_2}
                       {\tforalll{\lambda x.e\,x}
                         {(\tappl{\tfuser{y_1}{y_2}}{\tnil})}})}$

    \qquad $:: \frameoff{\Theta}
                 {\matchconj
                  {y_1{:}\susp{{\sf retn}\,v_2}}
                  {y_2{:}\susp{{\sf cont}\,({\sf app2}\,\lambda x.e\,x)}}}
               \leadsto
               \tackon{\Theta}
                 {z{:}\susp{{\sf eval}\,(e\,v_2)}} $

    \medskip

    $T =
     T'; 
     \tstep{y', y_2}{\sf gen/cont}{(\tforalll{{\sf app2}\,\lambda x.e\,x}{\tappl{x'}{\tnil}})};
     \tstep{y_1}{\sf gen/retn}{(\tforalll{v_2}{(\tappl{y'}{\tnil})})}$,\\
    so we construct\\
    $T'; 
     \tstep{z}{\sf gen/eval}{(\tforalll{e\,v_2}{\tappl{x'}{\tnil}})}$

    \medskip

  \end{itemize}

\noindent
This completes the proof. 
\end{proof}

It is straightfoward to see that the primary pattern match in this
proof covers all the cases. We postpone, for now, the less obvious
discussion of how we ensure that the secondary case analyses on
generative traces covered all the cases.\robnote{Fill in a forward
  reference above with a concrete reference when one exists}

\subsection{Adequacy of the transition system}
\label{sec:nat-ssos-adequacy-absmachine}

The most interesting part of the adequacy proof was showing that
formation by generative signature $\Sigma\sf gen$ was an invariant of
$\Sigma\ref{fig:cbv-ev-ssos-fun}$. With that property established, the
final step is as straightforward as 

\bigskip
\begin{theorem}[Adequacy of the transition system]
$s \mapsto s'$ if and only if there exist $\Delta$ and $\Delta'$
such that
$\Delta \leadsto_{\Sigma\ref{fig:cbv-ev-ssos-fun}} \Delta'$,
$\interp{s} :: x{:}\susp{\sf gen} \leadsto^*_{\Sigma\sf gen} \Delta$, and
$\interp{s'} :: x{:}\susp{\sf gen} \leadsto^*_{\Sigma\sf gen} \Delta'$. 
\end{theorem}

\begin{proof} The proof is by straightforward case analysis and
  construction; we will give the case associated with ${\sf ev/app}$
  in both directions.

  The forward direction prooceeds by case analysis over the definition
  of the transition system from the beginning of the chapter.  For
  instance, if $k \rhd e_1\,e_2 \mapsto (k; \Box\,e_2) \rhd e_1$ by
  rule ${\sf absmachine/app}$ then we can form (by
  Theorem~\ref{thm:adequacy-states}) the following traces:
  \begin{align*}
  \interp{k \rhd e_1\,e_2} 
  & :: x{:}\susp{\sf gen} \leadsto^*_{\Sigma\sf gen}
       \tackon{\Theta}
        {y{:}\susp{{\sf eval}\,({\sf app}\,\interp{e_1}\,\interp{e_2})}}
\\
  \interp{(k; \Box\,e_2) \rhd e_1} 
  & :: x{:}\susp{\sf gen} \leadsto^*_{\Sigma\sf gen}
       \tackon{\Theta}
        {\mkconj{z_1{:}\susp{{\sf eval}\,\interp{e_1}}}
         {z_2{:}\susp{{\sf cont}\,({\sf app1}\,\interp{e_2})}}}
  \end{align*}
  It is then possible to construct the required step:
  \begin{align*} 
  &\tstep{z_1, z_2}{\sf ev/app}{(\tforalll{\interp{e_1}}{\tforalll{\interp{e_2}}{(\tappl{y}{\tnil})}})}
  \\
  &\qquad\qquad :: \tackon{\Theta}
        {y{:}\susp{{\sf eval}\,({\sf app}\,\interp{e_1}\,\interp{e_2})}}
     \leadsto_{\Sigma\ref{fig:cbv-ev-ssos-fun}}
     \tackon{\Theta}
        {\mkconj{z_1{:}\susp{{\sf eval}\,\interp{e_1}}}
         {z_2{:}\susp{{\sf cont}\,({\sf app1}\,\interp{e_2})}}}
  \end{align*}

  In the backward direciton, we are given a step in the dynamic 
  semantics, such as the one above, as well as the two traces 
  \begin{align*}
  T_1
  & :: x{:}\susp{\sf gen} \leadsto^*_{\Sigma\sf gen}
       \tackon{\Theta}
        {y{:}\susp{{\sf eval}\,({\sf app}\,\interp{e_1}\,\interp{e_2})}}
\\
  T_2
  & :: x{:}\susp{\sf gen} \leadsto^*_{\Sigma\sf gen}
       \tackon{\Theta}
        {\mkconj{z_1{:}\susp{{\sf eval}\,\interp{e_1}}}
         {z_2{:}\susp{{\sf cont}\,({\sf app1}\,\interp{e_2})}}}
  \end{align*}
  By the same case analysis on the structure of the trace that we performed
  in the preservation theorem (Theorem~\ref{thm:adequate-pres}) and
  , we 
  need to establish that \medskip \\
  $T_1 = T'; \tstep{y}{\sf gen/eval}{(\tforalll{\interp{e_1}}{\tforalll{\interp{e_2}}{(\tappl{x'}{\tnil})}})}$ and \\
  $T_2 = T'; \tstep{z', z_2}{\sf gen/cont}{(\tforalll{{\sf app1}\,\interp{e_2}}{(\tappl{x'}{\tnil})})}; \tstep{z_1}{\sf gen/eval}{(\tforalll{\interp{e_1}}{(\tappl{z'}{\tnil})})}$ \medskip\\
  % 
  where $T' :: x{:}\susp{\sf gen} \leadsto^*_{\Sigma\sf gen}
  \tackon{\Theta}{x'{:}{\sf gen}}$ in both cases. Therefore,
  Theorem~\ref{thm:adequacy-states} there is a stack $k$ such that
  $\interp{k} = T'$, $\interp{k \rhd e_1\,e_2} = T_1$, and
  $\interp{(k; \Box\,e_2) \rhd e_1} = T_2$.
  We conclude, then, by observing that 
  $k \rhd e_1\,e_2 \mapsto (k; \Box\,e_2) \rhd e_1$ by rule 
  ${\sf absmachine/app}$.
\end{proof}

\section{Exploring the richer fragment}
\label{sec:richer-ordered-abstract}



\subsection{Mutable storage}
\label{sec:mutable-storage}

\begin{figure}
\fvset{fontsize=\small,boxwidth=229pt}
\VerbatimInput{sls/ssos-mutable.sls}
\caption{SSOS semantics of mutable storage.}
\label{fig:ssos-mutable}
\end{figure}


\subsubsection{Existential angst} 

No check for pointer inequality! This is a fundamental restriction of
the fact that we're using existential quantification rather than some
form of nominal quantification. (Hack due to Favonia and Bob, personal
communication, but dates back earlier - was it one of Karl's papers?
Cheney cites it in nominal abstraction.)

\subsection{Call-by-need evaluation}

\subsection{Recoverable failure}

\subsubsection{Failures and the parallel translation}

\subsection{Environment semantics}

\subsection{Looking back at natural semantics}
\label{sec:enriching-natsem}

\section{Partial transformation}
\label{sec:othertransform}

\subsection{Evaluation contexts}
\label{sec:evaluationcontexts}

Thus far, we have considered big-step operational semantics and abstract
machines, neglecting the third great tradition of programming language
specification, {\it structural operational semantics}. Structural
operational semantics (SOS) define single-step evaluation inductively over
the structure of expressions; the SOS semantics for our running example
language is the following:
\[
\infer
{\lambda x.e\,{\sf value} \mathstrut}
{}
\quad
\infer
{e_1\,e_2 \mapsto e_1'\,e_2 \mathstrut}
{e_1 \mapsto e_1' \mathstrut}
\quad
\infer
{e_1\,e_2 \mapsto e_1\,e_2' \mathstrut}
{e_1\,{\sf value}
 &
 e_2 \mapsto e_2' \mathstrut}
\quad
\infer
{(\lambda x. e)v \mapsto [v/x]e \mathstrut}
{v\,{\sf value} \mathstrut}
\]
This inductive specification is adequately encoded on the left-hand
side of Figure~\ref{fig:cbv-sos}, along with the proposition \Verb|ev|
that describes a big-step operational semantics in terms of repeated
application of the small-step operational semantics.

\begin{figure}[tp]
\fvset{fontsize=\small,boxwidth=229pt}
\BVerbatimInput{sls/cbv-sos.sls}
\BVerbatimInput{sls/cbv-sos-eval.sls}
\caption{Small-step evaluation, and one corresponding abstract machine.}
\label{fig:cbv-sos}
\end{figure}

\fvset{fontsize=\small}

There are a couple of possibilities for how the 
One obvious way to proceed is to simply translate the big-step portion
of our semantics as encoded 


If we just translate the ${\sf ev}$ portion of the semantics (using
the tail-recursion optimizing translation), then we will get what is
probably fair to call the most boring possible substructural
operational semantics: 

\smallskip
\VerbatimInput{sls/cbv-sos-proc.sls}
\smallskip

\noindent
Under this semantics, the substructural context contains a single
resource, \Verb|eval-steps(E)|, which takes steps according to the
rules of the small-step structural operational semantics until a value
is reached, at which point the context contains \Verb|retn-steps(V)|.


\begin{figure}[t]
\VerbatimInput{sls/cbv-sos-defun.sls}
\caption{The defunctionalized abstract machine from Figure~\ref{fig:cbv-sos}.}
\label{fig:cbv-sos-defun}
\end{figure}

The interesting observations are to be had from the other direction: what if

\subsection{Temporal logic}

The natural semantics of \rowan~are not, on a superficial level,
significantly more complex than other natural semantics. However, it
turns out that the usual set of techniques for adding state to a
natural semantics break down, and discussing a \rowan-like logic with
state remained a challenge for many years.\robnote{Figure out from
  Rowan what the recent work he told you about was.} Through the
logical correspondence, it is easy to see why: the natural SSOS
specification of \rowan~integrates both concurrent and deductive
reasoning in an arbitrarily nested way. In fact, Figure XXX is the
only SLS specification in this thesis that exhibits this form of
recursive dependency between concurrent and deductive reasoning.  In
particular, the \rowan~specification is way out of the image of the
extended natural semantics we considered in
Section~\ref{sec:enriching-natsem}. The natural encoding in state lies
in the ambient substructural context of a concurrent computation, but
that ambient computation cannot properly enter into a deductive
sub-computation. If we tried to add state to \rowan~the same way we
added it in Section~\ref{sec:mutable-storage}, the entire store
would effectively leave scope whenever computation considered
the subterm $e$ of ${\sf next}(e)$. That consideration happens
as deductive reasoning, not as concurrent reasoning!

 it is the only we
will consider in this thesis that has with property.

It's hard to include state in temporal logic! But the logical correspondence
helps us understand why: the natural SSOS specification of 



% Destination-passing style
\chapter{Destination-passing}
\label{chapter-destinations}

The natural notion of ordering provided by ordered linear logic
is quite convenient for encoding transition systems that 
have a stack or tree-based control structure. The ordered
abstract machine SSOS specifications from Chapter~6 demonstrate
this; another example is the push-down automaton for parenthesis
matching discussed in the introduction, which we can now present,
in Figure~\ref{fig:pda-ord}, as an \sls~specification.

\begin{figure}[ht]
\fvset{fontsize=\small,boxwidth=229pt}
\VerbatimInput{sls/pda-ord.sls}
\caption{Ordered \sls~specification of a PDA for parenthesis matching.}
\label{fig:pda-ord}
\end{figure}

The natural expression of order provided by \sls~makes ordered
abstract machine specifications and the PDA specification in
Figure~\ref{fig:pda-ord} much more concise. However, in a way that we
will make precise in this chapter, ordered logic does not actually add
any more {\it expressiveness} to concurrent specifications relative to
linear logic. In Chapter~5, we argued that ordered abstract machines
were at least as expressive as (moded) natural semantics by giving a
transformation, operationalization, from the latter to the
former. Analogously, in this chapter we will argue that concurrent
specifications in linear logic are just as expressive as concurrent
specifications in ordered logic by giving a transformation, {\it
  destination-adding}, from the latter to the former.  As 
originally presented by Pfenning and I in \cite{simmons11logical}, 
the destination-adding transformation turns all ordered
atomic propositions into linear atomic propositions, but tagged them
with two new arguments (the destinations of the destination-adding
transformation) that serve as a link between the formerly-ordered
atomic proposition and the formerly-ordered atomic propositions that
were previously to their left and to their right. 

Destinations (terms of type ${\sf dest}$) have no constructors, they
are only introduced as variables by existential quantification.  When
we perform the destination-adding transformation on the PDA in
Figure~\ref{fig:pda-ord}, we get the PDA in Figure~\ref{fig:pda-lin}.


\begin{figure}
\fvset{fontsize=\small,boxwidth=229pt}
\VerbatimInput{sls/pda-lin.sls}
\caption{Linear \sls~specification of a PDA for parenthesis matching.}
\label{fig:pda-lin}
\end{figure}

As an aside, the specification in Figure~\ref{fig:pda-lin}, like every
other specification that results from destination-adding, has no
occurrences of ${\downarrow}A^-$ and no ordered atomic propositions. By
the discussion in Section~\ref{sec:perm-fragments}, we would therefore
be justified in viewing this specification as a linear logical
specification (or a CLF specification) instead of a ordered logical
specification in \sls.  This would not impact the structure of the
derivations significantly; essentially, it just means that we would
write $A^+_1 \lolli \{ A^+_2 \}$ instead of $A^+_1 \lefti \{ A^+_2
\}$.  This reinterpretation was used in \cite{simmons11logical}, but
we will stick with the notation of ordered logic for consistency,
while recognizing that there is nothing ordered
about specifications like the one in Figure~\ref{fig:pda-lin}. 

When the destination-adding translation is applied to ordered abstract
machine SSOS specifications, the result is a style of SSOS
specification called {\it destination-passing}. Destination-passing
specifications were the original style of SSOS specification proposed
in the CLF tech reports~\cite{cervesato02concurrent}. Whereas the
operationalization transformation exposed the structure of natural
semantics proofs so that they could be modularly extended with
stateful features, the destination-adding translation exposes the
control structure of specifications, allowing the language to be
modularly extended with control effects and effects like
synchronization in concurrent specifications.

\section{Logical transformation: destination-adding}
\label{sec:destination-adding}

The destination-adding translation presented here is essentially the
same as the one presented and proved correct in
\cite{simmons11logical}. The translation in that paper operated over
rules of the form $\forall \overline{x}. S_1 \righti \{ S_2 \}$,
whereas ours will operate over rules of the form $\forall
\overline{x}. S_1 \lefti \{ S_2 \}$, but the difference between
$\righti$ and $\lefti$ is irrelevant for first-order
\sls~specifications.\footnote{The monad $\{ S_2 \}$ did not actually
  appear in \cite{simmons11logical}, and the presentation took
  polarity into account but was not explicitly polarized. We are
  justified in reading the lax modality back in by the sort of erasure
  argument discussed in Section~\ref{sec:perm-fragments}.} The syntactic
category $S$ is a refinement of the positive types $A^+$ defined by
the following grammar:
\[
S ::= p^+_\mpers \mid p^+_\meph \mid p^+ \mid \one
\mid t \doteq s \mid S_1 \fuse S_2 \mid \exists x{:}\tau. S
\]
The translation of a rule $\forall \overline{x}. S_1 \lefti \{ S_2 \}$
is then $\forall \overline{x}.\, \forall d_L{:}{\sf dest}.\, \forall
d_R{:}{\sf dest}.\, \llbracket S_1 \rrbracket^{d_L}_{d_R} \lefti \{
\llbracket S_2 \rrbracket^{d_L}_{d_R} \}$, where $\llbracket S
\rrbracket^{d_L}_{d_R}$ is defined in Figure~\ref{fig:destadd-pos}. It
is also necessary to transform all ordered predicates with kind
$\Pi. x_1{:}\tau_1\ldots \Pi.x_n{:}\tau_n.\,{\sf prop\,ord}$ that are
declared in the signature into predicates with kind
$\Pi. x_1{:}\tau_1\ldots \Pi.x_n{:}\tau_n.\, \Pi.d_L{:}{\sf dest}.\,
\Pi.d_R{:}{\sf dest}.\, {\sf prop\,ord}$ in order for the translation
of an ordered atomic proposition $p^+$ to remain a valid type in
the transformed signature.

\begin{figure}
\begin{align*}
\llbracket p^+ \rrbracket^{d_L}_{d_R} & := 
 {\sf a}\,t_1\ldots t_n\,d_L\,d_R ~~~ \mbox{(where $p^+ = {\sf a}\,t_1\ldots t_n$)}
\\
\llbracket p^+_\meph \rrbracket^{d_L}_{d_R} & := p^+_\meph \fuse d_L \doteq d_R
\\
\llbracket p^+_\mpers \rrbracket^{d_L}_{d_R} & := p^+_\mpers \fuse d_L \doteq d_R
\\
\llbracket \one \rrbracket^{d_L}_{d_R} & := d_L \doteq d_R
\\
\llbracket t \doteq s \rrbracket^{d_L}_{d_R} & := t \doteq s \fuse d_L \doteq d_R
\\
\llbracket S_1 \fuse S_2 \rrbracket^{d_L}_{d_R} & := 
 \exists d_M{:}{\sf dest}.\, 
   \llbracket S_1 \rrbracket^{d_L}_{d_M}
   \fuse
   \llbracket S_2 \rrbracket^{d_M}_{d_R}
\\
\llbracket \exists x{:}\tau.\,S \rrbracket^{d_L}_{d_R} & := 
 \exists x{:}\tau.\, \llbracket S \rrbracket^{d_L}_{d_R}
\end{align*}
\caption{Destination-adding transformation on positive propositions.}
\label{fig:destadd-pos}
\end{figure}

According to Figure~\ref{fig:destadd-pos}, the rule 
${\sf pop}$ in Figure~\ref{fig:pda-lin} should actually be written as
follows:
\begin{align*} 
  {\sf pop} & : 
  \forall x{:}{\sf tok}.\,
  \forall l{:}{\sf dest}.\,
  \forall r{:}{\sf dest}.\,
  \\
  & \qquad (\exists m_1{:}{\sf dest}.\, {\sf stack}\,x\,l\,m \fuse
   (\exists m_2{:}{\sf dest}.\, {\sf hd}\,m_1\,m_2 \fuse
     {\sf right}\,x\,m_2\,r))
  \\ 
  & \qquad\quad \lefti
  \{ 
    {\sf hd}\,l\,r
  \}
\end{align*}
The destination-adding transformation as
implemented\robnote{Implement, or don't say this; also discuss the
  implementation of the probably-correct transformation on
  higher-order specs if you get to that.} produces rules that are
equivalent to the specification in Figure~\ref{fig:destadd-pos}
but that avoid unnecessary equalities and push existential quantifiers
as far out as possible to get specifications that look more like
Figure~\ref{fig:pda-lin}. We write the result of the destination-adding
transformation on the signature $\Sigma$ as ${\it Dest}(\Sigma)$. 

We could consider another simplification: is it necessary to generate
a new destination $m$ by existential quantification in the head
$\exists m.\,{\sf stack}\,x\,l\,m \fuse {\sf hd}\,m\,r$ of ${\sf
  push}$ in Figure~\ref{fig:pda-lin}? There is already a destination
$m$ mentioned in the head that will be unused in the conclusion.  And
for the translation that takes all formerly-ordered atomic
propositions to linear atomic propositions, it would, in fact, be
possible to avoid generating new destinations in the transformation of
rules $\forall \overline{x}.\,S_1 \lefti \{ S_2 \}$ where the head
$S_2$ contains no more ordered atomic propositions than the premise
$S_1$. 

We preserve this quantifier in part because, as presented above, our
translation closely follows the contours of work by Morrill, Moot, and
Piazza on translating ordered logic into linear logic
\cite{morrill95higher,moot01linguistic}. That work is, in turn, based
on van Benthem's relational models of ordered logic, which closely
associate multiplicative conjunction $A \fuse B$ with existential
quantification \cite{vanbenthem91relational}. In some ways, the
aforementioned translations are more general than our
destination-adding transformation, as they handle a uniform logic
instead of the concurrent fragment presented here and in here
\cite{simmons11logical}. On the other hand, those translations only
operate on a propositional fragment without the unit of multiplicative
conjunction $\one$; as discussed in \cite[p.~57]{simmons11logical},
the addition non-ordered atomic propositions, $\one$, and $t \doteq s$
complicates matters significantly. 

In addition to following van Bentham's relational models, the
transformation as we have given it simplifies the correctness
proof (Theorem~\ref{thm:destcorrect}). The additional existential
quantifiers give us more structure to work with when considering
program abstraction in Chapter 8, and the result of applying the
transformation to ordered abstract machines is more in line with 
existing destination-passing SSOS specifications.

% Another reason for preserving the existential quantifier in the head
% of the ${\sf push}$ rule is that it allows us to make an extension to
% the destination-adding transformation beyond what was considered in
% \cite{simmons11logical}. As long as the head of every translated rule
% contains at least one formerly-ordered atomic proposition that has
% been turned into a linear atomic propsition, it is possible to 
% without breaking Theorem~\ref{thm:destcorrect}. 

%  The correctness of the transformation 
% depends critically on the fact that every portion of the context
% that might be used to successfully right focus on a translated positive 
% proposition $\llbracket S \rrbracket^{d_L}_{d_R}$ is a 

To prove the correctness of destination-adding, we must describe a 
translation $\llbracket \Psi; \Delta \rrbracket$ from process states
with ordered, linear, and persistent atomic propositions to ones
with only linear and persistent atomic propositions:
\begin{align*}
\llbracket \Psi; \cdot \rrbracket & = (\Psi, d_L{:}{\sf dest}; \cdot) 
\\
\llbracket \Psi; \Delta, x{:}\susp{{\sf a}\,t_1\ldots t_n} \rrbracket 
& = (\Psi', d_L{:}{\sf dest}, d_R{:}{\sf dest}; 
     \Delta', x{:}\susp{{\sf a}\,t_1\ldots t_n\,d_L\,d_R})\\
& \qquad
  \mbox{(where $\sf a$ is ordered and
  $\llbracket \Psi; \Delta \rrbracket = (\Psi', d_L{:}{\sf dest}; \Delta') $)}
\\
\llbracket \Psi; \Delta, x{:}\susp{p^+_\meph} \rrbracket 
& = (\Psi'; \Delta', x{:}\susp{p^+\meph})
 \qquad \mbox{(where
       $\llbracket \Psi; \Delta \rrbracket = (\Psi'; \Delta') $)}
\\
\llbracket \Psi; \Delta, x{:}\susp{p^+_\mpers} \rrbracket 
& = (\Psi'; \Delta', x{:}\susp{p^+\mpers})
  \qquad \mbox{(where 
       $\llbracket \Psi; \Delta \rrbracket = (\Psi'; \Delta') $)}
\end{align*}

\begin{theorem}[Correctness of destination-adding]~\\\label{thm:destcorrect}
$\llbracket \Psi; \Delta \rrbracket \leadsto_{{\it Dest}(\Sigma)}
 (\Psi_l; \Delta_l)$ if and only if 
$(\Psi; \Delta) \leadsto_\Sigma (\Psi_o; \Delta_o)$ and
$(\Psi_l; \Delta_l) = \llbracket \Psi_o, \Psi''; \Delta_o \rrbracket$ 
for some variable 
context $\Psi''$ containing destinations free in the first translated
context but not the second.
\end{theorem}

\begin{proof}
As in \cite[Theorem 2]{simmons11logical}.\robnote{Double check how the 
variable slack works.}
\end{proof}


If we leave off explicitly mentioning the variable context $\Psi$, 
then the trace that represents successfully processing 
the string {\sf [ ( ) ] }
with the transformed push-down automaton 
specification in Figure~\ref{fig:pda-lin} 
is as follows (we again underline ${\sf hd}$
for emphasis):
\begin{align*}
           & y_0{:}\susp{\underline{\sf hd}\,d_0\,d_1},
             x_1{:}\susp{{\sf left}\,{\sf sq}\,d_1\,d_2},
             x_2{:}\susp{{\sf left}\,{\sf pa}\,d_2\,d_3},
             x_3{:}\susp{{\sf right}\,{\sf pa}\,d_3\,d_4},
             x_4{:}\susp{{\sf right}\,{\sf sq}\,d_4\,d_5}
\\
\leadsto ~ & z_1{:}\susp{{\sf stack}\,{\sf sq}\,d_0\,d_6},
             y_1{:}\susp{\underline{\sf hd}\,d_6\,d_2}
             x_2{:}\susp{{\sf left}\,{\sf pa}\,d_2\,d_3},
             x_3{:}\susp{{\sf right}\,{\sf pa}\,d_3\,d_4},
             x_4{:}\susp{{\sf right}\,{\sf sq}\,d_4\,d_5}
\\
\leadsto ~ & z_1{:}\susp{{\sf stack}\,{\sf sq}\,d_0\,d_6},
             z_2{:}\susp{{\sf stack}\,{\sf pa}\,d_6\,d_7},
             y_2{:}\susp{\underline{\sf hd}\,d_7\,d_3}
             x_3{:}\susp{{\sf right}\,{\sf pa}\,d_3\,d_4},
             x_4{:}\susp{{\sf right}\,{\sf sq}\,d_4\,d_5}
\\
\leadsto ~ & z_1{:}\susp{{\sf stack}\,{\sf sq}\,d_0\,d_6},
             y_3{:}\susp{\underline{\sf hd}\,d_6\,d_4}
             x_4{:}\susp{{\sf right}\,{\sf sq}\,d_4\,d_5}
\\
\leadsto ~ & y_4{:}\susp{\underline{\sf hd}\,d_0\,d_5}
\end{align*}


\begin{figure}[t]
\fvset{fontsize=\small,boxwidth=229pt}
\VerbatimInput{sls/dest-vestige.sls}
\caption{Translation of Figure~\ref{fig:cbv-ev-ssos-fun} with vestigial destinations.}
\label{fig:dest-vestige}
\end{figure}

\begin{figure}
\fvset{fontsize=\small,boxwidth=229pt}
\VerbatimInput{sls/dest-cbv.sls}
\caption{Translation of Figure~\ref{fig:cbv-ev-ssos-fun} without vestigial destinations.}
\label{fig:dest-cbv}
\end{figure}



\subsection{Vestigial destinations}

When we apply the translation of expressions to the call-by-value
lambda calculus specification from Figure~\ref{fig:cbv-ev-ssos-fun},
we get the specification in Figure~\ref{fig:dest-vestige}, which is
has one problem: the second argument to ${\sf eval}$ and ${\sf retn}$
is always $d'$, and the destination never changes; it is essentially a
vestige of the destination-adding transformation. As long as we are
transforming a sequential ordered abstract machine, we can eliminate
this vestigial destination, giving us the specification in
Figure~\ref{fig:dest-cbv}. This extra destination is {\it not}
vestigial when we translate a parallel specification, but as we
discuss in Section~\ref{sec:modular-parallelism}, we don't necessarily
want to run destination-adding on parallel ordered abstract machines
anyway.

\subsection{Persistent destination passing}

When we translate our PDA specification, it is actually not necessary
to translate ${\sf hd}$, ${\sf left}$, ${\sf right}$ and ${\sf stack}$ into
linear atomic propositions; if we translate ${\sf hd}$ as
a linear predicate but translate the other predicates as persistent
predicates, it will still be the case that there is always exactly one
linear atomic proposition ${\sf hd}\,d_L\,d_R$ in the context, at most one
${\sf stack}\,x\,d\,d_L$ proposition with the same destination $d_L$, 
and at most one ${\sf right}\,x\,d_R\,d$ or ${\sf left}\,x\,d_R\,d$ 
with the same destination $d_R$. This means it is still the case that the
PDA accepts the string if and only if there is the following series of 
transitions:
\begin{align*}
 %(d_0{:}{\sf dest}, \ldots, d_{n+1}{:}{\sf dest}; 
(    x{:}\susp{{\sf hd}\,d_0\,d_1}, 
    y_1{:}\susp{{\sf left}\,x_1\,d_1\,d_2},
    \ldots&,
    y_n{:}\susp{{\sf right}\,x_n\,d_n\,d_{n+1}})% )
%\\
% &
~~ \leadsto^* ~~
%   (\Psi; 
(\Gamma, z{:}\susp{{\sf hd}\,d_0\,d_{n+1}})%)
\end{align*}
Unlike the entirely-linear PDA specification, the final state may include
some additional 
persistent propositions, represented by $\Gamma$. Specifically, the final state
contains all the original ${\sf left}\,x\,d_i\,d_{i+1}$ and
${\sf right}\,x\,d_i\,d_{i+1}$ propositions 
along with all the ${\sf stack}\,x\,d\,d'$ propositions that were created
during the course of evaluation.

% \begin{figure}[t]
% \fvset{fontsize=\small,boxwidth=229pt}
% \VerbatimInput{sls/pda-pers.sls}
% \caption{Linear/persistent \sls~specification of a PDA for parenthesis
%   matching.}
% \label{fig:pda-pers}
% \end{figure}

I originally conjectured that a version of
Theorem~\ref{thm:destcorrect} would hold in any specification that
turned some ordered atomic propositions linear and others
persistent just as long as at least one atomic proposition in
the premise of every rule remained linear after transformation.  
This would have given a
generic justification for turning ${\sf left}$, ${\sf right}$ and ${\sf
  stack}$ persistent in Figure~\ref{fig:pda-lin} and to turning ${\sf
  cont}$ persistent in Figure~\ref{fig:dest-cbv}. However, that
condition is not strong enough.  To see why, consider a signature with
one rule, ${\sf a} \fuse {\sf b} \fuse {\sf a} \lefti \{ {\sf b} \}$,
where ${\sf a}$ and ${\sf b}$ are ordered atomic propositions.  We can
construct the following trace:
\begin{align*}
& (x_1{:}\susp{\sf a}, x_2{:}\susp{\sf b}, x_3{:}\susp{\sf a}, 
  x_4{:}\susp{\sf b}, x_5{:}\susp{\sf a})
\leadsto 
(x{:}\susp{\sf b},
  x_4{:}\susp{\sf b}, x_5{:}\susp{\sf a})
\not\leadsto  
\intertext{From the same starting point, exactly one
other trace is possible:}
& (x_1{:}\susp{\sf a}, x_2{:}\susp{\sf b}, x_3{:}\susp{\sf a}, 
  x_4{:}\susp{\sf b}, x_5{:}\susp{\sf a})
\leadsto 
(x_1{:}\susp{\sf a}, x_2{:}\susp{\sf b}, x{:}\susp{\sf b})
\not\leadsto 
\end{align*}
However, if we perform the destination-passing transformation, letting
${\sf a}\,d\,d'$ be a persistent atomic proposition and letting ${\sf
  b}\,d\,d'$ be a linear atomic proposition, then we have a series of
transitions in the transformed specification that can reuse the atomic
proposition ${\sf a}\,d_2\,d_3$ in a way that doesn't correspond to
any series of transitions in ordered logic:
\begin{align*}
&  x_1{:}\susp{{\sf a}\,d_0\,d_1}, 
   x_2{:}\susp{{\sf b}\,d_1\,d_2}, 
   x_3{:}\susp{{\sf a}\,d_2\,d_3}, 
   x_4{:}\susp{{\sf b}\,d_3\,d_4}, 
   x_5{:}\susp{{\sf a}\,d_4\,d_5}
\\ \leadsto~
&  x_1{:}\susp{{\sf a}\,d_0\,d_1}, 
   \underline{x{:}\susp{{\sf b}\,d_0\,d_3}}, 
   x_3{:}\susp{{\sf a}\,d_2\,d_3}, 
   x_4{:}\susp{{\sf b}\,d_3\,d_4}, 
   x_5{:}\susp{{\sf a}\,d_4\,d_5}
\\ \leadsto~
&  x_1{:}\susp{{\sf a}\,d_0\,d_1}, 
   x{:}\susp{{\sf b}\,d_0\,d_3}, 
   x_3{:}\susp{{\sf a}\,d_2\,d_3}, 
   \underline{x'{:}\susp{{\sf b}\,d_2\,d_5}}, 
   x_5{:}\susp{{\sf a}\,d_4\,d_5}
\end{align*}
In the
first process state, there is a path $d_0, d_1, d_2, d_3, d_4, d_5$ through
the context that reconstructs the ordering in the original ordered context.
In the second process state, there is still a path $d_0, d_3, d_4, d_5$ that
allows us to reconstruct the ordered context
$(x{:}\susp{\sf b},
  x_4{:}\susp{\sf b}, x_5{:}\susp{\sf a})$ by ignoring the persistent
propositions associated with $x_1$ and $x_3$. 
However, in the third process state above, no path exists, so the final
state cannot be reconstructed as any ordered context. 

It would be good to identify a condition that allowed us to
selectively turn some ordered propositions persistent when
destination-adding without violating (a version of)
Theorem~\ref{thm:destcorrect}. In the absence of such a generic
condition, it is still straightforward to see that performing
destination-passing and then turning some propositions persistent is
an {\it abstraction}: if the original system can make a series of
transitions, the transformed system can simulate those transitions,
but the reverse may not be true. Furthermore, for systems we are
interested in (like push-down automata with persistent ${\sf stack}$,
${\sf left}$, and ${\sf right}$ or the sequential ordered abstract
machines with persistent ${\sf cont}$), it happens to be the case that
the abstraction is precise: the destination-passing specificiation can
only make transitions which were possible in the ordered
specification.

\section{Exploring the richer fragment}

In \cite{simmons11logical}, the destination-adding transformation was
used to to expose information about the control structure of
computations; this control structure can be harnessed by the program
abstraction methodology described in the next chapter to derive
program analyses. The exposure of control structure also allows us to
modularly extend specifications with additional features relating to
control. In this section, we explore programming language features
that are not easily expressible with ordered abstract machine SSOS
specifications but that can be easily expressed with
destination-passing SSOS specifications.

The semantics of parallelism and failure presented in
Section~\ref{sec:modular-parallelism} are new. The semantics of
futures (Section~\ref{sec:dest-futures}) and syncronization
(Section~\ref{sec:dest-synch}) are a based on the specifications first
presented in the CLF tech report \cite{cervesato02concurrent}; the
semantics of first-class continuations
(Section~\ref{sec:dest-continuations}) were presented previously in
\cite{pfenning04substructural,pfenning09substructural}.

\begin{figure}
\fvset{fontsize=\small,boxwidth=229pt}
\VerbatimInput{sls/dest-pair.sls}
\caption{Destination-passing semantics for parallel evaluation of pairs.}
\label{fig:dest-pair}
\end{figure}

\subsection{Alternate semantics for parallelism and exceptions}
\label{sec:modular-parallelism}

In Section~\ref{sec:failure}, we discussed how parallel evaluation and
recoverable failure can be combined in an ordered abstract machine
SSOS specification. Due to the fact that the two parts of a parallel
ordered abstract machine are separated by an arbitrary amount of
ordered context, some potentially desirable semantics ways of
integrating parallelism and failure were difficult or impossible to
express, however. 

Once we transition to destination-passing SSOS specifications, it is
possible to give a more direct semantics to parallel evalution that
better facilitates talking about failure. Instead of having the stack
frame associated with parallel pairs be ${\sf cont}\,{\sf pair1}$ (as
in Figure~\ref{fig:ssos-minml-core}) or ${\sf cont2}\,{\sf pair1}$ (as
discussed in Section~\ref{sec:failure}), we create a continuation
${\sf cont2}\,{\sf pair1}\,d_1\,d_2\,d$ with {\it three} destinations;
$d_1$ and $d_2$ represent the return destinations points of the two
subcomputations, whereas $d$ represents the destination to which the
evaluated pair is to be returned. This strategy applied to the
parallel evaluation of paris is shown in Figure~\ref{fig:dest-pair}.

In ordered specifications, an ordered atomic proposition can be
directly connected to at most two other ordered propositions: the
proposition immediately to the left in the ordered context, and the
proposition immediately to the right in the ordered context. What
Figure~\ref{fig:dest-pair} demonstrates is that, with destinations, a
linear proposition can be locally connected to {\it any finite number}
of other propositions. Whereas in ordered abstract machine
specifications the parallel structure of a computation had to be
reconstructed by parsing the context in postfix, a linear
destination-passing specification uses destinations to thread together
the treelike dependencies in the context. It would presumably be
possible to consider a different version of the parallel
operationalization that targeted this form of destination-passing
specification specifically, but we will not consider such a
transformation in this thesis. 

\begin{figure}
\fvset{fontsize=\small,boxwidth=229pt}
\VerbatimInput{sls/dest-fail-paror.sls}
\caption{Integration of parallelism and exceptions; signals failure as
  soon as possible.}
\label{fig:dest-fail-paror}
\end{figure}

By the destination-based parallel continuations, we give, in
Figure~\ref{fig:dest-fail-paror}, a semantics for recoverable failure
that eagerly returns errors from either branch of a parallel
computation. The rules ${\sf ev/errorL}$ and ${\sf ev/errorR}$ pass on
errors returned to a place where the computation forks.  Those two
rules also leave behind a linear proposition ${\sf terminate}\,d$ that
will abort the other branch of computation if it returns succesfully
(rule ${\sf term/retn}$) or with an error (rule ${\sf term/err}$). It
would also be possible to add additional rules like ${\sf cont}\,d'\,d
\fuse {\sf terminate}\,d \lefti \{ {\sf terminate}\,d' \}$ that
actively aborted the useless branch instaead of passively waiting for
it to finish.

\subsection{Synchronization}
\label{sec:dest-synch}

The CLF tech report gives a presentation of the full set of Concurrent
ML primitives that is equally applicable to destination-passing
\sls~specifications \cite{cervesato02concurrent}.

\begin{figure}
\fvset{fontsize=\small,boxwidth=229pt}
\VerbatimInput{sls/dest-synch.sls}
\caption{Simple synchronization.}
\label{fig:dest-synch}
\end{figure}



\subsection{Futures}
\label{sec:dest-futures}

Futures is are a ; they do not interact well with parallelism as it is encoded
in the ordered 


\subsection{First-class continuations}
\label{sec:dest-continuations}

The interaction of parallel evaluation and first-class continuations
has been evaluated by Moreau and Ribbens in the context of Scheme
\cite{moreau96semantics}; giving an SSOS encoding of their abstract
machine would be interesting, but we do not do so here.



% Linear logical approximation
\chapter{Linear logical approximation}
\label{chapter-approx}

The abstract interpretation of programs relates the exact semantics of
a programming language to a finite approximation of those semantics. A
general recipe for constructing a sound program analysis is to (1)
specify the operational semantics of the underlying programming
language via an interpreter, and (2) specify a terminating
approximation of the interpreter itself. This is the basic idea behind
{\it abstract interpretation} \cite{cousot77abstract}, which provides
techniques for constructing approximations (for example, by exhibiting
a Galois connection between concrete and abstract domains). The
correctness proof must establish the appropriate relationship between
the concrete and abstract computations and show that the abstract
computation terminations. We need to vary both the specification of
the operational semantics and the form of the approximation in order
to obtain various kinds of program analyses, sometimes with
considerable ingenuity.

In this chapter, which is mostly derived from \cite{simmons11logical},
we consider a new class of instances in the general schema of abstract
interpretation that is based on the approximation of SSOS
specifications in \sls. We apply logically justified techniques for
manipulating and approximating SSOS specifications to yield
approximations that are correct by construction. The resulting
persistent logical specifications can be interpreted as saturating
logic programs, generalizing proposals of McAllester and Ganzinger
\cite{mcallester02complexity,ganzinger02logical} to specifications
that use higher-order abstract syntax.

\section{Saturating logic programming}

Concurrent \sls~specifications where all positive atomic propositions
are persistent (and where all inclusions of negative propositions in
positive propositions -- if there are any -- have the form ${!}A$, not
${\downarrow}A^-$ or ${\gnab}A^-$) have a distinct logical and
operational character. Logically, by the discussion in
Section~\ref{sec:perm-fragments} we are justified in reading such
specifications as specifications in persistent intuitionistic logic or
persistent lax logic. Operationally, while persistent specifications
have an interpretation as transition systems, that interpretation is
not very useful. This is because if we can take a transition once --
for instance, using the rule ${\sf a} \lefti \{ {\sf b} \}$ to derive
the persistent atomic proposition ${\sf b}$ from the persistent atomic
proposition ${\sf a}$ -- none of the facts that enabled that
transition can be consumed, as all facts are persistent. Therefore, we
can continue to make the same transition indefinitely; in the
above-mentioned example, such transitions will derive multiple
redundant copies of ${\sf b}$.

The way we will understand the meaning of persistent and concurrent
\sls~specifications is in terms of {\it saturation}. A process state
$(\Psi; \Delta)$ is saturated relative to the signature $\Sigma$ if,
for any step $(\Psi; \Delta) \leadsto_\Sigma (\Psi'; \Delta')$, it is
the case that $\Psi$ and $\Psi'$ are the same, $x{:}\susp{p^+_\mpers}
\in \Delta'$ implies $x{:}\susp{p^+_\mpers} \in \Delta$, and
$x{:}\ispers{A^-} \in \Delta'$ implies $x{:}\ispers{A^-} \in
\Delta$. This means that a signatures with a rule that
  produces new variables by existential quantification, like ${\sf a}
  \lefti \{ \exists \lf{x}. {\sf b}\lf{(x)} \}$ has no saturated process states
  where ${\sf a}$ is present. Notions of saturation that can cope with
  existentially generated parameters are interesting, but are beyond
  the scope of this thesis.

A {\it minimal} saturated process state
is one with no duplicated propositions; we can compute a process state
from any saturated process state by removing duplicates. For purely
persistent specifications and process states, minimal saturated process
states are unique when they exist: if $(\Psi; \Delta)
\leadsto^*_\Sigma (\Psi_1; \Delta_1)$ and $(\Psi; \Delta)
\leadsto^*_\Sigma (\Psi_2; \Delta_2)$ and both $(\Psi_1; \Delta_1)$
and $(\Psi_2; \Delta_2)$ are saturated, then $(\Psi_1; \Delta_1)$
and $(\Psi_2; \Delta_2)$ have minimal process states that differ only
in the names of variables.

Furthermore, if a saturated process state exists for a given initial
process state, the minimal saturated process state can be computed by
the usual forward-chaining semantics where only transitions that
derive ${\it new}$ persistent atomic propositions or equalities $\lf{t}
\doteq \lf{s}$ are allowed. This forward-chaining logic programming
interpretation of persistent logic is extremely common; in fact, it is
what is commonly meant by ``forward-chaining logic programming.'' Just
as the term {\it persistent logic} was introduced in Chapter 2 to
distinguish what is traditionally referred to as intuitionistic logic
from intuitionistic ordered and linear logic, we will use the term
{\it saturating logic programming} to distinguish what is
traditionally referred to as forward-chaining logic programming from
the forward-chaining logic programming interpretation that makes sense
for ordered and linear logical specifications.

\section{Logical transformation: approximation}
\label{sec:abstraction}

Our approximation strategy is simple: a signature in an ordered or
linear logical specification can be approximated by making all atomic
propositions persistent, and a flat, persistent rule $\forall
\lf{\overline{x}}. \,A^+ \lefti \{ B^+ \}$ can be further approximated by
removing premises from $A^+$ and adding conclusions to $B^+$. Of
particular practical importance are added conclusions that equate the
parameters introduced by existential quantification with terms: all
parameters introduced by existential quantification must be dealt with
as a necessary condition for interpreting a persistent signature as a
saturating logic program.

First, we define what it means for a specification to be an approximate
version of another specification:

\bigskip
\begin{definition}\label{def:approxversion}
  A flat, concurrent, and persistent specification $\Sigma_a$ is an
  {\em approximate version} of another specification $\Sigma$ if every
  predicate ${\sf a} : \Pi \lf{x_1}{:}\tau_1 \ldots \Pi \lf{x_n}{:}\tau_n.\,
  {\sf prop}\,{\sf lvl}$ declared in $\Sigma$ has a corresponding
  predicate ${\sf a} : \Pi \lf{x_1}{:}\tau_1 \ldots \Pi \lf{x_n}{:}\tau_n.\,
  {\sf prop}\,{\sf pers}$ in $\Sigma_a$ and if for every rule ${\sf
    r} : \forall \overline{x}.\,A_1^+ \lefti \{ A_2^+ \}$ in $\Sigma$ there
  is a corresponding rule ${\sf r} : \forall \lf{\overline{x}}.\,B_1^+ \lefti
  \{ B_2^+ \}$ in $\Sigma_a$ such that:
  \begin{itemize}
  \item The existential quantifiers in $A_1^+$ and $A_2^+$ are
    identical to the existential quantifiers in $B_1^+$ and $B_2^+$
    (respectively),
  \item For each premise ($p^+_\mpers$ or $\lf{t} \doteq \lf{s}$) in $B^+_1$,
    the same premise appears in $A^+_1$, and 
  \item For each conclusion ($p^+_\mlvl$ or $\lf{t}
    \doteq \lf{s}$) in $A^+_2$, the same premise appears in $B^+_2$.
  \end{itemize}
\end{definition}
\bigskip

\noindent
While approximation is a program transformation, it is not a
deterministic one. Even the nondeterministic operationalization
transformation was just a bit nondeterministic, giving several options
for operationalizing any given deductive rule. The approximation
transformation, in contrast, needs explicit information from the user:
which premises should be removed, and what new conclusions should be
introduced? While there is value in actually implementing the
operationalization, defunctionalization, and destination-adding
transformations, applying approximation requires
intelligence. Borrowing a phrase from Danvy, approximation is a
candidate for ``mechanization by graduate student'' rather than
mechanization by computer.

Next, we give a definition of what it means for a state to be an 
approximate version (we use the word ``generalization'') of another state
or a family of states. 

\bigskip
\begin{definition}
  The persistent process state state $(\Psi_g; \Delta_g)$ is a {\em
    generalization} of the process state $(\Psi; \Delta)$ if there is
  a substitution $\Psi_g \vdash \lf{\sigma} : \Psi$ such that, for all
  suspended atomic propositions $p^+_\mlvl = {\sf a}\,\lf{t_1}\ldots \lf{t_n}$
  in $\Delta$, there exists a suspended persistent proposition
  $p^+_\mpers = {\sf a}\,\lf{(\sigma t_1)}\ldots\lf{(\sigma t_n)}$ in
  $\Delta_g$.
\end{definition}
\bigskip

One thing we might prove about the relationship between process states
and their generalizations is that generalizations can {\it simulate}
the process states they generalize: that is, that if $(\Psi_g;
\Delta_g)$ is a generalization of $(\Psi; \Delta)$ and $(\Psi; \Delta)
\leadsto_\Sigma (\Psi'; \Delta')$ then $(\Psi_g; \Delta_g)
\leadsto_{\Sigma_a} (\Psi'_g; \Delta'_g)$ where $(\Psi'_g; \Delta'_g)$
is a generalization of $(\Psi'; \Delta')$. This is true, but we're not
interested in generalization per se; rather, we're interested in a
stronger property, {\em abstraction}, that is defined in terms of
generalization:

\bigskip
\begin{definition}
A process state $(\Psi_a; \Delta_a)$ is an {\em abstraction} of 
$(\Psi_0; \Delta_0)$ under the signature $\Sigma$ if, for any trace
$(\Psi_0; \Delta_0) \leadsto^*_\Sigma (\Psi_n; \Delta_n)$, 
$(\Psi_a; \Delta_a)$ is a generalization of $(\Psi_n; \Delta_n)$. 
\end{definition}
\bigskip

An abstraction of the process state $(\Psi_0; \Delta_0)$ is therefore
a single process state that captures {\it all possible future
  behaviors} of the state $(\Psi_0; \Delta_0)$ because, for any atomic
proposition $p^+_\mlvl = {\sf a}\,\lf{t_1}\ldots \lf{t_n}$ that may be derived
by evolving $(\Psi_0; \Delta_0)$, there is a substitution $\lf\sigma$
such that ${\sf a}\,\lf{(\sigma t_1)}\ldots\lf{(\sigma t_n)}$ is already present
in the abstraction. The meta-approximation theorem relates this definition
of abstraction to the concept of approximate versions of programs as
specified by Definition~\ref{def:approxversion}.  

\bigskip
\begin{theorem}[Meta-approximation]\label{thm:metapprox}
  If $\Sigma_a$ is an approximate version of $\Sigma$, and if there is a
  state $(\Psi_0; \Delta_0)$ well-formed according to $\Sigma$, and if
  for some $\Psi_0' \vdash \lf\sigma : \Psi_0$ there is a trace
  $(\Psi_0'; \lf\sigma\Delta_0) \leadsto^*_{\Sigma_a} (\Psi_a; \Delta_a)$
  such that $(\Psi_a; \Delta_a)$ is a saturated process state, then
  $(\Psi_a; \Delta_a)$ is an abstraction of $(\Psi_0; \Delta_0)$.
\end{theorem}

\begin{proof}
As in \cite[Theorem 3]{simmons11logical}.
\end{proof}

The meaning of the meta-approximation theorem is that if (1) we can
approximate a specification and an initial state and (2) we can obtain
a saturated process state from that approximate specification
and approximate initial state, then the saturated process state captures
all possible future behaviors of the (non-approximate) initial state. 

\begin{figure}
\fvset{fontsize=\small,boxwidth=229pt}
\VerbatimInput{sls/pda-pers.sls}
\caption{Skolemized approximate version of the PDA specificaiton from
Figure~\ref{fig:pda-lin}}
\label{fig:pda-pers}
\end{figure}



\section{Using approximation}

The meta-approximation theorem gives us a way of building abstractions
from specifications and initial process states: we interpret the
approximate version of the program as a saturating logic program over
that initial state. If we can obtain a saturated process state using
the logic programming interpretation, it is an abstraction of the
initial process state. It is not always possible to obtain a saturated
process state using the logic programming interpretation, however:
rules like 
$\forall \lf{x}.\,{\sf a}\lf{(x)} \lefti \{ {\sf a}\lf{({\sf s}(x))} \}$ and 
$\forall \lf{x}.\,{\sf
  a}\lf{(x)} \lefti \{ \exists \lf{y}. {\sf a}\lf{(y)} \}$ 
lead to non-termination
when interpreted as saturating logic programs. Important classes 
of programs are known to terminate in all cases, such as those in the
so-called ``Datalog fragment'' where the only terms in the program
are variables and constants. However, we want to consider programs
that fall outside of this fragment, which means that we must 
reason explicitly about termination.

Consider the destination-passing PDA specification from
Figure~\ref{fig:pda-lin}. If we simply turn all linear predicates
persistent, then the ${\sf push}$ rule will lead to non-termination
because the head $\exists \lf{m}. {\sf stack}\,\lf{x}\,\lf{l}\,\lf{m} \fuse {\sf
  hd}\,\lf{m}\,\lf{r}$ introduces a new existential parameter $\lf{m}$. 
We can cope
by adding a new conclusion $\lf{m} \doteq \lf{t}$, but this means we 
have to pick a $\lf{t}$.
The most general starting point for selecting a $\lf{t}$ is to apply
Skolemization to the rule. By moving the existential quantifier for
$\lf{m}$ in front of the implicitly quantified $\lf{X}$, $\lf{L}$, $\lf{M}$, 
and $\lf{R}$, we
get a Skolem function $\lf{{\sf fm}\,X\,L\,M\,R}$ that takes four
arguments. Letting $\lf{t} = \lf{{\sf fm}\,X\,L\,M\,R}$ results in the
\sls~specification/logic program shown in
Figure~\ref{fig:pda-pers}.

Skolem functions provide a natural starting point for approximations,
even though the Skolem constant that arises directly from
Skolemization is usually more precise than we want. From the starting
point in Figure~\ref{fig:pda-pers}, however, we can define
approximations simply by instantiating the Skolem constant.  For
instance, we can equate the existentially generated destination in the
conclusion with the one given in the premise (letting $\lf{\sf fm} =
\lf{\lambda X. \lambda L. \lambda M. \lambda R.\, M}$). The result is
equivalent to this specification:

\smallskip
\fvset{fontsize=\small,boxwidth=229pt}
\VerbatimInput{sls/pda-pers-precise.sls}
\smallskip

\noindent This substitution yields a precise approximation
that exactly captures the behavior of the original PDA as a saturating
logic program. On the other hand, if we set $\lf{m}$ equal to $\lf{l}$ 
(letting $\lf{\sf fm} = \lf{\lambda X. \lambda L. \lambda M. \lambda R.\, L}$), 
the result is equivalent to this specification:

\smallskip
\fvset{fontsize=\small,boxwidth=229pt}
\VerbatimInput{sls/pda-pers-approx.sls}
\smallskip


If the initial process state contains a single atomic proposition
${\sf hd}\,\lf{d_0}\,\lf{d_1}$ in addition to all the ${\sf left}$ and ${\sf
  right}$ facts, then the two rules above maintain the invariant that,
as new facts are derived, the first argument of ${\sf hd}$ and the
second and third arguments of ${\sf stack}$ will {\it always} be
$\lf{d_0}$.  These arguments are therefore vestigial, like the extra
arguments to ${\sf eval}$ and ${\sf retn}$ discussed in
Section~\ref{sec:vestigial}, and we can remove them from the
approximate specification, resulting in the specification in
Figure~\ref{fig:pda-pers-approx2}. 

\begin{figure}[ht]
\fvset{fontsize=\small,boxwidth=229pt}
\VerbatimInput{sls/pda-pers-approx2.sls}
\caption{Approximated PDA specification}
\label{fig:pda-pers-approx2}
\end{figure}

This logical approximation of the
original PDA accepts every string where, for every form of bracket
$\lf{x}$, at least one left $\lf{x}$ appears before any of the right 
$\lf{x}$. The
string \obj{\mbox{\sf [ ] ] ] ( ( )}} 
would be accepted by this approximated PDA,
but the string \obj{\mbox{\sf ( ) ] [ [ ]}} 
would not, as the first right square
bracket appears before any left square bracket.


\section{Control flow analysis}
\label{sec:0cfa}

The initial process state for destination-passing SSOS specifications
generally has the form 
$(\lf{d}{:}{\sf dest}; x{:}\susp{{\sf eval}\,\lf{t}\,\lf{d}})$
for some program represented by the LF term $\lf{t} = \interp{e}$. 
This means that we
can use the meta-approximation result to derive abstractions from
initial expressions $\obj{e}$ using the saturating logic programming
interpretation of persistent SSOS specifications.

\begin{figure}
\fvset{fontsize=\small,boxwidth=229pt}
\VerbatimInput{sls/dest-env.sls}
\caption{Alternate environment semantics for CBV evaluation}
\label{fig:dest-env}
\end{figure}

The starting point for deriving a control flow analysis is the
environment semantics for call-by-value shown in
Figure~\ref{fig:dest-env}. It differs from the environment semantics
shown in Figure~\ref{fig:ssos-by-env} in three ways. First and
foremost, it is a destination-passing specification instead of an
ordered abstract machine specification, but that difference is
accounted for by the destination-adding transformation in Chapter~7.
A second difference is that the existentially generated parameter $\lf{x}$
associated with the persistent proposition ${\sf bind}\,\lf{x}\,\lf{v}$ is
introduced as late as possible (rule ${\sf ev/app2}$ in
Figure~\ref{fig:dest-env}), not as early as possible (rule ${\sf
  ev/app1}$ in Figure~\ref{fig:ssos-by-env}). The third difference is
that there is an extra frame $\lf{\sf app3}$ and an extra rule ${\sf
  ev/app3}$ that consumes such frames. The $\lf{\sf app3}$ frame is
an important part of the control flow analysis we derive, but in
\cite{simmons11logical} the addition of these frames was otherwise
unmotivated. Based on our discussion of the logical correspondence in
Chapter~6, we now have a principled account for this extra frame and
rule: it is precisely the pattern we get from operationalizing a
natural semantics {\it without} tail-recursion optimization and then
applying defunctionalization.

In order for us to approximate Figure~\ref{fig:dest-env} to derive a
finite control flow analysis, our first step is to deal with the
variables introduced by existential quantification. The variable $\lf{x}$
introduced in ${\sf ev/app2}$ will be equated with $\lf{{\sf var}(\lambda
x.E\,x)}$. The new constructor $\lf{\sf var}$ is a greatly simplified
Skolem function for $\lf{x}$ that only mentions the term $\lf{E}$ with LF type
${\sf exp} \rightarrow {\sf exp}$ -- the most general Skolem function
in this setting would have also been dependent on $\lf{V}$, $\lf{D}$, and
$\lf{D_2}$. Adding $\lf{x} \doteq \lf{{\sf var}(\lambda x.E\,x)}$ 
as a conclusion to
the rule ${\sf ev/lam2}$ effectively causes us to associate all
variables ever passed to an LF function with the {\it function into
  which that parameter was passed}.

\begin{figure}[t]
\fvset{fontsize=\small,boxwidth=229pt}
\VerbatimInput{sls/dest-cfa-1.sls}
\caption{A control-flow analysis derived from Figure~\ref{fig:dest-env}}
\label{fig:dest-cfa-1}
\end{figure}

The pattern above is an important one in the design of saturating
forward-chaining logic programs that use higher-order abstract syntax,
because it is a simple way of getting a handle on the subterms of a
higher-order term. When given a term $\lf{{\sf a}\,({\sf b}\,{\sf c}\,{\sf
  c})}$, it is clear that there are three distinct subterms: the entire
term, $\lf{{\sf b}\,{\sf c}\,{\sf c}}$, and $\lf{{\sf c}}$. Therefore, it is
meaningful to bound the size of a saturated process state using some
function that depends on the number of subterms of the original
term. But what are the subterms of $\lf{{\sf lam}\,(\lambda x.\,{\sf
  app}\,x\,x)}$?  Because we ensure that we {\it only} substitute terms
of the form $\lf{{\sf var}(\lambda x.e\,x)}$ into functions of the form
$\lf{\lambda x.e\,x}$ we can actually answer this question: there are three
distinct subterms of $\lf{{\sf lam}\,(\lambda x.\,{\sf app}\,x\,x)}$: the
entire term, $\lf{{\sf app}\,({\sf var}(\lambda x.\,{\sf
  app}\,x\,x))\,({\sf var}(\lambda x.\,{\sf app}\,x\,x))}$, and $\lf{{\sf
  var}(\lambda x.\,{\sf app}\,x\,x)}$.  The subterms of any closed term
$\lf{e}$ of LF type ${\sf exp}$ can be enumerated by running this
saturating logic program starting with the fact ${\sf subterms}\lf{(e)}$,
where ${\sf subterms}$ is a persistent positive proposition.
\begin{align*}
{\sf sub/lam} &: ~~
  \forall \lf{e}{:}{\sf exp} \rightarrow {\sf exp}.\,
    {\sf subterms}\lf{({\sf lam}(\lambda x.e\,x))} \lefti
      \{ {\sf subterms}\lf{(e({\sf var}(\lambda x.e\,x)))} \}
\\
{\sf sub/app} &: ~~
  \forall \lf{e_1}{:}{\sf exp}.\, \forall \lf{e_2}{:}{\sf exp}.\,
    {\sf subterms}\lf{({\sf app}\,e_1\,e_2)} \lefti
      \{ {\sf subterms}\,\lf{e_1} \fuse {\sf subterms}\,\lf{e_2} \}
\\
{\sf sub/var} &: ~~ 
  \forall \lf{e}{:}{\sf exp} \rightarrow {\sf exp}.\,
    {\sf subterms}\lf{({\sf var}(\lambda x.e\,x))} \lefti
      \{ {\sf subterms}\lf{(e({\sf var}(\lambda x.e\,x)))} \}
\end{align*}
The last rule, ${\sf sub/var}$, is redundant: if we ever derive a fact
of the form ${\sf subterms}\lf{({\sf var}(\lambda x.e\,x))}$, we know that
we previously derived ${\sf subterms}\lf{({\sf lam}(\lambda x.e\,x))}$ and
therefore, by the first rule, we have already derived ${\sf
  subterms}\lf{(e({\sf var}(\lambda x.e\,x))}$.



The discussion above pertains to the existentially generated variable
$\lf{x}$ in rule ${\sf ev/app2}$, but we still need some method for
handling destinations $\lf{d_1}$, $\lf{d_2}$, and $\lf{d_3}$ in ${\sf ev/app1}$,
${\sf ev/app2}$, and ${\sf ev/app3}$ (respectively). To this end, we
need to have in mind the question that we intend to answer with
control flow analysis. The primary question that such a flow analysis
is intended to answer is, ``for any given call site in the source
program, what are the functions that may be invoked at that
location?''\footnote{This kind of ``{\it may}-'' analysis, where the
  intention is to over-approximate the events that might happen, is
  the kind of analysis (as opposed to a ``{\it must}-'' analysis) that
  maps easily onto the meta-approximation theorem.} Call sites
correspond to expressions of the form $\lf{{\sf app}\,e_1\,e_2}$, and
functions are expressions of the form $\lf{{\sf lam}(\lambda x.e\,x)}$;
therefore, our next step is to equate the destinations introduced in
the $\lf{\sf app}$ rules with the expressions we are evaluating at this
point. One way to do this would be to introduce a new constructor
$\lf{\sf d}: {\sf exp} \rightarrow {\sf dest}$, but we can equivalently
conflate the two types to get the specification in
Figure~\ref{fig:dest-cfa-1}. 


The specification in Figure~\ref{fig:dest-cfa-1} has a point of
redundancy along the lines of the redundancy in our second PDA
approximation: the rules maintain the invariants that the two
arguments to ${\sf eval}\,\lf{e}\,\lf{d}$ are always the same. Therefore, the
second argument to ${\sf eval}$ can be treated as vestigial; by
removing that argument, we get a specification equivalent to
Figure~\ref{fig:dest-cfa-2}. That figure includes another
simplifications as well: instead of introducing expressions $\lf{d_1}$,
$\lf{d_2}$, and $\lf{d_3}$ by existential quantification just to equate them
with expressions $\lf{e_1}$, $\lf{e_2}$, and $\lf{e}$, 
we substitute in the equated
expressions where the respective destinations appeared in
Figure~\ref{fig:dest-cfa-1}. 

\begin{figure}[t]
\fvset{fontsize=\small,boxwidth=229pt}
\VerbatimInput{sls/dest-cfa-2.sls}
\caption{Simplification of Figure~\ref{fig:dest-cfa-1} that
  eliminates the redundant argument to ${\sf eval}$}
\label{fig:dest-cfa-2}
\end{figure}

This specification in Figure~\ref{fig:dest-cfa-2} is terminating when
interpreted as a saturating logic program because the rules only break
expressions $\lf{e}$ and values $\lf{v}$ 
into their subexpressions in the sense
we have described above. If the expressions in the original state of a
program have $n$ subterms, the program can derive no more than $n$ new
${\sf eval}$ facts, $n^2$ new ${\sf retn}$ facts, and $2n^2 + n$ ${\sf
  cont}$ facts. This analysis combined with the meta-approximation
theorem ensures that we have derived some sort of program analysis. To
see that this program analysis is a control flow analysis, note that
the third argument $\lf{e}$ to ${\sf comp}\,\lf{f}\,\lf{e'}\,\lf{e}$ 
is always a term
$\lf{{\sf app}\,e_1\,e_2}$ -- that is, a call site. The rule ${\sf
  ev/app2}$ starts evaluating the function $\lf{{\sf lam}(\lambda
x.e_0\,x)}$ and generates the fact ${\sf comp}\,\lf{\sf app3}\,\lf{(e({\sf
  var}(\lambda x.e_0\,x)))}\,\lf{e}$. This means that, in the course of
evaluating some initial expression $\lf{e_{\it init}}$, the function $\lf{{\sf
  lam}(\lambda x.e_0\,x)}$ may be called from the call site $\lf{e}$ only if
${\sf comp}\,\lf{\sf app3}\,\lf{(e({\sf var}(\lambda x.e_0\,x)))}\,\lf{e}$ appears
in a saturated process state that includes the persistent 
atomic proposition ${\sf eval}\lf{(e_{\it init})}$.

There is one important caveat to this analysis. If for some value $\obj{v}$
we consider the program $\interp{((\lambda x.x)\,(\lambda y.y))\,v}$,
we might expect a reasonable control flow analysis to notice that only
$\interp{\lambda y.y}$ is passed to the function $\interp{\lambda
  x.x}$ and that only $\obj{v}$ is passed to the function $\interp{\lambda
  y.y}$. Because of our use of higher-order abstract syntax, however,
$\interp{\lambda y.y}$ and $\interp{\lambda x.x}$ are alpha-equivalent
and therefore equal in the eyes of the logic programming
interpreter. This is not a problem with correctness, but it means that
our analysis may be less precise than expected. One solution would be
to add distinct labels to terms. Adding a label on the inside of every
lambda-abstraction would seem to suffice, and in any real example
labels would already be present in the form of source-code positions
or line numbers. The alias analysis presented next demonstrates the
use of such labels.

\begin{figure}
\fvset{fontsize=\small,boxwidth=229pt}
\VerbatimInput{sls/ssos-monadic.sls}
\caption{Semantics of functions in the simple monadic language}
\label{fig:ssos-monadic}
\end{figure}

\section{Alias analysis}

The control flow analysis above was derived from the SSOS
specification of a language that looked much like the Mini-ML-like
languages considered in Chapters~6~and~7.  In this section, we try to
derive an alias analysis that matches the interprocedural
object-oriented alias analysis presented as a logic program in
\cite[Chapter 12.4]{aho07compilers}.  When we try to reverse-engineer
a source language from which this alias analysis can be derived, we
get the monadic language presented below.

\subsection{Monadic language}

The language we consider differentiates atomic actions, which we will
call {\it expressions} (and encode in the LF type ${\sf exp}$) and
procedures or {\it commands} (which we encode in the LF type ${\sf
  cmd}$). There are only two commands $\obj{m}$ in our monadic language. The
first command, $\obj{{\sf ret}\,x}$, is a command that returns the value
bound to the variable $\obj{x}$ (rule ${\sf ev/ret}$ in
Figure~\ref{fig:ssos-monadic}). The second command, $\interp{{\sf
    bnd}^l\, x \leftarrow e \,{\sf in}\,m} = \lf{{\sf
  bnd}\,l\,\interp{e}\,\lambda x.\interp{m}}$, evaluates $\obj{e}$ to a
value, binds that value to the variable $\obj{x}$, and then evaluates
$\obj{m}$. Note the presence of $\obj{l}$ 
in the bind syntax; we will call it a
{\it label}, and we can think of it as a line number or source-code
position from the original program.

In the previous languages we have considered, values $\obj{v}$ were a
syntactic refinement of the expressions $\obj{e}$. In contrast, our monadic
language will differentiate the two: there are five expression forms
and three values that we will consider. An expression $\interp{\lambda
  x.e} = \lf{{\sf fun}\,\lambda x.\interp{e}}$ evaluates to a value
$\interp{\lambda^l x.e} = \lf{{\sf lam}\,l\,\lambda x.\interp{e}}$, where
the label $\obj{l}$ represents the source code position where the function
was bound. When we evaluate the command $\interp{{\sf
    bnd}^l \, y \leftarrow \lambda x.e \,{\sf in}\,m}$, the value
$\interp{\lambda^l x.e}$ gets bound to $\obj{y}$ in the body of the command
$\obj{m}$ (rule ${\sf ev/fun}$ in Figure~\ref{fig:ssos-monadic}).

The second expression form is a function call: $\interp{f\,x} = \lf{{\sf
  app}\,f\,x}$. To evaluate a function call, we expect a function value
(which is a command $\obj{m_0}$ with one free variable) to be bound to the
variable $\obj{f}$; we then store the rest of the current command on the
stack and evaluate the command $\obj{m_0}$ to a value. Note that the rule
${\sf ev/call}$ in Figure~\ref{fig:ssos-monadic} also stores the call
site's source-code location $\obj{l}$ on the stack frame. The reason for
storing a label here is that we need it for the alias
analysis. However, it is possible to independently motivate adding
these source-code positions to the operational semantics: for instance, it
would allow us to model the process of giving a stack trace when an
exception is raised. When the function we have called returns (rule
${\sf ev/call1}$ in Figure~\ref{fig:ssos-monadic}), we continue
evaluating the command that was stored on the control stack.

\begin{figure}
\fvset{fontsize=\small,boxwidth=229pt}
\VerbatimInput{sls/ssos-monadic2.sls}
\caption{Semantics of mutable pairs in the simple monadic language}
\label{fig:ssos-monadic2}
\end{figure}

The rules for mutable pairs are given in
Figure~\ref{fig:ssos-monadic2}. Evaluating the expression $\obj{\sf
  newpair}$ allocates a tuple with two fields $\obj{\sf fst}$ and 
$\obj{\sf
  snd}$ and yields a value $\obj{{\sf loc}\,l}$ referring to the tuple; both
fields in the tuple are initialized to the value $\obj{\sf null}$, and
each field is represented by a separate linear ${\sf cell}$ resource
(rule ${\sf ev/new}$). The expressions $\interp{x.{\sf fst}} = \lf{{\sf
  proj}\,x\,{\sf fst}}$ and $\interp{x.{\sf snd}} = \lf{{\sf proj}\,x\,{\sf
  snd}}$ expect a pair location to be bound to $\obj{x}$, and yield the value
stored in the appropriate field of the mutable pair (rule ${\sf
  ev/proj}$). The expressions $\interp{x.{\sf fst} := y} = \lf{{\sf
  set}\,x\,{\sf fst}\,y}$ and $\interp{x.{\sf snd} := y} = \lf{{\sf
  set}\,x\,{\sf snd}\,y}$ work much the same way. The difference is
that the former expressions do not change the accessed field's
contents, whereas the latter expressions replace the accessed field's
contents with the value bound to $\obj{y}$ (rule ${\sf ev/set}$).

This language specification bears some similarity to Harper's
Modernized Algol with free assignables \cite[Chapter
36]{harper12practical}. The {\it free assignables} addendum is
critical: SSOS specifications do not have a mechanism for enforcing
the stack discipline of Algol-like languages. The other major
difference is that Harper's version of Algol does not allow
expressions to access the state.  For our language to behave
similarly, the expressions $\obj{\sf newpair}$, $\obj{x.{\sf fst}}$, 
$\obj{x.{\sf
  snd}}$, $\obj{x.{\sf fst} := y}$, and 
$\obj{x.{\sf snd} := y}$ would need to be
commands and not expressions; we make them expressions only for the
convenience of a more regular presentation.

\begin{figure}
\fvset{fontsize=\small,boxwidth=229pt}
\VerbatimInput{sls/ssos-monadic-approx.sls}
\caption{Alias analysis for the simple mondaic language}
\label{fig:ssos-monadic-approx}
\end{figure}



\subsection{Approximation and alias analysis}


To approximate the semantics of our monadic language, we can follow the
methodology from before and turn the program persistent. A further
approximation is to remove the last premise from ${\sf ev/set}$, as
the meta-approximation theorem allows -- the only purpose of this
premise in Figure~\ref{fig:ssos-monadic2} was to consume the ephemeral
proposition ${\sf cell}\,\lf{l'}\,\lf{\it fld}\,\lf{v}$, and this is unnecessary
if ${\sf cell}$ is not an ephemeral predicate.  Having
made these two moves (turning all propositions persistent, and removing
a premise from ${\sf ev/set}$), we are left with three types of
existentially-generated variables that must be equated with concrete
terms in order for our semantics to be interpreted as a saturating
logic program:

\smallskip
\begin{itemize}
\item Variables $\lf{y}$, introduced by every rule except for ${\sf ev/ret}$,
\item Mutable locations $\lf{l}$, introduced by rule ${\sf ev/new}$, and 
\item Destinations $\lf{d}$; the only place where a destination is created
by the destination-adding transformation is in rule ${\sf ev/call}$.
\end{itemize}
\smallskip

Variables $\lf{y}$ are generated to be substituted into the body of some
command, so we could equate them with the Skolemized function body as
we did when deriving a control flow analysis example. Another option
comes from noting that, for any initial source program, every command
is associated with a particular source code location, so a simpler
alternative is just to equate the variable with that source code
location. This is why we stored labels on the stack: if we had not
done so, then the label $\obj{l}$ associated with $\obj{m}$ in the command
$\interp{{\sf bnd}^l\, x \leftarrow \lambda x.m_0 \,{\sf in}\,m}$ would
no longer be available when we needed it in rule ${\sf ev/call1}$.

We deal with mutable locations $\lf{l}$ is a similar manner: we equate them
with the label $\lf{l}$ representing the line where that cell was
generated.

There are multiple ways to deal with the destination $\lf{d_0}$ generated
in rule ${\sf ev/call}$. We want our analysis, like Aho et al.'s, to
be insensitive to control flow, so we will equate $\lf{d_0}$ with the label
$\lf{l_0}$ associated with the function we are calling.  If we instead
equated $\lf{d_0}$ with the label $\lf{l}$ associated with the call-site or with
the pair of the call site and the called function, the result would be
an analysis that is more sensitive to control flow.

The choices described above are reflected in
Figure~\ref{fig:ssos-monadic-approx}, which takes the additional step
of inlining uses of equality in the conclusions of rules. We can
invoke this specification as a program analysis by packaging a program
as a single command $\obj{m}$ and deriving a saturated process state from the
initial process state $(\lf{l_{\it init}}{:}{\sf loc}; x{:}\susp{{\sf
    eval}\,\interp{m}\,\lf{l_{\it init}}})$. 
The use of source-code position labels
makes the answers to some of the primary questions asked of an alias
analysis quite concise. For instance:

\smallskip
\begin{itemize}
\item {\it Might the first component of a pair created at label
    $\obj{l_1}$ ever reference a pair created at label $\obj{l_2}$?}
  Only if ${\sf cell}\,\lf{l_1}\,\lf{\sf fst}\,\lf{({\sf loc}\,l_2)}$
  appears in the saturated process state (and likewise for the second
  component).
\item {\it Might the first component of a pair created at label
    $\obj{l_1}$ ever reference the same object as the first component
    of a pair created at label $\obj{l_2}$?} Only if there is some
  $\lf{l'}$ such that ${\sf cell}\,\lf{l_1}\,\lf{\sf fst}\,\lf{({\sf
      loc}\,l')}$ and ${\sf cell}\,\lf{l_1}\,\lf{\sf fst}\,\lf{({\sf
      loc}\,l')}$ both appear in the saturated process state.
\end{itemize}

\section{Related work}
\label{sec:approximately-related}

The technical aspects of linear logical approximation are similar to
work done by Bozzano et al.~\cite{bozzano02effective,bozzano04model},
which was also based on the abstract interpretation of a logical
specification in linear logic.
They encode distributed systems and communication protocols in
a framework that is similar to the linear fragment of \sls~without
equality. Abstractions of those programs are then used to verify
properties of concurrent protocols that were encoded in the logic
\cite{bozzano02protocol}. 

There are a number of significant difference between our work and
Bozzano et al.'s, however. The style they use to encode protocols is
significantly different from any of the SSOS specification styles
presented in this thesis. They used a general purpose approximation,
which could therefore potentially be mechanized in the same way we
mechanized transformations like operationalization; in contrast, the
meta-approximation result described here captures a whole class of
approximations. Furthermore, Bozzano et al.'s methods are designed to
consider properties of a system as a whole, not static analyses of
individual inputs as is the case in our work.

Work by Might and Van Horn on abstracting abstract machines can be
seen as a parallel approach to our methodology in a very different
setting
\cite{might10resolving,might10abstract,might10abstracting}. Their
emphasis is on deriving a program approximation by approximating a
{\it functional} abstract interpreter for a programming language's
operational semantics. Their methodology is similar to ours in large
part because we are doing the same thing in a different setting,
deriving a program approximation by approximating a
destination-passing SSOS specification (which we could, in turn, have
derived from an ordered abstract machine by destination-adding).

Many of the steps that they suggest for approximating programs have
close analogues in our setting. For instance, their {\it
  store-allocated bindings} are analogous to the SSOS environment
semantics, and their {\it store-allocated continuations} -- which they
motivate by analogy to implementation techniques for functional
languages like SML/NJ -- are precisely the destinations that arise
naturally from the destination-adding transformation. The first
approximation step we take is forgetting about linearity in order to
obtain a (non-terminating) persistent logical specification. This step
is comparable to Might's first approximation step of ``throwing hats
on everything'' (named after the convention in abstract interpretation
of denoting the abstract version of a state space $\Sigma$ as
$\hat{\Sigma}$. The ``mysterious'' introduction of power domains that
this entails is, in our setting, a perfectly natural result of
relaxing the requirement that there be at most one persistent
proposition ${\sf bind}\,\lf{x}\,\lf{v}$ for every $\lf{x}$. 
As a final point of
comparison, the ``abstract allocation strategy'' discussed in
\cite{might10abstracting} is quite similar to our strategy of
introducing and then approximating Skolem functions as a means of
deriving a finite approximation. Our current discussion of Skolem
functions in Section~\ref{sec:0cfa} is partially inspired by the
relationship between our use of Skolemization and the discussion of
abstract allocation in \cite{might10abstracting}.

The independent discovery of a similar set of techniques for achieving
similar goals in such different settings (though both approaches were
to some degree inspired by Van Horn and Mairson's investigations of
the complexity of $k$-CFA \cite{vanhorn07relating}) is another
indication of the generality of both techniques, and the similarity
also suggests that the wide variety of approximations considered in
\cite{might10abstracting}, as well as the approximations of
object-oriented programming languages in \cite{might10abstract}, can
be adapted to this setting.


\part{Reasoning about substructural logical specifications}

\chapter{Generative invariants}
\label{chapter-gen}

So far in this thesis, we have presented \sls~as a framework for
presenting transition systems. This view focuses on synthetic
transitions as a way of relating pairs of process states, either with
one transition $(\Psi; \Delta) \leadsto (\Psi'; \Delta')$ or with a
series of transitions $(\Psi; \Delta) \leadsto^* (\Psi';
\Delta')$. This chapter will focus on another view of concurrent
\sls~specifications as {\it grammars} for describing well-formed
process states. This view was presented previously in the discussions
of adequacy in Section~\ref{sec:framework-reggenworld} and in
Section~\ref{sec:nat-ssos-adequacy}.

The grammar-like specifications that describe well-formed process
states are called {\it generative signatures}, and generative
signatures can be used to specify sets of process states, or {\it
  worlds}. By the analogy with grammars, we could also describe worlds
as {\it languages} of process states recognized by the grammar. In our
previous discussions of adequacy in
Section~\ref{sec:framework-reggenworld} and in
Section~\ref{sec:nat-ssos-adequacy}, the relevant world was a set of
process states that we could put in bijective correspondence with the
states of an abstract machine.

Our primary use of generative specifications in this thesis is showing
that, under some generative signature $\Sigma_{\it Gen}$ that defines
a world $\mathcal W$, whenever $(\Psi; \Delta) \in \mathcal W$ and
$(\Psi; \Delta) \leadsto_\Sigma (\Psi'; \Delta')$ it is always the
case that $(\Psi'; \Delta') \in \mathcal W$.  (The signature $\Sigma$
encodes the transition system we are studying.)  In such a case, the
world or language of well-formed process states is called a {\it
  generative invariant} of $\Sigma$.

\subsection*{Type preservation}

The purpose of this chapter is to demonstrate that generative
invariants are a reasonable way of talking about invariants of
\sls~specifications, especially well-formedness and well-typedness
invariants of substructural operational semantics specifications like
the ones presented in Part II. As we have already
seen, well-formedness invariants are major part of adequacy
theorems. Well-typedness invariants are important because they allow us to
prove {\it language safety}, the property (discussed way back in the
introduction) that a language specification is completely free
from undefined behavior.

We don't generally expect all syntactically well-formed expressions
$\obj{e}$ to be free of undefined behavior. When we want to prove
language safety for a small-step SOS specification like $\obj{e
  \mapsto e'}$ from Section~\ref{sec:evaluationcontexts} and the
beginning of Chapter~\ref{chapter-absmachine}, we also define a
judgment $\obj{x_1{:}{\it tp_1},\ldots, x_n{:}{\it tp_n} \vdash e :
  {\it tp}}$.  This {\it typing judgment} expresses that $\obj{e}$ has
type $\obj{\it tp}$ if the expression variables $\obj{x_1, \ldots,
  x_n}$ are respectively assumed to have the types $\obj{\it tp_1,
  \ldots, tp_n}$. (Note that $\obj{\it tp}$ is an {\it object-level
  type} as described in Section~\ref{sec:gen-ordertp}, not an LF type
$\tau$ from Chapter~\ref{chapter-framework}.) Using the typing
judgment, we can prove the safety theorem -- all well-typed
expressions are free from undefined behavior -- by way of two
theorems. The first theorem is {\it preservation}: if $\obj{\cdot
  \vdash e : {\it tp}}$ and $\obj{e \mapsto e'}$ then $\obj{\cdot
  \vdash e' : {\it tp}}$. The second theorem is {\it progress}: if
$\obj{\cdot \vdash e : {\it tp}}$, then either there is some
$\obj{e'}$ such that $\obj{e \mapsto e'}$ or else $\obj{e}$ is already
a value.

In SSOS specifications, generative specifications are analogous to
typing judgments, and establishing a generative invariant is analogous
to proving a preservation theorem. Progress theorems for SSOS
specifications will be discussed in Chapter~\ref{chapter-safety}.
These two chapters establish the centerpiece of the third refinement of
our central thesis:

\smallskip
\begin{quote} 
  {\bf Thesis (Part III):} {\it The \sls~specification of the operational
    semantics of a programming language is a suitable basis for formal
    reasoning about properties of the specified language.}
\end{quote} 


\subsection*{Overview}

In Section~\ref{sec:gen-worlds} we review how generative signatures
define a world and show how the {\it regular worlds} that Sch\"urmann
implemented in Twelf \cite{schurmann00automating} fall out as a
special case of the worlds described by generative signatures.  We
will then discuss invariants of operationalized ordered abstract
machines, introduced in Section~\ref{sec:nat-ssos-adequacy}, more
generally (Section~\ref{sec:gen-order}). In
Section~\ref{sec:gen-ordertp} we will extend that discussion from
well-{\it formed} process states to well-{\it typed} process states.
This is not a large technical shift, but conceptually it is an
important step from thinking about adequacy-like properties to
thinking about preservation theorems. 
In Section~\ref{sec:gen-state}
we describe how generative invariants can be established for the sorts
of stateful signatures considered in
Section~\ref{sec:richer-ordered-abstract}. In
Section~\ref{sec:gen-destinations} we consider invariants for
specifications in the image of the destination-adding transformation
from Chapter~\ref{chapter-destinations}.  In
Section~\ref{sec:gen-letcc} we consider the peculiar case of
first-class continuations, which require us to use persistent
continuation frames as described in Section~\ref{sec:dest-continuations}. 
%Finally, in Section~\ref{sec:gen-count} we
%introduce a more complicated class of generative invariants that
%capture the numerical properties of specifications that appear in the
%\sls~encoding of voting and auction protocols.

All of these sections are essentially replaying the same
two-or-three-step proof with variations. The first step, which is not
always necessary, is reasoning about unique indices, a topic
introduced in Section~\ref{sec:uniqueness-genstate} and refined in
Section~\ref{sec:uniqueness-gendests}. The second step is to prove an
inversion lemma which allows us to gather knowledge about the last
steps of a generative trace based on our knowledge of the structure of
the generated process state. The third step is a preservation lemma
where we take in a generative trace, perform inversion upon it, and
then use the results of inversion to construct a new generative trace.

\section{Worlds}
\label{sec:gen-worlds}

{\it Worlds} are nothing more or less than sets of process states
$(\Psi; \Delta)$ -- in the discussion here, we will specifically
exclude process states $(\Psi; \Delta)_{\lf\sigma}$ with a non-empty
accompanying substitution $\lf\sigma$.\footnote{These accompanying
  substitutions were presented in
  Section~\ref{sec:sls-processstates}.}
In this chapter, we will specify worlds by the combination of an
initial process state and a generative signature.

\bigskip
\begin{definition}\label{def:gensig}
  A {\em generative signature} is a \sls~signature where the ordered,
  mobile, and persistent atomic propositions can be separated into two
  sets -- the {\em terminals} and the {\em nonterminals}. Synthetic
  transitions enabled by a generative signature only consume (or
  reference) nonterminals and LF terms, but their output variables can
  include LF variables, variables associated with terminals, and
  variables associated with nonterminals.
\end{definition}
\bigskip

\noindent
A generative signature describes a world with the help of the
restriction operator $\restrictsig{(\Psi; \Delta)}{\Sigma}$ introduced
in Section~\ref{sec:framework-restriction}. If $(\Psi; \Delta)$ is
well-defined under the generative signature $\Sigma_{\it Gen}$, and
$\Sigma$ is any signature that includes all of the generative
signature's terminals (and all of its LF bits) but none of its
nonterminals, then $\restrictsig{(\Psi; \Delta)}{\Sigma}$ is only
defined when the only remaining nonterminals in $\Delta$ are
persistent and can therefore be filtered out of $\Delta$. (As long as
$\Sigma_{\it Gen}$ and $\Sigma$ have the same LF declarations, the LF
context $\Psi$ won't have anything filtered out.) When the
classification of terminals and nonterminals is clear, we will leave
off the restricting signature and just write $\restrictsig{(\Psi;
  \Delta)}{}$.

Definition~\ref{def:gensig} is intentionally quite broad -- it need
not even be decidable whether a process state belongs to a particular
world.\footnote{Proof: consider the initial state
  $(x{:}\istrue{\susp{\sf gen}})$ and the rule $\forall
  \lf{e}.\,\forall\lf{v}.\,{\sf gen} \fuse {!}({\sf
    ev}\,\lf{e}\,\lf{v}) \lefti \{ {\sf terminating}\,\lf{e} \}$. The
  predicate ${\sf gen}$ is a nonterminal, the predicate ${\sf
    terminating}$ is a terminal, and ${\sf ev}$ is the encoding of
  big-step evaluation $\obj{e \Downarrow v}$ from
  Figure~\ref{fig:example-transform-cbv}.  The language described is
  isomorphic to the set of $\lambda$-calculus expressions that terminate
  under a call-by-value strategy,
  and membership in that set is undecidable.} Future tractable
analyses will therefore presumably be based upon further restrictions
of the very general Definition~\ref{def:gensig}.  Context-free
grammars are one obvious specialization of generative signatures; we
used this correspondence as an intuitive guide in
Section~\ref{sec:framework-reggenworld}.  Perhaps less obviously,
the regular worlds of Twelf \cite{schurmann00automating} are another
specialization of generative signatures.


\subsection{Regular worlds}
\label{sec:gen-regularworlds}

The
regular world specifications used in Twelf
\cite{schurmann00automating} are made up of {\it blocks}. A block
describes a little piece of an LF context, and is declared in the LF
signature as follows:
\[
 {\sf blockname} :
 {\sf some}~\{\lf{a_1}{:}\tau_1\}\ldots\{\lf{a_n}{:}\tau_n\}
~{\sf block}~\{\lf{b_1}{:}\tau'_1\}\ldots\{\lf{b_m}{:}\tau'_m\}
\]
A block declaration is well formed in the signature $\Sigma$ if 
$\left(\lf{a_1}{:}\tau_1,\ldots,\lf{a_{i-1}}{:} \tau_{i-1} \right)\vdash_\Sigma
\tau_i \,{\sf type}$ for $1 \leq i \leq n$ and if 
$\left(\lf{a_1}{:}\tau_1,\ldots,\lf{a_{n}}{:} \tau_{n},
 \lf{b_1}{:}\tau'_1,\ldots,\lf{b_{j-1}}{:}\tau'_{j-1}\right) \vdash_\Sigma
\tau_j \,{\sf type}$ for $1 \leq j \leq m$. 

The first list of LF variable bindings
$\{\lf{a_1}{:}\tau_1\}\ldots\{\lf{a_n}{:}\tau_n\}$ that 
come after the ${\sf some}$ keyword describe the types
of concrete LF terms that must exist for the block to be well formed.
The second list of LF variable bindings represents the bindings that
the block actually adds to the LF context. The regular worlds of 
Twelf are defined as sets of block identifiers 
$({\sf block1} \mid \ldots \mid {\sf blockn})$. A set of block identifiers
and a Twelf signature $\Sigma$ define a world as follows: if
\smallskip
\begin{itemize}
\item $\Psi$ is a well-formed
LF context in the current world, 
\item ${\sf blockname} :
 {\sf some}~\{\lf{a_1}{:}\tau_1\}\ldots\{\lf{a_n}{:}\tau_n\}
~{\sf block}~\{\lf{b_1}{:}\tau'_1\}\ldots\{\lf{b_m}{:}\tau'_m\} \in \Sigma$
 is one of the blocks in the current world, and
\item there is a $\lf{\sigma}$ such that
$\Psi \vdash_\Sigma \lf{\sigma} :
\lf{a_1}{:}\tau_1,\ldots,\lf{a_n}{:}\tau_n$, 
\end{itemize}
\smallskip then $\Psi, \lf{b_1}{:}\lf\sigma\tau'_1,\ldots,
\lf{b_m}{:}\lf\sigma\tau'_m$ is also a well-formed LF context in the
current world. The {\it closed world}, which contains only the empty
process state, is a subset of every regular world, so this definition
gives us the rules for generating a world -- a set of contexts -- from
a series of block declarations.

One simple example of a regular world (previously discussed in
Section~\ref{sec:framework-reggenworld}) is the world that just
contains arbitrary expression variables with LF type ${\sf exp}$. This
world can be described with the block ${\sf blockexp}$:
\[
 {\sf blockexp} : 
 {\sf some}
~{\sf block}~\{{\lf x}{:}{\sf exp}\}
\]
If we had a judgment ${\sf natvar}\,\lf{x}\,\lf{n}$ that associated
every LF variable $\lf{x}{:}{\sf exp}$ with some natural number
$\lf{n}{:}{\sf nat}$, then in order to make sure that every expression
variable was associated with some natural number we would use the world
described by this block:
\[
 {\sf blocknatvar} : 
 {\sf some}~\{{\lf n}{:}{\sf nat}\}
~{\sf block}~\{{\lf x}{:}{\sf exp}\}~
               \{\lf{\it nv}{:}{\sf natvar}\,\lf{x}\,\lf{n}\}
\]
The world described by the combination of ${\sf blockexp}$ and ${\sf
  blocknatvar}$ is one where every LF variable $\lf{x}{:}{\sf exp}$
is associated with at most one LF variable of type ${\sf
  natvar}\,\lf{x}\,\lf{n}$. Assuming that there are no constants of
type ${\sf natvar}$, a property we can easily enforce with subordination, this
gives us a uniqueness property: if ${\sf natvar}\,\lf{x}\,\lf{n}$ and
${\sf natvar}\,\lf{x}\,\lf{m}$, then $\lf{m} = \lf{n}$. 

\subsection{Regular worlds from generative signatures}

A block declaration ${\sf blockname} :
 {\sf some}~\{\lf{a_1}{:}\tau_1\}\ldots\{\lf{a_n}{:}\tau_n\}
~{\sf block}~\{\lf{b_1}{:}\tau'_1\}\ldots\{\lf{b_m}{:}\tau'_m\}$ can
be described by one rule in a generative signature:
\begin{align*}
&{\sf blockname} : 
  \forall \lf{a_1}{:}\tau_1\ldots \forall\lf{a_n}{:}\tau_n.\,
  \{ \exists \lf{b_1}{:}\tau'_1 \ldots \lf{b_m}{:}\tau'_m.\,
     \one
  \}
\intertext{Because a regular world is just a set of blocks, 
the generative signature corresponding
to a regular world contains one rule for each block in the regular
worlds description.
The world $({\sf blockexp} \mid {\sf blockvar})$ corresponds
to the following generative signature:}
&{\sf nat} : {\sf type}, 
\\
&\mbox{\it \ldots declare constants of type ${\sf nat}$ \ldots }
\\
&{\sf exp} : {\sf type}, 
\\
&\mbox{\it \ldots declare constants of type ${\sf exp}$ \ldots }
\\
&{\sf blockexp} : 
  \{ \exists \lf{x}{:}{\sf exp}.\,\one\},
\\
&{\sf blocknatvar} : \forall \lf{n}{:}{\sf nat}.\,
  \{ \exists \lf{x}{:}{\sf exp}.\,
     \exists \lf{\it nv}{:}{\sf natvar}\,\lf{x}\,\lf{n}.\, \one \}
\end{align*}
Call this regular world signature $\Sigma_{\it RW}$. It is an extremely
simple example of a generative signature -- there are no
terminals and no nonterminals -- so the restriction operator has
no effect. The world described by $({\sf blockexp} \mid {\sf blocknatvar})$
is identical to the set of LF contexts $\Psi$ such that
$(\cdot; \cdot) \leadsto_{\Sigma_{RW}} (\Psi; \cdot)$.

\subsection{Regular worlds in substructural specifications}

From the structure of translated LF regular worlds, it is hopefully
apparent that by replacing the proposition $\one$ in the heads of the
generative ${\sf block*}$ rules with more interesting positive
\sls~propositions,  we can extend the language of regular
worlds to allow the introduction of ordered, mobile, and persistent
\sls~propositions as well. For instance, the rule
${\sf blockitem} : 
\forall \lf{n}. \, \{ {\sf item}\,\lf{n} \}$,
where ${\sf item}$ is a mobile predicate,
describes the world of contexts that take the form
$\left(\cdot; ~ x_1{:}\iseph{\susp{{\sf item}\,\lf{n_1}}}, ~
         \ldots, ~
         x_k{:}\iseph{\susp{{\sf item}\,\lf{n_k}}}\right)$
for some numbers $\lf{n_1}\ldots\lf{n_k}$. 
The world described by this generative signature is an invariant of a
rule like
\begin{align*}
  {\sf merge} : 
  \forall \lf{n}.\,\forall\lf{m}.\,\forall\lf{p}.\,
   {\sf item}\,\lf{n} \fuse
   {\sf item}\,\lf{m} \fuse
   {!}({\sf plus}\,\lf{n}\,\lf{m}\,\lf{p}) 
    \lefti \{ {\sf item}\,\lf{p} \}
\end{align*}
that combines two items,
where ${\sf plus}$ is  negative predicate defined with a deductive
specification as in
Figure~\ref{fig:plus}.  

Such substructural generalizations of regular worlds are sufficient
for the encoding of stores in Linear LF \cite{cervesato02linear} and
stacks in Ordered LF \cite{polakow01ordered}. They also suffice
to describe well-formedness invariants in Felty and Momigliano's
sequential specifications \cite{felty12hybrid}. However, regular
worlds are insufficient for the
invariants discussed in the remainder of this chapter.

\subsection{Generative versus consumptive signatures}

The example of regular worlds helps explain why generative signatures are
{\it generative}. A earlier version of the results in this chapter and
the next used {\it consumptive} signatures
\cite{simmons10type}. Consumptive signatures are generative
signatures with the arrows turned around: we consume well-formed
contexts using consumptive rules like $\forall\lf{e}.\,{\sf
  eval}\,\lf{e} \lefti \{ {\sf gen} \}$ and $\forall \lf{f}.\,{\sf
  gen} \fuse {\sf cont}\,\lf{f} \lefti \{ {\sf gen} \}$ instead of
creating them with generative 
rules like $\forall\lf{e}.\,{\sf gen}\lefti \{ {\sf eval}\,\lf{e} \}$
and $\forall \lf{f}.\,{\sf gen} \lefti \{ {\sf gen} \fuse {\sf
  cont}\,\lf{f} \}$. 

One arguable advantage of consumptive signatures is that it lets us
work with complete derivations, rather than traces. That is, using a
consumptive signature, we can talk about the set of process states
$(\Psi; \Delta)$ where $\Psi; \Delta \vdash \islax{{\sf gen}}$ rather
than the set of process states where $(\cdot; x{:}\istrue{\susp{\sf
    gen}}) \leadsto^* (\Psi; \Delta)$.\footnote{As long as $\Psi$ and
  $\Delta$ contain only nonterminals -- using consumptive signatures
  doesn't obviate the need for the restriction operation
  $\restrictsig{(\Psi; \Delta)}{}$ or some equivalent restriction
  operation.} For purely context-free-grammar-like invariants, such as
the PDA invariant from Section~\ref{sec:framework-reggenworld} and the
SSOS invariant from Section~\ref{sec:nat-ssos-adequacy}, generative
and consumptive signatures are effectively equivalent.

However, for generative signatures describing regular worlds, there is
no notion of turning the arrows around to get an appropriate
consumptive signature. In particular, say
we want to treat 
\begin{align*}
\Psi_{\it good} & = 
 \left(
 \lf{x_1}{:}{\sf exp}, \lf{{\it nv}_1}{:}{\sf natvar}\,\lf{x_1}\,\lf{n_1},
 \lf{x_2}{:}{\sf exp}, \lf{{\it nv}_2}{:}{\sf natvar}\,\lf{x_2}\,\lf{n_2}
 \right)
\intertext{as a well-formed LF context but {\it not} treat }
\Psi_{\it bad} & = 
 \left(
 \lf{x}{:}{\sf exp}, \lf{{\it nv}_1}{:}{\sf natvar}\,\lf{x}\,\lf{n_1},
  \lf{{\it nv}_2}{:}{\sf natvar}\,\lf{x}\,\lf{n_2}
 \right)
\end{align*} as well-formed. It is trivial to use Twelf's regular
worlds or generative signatures to impose this condition, but it does
not seem possible to use consumptive signatures for this
purpose. There exists a substitution \mbox{$\lf{(x/\!\!/x_1, {\it
      nv}_1/\!\!/{\it nv}_1, x/\!\!/x_2, {\it nv}_2/\!\!/{\it
      nv}_2)}$} from $\Psi_{\it good}$ to $\Psi_{\it bad}$; therefore,
by variable substitution (Theorem~\ref{thm:varsubst}), if there exists
a derivation of $\Psi_{\it good} \vdash_\Sigma \islax{\sf gen}$ there
also exists a derivation of $\Psi_{\it bad} \vdash_\Sigma \islax{\sf
  gen}$. 
%This means that consumptive signatures as presented in
%\cite{simmons10type} cannot describe the (regular) world that includes
%$\Psi_{\it good}$ and rejects $\Psi_{\it bad}$. 
This is related to the
issues of variable and pointer equality discussed in
Section~\ref{sec:mutable-storage}.

The generative signatures used to describe state in
Section~\ref{sec:gen-state} and destination-passing style in
Section~\ref{sec:gen-destinations} rely critically on the
regular-worlds-like uniqueness invariants that are provided by
generative signatures and not by consumptive signatures. (The progress
and preservation proofs in \cite{simmons10type} consider neither
mutable state nor destination-passing style.)

\section{Invariants of ordered specifications}
\label{sec:gen-order}

We already introduced generative invariants for ordered
abstract machine SSOS specifications in
Section~\ref{sec:nat-ssos-adequacy}. In this section, we will extend 
that generative invariant to ordered abstract machines
with parallel evaluation and recoverable failure.


\begin{figure}[t]
\fvset{fontsize=\small,boxwidth=229pt}
\VerbatimInput{sls/gen-order-prog.sls}
\caption{Ordered abstract machine with parallel evaluation and failure}
\label{fig:gen-order-prog}
\end{figure}

In Figure~\ref{fig:gen-order-prog} we describe a flat ordered abstract
machine with parallel features (parallel evaluation of the function
and argument in an application, as discussed in
Section~\ref{sec:trans-par} and Figure~\ref{fig:cbv-ev-ssos-par}) and
recoverable failure (as presented in Section~\ref{sec:failure} and
Figure~\ref{fig:ssos-fail}). To make sure there is still an
interesting sequential feature, we also introduce a let-expression
$\interp{{\sf let}\,x = e\,{\sf in}\,e'} = \lf{{\sf
    let}\,\interp{e}\,\lambda x.\interp{e'}}$. The particular features
are less important than the general setup, which effectively
represents all the specifications from
Chapter~\ref{chapter-absmachine} that used only ordered atomic
propositions.


Our goal is to describe a generative signature that represents the
well-formed process states of the specification in
Figure~\ref{fig:gen-order-prog}. What determines whether a process
state is well formed? The intended adequacy theorem was our guide in
Section~\ref{sec:nat-ssos-adequacy}, and the intended progress theorem
will guide our hand in Section~\ref{sec:gen-ordertp}. In this case,
our goal is that every well-formed state should be {\it
  reachable}. That is, if $(x{:}\istrue{\susp{\sf gen}})
\leadsto^* \Delta$ under the generative signature and if $\Delta$ contains no
instances of ${\sf gen}$, then there should be an expression $\obj{e}$
such that $(x{:}\istrue{\susp{{\sf eval}\,\interp{e}}}) \leadsto^*
\Delta$ under the signature from
Figure~\ref{fig:gen-order-prog}. (Because ${\sf gen}$ is the only
nonterminal, we can express that $\Delta$ contains no instances of
${\sf gen}$ with the restriction operator, writing
$\restrictsig{\Delta}{}$.) We will discuss the proof of this property
in Section~\ref{sec:well-formed-reachable}.

The analogues of the unary grammar productions, associated with the
terminals ${\sf eval}\,\lf{e}$, ${\sf retn}\,\lf{v}$, and ${\sf
  error}$, are straightforward:
%
\smallskip
\fvset{fontsize=\small,boxwidth=229pt}
\VerbatimInput{sls/gen-order-core.sls} 
\smallskip 
%
As in Section~\ref{sec:nat-ssos-adequacy}, we use a
deductively-defined judgment ${\sf value}\,\lf{v}$ to stipulate that
we only return values. The process state $(y{:}\istrue{\susp{{\sf
      retn}\,\interp{e_1\,e_2}}})$ is not well formed: the
application expression $\obj{e_1\,e_2}$ is not a value, and
there is no $\obj{e}$ such that $(x{:}\istrue{\susp{{\sf
      eval}\,\interp{e}}}) \leadsto^* (y{:}\istrue{\susp{{\sf
      retn}\,\interp{e_1\,e_2}}})$ under the signature from
Figure~\ref{fig:gen-order-prog}.

There is a potential catch when we consider the rules for sequential
continuations ${\sf cont}\,\lf{f}$ and parallel continuations ${\sf
  cont2}\,\lf{f}$. We expect a sequential continuation frame to be
preceded by a single well-formed computation, and for a parallel
continuation frame to be preceded by {\it two} well-formed
computations, suggesting these rules:
%
\smallskip
\fvset{fontsize=\small,boxwidth=229pt}
\VerbatimInput{sls/gen-order-bad.sls} 
\smallskip 
%
Even though ${\sf gen/cont}$ is exactly the rule for sequential
continuations in Section~\ref{sec:nat-ssos-adequacy}, this approach
conflicts with our guiding principle of reachability.
Both parallel and sequential continuations are indexed by
frames, but the parallel frame ${\sf app1}$ cannot appear in a
sequential continuation, nor can the sequential frame $\lf{({\sf
  let1}\,\lf{\lambda x.e\,x})}$ appear in a parallel frame. 

\begin{figure}[tp]
\fvset{fontsize=\small,boxwidth=229pt}
\VerbatimInput{sls/gen-order.sls} 
\caption{Generative invariant: well-formed process states}
\label{fig:gen-order} 
\end{figure}

This is fundamentally no more complicated than the restrictions we
placed on the ${\sf retn}\,\lf{v}$ terminal. All expressions (LF
variables of type ${\sf exp}$) can appear in ${\sf exp}\,\lf{e}$
propositions (and in ${\sf handle}\,\lf{e}$ propositions), but only
some can appear in ${\sf retn}\,\lf{v}$ frames. We describe that
subset of frames with the negative atomic proposition ${\sf
  value}\,\lf{v}$, which is deductively defined. Similarly, only some
frames can appear in ${\sf cont}\,\lf{f}$ terminals, and only some
frames can appear in ${\sf cont2}\,\lf{f}$ terminals. The former subset
can be expressed by a negative atomic proposition ${\sf okf}\,\lf{f}$,
and the latter by a negative atomic proposition ${\sf okf2}\,\lf{f}$.
Both of these are deductively defined.  The full specification of this
generative invariant is shown in Figure~\ref{fig:gen-order}; we will
refer to this generative signature as $\siggenorder$.

\subsection{Inversion}\label{sec:inversion-genorder}

Traditional inversion lemmas are a critical part of preservation
properties for small-step operational semantics specifications. In 
traditional preservation theorems, we are often start with a derivation of
$\obj{e_1\,e_2 \mapsto e_1'\,e_2}$ and another derivation of
$\obj{\cdot \vdash e_1\,e_2 : {\it tp}}$. An inversion lemma then proceeds
by case analysis on the structure of the derivation 
$\obj{\cdot \vdash e : {\it tp}}$, and allows us to conclude that
$\obj{\cdot \vdash e_1 : {\it tp'} \Rightarrow {\it tp}}$
and that $\obj{\cdot \vdash e_2 : {\it tp'} }$ for some object-level
type $\obj{\it tp'}$.  In other words, an inversion lemma allows us to 
take knowledge about a term's structure and obtain information about 
the typing derivation's structure. 

Inversion on a generative signature is intuitively similar: we take
information about the structure of a process state and use it to learn
about the generative trace that formed that process state. Concurrent
equality (Section~\ref{sec:framework-concurrenteq}) is critical.
None the parts of the lemma below would hold if we did not
equate traces such as such as
\[
\trstep{x_1, x_2, x_3}{{\sf gen/cont2}\,\lf{f}\,(x' \fuse {\sf okf2/app1})}; ~
\trstep{y_1}{{\sf gen/eval}\,\lf{e_1}\,x_1}; ~
\trstep{y_2}{{\sf gen/eval}\,\lf{e_2}\,x_2}
\]and\[
\trstep{x_1, x_2, x_3}{{\sf gen/cont2}\,\lf{f}\,(x' \fuse {\sf okf2/app1})}; ~
\trstep{y_2}{{\sf gen/eval}\,\lf{e_2}\,x_2}; ~
\trstep{y_1}{{\sf gen/eval}\,\lf{e_1}\,x_1}
\]
by concurrent equality.

\bigskip
\begin{lemma}[Inversion -- Figure~\ref{fig:gen-order}]~
\begin{enumerate}
\item If 
   $T :: (x_0{:}\istrue{\susp{\sf gen}}) \leadsto^*_{\siggenorder}
         \tackon{\Theta}{y{:}\istrue{\susp{{\sf eval}\,\lf{e}}}}$,
\\ then 
   $T = \left(T'; \trstep{y}{{\sf gen/eval}\,\lf{e}\,x'}\right)$.
\medskip
\item If 
   $T :: (x_0{:}\istrue{\susp{\sf gen}}) \leadsto^*_{\siggenorder}
         \tackon{\Theta}{y{:}\istrue{\susp{{\sf retn}\,\lf{v}}}}$,
\\ then 
   $T = \left(T'; \trstep{y}{{\sf gen/retn}\,\lf{v}\,(\tfuser{x'}{\tbangr{N}})}\right)$,
\\ where 
   $\cdot \vdash N : \isconc{{{{\sf value}\,\lf{v}}}}$.\footnote{In this
chapter, the signature associated with every deductive derivation
($\siggenorder$ in this case)
is clear from the context and so we write 
$\cdot \vdash N : \isconc{{{{\sf value}\,\lf{v}}}}$ instead of 
$\cdot \vdash N : \isconc{{{{\sf value}\,\lf{v}}}}$.}
\medskip
\item If 
   $T :: (x_0{:}\istrue{\susp{\sf gen}}) \leadsto^*_{\siggenorder}
         \tackon{\Theta}{y_1{:}\istrue{\susp{{\sf gen}}}, ~
                         y_2{:}\istrue{\susp{{\sf cont}\,\lf{f}}}}$,
\\ then 
   $T = \left(T'; \trstep{y_1,y_2}{{\sf gen/cont}\,\lf{f}\,(\tfuser{x'}{\tbangr{N}})}\right)$,
\\ where 
   $\cdot \vdash N : \isconc{{{{\sf okf}\,\lf{f}}}}$.
\medskip
\item If
   $T :: (x_0{:}\istrue{\susp{\sf gen}}) \leadsto^*_{\siggenorder}
         \tackon{\Theta}{y_1{:}\istrue{\susp{{\sf gen}}}, ~
                         y_2{:}\istrue{\susp{{\sf gen}}}, ~
                         y_3{:}\istrue{\susp{{\sf cont2}\,\lf{f}}}}$,
\\ then 
   $T = \left(T'; \trstep{y_1,y_2,y_3}{{\sf gen/cont2}\,\lf{f}\,(\tfuser{x'}{\tbangr{N}})}\right)$,
\\ where 
   $\cdot \vdash N : \isconc{{{{\sf okf2}\,\lf{f}}}}$.
\medskip
\item If 
   $T :: (x_0{:}\istrue{\susp{\sf gen}}) \leadsto^*_{\siggenorder}
         \tackon{\Theta}{y{:}\istrue{\susp{{\sf error}}}}$,
\\ then 
   $T = \left(T'; \trstep{y}{{\sf gen/error}\,x'}\right)$.
\medskip
\item If 
   $T :: (x_0{:}\istrue{\susp{\sf gen}}) \leadsto^*_{\siggenorder}
         \tackon{\Theta}{y_1{:}\istrue{\susp{{\sf gen}}}, ~
                         y_2{:}\istrue{\susp{{\sf handle}\,\lf{e}}}}$,
\\ then 
   $T = \left(T'; \trstep{y_1,y_2}{{\sf gen/handle}\,\lf{e}\,(\tfuser{x'}{\tbangr{N}})}\right)$.
\medskip
\end{enumerate}
In each instance above, 
$T' :: (x_0{:}\istrue{\susp{\sf gen}}) \leadsto^*_{\siggenorder}
          \tackon{\Theta}{x'{:}\istrue{\susp{{\sf gen}}}}$,
where the variables $x_0$ and $x'$ may or may not
be the same. (They are the same iff $T' = \emptytrace$.)
\end{lemma}

\begin{proof}
  Each part follows by induction and case analysis on the last steps
  of $T$.  In each case, we know that the trace cannot be empty,
  because the variable bindings $y{:}\istrue{\susp{{\sf
        eval}\,\lf{e}}}$, $y{:}\istrue{\susp{{\sf retn}\,\lf{v}}}$,
  $y_2{:}\istrue{\susp{{\sf cont}\,\lf{f}}}$,
  $y_3{:}\istrue{\susp{{\sf cont2}\,f}}$, $y{:}\istrue{\susp{{\sf
        error}}}$, and $y_2{:}\istrue{\susp{{\sf handle}\,\lf{e}}}$,
  respectively, appear in the final process state but not the initial
  process state. Therefore, $T =
  T''; S$ for some $T''$ and $S$.  Let ${\it Var}$ be the set of relevant
  variables -- $\{y\}$ in parts 1, 2, and 5, $\{y_1, y_2\}$ in parts 3
  and 6, and $\{y_1,y_2,y_3\}$ in part 4. 

  One possibility is that $\emptyset = S^{\bullet} \cap {\it Var}$. If so, it
  is always the case that $\emptyset = {^{\bullet}S} \cap {\it Var}$ as well,
  because ${\it Var}$ contains no persistent atomic propositions or LF
  variables. By the induction hypothesis we then get that $T'' = T''';
  S'$, where $S' = \trstep{y}{{\sf gen/eval}\,\lf{e}\,x'}$ in part 1,
  $S' = \trstep{y}{{\sf gen/retn}\,\lf{v}\,(\tfuser{x'}{\tbangr{N}})}$
  in part 2, and so on.  In each of the six parts, of course
  $S'{^\bullet} = {\it Var}$, so $\emptyset = S'{^\bullet} \cap {^\bullet}S$
  and $\left(T'''; S'; S\right) = \left(T'''; S; S'\right)$, and so we can
  conclude letting $T' = \left(T''; S\right)$.

  If $S^{\bullet} \cap {\it Var}$ is nonempty, we must show by case
  analysis that $S^{\bullet} = {\it Var}$ and that furthermore $S$ is
  the step we were looking for. This is easy in parts 1, 2, and 5
  where ${\it Var}$ is a singleton set: there is only one rule that
  can produce an atomic proposition of type ${\sf gen}\,\lf{e}$, ${\sf
    retn}\,\lf{v}$, or ${\sf error}$, respectively.  In part 3, we
  observe that, if the variable bindings $y_1{:}\istrue{\susp{{\sf
        gen}}}$ and $y_2{:}\istrue{\susp{{\sf cont}\,\lf{f}}}$ appear
  in order in the substructural context, there is no step in the
  signature $\siggenorder$ that has $y_1$ among its output variables
  that does not also have $y_2$ among its output variables, and vice
  versa. The rule ${\sf gen/cont2}$ cannot be used to generate $y_1$,
  for instance, because it would have to also place either another
  ${\sf gen}$ proposition or a ${\sf cont2}\,{\lf f}$ proposition to
  the right of $y_1$, and we know that a ${\sf cont}\,{\lf f}$
  proposition actually appears in this position. Parts 4 and 6 work
  by similar reasoning.
\end{proof}

The critical step of inversion can be intuitively connected with the
idea that the grammar described by a generative signature is {\it
  unambiguous}. This will not hold in general. If there was
a rule ${\sf gen/redundant} : {\sf gen} \lefti \{ {\sf gen} \}$ in
$\Sigma_{\it Gen\ref{fig:gen-order}}$, for instance, then the final
step $S$ could be
$\trstep{y_1}{{\sf gen/redundant}\,y'}$, and this would invalidate our
inversion proof
for parts 3, 4, and 6, as $S^{\bullet}$ would be $\{y_1\}$, a strict
subset of the set $V$. 
% If we added a rule ${\sf gen/contback} :
% \forall{\lf f}.\,{\sf gen} \fuse {!}{\sf okf}\,\lf{f} \lefti \{ {\sf
%   cont}\,\lf{f} \fuse {\sf gen} \}$, where the conclusions are in the
% wrong order, then a trace $(x{:}\istrue{\susp{\sf gen}})
% \leadsto^*_{\siggenorder} \tackon{\Theta}{y{:}\istrue{\susp{{\sf
%         gen}}}}$
Conversely, if we tried to prove an inversion
property about traces $(x{:}\istrue{\susp{\sf gen}})
\leadsto^*_{\siggenorder} \tackon{\Theta}{y{:}\istrue{\susp{{\sf
        gen}}}}$, this property would again fail: $V = \{ y
\}$, and in the case where the last step $S$ is driven by one of the
rules ${\sf gen/cont}$, ${\sf gen/cont2}$, or ${\sf gen/handle}$,
$S^{\bullet}$ will be a strict superset of $V$.

The proof above could also be stated in form of case analysis on the
form of the last step and induction. The reason for {\it not} doing
that is more an issue of proof engineering: such a proof would require
enumerating 7 cases for each of the 6 inversion lemmas, leading to
proof whose size is in $O(n^2)$ where $n$ is the number of rules
in the generative signature. In this chapter, we will emphasize the principles
by which we can use to reason {\it concisely} about specifications.

\subsection{Preservation}

\begin{theorem}[$\siggenorder$ is a generative invariant]\label{thm:siggenorder}
If $T_1 :: (x_0{:}\istrue{\susp{\sf gen}}) \leadsto^*_{\siggenorder} 
   \Delta$ and $S :: \restrictsig{\Delta}{} \leadsto \Delta'$
under the signature from Figure~\ref{fig:gen-order-prog}, then
$T_2 :: (x_0{:}\istrue{\susp{\sf gen}}) \leadsto^*_{\siggenorder} 
   \Delta'$
\end{theorem}

\begin{proof} As in the proofs of Theorem~\ref{thm:pda-preservation}
  and Theorem~\ref{thm:adequate-pres}, we enumerate the synthetic
  transitions possible under the signature in
  Figure~\ref{fig:gen-order-prog}, perform inversion on the structure
  of $T_1$, and then use the results of inversion to construct
  $T_2$. We give three illustrative cases corresponding to the
  fragment dealing with functions and parallel application.

\begin{description}
%%%%%%%%
%%%%%%%%
%%%%%%%%
\item 
  [Case $\trstep{y}{{\sf ev/lam}\,\lf{(\lambda x. e)}\,x}
   ::
   \tackon{\Theta}
     {x{:}\istrue{\susp{{\sf eval}\,\lf{({\sf lam}\,\lambda x.e)}}}}
   \leadsto
   \tackon{\Theta}
     {y{:}\istrue{\susp{{\sf retn}\,\lf{({\sf lam}\,\lambda x.e)}}}}$]~

\medskip
Applying inversion (Part 1) to $T_1$, we have 

\begin{tabbing}
$T_1 = ~$ \= \qquad \= $(x_0{:}\istrue{\susp{\sf gen}})$
\\
\>$T'$
\\
\>\>$\tackon{\Theta}{x'{:}\istrue{\susp{\sf gen}}}$
\\
\>$\trstep{x}{{\sf gen/eval}\,\lf{({\sf lam}\,\lambda x.e)}\,x'}$
\\
\>\>$\tackon{\Theta}
       {x{:}\istrue{\susp{{\sf eval}\,\lf{({\sf lam}\,\lambda x.e)}}}}$
\end{tabbing}

We can use $T'$ to construct $T_2$ as follows:

\begin{tabbing}
$T_2 = ~$ \= \qquad \= $(x_0{:}\istrue{\susp{\sf gen}})$
\\
\>$T'$
\\
\>\>$\tackon{\Theta}{x'{:}\istrue{\susp{\sf gen}}}$
\\
\>$\trstep{y}{{\sf gen/retn}\,\lf{({\sf lam}\,\lambda x.e)}\,(\tfuser{x'}{\tbangr{({\sf value/lam}\,\lf{(\lambda x.e)})}})}$
\\
\>\>$\tackon{\Theta}
     {y{:}\istrue{\susp{{\sf retn}\,\lf{({\sf lam}\,\lambda x.e)}}}}$
\end{tabbing}

%%%%%%%%
%%%%%%%%
%%%%%%%%
\item 
  [Case $\trstep{y_1, y_2, y_3}{{\sf ev/app}\,\lf{e_1}\,\lf{e_2}\,x}$]~

\qquad
  $::
   \tackon{\Theta}
     {x{:}\istrue{\susp{{\sf eval}\,\lf{({\sf app}\,e_1\,e_2)}}}}$

\qquad\qquad
  $\leadsto
   \tackon{\Theta}
     {y_1{:}\istrue{\susp{{\sf eval}\,\lf{e_1}}}, ~
      y_2{:}\istrue{\susp{{\sf eval}\,\lf{e_2}}}, ~
      y_3{:}\istrue{\susp{{\sf cont2}\,\lf{\sf app1}}}}$
\medskip

Applying inversion (Part 1) to $T_1$, we have 

\begin{tabbing}
$T_1 = ~$ \= \qquad \= $(x_0{:}\istrue{\susp{\sf gen}})$
\\
\>$T'$
\\
\>\>$\tackon{\Theta}{x'{:}\istrue{\susp{\sf gen}}}$
\\
\>$\trstep{x}{{\sf gen/eval}\,\lf{({\sf app}\,e_1\,e_2)}\,x'}$
\\
\>\>$\tackon{\Theta}{x{:}\istrue{\susp{{\sf eval}\,\lf{({\sf app}\,e_1\,e_2)}}}}$
\end{tabbing}

We can use $T'$ to construct $T_2$ as follows:

\begin{tabbing}
$T_2 = ~$ \= \qquad \= $(x_0{:}\istrue{\susp{\sf gen}})$
\\
\>$T'$
\\
\>\>$\tackon{\Theta}{x'{:}\istrue{\susp{\sf gen}}}$
\\
\>$\trstep{y_1',y_2',y}{{\sf gen/cont2}\,\lf{{\sf app1}}\,(\tfuser{x'}{\tbangr{{\sf okf2/app1}}})}$
\\
\>$\trstep{y_1}{{\sf gen/eval}\,\lf{e_1}\,y_1'}$
\\
\>$\trstep{y_2}{{\sf gen/eval}\,\lf{e_2}\,y_2'}$
\\
\>\>$\tackon{\Theta}{y_1{:}\istrue{\susp{{\sf eval}\,\lf{e_1}}}, ~
      y_2{:}\istrue{\susp{{\sf eval}\,\lf{e_2}}}, ~
      y_3{:}\istrue{\susp{{\sf cont2}\,\lf{\sf app1}}}}$
\end{tabbing}

%%%%%%%%
%%%%%%%%
%%%%%%%%
\item 
  [Case $\trstep{y}{{\sf ev/app1}\,\lf{(\lambda x.\,e)}\,\lf{v_2}\,(\tfuser{x_1}{\tfuser{x_2}{x_3}})}$]~

\qquad
  $::
   \tackon{\Theta}{x_1{:}\istrue{\susp{{\sf retn}\,\lf{({\sf lam}\,\lambda x.\,e)}}}, ~
                   x_2{:}\istrue{\susp{{\sf retn}\,\lf{v_2}}}, ~
                   x_3{:}\istrue{\susp{{\sf cont2}\,\lf{\sf app1}}}}$

\qquad\qquad
  $\leadsto
   \tackon{\Theta}{y{:}\istrue{\susp{{\sf eval}\,\lf{([v_2/x]e)}}}}$

\medskip
Applying inversion (Part 2, twice, and then Part 4) to $T_1$, we have

\begin{tabbing}
$T_1 = ~$ \= \qquad \= $(x_0{:}\istrue{\susp{\sf gen}})$
\\
\>$T'$
\\
\>\>$\tackon{\Theta}{x'{:}\istrue{\susp{\sf gen}}}$
\\
\>$\trstep{x_1', x_2', x_3}
     {{\sf gen/cont2}\,\lf{\sf app1}\,(\tfuser{x'}{\tbangr{N}})}$
\\
\>$\trstep{x_1}
     {{\sf gen/retn}\,\lf{({\sf lam}\,\lambda x.e)}\,
        (\tfuser{x_1'}{\tbangr{N_1}})}$ 
\\
\>$\trstep{x_2}
     {{\sf gen/retn}\,\lf{v_2}\,(\tfuser{x_2'}{\tbangr{N_2}})}$
\\
\>\>$\tackon{\Theta}{x_1{:}\istrue{\susp{{\sf retn}\,\lf{({\sf lam}\,\lambda x.\,e)}}}, ~
                   x_2{:}\istrue{\susp{{\sf retn}\,\lf{v_2}}}, ~
                   x_3{:}\istrue{\susp{{\sf cont2}\,\lf{\sf app1}}}}$
\end{tabbing}

We can use $T'$ to construct $T_2$ as follows:
\begin{tabbing}
$T_2 = ~$ \= \qquad \= $(x_0{:}\istrue{\susp{\sf gen}})$
\\
\>$T'$
\\
\>\>$\tackon{\Theta}{x'{:}\istrue{\susp{\sf gen}}}$
\\
\>$\trstep{y}
     {{\sf gen/eval}\,\lf{([v_2/x]e)}\,x'}$
\\
\>\>$\tackon{\Theta}{y{:}\istrue{\susp{{\sf eval}\,\lf{([v_2/x]e)}}}}$
\end{tabbing}


\end{description}

\noindent
The other cases, corresponding to the rules ${\sf ev/unit}$, ${\sf
  ev/fail}$, ${\sf ev/catch}$, ${\sf ev/catcha}$, ${\sf ev/catchb}$,
${\sf ev/error}$, ${\sf ev/errerr}$, ${\sf ev/errret}$, and ${\sf
  ev/reterr}$ all follow the same lines: enumeration, inversion, and
reconstruction. 
\end{proof}

Note that, in the case corresponding to the rule ${\sf ev/app1}$, we
obtained but did not use three terms $\cdot \vdash N : \isconc{{{\sf
    okf2}\,\lf{\sf app1}}}$, $\cdot \vdash N_1 : \isconc{{\sf value}\,\lf{({\sf
    lam}\,\lambda x.e)}}$, and $\cdot \vdash N_2 : \isconc{{{\sf
    value}\,\lf{v_2}}}$. By traditional inversion on the structure of a
deductive derivation, we know that $N = {\sf okf2/app1}$ and $N_1 =
{\sf value/lam}\,\lf{(\lambda x.e)}$, but that was not necessary to prove
in
this theorem.


\section{From well-formed to well-typed states}
\label{sec:gen-ordertp}

In order to describe those expressions whose evaluations never get
stuck, we introduce object level types $\obj{\it tp}$ and define a
typing judgment $\obj{\Gamma \vdash e : {\it tp}}$.  We encode
object-level types as LF terms classified by the LF type ${\sf
  typ}$. The unit type $\interp{\one} = \lf{\sf unittp}$ classifies units
$\interp{\langle\rangle} = \lf{\sf unit}$, and the function type $\interp{{\it tp}_1
  \Rightarrow {\it tp}_2} = \lf{\sf arr}\,\interp{{\it
    tp}_1}\,\interp{{\it tp}_2}$ classifies lambda expressions.

\begin{figure}[t]
\fvset{fontsize=\small,boxwidth=229pt}
\VerbatimInput{sls/gen-order-of.sls} 
\label{fig:gen-order-of} 
\fvset{fontsize=\small,boxwidth=229pt}
\VerbatimInput{sls/gen-ordertp.sls} 
\caption{Generative invariant: well-typed process states}
\label{fig:gen-ordertp} 
\end{figure}


In a syntax-directed type system, each syntactic construct is associated
with a different typing rule.
These are the typing rules necessary for describing the language
constructs in Figure~\ref{fig:gen-order-prog}:
\[
\infer
{\obj{\Gamma \vdash \langle\rangle : \one} \mathstrut}
{\mathstrut}
\qquad
\infer
{\obj{\Gamma \vdash \lambda x.e : {\it tp' \Rightarrow tp}} \mathstrut}
{\obj{\Gamma, x{:}{\it tp'} \vdash e : {\it tp}} \mathstrut}
\qquad
\infer
{\obj{\Gamma \vdash e_1\,e_2 : {\it tp}} \mathstrut}
{\obj{\Gamma \vdash e_1 : {\it tp' \Rightarrow tp}}
 &
 \obj{\Gamma \vdash e_2 : {\it tp'}}
 \mathstrut}
\]
\[
\infer
{\obj{\Gamma \vdash {\sf fail} : {\it tp}} \mathstrut}
{}
\qquad
\infer
{\obj{\Gamma \vdash {\sf try}\,e_1\,{\sf ow}\,e_2 : {\it tp}} \mathstrut}
{\obj{\Gamma \vdash e_1 : {\it tp}} 
 &
 \obj{\Gamma \vdash e_2 : {\it tp}}
 \mathstrut}
\]
We can adequately encode derivations of the judgment
$\obj{x_1{:}{\it tp}_1, \ldots, x_n{:}{\it tp}_n \vdash e : {\it tp}}$ as 
\sls~derivations $\lf{x_1}{:}{\sf exp}, \ldots, \lf{x_n}{:}{\sf exp}; 
y_1 : \ispers{({\sf of}\,\lf{x_1}\,\interp{{\it tp}_1})}, \ldots,
y_n : \ispers{({\sf of}\,\lf{x_1}\,\interp{{\it tp}_n})}
\vdash {\sf of}\,\interp{e}\,\interp{\it tp}$ under the signature
in Figure~\ref{fig:gen-order-of}.



This typing judgment allows us to describe well-formed initial states,
but it is not sufficient to describe intermediate states. To this end,
we describe typing rules for frames, refining the negative predicates
${\sf okf}\,\lf{f}$ and ${\sf okf2}\,\lf{f}$ from
Figure~\ref{fig:gen-order}. The \sls~proposition describing well-typed
sequential frames is $({\sf off}\,\lf{f}\,\interp{{\it
    tp}'}\,\interp{\it tp})$. This proposition expresses that the frame
$\lf{f}$ {\it expects} a returned result with type $\obj{\it tp'}$ and
{\it produces} a computation with type $\obj{{\it tp}}$.\footnote{The
  judgment we encode in \sls~as $({\sf off}\,\lf{f}\,\interp{{\it
      tp}'}\,\interp{\it tp})$ is written $\obj{f : {\it tp}'
    \Rightarrow {\it tp}}$ in \cite[Chapter 27]{harper12practical}.}
The parallel version is $({\sf off}\,\lf{f}\,\interp{{\it
    tp}_1}\,\interp{{\it tp}_2}\,\interp{\it tp})$, and expects two
sub-computation with types $\obj{{\it tp}_1}$ and $\obj{{\it tp}_2}$,
respectively, in order to produce a computation of type $\obj{{\it
    tp}}$. These judgments are given in Figure~\ref{fig:gen-ordertp}. 

The generative rules in Figure~\ref{fig:gen-ordertp} are our first use
of an {\it indexed} nonterminal, ${\sf gen}\,\interp{\it tp}$, which
generates computations that, upon successful return, will produce
values $\obj{v}$ such that $\obj{\cdot \vdash v : {\it tp}}$. 

\subsection{Inversion}

The structure of inversion lemmas is entirely unchanged
aside for accounting for type indices. We only state
two cases of the inversion lemma, the one corresponding to 
${\sf gen/eval}$ and the one corresponding to ${\sf gen/cont}$. 
These two cases suffice to set up the template that all other cases
follow. 

\bigskip
\begin{lemma}[Inversion -- Figure~\ref{fig:gen-order}, partial]~
\begin{enumerate}
\item If 
   $T :: (x_0{:}\istrue{\susp{{\sf gen}\,\lf{{\it tp}_0}}}) 
         \leadsto^*_{\siggenordertp}
         \tackon{\Theta}{y{:}\istrue{\susp{{\sf eval}\,\lf{e}}}}$,
\\ then 
   $T = \left(T'; \trstep{y}{{\sf gen/eval}\,\lf{{\it tp}}\,\lf{e}\,(\tfuser{x'}{\tbangr{N}})}\right)$,
\\ where $\cdot \vdash N : \isconc{{\sf of}\,\lf{e}\,\lf{{\it tp}}}$
\medskip
\item If 
   $T :: (x_0{:}\istrue{\susp{{\sf gen}\,\lf{{\it tp}_0}}})
         \leadsto^*_{\siggenordertp}
         \tackon{\Theta}{y_1{:}\istrue{\susp{{\sf gen}\,\lf{{\it tp}'}}}, ~
                         y_2{:}\istrue{\susp{{\sf cont}\,\lf{f}}}}$,
\\ then 
   $T = \left(T'; \trstep{y_1,y_2}{{\sf gen/cont}\,\lf{{\it tp}}\,\lf{f}\,\lf{{\it tp}'}\,(\tfuser{x'}{\tbangr{N}})}\right)$,
\\ where 
   $\cdot \vdash N : \isconc{{{{\sf off}\,\lf{f}\,\lf{{\it tp}'}\,\lf{{\it tp}}}}}$.
\medskip
\end{enumerate}
In each instance above, 
$T' :: (x_0{:}\istrue{\susp{{\sf gen}\,\lf{{\it tp}_0}}}) \leadsto^*_{\siggenordertp}
          \tackon{\Theta}{x'{:}\istrue{\susp{{\sf gen}\,\lf{{\it tp}}}}}$,
where the variables $x_0$ and $x'$ may or may not
be the same. (They are the same iff $T' = \emptytrace$, and if they
are the same that implies $\lf{{\it tp}_0} = \lf{{\it tp}}$.)
\end{lemma}
\bigskip



\subsection{Preservation}


Theorem~\ref{thm:siggenordertp} only differs from
Theorem~\ref{thm:siggenorder} because it mentions the type index.
Each object-level type $\lf{{\it tp}_0}$ describes a different world,
and evaluation under the rules in Figure~\ref{fig:gen-order-prog}
always stays within the same world.

\bigskip
\begin{theorem}[$\siggenordertp$ is a generative invariant]
\label{thm:siggenordertp}
If $T_1 :: (x_0{:}\istrue{\susp{{\sf gen}\,\lf{{\it tp}_0}}}) \leadsto^*_{\siggenordertp} 
   \Delta$ and $S :: \restrictsig{\Delta}{} \leadsto \Delta'$
under the signature from Figure~\ref{fig:gen-order-prog}, then
$T_2 :: (x_0{:}\istrue{\susp{{\sf gen}\,\lf{{\it tp}_0}}}) \leadsto^*_{\siggenordertp} 
   \Delta'$.
\end{theorem}

\bigskip
In the proof of Theorem~\ref{thm:siggenorder}, we observed that the
applicable inversion on the generative trace gave us derivations like
$\cdot \vdash N : \isconc{{{\sf okf2}\,\lf{\sf app1}}}$. We did not need these
side derivations to complete the proof, but we noted that they were
amenable to traditional inversion. Traditional inversion will be
critical in proving that the generative invariant described by
$\siggenordertp$ is preserved. Describing, proving, and mechanizing
traditional inversion lemmas on deductive derivation is a solved
problem; we merely point out when we are using a traditional inversion
property in the proof below.


\begin{proof} As always, the proof proceeds by enumeration, inversion,
  and reconstruction. We give two representative cases: 

\begin{description}

\item 
  [Case $\trstep{y}{{\sf ev/app1}\,\lf{(\lambda x.\,e)}\,\lf{v_2}\,(\tfuser{x_1}{\tfuser{x_2}{x_2}})}$]~

\qquad
  $::
   \tackon{\Theta}{x_1{:}\istrue{\susp{{\sf retn}\,\lf{({\sf lam}\,\lambda x.\,e)}}}, ~
                   x_2{:}\istrue{\susp{{\sf retn}\,\lf{v_2}}}, ~
                   x_3{:}\istrue{\susp{{\sf cont2}\,\lf{\sf app1}}}}$

\qquad\qquad
  $\leadsto
   \tackon{\Theta}{y{:}\istrue{\susp{{\sf eval}\,\lf{([v_2/x]e)}}}}$

\medskip
Applying inversion to $T_1$, we have

\begin{tabbing}
$T_1 = ~$ \= \qquad \= $(x_0{:}\istrue{\susp{{\sf gen}\,\lf{{\it tp}_0}}})$
\\
\>$T'$
\\
\>\>$\tackon{\Theta}{x'{:}\istrue{\susp{{\sf gen}\,\lf{{\it tp}}}}}$
\\
\>$\trstep{x_1', x_2', x_3}
     {{\sf gen/cont2}\,\lf{{\it tp}}\,\lf{\sf app1}\,
         \lf{{\it tp}''}\,\lf{{\it tp}'}\,
         (\tfuser{x'}{\tbangr{N}})}$
\\ %%
\>\>$\tackon{\Theta}
     {x_1'{:}\istrue{\susp{{\sf gen}\,\lf{{\it tp}''}}}, ~
      x_2'{:}\istrue{\susp{{\sf gen}\,\lf{{\it tp}'}}}, ~
      x_3{:}\istrue{\susp{{\sf cont2}\,\lf{\sf app1}}}}$
\\
\>$\trstep{x_1}
     {{\sf gen/retn}\,\lf{tp''}\,\lf{({\sf lam}\,\lambda x.e)}\,}
         (\tfuser{x_1'}{\tfuser{\tbangr{N_1}}{\tbangr{N_{v1}}}})$
%        (\tfuser{x_1'}{\tfuser{\tbangr{N_1}}{\tbangr{N_{v1}}}})}$ 
\\
\>$\trstep{x_2}
     {{\sf gen/retn}\,\lf{v_2}\,(\tfuser{x_2'}{\tfuser{\tbangr{N_2}}{\tbangr{N_{v2}}}})}$
\\
\>\>$\tackon{\Theta}{x_1{:}\istrue{\susp{{\sf retn}\,\lf{({\sf lam}\,\lambda x.\,e)}}}, ~
                   x_2{:}\istrue{\susp{{\sf retn}\,\lf{v_2}}}, ~
                   x_3{:}\istrue{\susp{{\sf cont2}\,\lf{\sf app1}}}}$
\end{tabbing}
where
\begin{itemize}
\item[$\bullet$] $\cdot \vdash N : \isconc{{\sf off2}\,\lf{\sf app1}\,\lf{{\it
      tp}''}\,\lf{{\it tp}'}\,\lf{{\it tp}}}$. \\ By traditional
  inversion we know $\lf{\it{tp}''} = \lf{{\sf arr}\,{\it tp}'\,{\it
      tp}}$ and $N = {\sf off2/app1}\,\lf{{\it tp}'}\,\lf{{\it tp}}$.
\item[$\bullet$] $\cdot \vdash N_1 : \isconc{{\sf of}\,\lf{({\sf lam}\,\lambda
    x.e)}\,\lf{{\sf arr}\,{\it tp}'\,{\it tp}}}$. \\
By traditional inversion we know 
    $\lf{x}{:}{\sf exp}; {\it dx} : \ispers{{\sf of}\,\lf{x}\,\lf{\it
      tp'}} \vdash N_1' : \isconc{{\sf of}\,\lf{e}\,\lf{\it
        tp}}$.
\item[$\bullet$] $\cdot \vdash N_2 : {\sf of}\,\lf{v_2}\,\lf{{\it tp}'}$.
\end{itemize}

With these derivations, 
variable substitution (Theorem~\ref{thm:varsubst}), and cut
admissibility (Theorem~\ref{thm:ord-cut}), we have a derivation of
$\cdot \vdash \rsubst{N_2}{\it dx}{(\lf{[v_2/x]}N_1')} : \isconc{{\sf
  of}\,\lf{([v_2/x]e)}\,\lf{\it tp}}$.\footnote{We know by
  subordination that $\lf{x}$ is not free in $\lf{\it tp}$, so
  $\lf{[v_2/x]{\it tp}} = \lf{\it tp}$.}  We can therefore use $T'$ to
construct $T_2$ as follows:
\begin{tabbing}
$T_2 = ~$ \= \qquad \= $(x_0{:}\istrue{\susp{{\sf gen}\,\lf{{\it tp}_0}}})$
\\
\>$T'$
\\
\>\>$\tackon{\Theta}{x'{:}\istrue{\susp{{\sf gen}\,\lf{\it tp}}}}$
\\
\>$\trstep{y}
     {{\sf gen/eval}\,\lf{tp}\,\lf{([v_2/x]e)}\,(\tfuser{x'}{\tbangr{(\rsubst{N_2}{\it
  dx}{(\lf{[v_2/x]}N_1')})}})}$
\\
\>\>$\tackon{\Theta}{y{:}\istrue{\susp{{\sf eval}\,\lf{([v_2/x]e)}}}}$
\end{tabbing}

%%%%
%%%%
%%%%
\item 
  [Case $\trstep{y_1,y_2}{{\sf ev/catch}\,\lf{(\lambda x.\,e)}\,\lf{v_2}\,x}$]~
 
\qquad $:: \tackon{\Theta}{x{:}\istrue{\susp{{\sf eval}\,\lf{({\sf catch}\,e_1\,e_2)}}}}
       \leadsto 
        \tackon{\Theta}{y_1{:}\istrue{\susp{{\sf eval}\,\lf{e_1}}}, ~
                   y_2{:}\istrue{\susp{{\sf handle}\,\lf{e_2}}}}$]~

\medskip
Applying inversion to $T_1$, we have

\begin{tabbing}
$T_1 = ~$ \= \qquad \= $(x_0{:}\istrue{\susp{{\sf gen}\,\lf{{\it tp}_0}}})$
\\
\>$T'$
\\
\>\>$\tackon{\Theta}{x'{:}\istrue{\susp{{\sf gen}\,\lf{{\it tp}}}}}$
\\
\>$\trstep{x}
     {{\sf gen/eval}\,\lf{\it tp}\,\lf{({\sf catch}\,e_1\,e_2)}\,(\tfuser{x'}{\tbangr{N}})}$
\\
\>\>$\tackon{\Theta}{x{:}\istrue{\susp{{\sf eval}\,\lf{({\sf catch}\,e_1\,e_2)}}}}$
\end{tabbing}
where $\cdot \vdash N : \sf of\,\lf{({\sf catch}\,e_1\,e_2)}\,\lf{\it tp}$. 

\medskip
By traditional inversion  on $N$ we 
know
 $\cdot \vdash N_1 : \isconc{{\sf of}\,\lf{e_1}\,\lf{\it tp}}$ and 
$\cdot \vdash N_2 :
\isconc{{\sf of}\,\lf{e_2}\,\lf{\it tp}}$.
We can therefore use $T'$ to construct $T_2$ as follows:

\begin{tabbing}
$T_1 = ~$ \= \qquad \= $(x_0{:}\istrue{\susp{{\sf gen}\,\lf{{\it tp}_0}}})$
\\
\>$T'$
\\
\>\>$\tackon{\Theta}{x'{:}\istrue{\susp{{\sf gen}\,\lf{{\it tp}}}}}$
\\
\>$\trstep{y_1',y_2}{{\sf gen/handle}\,\lf{\it tp}\,\lf{e_2}\,(\tfuser{x'}{\tbangr{N_2}})}$
\\
\>$\trstep{y_1}{{\sf gen/eval}\,\lf{\it tp}\,\lf{e_1}\,(\tfuser{y_1'}{\tbangr{N_1}})}$
\\
\>\>$\tackon{\Theta}{y_1{:}\istrue{\susp{{\sf eval}\,\lf{e_1}}}, ~
                   y_2{:}\istrue{\susp{{\sf handle}\,\lf{e_2}}}}$
\end{tabbing}


\end{description}

\noindent
The other cases follow the same pattern.
\end{proof}

Dealing with type preservation is, in an
sense, no more difficult than dealing with well-formedness
invariants. % Aside from the issues of adapting inversion theorems to
%generative traces, which were considered already in
%Section~\ref{sec:gen-order} and Theorem~\ref{thm:siggenorder},
Theorem~\ref{thm:siggenordertp} furthermore follows the contours of a standard
progress and preservation proof for an abstract machine like Harper's
$\mathcal K\{{\sf nat}{\rightharpoonup}\}$ \cite[Chapter
27]{harper12practical}.  Unlike the on-paper formalism used by Harper,
the addition of parallel evaluation in our specification does not
further complicate the statement or the proof of the preservation theorem.


\section{State}
\label{sec:gen-state}

Ambient state, encoded in mobile and persistent propositions, was used
to describe mutable storage in Section~\ref{sec:mutable-storage},
call-by-need evaluation in Section~\ref{sec:call-by-need}, and the
environment semantics in Section~\ref{sec:environment-semantics}. The
technology needed to describe generative invariants for each of these
specifications is similar. We will consider the extension of our
program from Figure~\ref{fig:gen-order-prog} with the semantics of
mutable storage from Figure~\ref{fig:ssos-mutable}. This specification
adds a mobile atomic proposition ${\sf cell}\,\lf{l}\,\lf{v}$, which
the generative signature will treat as a new terminal.

The intuition behind mutable cells is that they exist in 
tandem with locations $\lf{l}$ of LF type ${\sf mutable\_loc}$,
giving the non-control part of a process state the following
general form:
\[\left(\lf{l_1}{:}{\sf mutable\_loc},\ldots,\lf{l_n}{:}{\sf mutable\_loc};
  ~~ \iseph{\susp{{\sf cell}\,\lf{l_1}\,\lf{v_1}}}, 
  ~~ \ldots, 
  ~~ \iseph{\susp{{\sf cell}\,\lf{l_n}\,\lf{v_n}}}, 
  ~~ \ldots\right)\]
%
Naively, we might attempt to describe such process 
states with the block-like rule 
${\sf gen/cell/bad} : \forall \lf{v}.\, {!}{\sf
  value}\,\lf{v} \lefti \{ \exists \lf{l}. {\sf cell}\,\lf{l}\,\lf{v}
\}$. The problem with such a specification is that it makes 
cells unable to refer to themselves, a situation that can easily
happen in practice. A canonical example, using back-patching to
implement recursion, is traced out 
in Figure~\ref{fig:bigbackpatch}, which
describes a trace classified by:

\vspace{-10pt}

{\small\begin{align*}
&
\left(\cdot; ~~
 x_0{:}\istrue{\susp{{\sf eval}\,
  \interp{{\sf let}\,f = ({\sf ref}\,\lambda x. \langle\rangle) \,{\sf in}\,
          {\sf let}\,x = (f := \lambda x. ({!}f) x)\,{\sf in}\,e}}}
\right) ~~ \leadsto^*
\\
& 
\left(
\lf{l_1}{:}{\sf mutable\_loc}; ~~
y_2{:}\iseph{\susp{{\sf cell}\,\lf{l_1}\,
  \lf{({\sf lam}\,\lambda x.\,{\sf app}\,({\sf get}\,({\sf loc}\,l_1))\,x)}}},
 ~~
x_{17}{:}\istrue{\susp{{\sf eval}\,
  \lf{[({\sf loc}\,l_1)/f, {\sf unit}/x]\interp{e})}}}
\right)
\end{align*}}

\vspace{-10pt}

\begin{sidewaysfigure}[p]\label{fig:bigbackpatch}\small
\begin{align*}
&\qquad 
x_0{:}\susp{{\sf eval}\,
  \interp{{\sf let}\,f = ({\sf ref}\,\lambda x. \langle\rangle) \,{\sf in}\,
          {\sf let}\,x = (f := \lambda x. ({!}f) x)\,{\sf in}\,e}}
\\
& \trstep{x_1,x_2}{{\sf ev/let1}\,\ldots\,x_0}
\\
&\qquad
x_1{:}\susp{{\sf eval}\,\interp{{\sf ref}\,\lambda x. \langle\rangle}}, ~~
x_2{:}\susp{{\sf cont}\,
  \lf{({\sf let1}\,\lambda f.\,
  \interp{{\sf let}\, x = (f := \lambda x. ({!}f) x)\,{\sf in}\,e})}}
\\
& \trstep{x_3, x_4}{{\sf ev/ref}\,\ldots\,x_1}
\\
&\qquad
x_3{:}\susp{{\sf eval}\,\interp{\lambda x. \langle\rangle}}, ~~
x_4{:}\susp{{\sf cont}\,\lf{\sf ref1}}, ~~
x_2{:}\susp{{\sf cont}\,
  \lf{({\sf let1}\,\lambda f.\,
  \interp{{\sf let}\, x = (f := \lambda x. ({!}f) x)\,{\sf in}\,e})}}
\\
& \trstep{x_5}{{\sf ev/lam}\,\ldots\,x_3}
\\
&\qquad 
x_5{:}\susp{{\sf retn}\,\interp{\lambda x. \langle\rangle}}, ~~
x_4{:}\susp{{\sf cont}\,\lf{\sf ref1}}, ~~
x_2{:}\susp{{\sf cont}\,
  \lf{({\sf let1}\,\lambda f.\,
  \interp{{\sf let}\, x = (f := \lambda x. ({!}f) x)\,{\sf in}\,e})}}
\\
& \trstep{\lf{l_1}, x_6,y_1}{{\sf ev/ref1}\,\ldots\,(\tfuser{x_5}{x_4})}
\\
&\qquad
y_1{:}\susp{{\sf cell}\,\lf{l_1}\,\interp{\lambda x. \langle\rangle}}, ~~
x_6{:}\susp{{\sf retn}\,\lf{{\sf loc}\,l_1}}, ~~
x_2{:}\susp{{\sf cont}\,
  \lf{({\sf let1}\,\lambda f.\,
  \interp{{\sf let}\, x = (f := \lambda x. ({!}f) x)\,{\sf in}\,e})}}
\\
& \trstep{x_7}{{\sf ev/let1}\,\ldots\,(\tfuser{x_6}{x_2})}
\\
&\qquad
y_1{:}\susp{{\sf cell}\,\lf{l_1}\,\interp{\lambda x. \langle\rangle}}, ~~
x_7{:}\susp{{\sf eval}\,
  \lf{({\sf let}\,({\sf set}\,\lf{({\sf loc}\,l_1)}\,({\sf lam}\,\lambda x.\,{\sf app}\,({\sf get}\,({\sf loc}\,l_1))\,x))\,\lambda x.\,([({\sf loc}\,l_1)/f]\interp{e}))}}
\\
& \trstep{x_8,x_9}{{\sf ev/let}\,\ldots x_7}
\\
&\qquad
y_1{:}\susp{{\sf cell}\,\lf{l_1}\,\interp{\lambda x. \langle\rangle}}, ~~
x_8{:}\susp{{\sf eval}\,
  \lf{({\sf set}\,\lf{({\sf loc}\,l_1)}\,({\sf lam}\,\lambda x.\,{\sf app}\,({\sf get}\,({\sf loc}\,l_1))\,x))}}, ~~
x_9{:}\susp{{\sf cont}\,
  \lf{({\sf let1}\,\lambda x.\,([({\sf loc}\,l_1)/f]\interp{e}))}}
\\
& \trstep{x_{10},x_{11}}{{\sf ev/set}\,\ldots\,x_8}
\\
&\qquad
y_1{:}\susp{{\sf cell}\,\lf{l_1}\,\interp{\lambda x. \langle\rangle}}, ~~
x_{10}{:}\susp{{\sf eval}\,
  \lf{({\sf loc}\,l_1)}}, ~~
x_{11}{:}\susp{{\sf cont}\,
  \lf{({\sf set1}\,({\sf lam}\,\lambda x.\,{\sf app}\,({\sf get}\,({\sf loc}\,l_1))\,x))}}, ~~
x_9{:}\susp{{\sf cont}\,
  \lf{({\sf let1}\,\lambda x.\,([({\sf loc}\,l_1)/f]\interp{e}))}}
\\
& \trstep{x_{12}}{{\sf ev/loc}\,\ldots\,x_{10}}
\\
&\qquad
y_1{:}\susp{{\sf cell}\,\lf{l_1}\,\interp{\lambda x. \langle\rangle}}, ~~
x_{12}{:}\susp{{\sf retn}\,
  \lf{({\sf loc}\,l_1)}}, ~~
x_{11}{:}\susp{{\sf cont}\,
  \lf{({\sf set1}\,({\sf lam}\,\lambda x.\,{\sf app}\,({\sf get}\,({\sf loc}\,l_1))\,x))}}, ~~
x_9{:}\susp{{\sf cont}\,
  \lf{({\sf let1}\,\lambda x.\,([({\sf loc}\,l_1)/f]\interp{e}))}}
\\
& \trstep{x_{13},x_{14}}{{\sf ev/set1}\,\ldots\,(\tfuser{x_{12}}{x_{11}})}
\\
&\qquad
y_1{:}\susp{{\sf cell}\,\lf{l_1}\,\interp{\lambda x. \langle\rangle}}, ~~
x_{13}{:}\susp{{\sf eval}\,
  \lf{({\sf lam}\,\lambda x.\,{\sf app}\,({\sf get}\,({\sf loc}\,l_1))\,x)}}, ~~
x_{14}{:}\susp{{\sf cont}\,
  \lf{({\sf set2}\,l_1)}}, ~~
x_9{:}\susp{{\sf cont}\,
  \lf{({\sf let1}\,\lambda x.\,([({\sf loc}\,l_1)/f]\interp{e}))}}
\\
& \trstep{x_{15}}{{\sf ev/lam}\,\ldots\,x_{13}}
\\
&\qquad
y_1{:}\susp{{\sf cell}\,\lf{l_1}\,\interp{\lambda x. \langle\rangle}}, ~~
x_{15}{:}\susp{{\sf retn}\,
  \lf{({\sf lam}\,\lambda x.\,{\sf app}\,({\sf get}\,({\sf loc}\,l_1))\,x)}}, ~~
x_{14}{:}\susp{{\sf cont}\,
  \lf{({\sf set2}\,l_1)}}, ~~
x_9{:}\susp{{\sf cont}\,
  \lf{({\sf let1}\,\lambda x.\,([({\sf loc}\,l_1)/f]\interp{e}))}}
\\
& \trstep{x_{16},y_2}{{\sf ev/set2}\,\ldots\,(\tfuser{x_{15}}{\tfuser{x_{14}}{y_1}})}
\\
&\qquad
y_2{:}\susp{{\sf cell}\,\lf{l_1}\,
  \lf{({\sf lam}\,\lambda x.\,{\sf app}\,({\sf get}\,({\sf loc}\,l_1))\,x)}}, ~~
x_{16}{:}\susp{{\sf retn}\,
  \lf{\sf unit}}, ~~
x_9\susp{{\sf cont}\,
  \lf{({\sf let1}\,\lambda x.\,([({\sf loc}\,l_1)/f]\interp{e}))}}
\\
& \trstep{x_{17}}{{\sf ev/let1}\,\ldots\,(\tfuser{x_{16}}{x_9})}
\\
&\qquad
y_2{:}\susp{{\sf cell}\,\lf{l_1}\,
  \lf{({\sf lam}\,\lambda x.\,{\sf app}\,({\sf get}\,({\sf loc}\,l_1))\,x)}}, ~~
x_{17}{:}\susp{{\sf eval}\,
  \lf{[({\sf loc}\,l_1)/f, {\sf unit}/x]\interp{e})}}
\end{align*}
\caption{Back-patching. Judgments ($\mtrue$ and $\meph$) and
  arguments corresponding to implicit quantifiers are elided.}
\end{sidewaysfigure}

The name of this problem is {\it parameter dependency} -- the term
$\lf{v}$ in ${\sf gen/cell/bad}$ has to be instantiated before the
parameter $\lf{l}$ is introduced. As a result, the trace in
Figure~\ref{fig:bigbackpatch} includes a step
\[\trstep{x_{16},y_2}{{\sf
    ev/set2}\,\ldots\,(\tfuser{x_{15}}{\tfuser{x_{14}}{y_1}})}\] that
transitions from a state that can be described by
Figure~\ref{fig:gen-order} extended with ${\sf gen/cell/bad}$ to a
state that cannot be described by this signature. Thus, the extended
signature is not a generative invariant.

\begin{figure}[t]
\fvset{fontsize=\small,boxwidth=229pt}
\VerbatimInput{sls/gen-state.sls} 
\caption{Generative invariant: well-typed mutable storage}
\label{fig:gen-state} 
\end{figure}

The solution is to create cells in two steps. The first rule, a ${\sf
  promise}$ rule, creates the location $\lf{l}$ and associates a mobile
nonterminal ${\sf gencell}$ with that location. A second ${\sf
  fulfill}$ rule consumes that nonterminal and creates the actual
mutable cell.  Because ${\sf gencell}$ is a mobile nonterminal, the
promise {\it must} be fulfilled in order for the final state to pass
through the restriction operation. As we have already seen,
there is not much of a technical difference between well-formedness
invariants and well-typedness invariants; Figure~\ref{fig:gen-state}
describes a generative signature that captures type information. 
This specification introduces
two nonterminals. The first is the aforementioned mobile nonterminal
${\sf gencell}\,\lf{l}$, representing the promise to
eventually create a cell corresponding to the location $\lf{l}$.  
The second is a
persistent nonterminal ${\sf ofcell}\,\lf{l}\,\lf{\it tp}$. The
collection of ${\sf ofcell}$ propositions introduced by a generative
trace collectively plays the role of a {\it store typing} in
\cite[Chapter 13]{pierce02types} or a {\it signature} in \cite[Chapter
35]{harper12practical}.

\subsection{Inversion}

When we add mutable state, we must significantly generalize the 
{\it statement} of inversion lemmas. Derivations and
expressions now exist in a world with arbitrary locations $\lf{l}{:}{\sf
  mutable\_loc}$ that are paired with persistent propositions
${{\sf ofcell}\,\lf{l}\,\lf{\it tp}}$.\footnote{This purely persistent
world fits the pattern of regular worlds. 
As such, it can be described either with the
single rule
$\forall \lf{\it tp}.\,\{ \exists \lf{l}.\,{\sf
    ofcell}\,\lf{l}\,\lf{\it tp}\}$ or with the equivalent block
${\sf some} ~ \lf{\it tp}{:}{\sf
  typ} ~ {\sf block} ~ \lf{l}{:}{\sf mutable\_loc}, x : \ispers{\susp{{\sf
    ofcell}\,\lf{l}\,\lf{\it tp}}}.$}

\bigskip
\begin{lemma}[Inversion -- Figure~\ref{fig:gen-state}, partial]~
\begin{enumerate}
\item If 
   $T :: (\cdot; x_0{:}\istrue{\susp{{\sf gen}\,\lf{{\it tp}_0}}})
         \leadsto^*_{\siggenstate}
         (\Psi; \tackon{\Theta}
            {y{:}\istrue{\susp{{\sf eval}\,\lf{e}}}})$,
\\ then 
   $T = \left(T'; \trstep{y}{{\sf gen/eval}\,\lf{{\it tp}}\,\lf{e}\,
                                  (\tfuser{x'}{\tbangr{N}})}\right)$,
\\ where
   $\Psi; \Delta \vdash N : \isconc{{{\sf of}\,\lf{e}\,\lf{{\it tp}}}}$,
\\ $T' :: (\cdot; x_0{:}\istrue{\susp{{\sf gen}\,\lf{{\it tp}_0}}})
       \leadsto^*_{\siggenstate}
       (\Psi'; \tackon{\Theta}
          {x'{:}\istrue{\susp{{\sf gen}\,\lf{{\it tp}}}}})$, 
\\ and $\Delta$ is the persistent part of 
   $\tackon{\Theta}
          {x'{:}\istrue{\susp{{\sf gen}\,\lf{{\it tp}}}}}$.
   
\medskip
\item If 
   $T :: (\cdot; x_0{:}\istrue{\susp{{\sf gen}\,\lf{{\it tp}_0}}})
         \leadsto^*_{\siggenstate}
         (\Psi; \tackon{\Theta}
            {y_1{:}\istrue{\susp{{\sf gen}\,\lf{{\it tp}'}}}, ~
             y_2{:}\istrue{\susp{{\sf cont}\,\lf{f}}}})$,
\\ then 
   $T = \left(T'; \trstep{y_1,y_2}
              {{\sf gen/cont}\,\lf{{\it tp}}\,\lf{f}\,\lf{{\it tp}'}\,
                (\tfuser{x'}{\tbangr{N}})}\right)$,
\\ where 
   $\Psi; \Delta \vdash N : 
       \isconc{{{{\sf off}\,\lf{f}\,\lf{{\it tp}'}\,\lf{{\it tp}}}}}$,
\\ $T' :: (\cdot; x_0{:}\istrue{\susp{{\sf gen}\,\lf{{\it tp}_0}}})
       \leadsto^*_{\siggenstate}
       (\Psi'; \tackon{\Theta}
          {x'{:}\istrue{\susp{{\sf gen}\,\lf{{\it tp}}}}})$,
\\ and $\Delta$ is the persistent part of 
   $\tackon{\Theta}
          {x'{:}\istrue{\susp{{\sf gen}\,\lf{{\it tp}}}}}$.


\medskip
\item If $T :: (\cdot; x_0{:}\istrue{\susp{{\sf gen}\,\lf{{\it tp}_0}}})
         \leadsto^*_{\siggenstate} 
         (\Psi; \tackon{\Theta}
            {y{:}\istrue{\susp{{\sf cell}\,\lf{{\it l}}\,\lf{v}}}})$,
\\ then $T = (T'; \trstep{y}{{\sf gencell/fulfill}\,\lf{l}\,\lf{\it tp}\,\lf{v}\,(\tfuser{x'}{\tfuser{x_t}{\tfuser{\tbangr{N}}{\tbangr{N_v}}}})})$,
\\ where $x_t{:}\ispers{\susp{{\sf ofcell}\,\lf{l}\,\lf{\it tp}}} \in \Delta$,
   $\Psi; \Delta \vdash N : \isconc{{\sf of}\,\lf{v}\,\lf{\it tp}}$,
   $\Psi; \Delta \vdash N_v : \isconc{{\sf value}\,\lf{v}}$,
\\ $T' :: (\cdot; x_0{:}\istrue{\susp{{\sf gen}\,\lf{{\it tp}_0}}})
       \leadsto^*_{\siggenstate}
       (\Psi'; \tackon{\Theta}
          {x'{:}\iseph{\susp{{\sf gencell}\,\lf{l}}}})$,
\\ and $\Delta$ is the persistent part of 
   $\tackon{\Theta}
          {x'{:}\iseph{\susp{{\sf gencell}\,\lf{l}}}}$.
\end{enumerate}
\end{lemma}
\bigskip

Despite complicating the statement of inversion theorems, the addition
of mutable state does nothing to change the structure 
of these theorems.
The new inversion lemma (part 3 above) follows the
pattern established in Section~\ref{sec:inversion-genorder}.

\subsection{Uniqueness}
\label{sec:uniqueness-genstate}

To prove that our generative invariant for mutable storage is
maintained, we need one property besides inversion; we'll refer to it
as the {\it unique index} property. This is the property that, under
the generative signature described by $\siggenstate$, locations always
map {\it uniquely} to persistent positive propositions $x_t{:}{\sf
  ofcell}\,\lf{l}\,\lf{\it tp}$.

\bigskip
\begin{lemma}[Unique indices of $\siggenstate$]~
\begin{enumerate}
\item
If $T :: (\cdot; x_0{:}\istrue{\susp{{\sf gen}\,\lf{{\it tp}_0}}})
         \leadsto^*_{\siggenstate}
         (\Psi; \Delta)$,
\\ ${x}{:}\ispers{\susp{{\sf ofcell}\,\lf{l}\,\lf{\it tp}}} \in \Delta$, 
\\ and 
${y}{:}\ispers{\susp{{\sf ofcell}\,\lf{l}\,\lf{{\it tp}'}}} \in \Delta$, 
\\ then $x = y$ and $\lf{tp} = \lf{{\it tp}'}$.
\end{enumerate}
\end{lemma}

\begin{proof}
Induction and case analysis on the last steps of the trace $T$. 
\end{proof}


% We have to know that, given
% the persistent proposition ${\sf ofcell}\,\lf{l}\,\lf{{\it tp}}$ and
% the mobile proposition ${\sf gencell}\,\lf{l}\,\lf{{\it tp}'}$, we
% need to know that $\lf{{\it tp}} = \lf{{\it tp}'}$.

% $\{ {\sf ofcell/1} \}$
% $\{ {\sf gencell/1}, {\sf cell/1} \}$
% $\{ {\sf susp/1}, {\sf blackhole/1}, {\sf bind/1} \}$

% While it is a subtle point, unique indices are a property of the logical
% framework, but not the logic. 

\subsection{Preservation}

As it was with inversion, the statement of preservation is
substantially altered by the addition of locations and mutable state,
even though the structure of the proof is not.  In particular, because
${\sf ofcell}$ is a {\it persistent} nonterminal, we have to expressly
represent the fact that the restriction operator $\restrictsig{(\Psi;
  \Delta)}{}$ will modify the context $\Delta$ by erasing the store
typing.

\bigskip
\begin{theorem}[$\siggenstate$ is a generative invariant]
\label{thm:siggenstate}
If $T_1 :: (\cdot; x_0{:}\istrue{\susp{{\sf gen}\,\lf{{\it tp}_0}}}) 
   \leadsto^*_{\siggenstate} 
   (\Psi; \Delta)$ and 
   $S :: \restrictsig{(\Psi; \Delta)}{} \leadsto (\Psi'; \Delta')$
under the combined signature from Figure~\ref{fig:gen-order-prog}
and Figure~\ref{fig:ssos-mutable}, then
$(\Psi'; \Delta') = \restrictsig{(\Psi'; \Delta'')}{}$ 
for some $\Delta''$ such that 
$T_2 :: (\cdot; x_0{:}\istrue{\susp{{\sf gen}\,\lf{{\it tp}_0}}}) 
   \leadsto^*_{\siggenstate} 
   (\Psi'; \Delta'')$.
\end{theorem}

\begin{proof}
As always, the proof proceeds by enumeration, inversion, and reconstruction. 
The only interesting cases are the three that actually manipulate state,
corresponding to ${\sf ev/ref1}$, ${\sf ev/get1}$, and ${\sf ev/set2}$.
The last two are similar, so we only give the case for ${\sf ev/get1}$.

\begin{description}
%%%%%%%%
%%%%%%%% ev/ref1
%%%%%%%%
\item
  [Case $\trstep{\lf{l}, y_1, y_2}
         {{\sf ev/ref1}\,\lf{v}\,(\tfuser{x_1}{x_2})}$]~

\qquad 
  $:: \left(\Psi; ~ 
   \tackon{\Theta}{x_1{:}\istrue{\susp{{\sf retn}\,\lf{v}}}, ~
                   x_2{:}\istrue{\susp{{\sf cont}\,\lf{\sf ref1}}}}
   \right)$

\qquad\qquad
  $\leadsto \left(\Psi, ~ \lf{l}{:}{\sf mutable\_loc}; ~
   \tackon{\Theta}{y_1{:}\iseph{\susp{{\sf cell}\,\lf{l}\,\lf{v}}}, ~
                   y_2{:}\istrue{\susp{{\sf retn}\,\lf{({\sf loc}\,\lf{l})}}}}
   \right)$

   \medskip 
   $T_1 :: (\cdot; ~ x_0{:}\istrue{\susp{{\sf gen}\,\lf{{\it tp}_0}}}) 
    \leadsto^* \left(\Psi; ~
     \tackon{\Theta'}{x_1{:}\istrue{\susp{{\sf retn}\,\lf{v}}}, ~
                   x_2{:}\istrue{\susp{{\sf cont}\,\lf{\sf ref1}}}}
     \right)$ for some $\Theta'$ such that, for all $\Delta$,
   $\restrictsig{(\Psi; \tackon{\Theta'}{\Delta})}{} = (\Psi;
   \tackon{\Theta}{\restrictsig{\Delta}{}})$. Applying 
   inversion to $T_1$, we have

   \medskip

\begin{tabbing}
$T_1 = ~$ \= \qquad \= 
    $(\cdot; ~ x_0{:}\istrue{\susp{{\sf gen}\,\lf{{\it tp}_0}}})$
\\
\>$T'$
\\
\>\>$(\Psi; ~ \tackon{\Theta'}
      {x'{:}\istrue{\susp{{\sf gen}\,\lf{{\it tp}}}}})$
\\
\>$\trstep{x_1', x_2}
     {{\sf gen/cont}\,\lf{{\it tp}}\,\lf{\sf ref1}\,\lf{{\it tp}'}\,
         (\tfuser{x'}{\tbangr{N}})}$
\\ %%
\>\>$(\Psi; ~ \tackon{\Theta'}
     {x_1'{:}\istrue{\susp{{\sf gen}\,\lf{{\it tp}'}}}, ~
      x_2{:}\istrue{\susp{{\sf cont}\,\lf{\sf ref1}}}})$
\\
\>$\trstep{x_1}
     {{\sf gen/retn}\,\lf{v}\,(\tfuser{x_1'}{\tfuser{\tbangr{N_1}}{\tbangr{N_{v1}}}})}$
\\
\>\>$(\Psi; ~ \tackon{\Theta'}{x_1{:}\istrue{\susp{{\sf retn}\,\lf{v}}}, ~
                   x_2{:}\istrue{\susp{{\sf cont}\,\lf{\sf ref1}}}})$
\end{tabbing}
where $\Delta$ contains the persistent propositions from $\Theta'$ and where
\begin{itemize}
\item[$\bullet$] $\Psi; \Delta \vdash N : \isconc{{\sf off}\,\lf{\sf ref1}\,\lf{{\it tp}'}\,\lf{{\it tp}}}$.  By traditional
  inversion we know $\lf{\it{tp}} = \lf{{\sf reftp}\,{\it tp}'}$ and $N = {\sf off/ref1}\,\lf{{\it tp}'}$.
\item[$\bullet$] $\Psi; \Delta \vdash N_1 : \isconc{{\sf of}\,\lf{v}\,\lf{tp'}}$. 
\item[$\bullet$] $\Psi; \Delta \vdash N_{v1} : \isconc{{\sf value}\,\lf{v}}$.
\end{itemize}

We can use $T'$ to construct $T_2$ as follows:
\begin{tabbing}
$T_2 = ~$ \= \qquad \= 
    $(\cdot; ~x_0{:}\istrue{\susp{{\sf gen}\,\lf{{\it tp}_0}}})$
\\
\>$T'$
\\
\>\>$(\Psi; ~
     \tackon{\Theta'}
     {x'{:}\istrue{\susp{{\sf gen}\,\lf{({\sf reftp}\,{\it tp}')}}}})$
\\
\>$\trstep{\lf{l}, z, y_1'}{{\sf gencell/promise}\,\lf{{\it tp}'}}$
\\
\>$\trstep{y_1}{{\sf gencell/fulfill}\,\lf{l}\,\lf{{\it tp}'}\,\lf{v}\,
      (\tfuser{y_1'}{\tfuser{z}{\tfuser{\tbangr{N_1}}{\tbangr{N_{v1}}}}})}$
\\
\>\>$(\Psi, \lf{l}{:}{\sf mutable\_loc};$
\\
\>\>$\quad 
     \tackon{\Theta'}
     {z{:}\ispers{\susp{{\sf ofcell}\,\lf{l}\,\lf{{\it tp}'}}}, ~
      y_1{:}\iseph{\susp{{\sf cell}\,\lf{l}\,\lf{v}}}, ~
      x'{:}\istrue{\susp{{\sf gen}\,\lf{({\sf ref}\,{\it tp}')}}}})$
\\
\>$\trstep{y_2}
     {{\sf gen/retn}\,\lf{({\sf reftp}\,{\it tp}')}\,\lf{({\sf loc}\,l)}\,(\tfuser{x'}{\tfuser{\tbangr{({\sf of/loc}\,\lf{l}\,\lf{{\it tp}'}\,z)}}{\tbangr{({\sf value/loc}\,\lf{l})}}})}$
\\
\>\>$(\Psi, \lf{l}{:}{\sf mutable\_loc};$
\\
\>\>$\quad 
     \tackon{\Theta'}
     {z{:}\ispers{\susp{{\sf ofcell}\,\lf{l}\,\lf{{\it tp}'}}}, ~
      y_1{:}\iseph{\susp{{\sf cell}\,\lf{l}\,\lf{v}}}, ~
      x'{:}\istrue{\susp{{\sf retn}\,\lf{({\sf loc}\,l)}}}})$
\end{tabbing}

Restriction removes the persistent nonterminal $z{:}\ispers{\susp{{\sf
      ofcell}\,\lf{l}\,\lf{{\it tp}'}}}$ from the context, so the
restriction of $T_2$'s output is $\left(\Psi, ~ \lf{l}{:}{\sf mutable\_loc}; ~
   \tackon{\Theta}{y_1{:}\iseph{\susp{{\sf cell}\,\lf{l}\,\lf{v}}}, ~
                   y_2{:}\istrue{\susp{{\sf retn}\,\lf{({\sf loc}\,\lf{l})}}}}
   \right)$ as
required.

\medskip

%%%%%%%%
%%%%%%%% ev/get1
%%%%%%%%
\item
  [Case $\trstep{y_1, y_2}
         {{\sf ev/get1}\,\lf{l}\,\lf{v}\,(\tfuser{x_1}{\tfuser{x_2}{x_3}})}$]~

\qquad 
  $:: \left(\Psi; ~
   \tackon{\Theta}
    {x_1{:}\istrue{\susp{{\sf retn}\,\lf{({\sf loc}\,l)}}}, ~
     x_2{:}\istrue{\susp{{\sf cont}\,\lf{{\sf get1}}}}, ~
     x_3{:}\iseph{\susp{{\sf cell}\,\lf{l}\,\lf{v}}}
    }\right)$

\qquad\qquad
  $\leadsto \left(\Psi; ~
   \tackon{\Theta}
    {y_1{:}\istrue{\susp{{\sf retn}\,\lf{v}}}, ~
     y_2{:}\iseph{\susp{{\sf cell}\,\lf{l}\,\lf{v}}}
    }\right)$

   \medskip
   $T_1 :: (\cdot; x_0{:}\istrue{\susp{{\sf gen}\,\lf{{\it tp}_0}}}) $\\
    $ ~ \quad\qquad\leadsto^* \left(\Psi; ~
     \tackon{\Theta'}{x_1{:}\istrue{\susp{{\sf retn}\,\lf{({\sf loc}\,l)}}}, ~ 
     x_2{:}\istrue{\susp{{\sf cont}\,\lf{{\sf get1}}}}, ~
     x_3{:}\iseph{\susp{{\sf cell}\,\lf{l}\,\lf{v}}}
    }
     \right)$ \\
   for some $\Theta'$ such that, for all $\Delta$,
   $\restrictsig{(\Psi; \tackon{\Theta'}{\Delta})}{} = (\Psi;
   \tackon{\Theta}{\restrictsig{\Delta}{}})$. Applying 
   inversion to $T_1$, we have

   \medskip

\begin{tabbing}
$T_1 = ~$ \= \qquad \= 
    $(\cdot; ~ x_0{:}\istrue{\susp{{\sf gen}\,\lf{{\it tp}_0}}})$
\\
\>$T'$
\\
\>\> $\left(\Psi; ~
       \tackon{\Theta'}
       {x'{:}\istrue{\susp{{\sf gen}\,\lf{{\it tp}}}}, ~
        x_3'{:}\iseph{\susp{{\sf gencell}\,\lf{l}}}
       }\right)$
\\
\>$\trstep{x_3}{{\sf gencell/fulfill}\,\lf{l}\,\lf{{\it tp}'}\,\lf{v}\,
      (\tfuser{x_3'}{\tfuser{z_1}{\tfuser{\tbangr{N_3}}{\tbangr{N_{v3}}}}})}$
\\
\>\> $\left(\Psi; ~
       \tackon{\Theta'}
       {x'{:}\istrue{\susp{{\sf gen}\,\lf{{\it tp}}}}, ~
        x_3{:}\iseph{\susp{{\sf cell}\,\lf{l}\,\lf{v}}}
       }\right)$
\\
\>$\trstep{x_1',x_2}{{\sf gen/cont}\,
     \lf{{\it tp}}\,\lf{{\sf get1}}\,\lf{{\it tp}''}\,
     (\tfuser{x'}{\tbangr{N_2}})}$
\\
\>$\trstep{x_1}{{\sf gen/retn}\,\lf{{\it tp}''}\,\lf{({\sf loc}\,l)}\,
     (\tfuser{x_1'}{\tfuser{\tbangr{N_1}}{\tbangr{N_{v1}}}})}$ 
\\
\>\>$\left(\Psi; ~
   \tackon{\Theta}
    {x_1{:}\istrue{\susp{{\sf retn}\,\lf{({\sf loc}\,l)}}}, ~
     x_2{:}\istrue{\susp{{\sf cont}\,\lf{{\sf get1}}}}, ~
     x_2{:}\iseph{\susp{{\sf cell}\,\lf{l}\,\lf{v}}}
    }\right)$
\end{tabbing}

where $\Delta$ contains the persistent propositions from $\Theta'$ and where
\begin{itemize}
\item[$\bullet$] $\Psi; \Delta \vdash N_2 : \isconc{{\sf off}\,\lf{{\sf get1}}\,\lf{{\it tp}''}\,\lf{{\it tp}}}$. By traditional inversion we know $\lf{{\it tp}''} = \lf{{\sf reftp}\,{\it tp}}$ and $N_2 = {{\sf off/get1}\,\lf{\it tp}}$. 
\item[$\bullet$] $\Psi; \Delta \vdash N_1 : \isconc{{\sf of}\,\lf{({\sf loc}\,l)}\,\lf{({\sf reftp}\,\lf{{\it tp}})}}$. By traditional inversion we know $N_1 = {\sf of/loc}\,\lf{l}\,\lf{{\it tp}}\,x_1''$ where $x_1''{:}\ispers{\susp{{\sf ofcell}\,\lf{l}\,\lf{{\it tp}}}} \in \Delta$.
\item[$\bullet$] $x_3'{:}\ispers{\susp{{\sf ofcell}\,\lf{l}\,\lf{{\it tp}'}}} \in \Delta$.
\item[$\bullet$] $\Psi; \Delta \vdash N_3 : \isconc{{\sf of}\,\lf{v}\,\lf{{\it tp}'}}$.
\item[$\bullet$] $\Psi; \Delta \vdash N_{3v} : \isconc{{\sf value}\,\lf{v}}$.
\end{itemize}
By the uniqueness lemma, we have that $x_3' = x_1''$ and $\lf{{\it
    tp}'} = \lf{{\it tp}}$. Therefore, we can use $T'$ to construct $T_2$
as follows:

\begin{tabbing}
$T_1 = ~$ \= \qquad \= 
    $(\cdot; ~ x_0{:}\istrue{\susp{{\sf gen}\,\lf{{\it tp}_0}}})$
\\
\>$T'$
\\
\>\> $\left(\Psi; ~
       \tackon{\Theta'}
       {x'{:}\istrue{\susp{{\sf gen}\,\lf{{\it tp}}}}, ~
        x_3'{:}\iseph{\susp{{\sf gencell}\,\lf{l}}}
       }\right)$
\\
\>$\trstep{y_2}{{\sf gencell/fulfill}\,\lf{l}\,\lf{{\it tp}}\,\lf{v}\,
      (\tfuser{x_3'}{\tfuser{z_1}{\tfuser{\tbangr{N_3}}{\tbangr{N_{v3}}}}})}$
\\
\>\> $\left(\Psi; ~
       \tackon{\Theta'}
       {x'{:}\istrue{\susp{{\sf gen}\,\lf{{\it tp}}}}, ~
        y_2{:}\iseph{\susp{{\sf cell}\,\lf{l}\,\lf{v}}}
       }\right)$
\\
\>$\trstep{y_1}{{\sf gen/retn}\,\lf{{\it tp}}\,\lf{v}\,
     (\tfuser{x'}{\tfuser{\tbangr{N_3}}{\tbangr{N_{v3}}}})}$
\\
\>\> $\left(\Psi; ~
       \tackon{\Theta'}
       {y_1{:}\istrue{\susp{{\sf retn}\,\lf{v}}}, ~
        y_2{:}\iseph{\susp{{\sf cell}\,\lf{l}\,\lf{v}}}
       }\right)$
\end{tabbing}

$\left(\Psi; ~
   \tackon{\Theta}{y_1{:}\istrue{\susp{{\sf retn}\,\lf{({\sf loc}\,\lf{l})},
                   y_2{:}\iseph{\susp{{\sf cell}\,\lf{l}\,\lf{v}}}}}}
   \right)$ is the restriction of $T_2$'s output, as
required.


%%%%%%%%
%%%%%%%% ev/set2
%%%%%%%%
% \item
%   [Case $\trstep{y_1, y_2}
%          {{\sf ev/set2}\,\lf{v_2}\,\lf{l}\,\lf{v_x}\,
%           (\tfuser{x_1}{\tfuser{x_2}{x_3}})}$]~

% \qquad 
%   $:: \left(\Psi; ~
%    \tackon{\Theta}
%     {}\right)$

% \qquad\qquad
%   $\leadsto \left(\Psi; ~
%    \tackon{\Theta}
%     {}\right)$

%    \medskip
%    We have a $\Theta'$ such that for all $\Delta$ composed 
%    only of nonterminals, $\restrictsig{(\Psi;
%      \tackon{\Theta'}{\Delta})}{} = 
%      \tackon{\Theta}{\Delta}$. Applying 
%    inversion to $T_1$, we have

\end{description}
The other cases, notably ${\sf ev/set2}$, follow the 
same pattern.
\end{proof}


\begin{figure}
\fvset{fontsize=\small,boxwidth=229pt}
\VerbatimInput{sls/gen-countstate.sls} 
\caption{Two generative invariants for counting invariants}
\label{fig:gen-countstate}
\end{figure}
\subsection{Revisiting pointer inequality}
\label{sec:pointer-inequality}

As we discussed in Section~\ref{sec:mutable-storage}, the fact that
\sls~variables cannot be directly checked for inequality complicates
the representation of languages that can check for the inequality of
locations. One way of circumventing this shortcoming is by keeping a
runtime counter in the form of an ephemeral atomic proposition ${\sf
  count}\,\lf{n}$ that counts the number of currently allocated cells;
the rule ${\sf ev/ref1}$ that allocates a new cell must be modified
to access and increment this counter and to attach the counter's value
to the new cell. Inequality of those natural number tags can then be
used as a proxy for inequality of locations.

A generative signature like the one in Figure~\ref{fig:gen-countstate}
could be used to represent the invariant that each location and each
cell is associated with a unique natural number.


\section{Destination-passing}
\label{sec:gen-destinations}

Destination-passing style specifications, as discussed in
Chapter~\ref{chapter-destinations}, are not a focus of this thesis,
but they deserve mention due to their importance in the context of the
logical framework CLF. In this section, we will work with an
operational semantics derived from the signature given in
Figure~\ref{fig:dest-cbv} (sequential evaluation of function
application) in Chapter~\ref{chapter-destinations}. To use sequential
application instead of parallel evaluation of function application, we
will need to give different typing rules for frames:

\smallskip
\fvset{fontsize=\small,boxwidth=229pt}
\VerbatimInput{sls/gen-sequential-app.sls}
\smallskip

\noindent
Other than this change, our deductive typing rules stay the same.


\begin{figure}[tp]
\fvset{fontsize=\small,boxwidth=229pt}
\VerbatimInput{sls/gen-destinations2.sls} 
\caption{Generative invariant: destination-passing (``obvious'' formulation)}
\label{fig:gen-destinations2} 
\end{figure}

When we move from ordered abstract machines to destination-passing
style, the most natural adaptation of generative invariants is
arguably the one given in Figure~\ref{fig:gen-destinations}. In that
figure, the core nonterminal is the mobile proposition ${\sf
  gen}\,\lf{{\it tp}}\,\lf{d}$. The rule ${\sf gen/dest}$, which 
creates destinations freely, is
necessary, as we can see from the following sequence of 
process states:
\begin{align*}
( \lf{d_0}{:}{\sf dest}&; ~
       x_1{:}\iseph{\susp{{\sf eval}
          \,\lf{\interp{(\lambda x.e)\,e_2}}\,\lf{d_0}}}
) 
\leadsto
\\
( \lf{d_0}{:}{\sf dest}, \lf{d_1}{:}{\sf dest}&; ~ 
       x_2{:}\iseph{\susp{{\sf eval}\,\lf{\interp{(\lambda x.e)}}\,\lf{d_1}}}, ~
       x_3{:}\iseph{\susp{{\sf cont}
          \,\lf{({\sf app1}\,\interp{e_2})}\,\lf{d_1}\,\lf{d_0}}}
)
\leadsto
\\
( \lf{d_0}{:}{\sf dest}, \lf{d_1}{:}{\sf dest}&; ~ 
       x_4{:}\iseph{\susp{{\sf retn}\,\lf{\interp{(\lambda x.e)}}\,\lf{d_1}}}, ~
       x_3{:}\iseph{\susp{{\sf cont}
          \,\lf{({\sf app1}\,\interp{e_2})}\,\lf{d_1}\,\lf{d_0}}}
)
\leadsto
\\
( \lf{d_0}{:}{\sf dest}, \lf{d_1}{:}{\sf dest}, \lf{d_2}{:}{\sf dest}&; ~ 
       x_5{:}\iseph{\susp{{\sf eval}\,\lf{\interp{e_2}}\,\lf{d_2}}}, ~
       x_6{:}\iseph{\susp{{\sf cont}
          \,\lf{({\sf app2}\,\interp{\lambda x.e})}\,\lf{d_2}\,\lf{d_0}}}
)
\leadsto \ldots
\end{align*}
In the final state, $\lf{d_1}$ is isolated, no longer mentioned
anywhere else in the process state, so ${\sf gen/dest}$ must be used
in the generative trace showing that the last state above is well-typed.

\begin{figure}[tp]
\fvset{fontsize=\small,boxwidth=229pt}
\VerbatimInput{sls/gen-destinations.sls} 
\caption{Generative invariant: destination-passing (modified formulation)}
\label{fig:gen-destinations} 
\end{figure}

We will not use the form described in
Figure~\ref{fig:gen-destinations2} in this chapter, however. Instead,
we will prefer the presentation in Figure~\ref{fig:gen-destinations}.
There are two reasons for this. First and foremost, this formulation
meshes better with the promise-then-fulfill pattern that was necessary
for state in Figure~\ref{fig:gen-state} and that is also necessary for
continuations in Section~\ref{sec:gen-letcc} below. As a secondary
consideration, using the first formulation would require us to 
significantly change the structure of our inversion lemmas.
In previous inversion lemmas, proving
that ${\sf gen/cont}$ could always be rotated to the end of a
generative trace was simple, because it introduced no LF variables or
persistent nonterminals. The ${\sf gen/cont}$ rule in
Figure~\ref{fig:gen-destinations2} does introduce an LF variable
$\lf{d'}$, invalidating the principle used in
Section~\ref{sec:inversion-genorder}.

The ${\sf dest/promise}$ run in Figure~\ref{fig:gen-destinations} is
interesting in that it requires each destination $\lf{d'}$ to be
created along with foreknowledge of the destination, $\lf{d}$, that
the destination $\lf{d'}$ will return to. This effectively forces
all the destinations into a tree structure from the moment of their
creation onwards, a point that will become important when we modify
Figure~\ref{fig:gen-destinations} to account for persistent
destinations and first-class continuations.

\subsection{Uniqueness and index sets}
\label{sec:uniqueness-gendests}

One consequence of using the promise-then-fulfill pattern is that our
unique index lemma becomes conceptually prior to our inversion lemma.

\bigskip
\begin{lemma}[Unique indices of $\siggendests$]~
\begin{enumerate}
\item
If $T :: (\lf{d_0}{:}{\sf dest}; x_0{:}\istrue{\susp{{\sf gen}\,\lf{{\it tp}_0}\,\lf{d_0}}})
         \leadsto^*_{\siggendests}
         (\Psi; \Delta)$,
\\ $x{:}\iseph{\susp{{\sf gendest}\,\lf{d}\,\lf{d_1}}} \in \Delta$, 
 and 
$y{:}\iseph{\susp{{\sf gendest}\,\lf{d}\,\lf{d_2}}} \in \Delta$, 
\\ then $x = y$ and $\lf{d_1} = \lf{d_2}$.

\medskip
\item
If $T :: (\lf{d_0}{:}{\sf dest}; x_0{:}\istrue{\susp{{\sf gen}\,\lf{{\it tp}_0}\,\lf{d_0}}})
         \leadsto^*_{\siggendests}
         (\Psi; \Delta)$,
\\ $x{:}\iseph{\susp{{\sf gendest}\,\lf{d}\,\lf{d_1}}} \in \Delta$, 
 and
$y{:}\iseph{\susp{{\sf gen}\,\lf{\it tp}\,\lf{d}}} \in \Delta$,
\\
then there is a contradiction.

\medskip
\item
If $T :: (\lf{d_0}{:}{\sf dest}; x_0{:}\istrue{\susp{{\sf gen}\,\lf{{\it tp}_0}\,\lf{d_0}}})
         \leadsto^*_{\siggendests}
         (\Psi; \Delta)$,
\\ $x{:}\iseph{\susp{{\sf gendest}\,\lf{d}\,\lf{d_1}}} \in \Delta$, 
 and
$y{:}\iseph{\susp{{\sf cont}\,\lf{\it f}\,\lf{d}\,\lf{d'}}} \in \Delta$,
\\
then there is a contradiction.

\medskip
\item
If $T :: (\lf{d_0}{:}{\sf dest}; x_0{:}\istrue{\susp{{\sf gen}\,\lf{{\it tp}_0}\,\lf{d_0}}})
         \leadsto^*_{\siggendests}
         (\Psi; \Delta)$,
\\ $x{:}\iseph{\susp{{\sf gen}\,\lf{{\it tp}_1}\,\lf{d}}} \in \Delta$, 
 and 
$y{:}\iseph{\susp{{\sf gen}\,\lf{{\it tp}_2}\,\lf{d}}} \in \Delta$, 
\\ then $x = y$ and $\lf{{\it tp}_1} = \lf{{\it tp}_2}$.

\medskip
\item
If $T :: (\lf{d_0}{:}{\sf dest}; x_0{:}\istrue{\susp{{\sf gen}\,\lf{{\it tp}_0}\,\lf{d_0}}})
         \leadsto^*_{\siggendests}
         (\Psi; \Delta)$,
\\ $x{:}\iseph{\susp{{\sf cont}\,\lf{f_1}\,\lf{d}\,\lf{d_1}}} \in \Delta$, 
 and 
$y{:}\iseph{\susp{{\sf cont}\,\lf{f_2}\,\lf{d}\,\lf{d_2}}} \in \Delta$, 
\\ then $x = y$, $\lf{f_1} = \lf{f_2}$, and $\lf{d_1} = \lf{d_2}$.

\end{enumerate}
\end{lemma}

\begin{proof}
  Induction and case analysis on last steps of the trace $T$; each
  part uses the previous parts (parts 2 and 3 use part 1, and parts 4
  and 5 use parts 2 and 3).
\end{proof}

This lemma is a complicated way of expressing what is ultimately a
very simple property: that the second position of ${\sf gendest}$ is a
unique index and that it passes on that unique indexing to the second
position of ${\sf gen}$ and the second position of ${\sf cont}$. 

\bigskip
\begin{definition}
A set $S$ is a {\em unique index set} under a generative signature
$\Sigma$ and an initial state $(\Psi; \Delta)$ if, whenever
\begin{itemize}
\item ${\sf a}/i \in S$, 
\item ${\sf b}/j \in S$, 
\item $(\Psi; \Delta) \leadsto^*_{\Sigma} (\Psi; \Delta)$, 
\item $x{:}\islvl{\susp{{\sf a}\,\lf{t_1}\ldots \lf{t_n}}} \in \Delta$, and
\item $y{:}\islvl{\susp{{\sf b}\,\lf{s_1}\ldots \lf{s_m}}}' \in \Delta$,
\end{itemize}
it is the case that $\lf{t_i} = \lf{s_j}$ implies $x = y$. Of course, if
$x = y$, that in turn implies that ${\sf a} = {\sf b}$, 
$i = j$, $n = m$, $\lf{t_k} = \lf{s_k}$ for $1 \leq k \leq n$, and 
$\mlvl = \mlvl'$. 
\end{definition}
\bigskip

The complicated lemma above can then be captured by the dramatically
more concise statement: $\{ {\sf gendest}/2, {\sf gen}/2 \}$ and $\{
{\sf gendest}/2, {\sf cont}/2 \}$ are both unique index sets under the
signature $\siggendests$ and the initial state $(\lf{d_0}{:}{\sf
  dest}; x_0{:}\istrue{\susp{{\sf gen}\,\lf{{\it
        tp}_0}\,\lf{d_0}}})$. In fact, we can extend the first unique
index set to $\{ {\sf gendest}/2, {\sf gen}/2, {\sf eval}/2, {\sf
  retn}/2 \}$. Stating that $\{ {\sf gendest}/2, {\sf gen}/2 \}$ was a
unique index property previously required 3 distinct statements, and
it would take 10 distinct statements to express that $\{ {\sf
  gendest}/2, {\sf gen}/2, {\sf eval}/2, {\sf retn}/2 \}$ is a unique
index property.\footnote{Four positive statements (similar to parts 1, 4, and 5 of the lemma above) along ${{4}\choose{2}} =
  6$ negative ones (similar to parts 3 and 4 of the lemma above).}

It's also possible for unique index sets to be simply (and,
presumably, mechanically) checked. This amounts to a very simple
preservation property.

\subsection{Inversion}

In order to prove inversion using uniqueness and the
promise-then-fulfill style generative specification in
Figure~\ref{fig:gen-destinations2}, we need a more general inversion
lemma than the one we had before. Not only do we have to consider
cases analogous to the ones described in
Section~\ref{sec:inversion-genorder} (cases 1-3), we need another set
of cases to deal with {\it partial} matches (cases 4 and 5). 

\bigskip
\begin{lemma}[Inversion -- Figure~\ref{fig:gen-destinations}]~
\begin{enumerate}
\item If 
   $T :: (\lf{d_0}{:}{\sf dest}; ~
            x_0{:}\iseph{\susp{{\sf gen}\,\lf{{\it tp}_0}\,\lf{d_0}}})
         \leadsto^*_{\siggendests}
         (\Psi; ~
          \tackon{\Theta}
           {y{:}\iseph{\susp{{\sf eval}\,\lf{e}\,\lf{d}}}})$,
\\ then 
   $T = \left(T';
       \trstep{y}{{\sf gen/eval}\,\lf{{\it tp}}\,\lf{d}\,\lf{e}
       \,(\tfuser{x'}{\tbangr{N}})}\right)$,
\\ where
   $\Psi; \Delta \vdash N : \isconc{{\sf of}\,\lf{e}\,\lf{{\it tp}}}$,
\\ $T' :: (\lf{d_0}{:}{\sf dest}; ~
            x_0{:}\iseph{\susp{{\sf gen}\,\lf{{\it tp}_0}\,\lf{d_0}}})
         \leadsto^*_{\siggendests}
         (\Psi; ~
          \tackon{\Theta}
           {x'{:}\iseph{\susp{{\sf gen}\,\lf{{\it tp}}\,\lf{d}}}})$,
\\ and $\Delta$ is the persistent part of 
   $\tackon{\Theta}
           {x'{:}\iseph{\susp{{\sf gen}\,\lf{{\it tp}}\,\lf{d}}}}$.


\medskip
\item If 
   $T :: (\lf{d_0}{:}{\sf dest}; ~
            x_0{:}\iseph{\susp{{\sf gen}\,\lf{{\it tp}_0}\,\lf{d_0}}})
         \leadsto^*_{\siggendests}
         (\Psi; ~
          \tackon{\Theta}
           {y{:}\iseph{\susp{{\sf retn}\,\lf{v}\,\lf{d}}}})$,
\\ then 
   $T = \left(T';
       \trstep{y}{{\sf gen/retn}\,\lf{{\it tp}}\,\lf{d}\,\lf{v}
       \,(\tfuser{x'}{\tfuser{\tbangr{N}}{\tbangr{N_v}}})}\right)$,
\\ where
   $\Psi; \Delta \vdash N : \isconc{{\sf of}\,\lf{e}\,\lf{{\it tp}}}$,
   $\Psi; \Delta \vdash N_v : \isconc{{\sf value}\,\lf{v}}$, 
\\ $T' :: (\lf{d_0}{:}{\sf dest}; ~
            x_0{:}\iseph{\susp{{\sf gen}\,\lf{{\it tp}_0}\,\lf{d_0}}})
         \leadsto^*_{\siggendests}
         (\Psi; ~
          \tackon{\Theta}
           {x'{:}\iseph{\susp{{\sf gen}\,\lf{{\it tp}}\,\lf{d}}}})$,
\\ and $\Delta$ is the persistent part of 
   $\tackon{\Theta}
           {x'{:}\iseph{\susp{{\sf gen}\,\lf{{\it tp}}\,\lf{d}}}}$.


\medskip
\item If 
   $T :: (\lf{d_0}{:}{\sf dest}; ~
            x_0{:}\iseph{\susp{{\sf gen}\,\lf{{\it tp}_0}\,\lf{d_0}}})$
\\
   $~\qquad \qquad \leadsto^*_{\siggendests}
         (\Psi; ~
          \tackon{\Theta}
           {y_1{:}\iseph{\susp{{\sf gen}\,\lf{{\it tp}'}\,\lf{d'}}}, ~
            y_2{:}\iseph{\susp{{\sf cont}\,\lf{f}\,\lf{d'}\,\lf{d}}}})$,
\smallskip
\\ then 
   $T = \left(T';
       \trstep{y_1, y_2}{{\sf gen/cont}
       \,\lf{{\it tp}}\,\lf{d}\,\lf{f}\,\lf{{\it tp}'}\,\lf{d'}
       \,(\tfuser{x'}{\tfuser{\tbangr{N}}{z}})}\right)$,
\\ where
   $\Psi; \Delta \vdash N : \isconc{{\sf off}\,\lf{e}\,\lf{{\it tp}'}\,\lf{{\it tp}}}$,
\\ $T' :: (\lf{d_0}{:}{\sf dest}; ~
            x_0{:}\iseph{\susp{{\sf gen}\,\lf{{\it tp}_0}\,\lf{d_0}}})$
\\ $~\qquad\qquad        \leadsto^*_{\siggendests}
         (\Psi; ~
          \tackon{\Theta}
           {x'{:}\iseph{\susp{{\sf gen}\,\lf{{\it tp}}\,\lf{d}}}, ~
            z{:}\iseph{\susp{{\sf gendest}\,\lf{d'}\,\lf{d}}}})$,
\smallskip
\\ and $\Delta$ is the persistent part of 
   $\tackon{\Theta}
           {x'{:}\iseph{\susp{{\sf gen}\,\lf{{\it tp}}\,\lf{d}}}, ~
            z{:}\iseph{\susp{{\sf gendest}\,\lf{d'}\,\lf{d}}}}$.

% \medskip
% \item If 
%    $T :: (\lf{d_0}{:}{\sf dest}; ~
%             x_0{:}\iseph{\susp{{\sf gen}\,\lf{{\it tp}_0}\,\lf{d_0}}})
%          \leadsto^*_{\siggendests}
%          (\Psi; ~
%           \tackon{\Theta}
%            {y{:}\iseph{\susp{{\sf gen}\,\lf{{\it tp}}\,\lf{d}}}})$,


% \medskip
% \item If 
%    $T :: (\lf{d_0}{:}{\sf dest}; ~
%             x_0{:}\iseph{\susp{{\sf gen}\,\lf{{\it tp}_0}\,\lf{d_0}}})
%          \leadsto^*_{\siggendests}
%          (\Psi; ~
%           \tackon{\Theta}
%            {y{:}\iseph{\susp{{\sf cont}\,\lf{f}\,\lf{d'}\,\lf{d}}}})$,


\medskip
\end{enumerate}
\end{lemma}

\begin{proof}
  As with other inversion lemmas, each case follows by induction and
  case analysis on the last steps of $T$. The trace cannot be empty,
  so $T = T''; S$ for some $T''$ and $S$, and we let ${\it Var}$ be
  the set of relevant variables $\{y\}$ in parts 1 and 2, and
  $\{y_1,y_2\}$ in part 3.

  If $\emptyset = S^{\bullet} \cap {\it Var}$, the proof proceeds by
  induction as it did in Section~\ref{sec:inversion-genorder}.

  If $S^{\bullet} \cap {\it Var}$ is nonempty, then we must again show
  by case analysis that $S^{\bullet} = {\it Var}$ and that furthermore
  $S$ is the step we were looking for. As before, this is easy for the
  unary grammar productions where ${\it Var}$ is a singleton set:
  there is only one rule that can produce an atomic proposition 
  ${\sf eval}\,\lf{e}\,\lf{d}$ or ${\sf retn}\,\lf{v}\,\lf{d}$. 

  When ${\it Var}$ is not a singleton (which only happens in part 3
  for this lemma), we must use the unique index property to reason
  that if there is any overlap, that overlap must be total.
  \begin{itemize}
  \item Say $S = \trstep{y_1, y_2''}
    {{\sf gen/cont}\,
      \lf{{\it tp}}\,\lf{d''}\,\lf{f''}\,\lf{{\it tp}'}\,\lf{d'}\,
      (\tfuser{x'}{\tfuser{\tbangr{N}}{z}})}$.

      Then the final state contains
      $y_2{:}\iseph{\susp{{\sf cont}\,\lf{f}\,\lf{d'}\,\lf{d}}}$
      and
      $y_2''{:}\iseph{\susp{{\sf cont}\,\lf{f''}\,\lf{d'}\,\lf{d''}}}$.
      The shared $\lf{d'}$ and the unique index property 
      ensures that $y_2 = y_2''$, $\lf{f} = \lf{f''}$, and
      $\lf{d} = \lf{d''}$.

  \smallskip
  \item Say  $S = \trstep{y_1'', y_2}
    {{\sf gen/cont}\,
      \lf{{\it tp}}\,\lf{d}\,\lf{f}\,\lf{{\it tp}'''}\,\lf{d'}\,
      (\tfuser{x'}{\tfuser{\tbangr{N}}{z}})}$.

    Then the final state contains
    $y_1{:}\iseph{\susp{{\sf gen}\,\lf{{\it tp}'}\,\lf{d'}}}$
    and 
    $y_1''{:}\iseph{\susp{{\sf gen}\,\lf{{\it tp}'''}\,\lf{d'}}}$.
    The shared $\lf{d'}$ and the unique index property ensures that
    $y_1 = y_1''$ and $\lf{{\it tp}'} = \lf{{\it tp}'''}$. 
  \end{itemize}
Therefore, $S = \trstep{y_1, y_2}
    {{\sf gen/cont}\,
      \lf{{\it tp}}\,\lf{d}\,\lf{f}\,\lf{{\it tp}'}\,\lf{d'}\,
      (\tfuser{x'}{\tfuser{\tbangr{N}}{z}})}$.
\end{proof}

\subsection{Preservation}

As we once again have no persistent nonterminals, we
can return to the simpler form of the preservation theorem used in
Theorem~\ref{thm:siggenorder} and Theorem~\ref{thm:siggenordertp}
(compared to the more complex formulation needed for
Theorem~\ref{thm:siggenstate}).

\medskip
\begin{theorem}[$\siggendests$ is a generative invariant]
\label{thm:siggendests}
If $T_1 :: (\lf{d_0}{:}{\sf dest}; x_0{:}\iseph{\susp{{\sf gen}\,\lf{{\it tp}_0}\,\lf{d_0}}}) 
   \leadsto^*_{\siggendests} 
   (\Psi; \Delta)$ and there is a step
   $S :: \restrictsig{(\Psi; \Delta)}{} \leadsto (\Psi'; \Delta')$
under the signature from Figure~\ref{fig:dest-cbv}, then
$T_2 :: (\lf{d_0}{:}{\sf dest}; x_0{:}\istrue{\susp{{\sf gen}\,\lf{{\it tp}_0}\,\lf{d_0}}}) 
   \leadsto^*_{\siggendests} 
   (\Psi'; \Delta')$.
\end{theorem}

\begin{proof} As usual, we enumerate the synthetic transitions possible under
the signature in Figure~\ref{fig:dest-cbv}, perform inversion 
on the structure of $T_1$, and then use the results of inversion to
construct $T_2$. We give one illustrative case. 

\begin{description}
%%%%%%%%
%%%%%%%% ev/app1
%%%%%%%%

\item
  [Case $\trstep{\lf{d_2}, y_1, y_2}
    {{\sf ev/app1}\,\lf{(\lambda x. e)}\,\lf{d_1}\,\lf{e_2}\,\lf{d}\,
      (\tfuser{x_1}{x_2})}$]~

\qquad 
  $:: \left(\Psi; ~
   \tackon{\Theta}
    {x_1{:}\iseph{\susp{{\sf retn}\,
       \lf{({\sf lam}\, \lambda x.e)}\,\lf{d_1}}}, ~
     x_2{:}\iseph{\susp{{\sf cont}\,\lf{({\sf app1}\,e_2)}\,\lf{d_1}\,\lf{d}}}
    }\right)$

\qquad\qquad
  $\leadsto \left(\Psi, \lf{d_2}{:}{\sf dest}; ~
   \tackon{\Theta}
    {y_1{:}\iseph{\susp{{\sf eval}\,\lf{e_2}\,\lf{d_2}}}, ~
     y_2{:}\iseph{\susp{{\sf cont}\,
       \lf{({\sf app2}\,\lambda x.e)}\,\lf{d_2}\,\lf{d}}}
    }\right)$

\medskip
Applying inversion (Part 2, then Part 3) to $T_1$, we have

\begin{tabbing}
$T_1 = ~$ \= \qquad \= $(\lf{d_0}{:}{\sf dest}; ~ x_0{:}\iseph{\susp{{\sf gen}\,\lf{{\it tp}_0}\,\lf{d_0}}})$
\\
\> $T'$
\\
\>\> $\left(\Psi; ~
   \tackon{\Theta}
    {x'{:}\iseph{\susp{{\sf gen}\,\lf{{\it tp}}\,\lf{d}}}, ~
     z_1{:}\iseph{\susp{{\sf gendest}\,\lf{d_1}\,\lf{d}}}
    }\right)$
\\
\> $\trstep{x_1', x_2}{{\sf gen/cont}\,
       \lf{{\it tp}}\,\lf{d}\,\lf{({\sf app1}\,e_2)}\,
       \lf{{\it tp}'}\,\lf{d_1}\,
       (\tfuser{x'}{\tfuser{\tbangr{N_2}}{z_1}})}$
\\
\>\> $\left(\Psi; ~
   \tackon{\Theta}
    {x_1'{:}\iseph{\susp{{\sf gen}\,\lf{{\it tp}'}\,\lf{d_1}}}, ~
     x_2{:}\iseph{\susp{{\sf cont}\,\lf{({\sf app1}\,e_2)}\,\lf{d_1}\,\lf{d}}}
    }\right)$
\\
\> $\trstep{x_1}{{\sf gen/retn}\,
       \lf{{\it tp}'}\,\lf{d_1}\,\lf{v}\,
       (\tfuser{x_1'}{\tfuser{\tbangr{N_1}}{\tbangr{N_{v1}}}})}$
\\
\>\> $\left(\Psi; ~
   \tackon{\Theta}
    {x_1{:}\iseph{\susp{{\sf retn}\,\lf{({\sf lam}\,\lambda x.e)}\,\lf{d_1}}}, ~
     x_2{:}\iseph{\susp{{\sf cont}\,\lf{({\sf app1}\,e_2)}\,\lf{d_1}\,\lf{d}}}
    }\right)$
\end{tabbing}

where $\Delta$ contains the persistent propositions from $\Theta$ and where
\begin{itemize}
\item[$\bullet$]  
  $\Psi; \Delta \vdash N_2 : \isconc{{\sf off}\,\lf{({\sf app1}\,\lf{e_2})}\,\lf{{\it tp}'}\,\lf{{\it tp}}}$. 
  By traditional inversion we know there exists $\lf{{\it tp}''}$
  and $N_2'$ such that
  $\lf{{\it tp}'} = \lf{{\sf arr}\,{\it tp}''\,{\it tp}}$ and
  $\Psi; \Delta \vdash N_2' : \isconc{{\sf of}\,\lf{e_2}\,\lf{{\it tp}''}}$.
\item[$\bullet$] $\Psi; \Delta \vdash N_1 : \isconc{{\sf of}\,\lf{({\sf lam}\,\lambda
    x.e)}\,\lf{{\sf arr}\,{\it tp}'\,{\it tp}}}$. By traditional inversion we know there 
    exists $N_1'$ where 
    $\Psi, \lf{x}{:}{\sf exp}; \Delta, {\it dx} : \ispers{({\sf of}\,\lf{x}\,\lf{\it
      tp'})} \vdash N_1' : \isconc{{\sf of}\,\lf{e}\,\lf{\it
        tp}}$.
\end{itemize}
We can use $T'$ to construct $T_2$ as follows: 

\begin{tabbing}
$T_2 = ~$ \= \qquad \= $(\lf{d_0}{:}{\sf dest}; ~ x_0{:}\iseph{\susp{{\sf gen}\,\lf{{\it tp}_0}\,\lf{d_0}}})$
\\
\> $T'$
\\
\>\> $\left(\Psi; ~
   \tackon{\Theta}
    {x'{:}\iseph{\susp{{\sf gen}\,\lf{{\it tp}}\,\lf{d}}}, ~
     z_1{:}\iseph{\susp{{\sf gendest}\,\lf{d_1}\,\lf{d}}}
    }\right)$
\\
\> $\trstep{}{{\sf dest/unused}\,\lf{d_1}\,\lf{d}\,z_1}$
\\
\> $\trstep{\lf{d_2}, z_2}{{\sf dest/promise}\,\lf{d}}$
\\
\>\> $\left(\Psi, \lf{d_2}{:}{\sf dest}; ~
   \tackon{\Theta}
    {x'{:}\iseph{\susp{{\sf gen}\,\lf{{\it tp}}\,\lf{d}}}, ~
     z_2{:}\iseph{\susp{{\sf gendest}\,\lf{d_2}\,\lf{d}}}
    }\right)$
\\
\> $\trstep{y_1',y_2}{{\sf gen/cont}\,\lf{{\it tp}}\,\lf{d}\,
      \lf{({\sf app2}\lambda x. e)}\,\lf{{\it tp}''}\,\lf{d_2}}$
\\
\> \qquad\qquad\qquad\qquad
     $(\tfuser{x'}{\tfuser{!({\sf off/app2}\,\lf{{\it tp}''}\,\lf{(\lambda x.e)}\,\lf{{\it tp}}\,(\lambda \lf{x}, {\it dx}.\,\tbangr{N_1'}))}{z_2}})$
\\
\>\> $\left(\Psi, \lf{d_2}{:}{\sf dest}; ~
   \tackon{\Theta}
    {y_1'{:}\iseph{\susp{{\sf gen}\,\lf{{\it tp}''}\,\lf{d_2}}}, ~
     y_2{:}\iseph{\susp{{\sf cont}\,\lf{({\sf app2}\,\lambda x.e)}\,\lf{d_2}\,\lf{d}}}
    }\right)$
\\
\> $\trstep{y_1, y_2}{{\sf gen/eval}\,\lf{{\it tp}''}\,\lf{d_2}\,\lf{e_2}\,
      (\tfuser{y_1'}{\tbangr{N_2'}})}$
\\
\>\> $\left(\Psi, \lf{d_2}{:}{\sf dest}; ~
   \tackon{\Theta}
    {y_1{:}\iseph{\susp{{\sf eval}\,\lf{e_2}\,\lf{d_2}}}, ~
     y_2{:}\iseph{\susp{{\sf cont}\,\lf{({\sf app2}\,\lambda x.e)}\,\lf{d_2}\,\lf{d}}}
    }\right)$
\end{tabbing}

\medskip

\end{description}

\noindent
The other cases follow the same pattern.
\end{proof}

\subsection{Extensions}

Generative invariants for parallel evaluation
(Figure~\ref{fig:dest-pair}) and the alternate semantics of
parallelism and failure (Figure~\ref{fig:dest-fail-paror}) as
described in Section~\ref{sec:modular-parallelism} are
straightforward extensions of the development in this section.

\begin{figure}[tp]
\fvset{fontsize=\small,boxwidth=229pt}
\VerbatimInput{sls/gen-future.sls} 
\caption{Generative invariant: futures}
\label{fig:gen-future} 
\end{figure}

Synchronization (Section~\ref{sec:dest-synch}) and futures
(Section~\ref{sec:dest-futures}) are just a bit more interesting from
the perspective of generative invariants and preservation. The
generative invariant for synchronization should be straightforward if
we associate types with channels in the style of \cite[Section
41.5]{harper12practical}. One possible approach to futures is sketched
out in Figure~\ref{fig:gen-future}; it follows the same
promise-then-fulfill pattern used for mutable storage and
destination-passing in all our examples so far. The reason we do not
pursue either of these examples further is that, while preservation is
doable, the progress theorems that we turn to in
Chapter~\ref{chapter-safety} are complicated. Progress as it is
usually stated does not even hold for 
synchronization due to the possibility of deadlocks. Progress does
hold for futures, but it is not obvious that the generative invariant
in Figure~\ref{fig:gen-future} is strong enough to prove progress.


\section{Persistent continuations}
\label{sec:gen-letcc}

The final specification style we will cover in detail is the use of
persistent continuations as discussed in
Section~\ref{sec:dest-continuations} as a way of giving an SSOS
semantics for first-class continuations (Figure~\ref{fig:dest-letcc}).
While the setup of Figure~\ref{fig:gen-destinations} were designed to
make the transition to persistent continuations and $\obj{\sf letcc}$
seem less unusual, this section still represents a radical shift.  

It should not be terribly surprising that the generative invariant for
persistent continuations is rather different than the other generative
invariants. Generative invariants capture patterns of
specification, and the we have mostly concentrated on patterns that
facilitate
concurrency and communication. Persistent continuations, on the other
hand, are a pattern mostly associated with first-class
continuations. The integration of continuations and parallel or
concurrent evaluation is non-obvious; the proposal by Moreau and
Ribbens in \cite{moreau96semantics} is not straightforward to adapt to
the semantic specifications we gave in
Chapter~\ref{chapter-destinations}.

Consider again the ${\sf gendest/promise}$ rule from
Figure~\ref{fig:gen-destinations}. Because the rule consumes no
nonterminals and is the only rule to introduce LF variables, any trace
$T :: \left(\lf{d_0}{:}{\sf dest}; x_0{:}{\sf gen}\,\lf{{\it
      tp}_0}\,\lf{d_0}\right) \leadsto^* {(\Psi;
  \Delta)}{}$ under $\siggendests$ can be factored into two parts $T =
T_1; T_2$ where $T_1$ contains only steps that use ${\sf
  gendest/promise}$. The computational effect of
Theorem~\ref{thm:siggendests} is that $T_1$ grows to track the
tree-structured shape of the stack, both past and present. We could
record, if we wanted to, the {\it past} structure of the control stack
by adding a persistent nonterminal ${\sf
  ghostcont}\,\lf{f}\,\lf{d'}\,\lf{d}$ and modifying ${\sf dest/unused}$
in Figure~\ref{fig:gen-destinations} as follows:

\smallskip
\fvset{fontsize=\small,boxwidth=229pt}
\VerbatimInput{sls/gen-ghost.sls}
\smallskip

\begin{figure}[tp]
\fvset{fontsize=\small,boxwidth=229pt}
\VerbatimInput{sls/gen-letcc2.sls} 
\caption{Generative invariant: persistent destinations and first-class
  continuations}
\label{fig:gen-letcc2} 
\end{figure}

Once we make the move to persistent continuations, however, there's no
need to create a ghost continuation, we can just have the rule ${\sf
  dest/unused}$ (renamed to ${\sf dest/fulfill}$ in
Figure~\ref{fig:gen-letcc2}) create the continuation itself.  To make
this work, ${\sf dest/promise}$ predicts the type that will be
associated with a newly-generated destination $\lf{d}$ by generating a
persistent nonterminal ${\sf ofdest}\,\lf{d}\,\lf{{\it tp}}$. (This is
just like how ${\sf gencell/promise}$ in Figure~\ref{fig:gen-state}
predicts the type of a location $\lf{l}$ by generating a persistent
nonterminal ${\sf ofcell}\,\lf{l}\,\lf{{\it tp}}$.) Then, ${\sf
  dest/fulfill}$ checks to make sure that the generated continuation
frame has the right type relative to the destinations it connects.

Taken together, the rules ${\sf dest/promise}$ and ${\sf
  dest/fulfill}$ in Figure~\ref{fig:gen-letcc2} create a
tree-structured map of destinations starting from an initial
destination $\lf{d_0}$ and an initial persistent atomic proposition
${\sf ofdest}\,\lf{d_0}\,\lf{{\it tp}_0}$. This initial proposition
takes over for the mobile proposition ${\sf gen}\,\lf{{\it
    tp}_0}\,\lf{d_0}$ as the root of the stack; the mobile ${\sf gen}$
nonterminal no longer needs indices, and just serves to place a single
${\sf eval}$ or ${\sf retn}$ somewhere on the well-typed map of
destinations. The initial state of our generative traces is therefore
$\left(\lf{d_0}{:}{\sf dest}; x_0{:}\ispers{\susp{{\sf
        ofdest}\,\lf{d_0}\,\lf{{\it tp}_0}}}, z{:}\iseph{\susp{{\sf
        gen}}}\right)$; this is reflected in the the preservation
theorem.

\bigskip
\begin{lemma}[Unique indices of $\siggenletcc$] Both $\{ {\sf
    ofdest}/1 \}$ and $\{ {\sf gendest}/1, {\sf cont}/2 \}$ are unique
  index sets under the initial state
  $\left(\lf{d_0}{:}{\sf dest}; x_0{:}\ispers{\susp{{\sf
          ofdest}\,\lf{d_0}\,\lf{{\it tp}_0}}}, z{:}\iseph{\susp{{\sf gen}}}\right)$ and the signature $\siggenletcc$.
\end{lemma}

\begin{proof}
Induction and case analysis on the last steps of a given trace.
\end{proof}

\newpage
\begin{lemma}[Inversion -- Figure~\ref{fig:gen-letcc2}] ~
\begin{enumerate}
\item If $T :: \left(\lf{d_0}{:}{\sf dest}; ~ x_0{:}\ispers{\susp{{\sf
          ofdest}\,\lf{d_0}\,\lf{{\it tp}_0}}}, ~ z{:}\iseph{\susp{{\sf
          gen}}}\right) \leadsto^*_{\siggenletcc} \left(\Psi; ~
    \tackon{\Theta}{y{:}\iseph{\susp{{\sf
            eval}\,\lf{e}\,\lf{d}}}}\right)$,

    then $T = (T'; \trstep{y}{{\sf gen/eval}\,\lf{d}\,\lf{{\it tp}}\,\lf{e}\,
      (\tfuser{z'}{\tfuser{x}{\tbangr{N}}})})$,

    where $x{:}\ispers{\susp{{\sf ofdest}\,\lf{d}\,\lf{{\it tp}}}} \in \Delta$,
    $\Psi; \Delta \vdash N : \isconc{{\sf of}\,\lf{e}\,\lf{{\it tp}}}$,
    
    $T' :: \left(\lf{d_0}{:}{\sf dest}; ~ x_0{:}\ispers{\susp{{\sf
          ofdest}\,\lf{d_0}\,\lf{{\it tp}_0}}}, ~ z{:}\iseph{\susp{{\sf
          gen}}}\right) \leadsto^*_{\siggenletcc}
          \left(\Psi; ~ \tackon{\Theta}{z'{:}\iseph{\susp{{\sf gen}}}}\right)$,

    and $\Delta$ is the persistent part of 
    $\left(\Psi; ~ \tackon{\Theta}{z'{:}\iseph{\susp{{\sf gen}}}}\right)$.

\medskip
\item If $T :: \left(\lf{d_0}{:}{\sf dest}; ~ x_0{:}\ispers{\susp{{\sf
          ofdest}\,\lf{d_0}\,\lf{{\it tp}_0}}}, ~ z{:}\iseph{\susp{{\sf
          gen}}}\right) \leadsto^*_{\siggenletcc} \left(\Psi; ~
    \tackon{\Theta}{y{:}\iseph{\susp{{\sf
            retn}\,\lf{v}\,\lf{d}}}}\right)$,

    then $T = (T'; \trstep{y}{{\sf gen/retn}\,\lf{d}\,\lf{{\it tp}}\,\lf{v}\,
      (\tfuser{z'}{\tfuser{x}{\tfuser{\tbangr{N}}{\tbangr{N_v}}}})})$,

    where $x{:}\ispers{\susp{{\sf ofdest}\,\lf{d}\,\lf{{\it tp}}}} \in \Delta$,
    $\Psi; \Delta \vdash N : \isconc{{\sf of}\,\lf{v}\,\lf{{\it tp}}}$,
    $\Psi; \Delta \vdash N : \isconc{{\sf value}\,\lf{v}}$,
    
    $T' :: \left(\lf{d_0}{:}{\sf dest}; ~ x_0{:}\ispers{\susp{{\sf
          ofdest}\,\lf{d_0}\,\lf{{\it tp}_0}}}, ~ z{:}\iseph{\susp{{\sf
          gen}}}\right) \leadsto^*_{\siggenletcc}
          \left(\Psi; ~ \tackon{\Theta}{z'{:}\iseph{\susp{{\sf gen}}}}\right)$,

    and $\Delta$ is the persistent part of 
    $\left(\Psi; ~ \tackon{\Theta}{z'{:}\iseph{\susp{{\sf gen}}}}\right)$.


\medskip
\item If $T :: \left(\lf{d_0}{:}{\sf dest}; ~ x_0{:}\ispers{\susp{{\sf
          ofdest}\,\lf{d_0}\,\lf{{\it tp}_0}}},  ~ z{:}\iseph{\susp{{\sf
          gen}}}\right)$

   $~\qquad \qquad \leadsto^*_{\siggenletcc} \left(\Psi; ~
    \tackon{\Theta}{y{:}\ispers{\susp{{\sf
            cont}\,\lf{f}\,\lf{d'}\,\lf{d}}}}\right)$,
\smallskip

    then $T = (T'; \trstep{y}{{\sf dest/fulfill}\,\lf{d'}\,\lf{d}\,\lf{f}\,\lf{{\it tp}'}\,\lf{{\it tp}}\,
      (\tfuser{y'}{\tfuser{\tbangr{N}}{\tfuser{x'}{x}}})})$,

    where $x'{:}\ispers{\susp{{\sf ofdest}\,\lf{d'}\,\lf{{\it tp}'}}} 
           \in \Delta$,
    $x{:}\ispers{\susp{{\sf ofdest}\,\lf{d}\,\lf{{\it tp}}}} \in \Delta$,
    $\Psi; \Delta \vdash N : 
       \isconc{{\sf off}\,\lf{f}\,\lf{{\it tp}'}\,\lf{{\it tp}}}$

    $T' :: \left(\lf{d_0}{:}{\sf dest}; x_0{:}\ispers{\susp{{\sf
          ofdest}\,\lf{d_0}\,\lf{{\it tp}_0}}}, z{:}\iseph{\susp{{\sf
          gen}}}\right)$

   $~\qquad \qquad \leadsto^*_{\siggenletcc} \left(\Psi; ~
          \tackon{\Theta}
            {y'{:}\iseph{\susp{{\sf gendest}\,\lf{d'}\,\lf{d}}}}\right)$,

    and $\Delta$ is the persistent part of 
    $\left(\Psi; ~ \tackon{\Theta}
      {z'{:}\iseph{\susp{{\sf gendest}\,\lf{d'}\,\lf{d}}}}\right)$.
\smallskip
 
\end{enumerate}
\end{lemma}

\begin{proof}
  Induction and case analysis on last steps of the trace $T$; each
  case is individually quite simple because ${\it Var}$ is always a
  singleton $\{ y \}$.  While we introduce a persistent atomic
  proposition ${\sf cont}$ in Part 3, the step that introduces this
  proposition can always be rotated to the end of a trace because
  ${\sf cont}$ propositions cannot appear in the input interface of
  any step in under the generative signature $\siggenletcc$. This is a
  specific property of $\siggenletcc$, but it also follows from the
  definition of generative signatures (Definition~\ref{def:gensig}),
  which stipulates that transitions enabled by a generative signature
  cannot consume or mention terminals like ${\sf cont}$.

  As an aside, $z$ will always equal $z'$ in Parts 1 and 2, but we'll
  never need to rely on this fact.
\end{proof}


\bigskip
\begin{theorem}[$\siggenletcc$ is a generative invariant] ~
  \label{thm:siggenletcc}
  \smallskip

  If $T_1 :: \left(\lf{d_0}{:}{\sf dest}; x_0{:}\ispers{\susp{{\sf
          ofdest}\,\lf{d_0}\,\lf{{\it tp}_0}}}, z{:}\iseph{\susp{{\sf
          gen}}}\right) \leadsto^*_{\siggenletcc} \left(\Psi; \Delta
  \right)$ and $S :: \restrictsig{\left(\Psi; \Delta\right)}{}
  \leadsto \left(\Psi'; \Delta'\right)$ under the signature from
  Figure~\ref{sec:dest-continuations}, then $(\Psi'; \Delta') =
  \restrictsig{\left(\Psi; \Delta\right)}{}$ for some $\Delta''$
  such that $T_2 :: \left(\lf{d_0}{:}{\sf dest}; x_0{:}\ispers{\susp{{\sf
          ofdest}\,\lf{d_0}\,\lf{{\it tp}_0}}}, z{:}\iseph{\susp{{\sf
          gen}}}\right) \leadsto^*_{\siggenletcc} \left(\Psi'; \Delta''
  \right)$.
\end{theorem}

\begin{proof}
  As always, the proof proceeds by enumeration, inversion, and
  reconstruction; the cases are all fundamentally similar to the ones
  we have already seen.
\end{proof}

\newpage

\section{On mechanization}

In this chapter, we have shown that generative invariants can describe
well-formedness and well-typedness properties of the full range of
specifications discussed in Part II of this thesis. We have
furthermore shown that these generative invariants are a suitable
basis for reasoning about type preservation in these specifications.
All of these proofs have a common 3-step structure: 

\smallskip
\begin{enumerate}
\item Straightforward unique index properties,
\item An inversion lemma that mimics the
  structure of the generative signature, and
\item A preservation theorem that proceeds by enumerating transitions,
applying inversion to the given generative trace, and using the result
to construct a new generative trace.
\end{enumerate}
\smallskip

\noindent
Despite the fact that the inversion lemmas in this chapter technically
use induction, they do so in such a trivial way that is quite possible
to imagine that inversion lemmas could be automatically synthesized
from a generative signature. Unique index properties may be less
straightforward to synthesize, but like termination and mode
properties in Twelf, they should be entirely straightforward to
verify. Only the last part of step 3, the reconstruction that happens
in a preservation theorem, has the structure of a more general theorem
proving task. Therefore, there is reason to hope that we can mechanize the
tedious results in the results in this chapter in a framework that
does much of the work of steps 1 and 2 automatically.



% \begin{figure}
% \noindent
% {\bf Counting down:}
% \smallskip
% \fvset{fontsize=\small,boxwidth=229pt}
% \VerbatimInput{sls/gen-count.sls} 

% \bigskip
% \noindent
% {\bf Counting up:}
% \smallskip
% \fvset{fontsize=\small,boxwidth=229pt}
% \VerbatimInput{sls/gen-count2.sls} 
% \caption{Two generative invariants for counting invaraints}
% \label{fig:gen-count}
% \end{figure}

% Certainly, however, many potentially interesting generative invariants
% will not be so easy to capture and verify correct. One example of this
% has come up in work by Henry de Young on voting protocols
% \cite{deyoung12linear}. In that setting, it is frequently crucial to
% maintain a single ephemeral proposition that collects information
% about a set of ephemeral propositions. The simplest example would
% be a mobile proposition ${\sf count}\,\lf{n}$ where the natural
% nubmer $\lf{n}$ represents the number of ${\sf item}$ propositions 
% present in the state. There are two obvious ways of representing
% this situation with a generative invariant, both shown in
% Figure~\ref{fig:gen-count}. However, reasoning about preservation



%  a counter can be used to ensure that each destination
% in a mutable store . While \sls~variables cannot be directly
% checked for 

% \subsection{Parity}

% \subsection{}

% \subsection{Pointer inequality}


% \chapter{Checking invariants}

\chapter{Safety for substructural specifications}

% \chapter{Programming with canonical forms}

% \newcommand{\F}[1]{\ensuremath{F({#1})}}
% \newcommand{\G}[1]{\ensuremath{G(\textcolor{TrueBlue}{#1})}}
% \newcommand{\upX}[1]{\ensuremath{{\uparrow}\textcolor{ValidBlue}{#1}}}
% \newcommand{\downX}[1]{\ensuremath{{\downarrow}\textcolor{ValidRed}{#1}}}
% \newcommand{\upA}[1]{\ensuremath{{\uparrow}\textcolor{TrueBlue}{#1}}}
% \newcommand{\downA}[1]{\ensuremath{{\downarrow}\textcolor{TrueRed}{#1}}}

% \newcommand{\valid}[1]{\ensuremath{{\downarrow}\textcolor{ValidBlue}{{#1}\,\mathit{valid}}}}
% \newcommand{\true}[1]{\ensuremath{{\downarrow}\textcolor{TrueBlue}{{#1}\,\mathit{true}}}}

% \newcommand{\ajseq}[2]{\ensuremath{\mathstrut{#1} \vdash {#2}}}
% \newcommand{\ajinv}[3]{\ajseq{{#1}; {#2}}{\textcolor{ValidRed}{#3}}}
% \newcommand{\ajrfoc}[2]{\ajseq{{#1}}{[\textcolor{ValidBlue}{#2}]}}
% \newcommand{\ajlfoc}[3]{\ajseq{{#1} [{#2}]}{\textcolor{ValidRed}{#3}}}
% \newcommand{\ajAseq}[3]{\ensuremath{\mathstrut{#1} \vdash {#2}}}
% \newcommand{\ajAinv}[4]{\ajseq{{#1}; {#2}}{\textcolor{ValidRed}{#3}}}
% \newcommand{\ajArfoc}[3]{\ajseq{{#1}}{[\textcolor{ValidBlue}{#2}]}}
% \newcommand{\ajAlfoc}[4]{\ajseq{{#1} [{#2}]}{\textcolor{ValidRed}{#3}}}
% \newcommand{\ajXseq}[2]{\ensuremath{\mathstrut{#1} \vdash {#2}}}
% \newcommand{\ajXinv}[3]{\ajseq{{#1}; {#2}}{\textcolor{ValidRed}{#3}}}
% \newcommand{\ajXrfoc}[2]{\ajseq{{#1}}{[\textcolor{ValidBlue}{#2}]}}
% \newcommand{\ajXlfoc}[3]{\ajseq{{#1} [{#2}]}{\textcolor{ValidRed}{#3}}}

% \begin{figure}
% \fbox{\ajXrfoc{\Gamma}{X^+}}
% \[
% \infer[x^+_R]
% {\ajrfoc{\valid{x^+}}{x^+}}
% {}
% \qquad
% \infer[{\downarrow}_{XR}]
% {\ajXrfoc{\Gamma}{\downX{X^-}}}
% {\ajXinv{\Gamma}{\cdot}{X^-}}
% \qquad
% \infer[G_R]
% {\ajXrfoc{\Gamma}{\G{A}}}
% {\ajArfoc{\Gamma}{\cdot}{A}}
% \]

% \fbox{\ajArfoc{\Gamma}{\Delta}{A^+}}
% \[
% \infer[a^+_R]
% {\ajArfoc{\cdot}{a^+}{a^+}}
% {}
% \qquad
% \infer[{\downarrow}_{AR}]
% {\ajArfoc{\Gamma}{\Delta}{\downA{A^-}}}
% {\ajAinv{\Gamma}{\Delta}{\cdot}{A^-}}
% \]

% \caption{Focused adjoint logic}
% \end{figure}

\chapter{Conclusion}

\appendix

\chapter{Hybrid SSOS specification}
\label{chapter-appendix-hybrid}

%\include{appendix}

\backmatter

%\renewcommand{\baselinestretch}{1.0}\normalsize

% By default \bibsection is \chapter*, but we really want this to show
% up in the table of contents and pdf bookmarks.
\renewcommand{\bibsection}{\chapter{\bibname}}
%\newcommand{\bibpreamble}{This text goes between the ``Bibliography''
%  header and the actual list of references}
\bibliographystyle{alpha}
\bibliography{ref} %your bib file

\end{document}
