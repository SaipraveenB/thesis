%for a more compact document, add the option openany to avoid
%starting all chapters on odd numbered pages
\documentclass[12pt,openany]{cmuthesis}

% This is a template for a CMU thesis.  It is 18 pages without any content :-)
% The source for this is pulled from a variety of sources and people.
% Here's a partial list of people who may or may have not contributed:
%
%        bnoble   = Brian Noble
%        caruana  = Rich Caruana
%        colohan  = Chris Colohan
%        jab      = Justin Boyan
%        josullvn = Joseph O'Sullivan
%        jrs      = Jonathan Shewchuk
%        kosak    = Corey Kosak
%        mjz      = Matt Zekauskas (mattz@cs)
%        pdinda   = Peter Dinda
%        pfr      = Patrick Riley
%        dkoes = David Koes (me)

% My main contribution is putting everything into a single class files and small
% template since I prefer this to some complicated sprawling directory tree with
% makefiles.

\definecolor{ValidRed}{rgb}{0.5,0,0}
\definecolor{TrueRed}{rgb}{1,0.6,0.2}
\definecolor{ValidBlue}{rgb}{0,0,0.5}
\definecolor{TrueBlue}{rgb}{0.2,0.6,1}

% some useful packages
\usepackage{fancyvrb}
\usepackage{times}
\usepackage{dashrule}
\usepackage{proof-dashed}
\usepackage[show]{ed}
\usepackage{fullpage}
\usepackage{graphicx}
\usepackage{xcolor}
\usepackage{tikz}
\usepackage{rotating}
\usetikzlibrary{arrows,decorations.pathmorphing,backgrounds,fit}
\usepackage{amsthm}
\usepackage{amsmath}
\usepackage{latexsym}
\usepackage{amssymb}            % for \multimap (-o)
\usepackage{stmaryrd}           % for \binampersand (&), \bindnasrepma (\paar)
\usepackage{wasysym}            % for \ocircle
\usepackage[numbers,sort]{natbib}
\usepackage[backref,pageanchor=true,plainpages=false, pdfpagelabels, bookmarks,bookmarksnumbered,
%pdfborder=0 0 0,  %removes outlines around hyper links in online display
]{hyperref}
\usepackage{subfigure}

% Approximately 1" margins, more space on binding side
%\usepackage[letterpaper,twoside,vscale=.8,hscale=.75,nomarginpar]{geometry}
%for general printing (not binding)
\usepackage[letterpaper,twoside,vscale=.8,hscale=.75,nomarginpar,hmarginratio=1:1]{geometry}

% Provides a draft mark at the top of the document. 

\definecolor{grayout}{rgb}{.8,.8,.8}
\definecolor{plesantgreen}{rgb}{.1,.5,0}
\newcommand{\gr}[1]{\textcolor{plesantgreen}{\ensuremath{{#1}}}}

\definecolor{lfblue}{rgb}{.5,.2,0}
\newcommand{\lf}[1]{\textcolor{lfblue}{\ensuremath{{#1}}}}



\definecolor{justblack}{rgb}{0,0,0}
\newcommand{\no}[1]{\textcolor{justblack}{\ensuremath{{#1}}}}

\definecolor{objgreen}{rgb}{0,.3,.5}
\newcommand{\obj}[1]{\textcolor{objgreen}{\ensuremath{{#1}}}}

\hypersetup{colorlinks=true,citecolor=blue,urlcolor=blue,linkcolor=black}

\newcommand{\robnote}[1]{\footnote{{\bf NOTE TO SELF:} ~~ {#1}}}
\newcommand{\futurework}[1]{}

\newcommand{\ollll}{OL$_3$}
\newcommand{\sls}{SLS}
\newcommand{\simplearrow}{\rightarrow}
\newcommand{\rowan}{${\lambda}^{\ocircle}$}

\newcommand{\mb}{Coq}

\renewcommand{\labelitemi}{$\ast$}



% Theorems
\newtheorem{theorem}{Theorem}[chapter]
\newtheorem*{lemma}{Lemma}
\newtheorem{proposition}[theorem]{Proposition}
\newtheorem{definition}[theorem]{Definition}

% symbols of linear logic
\newcommand{\lolli}{\multimap}
\newcommand{\tensor}{\otimes}
\newcommand{\with}{\mathbin{\binampersand}}
\newcommand{\paar}{\mathbin{\bindnasrepma}}
\newcommand{\one}{\mathbf{1}}
\newcommand{\zero}{\mathbf{0}}
\newcommand{\bang}{{!}}
\newcommand{\pbang}{\mbox{\hspace{2pt}$\mathbb !$\hspace{-4.7pt}$_\curlyvee$\hspace{-1pt}}}
\newcommand{\whynot}{{?}}
\newcommand{\bilolli}{\mathrel{\raisebox{1pt}{\ensuremath{\scriptstyle\circ}}{\lolli}}}
% \oplus, \top, \bot
\newcommand{\deupdown}{\mbox{${\uparrow}{\downarrow}\hspace{-11.4pt}\diagup$}}
\newcommand{\dedownup}{\mbox{${\downarrow}{\uparrow}\hspace{-11.4pt}\diagdown$}}

\newcommand{\restrictto}[2]{\ensuremath{{#1}{\upharpoonright}_{#2}}}
\newcommand{\restrictsig}[2]{\ensuremath{{#1}{\mbox{\lightning}}_{#2}}}
\newcommand{\restrictfrom}[2]{\ensuremath{{#1}{\downharpoonright}^{#2}}}

% symbols of ordered logic
\newcommand{\fuse}{\mathbin{\bullet}}
\newcommand{\righti}{\twoheadrightarrow}
\newcommand{\lefti}{\rightarrowtail}
\newcommand{\gnab}{\mbox{\textexclamdown}}
\newcommand{\scriptgnab}{\mbox{\scriptsize{\textexclamdown}}}

\newcommand{\mconc}{{\it true}}
\newcommand{\mvalid}{{\it valid}}

\newcommand{\mlax}{{\it lax}}
\newcommand{\mtrue}{{\it ord}}
\newcommand{\meph}{{\it eph}}
\newcommand{\mpers}{{\it pers}}
\newcommand{\mlvl}{{\it lvl}}

\newcommand{\isconc}[1]{{#1}\,{\mconc}}
\newcommand{\isvalid}[1]{{#1}\,{\mvalid}}

\newcommand{\islax}[1]{{#1}\,{\mlax}}
\newcommand{\istrue}[1]{{#1}\,{\mtrue}}
\newcommand{\iseph}[1]{{#1}\,{\meph}}
\newcommand{\ispers}[1]{{#1}\,{\mpers}}
\newcommand{\islvl}[1]{{#1}\,{\mlvl}}

% judgments of linear logic
\newcommand{\altv}{\Longrightarrow}
\newcommand{\seq}[3]{{#1};{#2} \longrightarrow {#3} \mathstrut}
\newcommand{\altseq}[3]{{#1};{#2} \altv {#3} \mathstrut}
\newcommand{\pseq}[2]{{#1} \longrightarrow {#2} \mathstrut}

\newcommand{\mildseq}[3]{{#1};{#2} \vdash {#3} \mathstrut}
\newcommand{\andseq}[3]{{#1};{#2} \Vdash {#3} \mathstrut}
\newcommand{\mildrfoc}[3]{{#1};{#2} \vdash [{#3}] \mathstrut}
\newcommand{\mildinv}[3]{{#1};{#2} \vdash {#3} \mathstrut}
\newcommand{\mildlfoc}[4]{{#1};{#2}, [{#3}] \vdash {#4} \mathstrut}

\newcommand{\foc}[3]{{#1};{#2} \vdash {#3} \mathstrut}
\newcommand{\focx}[3]{{#1};{#2} \vdash_{\Sigma,\subord} {#3} \mathstrut}
\newcommand{\rfoc}[3]{{#1};{#2} \vdash [{#3}] \mathstrut}
\newcommand{\ifoc}[4]{{#1};{#2} {\mid} {#3} \vdash {#4} \mathstrut}
\newcommand{\lfoc}[4]{{#1};{#2}[{#3}] \vdash {#4} \mathstrut}

\newcommand{\foct}[4]{{#1}; {#2} \vdash {#3} : {#4} \mathstrut}
\newcommand{\foctx}[4]{{#1}; {#2} \vdash_{\Sigma,\subord} {#3} : {#4}
 \mathstrut}
\newcommand{\rfoct}[4]{{#1};{#2} \vdash {#3} : [{#4}] \mathstrut}
\newcommand{\lfoct}[4]{{#1};{#2}[{#3}] \vdash {#3} : [{#4}] \mathstrut}

\newcommand{\focsx}[5]{{#1}; {#2} \vdash_{\Sigma,\subord} \{{#3}\}{#4} : {#5}
 \mathstrut}

\newcommand{\slss}[4]{{#2}; {#3} \vdash_{#1} {#4}}
\newcommand{\slst}[5]{{#2}; {#3} \vdash_{#1} {#4} : {#5}}


\newcommand{\tfocusr}[1]{{#1}}
\newcommand{\tfocusl}[2]{{#1} \cdot {#2}}
\newcommand{\tetan}[1]{\langle {#1} \rangle}
\newcommand{\tetapp}[2]{\langle {#1} \rangle_p . {#2} }
\newcommand{\tetapl}[2]{\langle {#1} \rangle_l . {#2} }
\newcommand{\tetap}[2]{\langle {#1} \rangle . {#2}}
\newcommand{\tupr}[1]{{\uparrow}{#1}}
\newcommand{\tupl}[1]{{\uparrow}{#1}}
\newcommand{\tlaxr}[1]{\{{#1}\}}
\newcommand{\tlaxl}[1]{\{{#1}\}}
\newcommand{\tdownr}[1]{{\downarrow}{#1}}
\newcommand{\tdownl}[2]{{\downarrow}{#1}.{#2}}
\newcommand{\tbangr}[1]{{!}{#1}}
\newcommand{\tbangl}[2]{{!}{#1}.{#2}}
\newcommand{\tgnabr}[1]{{\gnab}{#1}}
\newcommand{\tgnabl}[2]{{\gnab}{#1}.{#2}}
\newcommand{\tnil}{\textsc{nil}}
\newcommand{\tabort}{\textsc{abort}}
\newcommand{\tinl}[1]{\textsc{inl}({#1})}
\newcommand{\tinr}[1]{\textsc{inr}({#1})}
\newcommand{\toplusl}[2]{[{#1},{#2}]}
\newcommand{\toner}{()}
\newcommand{\ttopr}{\top}
\newcommand{\tfuser}[2]{{#1} \fuse {#2}}
\newcommand{\twithr}[2]{{#1} \with {#2}}
\newcommand{\tpione}[1]{\pi_1;{#1}}
\newcommand{\tpitwo}[1]{\pi_2;{#1}}
\newcommand{\tfusel}[1]{{\fuse}{#1}}
\newcommand{\tlamr}[1]{{\lambda}^{>}{#1}}
\newcommand{\tappr}[2]{{#1}{^{>}}{#2}}
\newcommand{\tlaml}[1]{{\lambda}^{<}{#1}}
\newcommand{\tappl}[2]{{#1}{^{<}}{#2}}
\newcommand{\tonel}[1]{().{#1}}
\newcommand{\Sp}{{\it Sp}}
\newcommand{\texistsr}[2]{{#1},{#2}}
\newcommand{\texistsl}[2]{{#1}.{#2}}
\newcommand{\tforallr}[2]{[{#1}].{#2}}
\newcommand{\tforalll}[2]{[{#1}]; {#2}}
\newcommand{\tunifr}{\textsc{refl}}

\newcommand{\spi}{{\it sp}}
\newcommand{\lfnil}{\lf{()}}
\newcommand{\lflam}[2]{\lf{\lambda {#1}.{#2}}}
\newcommand{\lfapp}[2]{\lf{{#1};{#2}}}
\newcommand{\lfroot}[2]{{{#1} \cdot \lf{#2}}}
\newcommand{\lfpi}[3]{\Pi{\lf{{#1}}{:}{#2}.{#3}}}

\newcommand{\susp}[1]{\langle {#1} \rangle}

\newcommand{\ofirstseq}[5]{{#1};{#2};{#3};{#4} \altv {#5} \mathstrut}
\newcommand{\oseq}[4]{{#1};{#2};{#3} \altv {#4} \mathstrut}
\newcommand{\oiseq}[2]{\oseq{#1}{\cdot}{/{#2}/}{\islvl{#2}}}
\newcommand{\orseq}[4]{\oseq{#1}{#2}{{#3}}{\islvl{#4}}}
\newcommand{\orfseq}[4]{\ofirstseq{\Psi}{#1}{#2}{{#3}}{\islvl{#4}}}
\newcommand{\otseq}[4]{\oseq{#1}{#2}{{#3}}{\isconc{#4}}}
\newcommand{\olseq}[5]{\oseq{#1}{#2}{{#3}/{#4}/{#5}}{U}}
\newcommand{\olfseq}[5]{\ofirstseq{\Psi}{#1}{#2}{{#3}/{#4}/{#5}}{U}}
\newcommand{\opseq}[4]{\oseq{#1}{#2}{{#3},{#4}}{U}}
\newcommand{\opfseq}[4]{\ofirstseq{\Psi}{#1}{#2}{{#3},{#4}}{U}}

\newcommand{\mkunit}{\cdot}
\newcommand{\matchunit}{\cdot}
\newcommand{\mkconj}[2]{{#1},{#2}}
\newcommand{\matchconj}[2]{{#1},{#2}}

\newcommand{\invoff}[2]{{#1}\{{#2}\mbox\}}
\newcommand{\tackon}[2]{{#1}\{{#2}\}}
\newcommand{\frameoff}[2]{{#1}\mbox{$\{\hspace{-4pt}\{$}{#2}\mbox{$\}\hspace{-4pt}\}$}}

\newcommand{\tackonstart}{\{}
\newcommand{\tackonstop}{\}}
\newcommand{\frameoffstart}{\mbox{$\{\hspace{-4pt}\{$}}
\newcommand{\frameoffstop}{\mbox{$\}\hspace{-4pt}\}$}}

\newcommand{\urfoc}[3]{{#1};{#2} \longrightarrow [{#3}] \mathstrut}
\newcommand{\ulfoc}[4]{{#1};{#2} \,[#3] \longrightarrow {#4} \mathstrut}
\newcommand{\uinv}[4]{{#1};{#2};{#3} \longrightarrow {#4} \mathstrut}

\newcommand{\stableR}[1]{{#1}\,\mathit{stable_R} \mathstrut}
\newcommand{\stableL}[1]{{#1}\,\mathit{stable_L} \mathstrut}

\newcommand{\etana}[2]{\eta_{#2}({#1})}
\newcommand{\etapa}[3]{\eta_{#3}({#1}.{#2})}

\newcommand{\subst}[2]{{#1} \circ {#2}}
\newcommand{\rsubsta}[4]{\llbracket {#1}/{#2} \rrbracket^{#4} {#3}}
\newcommand{\lsubsta}[3]{\llbracket {#1} \rrbracket^{#3} {#2}}
\newcommand{\rsubst}[3]{\rsubsta{#1}{#2}{#3}{}}
\newcommand{\lsubst}[2]{\lsubsta{#1}{#2}{}}

\newcommand{\arb}{\mathbin{\mbox{$\ocircle$\hspace{-7pt}{\footnotesize ?}\hspace{3pt}}}}
\newcommand{\softinterp}[1]{{{\ulcorner{{#1}}\urcorner}}}
\newcommand{\interp}[1]{{\no{\ulcorner\obj{{#1}}\urcorner}}}
\newcommand{\ctxinterp}[1]{\no{\mbox{$\ulcorner\!\!\ulcorner$}\obj{{#1}}\mbox{$\urcorner\!\!\urcorner$}}}

\newcommand{\transop}[1]{{\it Op}({#1})}

\newcommand{\opbasic}[3]{\llbracket {#1} \rrbracket({\sf a},\lf{#2},\lf{#3})}
\newcommand{\opsubst}[1]{{{#1}^\dag}}

\newcommand{\tlet}[2]{\{{\sf let}\,{#1}\,{\sf in}\, {#2} \}}
\newcommand{\tstep}[3]{\{{#1}\} \leftarrow \tfocusl{#2}{#3}}
\newcommand{\trstep}[2]{\{{#1}\} \leftarrow {#2}}
\newcommand{\mkpat}[1]{{\it pat}({#1})}
\newcommand{\emptytrace}{\diamond}

\newcommand{\subord}{\mathcal R}


\newcommand{\siggen}{\Sigma_{\it Gen}}
\newcommand{\siggenorder}{\Sigma_{\it Gen\ref{fig:gen-order}}}
\newcommand{\siggenordertp}{\Sigma_{\it Gen\ref{fig:gen-ordertp}}}
\newcommand{\siggenstate}{\Sigma_{\it Gen\ref{fig:gen-state}}}
\newcommand{\siggendests}{\Sigma_{\it Gen\ref{fig:gen-destinations}}}
\newcommand{\siggenletcc}{\Sigma_{\it Gen\ref{fig:gen-letcc2}}}

\draftstamp{\today}{DRAFT}

% \usepackage{titlesec,minitoc}  
% \titleclass{\part}{top}
% \titleformat{\part}
% {\centering\normalfont\Huge\bfseries} 
% {Foo} 
% {0pt} 
% {Bar}


\begin {document} 
\frontmatter

%initialize page style, so contents come out right (see bot) -mjz
\pagestyle{empty}

\title{ %% {\it \huge Thesis Proposal}\\
{\bf Substructural Logical Specifications}}
\author{Robert J. Simmons}
\date{The Future}
\Year{Year In The Future}
\trnumber{CMU-CS-THE-FUTURE}

\committee{
Frank Pfenning, Chair \\
Robert Harper \\
Andr{\'e} Platzer \\
Iliano Cervesato, Carnegie Mellon Qatar \\
Dale Miller, INRIA-Saclay \& LIX/Ecole Polytechnique
}

\support{}
\disclaimer{}

% copyright notice generated automatically from Year and author.
% permission added if \permission{} given.

\keywords{Stuff, More Stuff}

\maketitle

% XXX MAKE DEDICATION
% \begin{dedication}
% For my dog
% \end{dedication}

\pagestyle{plain} % for toc, was empty

%% Obviously, it's probably a good idea to break the various sections of your thesis
%% into different files and input them into this file...

% XXX MAKE ABSTRACT
% \begin{abstract}
% A short summary.
% \end{abstract}

% XXX MAKE ACKNOLWEDGEMENTS
% \begin{acknowledgments}
% My advisor is cool.
% \end{acknowledgments}



\tableofcontents
\listoffigures % XXX Do I want this?
% \listoftables XXX Probably don't need this at all.

\mainmatter

%% Double space document for easy review:
%\renewcommand{\baselinestretch}{1.66}\normalsize

% The other requirements Catherine has:
%
%  - avoid large margins.  She wants the thesis to use fewer pages, 
%    especially if it requires colour printing.
%
%  - The thesis should be formatted for double-sided printing.  This
%    means that all chapters, acknowledgements, table of contents, etc.
%    should start on odd numbered (right facing) pages.
%
%  - You need to use the department standard tech report title page.  I
%    have tried to ensure that the title page here conforms to this
%    standard.
%
%  - Use a nice serif font, such as Times Roman.  Sans serif looks bad.
%
% Other than that, just make it look good...


% Introduction
\chapter{Introduction}


\section{Substructural logical specifications}



\section{Substructural operational semantics}
\label{sec:intro-ssos}

Abstract machine rules

\subsection{Modular and non-modular specification}
\label{sec:modularnonmodular}

(This can mostly come out of the thesis proposal, but use 
natural semantics instead of SOS.)

Natural semantics

Not modular (state), underspecified (parallel or not).


Add state in a modular way.

\section{Invariants in substructural logic}


\part{Substructural logic}

% Linear logic

\chapter{Linear logic}

Logic as it has been traditionally understood and studied -- both in
its classical and intuitionistic varieties -- treats the truth of a
proposition as a {\it persistent resource}. That is, if we have
evidence for the truth of a proposition, we can ignore that evidence
if it is not needed and reuse the evidence as many times as we need
to. Throughout this thesis, ``logic as it has been traditionally
understood as studied'' will be referred to as {\it persistent} logic
to emphasize this treatment of evidence. 

Linear logic, which was studied and
popularized by Girard \cite{girard87linear},
treats evidence as an {\it ephemeral} resource; the use of an
ephemeral resource consumes it, at which point it is unavailable for
further use.  Linear logic, like persistent logic, comes in classical
and intuitionistic flavors. We will favor intuitionistic linear logic
in part because the propositions of intuitionistic linear logic
(written $A$, $B$, $C$, \ldots) have a more natural correspondence
with our physical intuitions about consumable resources. Linear
conjunction $A \tensor B$ represents the resource built from the
resources $A$ and $B$; if you have both a bowl of soup {\it and} a
sandwich, that resource can be represented by the proposition ${\sf
  soup} \otimes {\sf sandwich}$. Linear implication $A \lolli B$
represents a resource that can interact with another resource $A$ to
produce a resource $B$. One robot with batteries not included could be
represented as the linear resource $({\sf battery} \lolli {\sf
  robot})$, and the linear resource $({\sf 6bucks} \lolli {\sf soup}
\tensor {\sf sandwich})$ represents the ability to use \$6 to obtain
lunch -- but only once!\footnote{Conjunction will always bind more
  tightly than implication, so this is equivalent to the proposition
  ${\sf 6bucks} \lolli ({\sf soup} \tensor {\sf sandwich})$.} Linear
logic also has a modal connective ${!}A$ representing a persistent
resource that can be
used to generate any number of $A$ resources, including zero. The
Panera ``You Pick Two'' menu might be represented as
\[ {!}({\sf 6bucks} \lolli {\sf soup} \tensor {\sf sandwich}) \otimes
{!}({\sf 6bucks} \lolli {\sf soup} \tensor {\sf salad}) \otimes
{!}({\sf 6bucks} \lolli {\sf sandwich} \tensor {\sf salad}),\] as the
menu gives you the opportunity to exchange six dollars for two
distinct members of the set $\{ {\sf soup}, {\sf salad}, {\sf
  sandwich} \}$ any number of times.

\begin{figure}[t]
\begin{tabbing}
\quad $A$ \,\, \=  $::= p \mid {!}A \mid A \lolli B \mid \one \mid A \tensor B$\\
\quad $\Gamma$ \> $::= \cdot \mid \Gamma, A$ \qquad \= {\it (multiset)}\\
\quad $\Delta$ \> $::= \cdot \mid \Delta, A$ \> {\it (multiset)}\\
\end{tabbing}
%
%
\quad \fbox{$\seq{\Gamma}{\Delta}{A}$}
\[
\infer[{\it init}]
{\seq{\Gamma}{p}{p}}
{}
\qquad
\infer[{\it copy}]
{\seq{\Gamma, A}{\Delta}{C}}
{\seq{\Gamma, A}{\Delta, A}{C}}
%
\]

\[
%
\infer[{!}_R]
{\seq{\Gamma}{\cdot}{{!}A}}
{\seq{\Gamma}{\cdot}{A}}
\qquad
\infer[{!}_L]
{\seq{\Gamma}{\Delta, {!}A}{C}}
{\seq{\Gamma, A}{\Delta}{C}}
\qquad
\infer[\one_R]
{\seq{\Gamma}{\cdot}{\one}}
{}
\qquad
\infer[\one_L]
{\seq{\Gamma}{\Delta, \one}{C}}
{\seq{\Gamma}{\Delta}{C}}
\]

\[
%
\infer[{\tensor}_R]
{\seq{\Gamma}{\Delta_1,\Delta_2}{A \tensor B}}
{\seq{\Gamma}{\Delta_1}{A}
 &
 \seq{\Gamma}{\Delta_2}{B}}
\qquad
\infer[{\tensor}_L]
{\seq{\Gamma}{\Delta, A \tensor B}{C}}
{\seq{\Gamma}{\Delta, A, B}{C}}
\]

\[
%
\infer[{\lolli}_R]
{\seq{\Gamma}{\Delta}{A \lolli B}}
{\seq{\Gamma}{\Delta, A}{B}}
\qquad
\infer[{\lolli}_L]
{\seq{\Gamma}{\Delta_1,\Delta_2, A \lolli B}{C}}
{\seq{\Gamma}{\Delta_1}{A}
 &
 \seq{\Gamma}{\Delta_2, B}{C}}
%
\]
\caption{Intuitionstic linear logic}
\label{fig:linear}
\end{figure}


Figure~\ref{fig:linear} presents a standard sequent calculus for
linear logic, in particular the so-called {\it multiplicative,
  exponential} fragment of intuitionistic linear logic (or {\it
  MELL}). It corresponds most closely to Barber's dual intuitionistic
linear logic \cite{barber96dual}, but also to Andreoli's dyadic system
\cite{andreoli92logic} and Chang et al.'s judgmental analysis of
intuitionistic linear logic \cite{chang03judgmental}.

\subsection*{Transitions in linear logic}

The propositions of intuitionistic linear logic, and linear implication
in particular, capture a notion of state change: we can {\it
  transition} from a state where we have both a ${\sf battery}$ and
the battery-less robot (represented, as before, by the linear
implication ${\sf battery} \lolli {\sf robot}$) to a state where we
have the battery-endowed (and therefore presumably functional) robot
(represented by the proposition ${\sf robot}$). In other words, the
proposition
%
\[{\sf battery} \otimes ({\sf battery} \lolli {\sf robot}) \lolli
{\sf robot}\] 
%
is provable in linear logic. These transitions can be chained
together as well: if we start out with ${\sf
  6bucks}$ instead of ${\sf battery}$ but we also have the
persistent ability to turn ${\sf 6bucks}$ into a ${\sf battery}$ --
just like we turned \$6 into a bowl of soup and a salad at Panera --
then we can ultimately get our working robot as well.
Written as a series of transitions, the picture looks like this:
\[
\begin{array}{ccccc}
\begin{array}{c}
\mbox{\it \$6 (1)}\medskip\\ 
\mbox{\it battery-free robot (1)} \medskip\\ 
\mbox{\it turn \$6 into a battery}\\
\mbox{\it (all you want)}
\end{array}
& \leadsto &
\begin{array}{c}
\mbox{\it battery  (1)}\medskip\\ 
\mbox{\it battery-free robot (1)} \medskip\\ 
\mbox{\it turn \$6 into a battery}\\
\mbox{\it (all you want)}
\end{array}
& \leadsto &
\begin{array}{c}
\mbox{\it robot (1)} \medskip\\ 
\mbox{\it turn \$6 into a battery}\\
\mbox{\it (all you want)}\medskip\\~\\
\end{array}
\end{array}
\]
In linear logic, these transitions correspond to the provability
of the proposition
\[{!}({\sf 6bucks} \lolli {\sf battery}) \otimes {\sf 6bucks} \otimes
({\sf battery} \lolli {\sf robot}) \lolli {\sf robot}.\] 
A derivation of this proposition is given in
Figure~\ref{fig:unfocused-robot}.\footnote{In Chapter XXX, I will
  argue that this view isn't quite precise enough, and that the most
  natural representation of state change from the state $A$ to the
  state $B$ isn't really captured by derivations of the proposition $A
  \lolli B$ or by derivations of the hypothetical judgment
  $\seq{\cdot}{A}{B}$.  However, this view remains a simple and useful
  one; Cervesato and Scedrov cover it thoroughly in the context of
  intuitionistic linear logic \cite{cervesato09relating}.}  

\begin{figure}
\[
\infer[{\lolli}_R]
{\seq{\cdot}{\cdot}{{!}({\sf 6bucks} \lolli {\sf battery}) \otimes
                    {\sf 6bucks} \otimes 
                    ({\sf battery} \lolli {\sf robot}) \lolli {\sf robot}}}
{\infer[{\otimes}_L]
{\seq{\cdot}{{!}({\sf 6bucks} \lolli {\sf battery}) \otimes
                    {\sf 6bucks} \otimes 
                    ({\sf battery} \lolli {\sf robot})}{{\sf robot}}}
{\infer[{!}_L]
{\seq{\cdot}{{!}({\sf 6bucks} \lolli {\sf battery}),
                    {\sf 6bucks} \otimes 
                    ({\sf battery} \lolli {\sf robot})}{{\sf robot}}}
{\infer[{\otimes}_L]
{\seq{\Gamma}{{\sf 6bucks} \otimes 
                    ({\sf battery} \lolli {\sf robot})}{{\sf robot}}}
{\infer[{\lolli}_L]
{\seq{\Gamma}{{\sf 6bucks}, {\sf battery} \lolli {\sf robot}}{{\sf robot}}}
{\infer[{\it copy}]
 {\seq{\Gamma}{{\sf 6bucks}}{{\sf battery}}}
 {\infer[{\lolli}_L] 
  {\seq{\Gamma}{{\sf 6bucks}, {\sf 6bucks} \lolli {\sf battery}}{{\sf battery}}}
  {\infer[{\it init}]
   {\seq{\Gamma}{{\sf 6bucks}}{{\sf 6bucks}}}
   {}
   &
   \infer[{\it init}]
   {\seq{\Gamma}{{\sf battery}}{{\sf battery}}}
   {}}}
 &
 \infer[{\it init}]
 {\seq{\Gamma}{{\sf robot}}{{\sf robot}}}
 {}}}}}}
\] 
\caption{Proving that a transition is possible 
(where we let $\Gamma = {\sf 6bucks} \lolli {\sf battery}$)}
\label{fig:unfocused-robot}
\end{figure}


It is precisely because linear logic contains this natural notion of
state and state transition that a rich line of work, dating back to
Chirmar's 1995 Ph.D. thesis, has sought to use linear logic as a {\it
  logical framework} for describing stateful systems
\cite{chirimar95proof,cervesato02linear,
  cervesato02concurrent,pfenning04substructural,miller09formalizing,
  pfenning09substructural,cervesato09relating}.  

\subsection*{Logical frameworks}

Generally speaking, logical frameworks use the {\it structure} of
proofs in a logic (like linear logic) to describe the structures we're
really interested in (like the process of obtaining a robot).  There
are two related reasons why linear logic as described in
Figure~\ref{fig:linear} is not immediately useful as a logical
framework. First, the structure of the proof in
Figure~\ref{fig:unfocused-robot} doesn't really match the intuitive
two-step transition that we sketched out above. Second, there are {\it
  lots} of derivations of our example proposition according to the
rules in Figure~\ref{fig:linear}, even though there's only one
``real'' series of transitions that get us to a working robot. The use
of ${!}L$, for instance, could be permuted up past the ${\otimes}L$
and then past the ${\lolli}L$ into the left branch of the proof. These
differences represent inessential nondeterminism in proof construction
or in proof search -- they just get in the way of the structure that
we are trying to capture. 

This is a general problem in the construction of logical frameworks,
and we'll discuss two solutions in the context of LF, a logical
framework based on dependent type theory that has proved to be a
suitable means of encoding a wide variety of deductive systems, such
as logics and programming languages \cite{harper93framework}.  The
first solution is to define an appropriate equivalence class of
proofs, and the second solution is to define an appropriate fragment
of canonical proofs.

Using an appropriate equivalence class of proofs can be an effective
way of defining away the problem of inessential nondeterminism.  In
linear logic as presented above, if the permutability of rules like
${!}_L$ and ${\otimes}_L$ is problematic, we can instead reason about
{\it equivalence classes} of derivations where proofs that differ only
in the ordering of ${!}_L$ and ${\otimes}_L$ rules are treated as
equivalent (that is, as members of the same equivalence class):
\[
\infer[{!}_L]
{\seq{\Gamma}{\Delta, {!}A, B \otimes C}{D}}
{\infer[{\otimes}_L]
 {\seq{\Gamma,A}{\Delta, B \otimes C}{D}}
 {\deduce{\seq{\Gamma,A}{\Delta, B, C}{D}}{\mathcal D}}}
\quad
\deduce{\mathstrut}{\mathstrut{\equiv}}
\quad
\infer[{\otimes}_L]
{\seq{\Gamma}{\Delta, {!}A, B \otimes C}{D}}
{\infer[{!}_L]
 {\seq{\Gamma}{\Delta, {!}A, B, C}{D}}
 {\deduce{\seq{\Gamma,A}{\Delta, B, C}{D}}{\mathcal D}}}
\]

In LF, lambda calculus terms (which correspond to derivations by the
Curry-Howard) are considered modulo the least equivalence class that
includes
\begin{itemize}
\item $\alpha$-equivalence ($\lambda x.N \equiv \lambda y.N[y/x]$ if 
$y \not\in {\it FV}(N)$), 
\item $\beta$-equivalence 
($(\lambda x.\,M)N \equiv M[N/x]$ if $x \not\in {\it FV}(N)$), and 
\item $\eta$-equivalence ($N \equiv \lambda x.N\,x$).
\end{itemize}
The weak normalization property for LF establishes that given any
typed LF term, we can find an equivalent term that is $\beta$-normal
(no $\beta$-redexes of the form $(\lambda x.M) N$ exist) and
$\eta$-long (replacing $N$ with $\lambda x.N\,x$ anywhere would
introduce a $\beta$-redex or make the term ill-typed).  Furthermore,
in any given equivalence class of typed LF terms, all the
$\beta$-normal and $\eta$-long terms are $\alpha$-equivalent.
Therefore, because $\alpha$-equivalence is decidable, the equivalence
of typed LF terms is also decidable. 

The uniqueness of $\beta$-normal and $\eta$-long terms within an
equivalence class of lambda calculus terms (modulo
$\alpha$-equivalence, which we will henceforth take for granted) makes
these terms useful as canonical representatives of equivalence
classes. In Harper, Honsell, and Plotkin's original formulation
of LF, a deductive system was said to be {\it adequately encoded} as
an LF type family in the case that there is a compositional bijection
between the formal objects in the deductive system and these
$\beta$-normal, $\eta$-long representatives of equivalence classes
\cite{harper93framework}.

More modern presentations of LF, such as Harper and Licata's
\cite{harper07mechanizing}, follow the approach developed by Watkins
et al.~\cite{watkins02concurrent} and define the logical framework so
that it only contains these $\beta$-normal, $\eta$-long {\it canonical
  forms} of LF. This presentations of LF is called Canonical LF to
distinguish it from the original presentation of LF in which the
canonical forms are just a subset of the possible terms. A central
component in this approach is {\it hereditary substitution}.
Hereditary substitution also establishes a normalization property for
LF; using hereditary substitution we can easily take a regular LF
term and transform it into a Canonical LF term.\footnote{This process
  is the same as the way we use cut admissibility to prove cut
  elimination.} An oft-overlooked point, which we will return to in
Section~\ref{sec:warning}, is that the normalization theorem we prove
this way is a strictly weaker theorem than so-called weak
normalization.

One analogue to the canonical forms of LF will be the {\it focused
  derivations} of linear logic that are presented in the next
section. In Section~\ref{sec:foclinlog} below, we will present 
focused linear logic and see that there is exactly 
one focused derivation that derives the proposition
\[{!}({\sf 6bucks} \lolli {\sf battery}) \otimes {\sf 6bucks} \otimes
({\sf battery} \lolli {\sf robot}) \lolli {\sf robot}.\] 
%
We will furthermore see that the structure of this derivation matches
the intuitive transition interpretation of the proposition, a point
that is reinforced by the discussion of {\it synthetic inference
  rules} in Section~\ref{sec:linsynthetic}. In
Section~\ref{sec:linhack}, we argue that our focused system, while it
may be the most natural one for linear logic, does not precisely meet
the demands we will place upon it. This, in turn, motivates a
discussion of notation (Section~\ref{sec:linnote}) 
which we will continue in the next chapter.

\section{Focused linear logic}
\label{sec:foclinlog}

Andreoli's original motivation for introducing focusing was not to
describe a logical framework, it was to describe a paradigm of logic
programming based on proof search in classical linear logic
\cite{andreoli92logic}. The existence of multiple proofs that differ
in inessential ways is particularly problematic for proof search, as
inessential differences between derivations correspond to unnecessary
choice points that a proof search procedure will need to backtrack
over. The presentation of focusing for intuitionistic linear logic in
this section most closely resembles Chaudhuri's focused intuitionistic
linear logic \cite{chaudhuri06focused} and my presentation of
polarized intuitionistic persistent logic
\cite{simmons11structural}. The major exception is the treatment of
asynchronous rules as confluent rather than fixed and arbitrary
(discussed in Section~\ref{sec:confluent-v-fixed}).

\subsection{Polarization}
\label{sec:linpolar}

The first step in describing a focused sequent calculus is to classify
connectives into two groups.  Some connectives, such as linear
implication $A \lolli B$, are called {\it asynchronous} because their
right rules can always be applied eagerly, without backtracking,
during bottom-up proof search. Other connectives, such as disjunction
$A \tensor B$, are called {\it synchronous} because their right rules
cannot be applied eagerly. For instance, if we are trying to prove
$\seq{\Gamma}{A \tensor B}{B \tensor A}$, the ${\tensor}R$ rule cannot
be applied eagerly; we first have to decompose $A \tensor B$ on the
left using the ${\tensor}L$ rule.\footnote{Andreoli dealt with a
  one-sided classical sequent calculus; in intuitionistic logics, it
  is common to call asynchronous connectives {\it right}-asynchronous
  and {\it left}-synchronous. Similarly, it is common to call
  synchronous connectives {\it right}-synchronous and {\it
    left}-asynchronous.

  Synchronicity, a property of connectives, is closely connected to
  (and sometimes conflated with) a property of rules called {\it
    invertibility}; a rule is invertible if the conclusion of the rule
  implies the premises. So ${\lolli}R$ is invertible
  ($\seq{\Gamma}{\Delta}{A \lolli B}$ implies $\seq{\Gamma}{\Delta,
    A}{B}$) but ${\lolli}L$ is not ($\seq{\Gamma}{\Delta, A \lolli
    B}{C}$ does not imply that $\Delta = \Delta_1, \Delta_2$ such that
  $\seq{\Gamma}{\Delta_1}{A}$ and $\seq{\Gamma}{\Delta_2, B}{C}$).
  Rules that can be applied eagerly need to be invertible, so
  asynchronous connectives have invertible right rules and synchronous
  connectives have invertible left rules. Therefore, another synonym
  for asynchronous/negative is {\it right-invertible}, and another
  synonym for synchronous/positive is {\it left-invertible}.}  
The nontrivial result of focusing is that it is possible to separate a
proof into phases: inversion phases in which all asynchronous rules
are applied exhaustively, and focused phases where synchronous rules
are applied repeatedly and exhaustively to a single proposition (the
proposition {\it in focus}). 

We call the asynchronous connectives {\it negative} ($\lolli$, $\top$
and $\with$ in full propositional linear logic) and call the
synchronous connectives {\it positive} ($\zero$, $\oplus$, $\one$, and
$\otimes$ in full propositional linear logic). Each atomic proposition
must be assigned to be either positive or negative, though this
assignment can be arbitrary. At this point, there is an important
choice to make. One way forward is to treat positive and negative
propositions as a syntactic refinements of all propositions, in which
case we end up focusing a standard intuitionistic linear logic. The
other way forward is to treat positive and negative propositions as
distinct syntactic classes $A^+$ and $A^-$ with explicit inclusions
between them. In this second case, we end up focusing a {\it
  polarized} linear logic.  These inclusions are traditionally called
{\it shifts}. The positive proposition ${\downarrow}A^-$, pronounced
``downshift $A$,'' has a subterm that is a negative proposition; the
negative proposition ${\uparrow}A^+$, pronounced ``upshift $A$,'' has
a subterm that is a positive proposition.

The choice doesn't make a large difference for our purposes.
Polarized logics are interesting, and polarized linear logic is a bit
more expressive than regular linear logic, as heavily-shifted
propositions like ${\uparrow}{\downarrow}{\uparrow}{\downarrow}A^-$
can be expressed. This extra expressiveness won't help us in
the design of logical frameworks, but the use of shifts is helpful
when explaining identity expansion in Section~\ref{sec:linindentity}, 
so we will focus a polarized linear logic with shifts.

\begin{figure}
{\small \[
\begin{array}{rcl|rcl|rcl}
({\downarrow}A^-)^\circ & \!\!\!=\!\!\! & (A^-)^\circ & & & & & & 
\\
(p^+)^\circ & \!\!\!=\!\!\! & p^+ &
(p^+)^\oplus & \!\!\!=\!\!\! & p^+ &
(p^+)^\ominus & \!\!\!=\!\!\! & {\uparrow}p^+
\\
({!}A^-)^\circ & \!\!\!=\!\!\! & {!}(A^-)^\circ &
({!}A)^\oplus & \!\!\!=\!\!\! & {!}A^\ominus &
({!}A)^\ominus & \!\!\!=\!\!\! & {\uparrow}({!}A^\ominus)
\\
(\one)^\circ & \!\!\!=\!\!\! & \one &
(\one)^\oplus & \!\!\!=\!\!\! & \one &
(\one)^\ominus & \!\!\!=\!\!\! & {\uparrow}\one 
\\
(A^+ \otimes B^+)^\circ & \!\!\!=\!\!\! & (A^+)^\circ \otimes (B^+)^\circ &
(A \otimes B)^\oplus & \!\!\!=\!\!\! & A^\oplus \otimes B^\oplus &
(A \otimes B)^\ominus & \!\!\!=\!\!\! & {\uparrow}(A^\oplus \otimes B^\oplus)
\\
({\uparrow}A^+)^\circ & \!\!\!=\!\!\! & (A^+)^\circ & & & & & & 
\\
(p^-)^\circ & \!\!\!=\!\!\! & p^- &
(p^-)^\oplus & \!\!\!=\!\!\! & {\downarrow}p^- &
(p^-)^\ominus & \!\!\!=\!\!\! & p^- 
\\
(A^+ \lolli B^-)^\circ & \!\!\!=\!\!\! & (A^+)^\circ \lolli (B^-)^\circ &
(A \lolli B)^\oplus & \!\!\!=\!\!\! & {\downarrow}(A^\oplus \lolli B^\ominus) &
(A \lolli B)^\ominus & \!\!\!=\!\!\! & A^\oplus \lolli B^\ominus
\end{array}
\]}

\caption{De-polarizing and polarizing (with minimal shifts) propositions of MELL}
\label{fig:lin-shift}
\end{figure}


The relationship between unpolarized and polarized linear logic is
given by two erasure functions $(A^+)^\circ$ and $(A^-)^\circ$ that
wipe away all the shifts; this function is defined in
Figure~\ref{fig:lin-shift}. While shifts turn out to have a profound
impact on the structure of focused proofs, they are intended to have
no impact on provability. Therefore, the strongest statement of the
correctness of focusing is based on erasure: there is an unfocused
derivation of $(A^-)^\circ$ if and only if there is a focused
derivation of $A^-$.\footnote{I chose $A^-$ only to be brief; the
  condition that $(A^+)^\circ$ is derivable iff $A^+$ is could, of
  course, be added.}  However, most proofs of the correctness of
focusing prove a weaker property. Every proposition in linear logic
has an obvious polarized analogue with a minimal number of shifts;
this analogue is formalized as the two functions $A^\oplus$ and
$A^\ominus$ in Figure~\ref{fig:lin-shift}. Note that both of these
functions are partial inverses of erasure: $(A^\oplus)^\circ =
(A^\ominus)^\circ = A$. Almost all proofs of the correctness of
focusing work on the basis of these partial inverses, which we call
{\it polarization strategies}, establishing that there is an unfocused
proof of $A$ if and only if there is a focused proof of
$A^\ominus$.\footnote{The two exceptions are Zeilberger completeness
  proof in classical persistent logic \cite{zeilberger08unity} and my
  proof in intuitionistic persistent logic \cite{simmons11structural}.}
The weaker formulation is sufficient for our current purposes, so we
will discuss the weaker property of polarization-strategy-based
correctness, not erasure-based correctness.

\subsection{Sequent calculus}

Usually, focused logics are described as
having multiple judgments:
\begin{itemize}
\item $\mildrfoc{\Gamma}{\Delta}{A^+}$ (the {\it right focus} sequent, where
the proposition $A^+$ is in focus),
\item $\mildinv{\Gamma}{\Delta}{C}$ (the {\it inversion} sequent), and
\item $\mildlfoc{\Gamma}{\Delta}{A^-}{C}$ (the {\it left focus} sequent,
where the proposition $A^-$ is in focus).
\end{itemize}
Another reasonable presentation of linear logic uses only one sequent
$\mildseq{\Gamma}{\Delta}{U}$, but generalizes what is to allowed to
to appear in the linear context $\Delta$ or in the succeedant, which
we write $U$. We will use this interpretation to understand the logic
described in Figure~\ref{fig:kaustuv-focused}.

\begin{figure}[t]
\begin{tabbing}
\quad $A^+$ \= $::= p^+ 
              \mid {\downarrow}A^- 
              \mid {!}A^- 
              \mid \one
              \mid A \otimes B$\\
\quad $A^-$ \> $::= p^-
              \mid {\uparrow}A^+
              \mid A \lolli B$\\
\quad $\Gamma$ \> $::= \cdot \mid \Gamma, A^-$ \qquad\qquad\qquad\qquad\qquad\qquad\quad \= {\it (multiset)}\\
\quad $\Delta$ \> $::= \cdot \mid \Delta, A^+ \mid \Delta, A^- \mid \Delta, [A^-] \mid \Delta, \langle A^+ \rangle$ \> {\it (multiset)}\\
\quad $U$ \> $::= A^- \mid A^+ \mid [ A^+ ] \mid \langle A^- \rangle$\\
\end{tabbing}
%
%
\quad \fbox{$\mildseq{\Gamma}{\Delta}{U}$}
\[
\infer[{\it focus}^*_R]
{\mildseq{\Gamma}{\Delta}{A^+}}
{\mildseq{\Gamma}{\Delta}{[A^+]}}
\quad
\infer[{\it focus}^*_L]
{\mildseq{\Gamma}{\Delta,A^-}{U}}
{\mildseq{\Gamma}{\Delta,[A^-]}{U}}
\quad
\infer[{\it copy}^*]
{\mildseq{\Gamma, A^-}{\Delta}{U}}
{\mildseq{\Gamma, A^-}{\Delta, [A^-]}{U}}
\]

\[
\infer[\eta^+]
{\mildseq{\Gamma}{\Delta, p^+}{U}}
{\mildseq{\Gamma}{\Delta, \langle p^+ \rangle}{U}}
\quad
\infer[{\it id}^+]
{\mildseq{\Gamma}{\langle A^+ \rangle}{[A^+]}}
{}
\quad
\infer[\eta^-]
{\mildseq{\Gamma}{\Delta}{p^-}}
{\mildseq{\Gamma}{\Delta}{\langle p^- \rangle}}
\quad
\infer[{\it id}^-]
{\mildseq{\Gamma}{[A^-]}{\langle A^- \rangle}}
{}
\]

\[
\infer[{\uparrow}_R]
{\mildseq{\Gamma}{\Delta}{{\uparrow}A^+}}
{\mildseq{\Gamma}{\Delta}{A^+}}
\quad
\infer[{\uparrow}_L]
{\mildseq{\Gamma}{\Delta, [{\uparrow}A^+]}{U}}
{\mildseq{\Gamma}{\Delta, A^+}{U}}
\quad
\infer[{\downarrow}_R]
{\mildseq{\Gamma}{\Delta}{[{\downarrow}A^-]}}
{\mildseq{\Gamma}{\Delta}{A^-}}
\quad
\infer[{\downarrow}_L]
{\mildseq{\Gamma}{\Delta, {\downarrow}A^-}{U}}
{\mildseq{\Gamma}{\Delta, A^-}{U}}
\]

\[
%
\infer[{!}_R]
{\mildseq{\Gamma}{\cdot}{[{!}A^-]}}
{\mildseq{\Gamma}{\cdot}{A^-}}
\quad
\infer[{!}_L]
{\mildseq{\Gamma}{\Delta, {!}A^-}{U}}
{\mildseq{\Gamma, A^-}{\Delta}{U}}
\quad
\infer[\one_R]
{\mildseq{\Gamma}{\cdot}{[\one]}}
{}
\quad
\infer[\one_L]
{\mildseq{\Gamma}{\Delta, \one}{U}}
{\mildseq{\Gamma}{\Delta}{U}}
\]

\[
%
\infer[{\tensor}_R]
{\mildseq{\Gamma}{\Delta_1,\Delta_2}{[A^+ \tensor B^+]}}
{\mildseq{\Gamma}{\Delta_1}{[A^+]}
 &
 \mildseq{\Gamma}{\Delta_2}{[B^+]}}
\quad
\infer[{\tensor}_L]
{\mildseq{\Gamma}{\Delta, A^+ \tensor B^+}{U}}
{\mildseq{\Gamma}{\Delta, A^+, B^+}{U}}
\]

\[
%
\infer[{\lolli}_R]
{\mildseq{\Gamma}{\Delta}{A^+ \lolli B^-}}
{\mildseq{\Gamma}{\Delta, A^+}{B^-}}
\quad
\infer[{\lolli}_L]
{\mildseq{\Gamma}{\Delta_1,\Delta_2, [A^+ \lolli B^-]}{U}}
{\mildseq{\Gamma}{\Delta_1}{[A^+]}
 &
 \mildseq{\Gamma}{\Delta_2, [B^-]}{U}}
%
\]
\caption{Focused intuitionstic linear logic.}
\label{fig:kaustuv-focused}
\end{figure}


By adding a side condition to the three rules ${\it focus}_R$, ${\it
  focus}_L$, and ${\it copy}$ that neither the context $\Delta$ nor
the succeedant $U$ can contain an in-focus proposition $[A^+]$ or
$[A^-]$, derivations can maintain the invariant that there is always
at most one proposition in focus, effectively restoring the situation
in which there are three distinct judgments.  This restriction alone
gives us what Pfenning calls a {\it chaining} logic
\cite{pfenning02chaining} and which Laurent calls a {\it weakly
  focused} logic \cite{laurent04proof}.\footnote{This is not what I
  called a weakly focused logic \cite{simmons09weak}. That weakly
  focused system had an additional restriction that invertible rules
  could not be applied when any other proposition was in focus; this
  corresponded to what Laurent called a strongly $+$-focused logic.}
We obtain a fully focused logic by further restricting these three
rules so that they only apply when the sequent below the line is {\it
  stable}.  A sequent $\mildseq{\Gamma}{\Delta}{U}$ is stable if the
context $\Delta$ contains only negative propositions $A^-$ and
suspended positive propositions $\langle A^+ \rangle$ and the
succeedant $U$ is either a positive proposition $A^+$ or a suspended
negative proposition $\langle A^- \rangle$. 

We will now turn our attention to the meaning of these suspended
propositions and the four rules that interact with them: ${\it id}^+$,
${\it id}^-$, $\eta^+$, and $\eta^-$.

\subsection{Suspended propositions}

In unfocused sequent calculi, such as the one for linear logic in
Figure~\ref{fig:linear}, initial sequents are restricted to atomic
propositions. All sequent calculi, focused or unfocused, have the
subformula property: every rule breaks down a proposition, either on
the left or the right. Since the logical interpretation of atomic
propositions is that they are stand-ins for unknown propositions, we
are unable to break them down any further. We are therefore only able
to derive an atomic conclusion or use an atomic premise with the {\it
  init} rule that concludes $\seq{\Gamma}{p}{p}$ and has no premises.
This {\it init} rule is the only instance of the admissible identity
theorem $\seq{\Gamma}{A}{A}$ that must be explicitly included as a
proof rule. If we substitute in a proposition for an atomic
propositions, the structure of the proof stays exactly the same,
except that instances of initial sequents become admissible instances
of the general identity theorem.

To my knowledge, all published proof systems for focused logic have
attempted to replicate this initial rule {\it init}. This is a design
error, and it is one that has historically made it enormously (and
unnecessarily) difficult to prove the identity theorem for focused
systems. Our presentation uses {\it suspensions}: suspended positive
propositions $\langle A^+ \rangle$ only appear in the linear context
$\Delta$, and suspended negative propositions $\langle A^- \rangle$
only appear as succeedants. They treated as stable (we never break
down a suspended proposition) and are only used to immediately
prove a proposition in focus with one of the identity rules
${\it id}^+$ or ${\it id}^-$.

Suspended positive propositions act much like regular variables in a
natural deduction system. The positive identity rule ${\it id}^+$
allows us to prove any positive proposition given that positive
proposition appears suspended in the context.  There is a
corresponding substitution principle for focal substitutions that has
a natural-deduction-like flavor: we can substitute a derivation
right-focused on $A^+$ for a suspended positive proposition $\langle
A^+ \rangle$ in a context.

\bigskip
\begin{theorem}[Focal substitution (positive)]\label{thm:fsubst-pos}~\\
For stable $\Delta$,
if $\mildseq{\Gamma}{\Delta}{[A^+]}$ 
and $\mildseq{\Gamma}{\Delta', \langle A^+ \rangle}{U}$, 
then $\mildseq{\Gamma}{\Delta', \Delta}{U}$.
\end{theorem}

\begin{proof}
  Straightforward induction over the second given derivation, as in a
  proof of regular substitution in a natural deduction system. If the
  second derivation is the axiom ${\it id}^+$, the result follows
  immediately using the first given derivation.
\end{proof}

\noindent
Note that, in the statement of Theorem~\ref{thm:fsubst-pos}, the
second premise $\mildseq{\Gamma}{\Delta', \langle A^+ \rangle}{U}$ may
be a right-focused sequent $\mildseq{\Gamma}{\Delta', \langle A^+
  \rangle}{[B^+]}$, a left-focused sequent $\mildseq{\Gamma}{\Delta'',
  [B^-], \langle A^+ \rangle}{U}$, or an inverting sequent. 

Suspended negative propositions are a bit weirder. While a derivation
of $\mildseq{\Gamma}{\Delta', \langle A^+ \rangle}{U}$ is missing a
premise that can be satisfied by a derivation of
$\mildseq{\Gamma}{\Delta}{[A^+]}$, a derivation of 
$\mildseq{\Gamma}{\Delta}{\langle A^- \rangle}$ is missing a 
{\it continuation} that can be satisfied by a derivation of
$\mildseq{\Gamma}{\Delta', [A^-]}{U}$. The focal substitution principle,
however, still takes the basic form of a substitution principle.

\bigskip
\begin{theorem}[Focal substitution (negative)]\label{thm:fsubst-neg}~\\
For stable $\Delta'$ and $U$, 
if $\mildseq{\Gamma}{\Delta}{\langle A^- \rangle}$
and $\mildseq{\Gamma}{\Delta', [A^-]}{U}$, 
then $\mildseq{\Gamma}{\Delta', \Delta}{U}$. 
\end{theorem}

\begin{proof}
  Straightforward induction over the {\it first} given derivation; if
  the first derivation is the axiom ${\it id}^-$, the result follows
  immediately using the second given derivation.
\end{proof}

\noindent
As a regular substitution principle that is inductive over the structure
of the first given proposition, focal substitution is reminiscent of 
the {\it leftist substitutions} introduced by Pfenning and Davies in the 
context of the possibility modality \cite{pfenning01judgmental}.

Unlike cut admissibility, which we discuss in Section~\ref{sec:lincut}, both
of the focal substitution principles are straightforward inductions
over the structure of the derivation containing the suspended
proposition. In the development of structural focalization, I discuss
how, in a focused presentation of persistent intuitionistic logic that
is encoded in LF, a suspended positive premise can be encoded as a
hypothetical right focus. This encoding makes the ${\it id}^+$ rule an
instance of the hypothesis rule provided by LF and establishes
Theorem~\ref{thm:fsubst-pos} ``for free'' as an instance of LF
substitution. This is possible to do for negative focal substitution
as well, but it is somewhat counter-intuitive
\cite{simmons11structural}.

The two substitution
principles can be phrased as admissible rules for building derivations,
which we indicate using a dashed line:
\[
\infer-[{\it subst}^+]
{\mildseq{\Gamma}{\Delta', \Delta}{U}}
{\mildseq{\Gamma}{\Delta}{[A^+]}
 &
 \mildseq{\Gamma}{\Delta', \langle A^+ \rangle}{U}}
\qquad
\infer-[{\it subst}^-]
{\mildseq{\Gamma}{\Delta', \Delta}{U}}
{\mildseq{\Gamma}{\Delta}{\langle A^- \rangle}
 &
 \mildseq{\Gamma}{\Delta', [A^-]}{U}}
\]

\subsection{Identity expansions}
\label{sec:linindentity}

Suspended propositions appear in Figure~\ref{fig:kaustuv-focused} in
two places: first in the identity rules that we have just discussed
and connected with the focal substitution principles, and second in
the rules marked $\eta^+$ and $\eta^-$, which are also the only
mention of atomic propositions in the presentation. It is here that we
need to make an absolutely critical shift of perspective from
unfocused to focused logic. In an unfocused logic, the rules
nondeterministically break down propositions, and the initial rule {\it
  init} puts an end to this process when an atomic proposition is
reached. In a focused logic, the focus and inversion phases {\it must}
break down a proposition all the way until a shift is reached. The two
$\eta$ rules are what put an end to this when an atomic proposition is
reached, and they correspond to the two ${\it id}$ rules that allow
these necessarily suspended propositions to successfully conclude a
right or left focus.

\begin{figure}
{\small 
\[
\infer-[\eta^+]
{\mildseq{\Gamma}{\Delta, {\downarrow}A}{U}}
{\deduce
 {\mildseq{\Gamma}{\Delta, \langle {\downarrow}A \rangle}{U}}
 {\mathcal D}}
\quad
\deduce{\mathstrut}{\Longrightarrow}
\quad
\infer[{\downarrow}_L]
{\mildseq{\Gamma}{\Delta, {\downarrow}A}{U}}
{\infer-[{\it subst}^+]
 {\mildseq{\Gamma}{\Delta, A}{U}}
 {\infer[{\downarrow}_R]
  {\mildseq{\Gamma}{A}{[ {\downarrow}A ]}}
  {\infer-[{\eta}^-]
   {\mildseq{\Gamma}{A}{A}}
   {\infer-[{\it focus}_L]
    {\mildseq{\Gamma}{A}{\langle A \rangle}}
    {\infer[{\it id}^-]
     {\mildseq{\Gamma}{[ A ]}{\langle A \rangle}}
     {}}}}
  &
  \deduce
  {\mildseq{\Gamma}{\Delta, \langle {\downarrow}A \rangle}{U}}
  {\mathcal D}}}
\]

\[
\infer-[\eta^+]
{\mildseq{\Gamma}{\Delta,{!}A}{U}}
{\deduce
 {\mildseq{\Gamma}{\Delta, \langle {!}A \rangle}{U}}
 {\mathcal D}}
\quad
\deduce{\mathstrut}{\Longrightarrow}
\infer[{!}_L]
{\mildseq{\Gamma}{\Delta,{!}A}{U}}
{\infer-[{\it subst}^+]
 {\mildseq{\Gamma,A}{\Delta}{U}}
 {\infer[{!}_R]
  {\mildseq{\Gamma, A}{\cdot}{[{!}A]}}
  {\infer-[\eta^-]
   {\mildseq{\Gamma, A}{\cdot}{A}}
   {\infer[\it copy]
    {\mildseq{\Gamma, A}{\cdot}{\langle A \rangle}}
    {\infer[{\it id}^-]
     {\mildseq{\Gamma, A}{[ A ]}{\langle A \rangle}}
     {}}}}
  &
  \infer-[{\it weaken}]
  {\mildseq{\Gamma,A}{\Delta, \langle {!}A \rangle}{U}}
  {\deduce
   {\mildseq{\Gamma}{\Delta, \langle {!}A \rangle}{U}}
   {\mathcal D}}}}
\]

\[
\infer-[\eta^+]
{\mildseq{\Gamma}{\Delta, A \otimes B}{U}}
{\deduce{\mildseq{\Gamma}{\Delta, \langle A \otimes B \rangle}{U}}{\mathcal D}}
\quad
\deduce{\mathstrut}{\Longrightarrow}
\!\!\!\!\!\!\!\!\!
\infer[{\otimes}_L]
{\mildseq{\Gamma}{\Delta, A \otimes B}{U}}
{\infer-[\eta^+]
 {\mildseq{\Gamma}{\Delta, A, B}{U}}
 {\infer-[\eta^+]
 {\mildseq{\Gamma}{\Delta, \langle A \rangle, B}{U}}
 {\infer-[{\it subst}^+]
  {\mildseq{\Gamma}{\Delta, \langle A \rangle, \langle B \rangle}{U}}
  {\infer
   {\mildseq{\Gamma}{\langle A \rangle, \langle B \rangle}{[A \otimes B]}}
   {\infer[{\it id}^+]
    {\mildseq{\Gamma}{\langle A \rangle}{[A]}}
    {}
    & 
    \infer[{\it id}^+]
    {\mildseq{\Gamma}{\langle B \rangle}{[B]}}
    {}}
   & 
   \deduce
   {\mildseq{\Gamma}{\Delta, \langle A \otimes B \rangle}{U}}
   {\mathcal D}}}}}
\]

\[
\infer-[\eta^-]
{\mildseq{\Gamma}{\Delta}{{\uparrow}A}}
{\deduce
 {\mildseq{\Gamma}{\Delta}{\langle {\uparrow}A \rangle}}
 {\mathcal D}}
\quad
\deduce{\mathstrut}{\Longrightarrow}
\quad
\infer[{\uparrow}_R]
{\mildseq{\Gamma}{\Delta}{{\uparrow}A}}
{\infer-[{\it subst}^-]
 {\mildseq{\Gamma}{\Delta}{A}}
 {\deduce
  {\mildseq{\Gamma}{\Delta}{\langle {\uparrow}A \rangle}}
  {\mathcal D}
  &
  \infer[{\uparrow}_L]
  {\mildseq{\Gamma}{[{\uparrow}A]}{A}}
  {\infer-[\eta^+]
   {\mildseq{\Gamma}{A}{A}}
   {\infer[{\it focus}_R]
    {\mildseq{\Gamma}{\langle A \rangle}{A}} 
    {\infer[{\it id}^+]
     {\mildseq{\Gamma}{\langle A \rangle}{[ A ]}}
     {}}}}}}
\]

\[
\infer-[\eta^-]
{\mildseq{\Gamma}{\Delta}{A \lolli B}}
{\deduce
 {\mildseq{\Gamma}{\Delta}{\langle A \lolli B \rangle}}
 {\mathcal D}}
\quad
\deduce{\mathstrut}{\Longrightarrow}
\!\!\!\!\!\!
\infer[{\lolli}_R]
{\mildseq{\Gamma}{\Delta}{A \lolli B}}
{\infer-[\eta^+]
 {\mildseq{\Gamma}{\Delta, A}{B}}
 {\infer-[\eta^-]
  {\mildseq{\Gamma}{\Delta, \langle A \rangle}{B}}
  {\infer-[{\it subst}^-]
   {\mildseq{\Gamma}{\Delta, \langle A \rangle}{\langle B \rangle}}
   {\deduce
    {\mildseq{\Gamma}{\Delta}{\langle A \lolli B \rangle}}
    {\mathcal D}
    &
    \infer[{\lolli}_L]
    {\mildseq{\Gamma}{\langle A \rangle, [ A \lolli B ]}{\langle B \rangle}}
    {\infer[{\it id}^+]
     {\mildseq{\Gamma}{\langle A \rangle}{[ A ]}}
     {}
     &
     \infer[{\it id}^-]
     {\mildseq{\Gamma}{[ B ]}{\langle B \rangle}}
     {}}}}}}
\]}
\caption{Identity expansion -- restricting $\eta^+$ and $\eta^-$ to atomic 
 propositions}
\label{fig:lineta-1}
\end{figure}

\begin{figure}[t]
{\small

\[
\infer-[\eta^+]
{\mildseq{\Gamma}{\Delta, \one}{U}}
{\deduce
 {\mildseq{\Gamma}{\Delta, \langle \one \rangle}{U}}
 {\mathcal D}}
\quad
\Longrightarrow
\infer[{\one}_L]
{\mildseq{\Gamma}{\Delta, \one}{U}}
{\infer-[{\it subst}^+]
 {\mildseq{\Gamma}{\Delta}{U}}
 {\infer[{\one}_R]
  {\mildseq{\Gamma}{\cdot}{[ \one ]}}
  {}
  &
  \deduce
  {\mildseq{\Gamma}{\Delta, \langle \one \rangle}{U}}
  {\mathcal D}}}
\]

\[
\infer-[\eta^+]
{\mildseq{\Gamma}{\Delta, \zero}{U}}
{\deduce
 {\mildseq{\Gamma}{\Delta, \langle \zero \rangle}{U}}
 {\mathcal D}}
\quad
\Longrightarrow
\infer[\zero_L]
{\mildseq{\Gamma}{\Delta, \zero}{U}}
{}
\]

\[
\infer-[\eta^+]
{\mildseq{\Gamma}{\Delta, A \oplus B}{U}}
{\deduce
 {\mildseq{\Gamma}{\Delta, \langle A \oplus B \rangle}{U}}
 {\mathcal D}}
\quad
\Longrightarrow
\!\!\!\!\!\!\!\!\!\!\!\!\!\!\!\!
\infer[{\oplus}_L]
{\mildseq{\Gamma}{\Delta, A \oplus B}{U}}
{\infer-[\eta^+]
 {\mildseq{\Gamma}{\Delta, A}{U}}
 {\infer-[{\it subst}^+]
  {\mildseq{\Gamma}{\Delta, \langle A \rangle}{U}}
  {\infer[{\oplus}_{R1}]
   {\mildseq{\Gamma}{\Delta, \langle A \rangle}{[ A \oplus B ]}}
   {\infer[{\it id}^+]
    {\mildseq{\Gamma}{\langle A \rangle}{[ A ]}}
    {}}
   &
   \deduce
   {\mildseq{\Gamma}{\Delta, \langle A \oplus B \rangle}{U}}
   {\mathcal D}}}
 &
 \deduce
 {\mildseq{\Gamma}{\Delta, B}{U}}
 {\vdots}
 }
\]


\[
\infer-[\eta^-]
{\mildseq{\Gamma}{\Delta}{\top}}
{\deduce
 {\mildseq{\Gamma}{\Delta}{\langle \top \rangle}}
 {\mathcal D}}
\quad
\Longrightarrow
\quad
\infer[{\top}_R]
{\mildseq{\Gamma}{\Delta}{\top}}
{}
\]

\[
\infer-[\eta^-]
{\mildseq{\Gamma}{\Delta}{A \with B}}
{\deduce
 {\mildseq{\Gamma}{\Delta}{\langle A \with B \rangle}}
 {\mathcal D}}
\quad
\Longrightarrow
\!\!\!\!
\infer[{\with}_R]
{\mildseq{\Gamma}{\Delta}{A \with B}}
{\infer[\eta^-]
 {\mildseq{\Gamma}{\Delta}{A}}
 {\infer-[{\it subst}^-]
  {\mildseq{\Gamma}{\Delta}{\langle A \rangle}}
  {\deduce
   {\mildseq{\Gamma}{\Delta}{\langle A \with B \rangle}}
   {\mathcal D}
   &
   \infer[{\with}_{L1}]
   {\mildseq{\Gamma}{[A \with B]}{\langle A \rangle}}
   {\infer[{\it id}^-]
    {\mildseq{\Gamma}{[A]}{\langle A \rangle}}
    {}}}}
 & 
 \deduce
 {\mildseq{\Gamma}{\Delta}{B}}
 {\vdots}}
\]}

\caption{Identity expansion for units and additive connectives}
\label{fig:lineta-2}
\end{figure}


Just as the {\it init} rule is a particular instance of the admissible
identity sequent $\seq{\Gamma}{A}{A}$ in unfocused linear logic, the
atomic suspension rules $\eta^+$ and $\eta^-$ are instances of an admissible
identity expansion rule in focused linear logic:
\[
\infer-[\eta^+]
{\mildseq{\Gamma}{\Delta, A^+}{U}}
{\mildseq{\Gamma}{\Delta, \langle A^+ \rangle}{U}}
\qquad
\infer-[\eta^-]
{\mildseq{\Gamma}{\Delta}{A^-}}
{\mildseq{\Gamma}{\Delta}{\langle A^- \rangle}}
\]
This admissible rule must be established by mutual recursion; the
proof is structurally inductive on the structure of the proposition,
and uses focal substitution in a critical way. Most of the cases of
this proof are represented in Figure~\ref{fig:lineta-1}. (Note that in
this figure we omit polarity annotations from propositions as they are
always clear from the context.) The remaining case (for the
multiplicative unit $\one$) is presented in Figure~\ref{fig:lineta-2}
along with the cases for the additive connectives $\zero$, $\oplus$,
$\top$, and $\with$, which are neglected elsewhere in this chapter.

The admissible identity expansion rules fit with an interpretation of
positive atomic propositions as stand-ins for arbitrary positive
propositions and of negative atomic propositions as stand-ins for
negative atomic propositions: if we substitute a proposition in for
some atomic proposition, all the instances of atomic suspension
corresponding to that rule become admissible instances of identity
expansion. Furthermore, the usual identity principles are simple
corollaries of identity expansion:
\[
\infer-[\eta^+]
{\mildseq{\Gamma}{A^+}{A^+}}
{\infer[{\it focus}_R]
 {\mildseq{\Gamma}{\langle A^+ \rangle}{A^+}}
 {\infer[{\it id}^+]
  {\mildseq{\Gamma}{\langle A^+ \rangle}{[A^+]}}
  {}}}
\qquad
\infer-[\eta^-]
{\mildseq{\Gamma}{A^-}{A^-}}
{\infer[{\it focus}_L]
 {\mildseq{\Gamma}{A^-}{\langle A^- \rangle}}
 {\infer[{\it id}^-]
  {\mildseq{\Gamma}{[A^-]}{\langle A^- \rangle}}
  {}}}
\]

\subsection{Cut admissibility}
\label{sec:lincut}

Theorem~\ref{thm:lincut} mostly follows the well-worn contours of a
structural cut admissibility argument \cite{pfenning00structural}, so
we defer a full discussion of cut admissibility until the next
chapter, where the use of structural focalization will allow us to
give a tidier proof of the theorem.\footnote{The main ``untidy''
  aspect of Theorem~\ref{thm:lincut} is that the lack of a forced
  inversion order means that the right commutative cases dealing with
  invertible rules must be repeated in parts 1 and 3, and likewise for
  the left commutative cases dealing with invertible rules and parts 2
  and 4. The repetition of {\it all} cases between parts 3 and 5 will
  also be addressed in the next chapter.}
%
The only important caveat to emphasize about Theorem~\ref{thm:lincut}
is that cut admissibility is only applicable in the absence of any
non-atomic suspended propositions. If we did not make this
restriction, then in Theorem~\ref{thm:lincut}, part 1, we might encounter
a derivation of $\mildseq{\Gamma}{\langle A \tensor B \rangle}{[ A \tensor B ]}$
being cut into the derivation
\[
\infer[{\otimes}_R]
{\mildseq{\Gamma}{\Delta',A \tensor B}{U}}
{\deduce{\mildseq{\Gamma}{\Delta', A, B}{U}}{\mathcal E}}
\]
in which case there is no clear way to proceed and prove 
$\mildseq{\Gamma}{\Delta', \langle A \tensor B \rangle}{U}$. 

\bigskip
\begin{theorem}[Cut admissibility]\label{thm:lincut}
For all $\Gamma$, $A^+$, $A^-$, $\Delta$, $\Delta'$, and $U$ that
do not contain any non-atomic suspended propositions:
\begin{enumerate}
\item If $\mildseq{\Gamma}{\Delta}{[A^+]}$
      and $\mildseq{\Gamma}{\Delta',A^+}{U}$
      (where $\Delta$ is stable), 
      then $\mildseq{\Gamma}{\Delta',\Delta}{U}$.
\item If $\mildseq{\Gamma}{\Delta}{A^-}$
      and $\mildseq{\Gamma}{\Delta', [A^-]}{U}$
      (where $\Delta$, $\Delta'$, and $U$ are stable),
      then $\mildseq{\Gamma}{\Delta',\Delta}{U}$. 
\item If $\mildseq{\Gamma}{\Delta}{A^-}$
      and $\mildseq{\Gamma}{\Delta', A^-}{U}$
      (where $\Delta$ is stable), 
      then $\mildseq{\Gamma}{\Delta',\Delta}{U}$. 
\item If $\mildseq{\Gamma}{\Delta}{A^+}$
      and $\mildseq{\Gamma}{\Delta', A^+}{U}$
      (where $\Delta'$ and $U$ are stable),
      then $\mildseq{\Gamma}{\Delta',\Delta}{U}$. 
\item If $\mildseq{\Gamma}{\cdot}{A^-}$
      and $\mildseq{\Gamma, A^-}{\Delta}{U}$,
      then $\mildseq{\Gamma}{\Delta}{U}$. 
\end{enumerate}
\end{theorem}

\begin{proof}
By lexicographic induction. In each invocation of the induction
hypothesis, either the principal cut formula $A^+$ or $A^-$ gets 
smaller, or else it stays the same and the number of the part
gets smaller (as when we invoke part 2 when proving part 5). 

Within parts 3 and 5, the first two metrics stay the same while the
second given derivation gets smaller, and within part 4, the first two
metrics stay the same while the first given derivation gets smaller.
\end{proof}

\subsection{Correctness of focusing}

Now we will make precise the correctness for a focused, polarized logic
that was discussed in Section~\ref{sec:linpolar}: that there is a unfocused
proof of $A$ if and only if there is a focused proof of $A^\oplus$. The
proof require lifting of our erasure function and our ``obvious'' 
polarization strategy to contexts and succeedents, which is done
in Figure~\ref{fig:lin-shift-ctx}. Soundness is established on the 
basis of erasure as in \cite{simmons11structural}, but as discussed
we give the slightly simpler polarization-strategy-based proof of
completeness.

\bigskip
\begin{theorem}[Soundness of focusing]
If $\mildseq{\Gamma}{\Delta}{U}$, where $\Delta$ and $U$ contain only
atomic suspensions, then $\seq{\Gamma^\circ}{\Delta^\circ}{U^\circ}$.
\end{theorem}

\begin{proof}
  By straightforward induction on the given derivation; in each case,
  the result either follows directly by invoking the induction
  hypothesis or by invoking the induction hypothesis and applying one
  rule from Figure~\ref{fig:linear}.
\end{proof}

\begin{theorem}[Completeness of focusing]
If $\seq{\Gamma}{\Delta}{C}$,
then $\mildseq{\Gamma^\bullet}{\Delta^\bullet}{C^\bullet}$. 
\end{theorem}

\begin{proof}
  By induction on the first given derivation. Each rule in 
  Figure~\ref{fig:linear}

  Rule {\it copy}:
  \[
  \]

  Rule $\lolli_L$:
  \[
  \]

  Rule $\lolli_R$:
  {\small \[
  \infer-[{\it cut}(3)]
  {\mildseq{\Gamma^\bullet}{\Delta^\bullet}
    {{\downarrow}(A^\oplus \lolli B^\ominus)}}
  {\infer[{\lolli}_R]
   {\mildseq{\Gamma^\bullet}{\Delta^\bullet}
     {{\downarrow}{\uparrow}A^\oplus \lolli {\uparrow}{\downarrow}B^\ominus}}  
   {\infer[{\downarrow}_L]
    {\mildseq{\Gamma^\bullet}{\Delta^\bullet, {\downarrow}{\uparrow}A^\oplus}
      {{\uparrow}{\downarrow}B^\ominus}}
    {\infer[{\uparrow}_R]
     {\mildseq{\Gamma^\bullet}{\Delta^\bullet, {\uparrow}A^\oplus}
      {{\uparrow}{\downarrow}B^\ominus}}
     {\mildseq{\Gamma^\bullet}{\Delta^\bullet, {\uparrow}A^\oplus}
      {{\downarrow}B^\ominus}}}}
   &
   \infer[{\it focus}_R]
   {\mildseq{\Gamma}{{\downarrow}{\uparrow}A^\oplus \multimap B^\ominus}
      {{\downarrow}(A^\oplus \lolli B^\ominus)}}
   {\infer[{\downarrow}_R]
    {\mildseq{\Gamma^\bullet}
      {{\downarrow}{\uparrow}A^\oplus \multimap {\uparrow}{\downarrow}B^\ominus}
      {[ {\downarrow}(A^\oplus \lolli B^\ominus)} ]}
    {\infer[{\lolli}_R]
     {\mildseq{\Gamma^\bullet}
       {{\downarrow}{\uparrow}A^\oplus \multimap {\uparrow}{\downarrow}B^\ominus}
       {A^\oplus \lolli B^\ominus}}
     {\infer-[\eta^+]
      {\mildseq{\Gamma^\bullet}
        {{\downarrow}{\uparrow}A^\oplus 
           \multimap {\uparrow}{\downarrow}B^\ominus, 
         A^\oplus}
        {B^\ominus}}
      {\infer-[\eta^-]
       {\mildseq{\Gamma^\bullet}
         {{\downarrow}{\uparrow}A^\oplus
            \multimap {\uparrow}{\downarrow}B^\ominus, 
          \langle A^\oplus \rangle}
         {B^\ominus}}
       {\infer[{\it focus}_L]
        {\mildseq{\Gamma^\bullet}
          {{\downarrow}{\uparrow}A^\oplus 
             \multimap {\uparrow}{\downarrow}B^\ominus, 
           \langle A^\oplus \rangle}
          {\langle B^\ominus \rangle}}
        {\infer[{\lolli}_L]
         {\mildseq{\Gamma^\bullet}
           {[ {\downarrow}{\uparrow}A^\oplus 
              \multimap {\uparrow}{\downarrow}B^\ominus ], 
            \langle A^\oplus \rangle}
           {\langle B^\ominus \rangle}}
         {\infer[{\downarrow}_R]
          {\mildseq{\Gamma^\bullet}{\langle A^\oplus \rangle}
           {[{\downarrow}{\uparrow}A^\oplus]}}
          {\infer[{\uparrow}_R]
           {\mildseq{\Gamma^\bullet}{\langle A^+ \rangle}{{\uparrow}A^+}}
           {\infer[{\it focus}_R]
            {\mildseq{\Gamma^\bullet}{\langle A^+ \rangle}{A^+}} 
            {\infer[{\it id}^+]
             {\mildseq{\Gamma^\bullet}{\langle A^+ \rangle}{[A^+]}}
             {}}}}
          &
          \infer[{\uparrow}_L]
          {\mildseq{\Gamma^\bullet}{[{\uparrow}{\downarrow}B^\ominus]}
             {\langle B^\ominus \rangle}}
          {\infer[{\downarrow}_L]
           {\mildseq{\Gamma^\bullet}{{\downarrow}B^\ominus}
             {\langle B^\ominus \rangle}}
           {\infer[{\it focus}_L]
            {\mildseq{\Gamma^\bullet}{B^\ominus}{\langle B^\ominus \rangle}}
            {\infer[{\it id}^-]
             {\mildseq{\Gamma^\bullet}{[B^\ominus]}{\langle B^\ominus \rangle}}
             {}}}}}}}}}}}}
  \]}
\end{proof}

The correctness of focusing is established as in the structural focalization
methadology, except that we establish a weaker polarization-strategy-based
proof of completeness rather than an erasure-based proof as in 
\cite{simmons11structural}. Both the soundness and completeness of focusing


\begin{figure}
{\small \[
\begin{array}{rcl|rcl|rcl}
{(\Gamma)^\circ} & & &
{(\underline{\Delta})^\circ} & & &
{(\underline{U})^\circ} & & 
\\
(\cdot)^\circ & \!\!\!=\!\!\! & \cdot &
(\cdot)^\circ & \!\!\!=\!\!\! & \cdot &
(A^-)^\circ & \!\!\!=\!\!\! & (A^-)^\circ
\\
(\Gamma, A^-)^\circ & \!\!\!=\!\!\! & (\Gamma)^\circ, (A^-)^\circ &
(\Delta, A^+)^\circ & \!\!\!=\!\!\! & (\Delta)^\circ, (A^+)^\circ &
(A^+)^\circ & \!\!\!=\!\!\! & (A^+)^\circ
\\
& & & 
(\Delta, A^-)^\circ & \!\!\!=\!\!\! & (\Delta)^\circ, (A^-)^\circ &
([A^+])^\circ & \!\!\!=\!\!\! & (A^+)^\circ 
\\
& & &
(\Delta, [ A^- ])^\circ & \!\!\!=\!\!\! & (\Delta)^\circ, (A^-)^\circ & 
(\langle p^- \rangle)^\circ & \!\!\!=\!\!\! & p^-
\\
& & &
(\Delta, \langle p^+ \rangle)^\circ & \!\!\!=\!\!\! & (\Delta)^\circ, p^+ & 
& &
\end{array}\]}
\caption{Lifting erasure and polarization (Figure~\ref{fig:lin-shift}) to
contexts and succeedents}
\label{fig:lin-shift-ctx}
\end{figure}




\begin{proof}

The reverse direction is the soundness of focusing. It is completely
straightforward, 
\end{proof}

\subsection{Confluent versus fixed inversion}
\label{sec:confluent-v-fixed}

\subsection{Running example}

\begin{figure}
\[
\infer[{\lolli}_R]
{\mildseq{\cdot}{\cdot}{{!}({\sf 6bucks} \lolli {\sf battery}) \otimes
                    {\sf 6bucks} \otimes 
                    ({\sf battery} \lolli {\sf robot}) \lolli {\sf robot}}}
{\infer[{\otimes}_L]
{\mildseq{\cdot}{{!}({\sf 6bucks} \lolli {\sf battery}) \otimes
                    {\sf 6bucks} \otimes 
                    ({\sf battery} \lolli {\sf robot})}{{\sf robot}}}
{\infer[{!}_L]
{\mildseq{\cdot}{{!}({\sf 6bucks} \lolli {\sf battery}),
                    {\sf 6bucks} \otimes 
                    ({\sf battery} \lolli {\sf robot})}{{\sf robot}}}
{\infer[{\otimes}_L]
{\mildseq{\Gamma}{{\sf 6bucks} \otimes 
                    ({\sf battery} \lolli {\sf robot})}{{\sf robot}}}
{\infer[{\it copy}]
{\mildseq{\Gamma}{{\sf 6bucks}, {\sf battery} \lolli {\sf robot}}{{\sf robot}}}
{\infer[{\lolli}_L]
{\mildseq{\Gamma}{{\sf 6bucks}, {\sf battery} \lolli {\sf robot}, [{\sf 6bucks} \lolli {\sf battery}]}{{\sf robot}}}
{\infer[{\it init}^+]
 {\mildseq{\Gamma}{{\sf 6bucks}}{[{\sf 6bucks}]}}
 {}
 &
 \infer[{\it blur}_L]
 {\mildseq{\Gamma}{{\sf battery} \lolli {\sf robot}, [{\sf battery}]}{{\sf robot}}}
 {\infer[{\it focus}_L]
 {\mildseq{\Gamma}{{\sf battery} \lolli {\sf robot}, {\sf battery}}{{\sf robot}}}
 {\infer[{\lolli}_L]
 {\mildseq{\Gamma}{{\sf battery}, [{\sf battery} \lolli {\sf robot}]}{{\sf robot}}}
 {\infer[{\it init^+}]
  {\mildseq{\Gamma}{{\sf battery}}{[{\sf battery}]}}
  {}
  &
  \infer[{\it blur}_L]
  {\mildseq{\Gamma}{[{\sf robot}]}{{\sf robot}}}
  {\infer[{\it focus}_R]
  {\mildseq{\Gamma}{{\sf robot}}{{\sf robot}}}
  {\infer[{\it init}^+]
  {\mildseq{\Gamma}{{\sf robot}}{[{\sf robot}]}}
  {}}}}}}}}}}}}
\] 
\caption{The single focused transition is possible 
(where we let $\Gamma = {\sf 6bucks} \lolli {\sf battery}$).}
\label{fig:focused-robot}
\end{figure}

Figure

\section{Synthetic inference rules}
\label{sec:linsynthetic}

\section{Hacking the focusing system}
\label{sec:linhack}

\subsection{Atom optimization}

\subsection{Bang optimization}

\subsection{A more primitive logic?}

\paragraph{Adjoint logic}

\paragraph{Tensor logic}

\subsection{Concurrent equality}

\section{Revisiting our notation}
\label{sec:linnote}

\section{A warning about normalization}
\label{sec:warning}

Talk about equivalence, Chris's unpublished work, and where
the focalization theorem given by this approach is deficient -- 

% Substructural logic
\chapter{Substructural logic}

Another name that would work
is {\it structural} logic. In formal presentations of logic, this
persistence manifests itself as the so-called {\it structural
  properties} of hypothetical reasoning -- {\it weakening} (which
allows hypotheses to go unused), {\it contraction} (which allows
hypotheses to be duplicated for reuse), and {\it exchange} (which
enforces that the ordering of hypotheses is not meaningful)
\cite{gentzen35untersuchungen}.

\section{Ordered linear logic}

Ordered linear logic 

\begin{figure}
\small
{\it Atomic propositions}
\[
\infer[{\it init}^+]
{\oiseq{\Gamma}{p}}
{}
\]

\medskip
{\it Modalities}
\[
\infer[{\gnab}_R]
{\orseq{\Gamma}{\Delta}{\cdot}{{\gnab}A}}
{\otseq{\Gamma}{\Delta}{\cdot}{A}}
\quad
\infer[{\gnab}_L]
{\olseq{\Gamma}{\Delta}{\Omega_L}{{\gnab}A}{\Omega_R}}
{\opseq{\Gamma}{\Delta, A}{\Omega_L}{\Omega_R}}
\]
\vspace{-5pt}
\[
\infer[{!}_R]
{\orseq{\Gamma}{\cdot}{\cdot}{{!}A}}
{\otseq{\Gamma}{\cdot}{\cdot}{A}}
\quad
\infer[{!}_L]
{\olseq{\Gamma}{\Delta}{\Omega_L}{{!}A}{\Omega_R}}
{\opseq{\Gamma,A}{\Delta}{\Omega_L}{\Omega_R}}
\]
\vspace{-5pt}
\[
\infer[{\ocircle}_R]
{\orseq{\Gamma}{\Delta}{\Omega}{{\ocircle}A}}
{\oseq{\Gamma}{\Delta}{\Omega}{\islax{A}}}
\quad
\infer[{\ocircle}_L]
{\oseq{\Gamma}{\Delta}{\Omega_L /{\ocircle}A/ \Omega_R}{\islax{C}}}
{\oseq{\Gamma}{\Delta}{\Omega_L, A, \Omega_R}{\islax{C}}}
\]

\medskip
{\it Multiplicative connectives}
\[
\infer[{\one}_R]
{\orseq{\Gamma}{\cdot}{\cdot}{\one}}
{}
\quad
\infer[{\one}_L]
{\olseq{\Gamma}{\Delta}{\Omega_L}{\one}{\Omega_R}}
{\opseq{\Gamma}{\Delta}{\Omega_L}{\Omega_R}}
\]
\[
\infer[{\fuse}_R]
{\orseq{\Gamma}{\Delta_1,\Delta_2}{\Omega_L,\Omega_R}{A \fuse B}}
{\otseq{\Gamma}{\Delta}{\Omega_L}{A}
 &
 \otseq{\Gamma}{\Delta}{\Omega_R}{B}}
\quad
\infer[{\fuse}_L]
{\olseq{\Gamma}{\Delta}{\Omega_L}{A \fuse B}{\Omega_R}}
{\opseq{\Gamma}{\Delta}{\Omega_L}{A,B,\Omega_R}}
\]
\vspace{-5pt}
\[
\infer[{\lefti}_R]
{\orseq{\Gamma}{\Delta}{\Omega}{A \lefti B}}
{\otseq{\Gamma}{\Delta}{A, \Omega}{B}}
\quad
\infer[{\lefti}_L]
{\olseq{\Gamma}{\Delta_A, \Delta}{\Omega_L, \Omega_A}{A \lefti B}{\Omega_R}}
{\otseq{\Gamma}{\Delta_A}{\Omega_A}{A}
 &
 \opseq{\Gamma}{\Delta}{\Omega_L}{B,\Omega_R}}
\]
\vspace{-5pt}
\[
\infer[{\righti}_R]
{\orseq{\Gamma}{\Delta}{\Omega}{A \righti B}}
{\otseq{\Gamma}{\Delta}{\Omega, A}{B}}
\quad
\infer[{\righti}_L]
{\olseq{\Gamma}{\Delta_A, \Delta}{\Omega_L}{A \righti B}{\Omega_A, \Omega_R}}
{\otseq{\Gamma}{\Delta_A}{\Omega_A}{A}
 &
 \opseq{\Gamma}{\Delta}{\Omega_L}{B, \Omega_R}}
\]

\medskip
{\it Additive connectives}
\[
\infer[{\zero}_L]
{\olseq{\Gamma}{\Delta}{\Omega_L}{\zero}{\Omega_R}}
{}
\quad
\infer[{\oplus}_{R1}]
{\orseq{\Gamma}{\Delta}{\Omega}{A \oplus B}}
{\otseq{\Gamma}{\Delta}{\Omega}{A}}
\quad
\infer[{\oplus}_{R2}]
{\orseq{\Gamma}{\Delta}{\Omega}{A \oplus B}}
{\otseq{\Gamma}{\Delta}{\Omega}{B}}
\]
\vspace{-5pt}
\[
\infer[{\oplus}_{L}]
{\olseq{\Gamma}{\Delta}{\Omega_L}{A \oplus B}{\Omega_R}}
{\opseq{\Gamma}{\Delta}{\Omega_L}{A,\Omega_R}
 &
 \opseq{\Gamma}{\Delta}{\Omega_L}{B,\Omega_R}}
\]
\vspace{-5pt}
\[
\infer[{\top}_R]
{\orseq{\Gamma}{\Delta}{\Omega}{\top}}
{}
\quad
\infer[{\with}_{L1}]
{\olseq{\Gamma}{\Delta}{\Omega_L}{A \with B}{\Omega_R}}
{\opseq{\Gamma}{\Delta}{\Omega_L}{A, \Omega_R}}
\quad
\infer[{\with}_{L2}]
{\olseq{\Gamma}{\Delta}{\Omega_L}{A \with B}{\Omega_R}}
{\opseq{\Gamma}{\Delta}{\Omega_L}{B, \Omega_R}}
\]
\vspace{-5pt}
\[
\infer[{\with}_R]
{\orseq{\Gamma}{\Delta}{\Omega}{A \with B}}
{\otseq{\Gamma}{\Delta}{\Omega}{A}
 &
 \otseq{\Gamma}{\Delta}{\Omega}{B}}
\]


\caption{Propositional ordered linear lax logic.}
\label{fig:ordered-prop}
\end{figure}




\section{Substructural contexts}

We will treat {\it substructural contexts} $\Delta$ are maps from {\it
  variables} to {\it judgments}. There are two kinds of variables: 
those that map {\it extent}.

${\mbox{\it [}}$


Two of the fundamental propeties of $\Delta$ are 
$x{:}\langle A^+ \rangle_l \sqsubseteq \Delta$. In linear logic, this
is the statement that $x{:}A \in \Gamma$, whereas in 

\[\small
\begin{array}{|c|c|c|c|}
%
\begin{array}{c}
\makebox[1.8in]{\it Dual intuitionstic linear logic}\medskip
\\
\infer
{\Gamma; p^+ \vdash p^+ \mathstrut}
{\mathstrut}
\end{array}
%
&
%
\begin{array}{c}
\makebox[1.8in]{\it Pfenning-Davies S4}\medskip
\\
\infer
{\Delta; \Gamma, p^+ \mathstrut \vdash p^+}
{\mathstrut}
\end{array}
%
&
%
\begin{array}{c}
\makebox[1.8in]{\it Unified}\medskip
\\
\infer
{\Delta \vdash [ p^+ ]_l \mathstrut}
{x{:}\langle p^+ \rangle_l \subseteq \Delta \mathstrut}
\end{array}
%
\end{array}
\]

\[\small
\begin{array}{|c|c|c|c|}
%
\begin{array}{c}
\makebox[1.8in]{\it Dual intuitionstic linear logic}\medskip
\\
\infer
{\Gamma; \Delta, A \with B \vdash C \mathstrut}
{\Gamma; \Delta, A \vdash C \mathstrut}
\end{array}
%
&
%
\begin{array}{c}
\makebox[1.8in]{\it Pfenning-Davies S4}\medskip
\\
\infer
{\Delta; \Gamma, A \wedge^- B \mathstrut \vdash C}
{\Delta; \Gamma, A \wedge^- B, A \vdash C \mathstrut}
\end{array}
%
&
%
\begin{array}{c}
\makebox[1.8in]{\it Unified}\medskip
\\
\infer
{\Theta\{ x{:} A \with B \} \vdash U \mathstrut}
{\Theta\{ x_1{:} A \} \vdash U \mathstrut}
\end{array}
%
\end{array}
\]


\[\small
\begin{array}{|c|c|c|c|}
%
\begin{array}{c}
\makebox[1.8in]{\it Dual intuitionstic linear logic}\medskip
\\
\infer
{\Gamma; \Delta_1, \Delta_2 \vdash A \otimes B \mathstrut}
{\Gamma; \Delta_1 \vdash A & \Gamma; \Delta_2 \vdash B \mathstrut}
\end{array}
%
&
%
\begin{array}{c}
\makebox[1.8in]{\it Pfenning-Davies S4}\medskip
\\
\infer
{\Delta; \Gamma \vdash A \wedge^+ B \mathstrut}
{\Delta; \Gamma \vdash A & \Delta; \Gamma \vdash B \mathstrut}
\end{array}
%
&
%
\begin{array}{c}
\makebox[1.8in]{\it Unified}\medskip
\\
\infer
{\Delta_1 \bowtie \Delta_2 \vdash A \otimes B \mathstrut}
{\Delta \vdash A & \Delta_2 \vdash B \mathstrut}
\end{array}
%
\end{array}
\]




% Substructural logical specifications
\chapter{Substructural logical specifications}
\label{chapter-framework}

In this chapter, we design a logical framework of substructural
logical specifications (\sls), a framework heavily inspired by the
Concurrent Logical Framework (CLF) \cite{watkins02concurrent}. The
framework is justified as a fragment of the logic \ollll~from
Chapter~\ref{chapter-order}. 
There are a number of reasons why we do not just use the
already-specified \ollll~outright as a logical framework.
%
\smallskip
\begin{itemize}
\item{\it Formality.} The specifics of the domain of first-order
  quantification in \ollll~were omitted in Chapter~\ref{chapter-order}, so in
  Section~\ref{sec:sls-termlanguage} we give a careful presentation of
  the term language for \sls, Spine Form LF.

\item{\it Clarity.} The syntax constructions that we presented for
  \ollll~proof terms had a 1-to-1 correspondence with the sequent
  calculus rules; the drawback of this presentation was that large
  proof terms are notationally heavy and difficult to read. The proof
  terms we present for \sls~will leave implicit some of the
  information present in the diacritical marks of \ollll~proof
  terms. 

  An implementation based on these proof terms would need to consider
  type reconstruction and/or bidirectional typechecking to recover the
  omitted information, but we will not consider those issues in this
  thesis.

\item{\it Separating concurrent and deductive reasoning.} Comparing
  CLF to the focused logics from previous chapters leads us to
  conclude that the single most critical design feature of CLF is its
  omission of the proposition ${\uparrow}A^+$. This single
  omission\footnote{In our development, the omission of
    right-permeable propositions $p^-_\mlax$ from \ollll~is equally
    important, but permeable propositions as we have presented them in
    Section~\ref{sec:permeable} were not a relevant consideration in
    the design of CLF.} means that stable sequents in CLF or \sls~are
  effectively restricted to have the succedent $\istrue{\susp{p^-}}$
  or the succedent $\islax{A^+}$.

  Furthermore, any left focus when the succedent is
  $\istrue{\susp{p^-}}$ must conclude with the rule ${\it id}^-$, and
  any left focus when the succedent is $\islax{A^+}$ must conclude
  with ${\ocircle}_L$ -- without the elimination of ${\uparrow}A^+$,
  left focus in both cases could additionally conclude with the rule
  ${\uparrow}_L$. This allows derivations that prove
  $\istrue{\susp{p^-}}$ -- the {\it deductive fragment} of CLF or SLS
  -- to adequately represent deductive systems, conservatively
  extending deductive logical frameworks like LF and LLF. Derivations
  that prove $\islax{A^+}$, on the other hand, fall into the {\it
    concurrent fragment} of CLF and SLS and can encode evolving
  systems.
 
\item{\it Partial proofs.} The design of CLF makes it difficult to
  reason about and manipulate the proof terms corresponding to partial
  evaluations of evolving systems in the concurrent fragment: the proof
  terms in CLF correspond to complete proofs and partial evaluations
  naturally correspond to partial proofs.

  The syntax of \sls~is designed to support the explicit
  representation of partial \ollll~proofs. The omission of the
  propositions $\zero$, $A^+ \oplus B^+$, and the restrictions we
  place on $\lf{t} \doteq_\tau \lf{s}$ are made in the service of
  presenting a convenient and simple syntax for partial proofs. The
  three syntactic objects representing partial proofs, {\it patterns}
  (Section~\ref{sec:framework-patterns}), {\it steps}, and {\it
    traces} (Section~\ref{sec:framework-concurrent}), allow us to
  treat proof terms for evolving systems as first-class members of
  \sls.

  The removal of $\zero$ and $A^+ \oplus B^+$, and the restrictions we
  place on $\lf{t} \doteq_\tau \lf{s}$, also assist in imposing
  equivalence relation, {\it concurrent equality}, on
  \sls~terms. Concurrent equality is a coarser equivalence relation
  than the $\alpha$-equivalence of \ollll~terms.

\item{\it Removal of $\top$.} The presence of $\top$ causes pervasive
  problems in the design of substructural logical frameworks. May of
  these problems arise at the level of implementation and type
  reconstruction, which motivated Schack-Nielson to remove $\top$ from
  the Celf implementation of CLF
  \cite{schacknielsen11implementing}. Even though those considerations
  are outside the scope of this thesis, the presence of $\top$ causes
  other pervasive difficulties: for instance, the presence of $\top$
  complicates the discussion of concurrent equality in CLF. We therefore
  follow Schack-Nielson in removing $\top$ from \sls.

\end{itemize}
\smallskip
\noindent
In summary, with \sls~we {\it simply} the presentation of \ollll~for
convenience and readability, {\it restrict} the propositions of
\ollll~to separate concurrent and deductive reasoning and to make the
syntax for partial proofs feasible, and {\it extend} \ollll~with a
syntax for partial proofs and a coarser equivalence relation.

In Section~\ref{sec:sls-termlanguage} we give a brief presentation of
the term language for \sls, Spine Form LF.
%
In Section~\ref{sec:slsframework} we present \sls~as a fragment of
\ollll, and in Section~\ref{sec:framework-concurrenteq} we discuss
concurrent equality. 
%
In Section~\ref{sec:sls-adequate} we adopt the methodology of adequate
encoding from LF to \sls, in the process introducing {\it generative
  signatures}, which play a critical role in Part 3 of this thesis.
%
In Section~\ref{sec:prototype} we cover the \sls~prototype
implementation, and in Section~\ref{sec:framework-logicprog} we review
some intuitions about logic programming in the framework. 
%
Finally, in Section~\ref{sec:designdecisions}, we discuss some of the
decisions reflected in the design of \sls~and how some decisions could
have been potentially been made differently.

\section{Spine Form LF as a term language}
\label{sec:sls-termlanguage}

Other substructural logical frameworks, like Cervesato and Pfenning's
LLF \cite{cervesato02linear}, Polakow's OLF \cite{polakow01ordered},
and Watkins et al.'s CLF \cite{watkins02concurrent} are {\it
  fully-dependent type theories}: the language of terms (that is, the
domain of first-order quantification) is the same as the language of
proof terms, the representatives of logical derivations (we will call
the domain of quantification the {\it object terms} when ``terms''
would be ambiguous). The logical framework \sls~presented in this
chapter breaks from this tradition -- a choice we discuss further in
Section~\ref{sec:why-not-fully-dependent}. The domain of first-order
quantification, which was left unspecified in Chapter~\ref{chapter-order}, 
will be
presently described as Spine Form LF, a well-understood logical
framework derived from the normal forms of the purely persistent type
theory LF \cite{harper93framework}.

All the information in this section is standard and adapted from
various sources, especially Harper, Honsell, and Plotkin's original
presentation of LF \cite{harper93framework}, Cervesato and Pfenning's
discussion of spine form terms \cite{cervesato02linear}, Watkins et
al.'s presentation of the canonical forms of CLF
\cite{watkins02concurrent}, Nanevski et al.'s dependent contextual
modal type theory \cite{nanevski08contextual}, Harper and Licata's
discussion of Canonical LF \cite{harper07mechanizing}, and Reed's
spine form presentation of HLF \cite{reed09hybrid}.

It would be entirely consistent for us to appropriate Harper and
Licata's Canonical LF presentation instead of presenting Spine Form
LF. Nevertheless, a spine-form presentation of canonical LF serves to
make our presentation more uniform, as spines are used in the proof
term language of \sls. Canonical term languages like Canonical LF
correspond to normal natural deduction presentations of logic, whereas
spine form term languages correspond to focused sequent calculus
presentations like the ones we have considered thus far.

\subsection{Core syntax}

The syntax of Spine Form LF is extended in two places to handle \sls:
rules ${\sf r} : A^-$ in the signature contain negative \sls~types
$A^-$ (it would be possible to separate out the LF portion of
signatures from the \sls~rules), and several new base kinds
are introduced for the sake of \sls~-- ${\sf prop}$, ${\sf prop}\,{\sf
  ord}$, ${\sf prop}\,{\sf lin}$, and ${\sf prop}\,{\sf pers}$.
% We also add four additional kinds, ${\sf prop}$, which classifies
% negative ordered atomic types $p^-$, ${\sf prop}\,{\sf ord}$, which
% classifies positive ordered atomic types $p^+$, ${\sf prop}\,{\sf
%   lin}$, which classifies positive linear/mobile/ephemeral atomic
% types $p^+_\meph$, and ${\sf prop}\,{\sf ord}$, which classifies
% positive persistent atomic types $p^+_\mpers$.  Other than the extra
% kinds classifying atomic \sls~propositions, kinds $\kappa$ are
% otherwise exactly as they are in other presentations of LF; kinds
% classify types $\tau$, and types $\tau$ classify normal terms
% $\lf{t}$ and spines $\lf{\spi}$. Kinds $\kappa$ and types $\tau$ are
% both treated as syntactic refinements of {\it classifiers} $\nu$.
\begin{align*}
& \mbox{Signatures} & \Sigma & ::= \cdot 
  \mid \Sigma, \lf{\sf c} : \tau
  \mid \Sigma, {\sf a} : \kappa
  \mid \Sigma, {\sf r} : A^-
\\
& \mbox{Variables} & \lf{a}, \lf{b} & ::= \ldots
\\
& \mbox{Variable contexts} & \Psi & ::= \cdot
  \mid \Psi, \lf{a} {:} \tau 
\\
& \mbox{Kinds} & \kappa & ::=  \lfpi{a}{\tau}{\kappa} 
  \mid {\sf type}
  \mid {\sf prop}
  \mid {\sf prop}\,{\sf ord}
  \mid {\sf prop}\,{\sf lin}
  \mid {\sf prop}\,{\sf pers}
\\
& \mbox{Types} & \tau & ::= \lfpi{a}{\tau}{\tau'} 
  \mid \lfroot{\sf a}{\spi}
\\
& \mbox{Heads} & \lf{h} & ::= \lf{a} \mid \lf{\sf c}
\\
& \mbox{Normal terms} & \lf{t}, \lf{s} & ::= \lf{\lambda a.t}
  \mid \lf{\lfroot{h}{\spi}}
\\
& \mbox{Spines} & \lf{\spi} & ::= \lf{t; \spi} \mid \lf{\lfnil}
\\
& \mbox{Substitutions} & \lf{\sigma} & ::= \lf{\cdot}
  \mid \lf{t/a, \sigma}
  \mid \lf{b/\!\!/a, \sigma}
\end{align*}
\noindent
Types $\tau$ and kinds $\kappa$ overlap, and will be referred to
generically as {\it classifiers} $\nu$ when it is convenient to do so;
types and kinds can be seen as refinements of classifiers. Another
important refinement are {\it atomic classifiers} $\lfroot{\sf
  a}{\spi}$, which we abbreviate as $p$.

LF spines $\lf\spi$ are just sequences of terms $\lf{(t_1; (\ldots;
  (t_n;())\ldots))}$; we will follow common convention and write
$\lf{h\,t_1\ldots t_n}$ as a convenient shorthand for the atomic term
$\lf{\lfroot{h}{(t_1; \ldots; (t_n;())\ldots)}}$; similarly, we will
write ${\sf a}\,\lf{t_1\ldots t_n}$ as a shorthand for atomic
classifiers $\lfroot{\sf a}{(t_1;
  (\ldots; (t_n;())\ldots))}$. This shorthand is given a formal justification
in \cite{cervesato02linear}; we will use the same shorthand for 
\sls~proof terms in Section~\ref{sec:framework-deductive}.

% This shorthand evokes a canonical-forms
% presentation, as an atomic term, type, or proposition is a head
% $\lf{h}$ or ${\sf a}$ with the terms $\lf{t_1\ldots t_n}$ applied to
% it.

\begin{figure}[t]
\begin{align*}
\fbox{$\subst{\lf{t}}{\lf{\spi}}$}
\\
\subst{(\lf{\lambda a. t'})}{(\lf{t; \spi})}
 & = \subst{\rsubst{\lf{t}}{\lf{a}}{\lf{t'}}}{\lf{\spi}}
\\
\subst{\lfroot{\lf h}{\spi}}{\lfnil}
 & = \lfroot{\lf h}{\spi}
\end{align*}\begin{align*}
\fbox{$\rsubst{\lf{t}}{\lf{a}}{\lf{\spi}}$}&
&
\fbox{$\rsubst{\lf{t}}{\lf{a}}{\lf{t'}}$}&
&
\\
\rsubst{\lf t}{\lf a}{(\lf{t'; \spi})}
 & = \lf{\no{\rsubst{\lf t}{\lf a}{\lf{t'}}}; 
         \no{\rsubst{\lf t}{\lf a}{\lf{\spi}}}} &
\rsubst{\lf t}{\lf a}(\lf{\lambda y. t'})
 & = \lf{\lambda b.\, \no{\rsubst{\lf t}{\lf a}{\lf{t'}}}} 
      & (\lf a \neq \lf b) 
\\
\rsubst{\lf t}{\lf a}{\lfnil} 
 & = \lfnil &
\rsubst{\lf t}{\lf a}{(\lf{\lfroot{a}{\spi}})}
 & = \subst{\lf t}{\rsubst{\lf t}{\lf a}{\lf{\spi}}}
\\
& & 
\rsubst{\lf t}{\lf a}{(\lf{\lfroot{h}{\spi}})}
 & = \lfroot{\lf h}{\no{\rsubst{\lf t}{\lf a}{\lf{\spi}}}}
      & ({\it if}~ \lf{h} \neq \lf{a})
\end{align*}
\caption{Hereditary substitution on terms, spines, and classifiers}
\label{fig:lf-hsubst}
\end{figure}

\subsection{Simple types and hereditary substitution}
\label{sec:lf-simpletypesandhsubst}

In addition to LF types like $\lfpi{a}{(\lfpi{z}{(\lfroot{\sf
      a1}{\spi_1})}{\,(\lfroot{\sf
      a2}{\spi_2})})}{\,\lfpi{y}{(\lfroot{\sf
      a3}{\spi_3})}{\,(\lfroot{\sf a4}{\spi_4})}}$, both Canonical LF
and Spine Form LF take {\it simple types} into consideration. The
simple type corresponding to the type above is $({\sf a1} \simplearrow {\sf
  a2}) \simplearrow {\sf a3} \simplearrow {\sf a4}$, where ${\simplearrow}$
associates to the right. The simple type associated with the 
LF type
$\tau$ is given by the function ${\mid}\tau{\mid}^- = \tau_s$, where
${\mid}\lfroot{\sf a}{\spi}{\mid}^- = {\sf a}$ and
${\mid}\lfpi{a}{\tau}{\tau'}{\mid}^- = {\mid}\tau{\mid}^- \simplearrow
{\mid}\tau'{\mid}^-$. 


\begin{figure}
\begin{align*}
\fbox{$\lf{\sigma}(\lf{\spi})$}&
&
\fbox{$\lf{\sigma}(\lf{t'})$}
\\
\lf{\sigma}(\lf{t'; \spi}) 
 & = \lf{\no{\lf{\sigma}(\lf{t'})}; \no{\lf{\sigma}(\lf{\spi})}} &
\lf{\sigma}(\lf{\lambda a.t'}) 
 & = \lf{\lambda a.\,\no{\lf{(\sigma, a/\!\!/a)}(\lf{t'})}}
 & (\lf{a} \# \lf{\sigma})
\\
\lf{\sigma}\lfnil 
 & = \lfnil &
\lf{\sigma}(\lf{\lfroot{a}{\spi}}) 
 & = \subst{\lf t}{\no{\lf{\sigma}(\lf{\spi})}}
      & \lf{t/a} \in \lf{\sigma} 
\\
& &
\lf{\sigma}(\lf{\lfroot{a}{\spi}}) 
 & = \lf{\lfroot{b}{\no{\lf{\sigma}(\lf{\spi})}}} 
      & \lf{b/\!\!/a} \in \lf{\sigma} 
\\
& &
\lf{\sigma}(\lf{\lfroot{\sf c}{\spi}}) 
 & = \lf{\lfroot{\sf c}{\no{\lf{\sigma}(\lf{\spi})}}} 
\end{align*}\begin{align*}
\fbox{$\lf{\sigma}{\nu}$} &
\\
\lf{\sigma}(\lfpi{b}{\nu}{\nu'})
 & = \lfpi{b}{\lf{\sigma}\nu}{\,\lf{(\sigma, b/\!\!/b)}\nu'}
     \qquad (\lf a \neq \lf b) 
\\
\lf{\sigma}({\sf type})
  & = {\sf type}
\\ 
\lf{\sigma}({\sf prop}) 
 & = {\sf prop} 
\\
\lf{\sigma}({\sf prop}\,{\sf ord}) 
 & = {\sf prop}\,{\sf ord} 
\\
\lf{\sigma}({\sf prop}\,{\sf lin}) 
 & = {\sf prop}\,{\sf lin} 
\\
\lf{\sigma}({\sf prop}\,{\sf pers}) 
 & = {\sf prop}\,{\sf pers} 
\\
\lf{\sigma}(\lfroot{\sf a}{\spi}) 
 & = \lfroot{\sf a}{{\lf{\sigma\spi}}}
\end{align*}
\caption{Simultaneous substitution on terms, spines, and classifiers}
\label{fig:simsubst}
\end{figure}

Variables and constants are treated as having an intrinsic simple
type; these intrinsic simple types are sometimes written explicitly as
annotations $\lf{a}^{\tau_s}$ or $\lf{\sf c}^{\tau_s}$ (see
\cite{pfenning08church} for an example), but we will leave them
implicit.  An atomic term $\lf{h\,t_1\ldots t_n}$ must have a
simple atomic type ${\sf a}$. This means that the head $\lf h$ must
have simple type $\tau_{s1} \simplearrow \ldots \simplearrow \tau_{sn} \simplearrow
{\sf a}$ and each $\lf{t_i}$ much have simple type
$\tau_{si}$. Similarly, a lambda term $\lf{\lambda a. t}$ must have
simple type $\tau_s \simplearrow \tau_s'$ where $\lf a$ is a variable with
simple type $\tau_s$ and $\lf t$ has simple type $\tau_s'$.

Simple types, which are treated in full detail elsewhere
\cite{harper07mechanizing,reed09hybrid}, are critical because they
allow us to define hereditary substitution and hereditary reduction as
total functions in Figure~\ref{fig:lf-hsubst}. Intrinsically-typed
Spine Form LF terms correspond to the proof terms for a focused
presentation of (non-dependent) minimal logic. Hereditary reduction
$\subst{\lf{t}}{\lf{\spi}}$ and hereditary substitution $\rsubst{\lf
  t}{\lf a}{\lf{t'}}$, which are both implicitly indexed by the simple
type $\tau_s$ of $\lf t$, capture the computational content of
structural cut admissibility on these proof terms. Informally, the
action of hereditary substitution is to perform a substitution into a
term and then continue to reduce any $\beta$-redexes that would
introduced by a traditional substitution operation.  Therefore,
$\rsubst{\lf{\lambda x. x}}{\lf f}{\lf{({\sf a}\,(f\,{\sf
      b})\,(f\,{\sf c}))}}$ is not $\lf{{\sf a}\,((\lambda x.x)\,{\sf
    b})\,((\lambda x.x)\,{\sf c})}$ -- that's not even a syntactically
well-formed term according to the grammar for Spine Form LF. Rather,
the result of that substitution is $\lf{{\sf a}\,{\sf b}\,{\sf c}}$.

\subsection{Judgments}

Hereditary substitution is necessary to define simultaneous
substitution into types and terms in Figure~\ref{fig:simsubst}.  We
will treat simultaneous substitutions in a mostly informal way,
relying on the more careful treatment by Nanevski et
al.~\cite{nanevski08contextual}. A substitution takes every variable
in the context and either substitutes a term for it (the form
$\lf{t/a,\sigma}$) or substitutes another variable for it (the form
$\lf{b/\!\!/a,\sigma}$). The latter form is helpful for defining
identity substitutions, which we write as $\lf{\sf id}$ or $\lf{\sf
  id}_\Psi$, as well as generic substitutions $\lf{[t/a]}$ that act
like the identity on all variables except for $\lf{a}$; the latter
notation is used in the definition of LF typing in in
Figure~\ref{fig:lf-form}, which is adapted to Spine Form LF from
Harper and Licata's Canonical LF presentation
\cite{harper07mechanizing}. The judgments $\lf{a}\#\lf{\sigma}$,
$\lf{a}\#\Psi$, $\lf{\sf c}\#\Sigma$, ${\sf a}\#\Sigma$, and ${\sf
  r}\#\Sigma$ assert that the relevant variable or constant does not
already appear in the context $\Psi$ (as a binding $\lf{a}{:}\tau$),
the signature $\Sigma$ (as a declaration $\lf{\sf c} : \tau$, ${\sf a}
: \nu$, or ${\sf r} : A^-$), or the substitution $\lf{\sigma}$ (as a
binding $\lf{t/a}$ or \mbox{$\lf{b/\!\!/a}$}).

\begin{figure}
\fbox{$\vdash_\subord \Sigma\,{\sf sig}$}\vspace{-10pt}
\[
\infer
{\vdash_\subord \cdot\,{\sf sig} \mathstrut}
{}
\quad
\infer
{\vdash_\subord (\Sigma, \lf{\sf c} : \tau)\,{\sf sig} \mathstrut}
{\vdash_\subord \Sigma\,{\sf sig} 
 &
 \cdot \vdash_{\Sigma,\subord} \tau\,{\sf type}
 &
 \tau \prec_\subord \tau
 &
 \lf{\sf c} \# \Sigma \mathstrut}
\]
\[
\infer
{\vdash_\subord (\Sigma, {\sf a} : \kappa)\,{\sf sig} \mathstrut}
{\vdash_\subord \Sigma\,{\sf sig}
 &
 \vdash_{\Sigma, \subord} \kappa \,{\sf kind}
 &
 \kappa  \sqsubset_\subord {\sf a} 
 &
 {\sf a} \# \Sigma\mathstrut}
\quad
\infer
{\vdash_\subord (\Sigma, {\sf r} : A^-)\,{\sf sig} \mathstrut}
{\vdash_\subord \Sigma\,{\sf sig}
 &
 \vdash_{\Sigma, \subord} A^- \,{\sf prop}^-
 &
 {\sf r} \# \Sigma \mathstrut}
\]

\medskip
\fbox{$\vdash_{\Sigma,\subord} \Psi\,{\sf ctx}$} -- presumes
  $\vdash_{\subord} \Sigma\,{\sf sig}$\vspace{-10pt}
\[
\infer
{\vdash_{\Sigma,\subord} \cdot\,{\sf ctx} \mathstrut}
{}
\quad
\infer
{\vdash_{\Sigma,\subord} (\Psi, \lf{a}{:}\tau)\,{\sf ctx} \mathstrut}
{\vdash_{\Sigma,\subord} \Psi\,{\sf ctx}
 &
 \Psi \vdash_{\Sigma, \subord} \tau\,{\sf type}
 &
 \lf a \# \Psi}
\]

\medskip
\fbox{$\Psi \vdash_{\Sigma,\subord} \kappa\,{\sf kind}$} -- presumes
  $\vdash_{\Sigma, \subord} \Psi\,{\sf ctx}$
\[
\infer
{\Psi \vdash_{\Sigma,\subord} (\lfpi{a}{\tau}{\kappa})\,{\sf kind} \mathstrut}
{\Psi \vdash_{\Sigma,\subord} \tau\,{\sf type}
 &
 \Psi, \lf{a}{:}\tau \vdash_{\Sigma,\subord} \kappa\,{\sf kind}}
\quad
\infer{\Psi \vdash_{\Sigma,\subord} {\sf type}\,{\sf kind} \mathstrut}{}
\quad
\infer{\Psi \vdash_{\Sigma,\subord} {\sf prop}\,{\sf kind} \mathstrut}{}
\]
\[
\infer
{\Psi \vdash_{\Sigma,\subord} ({\sf prop}\,{\sf ord})\,{\sf kind}\mathstrut}{}
\quad
\infer
{\Psi \vdash_{\Sigma,\subord} ({\sf prop}\,{\sf lin})\,{\sf kind}\mathstrut}{}
\quad
\infer
{\Psi \vdash_{\Sigma,\subord} ({\sf prop}\,{\sf pers})\,{\sf kind}\mathstrut}{}
\]

\medskip
\fbox{$\Psi \vdash_{\Sigma,\subord} \tau\,{\sf type}$} -- presumes
  $\vdash_{\Sigma, \subord} \Psi\,{\sf ctx}$
\[
\infer
{\Psi \vdash_{\Sigma,\subord}(\lfpi{a}{\tau}{\tau'})\,{\sf type} \mathstrut}
{\Psi \vdash_{\Sigma,\subord} \tau\,{\sf type}
 &
 \Psi, \lf{a}{:}\tau \vdash_{\Sigma,\subord} \tau'\,{\sf type}
 &
 \tau \preceq_\subord \tau' \mathstrut}
\quad
\infer
{\Psi \vdash_{\Sigma,\subord}(\lfroot{\sf a}{\spi})\,{\sf type} \mathstrut}
{a{:}\kappa \in \Sigma
 &
 \Psi, [\kappa] \vdash_{\Sigma,\subord} \lf{\spi} : {\sf type}
 \mathstrut}
\]

\medskip
\fbox{$\Psi \vdash_{\Sigma,\subord} \lf t : \tau$} -- presumes 
  $\Psi \vdash_{\Sigma,\subord} \tau\,{\sf type}$
\[
\infer
{\Psi \vdash_{\Sigma,\subord} \lf{\lambda a.t} : \lfpi{x}{\tau}{\tau'}\mathstrut}
{\Psi, \lf{a}{:}\tau \vdash_{\Sigma,\subord} \lf{t} : \tau'\mathstrut}
\quad
\infer
{\Psi \vdash_{\Sigma,\subord} \lf{\lfroot{\sf c}{\spi}} : p
 \mathstrut}
{\lf{\sf c} : \tau \in {\Sigma}
 &
 \Psi, [\tau] \vdash_{\Sigma,\subord} \lf{\spi} : \tau'
 &
 \tau' = p\mathstrut}
\]
\[
\infer
{\Psi \vdash_{\Sigma,\subord} \lf{\lfroot{a}{\spi}} : p
 \mathstrut}
{\lf{a} {:} \tau \in {\Psi}
 &
 \Psi, [\tau] \vdash_{\Sigma,\subord} \lf{\spi} : \tau'
 &
 \tau' = p}
\]

\medskip
\fbox{$\Psi, [\nu] \vdash_{\Sigma,\subord} \lf{\spi} : \nu_0$} --
presumes that either $\Psi \vdash_{\Sigma,\subord} \nu\, {\sf type}$
or that $\Psi \vdash_{\Sigma,\subord} \nu\, {\sf kind}$
\[
\infer
{\Psi, [\nu] \vdash_{\Sigma,\subord} \lfnil : \nu \mathstrut}
{}
\quad
\infer
{\Psi, [\lfpi{a}{\tau}{\nu}] \vdash_{\Sigma,\subord} \lf{t; \spi} : \nu_0
 \mathstrut}
{\Psi \vdash_{\Sigma,\subord} \lf{t} : \tau
 &
 \lf{[t/a]}\nu = \nu'
 &
 \Psi, [\nu'] \vdash_{\Sigma,\subord} \lf{\spi} : \nu_0 \mathstrut}
\]

\medskip
\fbox{$\Psi \vdash \lf{\sigma} : \Psi'$} -- presumes
 $\vdash_{\Sigma,\subord} \Psi\,{\sf ctx}$
 and
 $\vdash_{\Sigma,\subord} \Psi'\,{\sf ctx}$
\[
\infer
{\Psi \vdash_{\Sigma,\subord} \cdot : \cdot \mathstrut}
{}
\quad
\infer
{\Psi \vdash_{\Sigma,\subord} (\lf{\sigma, t/a}) : \Psi', \lf{a}{:}\tau
  \mathstrut}
{\Psi \vdash_{\Sigma,\subord} \lf{\sigma} : \Psi' 
 &
 \Psi \vdash_{\Sigma,\subord} \lf{t} : \lf{\sigma}\tau 
  \mathstrut}
\quad
\infer
{\Psi \vdash_{\Sigma,\subord} (\lf{\sigma, b/\!\!/a}) : \Psi', \lf{a}{:}\tau
  \mathstrut}
{\Psi \vdash_{\Sigma,\subord} \lf{\sigma} : \Psi'
 &
 \lf{b}{:}\lf{\sigma}\tau \in \Psi
  \mathstrut}
\]

\caption{LF formation judgments ($\tau' = p$ refers to $\alpha$-equivalence)}
\label{fig:lf-form}
\end{figure}

All the judgments in Figure~\ref{fig:lf-form} are indexed by a
transitive {\it subordination relation} $\subord$, similar to the one
introduced by Virga in \cite{virga99higherorder}. The subordination
relation is used to determine a term or variable of type $\tau_1$ can
be a (proper) subterm of a term of type $\tau_2$. Uses of
subordination appear in the definition of well-formed equality
propositions $\lf t \doteq_\tau \lf s$ in
Section~\ref{sec:slsframework}, in the preservation proofs in
Section~\ref{sec:gen-destinations}, and in adequacy arguments (as
discussed in \cite{harper07mechanizing}). We treat $\subord$ as a
binary relation on type family constants.  Let ${\sf head}(\tau) =
{\sf a}$ if $\tau =
\lfpi{a_1}{\tau_{1}}{\,.\,.\lfpi{a_{m}}{\tau_{m}}{\,\lfroot{\sf
      a}{\spi}}}$. The signature formation operations depend on three
judgments. The index subordination judgment, $\kappa \sqsubset_\subord
{\sf a}$, relates type family constants to types.
%
It is always the case that $\kappa =
\lfpi{a_1}{\tau_1}{\ldots\lfpi{a_n}{\tau_n}{\sf type}}$, and the
judgment $\kappa \sqsubset_\subord {\sf a}$ holds if $({\sf
  head}(\tau_i), {\sf a}) \in \subord$ for $1 \leq i \leq n$.
%
The type subordination judgment $\tau \prec_\subord \tau'$ holds if
$({\sf head}(\tau), {\sf head}(\tau')) \in \subord$, and the judgment
$\tau \preceq_\subord \tau'$ is the symmetric extension of this
relation.

In Figure~\ref{fig:lf-form}, we define the judgments $\vdash_\subord
\Sigma\,{\sf sig}$, which takes a context $\Sigma$ and determines
whether it is well-formed.  The premise $\tau \prec_\subord \tau$ is
used in the definition of term constants to enforce that only
self-subordinate types can have constructors. This, conversely, means
that types that are not self-subordinate can only be inhabited by
variables $\lf{a}$, which is important for one of the two types of
equality $\lf{t} \doteq_\tau \lf{s}$ that \sls~supports. The judgments
$\vdash_{\Sigma,\subord} \Psi\,{\sf ctx}$, $\Psi
\vdash_{\Sigma,\subord} \kappa\,{\sf kind}$, and $\Psi
\vdash_{\Sigma,\subord} \tau\,{\sf type}$ similarly take contexts
$\Psi$, kinds $\kappa$, and types $\tau$ and ensure that they are
well-formed in the current signature or (if applicable) context.  The
judgment $\Psi \vdash_{\Sigma,\subord} \lf{t}{:} \tau$ takes a term
and a type and typechecks the term against the type, and the judgment
$\Psi \vdash_{\Sigma,\subord} \lf{\sigma} : \Psi'$ checks that a
substitution $\lf{\sigma}$ can transport objects (terms, types, etc.)
defined in the context $\Psi'$ to objects defined in $\Psi$.

The judgment $\Psi, [\nu] \vdash_{\Sigma,\subord} \lf\spi : \nu_0$ is
read a bit differently than these other judgments. The notation, first
of all, is meant to evoke the (exactly analogous) left-focus judgments
from Chapters~\ref{chapter-foc}~and~\ref{chapter-order}. 
In most other sources (for example, in
\cite{cervesato02linear}) this judgment is instead written as $\Psi
\vdash_{\Sigma,\subord} \lf\spi : \nu > \nu_0$. In either case, we
read this judgment as checking a spine $\lf\spi$ against a classifier
$\nu$ (actually either a type $\tau$ or a kind $\kappa$) and {\it
  synthesizing} a return classifier $\nu_0$. In other words, $\nu_0$
is an output of the judgment $\Psi, [\nu] \vdash_{\Sigma,\subord}
\lf\spi : \nu_0$, and given that this judgment presumes that either
$\Psi \vdash_{\Sigma,\subord} \nu\,{\sf type}$ or $\Psi
\vdash_{\Sigma,\subord} \nu\,{\sf kind}$, it {\it ensures} that either
$\Psi \vdash_{\Sigma,\subord} \nu_0\,{\sf type}$ or $\Psi
\vdash_{\Sigma,\subord} \nu_0\,{\sf kind}$, where the classifiers of
$\nu$ and $\nu_0$ (${\sf type}$ or ${\sf kind}$) always match.  It is
because $\nu_0$ is an output that we add an explicit premise to check
that $\tau' = p$ in the typechecking rule for $\lf{\lfroot{\sf
    c}{\spi}}$; this equality refers to the $\alpha$-equality of Spine
Form LF terms.

There are a number of well-formedness theorems that we need to
consider, such as the fact that substitutions compose in a
well-behaved way and that hereditary substitution is always
well-typed.  However, as these theorems are adequately covered in the
aforementioned literature on LF, we will proceed with using LF as a
term language and will treat term-level operations like substitution
somewhat informally.

We will include annotations for the signature $\Sigma$ and the
subordination relation $\subord$ in the definitions of this section
and the next one. In the following sections and chapters, however, we
will often leave the signature $\Sigma$ implicit when it is
unambiguous or unimportant. We will almost always leave the
subordination relation implicit; we can assume where applicable that
we are working with the {\it strongest} (that is, the smallest)
subordination relation for the given signature
\cite{harper07mechanizing}.

\subsection{Adequacy}
\label{sec:lf-adequacy}

{\it Adequacy} was the name given by Harper, Honsell, and Plotkin to
the methodology of connecting inductive definitions to the canonical
forms of a particular type family in LF. Consider, as a standard
example, the untyped lambda calculus, which is generally specified by
a BNF grammar such as the following:
\[
\obj{e} ::= \obj{x} \mid \obj{\lambda x.e} \mid \obj{e_1\,e_2}
\]
We can adequately encode this language of terms into LF (with a
subordination relation $\subord$ such that $({\sf exp}, {\sf
  exp}) \in \subord$) by giving the following signature:
\begin{align*}
\Sigma & = \cdot, 
\\
 & ~\quad {\sf exp} : {\sf type}, 
\\
 & ~\quad \lf{\sf app} : 
     \lfpi{a}{{\sf exp}}{\,\lfpi{b}{\sf exp}{\,\sf exp}},
\\
 & ~\quad \lf{\sf lam} : 
     \lfpi{a}{(\lfpi{b}{\sf exp}{\,\sf exp})}{\,\sf exp}
\end{align*}
Note that the variables $\lf{a}$ and $\lf{b}$ are bound by
$\Pi$-quantifiers in the declaration of $\lf{\sf app}$ and $\lf{\sf lam}$ but
never used. The usual convention is to abbreviate
$\lfpi{a}{\tau}{\tau'}$ as $\tau \rightarrow \tau'$ when $\lf{a}$ is
not free in $\tau'$, which would give $\lf{\sf app}$ type ${\sf exp}
\rightarrow {\sf exp} \rightarrow {\sf exp}$ and $\lf{\sf lam}$ type
$({\sf exp} \rightarrow {\sf exp}) \rightarrow {\sf exp}$.

\bigskip
\begin{theorem}[Adequacy for terms]\label{thm:expadequacy}
  Up to standard $\alpha$-equivalence, there is a bijection between
  expressions $\obj{e}$ (with free variables in the set
  $\{\obj{x_1},\ldots,\obj{x_n}\}$) and Spine Form LF terms $\lf{t}$ such
  that $\lf{x_1}{:}\mathsf{exp}, \ldots, \lf{x_n}{:}\mathsf{exp} \vdash
  \lf{t} : \mathsf{exp}$. 
\end{theorem}

\begin{proof}
By induction on the structure of the inductive definition of $\obj{e}$
in the forward direction and by induction on the structure of 
terms $\lf{t}$ with type ${\sf exp}$ in the reverse direction.
\end{proof}

We express the constructive content of this theorem as a bijective
function $\interp{e} = \lf{t}$ from object language terms $\obj{e}$ to
representations LF terms $\lf{t}$ of type ${\sf exp}$:
\smallskip
\begin{itemize}
\item $\interp{x} = \lf{x}$, 
\item $\interp{e_1\,e_2} = \lf{{\sf
    app}\,\interp{e_1}\,\interp{e_2}}$, and
\item  $\interp{\lambda x.e} =
\lf{{\sf lam}\,\lambda x.\,\interp{e}}$.
\end{itemize}
\smallskip If we had also defined
substitution $\obj{[e/x]e'}$ on terms, it would be necessary to show
that the bijection is compositional: that is, that
$[\interp{e}/\lf{x}]\interp{e'} = \interp{[e/x]e'}$.  Note that
adequacy critically depends on the context having the form
$\lf{x_1}{:}{\sf exp},\ldots,\lf{x_n}{:}{\sf exp}$. If we had a
context with a variable $\lf{y}{:}({\sf exp} \rightarrow {\sf exp})$,
then we could form a term $\lf{y\,({\sf lam}\,\lambda x. x)}$ with
type ${\sf exp}$ that does {\it not} adequately encode any term
$\obj{e}$ in the untyped lambda calculus.

One of the reasons subordination is important in practice is that it
allows us to consider the adequate encoding of expressions in contexts
$\Psi$ that have other variables $\lf{x}{:}\tau$ as long as $({\sf
  head}(\tau),{\sf exp}) \notin \subord$. If $\Psi,\lf{x}{:}\tau
\vdash_{\Sigma,\subord} \lf{t} : {\sf exp}$ and $\tau
\not\preceq_\subord {\sf exp}$, then $\lf{x}$ cannot be free in
$\lf{t}$, so $\Psi \vdash_{\Sigma,\subord} \lf{t} : {\sf exp}$ holds as
well. By iterating this procedure, it may be possible to strengthen a
context $\Psi$ into one of the form $\lf{x_1}{:}{\sf
  exp},\ldots,\lf{x_n}{:}{\sf exp}$, in which case we can conclude
that $\lf t = \interp{e}$ for some untyped lambda calculus term $\obj
e$.



\section{The logical framework \sls}
\label{sec:slsframework}

In this section, we will describe a restricted set of polarized
\ollll~propositions and focused \ollll~proof terms that make up the
logical framework \sls. For the remainder of the thesis, we will work
exclusively with the following positive and negative
\sls~propositions, which are a syntactic refinement of the positive
and negative propositions of polarized \ollll:
\begin{align*}
A^+, B^+, C^+ & ::= p^+ \mid p^+_\meph \mid p^+_\mpers \mid {\downarrow}A^-
  \mid {\gnab}A^- \mid {!}A^- \mid \one \mid A^+ \fuse B^+
  \mid \exists \lf{a}{:}\tau.A^+ \mid \lf{t} \doteq_\tau \lf{s}
\\
A^-, B^-, C^- & ::= p^- \mid {\ocircle}A^+ \mid A^+ \lefti B^- 
  \mid A^+ \righti B^- \mid A^- \with B^-
  \mid \forall \lf{a}{:}\tau.A^-
\end{align*}
%Aside from the type annotation $\tau$ on unification $\lf{t}
%\doteq_\tau \lf{s}$ and on the quantifiers $\forall \lf{x}{:}\tau. A^-$
%and $\exists \lf{x}{:}\tau. A^+$, which we will in general leave implicit,
%this is exactly a refinement of the  
%Notably missing from this refinement are
%upshifts ${\uparrow}A^+$ and right-permeable atomic propositions
%$p^-_\mlax$.
% We will also continue to
% avoid using variable names $x$ with inverting positive propostiions
% and focused negative propositions. In the case of focused
% propositions, there is a straightforward justification: the form of
% context ensures there is at most one of them. In the case of positive
% propositions, we are justified by the convention discussed in the last
% chapter that we only ever frame off the leftmost positive proposition
% in the context.
We now have to deal with a point of notational dissonance: all
existing work on CLF, all existing implementations of CLF, and the
prototype implementation of \sls~(Section~\ref{sec:prototype}) use the
notation $\{ A^+ \}$ for the connective internalizing the judgment
$\islax{A^+}$, which we have written as ${\ocircle}A^+$, following
Fairtlough and Mendler \cite{fairtlough95propositional}. The
traditional notation overloads curly braces, which we also use for
the context-framing notation 
$\tackon{\Theta}{\Delta}$ introduced in Section~\ref{sec:contexts}. We
will treat ${\ocircle}A^+$ and $\{ A^+ \}$ as synonyms in \sls, preferring the
former in this chapter and the latter afterwards.

Positive ordered atomic propositions $p^+$ are atomic classifiers
${\sf a}\,\lf{t_1}\ldots\lf{t_n}$ with kind ${\sf prop}\,{\sf ord}$,
positive linear and persistent atomic propositions $p^+_\meph$ and
$p^+_\mpers$ are (respectively) atomic classifiers with kind ${\sf
  prop}\,{\sf lin}$ and ${\sf prop}\,{\sf pers}$, and negative ordered
atomic propositions $p^-$ are atomic classifiers with kind ${\sf
  prop}$.  From this point on, we will unambiguously refer to atomic
propositions $p^-$ as negative atomic propositions, omitting
``ordered.'' Similarly, we will refer to atomic propositions $p^+$,
$p^+_\meph$, and $p^+_\mpers$ collectively as positive atomic
propositions but individually as ordered, linear, and persistent
propositions, respectively, omitting ``positive.''  (``Mobile'' and
``ephemeral'' will continue to be used as synonyms for ``linear.'')

\subsection{Propositions}

\begin{figure}
\fbox{$\Psi; \mathcal S \vdash_{\Sigma,\subord} A^+\,{\sf prop}^+$} -- presumes
  $\vdash_{\Sigma,\subord} \Psi\,{\sf ctx}$ and $\mathcal S \subseteq \Psi$
\[
\infer
{\Psi; \mathcal S
   \vdash_{\Sigma,\subord} \lfroot{\sf a}{\spi}\,{\sf prop}^+ \mathstrut}
{{\sf a}{:}\kappa \in \Sigma
 &
 \Psi, [\kappa] \vdash_{\Sigma,\subord} \lf{\spi} : {\sf prop}\,{\sf ord} \mathstrut}
\quad
\infer
{\Psi; \mathcal S
   \vdash_{\Sigma,\subord} \lfroot{\sf a}{\spi}\,{\sf prop}^+ \mathstrut}
{{\sf a}{:}\kappa \in \Sigma
 &
 \Psi, [\kappa] \vdash_{\Sigma,\subord} \lf{\spi} : {\sf prop}\,{\sf lin} \mathstrut}
\]
\[
\infer
{\Psi; \mathcal S
   \vdash_{\Sigma,\subord} \lfroot{\sf a}{\spi}\,{\sf prop}^+ \mathstrut}
{{\sf a}{:}\kappa \in \Sigma
 &
 \Psi, [\kappa] \vdash_{\Sigma,\subord} \lf{\spi} : {\sf prop}\,{\sf pers} \mathstrut}
\]
\[
\infer
{\Psi; \mathcal S \vdash_{\Sigma,\subord} {\downarrow}A^-\,{\sf prop}^+ \mathstrut}
{\Psi; \cdot \vdash_{\Sigma,\subord} A^-\,{\sf prop}^- \mathstrut}
\quad
\infer
{\Psi; \mathcal S \vdash_{\Sigma,\subord} {\gnab}A^-\,{\sf prop}^+ \mathstrut}
{\Psi; \cdot \vdash_{\Sigma,\subord} A^-\,{\sf prop}^- \mathstrut}
\quad
\infer
{\Psi; \mathcal S \vdash_{\Sigma,\subord} {!}A^-\,{\sf prop}^+ \mathstrut}
{\Psi; \cdot \vdash_{\Sigma,\subord} A^-\,{\sf prop}^- \mathstrut}
\quad
\infer
{\Psi; \mathcal S \vdash_{\Sigma,\subord} \one\,{\sf prop}^+ \mathstrut}
{}
\] 
\[
\infer
{\Psi; \mathcal S_1, \mathcal S_2 \vdash_{\Sigma,\subord} A^+ \fuse B^+\,{\sf prop}^+ \mathstrut}
{\Psi; \mathcal S_1 \vdash_{\Sigma,\subord} A^+\,{\sf prop}^+ 
 &
 \Psi; \mathcal S_2 \vdash_{\Sigma,\subord} B^+\,{\sf prop}^+  \mathstrut}
\quad
\infer
{\Psi; \mathcal S \vdash_{\Sigma,\subord} \exists \lf{a}{:}\tau. A^+\,{\sf prop}^+ \mathstrut}
{\Psi \vdash_{\Sigma,\subord} \tau\,{\sf type}
 &
 \Psi, \lf{a}{:}\tau; \mathcal S, \lf{a}{:}\tau \vdash_{\Sigma,\subord} A^+\,{\sf prop}^+ \mathstrut}
\] 
\[
\infer
{\Psi; \mathcal S \vdash_{\Sigma,\subord} \lf{t} \doteq_p \lf{s}\,{\sf prop}^+}
{\Psi \vdash_{\Sigma,\subord} p\,{\sf type}
 &
 \Psi \vdash_{\Sigma,\subord} \lf{t} : p
 &
 \Psi \vdash_{\Sigma,\subord} \lf{s} : p
 & 
 p \not\prec_\subord p}
\]
\[
\infer
{\Psi; \mathcal S \vdash_{\Sigma,\subord} \lf{a} \doteq_p \lf{s}\,{\sf prop}^+}
{\Psi \vdash_{\Sigma,\subord} p\,{\sf type}
 &
 \Psi \vdash_{\Sigma,\subord} \lf{a} : p
 &
 \Psi \vdash_{\Sigma,\subord} \lf{s} : p
 & 
 \lf{a}{:}p \in \mathcal S}
\]


\medskip
\fbox{$\Psi \vdash_{\Sigma,\subord} A^-\,{\sf prop}^-$} -- presumes
  $\vdash_{\Sigma,\subord} \Psi\,{\sf ctx}$ and
  $\mathcal S \subseteq \Psi$
\[
\infer
{\Psi; \mathcal S
   \vdash_{\Sigma,\subord} \lfroot{\sf a}{\spi}\,{\sf prop}^- \mathstrut}
{{\sf a}{:}\kappa \in \Sigma
 &
 \Psi, [\kappa] \vdash_{\Sigma,\subord} \lf{\spi} : {\sf prop} \mathstrut}
\quad
\infer
{\Psi; \mathcal S \vdash_{\Sigma,\subord} {\ocircle}A^+\,{\sf prop}^- \mathstrut}
{\Psi; \cdot \vdash_{\Sigma,\subord} A^+ : {\sf prop}^+ \mathstrut}
\]
\[
\infer
{\Psi; \mathcal S_1, \mathcal S_2 \vdash_{\Sigma,\subord} A^+ \lefti B^-\,{\sf prop}^- \mathstrut}
{\Psi; \mathcal S_1 \vdash_{\Sigma,\subord} A^+\,{\sf prop}^+ 
 &
 \Psi; \mathcal S_2 \vdash_{\Sigma,\subord} B^-\,{\sf prop}^-  \mathstrut}
\quad
\infer
{\Psi; \mathcal S_1, \mathcal S_2 \vdash_{\Sigma,\subord} A^+ \righti B^-\,{\sf prop}^- \mathstrut}
{\Psi; \mathcal S_1 \vdash_{\Sigma,\subord} A^+\,{\sf prop}^+ 
 &
 \Psi; \mathcal S_2 \vdash_{\Sigma,\subord} B^-\,{\sf prop}^-  \mathstrut}
\] 
\[
\infer
{\Psi; \mathcal S \vdash_{\Sigma,\subord} A^- \with B^-\,{\sf prop}^- \mathstrut}
{\Psi; \mathcal S \vdash_{\Sigma,\subord} A^-\,{\sf prop}^- 
 &
 \Psi; \mathcal S \vdash_{\Sigma,\subord} B^-\,{\sf prop}^-  \mathstrut}
\quad
\infer
{\Psi; \mathcal S \vdash_{\Sigma,\subord} \forall \lf{a}{:}\tau. A^-\,{\sf prop}^- \mathstrut}
{\Psi; \mathcal S \vdash_{\Sigma,\subord} \tau\,{\sf type}
 &
 \Psi, \lf{a}{:}\tau; \mathcal S, \lf{a}{:}\tau \vdash_{\Sigma,\subord} A^-\,{\sf prop}^- \mathstrut}
\] 
\caption{\sls~proposition formation judgments}
\label{fig:sls-propform}
\end{figure}

The formation judgments for \sls~types are given in
Figure~\ref{fig:sls-propform}.  As discussed in the introduction to
this chapter, the removal of ${\uparrow}A^+$ and $p^-_\mlax$ is
fundamental to the separation of the deductive and concurrent
fragments of \sls; all the other restrictions made to the language are
for the purpose of giving partial proofs a linear structure.  In
particular, all positive propositions whose left rules have more or
less than one premise are restricted. The propositions $\zero$ and
$A^+ \oplus B^+$ are excluded from \sls~to this end, and we must place
rather draconian restrictions on the use of equality in order to ensure
that $\doteq_L$ can always be treated as having exactly one premise.

The formation rules for propositions are given in
Figure~\ref{fig:sls-propform}. Much of the complexity of this
presentation, such as the existence of an additional context $\mathcal
S$ that is treated like a multiset, is needed to support the inclusion
of equality \sls.  The restrictions ensure that, whenever we decompose
a positive proposition $\lf{s} \doteq \lf{t}$ on the left, we have
that $\lf{s} = \lf{a}$ for some variable $\lf{a}$ in the context. When
this is the case, $\lf{[t/a]}$, is always a most general unifier of
$\lf{s} = \lf{a}$ and $\lf{t}$, which in turn means that the left rule
for equality in \ollll
\[
\infer[{\doteq}_L]
{\foc{\Psi}{\frameoff{\Theta}{\lf{t} \doteq_\tau \lf{s}}}{U}}
{\forall(\Psi' \vdash \lf{\sigma} : \Psi).
 &
 \lf{\sigma t} = \lf{\sigma s}
 &
 \longrightarrow
 &
 \foc{\Psi'}{\tackon{\lf{\sigma}\Theta}{\cdot}}{\lf{\sigma} U}
 }
\]
is equivalent to a much simpler rule:
\[
\infer[{\doteq}_{\it yes}]
{\foc{\Psi, \lf{a}{:}\tau, \Psi'}{\frameoff{\Theta}{\lf{a} \doteq_\tau \lf{t}}}{U}}
{\foc{\Psi, \lf{[t/a]}\Psi'}{\tackon{\lf{[t/a]}\Theta}{\cdot}}{\lf{[t/a]} U}}
\]
Usually, when we require the ``existence of most general unifiers,''
that signals that a most general unifier must exist if any unifier
exists. The condition we are requiring is much stronger: for the
unification problems we will encounter due to the $\doteq_L$ rule, a
most general unifier {\it must} exist. Allowing unification problems
that could fail would require us to consider positive inversion rules
with zero premises, and the proposition $\zero$ was excluded from
\sls~precisely to prevent us from needing to deal with positive
inversion rules with zero premises.\footnote{The other side of this
  observation is that, if we allow the proposition $\zero$ and adapt
  the logical framework accordingly, it might be possible to reduce or
  eliminate the restrictions we have placed on equality.}

There are two distinct conditions under which we can be sure that
unification problems always have a most general solution -- when
equality is performed over {\it pure variables} and when equality is
used as a {\it notational definition}
\cite{pfenning99algorithms}. Equality of pure variable types is
necessary for the destination-adding transformation in
Chapter~\ref{chapter-destinations}, and notational definitions are
used extensively Chapter~\ref{chapter-approx}.

\paragraph{Pure variables} Equality at an atomic type $p$ that is {\it
  not subordinate to itself} ($p \not\prec_\subord p$) is always
allowed.  This is reflected in the first formation rule for $\lf{t}
\doteq \lf{s}$ in Figure~\ref{fig:sls-propform}.

Types that are not self-subordinate can only be inhabited by
variables: that is, if $p \not\prec_\subord p$ and $\Psi
\vdash_{\Sigma,\subord} \lf{t} : p$, then $\lf{t} = \lf{{a}}$ where
$\lf{a}:p \in \Psi$. For any unification problem $\lf{{a}} \doteq
\lf{{b}}$, both $\lf{[{a}/b]}$ and $\lf{[{b}/a]}$ are most general
unifiers.


\paragraph{Notational definitions} Using equality as a notational
definition allows us manipulate propositions in ways that have no
effect on the structure of synthetic inference rules. The subset
$\mathcal S$ of the LF context, which is treated as an a multiset,
enforces that each variable bound by an existential quantifier
$\exists \lf{a}{:}p.\,A^+$ or a universal quantifier $\forall
\lf{a}{:}p.\,A^-$ can be associated with at most one proposition
$\lf{a} \doteq \lf{t}$, where $\lf{t}$ is arbitrary. This condition is
handled by the second formation rule for $\lf{t} \doteq \lf{s}$ in
Figure~\ref{fig:sls-propform}. The rule ${\ocircle}(\exists \lf{a}.\,
\lf{a} \doteq \lf{t} \fuse \lf{a} \doteq \lf{s})$ does not satisfy
this condition because the introduced variable $\lf{a}$ is associated
with the left-hand side of two different equalities, $\lf{a} \doteq
\lf{t}$ and $\lf{a} \doteq \lf{s}$, that together encode an arbitrary
unification problem $\lf{t} \doteq \lf{s}$.


The proposition $\lf{a} \doteq \lf{t}$ must be reachable from its
associated quantifier without crossing a shift or an exponential -- in
Andreoli's terms, it must be in the same {\it monopole}
(Section~\ref{sec:linsynthetic}).  This is enforced by the formation
rules for shifts and exponentials, which clear the subset $\mathcal S$
in their premise.  The proposition $\forall \lf{a}.\,
{\downarrow}({\sf p}\,\lf{a}) \lefti \lf{a} \doteq \lf{t} \lefti {\sf
  p}\,\lf{t}$ satisfies this condition but $\forall \lf{a}.\,\forall
\lf{b}.\,{\ocircle}(\lf{a} \doteq \lf{b})$ does not ($\ocircle$ breaks
focus), and the proposition ${\ocircle}(\exists \lf{a}. \lf{a} \doteq
\lf{t})$ satisfies this condition but ${\ocircle}(\exists \lf{a}.
{\uparrow}(\lf{a} \doteq \lf{t} \lefti {\sf p}\,\lf{a}))$ does not
(${\uparrow}$ breaks focus).

%   This restriction ensures that, if the unification appears on the
%   left, the left-hand side of the unification will always be a
%   variable $\lf{a}$, meaning that $\lf{[t/a]}$ is always a most
%   general unifier.  This usage of unification is essentially just a
%   notational definition \cite{pfenning99algorithms}.


\subsection{Substructural contexts}

\begin{figure}[t]
\fbox{$\Psi \vdash_{\Sigma,\subord} T\,{\sf left} \mathstrut$} -- presumes
  $\vdash_{\Sigma,\subord} \Psi\,{\sf ctx}$
\[
\infer
{\Psi \vdash_{\Sigma,\subord} (\islvl{A^-})\,{\sf left} \mathstrut}
{\Psi; \cdot \vdash_{\Sigma,\subord} A^-\,{\sf prop}^- \mathstrut}
\quad
\infer
{\Psi \vdash_{\Sigma,\subord} 
   (\istrue{\susp{\lfroot{\sf a}{\spi}}})\,{\sf left} \mathstrut}
{{\sf a} : \kappa \in \Sigma
 & 
 \Psi, [\kappa] \vdash_{\Sigma,\subord} \lf{\spi} : {\sf prop}\,{\sf ord}
 \mathstrut}
\]
\[
\quad
\infer
{\Psi \vdash_{\Sigma,\subord} 
   (\iseph{\susp{\lfroot{\sf a}{\spi}}})\,{\sf left} \mathstrut}
{{\sf a} : \kappa \in \Sigma
 & 
 \Psi, [\kappa] \vdash_{\Sigma,\subord} \lf{\spi} : {\sf prop}\,{\sf lin}
 \mathstrut}
\quad
\infer
{\Psi \vdash_{\Sigma,\subord} 
   (\ispers{\susp{\lfroot{\sf a}{\spi}}})\,{\sf left} \mathstrut}
{{\sf a} : \kappa \in \Sigma
 & 
 \Psi, [\kappa] \vdash_{\Sigma,\subord} \lf{\spi} : {\sf prop}\,{\sf pers}
 \mathstrut}
\]

\medskip
\fbox{$\Psi \vdash_{\Sigma,\subord} \Delta\,{\sf stable} \mathstrut$} -- presumes
  $\vdash_{\Sigma,\subord} \Psi\,{\sf ctx}$
\[
\infer
{\Psi \vdash_{\Sigma,\subord} \cdot\,{\sf stable} \mathstrut}
{}
\quad
\infer
{\Psi \vdash_{\Sigma,\subord} (\Delta, x{:}T)\,{\sf stable} \mathstrut}
{\Psi \vdash_{\Sigma,\subord} \Delta\,{\sf stable}
 &
 \Psi \vdash_{\Sigma,\subord} T\,{\sf left}}
\]

\medskip
\fbox{$\Psi \vdash_{\Sigma,\subord} \Delta\,{\sf inv} \mathstrut$} -- presumes
  $\vdash_{\Sigma,\subord} \Psi\,{\sf ctx}$
\[
\infer
{\Psi \vdash_{\Sigma,\subord} \cdot\,{\sf inv} \mathstrut}
{}
\quad
\infer
{\Psi \vdash_{\Sigma,\subord} (\Delta, x{:}T)\,{\sf inv} \mathstrut}
{\Psi \vdash_{\Sigma,\subord} \Delta\,{\sf inv}
 &
 \Psi \vdash_{\Sigma,\subord} T\,{\sf left}}
\quad
\infer
{\Psi \vdash_{\Sigma,\subord} (\Delta, x{:}\istrue{A^+})\,{\sf inv} \mathstrut}
{\Psi \vdash_{\Sigma,\subord} \Delta\,{\sf inv}
 &
 \Psi; \mathcal C \vdash_{\Sigma,\subord} A^+\,{\sf prop}^+}
\]

\medskip
\fbox{$\Psi \vdash_{\Sigma,\subord} \Delta\,{\sf infoc} \mathstrut$} -- presumes
  $\vdash_{\Sigma,\subord} \Psi\,{\sf ctx}$
\[
\infer
{\Psi \vdash_{\Sigma,\subord} (\Delta, x{:}T)\,{\sf infoc} \mathstrut}
{\Psi \vdash_{\Sigma,\subord} \Delta\,{\sf infoc}
 &
 \Psi \vdash_{\Sigma,\subord} T\,{\sf left}}
\quad
\infer
{\Psi \vdash_{\Sigma,\subord} (\Delta, x{:}\istrue{[A^-]})\,{\sf infoc} 
 \mathstrut}
{\Psi \vdash_{\Sigma,\subord} \Delta\,{\sf stable}
 &
 \Psi; \mathcal C \vdash_{\Sigma,\subord} A^-\,{\sf prop}^-
 %&
 %\Psi \vdash_{\Sigma,\subord} \lf{\sigma} : \Psi, \Psi'
 }
\]
\caption{\sls~contexts}
\label{fig:sls-ctxform}
\end{figure}


Figure~\ref{fig:sls-ctxform} describes the well-formed substructural
contexts in \sls. The judgment $\Psi \vdash_{\Sigma,\subord} T\,{\sf
  left}$ is used to check stable bindings $x{:}\islvl{A^-}$ and
$z{:}\islvl{p^+_\mlvl}$ that can appear as a part of stable,
inverting, or focused sequents; the judgment $\Psi
\vdash_{\Sigma,\subord} \Delta\,{\sf stable}$ just maps this judgment
over the context. The judgment $\Psi \vdash_{\Sigma,\subord}
\Delta\,{\sf inv}$ describes contexts during the inversion phase,
which can also contain inverting positive propositions $A^+$. The
judgment $\Psi \vdash_{\Sigma,\subord} \Delta\,{\sf infoc}$ describes
a context that is stable aside from the one negative proposition in
focus. 

The last rule in Figure~\ref{fig:sls-ctxform} is strange because it
appears to make up a substitution $\lf\sigma$ out of thin air.  This
is necessary because the property of being a well-formed proposition
is not stable under substitution. Therefore, even though
$\forall\lf{a}{:}p.\,(\lf{a} \doteq_p \lf{\sf c}) \lefti A^-$ is a
well-formed negative type according to Figure~\ref{fig:sls-propform},
$(\lf{\sf c} \doteq_p \lf{\sf c}) \lefti \lf{[{\sf c}/a]}A^-$ is not.
Therefore, to explain why the intermediate states of a focusing phase
are well-formed, we need to keep track of the substitutions that we
have applied to the negative proposition; the proposition only needs
to be well-formed prior to applying that substitution for the logic to
behave correctly. The same trick will need to be played with the rules
for right-focus in Figure~\ref{fig:sls-values}.

The restrictions we make to contexts justify our continued practice of
omitting the $\mtrue$ annotation when talking about inverting positive
propositions $A^+$ or focused negative propositions $[A^-]$ in the
context, since these context constituents only appear in conjunction
with the $\mtrue$ judgment. 

This discussion of well-formed propositions and contexts takes care of
any issues dealing with variables that were swept under the rug in
Chapter~\ref{chapter-order}.  We could stop here and use the
refinement of \ollll~proof terms that corresponds to our refinement of
propositions as the language of \sls~proof terms. This is not
desirable for two main reasons. First, the proof terms of
focused \ollll~make it inconvenient (though not impossible) to talk
about concurrent equality
(Section~\ref{sec:framework-concurrenteq}). Second, one of our primary
uses of \sls~in this thesis will be to talk about {\it traces}, which
correspond roughly to partial proofs
\[
\deduce
{\foc{\Psi}{\Delta}{\islax{A^+}}\mathstrut}
{\deduce{\vdots\mathstrut}
  {\vspace{-4pt}\foc{\Psi'}{\Delta'}{\islax{A^+}}}}
\]
in \ollll, where both the top and bottom sequents are stable and where
$A^+$ is some unspecified, parametric positive proposition. Using
\ollll-derived proof terms makes it difficult to talk about about and
manipulate proofs of this form.

In the remainder of this section, we will present a proof term
assignment for \sls~that facilitates discussing concurrent equality
and partial proofs. \sls~proof terms are in bijective correspondence
with a refinement of \ollll~proof terms when we consider complete
(deductive) proofs, but the introduction of patterns and traces
reconfigures the structure of derivations and proof terms.

\subsection{Process states}

A {\it process state} is a disembodied left-hand side of a sequent that
we use to describe the intermediate states of concurrent systems. Traces,
which we will type in terms of process states in Section~\ref{sec:framework-concurrent}, are intended to capture the structure of partial proofs:
\[
\deduce
{\foc{\Psi}{\Delta}{\islax{A^+}}\mathstrut}
{\deduce{\vdots\mathstrut}
  {\vspace{-4pt}\foc{\Psi'}{\Delta'}{\islax{A^+}}}}
\]
as a relation between two 
process states. As a first cut, we can represent the initial state as 
$(\Psi; \Delta)$ and the final state as $(\Psi'; \Delta')$, and we can
omit $\Psi$ and just write $\Delta$ when that is sufficiently clear.

Representing a process state as merely an LF context $\Psi$ and a
substructural context $\Delta$ is insufficient because of the way
equality -- pure variable equality in particular -- can unify distinct
variables. Consider the following partial proof:
\[
\deduce
{\foc{\lf{a}{:}p, \lf{b}{:}p}
  {~~x{:}\iseph{{\ocircle}(\lf{a} \doteq_\tau \lf{b})}, ~~
   z{:}\iseph{\susp{{\sf foo}\,\lf{a}\,\lf{a}}}}
  {\islax{({\sf foo}\,\lf{a}\,\lf{b})}}\mathstrut}
{\deduce{\vdots\mathstrut}
  {\vspace{-4pt}\foc{\lf{b}{:}p}
   {~~ z{:}\iseph{\susp{{\sf foo}\,\lf{b}\,\lf{b}}}}
   {\islax{({\sf foo}\,\lf{b}\,\lf{b})}}}}
\]
This partial proof can be constructed in one focusing stage by a left 
focus on $x$. It is insufficient to capture the first process
state as 
$\left({\lf{a}{:}p, \lf{b}{:}p}; ~~
 {x{:}{\ocircle}(\lf{a} \doteq_\tau \lf{b}), ~~
  z{:}\iseph{\susp{{\sf foo}\,\lf{a}\,\lf{a}}}}\right)$
and the second process state as
$\left({\lf{b}{:}p};~~
 {z{:}\iseph{\susp{{\sf foo}\,\lf{b}\,\lf{b}}}}\right)$, as this would fail to 
capture that the succedent $\islax{({\sf foo}\,\lf{b}\,\lf{b})}$
is a substitution instance of
the succedent
$\islax{({\sf foo}\,\lf{a}\,\lf{b})}$. 
%
In general, if the derivation above proved some arbitrary
succedent $\islax{A^+}$ instead of $\islax{({\sf
    foo}\,\lf{a}\,\lf{b})}$, then the missing subproof has the succedent
$\islax{\lf{[b/a]}A^+}$.

A process state is therefore written as $(\Psi; \Delta)_{\lf{\sigma}}$
and is well-formed under signature $\Sigma$ and subordination relation
$\subord$ if $\Psi \vdash_{\Sigma,\subord} \Delta\,{\sf inv}$
(which presumes that $\vdash_{\Sigma,\subord} \Psi\,{\sf ctx}$, as
defined in Figure~\ref{fig:lf-form}) and if $\Psi \vdash \lf{\sigma} :
\Psi_0$, where $\Psi_0$ is some other context that represents the
starting point, the context in which the disconnected succedent
$\islax{A^+}$ is well-formed.
\[
\infer
{\vdash_{\Sigma,\subord} (\Psi; \Delta)_{\lf{\sigma}}\,{\sf state}
 \mathstrut}
{%\vdash_{\Sigma,\subord} \Psi : {\sf ctx}
 %&
 \Psi \vdash_{\Sigma,\subord} \Delta\,{\sf inv}
 &
 \vdash_{\Sigma,\subord} \Psi_0 : {\sf ctx}
 &
 \Psi \vdash \lf{\sigma} : \Psi_0
 \mathstrut}
\]
Taking $\Psi_0 = \lf{a}{:}p, \lf{b}{:}p$, the partial proof above can
thus be represented as a step (Section~\ref{sec:framework-concurrent})
between these two process states:
\[
\left({\lf{a}{:}p, \lf{b}{:}p}; ~~
 {x{:}{\ocircle}(\lf{a} \doteq_\tau \lf{b}),  ~~
  z{:}\iseph{\susp{{\sf foo}\,\lf{a}\,\lf{a}}}}\right)_{\lf{(a/a,\,b/b)}}
\leadsto_{\Sigma,\subord}
\left({\lf{b}{:}p}; ~~
 {z{:}\iseph{\susp{{\sf foo}\,\lf{b}\,\lf{b}}}}\right)_{\lf{(b/a,\,b/b)}}
\]

Substitutions are just one of several ways that we could cope with
free variables in succedents; another option, discussed in
Section~\ref{sec:constraint-store-solution}, is to track the set of
constraints $\lf{a} = \lf{b}$ that have been encountered by
unification.  When we consider traces in isolation, we will generally
let $\Psi_0 = \cdot$ and $\lf{\sigma} = \lf \cdot$, which corresponds
to the case where the parametric conclusion $A^+$ is a closed
proposition. When the substitution is not mentioned, it can therefore
be presumed to be empty. Additionally, when the LF context $\Psi$ is
empty or clear from the context, we will omit it as well. One further
simplification is that we will occasionally omit the judgment $\mlvl$
associated with a suspended positive atomic proposition
$\islvl{\susp{p^+_\mlvl}}$, but only when it is unambiguous from the
current signature that $p^+_\mlvl$ is an ordered, linear, or
persistent positive atomic proposition. In the examples above, we
tacitly assumed that ${\sf foo}$ was given kind $p \rightarrow p
\rightarrow {\sf prop}\,{\sf lin}$ in the signature $\Sigma$ when we
tagged the suspended atomic propositions with the judgment $\meph$. If
it had been clear that ${\sf foo}$ was linear, then this judgment
could have been omitted.

%In almost all cases, the substitution $\lf\sigma$ associated with a process
%state is the identity and will be omitted, and the LF context $\Psi$
%will frequently be omitted as well. 

\begin{figure}
\fbox{$P :: (\Psi; \Delta)_{\lf{\sigma}}
         \Longrightarrow_{\Sigma,\subord}
           (\Psi'; \Delta')_{\lf{\sigma'}}$}
 -- presumes
 $\vdash_{\Sigma,\subord} (\Psi; \Delta)_{\lf{\sigma}}\,{\sf state}$
\[
\infer[()]
{() :: (\Psi; \Delta)_{\lf{\sigma}}
           \Longrightarrow_{\Sigma,\subord} 
       (\Psi; \Delta)_{\lf{\sigma}}}
{\Psi \vdash_{\Sigma,\subord} \Delta\,{\sf stable}}
\]
\[
\infer[\eta^+]
{z,P :: (\Psi; \frameoff{\Theta}{p^+_\mlvl})_{\lf{\sigma}}
           \Longrightarrow_{\Sigma,\subord} 
        (\Psi'; \Delta')_{\lf{\sigma'}}}
{P :: (\Psi; \tackon{\Theta}{z{:}\islvl{\susp{p^+_\mlvl}}})_{\lf{\sigma}}
           \Longrightarrow_{\Sigma,\subord}
        (\Psi'; \Delta')_{\lf{\sigma'}}}
\quad
\infer[{\downarrow}_L]
{x,P :: (\Psi; \frameoff{\Theta}{{\downarrow}A^-})_{\lf{\sigma}}
           \Longrightarrow_{\Sigma,\subord}
        (\Psi'; \Delta')_{\lf{\sigma'}}}
{P :: (\Psi; \tackon{\Theta}{x{:}\istrue{A^-}})_{\lf{\sigma}}
           \Longrightarrow_{\Sigma,\subord}
        (\Psi'; \Delta')_{\lf{\sigma'}}}
\]
\[
\infer[{\gnab}_L]
{x,P :: (\Psi; \frameoff{\Theta}{{\gnab}A^-})_{\lf{\sigma}}
           \Longrightarrow_{\Sigma,\subord}
        (\Psi'; \Delta')_{\lf{\sigma'}}}
{P :: (\Psi; \tackon{\Theta}{x{:}\iseph{A^-}})_{\lf{\sigma}}
           \Longrightarrow_{\Sigma,\subord}
        (\Psi'; \Delta')_{\lf{\sigma'}}}
\quad
\infer[{\bang}_L]
{x,P :: (\Psi; \frameoff{\Theta}{{!}A^-})_{\lf{\sigma}}
           \Longrightarrow_{\Sigma,\subord}
        (\Psi'; \Delta')_{\lf{\sigma'}}}
{P :: (\Psi; \tackon{\Theta}{x{:}\ispers{A^-}})_{\lf{\sigma}}
           \Longrightarrow_{\Sigma,\subord}
        (\Psi'; \Delta')_{\lf{\sigma'}}}
\]
\[
\infer[{\one}_L]
{P :: (\Psi; \frameoff{\Theta}{\one})_{\lf{\sigma}}
           \Longrightarrow_{\Sigma,\subord}
        (\Psi'; \Delta')_{\lf{\sigma'}}}
{P :: (\Psi; \tackon{\Theta}{\cdot})_{\lf{\sigma}}
           \Longrightarrow_{\Sigma,\subord}
        (\Psi'; \Delta')_{\lf{\sigma'}}}
\quad 
\infer[{\fuse}_L]
{P :: (\Psi; \frameoff{\Theta}{A^+ \fuse B^+})_{\lf{\sigma}}
           \Longrightarrow_{\Sigma,\subord}
        (\Psi'; \Delta')_{\lf{\sigma'}}}
{P :: (\Psi; \tackon{\Theta}{\mkconj{A^+}{B^+}})_{\lf{\sigma}}
           \Longrightarrow_{\Sigma,\subord}
        (\Psi'; \Delta')_{\lf{\sigma'}}}
\]
\[
\infer[{\exists}_L]
{\lf{a},P :: (\Psi; \frameoff{\Theta}{\exists \lf{a}{:}\tau. A^+})_{\lf{\sigma}}
           \Longrightarrow_{\Sigma,\subord}
        (\Psi'; \Delta')_{\lf{\sigma'}}}
{P :: (\Psi, \lf{a}{:}\tau; \tackon{\Theta}{A^+})_{\lf{\sigma}}
           \Longrightarrow_{\Sigma,\subord}
        (\Psi'; \Delta')_{\lf{\sigma'}}}
\]
\[
\infer[{\doteq}_L]
{\lf{t/a},P :: 
  (\Psi, \lf{a}{:}\tau, \Psi'; \frameoff{\Theta}{\lf{a} \doteq_\tau \lf{t}}
   )_{\lf{\sigma}}
           \Longrightarrow_{\Sigma,\subord}
        (\Psi'; \Delta')_{\lf{\sigma'}}}
{P :: (\Psi, \lf{[t/a]}\Psi'; {\tackon{\lf{[t/a]}\Theta}{\cdot}}
       )_{\lf{[t/a]\sigma}}
           \Longrightarrow_{\Sigma,\subord}
        (\Psi'; \Delta')_{\lf{\sigma'}}}
\]
\caption{\sls~patterns}
\label{fig:sls-patterns}
\end{figure}


\subsection{Patterns}
\label{sec:framework-patterns}

 % With the restrictions we have made to unificaiton, we can 
% treat $\doteq_L$ as always having one premise, so all remaining
% left rules for positive atomic propositions have exactly one 
% premise. 

%
A {\it pattern} is a syntactic entity that captures the linear
structure of left inversion on positive propositions. The \ollll~proof
term for the proposition
%
$(\exists \lf{a}.\,{\sf p}\,\lf{a} 
             \fuse {\gnab}A^-
             \fuse {\downarrow}B^-) \lefti C^-$,
%
is somewhat inscrutable:
${\tlaml{\texistsl{\lf{a}}
    {\tfusel{\tfusel{\tetap{x}{\tgnabl{y}{\tdownl{z}{N}}}}}}}}$. The
\sls~proof of this proposition, which uses patterns, is
$(\lambda \lf{a},x,y,z.\, N)$. The pattern $P = \lf{a}, x,y,z$ captures
the structure of left inversion on the positive proposition 
$\exists \lf{a}.\,{\sf p}\,\lf{a} 
             \fuse {\gnab}A^-
             \fuse {\downarrow}B^-$.

The grammar of patterns is straightforward.
% 
Inversion on positive propositions
can only have the effect of introducing new bindings (either LF
variables $\lf{a}$ or \sls~variables $x$) or handling a unification
$\lf{a} \doteq_p \lf{t}$, which by our discussion above can always be
resolved by the most general unifier $\lf{[t/a]}$, so the pattern associated
with a proposition $\lf{a} \doteq_p \lf{t}$ is $\lf{t/a}$. 
\[
P ::= () \mid x, P \mid \lf{a}, P \mid \lf{t/a}, P
\] 
For sequences with one or more elements, we omit the trailing
comma and $()$, writing $x, \ldots, z$ 
instead of $x, \ldots, z, ()$. 

\sls~patterns have a linear structure (the comma is right associative)
because they capture the linear structure of proofs. The associated
decomposition judgment which takes two process states: $P :: (\Psi;
\Delta)_{\lf{\sigma}} \Longrightarrow_{\Sigma,\subord} (\Psi';
\Delta')_{\lf{\sigma'}}$, and operates a bit like the spine typing
judgment from Figure~\ref{fig:lf-form} in that the process state
$(\Psi; \Delta)_{\lf{\sigma}}$ (and the pattern $P$) are treated as an
input and the process state $(\Psi'; \Delta')_{\lf{\sigma'}}$ is
treated as an output.  The typing rules for \sls~patterns are given in
Figure~\ref{fig:sls-patterns}. We preserve the side conditions from
the previous chapter: when we frame off a inverting positive
proposition in the process state, it is required to be the left-most
one. As in focused \ollll, this justifies our omission of the
variables associated with positive propositions: the positive
proposition we frame off is always uniquely identified not by its
associated variable but by its position in the context.

Note that there no longer appears to be a one-to-one
correspondence between proof terms and rules: ${\downarrow}_L$,
${\gnab}_L$, and ${!}_L$ appear to have the same proof term, and
${\one}_L$ and ${\fuse}_L$ appear to have no proof term at all. To
view patterns as being intrinsically typed -- that is, to view them as
actual representatives of (incomplete) derivations -- we must think of
patterns as carrying extra annotations that allow them to continue
matching the structure of proof rules.

\subsection{Values, terms, and spines}
\label{sec:framework-deductive}

Notably missing from the \sls~types are the upshifts ${\uparrow}A^+$
and right-permeable negative atomic propositions $p^-_\mlax$. The
removal of these two propositions effectively means that the succedent
of a stable \sls~sequent can only be $\istrue{\susp{p^-}}$ or
$\islax{A^+}$. The \sls~framework only considers {\it complete} proofs
of judgments $\istrue{\susp{p^-}}$, whereas traces, associated with
proofs of $\islax{A^+}$ and introduced below in
Section~\ref{sec:framework-concurrent}, are a proof term assignment
for partial proofs. Excepting the proof term $\tlet{T}{V}$, which we
present as part of the {\it concurrent} fragment of \sls~in
Section~\ref{sec:framework-concurrent} below, the values, terms, and
spines that stand for complete proofs will be referred to as the {\it
  deductive fragment} of \sls.
\begin{align*}
&\mbox{\sls~values (Figure~\ref{fig:sls-values})}& 
V & ::= z
   \mid N
   \mid \tgnabr{N}
   \mid \tbangr{N}
   \mid \toner
   \mid \tfuser{V_1}{V_2}
   \mid \texistsr{\lf{t}}{V}
   \mid \tunifr
\\
&\mbox{\sls~atomic terms (Figure~\ref{fig:sls-atomic-terms})}&
R & ::= \tfocusl{x}{\Sp} 
   \mid \tfocusl{\sf r}{\Sp} 
\\
&\mbox{\sls~terms (Figure~\ref{fig:sls-terms})}&
N & ::= R
   \mid \lambda P.N 
   \mid N_1 \with N_2
   \mid \tforallr{\lf{a}}{N}
   \mid \tlet{T}{V}
\\
&\mbox{\sls~spines (Figure~\ref{fig:sls-spines})}&
\Sp & ::= \tnil 
   \mid V; \Sp
   \mid \pi_1; \Sp 
   \mid \pi_2; \Sp
   \mid \lf{t}; \Sp
\end{align*}

In contrast to \ollll, we distinguish the syntactic category $R$ of
atomic terms that correspond to stable sequents. As with patterns, we
appear to conflate the proof terms associated with different proof
rules -- we have a single $\lambda P.N$ constructor and a single
$V;\Sp$ spine rather than one term $\tlamr{N}$ and spine
$\tappr{V}{\Sp}$ associated with propositions $A^+ \righti B^-$ and
another term $\tlaml{N}$ and spine $\tappl{V}{\Sp}$ associated with
propositions $A^+ \lefti B^-$.  As with patterns, it is possible to
think of these terms as just having extra annotations ($\lambda^>$ or
$\lambda^<$) that we have omitted.  Without these annotations, proof
terms carry less information than derivations, and the rules for
values, terms, and spines in
Figures~\ref{fig:sls-values}--\ref{fig:sls-spines} must be seen as
typing rules. With these extra implicit annotations (or, possibly,
with some of the technology of bidirectional typechecking), values,
terms, and spines can continue to be seen as representatives of
derivations.

\begin{figure}
\fbox{$\foctx{\Psi}{\Delta}{V}{[A^+]}$} -- presumes
  $\Psi \vdash_{\Sigma,\subord} \Delta\,{\sf stable}$, 
  and  $\Psi; \mathcal C \vdash_{\Sigma,\subord} A^+\,{\sf prop}^+$
\[
\infer[{\it id}^+]
{\foctx{\Psi}{\Delta}{z}{[A^+]}}
{\Delta \mbox{\it ~matches~} z{:}\susp{A^+}}
\]
\[
\infer[{\downarrow}_R]
{\foctx{\Psi}{\Delta}{\tdownr{N}}{[{\downarrow}A^-]}}
{\foctx{\Psi}{\Delta}{N}{A^-}}
\quad
\infer[{\gnab}_R]
{\foctx{\Psi}{\restrictto{\Delta}{\meph}}{\tgnabr{N}}{[{\gnab}A^-]}}
{\foctx{\Psi}{\Delta}{N}{A^-}}
\quad
\infer[{!}_R]
{\foctx{\Psi}{\restrictto{\Delta}{\mpers}}{\tbangr{N}}{[{!}A^-]}}
{\foctx{\Psi}{\Delta}{N}{A^-}}
\]
\[
\infer[{\one}_R]
{\foctx{\Psi}{\Delta}{\toner}{[\one]}}
{\Delta \mbox{\it ~matches~} \cdot}
\quad
\infer[{\fuse}_R]
{\foctx{\Psi}
  {\matchconj{\Delta_1}{\Delta_2}}{\tfuser{V_1}{V_2}}{[A^+_1 \fuse A^+_2]}}
{\foctx{\Psi}{\Delta_1}{V_1}{[A^+_1]}
 &
 \foctx{\Psi}{\Delta_2}{V_2}{[A^+_2]}}
\]
\[
\infer[{\exists}_R]
{\foctx{\Psi}{\Delta}{\texistsr{\lf{t}}{V}}{[\exists \lf{a}{:}\tau. A^+]}}
{\Psi; \Delta \vdash_{\Sigma,\subord} \lf{t} : \tau
 &
 \foctx{\Psi}{\Delta}{V}{[\lf{[t/a]}A^+]}}
\quad
\infer[{\doteq}_R]
{\foctx{\Psi}{\cdot}{\tunifr}{\lf{t} \doteq \lf{t}}}
{\Delta \mbox{\it ~matches~} \cdot}
\]
\caption{\sls~values}
\label{fig:sls-values}
\end{figure}

\begin{figure}
\fbox{$\foctx{\Psi}{\Delta}{R}{U}$} -- presumes
  $\Psi \vdash_{\Sigma,\subord} \Delta\,{\sf stable}$ and
  $U = \istrue{\susp{C^-}}$
\[
\infer[{\it focus}_L]
{\foctx{\Psi}{\frameoff{\Theta}{x{:}A^-}}{\tfocusl{x}{\Sp}}{U}}
{\foctx{\Psi}{\tackon{\Theta}{[A^-]}}{\Sp}{U}}
\quad
\infer[{\it rule}]
{\foctx{\Psi}{\frameoff{\Theta}{\cdot}}{\tfocusl{\sf r}{\Sp}}{U}}
{{\sf r} : A^- \in \Sigma 
 &
 \foctx{\Psi}{\tackon{\Theta}{[A^-]}}{\Sp}{U}}
\]
\caption{\sls~atomic terms}
\label{fig:sls-atomic-terms}
\end{figure}

\begin{figure}
\fbox{$\foctx{\Psi}{\Delta}{N}{\istrue{A^-}}$} -- presumes
  $\Psi \vdash_{\Sigma,\subord} \Delta\,{\sf stable}$ and
  $\Psi; \mathcal C \vdash_{\Sigma,\subord} A^-\,{\sf prop}^-$
\[
\infer[\eta^-]
{\foctx{\Psi}{\Delta}{R}{\istrue{p^-}}}
{\foctx{\Psi}{\Delta}{R}{\istrue{\susp{p^-}}}}
\]
\[
\infer[{\lefti}_R]
{\foctx{\Psi}{\Delta}{\lambda P.N}{\istrue{A^+ \lefti B^-}}}
{P :: (\Psi; \mkconj{A^+}{\Delta})_{\lf{\sf id}_{\Psi}} 
  \Longrightarrow_{\Sigma,\subord}
 (\Psi';\Delta')_{\lf{\sigma}}
 &
 \foctx{\Psi'}{\Delta'}{N}{\istrue{\lf{\sigma}B^-}}}
\]
\[
\infer[{\righti}_R]
{\foctx{\Psi}{\Delta}{\lambda P.N}{\istrue{A^+ \righti B^-}}}
{P :: (\Psi; \mkconj{\Delta}{A^+})_{\lf{\sf id}_{\Psi}} 
  \Longrightarrow_{\Sigma,\subord}
 (\Psi';\Delta')_{\lf{\sigma}}
 &
 \foctx{\Psi'}{\Delta'}{N}{\istrue{\lf{\sigma}B^-}}}
\]
\[
\infer[{\with}_R]
{\foctx{\Psi}{\Delta}{\twithr{N_1}{N_2}}{\istrue{A_1^- \with A_2^-}}}
{\foctx{\Psi}{\Delta}{N_1}{\istrue{A_1^-}}
 &
 \foctx{\Psi}{\Delta}{N_2}{\istrue{A_2^-}}}
\]
\[
\infer[{\forall}_R]
{\foctx{\Psi}{\Delta}{\tforallr{\lf{a}}{N}}
    {\istrue{\forall \lf{a}{:}\tau. A^-}}}
{\foctx{\Psi, \lf{a}{:}\tau}{\Delta}{N}{\istrue{A^-}}}
\]
\[
\infer[{\ocircle}_R]
{\foctx{\Psi}{\Delta}{\tlet{T}{V}}{\istrue{{\ocircle}A^+}}}
{T :: (\Psi; \Delta)_{\lf{\sf id}_{\Psi}}
  \leadsto^*_{\Sigma,\subord}
 (\Psi';\Delta')_{\lf{\sigma'}}
 &
 \foctx{\Psi'}{\Delta'}{V}{[\lf{\sigma}A^+]}}
\]
\caption{\sls~terms}
\label{fig:sls-terms}
\end{figure}

%Spines
\begin{figure}
\fbox{$\foctx{\Psi}{\Delta}{\Sp}{U}$} --
  presumes
  $\Psi \vdash_{\Sigma,\subord} \Delta\,{\sf infoc}$
\[
\infer[{\it id}^-]
{\foctx{\Psi}{{\Delta}}{\tnil}{\istrue{\susp{A^-}}}}
{\Delta \mbox{\it ~matches~} [A^-] \mathstrut}
\]
\[
\infer[{\lefti}_L]
{\foctx{\Psi}{\frameoff{\Theta}{\matchconj{\Delta}{[A^+ \lefti B^-]}}}
  {V; \Sp}{U}}
{\foctx{\Psi}{\Delta}{V}{[A^+]}
 &
 \foctx{\Psi}{\tackon{\Theta}{[B^-]}}{\Sp}{U}}
\]
\[
\infer[{\righti}_L]
{\foctx{\Psi}{\frameoff{\Theta}{\matchconj{[A^+ \righti B^-]}{\Delta}}}
  {V; \Sp}{U}}
{\foctx{\Psi}{\Delta}{V}{[A^+]}
 &
 \foctx{\Psi}{\tackon{\Theta}{[B^-]}}{\Sp}{U}}
\]
\[
\infer[{\with}_{L1}]
{\foctx{\Psi}{\frameoff{\Theta}{[A_1^- \with A_2^-]}}
  {\pi_1; \Sp}{U}}
{\foctx{\Psi}{\tackon{\Theta}{[A_1^-]}}{\Sp}{U}}
\quad
\infer[{\with}_{L2}]
{\foctx{\Psi}{\frameoff{\Theta}{[A_1^- \with A_2^-]}}
  {\pi_2; \Sp}{U}}
{\foctx{\Psi}{\tackon{\Theta}{[A_2^-]}}{\Sp}{U}}
\]
\[
\infer[{\forall}_L]
{\foctx{\Psi}{\frameoff{\Theta}{[\forall \lf{a}{:}\tau. B^-]}}
  {\lf{t}; \Sp}{U}}
{\Psi \vdash_{\Sigma,\subord} \lf{t} : \tau
 &
 \lf{[t/a]}B^- = B'^-
 &
 \foctx{\Psi}{\tackon{\Theta}{[B'^-]}}{\Sp}{U}}
\]
\caption{\sls~spines}
\label{fig:sls-spines}
\end{figure}


Aside from ${\ocircle}_R$ and it's associated term $\tlet{T}{V}$,
which belong to the concurrent fragment of \sls, there is one rules in
Figures~\ref{fig:sls-values}--\ref{fig:sls-spines} that does not have
an exact analogues as rules in \ollll, the rule ${\it rule}$ in
Figure~\ref{fig:sls-atomic-terms}. This rule corresponds to an atomic
term $\tfocusl{\sf r}{\Sp}$ and accounts for the fact that there is an
additional source of persistent facts in \sls, the signature $\Sigma$,
that is not present in \ollll.  To preserve the bijective
correspondence between \ollll~and \sls~proof terms, we need to place
every rule ${\sf r} : A^-$ in the \sls~signature $\Sigma$ into the
corresponding \ollll~context as a persistent proposition.

As with LF terms, we will use a shorthand for atomic terms
$\tfocusl{x}{\Sp}$ and $\tfocusl{\sf r}{\Sp}$, writing $({\sf
  foo}\,\lf{t}\,\lf{s}\,V\,V')$ instead of $\tfocusl{\sf
  foo}{(\lf{t};\lf{s};V;V';\tnil)}$ when we are not concerned with the
fact that atomic terms consist of a variable and a spine.

\subsection{Steps and traces}
\label{sec:framework-concurrent}

The deductive fragment of \sls~presented in
Figures~\ref{fig:sls-patterns}--\ref{fig:sls-spines} covers every
\sls~proposition except for the lax modality ${\ocircle}A^+$. It is in
the context of the lax modality that we will present proof terms
corresponding to partial proofs; we call this fragment the {\it
  concurrent} fragment of \sls~because of its relationship with
concurrent equality, described in
Section~\ref{sec:framework-concurrenteq}.
\begin{align*}
\qquad\qquad\qquad\qquad
&\mbox{Steps}&
S & ::= \trstep{P}{R}
\qquad\qquad\qquad\qquad
\\
&\mbox{Traces}&
T & ::= \emptytrace \mid T_1; T_2 \mid S
\end{align*}

A step $S = \tstep{P}{x}{\Sp}$ corresponds precisely to the notion of
a {\it synthetic inference rule} as discussed in
Section~\ref{sec:linsynthetic}. A
step in \sls~corresponds to a use of left focus, a use of the
left rule for the lax modality, and a use of the 
admissible focal substitution lemma in \ollll:
\[
\infer-[{\it subst}^-]
{\foc{\Psi}{\frameoff{\Theta}{\tackon{\Theta'}{x{:}\istrue{A^-}}}}{\islax{\lf{\sigma}A^+}}}
{\infer[{\it focus}_L]
 {\foc{\Psi}{\tackon{\Theta'}{x{:}\istrue{A^-}}}{\istrue{\susp{{\ocircle}B^+}}}}
 {\deduce
  {\foc{\Psi}{\tackon{\Theta'}{[A^-]}}{\istrue{\susp{{\ocircle}B^+}}}}
  {\vdots\mathstrut}}
 &
 \infer[{\ocircle}_L]
 {\foc{\Psi}{\tackon{\Theta}{[{\ocircle}B^+]}}{\islax{\lf{\sigma}A^+}}}
 {\deduce
  {\foc{\Psi}
    {\tackon{\Theta}{B^+}}
    {\islax{\lf{\sigma}A^+}}\mathstrut} 
  {\deduce{\vdots\mathstrut}
    {\vspace{-4pt}\foc{\Psi'}
     {\Delta'}
     {\islax{\lf{\sigma'}A^+}}}}}}
\]
The spine $\Sp$ corresponds to the complete proof of
$\foc{\Psi}{\tackon{\Theta'}{[A^-]}}{\istrue{\susp{{\ocircle}B^+}}}$,
and the pattern $P$ corresponds to the partial proof from $\foc{\Psi'}
{\Delta'} {\islax{\lf{\sigma'}A^+}}$ to $\foc{\Psi}
{\tackon{\Theta}{B^+}} {\islax{\lf{\sigma}A^+}}$. The typing rules for
steps are given in Figure~\ref{fig:sls-steps}.  Because we understand
these synthetic inference rules as relations between process states,
we call the type of a step a {\it synthetic transition}. Traces $T$
are monoids over steps -- $\emptytrace$ is an empty trace, $S$ is a
trace consisting of a single step, and $T_1; T_2$ is the sequential
composition of traces. The typing rules for traces in
Figure~\ref{fig:sls-traces} straightforwardly reflect this monoid
structure. Both of the judgments $S :: (\Psi; \Delta)_{\lf{\sigma}} 
   \leadsto_{\Sigma,\subord}
  (\Psi'; \Delta')_{\lf{\sigma'}}$ and $T :: (\Psi; \Delta)_{\lf{\sigma}} 
   \leadsto^*_{\Sigma,\subord}
  (\Psi'; \Delta')_{\lf{\sigma'}}$ work like the rules for patterns,
in that the step $S$ or trace $T$ is treated as an input along with
the initial process state $(\Psi; \Delta)_{\lf{\sigma}}$, whereas the final
process state $(\Psi'; \Delta')_{\lf{\sigma'}}$ is treated as an output.

\begin{figure}
\fbox{$S :: (\Psi; \Delta)_{\lf{\sigma}} 
   \leadsto_{\Sigma,\subord}
  (\Psi'; \Delta')_{\lf{\sigma'}}$} -- presumes
  $\vdash_{\Sigma,\subord} (\Psi; \Delta)_{\lf{\sigma}}\,{\sf state}$
  and $\Psi \vdash_{\Sigma,\subord} \Delta\,{\sf stable}$
\[
\infer
{\trstep{P}{R} :: 
  (\Psi; \frameoff{\Theta}{\Delta})_{\lf{\sigma}} 
   \leadsto_{\Sigma,\subord}
  (\Psi'; \Delta')_{\lf{\sigma'}}}
{\foctx{\Psi}{\Delta}{R}{\istrue{\susp{{\ocircle}{B^+}}}}
 &
 P :: (\Psi,\tackon{\Theta}{B^+})_{\lf{\sigma}}
   \Longrightarrow_{\Sigma,\subord}
      (\Psi'; \Delta')_{\lf{\sigma'}}}
\]
\caption{\sls~steps}
\label{fig:sls-steps}
\end{figure}

\begin{figure}
\fbox{$T :: (\Psi; \Delta)_{\lf{\sigma}} 
   \leadsto^*_{\Sigma,\subord}
  (\Psi'; \Delta')_{\lf{\sigma'}}$} -- presumes
  $\vdash_{\Sigma,\subord} (\Psi; \Delta)_{\lf{\sigma}}\,{\sf state}$
  and $\Psi \vdash_{\Sigma,\subord} \Delta\,{\sf stable}$
\[
\infer
{\emptytrace :: (\Psi; \Delta)_{\lf{\sigma}} 
               \leadsto^*_{\Sigma,\subord}
             (\Psi; \Delta)_{\lf{\sigma}}}
{}
\quad
\infer
{S :: (\Psi; \Delta)_{\lf{\sigma}}
               \leadsto^*_{\Sigma,\subord}
             (\Psi'; \Delta')_{\lf{\sigma'}}}
{S :: (\Psi; \Delta)_{\lf{\sigma}}
               \leadsto_{\Sigma,\subord}
             (\Psi'; \Delta')_{\lf{\sigma'}}}
\]
\[
\infer
{T; T' :: (\Psi_1; \Delta_1)_{\lf{\sigma_1}}
               \leadsto^*_{\Sigma,\subord}
             (\Psi_3; \Delta_3)_{\lf{\sigma_3}}}
{T :: (\Psi_1; \Delta_1)_{\lf{\sigma_1}}
               \leadsto^*_{\Sigma,\subord}
             (\Psi_2; \Delta_2)_{\lf{\sigma_2}}
&
T' :: (\Psi_2; \Delta_2)_{\lf{\sigma_2}}
               \leadsto^*_{\Sigma,\subord}
             (\Psi_3; \Delta_3)_{\lf{\sigma_3}}}
\]
\caption{\sls~traces}
\label{fig:sls-traces}
\end{figure}

Steps incorporate left focus and the left rule for ${\ocircle}$, and
{\it let-expressions} $\tlet{T}{V}$, which include traces in deductive
terms, incorporate right focus and the right rule for
the lax modality in \ollll:

\[
\infer[{\ocircle}_R]
{\foc{\Psi}{\Delta}{{\ocircle}A^-}}
{\deduce
  {\foc{\Psi}
    {\Delta}
    {\islax{A^+}}\mathstrut} 
  {\deduce{\vdots\mathstrut}
    {\vspace{-4pt}\infer[]
     {\foc{\Psi'}
      {\Delta'}
      {\islax{\lf{\sigma'}A^+}}}
     {\deduce{\foc{\Psi'}{\Delta'}{[\lf{\sigma'}A^+]}}
      {\vdots}}}}}
\]
The trace $T$ represents the entirety of the partial
proof from $\foc{\Psi}
    {\Delta}
    {\islax{A^+}}$ to $\foc{\Psi'}
      {\Delta'}
      {\islax{\lf{\sigma'}A^+}}$ that proceeds by repeated use of steps
or synthetic transitions, and the eventual conclusion  $V$ represents the 
complete proof of $\foc{\Psi'}
      {\Delta'}
      {\islax{[\lf{\sigma'}A^+]}}$ that follows the series of synthetic
transitions.

Both of the endpoints of a trace are stable sequents, but it will
occasionally be useful to talk about steps and traces that start from
unstable sequents by decomposition positive propositions. We will use
the usual trace notation $(\Psi; \Delta)_{\lf\sigma}
\leadsto^*_{\Sigma,\subord} (\Psi'; \Delta')_{\lf{\sigma'}}$ to
describe the type of these partial proofs. The proof term
associated with this type will be written as $\lambda P.T$, where $P :: (\Psi;
\Delta)_{\lf\sigma} \Longrightarrow_{\Sigma,\subord} (\Psi'';
\Delta'')_{\lf{\sigma''}}$ and $T :: (\Psi'';
\Delta'')_{\lf{\sigma''}} \leadsto^*_{\Sigma,\subord} (\Psi';
\Delta')_{\lf{\sigma'}}$. 


\subsection{Presenting traces}

To present traces in a readable way, we can use a notation that
interleaves process states among the steps of a trace, a common
practice in Hoare-style reasoning \cite{hoare71proof}.  As an example,
recall the series of transitions that our money-store-battery-robot
system took Section~\ref{sec:linlogtrans}: 
%
\[
\begin{array}{ccccc}
\begin{array}{c}
\mbox{\it \$6 (1)}\medskip\\ 
\mbox{\it battery-less robot (1)} \medskip\\ 
\mbox{\it turn \$6 into a battery}\\
\mbox{\it (all you want)}
\end{array}
& \leadsto &
\begin{array}{c}
\mbox{\it battery  (1)}\medskip\\ 
\mbox{\it battery-less robot (1)} \medskip\\ 
\mbox{\it turn \$6 into a battery}\\
\mbox{\it (all you want)}
\end{array}
& \leadsto &
\begin{array}{c}
\mbox{\it robot (1)} \medskip\\ 
\mbox{\it turn \$6 into a battery}\\
\mbox{\it (all you want)}\medskip\\~\\
\end{array}
\end{array}
\]
%
This evolution can now be precisely captured as a trace in \sls:
\begin{align*}
&\qquad\qquad
\left(
 x{:}\iseph{\susp{\sf 6bucks}}, ~~
 f{:}\iseph{({\sf battery} \lefti {\ocircle}{\sf robot})}, ~~
 g{:}\ispers{({\sf 6bucks} \lefti {\ocircle}{\sf battery})}
\right)
\\
&\trstep{y}{g\,x};
\\
&\qquad\qquad
\left(
 y{:}\iseph{\susp{\sf battery}}, ~~
 f{:}\iseph{({\sf battery} \lefti {\ocircle}{\sf robot})}, ~~
 g{:}\ispers{({\sf 6bucks} \lefti {\ocircle}{\sf battery})}
\right)
\\
&\trstep{z}{f\,y}
\\
&\qquad\qquad
\left(
 z{:}\iseph{\susp{\sf robot}}, ~~
 g{:}\ispers{({\sf 6bucks} \lefti {\ocircle}{\sf battery})}
\right)
\end{align*}

\subsection{Frame properties}

The {\it frame rule} is a concept from separation logic
\cite{reynolds02separation}.  It states that if a property holds of
some program, then the property holds under any extension of the
mutable state. The frame rule increases the modularity of separation
logic proofs, because two program fragments that reason about
different parts of the state can be reasoned about independently.

Similar frame properties hold for \sls~traces. The direct analogue of
the frame rule is the observation that a trace can always have some
extra state framed on to the outside. This is a generalization of
weakening to \sls~traces.

\bigskip
\begin{theorem}[Frame weakening]\label{thm:frameweak}~\\
If $T :: (\Psi; \Delta) \leadsto^*_{\Sigma, \subord} (\Psi'; \Delta')$, then
$T :: (\Psi, \Psi''; \tackon{\Theta}{\Delta})
       \leadsto^*_{\Sigma, \subord} (\Psi', \Psi''; \tackon{\Theta}{\Delta})$.
\end{theorem}
\begin{proof}
Induction on $T$ and case analysis on the first steps of $T$, using admissible
weakening and the properties of matching constructs at each step.
\end{proof}

The frame rule is a weakening property which ensures that new,
irrelevant state can can always be added to a state. Conversely, any
state that is never accessed or modified by a trace can be always be 
removed without making the trace ill-typed. This property
is a generalization of strengthening to \sls~traces.

\bigskip
\begin{theorem}[Frame strengthening]\label{thm:framestrong}~\\
  If $T :: (\Psi; \tackon{\Theta}{x{:}\islvl{Y}}) \leadsto^*_{\Sigma,
    \subord} (\Psi'; \tackon{\Theta'}{x{:}\islvl{Y}})$ and $x$ is not
  free in any of the steps of $T$, then $T :: (\Psi;
  \tackon{\Theta}{\cdot}) \leadsto^*_{\Sigma, \subord} (\Psi';
  \tackon{\Theta}{\cdot})$.
\end{theorem}
\begin{proof}
  Induction on $T$ and case analysis on the first steps of $T$, using
  a lemma to enforce that, if $x$ is not free in an individual step,
  it is either not present in the context of the subderivation 
  (if $\mlvl = \mtrue$ or $\meph$) or else
  it can be strengthened away (if $\mlvl = \mpers$).
\end{proof}

\section{Concurrent equality}
\label{sec:linconcurrenteq}
\label{sec:framework-concurrenteq}

Concurrent equality is a notion of equivalence on traces that is
coarser than the equivalence relation we would derive from partial
\ollll~proofs. Consider the following \sls~signature:
\begin{align*}
 \Sigma = \cdot, 
~&{\sf a} : {\sf prop}\,{\sf lin},
~ {\sf b} : {\sf prop}\,{\sf lin},
~ {\sf c} : {\sf prop}\,{\sf lin},
~ {\sf d} : {\sf prop}\,{\sf lin},
~ {\sf e} : {\sf prop}\,{\sf lin},
~ {\sf f} : {\sf prop}\,{\sf lin},
\\ & 
  {\sf first}  : {\sf a} \lefti {\ocircle}({\sf b} \fuse {\sf c}), 
\\ &
  {\sf left}  : {\sf b} \lefti {\ocircle}{\sf d}, ~
\\ &
  {\sf right} : {\sf c} \lefti {\ocircle}{\sf e}, ~
\\ &
  {\sf last} : {\sf d} \fuse {\sf e} \lefti {\ocircle}{\sf f}
\end{align*}
Under the signature $\Sigma$,
we can create two traces with the type
$x_a{:}\susp{\sf a} \leadsto^*_{\Sigma,\subord} x_f{:}\susp{\sf f}$:
\[
\begin{array}{rcl}
T_1 & = 
 & \trstep{x_b, x_c}{\sf first}\,x_a;\\
&& \trstep{x_d}{\sf left}\,x_b;\\
&& \trstep{x_e}{\sf right}\,x_c;\\
&& \trstep{x_f}{\sf last}\,(\tfuser{x_d}{x_e})
\end{array}
\quad
\mbox{versus}
\quad
\begin{array}{rcl}
T_2 & = 
 & \trstep{x_b, x_c}{\sf first}\,x_a;\\
&& \trstep{x_e}{\sf right}\,x_c;\\
&& \trstep{x_d}{\sf left}\,x_b;\\
&& \trstep{x_f}{\sf last}\,(\tfuser{x_d}{x_e})
\end{array}
\]
In both cases, there is an $x_a{:}\susp{\sf a}$ resource
that transitions to a $x_b{:}\susp{\sf b}$ resource and a
$x_c{:}\susp{\sf c}$ resource, and then $x_b{:}\susp{\sf b}$
transitions to $x_d{:}\susp{\sf d}$ while, independently,
$x_c{}\susp{\sf c}$ transitions to $x_d{:}\susp{\sf d}$. Then,
finally, the $x_d{:}\susp{\sf d}$ and $x_e{:}\susp{\sf e}$ combine to
transition to $x_f{:}\susp{\sf f}$, which completes the trace. 

The independence here is key: if two steps consume different
resources, then we want to treat them as independent concurrent steps
that could have equivalently happened in the other order. However, if
we define equivalence only in terms of the $\alpha$-equivalence of
partial \ollll~derivations, the two traces above are distinct. In this
section, we introduce a coarser equivalence relation, {\it concurrent
  equality}, that allows us to treat traces that differ only in the
interleaving of independent and concurrent steps as being equal.  The
previous section considered the proof terms of \sls~as a fragment of
\ollll~better able to talk about partial proofs.  The introduction of
concurrent equality takes a step beyond \ollll, because it breaks the
bijective correspondence between \ollll~proofs and \sls~proofs. As the
example above indicates, there are simply more \ollll~proofs than
\sls~proofs when we quotient the latter modulo concurrent equality and
declare $T_1$ and $T_2$ to be (concurrently) equal.

Concurrent equality was first was introduced and explored in the
context of CLF \cite{watkins02concurrent}, but our presentation
follows the reformulation in \cite{cervesato12trace}, which defines
concurrent equivalence based on an analysis of the variables that are
used (inputs) and introduced (outputs) by a given step.  Specifically,
our strategy will be to take a particular well-typed trace $T$ and
define a set $I$ of pairs of states $(S_1, S_2)$ with the property
that, if $S_1; S_2$ is a well-typed trace, then $S_2; S_1$ is a
concurrently equivalent and well-typed trace.  This {\it independency
  relation} allows us to treat the trace $T$ as a {\it trace
  monoid}. Concurrent equality, in turn, is just $\alpha$-equality of
\sls~proof terms combined with treating $\tlet{T}{V}$ and
$\tlet{T'}{V}$ as equivalent if $T$ and $T'$ are equivalent
according to the equivalence relation imposed by treating $T$ and $T'$
as trace monoids.

This formulation of concurrent equality facilitates applying the rich
theory developed around trace monoids to \sls~traces.  For example, it
is decidable whether two traces $T$ and $T'$ are equivalent as trace
monoids, and there are algorithms for determining whether $T'$ is a
subtrace of $T$ (that is, whether there exist $T_{\it pre}$ and
$T_{\it post}$ such that $T$ is equivalent $T_{\it pre}; T'; T_{\it
  post}$) \cite{diekert90combinatorics}. A different sort of matching
problem, in which we are given $T$, $T_{\it pre}$, and $T_{\it post}$
and must determine whether there exists a $T'$ such that $T$ is
equivalent $T_{\it pre}; T'; T_{\it post}$, was considered in
\cite{cervesato12trace}. However, this treatment is complicated by 

Unfortunately, the presence of equality in \sls~complicates our
treatment of independency. The {\it interface} of a step is used to
define independency on steps $S = (\trstep{P}{R})$. Two components of
the interface, the input variables ${^\bullet}S$ and the output
variables $S{^\bullet}$ are standard in the literature on Petri nets
-- see, for example, \cite[p.~553]{murata89petri}. The third component
(unified variables ${^\circledast}S$) is unique to our presentation.
\bigskip
\begin{definition}[Interface of a step]~
\begin{itemize}
\item The {\em input variables} of a step, denoted ${^\bullet}S$, are
  all the LF variables $\lf{a}$ and \sls~variables $x$ free in the
  normal term $R$.

\item The {\em output variables} of a step $S = (\trstep{P}{R})$,
  denoted by $S{^\bullet}$, are all the LF variables $\lf{a}$ and
  \sls~variables $x$ bound by the pattern $P$ that are not
  subsequently consumed by a substitution $\lf{t/a}$ in the same
  pattern.

\item The {\em unified variables} of a step, denoted by ${^\circledast}S$, are
  the free variables of a step that are modified by unification. If
  $\lf{t/a}$ appears in a pattern and $\lf{a}$ is free in the pattern,
  then $\lf{t} = \lf{b}$ for some other variable $\lf{b}$; both
  $\lf{a}$ and $\lf{b}$ (if the latter is free in the pattern) are
  included in the step's unified variables.
\end{itemize}
\end{definition}
\bigskip

Consider a well-typed trace $S_1; S_2$ with two steps. It is possible,
by renaming variables bound in patterns, to ensure that $\emptyset =
{^\bullet}S_1 \cap S_2{^\bullet} = {^\bullet}S_1 \cap S_1{^\bullet} =
{^\bullet}S_2 \cap S_2{^\bullet}$. We will generally assume that, in the
traces we consider, the variables introduced in each step are renamed
to be distinct from variables bound or free in all previous steps.

If $S_1; S_2$ is a well-typed trace, then the order of $S_1$ and $S_2$
is fixed if $S_1$ introduces variables that are used by $S_2$ -- that
is, if $\emptyset \neq S_1{^\bullet} \cap {^\bullet}S_2$. For example,
if $S_1 = (\trstep{x_b, x_c}{\sf first}\,x_a)$ and $S_2 =
(\trstep{x_d}{\sf left}\,x_b)$, then $\{x_b\} = S_1{^\bullet} \cap
{^\bullet}S_2$, and the two steps cannot be reordered relative to one
another.  Conversely, the condition that $\emptyset = S_1{^\bullet}
\cap {^\bullet}S_2$ is sufficient to allow reordering in a CLF-like
framework \cite{cervesato12trace}, and is also sufficient to allow
reordering in \sls~when neither step contain no unified variables
(that is, when $\emptyset = {^\circledast}S_1 = {^\circledast}S_2$).
The unification driven by equality, however, can have subtle
effects. Consider the following two-step trace:
\begin{align*}
& \qquad\qquad
(\lf{a}{:}p, \lf{b}{:}p; ~~ x{:}\iseph{\left({\ocircle}(\lf{b} \doteq_p \lf{a})\right)}, ~~
 y{:}\iseph{\left({\sf foo}\,\lf{b} 
                 \lefti {\ocircle}({\sf bar}\, \lf{a})\right)}, ~~
 z{:}\iseph{\susp{{\sf foo}\,\lf{a}}})
\\
& \trstep{\lf{b/a}}{x};
\\
& \qquad\qquad
(\lf{b}{:}p; ~~ 
 y{:}\iseph{\left({\sf foo}\,\lf{b} 
                 \lefti {\ocircle}({\sf bar}\, \lf{b})\right)}, ~~
 z{:}\iseph{\susp{{\sf foo}\,\lf{b}}})
\\
& \trstep{w}{y\,z} 
\\
& \qquad\qquad
(\lf{b}{:}p; ~~ w{:}\iseph{\susp{{\sf bar}\,\lf{b}}})
\end{align*}
This trace cannot be reordered even though $\emptyset = \emptyset \cap
\{ y, z \} = (\trstep{\lf{b/a}}{x}){^\bullet} \cap
{^\bullet}(\trstep{w}{y\,z})$, because the atomic term $y\,z$ is only
well typed after the LF variables $\lf{a}$ and $\lf{b}$ are unified.
It is not even sufficient to compare the free and unified variables
(requiring that $\emptyset = {^\circledast}S_1 \cap {^\bullet}S_2$),
as in the example above ${^\circledast}(\trstep{\lf{b/a}}{x}) = \{
\lf{a},\lf{b} \}$ and ${^\bullet}(\trstep{w}{y\,z}) = \{ y, z \}$ --
and obviously $\emptyset = \{ \lf{a},\lf{b} \} \cap \{ y, z \}$.


The simplest solution is to forbid steps with unified variables from
being reordered at all: we can say that $(S_1,S_2) \in I$ if
$\emptyset = S_1{^\bullet} \cap {^\bullet}S_2 = {^\circledast}S_1 =
{^\circledast}S_2$. It is unlikely that this condition is satisfying
in general, but it is sufficient for all the examples in this
thesis. In the remainder of this thesis, we will define concurrent
equality on the basis of this simple solution.  Nevertheless, three
other possibilities are worth considering; all three are
equivalent to this simple solution as far as the examples in this
thesis are concerned.

\paragraph{Restricting open propositions} Part of the problem with the
example above was that there were variables free in the {\it type} of
a transition that were not free in the {\it term}.  A solution is to
restrict propositions so that negative propositions in the context are
always {\it closed} relative to the LF context (or at least relative
to the part of the LF context that mentions types subject to pure
variable equality, which would be simple enough to determine with a
subordination-based analysis). This restriction means that a step $S =
\trstep{P}{R}$ can only have the parameter $\lf{a}$ free in $R$'s type
if $\lf{a}$ is free in $R$, allowing us to declare that $S_1$ and
$S_2$ are reorderable -- meaning $(S_1,S_2)$ and $(S_2,S_1)$ are in the
independency relation $I$ -- whenever
$\emptyset = S_1{^\bullet} \cap {^\bullet}S_2 = {^\circledast}S_1 \cap
{^\bullet}S_2 = {^\bullet}S_1 \cap {^\circledast}S_2$. 

While this
restriction would be sufficient for the examples in this thesis, it
would preclude a conjectured extension of the destination-adding
transformation given in Chapter~\ref{chapter-destinations} to nested
specifications (nested versus flat specifications are discussed in
Section~\ref{section-introtologicalcorrespondence}).

\paragraph{Re-typing}
Another alternate solution would be to follow CLF and allow
any reordering permitted by the input and output interfaces, but then
forbid those that cannot be re-typed. (This was necessary in CLF to
deal with the presence of $\top$.) This is very undesirable, however, 
because it leads to strange asymmetries. The following trace would be 
reorderable by this definition, for example, but a symmetric case where
the equality was $\lf{b} \doteq_p \lf{a}$ instead of  
$\lf{a} \doteq_p \lf{b}$, that would no longer be the case.
\begin{align*}
& \qquad\qquad
(\lf{a}{:}p, \lf{b}{:}p; ~~ x{:}\iseph{\left({\ocircle}(\lf{a} \doteq_p \lf{b})\right)}, ~~
 y{:}\iseph{\left(\forall \lf{w}.\,{\sf foo}\,\lf{w} 
                 \lefti {\ocircle}({\sf bar}\, \lf{w})\right)}, ~~
 z{:}\iseph{\susp{{\sf foo}\,\lf{b}}})
\\
& \trstep{w}{y\,\lf{b}\,z};
\\
& \qquad\qquad
(\lf{a}{:}p, \lf{b}{:}p; ~~ x{:}\iseph{\left({\ocircle}(\lf{a} \doteq_p \lf{b})\right)}, ~~
 w{:}\iseph{\susp{{\sf bar}\,\lf{a}}})
\\
& \trstep{\lf{b/a}}{x}
\\
& \qquad\qquad
(\lf{b}{:}p; ~~ w{:}\iseph{\susp{{\sf bar}\,\lf{b}}})
\end{align*}

\paragraph{Process states with equality constraints}
A third possible solution is be to change the way we handle the
interaction of process states and unification. In this formulation of
\sls, the process state $(\Psi; \Delta)_{\lf\sigma}$ uses $\lf\sigma$
to capture the constraints that have been introduced by equality. As
an alternative, we could have process states mention an explicit
constraint store of equality propositions that have been encountered,
as in Jagadeesan et al.'s formulation of concurrent constraint
programming \cite{jagadeesan05testing}. Process states with equality
constraints might facilitate talking about the interaction of equality
and typing, which in our current formulation is left rather implicit. 

\subsection{Multifocusing}

Concurrent equality is related to the equivalence relation induced by
{\it multifocusing} \cite{chaudhuri08canonical}. Like concurrent
equality, multifocusing imposes a coarser equivalence relation on
focused proofs. The coarser equivalence relation is enabled by a
somewhat different mechanism: we are allowed to begin focus on
multiple propositions simultaneously.


\begin{figure}
\begin{center}
\begin{tikzpicture}
\draw[->] (0,20) -- node[above]{$u_1{:}\susp{\sf a}$} (1,20);
\draw (1,20) node[right]
{\footnotesize \gr{\trstep{v_1,w_1}{{\sf first}\,u_1}}};
%
\draw[->] (4,20.3) -- node[above]{$v_1{:}\susp{\sf b}$\qquad~} (5,21);
\draw (5,21) node[right]
{\footnotesize \gr{\trstep{x_1}{{\sf left}\,v_1}}};
\draw[->] (4,19.7) -- node[below]{$w_1{:}\susp{\sf c}$\qquad~} (5,19);
\draw (5,19) node[right]
{\footnotesize \gr{\trstep{y_1}{{\sf right}\,w_1}}};
%
\draw[->] (7.4,21) -- node[above]{~\qquad$x_1{:}\susp{\sf d}$} (8.7,20.3) ;
\draw[->] (7.7,19) -- node[below]{~\qquad$y_1{:}\susp{\sf e}$} (8.7,19.7) ;
\draw (8.7,20) node[right]
{\footnotesize \gr{\trstep{z_1}{{\sf last}\,(\tfuser{x_1}{y_1})}}};
%
\draw[->] (12.1,20) -- node[above]{$z_1{:}\susp{\sf f}$} (13.1,20);
%%%
%%%
%%%
\draw[->] (0,17) -- node[above]{$u_2{:}\susp{\sf a}$} (1,17);
\draw (1,17) node[right]
{\footnotesize \gr{\trstep{v_2,w_2}{{\sf first}\,u_2}}};
%
\draw[->] (4,17.3) -- node[above]{$v_2{:}\susp{\sf b}$\qquad~} (5,18);
\draw (5,18) node[right]
{\footnotesize \gr{\trstep{x_2}{{\sf left}\,v_2}}};
\draw[->] (4,16.7) -- node[below]{$w_2{:}\susp{\sf c}$\qquad~} (5,16);
\draw (5,16) node[right]
{\footnotesize \gr{\trstep{y_2}{{\sf right}\,w_2}}};
%
\draw[->] (7.4,18) -- % node[above]{\quad\qquad$x_2{:}\susp{\sf d}$}
 (8.7,14.3) ;
\draw[->] (7.7,16) -- %node[above, near end]{$y_2{:}\susp{\sf e}$} 
(8.7,17) ;
\draw (8.7,17) node[right]
{\footnotesize \gr{\trstep{z_2}{{\sf last}\,(\tfuser{x_3}{y_2})}}};
%
\draw[->] (12.1,17) -- node[above]{$z_2{:}\susp{\sf f}$} (13.1,17);
%%%
%%%
%%%
\draw[->] (0,14) -- node[above]{$u_3{:}\susp{\sf a}$} (1,14);
\draw (1,14) node[right]
{\footnotesize \gr{\trstep{v_3,w_3}{{\sf first}\,u_3}}};
%
\draw[->] (4,14.3) -- node[above]{$v_3{:}\susp{\sf b}$\qquad~} (5,15);
\draw (5,15) node[right]
{\footnotesize \gr{\trstep{x_3}{{\sf left}\,v_3}}};
\draw[->] (4,13.7) -- node[below]{$w_3{:}\susp{\sf c}$\qquad~} (5,13);
\draw (5,13) node[right]
{\footnotesize \gr{\trstep{y_3}{{\sf right}\,w_3}}};
%
\draw[->] (7.4,15) -- %node[below, near end]{\quad\qquad$x_3{:}\susp{\sf d}$} 
(8.7,16.7) ;
\draw (8.5,17.4) node{$y_2{:}\susp{\sf e}$};
\draw (8.5,16.4) node[right]{$x_3{:}\susp{\sf d}$};
\draw (8.5,15) node[right]{$x_2{:}\susp{\sf d}$};
\draw[->] (7.7,13) -- node[below]{~\qquad$y_3{:}\susp{\sf e}$} (8.7,13.7) ;
\draw (8.7,14) node[right]
{\footnotesize \gr{\trstep{z_3}{{\sf last}\,(\tfuser{x_2}{y_3})}}};
%
\draw[->] (12.1,14) -- node[above]{$z_3{:}\susp{\sf f}$} (13.1,14);
\end{tikzpicture}
\end{center}
\caption{Interaction diagram for a trace $(u_1{:}\susp{\sf a}, u_2{:}\susp{\sf a},u_3{:}\susp{\sf a}) \leadsto^*_{\Sigma} (z_1{:}\susp{\sf f}, z_2{:}\susp{\sf f},z_3{:}\susp{\sf f})$}
\label{fig:trace-dag}
\end{figure}

Both multifocusing and concurrent equality seek to address the
sequential structure of focused proofs. The sequential structure of
a computation obscures the fact that the interaction between resources
has the structure of a directed acyclic graph (DAG), not a list. We
sketch a radically different way of presenting traces in
Figure~\ref{fig:trace-dag}, where resources are the edges in the DAG
and steps or synthetic inference rules are the vertexes. (The crossed
edges that exchange $x_2$ and $x_3$ are only well-typed because, in
our example trace, ${\sf e}$ and ${\sf d}$ were both declared to be
ephemeral propositions.) Multifocusing gives a unique normal form to
proofs by gathering all the focusing steps that can be rotated all the
way to the beginning, then all the focusing steps that can happen as
soon as those first steps have been rotated all the way to the
beginning, etc. In \sls, by contrast, we are content to represent the
DAG structure as a list combined with the equivalence relation given
by concurrent equality.

Multifocusing has only been explored carefully in the context of
classical linear logic. We conjecture that a suitably-defined notion
of multifocusing for \ollll~would be in bijective correspondence with
\sls~terms modulo concurrent equivalence, at least if we omit
equality. Of course, without a formal notion of multifocusing for
intuitionistic logic, this conjecture is impossible to state
explicitly. The analogy with multifocusing may be able shed light on
our difficulties in integrating concurrent equality and unification
of pure variable types, because multifocusing has an independent
notion of correctness: the equivalence relation given by multifocusing
coincides with the the least equivalence relation that includes all
permutations of independent rules in an unfocused sequent calculus
proof \cite{chaudhuri08canonical}.

\section{Adequate encoding}
\label{sec:sls-adequate}

In Section~\ref{sec:lf-adequacy} we discussed encoding untyped
$\lambda$-calculus terms as LF terms of type ${\sf exp}$, captured by
the invertible function $\interp{e}$. Adequacy was extended to Linear
LF (LLF) by Cervesato and Pfenning \cite{cervesato02linear} and was
extended to Ordered LF (OLF) by Polakow \cite{polakow01ordered}. The
deductive fragment of \sls~approximately extends both LLF and OLF, and
the adequacy arguments made by Cervesato and Polakow extend
straightforwardly to the deductive fragment of \sls. These adequacy
arguments do not extend to the systems we want to encode in the
non-deductive fragment of \sls, however. The more general techniques
we consider in this section will be explored further in 
Chapter~\ref{chapter-gen} as a
general technique for capturing invariants of \sls~specifications.

The example that we will give to illustrate adequate encoding is the
following signature, the \sls~encoding of the push-down automata 
for parentheses matching from the introduction; we replace the atomic
proposition ${\sf <}$ with ${\sf L}$ and the proposition ${\sf >}$ with
${\sf R}$: 
\begin{align*}
 \Sigma_{\it PDA} = \cdot, 
~&{\sf L} : {\sf prop}\,{\sf ord},\\
~&{\sf R} : {\sf prop}\,{\sf ord},\\
~&{\sf hd} : {\sf prop}\,{\sf ord},\\
~&{\sf push} : 
     {\sf hd} \fuse {\sf L}
       \lefti {\ocircle}({\sf L} \fuse {\sf hd}),\\
~&{\sf pop} : 
     {\sf L} \fuse {\sf hd} \fuse {\sf R}
       \lefti {\ocircle}({\sf hd})
\end{align*}
We will relate this specification to a push-down automata defined in
terms of stacks $\obj{k}$ and strings $\obj{s}$, which we define inductively:
\begin{align*}
\obj{k} & ::= \obj{\cdot} \mid \obj{k{\sf <}}\\
\obj{s} & ::= \obj{\cdot} \mid \obj{{\sf <}s} \mid \obj{{\sf >}s}
\end{align*}
The transition system defined in terms of the stacks and strings
has two transitions:
\begin{align*}
\obj{(k
  \rhd ({\sf <}s))} & \mapsto \obj{((k{\sf <}) \rhd s)} \\
\obj{((k{\sf <}) \rhd ({\sf >}s))} & \mapsto \obj{(k \rhd s)}
\end{align*}

Existing adequacy arguments for CLF specifications by Cervesato et
al.~\cite{cervesato02concurrent} and by
Schack-Nielsen~\cite{schacknielsen07induction} have a three-part
structure structure.  The first step is to define an encoding
function $\ctxinterp{k \rhd s} = \Delta$ from PDA states $\obj{k \rhd
  s}$ to process states $\Delta$, so that, for example, the PDA state
$\obj{(\cdot{\sf <<}) \rhd ({\sf >>><}\cdot)}$ is encoded as the
process state
\[
x_2{:}\istrue{\susp{\sf L}}, ~~
x_1{:}\istrue{\susp{\sf L}}, ~~
h{:}\istrue{\susp{\sf hd}}, ~~
y_1{:}\istrue{\susp{\sf R}}, ~~
y_2{:}\istrue{\susp{\sf R}}, ~~
y_3{:}\istrue{\susp{\sf R}}, ~~
y_4{:}\istrue{\susp{\sf L}}
\]
The second step is to prove a preservation-like property: if
$\ctxinterp{\obj{k \rhd s}} \leadsto_{\Sigma_{\it PDA}}
\Delta'$, then $\Delta' = \ctxinterp{{k' \rhd s'}}$ for some
$\obj{k'}$ and $\obj{s'}$. The third step is the main adequacy result:
that $\ctxinterp{k \rhd s} \leadsto_{\Sigma_{\it PDA}} \ctxinterp{k'
  \rhd s'}$ if and only if $\obj{k \rhd s} \mapsto \obj{k' \rhd
  s'}$. 

The second step is crucial in general: without it, we might transition
in \sls~from the encoding of some $\obj{k \rhd s}$ to a state
$\Delta'$ that is not in the image of encoding. % With our
%deterministic example, this concern is less obvious: we could avoid
%the second proof by instead proving that if $\ctxinterp{k \rhd s}
%\leadsto_{\Sigma_{\it PDA}} \Delta$ and $\ctxinterp{k \rhd s}
%\leadsto_{\Sigma_{\it PDA}} \Delta'$ then $\Delta = \Delta'$. 
We will 
take the opportunity to re-factor Cervesato et al.'s approach, 
replacing the second step with a general statement about transitions
in $\Sigma_{\it PDA}$ preserving a well-formedness invariant. The 
invariant we discuss is a simple instance of
the well-formedness invariants that we will explore
further in Chapter~\ref{chapter-gen}. 

The first step in our revised methodology is to describe a generative
signature $\Sigma_{\it Gen}$ that precisely captures the set of
process states that encode machine states
(Theorem~\ref{thm:pda-encoding} below).  The second step is showing
that the generative signature $\Sigma_{\it Gen}$ describes an
invariant of the signature $\Sigma_{\it PDA}$
(Theorem~\ref{thm:pda-preservation}).  The third step, showing that
$\ctxinterp{k \rhd s} \leadsto_{\Sigma_{\it PDA}} \ctxinterp{k' \rhd
  s'}$ if and only if $\obj{k \rhd s} \mapsto \obj{k' \rhd s'}$, is
straightforward and follows other developments.

\subsection{Regular and generated worlds}

A critical aspect of any adequacy argument is an understanding of the
structure of the relevant context(s) (the LF context in LF encodings,
the substructural context in CLF encodings, both in \sls~encodings).
In the statement of adequacy for untyped $\lambda$-calculus terms
(Section~\ref{sec:lf-adequacy}), for instance, it was necessary to
require that the LF context $\Psi$ take the form $\lf{a_1}{:}{\sf
  exp},\ldots,\lf{a_n}{:}{\sf exp}$. In the adequacy theorems that
have been presented for deductive logical frameworks, the structure of
the context is always {\it regular} -- we can describe a set of
building blocks that build small pieces of the context, and then
define the set of valid contexts (the {\it world}) to be any context
that can be built from a particular set of building blocks.  The
simplest {\it regular worlds} are those that forbid the presence of
any LF variables -- when we adequately encode unary natural numbers
$\obj{n}$ as LF terms of type ${\sf nat}$, letting $\interp{\sf z} =
\lf{\sf z}$ and letting $\interp{{\sf s} n} = \lf{s}\,\interp{n}$, we
assume there aren't any free LF variables of type ${\sf nat}$! This is
the {\it closed world assumption}, and regular worlds can be presented
as a generalization of closed worlds.  Twelf contains both a syntax
for declaring regular worlds and facilities for checking that a
signature respects a particular regular world structure
\cite{schurmann00automating}.

The relevant building blocks in our $\lambda$-calculus encoding are
just LF variables of type $\lf{a}{:}{\sf exp}$. In other LF examples
regular worlds involve multiple context elements that must appear in
tandem. In proving adequacy for typing derivations, for instance, we
must generally assume that a variable $\lf{a}{:}{\sf exp}$ always
appears in tandem with a variable $\lf{d}{:}({\sf
  of}\,\lf{a}\,\lf{tp})$ that associates the variable with some closed
LF term $\lf{tp}$ of type ${\sf typ}$ that encodes the type of that
variable.

Regular worlds remain sufficient for the encoding of stores in Linear
LF \cite{cervesato02linear} and stacks in Ordered LF
\cite{polakow01ordered}. Our PDA states, on the other hand, do not
have the structure that can be described by Sch\"urmann's regular
worlds language, because the process state is organized into three
distinct zones:
\[
[~\mbox{the stack}~]
~
[~\mbox{the head}~]
~
[~\mbox{the string being read}~]
\]
This structure {\it can} be described with a regular expression, but
the structures that can be described by regular expressions are richer
than those that can be described by the language of regular worlds
given by Sch\"urmann and implemented in Twelf
\cite{schurmann00automating}.

The generalization we propose is a move from a description of worlds
based on regular worlds descriptions to a description of worlds based
on (something like) context-free grammars.  Conveniently, the
something-like-context-free grammars we are interested in can be
characterized within the machinery of \sls~itself by describing {\it
  generative signatures} that can generate the set of process states
we are interested in from a single seed context.  The signature
$\Sigma_{\it Gen}$ in Figure~\ref{fig:pda-gen} treats all the atomic
propositions of $\Sigma_{\it PDA}$ -- the atomic propositions ${\sf
  L}$, ${\sf R}$ and ${\sf hd}$ -- as {\it terminals}, and introduces
three {\it nonterminals} ${\sf gen}$, ${\sf gen\_stack}$, and ${\sf
  gen\_string}$.

\begin{figure}
\begin{align*}
 \Sigma_{\it Gen} = \cdot, 
~&{\sf L} : {\sf prop}\,{\sf ord},\\
~&{\sf R} : {\sf prop}\,{\sf ord},\\
~&{\sf hd} : {\sf prop}\,{\sf ord},\\
~&{\sf gen} : {\sf prop}\,{\sf ord},\\
~&{\sf gen\_stack} : {\sf prop}\,{\sf ord},\\
~&{\sf gen\_string} : {\sf prop}\,{\sf ord},\\
~&{\sf state} : {\sf gen} 
       \lefti {\ocircle}({\sf gen\_stack} \fuse {\sf hd} 
                         \fuse {\sf gen\_string})
&& G \rightarrow G_k\,{\sf hd}\,G_s\\
~&{\sf stack/left} : {\sf gen\_stack} 
       \lefti {\ocircle}({\sf L} \fuse {\sf gen\_stack})
&& G_k \rightarrow {\sf <}\,G_k \\
~&{\sf stack/done} : {\sf gen\_stack} \lefti {\ocircle}(\one)
&& G_k \rightarrow \epsilon\\
~&{\sf string/left} : 
     {\sf gen\_string} 
       \lefti {\ocircle}({\sf gen\_string} \fuse {\sf L})
&& G_s \rightarrow G_s\,{\sf <}\\
~&{\sf string/right} : 
     {\sf gen\_string} 
       \lefti {\ocircle}({\sf gen\_string} \fuse {\sf R})
&& G_s \rightarrow G_s\,{\sf >}\\
~&{\sf string/done} : 
     {\sf gen\_string} 
       \lefti {\ocircle}(\one)
&& G_s \rightarrow \epsilon
\end{align*}
\caption{Generative specification of PDA states and an analogous
  context-free grammar}
\label{fig:pda-gen}
\end{figure}

An informal translation of the signature $\Sigma_{\it Gen}$ as a
context-free grammar is given on the right-hand side of
Figure~\ref{fig:pda-gen}. Observe that the sentences in the language
$G$ encode the states of our PDA as a string. 

\subsection{Restriction}

The operation of {\it restriction} adapts the concept of ``terminal''
and ``non-terminal'' to \sls. Note that process states $\Delta$
such that $(x{:}\istrue{\susp{\sf gen}}) \leadsto^*_{\Sigma_{\it Gen}}
\Delta$ are only well-formed under the signature $\Sigma_{\it PDA}$ if
they are free of nonterminals; we can define an operation of {\it
  restriction} that filters out the non-terminal process states by
checking whether they are well-formed in a signature that only
declares the terminals.

\bigskip
\begin{definition}[Restriction]~
\begin{itemize}
\item
  $\restrictsig{\Psi}{\Sigma}$ is a total function that returns the
  largest context $\Psi' \subseteq \Psi$ such that $\vdash_\Sigma
  \Psi\,{\sf ctx}$ (defined in Figure~\ref{fig:lf-form}) by removing
  all the LF variables in $\Psi$ whose types are not-well-formed in
  the context $\Sigma$.

\item
  $\restrictsig{(\Psi; \Delta)}{\Sigma}$ is a partial function that is
  defined exactly when, for every variable declaration
  $x{:}\istrue{T}$ or $x{:}\iseph{T}$ in $\Delta$, we have that
  $(\restrictsig{\Psi}{\Sigma}) \vdash_\Sigma T\,{\sf left}$ (defined
  in Figure~\ref{fig:sls-ctxform}). When it is defined,
  $\restrictsig{(\Psi; \Delta)}{\Sigma} = ((\restrictsig{}{\Sigma});
  \Delta')$, where $\Delta'$ is $\Delta$ except for the variable
  declarations $x{:}\ispers{T}$ in $\Delta$ for which it was not the
  case that $(\restrictsig{\Psi}{\Sigma}) \vdash_\Sigma T\,{\sf
    left}$.

\item
  We will also use $\restrictsig{(\Psi; \Delta)}{\Sigma}$ as a
  judgment which expresses that the function as defined.

\end{itemize}
\end{definition}
\bigskip

Because 
restriction is only defined if
all the ordered and linear propositions in $\Delta$ are well-formed
in $\Sigma$; this means that $\restrictsig{(x{:}\istrue{\susp{\sf
      gen}})}{\Sigma_{\it PDA}}$ is not defined.
Restriction acts as a semi-permeable membrane on process states: some
process states cannot pass through at all, and others pass through
with some of their LF variables and persistent propositions
removed. We can represent context restriction $\restrictsig{(\Psi;
  \Delta)}{\Sigma} = (\Psi'; \Delta')$ in a two-dimensional notation
as a dashed line annotated with the restricting signature:
\begin{center}
\begin{tikzpicture} 
\draw (.8,.5) node{$(\Psi; \Delta)$};
\draw (-.3, .1) node{$\Sigma$};
\draw [thick,dash pattern = on 2.82842842712mm off 2mm,decorate,decoration={saw,amplitude=2mm,segment length=2mm}] 
(0,0) -- (1.6,0); 
\draw (.8,-.3) node{$(\Psi'; \Delta')$};
\end{tikzpicture} 
\end{center}

For all process states that evolve from the initial state
$(x{:}\istrue{\susp{\sf gen}})$ under the signature $\Sigma_{\it
  Gen}$, restriction to $\Sigma_{\it PDA}$ is the identity function
whenever it is defined. Therefore, in the statement of
Theorem~\ref{thm:pda-encoding}, we use restriction as a judgment
$\restrictsig{\Delta}{\Sigma_{\it PDA}}$ that holds whenever the partial 
function is defined.

\bigskip
\begin{theorem}[Encoding]\label{thm:pda-encoding}
  Up to variable renaming, there is a bijective correspondence between
  PDA states $\obj{k \rhd s}$ and process states $\Delta$ such that
  $T :: (x{:}\istrue{\susp{\sf gen}}) \leadsto^*_{\Sigma_{\it Gen}}
  \Delta$ and $\restrictsig{\Delta}{\Sigma_{\it
      PDA}}$.
\end{theorem}

\begin{proof}To establish the bijective correspondence, we first define
an encoding function from PDA states to process states:
\smallskip
\begin{itemize}
\item $\ctxinterp{k \rhd s} = 
  \ctxinterp{k}, ~~
  h{:}\istrue{\susp{\sf hd}}, ~~
  \ctxinterp{s}$
\item $\ctxinterp{\cdot} = \cdot$
\item $\ctxinterp{k{\sf <}} = \ctxinterp{k}, ~~ x{:}\istrue{\susp{\sf L}}$
\item $\ctxinterp{{\sf <}s} = y{:}\istrue{\susp{\sf L}}, ~~ \ctxinterp{s}$
\item $\ctxinterp{{\sf >}s} = y{:}\istrue{\susp{\sf R}}, ~~ \ctxinterp{s}$
\end{itemize}
\smallskip 
It is always the case that $\restrictsig{\ctxinterp{k \rhd
    s}}{\Sigma_{\it PDA}}$ -- the encoding only includes terminals.

It is straightforward to observe that if $\ctxinterp{k \rhd s} =
\ctxinterp{k' \rhd s'}$ if an only if $\obj{k} = \obj{k'}$ and
$\obj{s} = \obj{s'}$. The interesting part of showing that context
interpretation is an injective function is mostly just showing that it
is a function: that is, showing that, for any $\obj{k \rhd s}$, there
exists a trace $T :: (x{:}\istrue{\susp{\sf gen}}) \leadsto^*_{\it
  Gen} \ctxinterp{k \rhd s}$. To show that the encoding function is
surjective, we must show that if $T :: (x{:}\istrue{\susp{\sf gen}})
\leadsto^*_{\Sigma_{\it Gen}} \Delta$ and
$\restrictsig{\Delta}{\Sigma_{\it PDA}}$ then $\Delta = \ctxinterp{k
  \rhd s}$ for some $\obj{k}$ and $\obj{s}$. This will complete the
proof: an injective and surjective function is bijective.

\subsubsection{Encoding is injective}

We prove that for
any $\obj{k \rhd s}$, there exists a trace $T ::
(x{:}\istrue{\susp{\sf gen}}) \leadsto^*_{\it Gen} \ctxinterp{k \rhd
  s}$ with a series of three lemmas.

\begin{lemma} For all $\obj{k}$, there exists
$T :: ({x{:}\istrue{\susp{\sf gen\_stack}}}) \leadsto^*_{\Sigma_{\it Gen}} 
({\ctxinterp{k}, x'{:}\istrue{\susp{\sf gen\_stack}}})$.
\end{lemma}
\noindent
By induction on $\obj{k}$. 
\begin{itemize}
\item If $\obj{k} = \cdot$, $T = \emptytrace ::
({x{:}\istrue{\susp{\sf gen\_stack}}}) \leadsto^*_{\Sigma_{\it Gen}} 
(x{:}\istrue{\susp{\sf gen\_stack}})$
\item If $\obj{k} = \obj{k' {\sf <}}$, we have 
$T' :: ({x{:}\istrue{\susp{\sf gen\_stack}}}) \leadsto^*_{\Sigma_{\it Gen}} 
({\ctxinterp{k'}, x''{:}\istrue{\susp{\sf gen\_stack}}})$ by the induction
hypothesis, so $T = (T'; \trstep{x_1, x_2}{{\sf stack/left}\,x'}) :: 
({x{:}\istrue{\susp{\sf gen\_stack}}}) \leadsto^*_{\Sigma_{\it Gen}} 
({\ctxinterp{k'}, x_1{:}\istrue{\susp{\sf L}}, x_2{:}\istrue{\susp{\sf gen\_stack}}})$
\end{itemize}

\begin{lemma} For all $\obj{s}$, there exists
$T :: ({y{:}\istrue{\susp{\sf gen\_string}}}) \leadsto^*_{\Sigma_{\it Gen}} 
({y'{:}\istrue{\susp{\sf gen\_string}}, \ctxinterp{s}})$.
\end{lemma}
\noindent
By induction on $\obj{s}$.
\begin{itemize}
\item If $\obj{s} = \cdot$, $T = \emptytrace ::
({y{:}\istrue{\susp{\sf gen\_string}}}) \leadsto^*_{\Sigma_{\it Gen}} 
(y{:}\istrue{\susp{\sf gen\_string}})$
\item If $\obj{s} = \obj{{\sf <}s' }$, we have 
$T' :: ({y{:}\istrue{\susp{\sf gen\_string}}}) \leadsto^*_{\Sigma_{\it Gen}} 
(y'{:}\istrue{\susp{\sf gen\_string}}, \ctxinterp{s'})$ by the induction
hypothesis, so $T = (T'; \trstep{y_1, y_2}{{\sf string/left}\,y'}) :: 
({y{:}\istrue{\susp{\sf gen\_stack}}}) \leadsto^*_{\Sigma_{\it Gen}} 
(y_1{:}\istrue{\susp{\sf gen\_stack}},
 y_2{:}\istrue{\susp{\sf L}},
\ctxinterp{s'})$
\item If $\obj{s} = \obj{{\sf >}s' }$, we have 
$T' :: ({y{:}\istrue{\susp{\sf gen\_string}}}) \leadsto^*_{\Sigma_{\it Gen}} 
(y'{:}\istrue{\susp{\sf gen\_string}}, \ctxinterp{s'})$ by the induction
hypothesis, so $T = (T'; \trstep{y_1, y_2}{{\sf string/right}\,y'}) :: 
({y{:}\istrue{\susp{\sf gen\_stack}}}) \leadsto^*_{\Sigma_{\it Gen}} 
(y_1{:}\istrue{\susp{\sf gen\_stack}},
 y_2{:}\istrue{\susp{\sf R}},
\ctxinterp{s'})$
\end{itemize}

\begin{lemma} For all $\obj{k}$ and $\obj{s}$, there exists
$T :: ({g{:}\istrue{\susp{\sf gen}}}) \leadsto^*_{\Sigma_{\it Gen}} 
(\ctxinterp{k \rhd s})$. 
\end{lemma}
\noindent
By straightforward construction using the first two lemmas
and frame weakening (Theorem~\ref{thm:frameweak}): 
\begin{align*}
& \qquad\qquad (g{:}\istrue{\susp{\sf gen}})\\
& \trstep{x,h,y}{{\sf state}\,g};\\
& \qquad\qquad
       (x{:}\istrue{\susp{\sf gen\_stack}},  ~~
        h{:}\istrue{\susp{\sf hd}},~~ 
        y{:}\istrue{\susp{\sf gen\_string}})\\
& T_k; ~~ \mbox{\it (given by the first lemma and frame weakening)}\\
& \qquad\qquad
       (\ctxinterp{k},  ~~
        x'{:}\istrue{\susp{\sf gen\_stack}},  ~~
        h{:}\istrue{\susp{\sf hd}}, ~~
        y{:}\istrue{\susp{\sf gen\_string}})\\
& \trstep{()}{{\sf stack/done}\,x'}\\
& \qquad\qquad
       (\ctxinterp{k},  ~~
        h{:}\istrue{\susp{\sf hd}}, ~~
        y{:}\istrue{\susp{\sf gen\_string}})\\
& T_s; ~~ \mbox{\it (given by the second lemma and frame weakening)}\\
& \qquad\qquad
       (\ctxinterp{k}, ~~
        h{:}\istrue{\susp{\sf hd}}, ~~
        y'{:}\istrue{\susp{\sf gen\_string}}, ~~
        \ctxinterp{s})\\
& \trstep{()}{{\sf string/done}\,y'}\\
& \qquad\qquad 
       (\ctxinterp{k}, ~~
        h{:}\istrue{\susp{\sf hd}}, ~~
        \ctxinterp{s}) \\
& \qquad\qquad = \ctxinterp{k \rhd s}
\end{align*}


\subsubsection{Encoding is surjective}

We prove that if $T :: (x{:}\istrue{\susp{\sf gen}}) \leadsto^*_{\Sigma_{\it
    Gen}} \Delta$ and $\restrictsig{\Delta}{\Sigma_{\it PDA}}$ then
$\Delta = \ctxinterp{k \rhd s}$ for some $\obj{k}$ and $\obj{s}$
any $\obj{k \rhd s}$, there exists a trace $T ::
(x{:}\istrue{\susp{\sf gen}}) \leadsto^*_{\it Gen} \ctxinterp{k \rhd
  s}$ with a series of two lemmas.

\begin{lemma} If 
$T :: (\ctxinterp{k}, 
       x{:}\istrue{\susp{\sf gen\_stack}}, 
       h{:}\istrue{\susp{\sf hd}},
       y{:}\istrue{\susp{\sf gen\_store}},
       \ctxinterp{s}) 
  \leadsto^*_{\Sigma_{\it Gen}} \Delta$ and
$\restrictsig{\Delta}{\Sigma_{\it
      PDA}}$, then $\Delta = \ctxinterp{k' \rhd s'}$ for some $\obj{k'}$
and $\obj{s'}$.
\end{lemma}
\noindent
By induction on the structure of $T$ and case analysis on the 
first steps in $T$. Up to concurrent equality, there
are four possibilities:
\smallskip
\begin{itemize}
\item $T = (\trstep{()}{{\sf stack/done}\,x}; \trstep{()}{{\sf
      string/done}\,y})$ -- this is a base case, and we can finish by
  letting $\obj{k'} = \obj{k}$ and $\obj{s'} = \obj{s}$.
\item $T = (\trstep{x_1,x_2}{{\sf stack/left}\,x}; T')$ -- apply 
  the ind.~hyp.~(letting $x = x_2$, $\obj{k} =
  \obj{k{\sf <}}$).
\item $T = (\trstep{y_1,y_2}{{\sf string/left}\,y}; T')$ -- apply
  the ind.~hyp.~(letting $y = y_1$, $\obj{s} =
  \obj{{\sf <}s}$).
\item $T = (\trstep{y_1,y_2}{{\sf string/right}\,y}; T')$ -- apply
  the ind.~hyp.~(letting $y = y_1$, $\obj{s} =
  \obj{{\sf >}s}$).
\end{itemize}
\smallskip 
The proof above takes
a number of facts about concurrent equality for granted. 
%
For example, the trace $T = (\trstep{()}{{\sf stack/done}\,x};
\trstep{y_1,y_2}{{\sf string/right}\,y}; T')$ does not syntactically
match any of the traces above if we do not account for concurrent
equality. Modulo concurrent equality, on the other hand, $T =
(\trstep{y_1,y_2}{{\sf string/right}\,y}; \trstep{()}{{\sf
    stack/done}\,x}; T')$, matching the last branch of the case
analysis.  If we didn't implicitly rely on concurrent equality in this
way, the resulting proof would have twice as many cases.  We will take
these finite uses of concurrent equality for granted when we specify
that a proof proceeds by case analysis on the first steps of $T$ (or,
conversely, by case analysis on the last steps of $T$).

\begin{lemma} If 
$T :: (g{:}\istrue{\susp{\sf gen}}) 
  \leadsto^*_{\Sigma_{\it Gen}} \Delta$ and
$\restrictsig{\Delta}{\Sigma_{\it
      PDA}}$, then $\Delta = \ctxinterp{k' \rhd s'}$ for some $\obj{k'}$
and $\obj{s'}$.
\end{lemma}
\noindent
This is a corollary of the previous lemma, as it can only
be the case that $T = \trstep{x,h,y}{{\sf state}\,g}; T'$. We can apply the previous
lemma to $T'$, letting $\obj{k} = \obj{s} = \obj{\cdot}$.
This establishes that encoding is a surjective function, which in turn
completes the proof. 
\end{proof}

Theorem~\ref{thm:pda-encoding} establishes that the generative
signature $\Sigma_{\it Gen}$ describes a world -- a set of
\sls~process states -- that precisely corresponds to the states of a
push-down automata.  We can (imperfectly) illustrate the content of
this theorem in our two-dimensional notation as follows, where 
$\Delta \Leftrightarrow \obj{k \rhd s}$ indicates the presence of a
bijection:
\begin{center}
\begin{tikzpicture} 
\draw (.8,2.3) node{$(x{:}\istrue{\susp{\sf gen}})$};
\draw [->,decorate, 
decoration={snake,amplitude=.3mm,segment length=3mm,post length=1mm}] 
(0.8,2) -- (.8,.8); 
\draw (.95,.8) node{$_*$};
\draw (.3,1.4) node{$\Sigma_{\it Gen}$};
\draw (.8,.5) node{$\Delta$};
\draw (-.2, .1) node{$\Sigma_{\it PDA}$};
\draw [thick,dash pattern = on 2.82842842712mm off 2mm,decorate,decoration={saw,amplitude=2mm,segment length=2mm}] 
(.4,0) -- (1.2,0); 
\draw (.8,-.3) node{$\Delta$};
\draw (1.1,-.7) node{\begin{turn}{-45}$\Leftrightarrow$\end{turn}};
\draw (1.8,-1) node{$\obj{k \rhd s}$};
\end{tikzpicture} 
\end{center}

It is interesting to note how the proof of
Theorem~\ref{thm:pda-encoding} takes advantage of the associative
structure of traces: the inductive process that constructed traces in
the first two lemmas treated trace composition as left-associative,
but the induction we performed on traces in the next-to-last lemma
treated trace composition as right-associative.

\subsection{Generated world preservation}
\label{sec:sls-pda-preservation}

The generative signature $\Sigma_{\it Gen}$ precisely captures the
world of \sls~process states that are in the image of the encoding
$\ctxinterp{k \rhd s}$ of PDA states as process states. In order for
the signature $\Sigma_{\it PDA}$ to encode a reasonable notion of
transition between PDA states, we need to show that steps in this
signature only take encoded PDA states to encoded PDA states. Because 
the generative signature $\Sigma_{\it Gen}$ precisely captures the 
process states that represent encoded PDA states, we can describe
and prove this property without reference to the actual encoding function:
 
\bigskip
\begin{theorem}[Preservation]\label{thm:pda-preservation}
If $T_1 :: (x{:}\istrue{\susp{\sf gen}}) \leadsto^*_{\Sigma_{\it Gen}} \Delta_1$,
$\restrictsig{\Delta_1}{\Sigma_{\it PDA}}$, and 
$S :: \Delta_1 \leadsto_{\Sigma_{\it PDA}} \Delta_2$, then 
$T_2 :: (x{:}\istrue{\susp{\sf gen}}) \leadsto^*_{\Sigma_{\it Gen}} \Delta_2$.
\end{theorem}
\bigskip

\noindent
If we illustrate the given elements as solid lines and elements that we have
to prove as dashed lines, the big picture of the encoding and preservation
theorems is the following:

\begin{center}
\begin{tikzpicture} 
\draw (.8,2.3) node{$(x{:}\istrue{\susp{\sf gen}})$};
\draw [->,decorate, 
decoration={snake,amplitude=.3mm,segment length=3mm,post length=1mm}] 
(0.8,2) -- (.8,.8); 
\draw (.95,.8) node{$_*$};
\draw (.3,1.4) node{$\Sigma_{\it Gen}$};
\draw (.8,.5) node{$\Delta$};
\draw [thick,dash pattern = on 2.82842842712mm off 2mm,decorate,decoration={saw,amplitude=2mm,segment length=2mm}] 
(.4,0) -- (1.2,0); 
\draw (.8,-.3) node{$\Delta$};
\draw (1.1,-.7) node{\begin{turn}{-45}$\Leftrightarrow$\end{turn}};
\draw (1.8,-1) node{$\obj{k \rhd s}$};
%
\draw (4.8,2.3) node{$(x{:}\istrue{\susp{\sf gen}})$};
\draw [->,densely dotted,decorate, 
decoration={snake,amplitude=.3mm,segment length=3mm,post length=1mm}] 
(4.8,2) -- (4.8,.8); 
\draw (4.95,.8) node{$_*$};
\draw (4.3,1.4) node{$\Sigma_{\it Gen}$};
\draw (4.8,.5) node{$\Delta'$};
\draw (2.8, 0) node{$\Sigma_{\it PDA}$};
\draw [thick,dash pattern = on 0.677mm off .4mm on 0.676142375mm off .4mm on 0.676142375mm off 2mm,decorate,decoration={saw,amplitude=2mm,segment length=2mm}] 
(4.4,0) -- (5.2,0); 
\draw (4.8,-.3) node{$\Delta'$};
\draw (5.1,-.7) node{\begin{turn}{-45}$\Leftrightarrow$\end{turn}};
\draw (5.8,-1) node{$\obj{k' \rhd s'}$};
%
\draw [->,decorate, 
decoration={snake,amplitude=.3mm,segment length=3mm,post length=1mm}] 
(1.2,-.3) -- (4.4,-.3); 
\end{tikzpicture} 
\end{center}

\noindent
The proof of Theorem~\ref{thm:pda-preservation} relies on two lemmas, 
which we will consider before the proof itself. They are both
{\it inversion lemmas}: they help uncover the structure of the
trace based on the type of that trace. Treating
traces modulo concurrent equality is critical in both cases. 

\bigskip
\begin{lemma}
  Let $\Delta = \tackon{\Theta}{x{:}\istrue{\susp{\sf gen\_stack}},
    h{:}\istrue{\susp{\sf hd}}, y{:}\istrue{\susp{\sf gen\_string}}}$.
  If $T :: \Delta \leadsto^*_{\Sigma_{\it Gen}} \Delta'$ and
  $\restrictsig{\Delta'}{\Sigma_{\it PDA}}$, then $T = (T';
  \trstep{()}{{\sf stack/done}\,x'}; \trstep{()}{{\sf
      string/done}\,y'})$, where $T' ::
  \Delta \leadsto^*_{\Sigma_{\it Gen}}
  \tackon{\Theta'}{x'{:}\istrue{\susp{\sf gen\_stack}},
    h{:}\istrue{\susp{\sf hd}}, y'{:}\istrue{\susp{\sf gen\_string}}}$
  and $\Delta' = \tackon{\Theta'}{h{:}\istrue{\susp{\sf hd}}}$.
Or, as a picture: 

\begin{center}
\begin{tikzpicture} 
\draw (0,2.3) 
  node{$\Delta = \tackon{\Theta}{x{:}\istrue{\susp{\sf gen\_stack}},
    h{:}\istrue{\susp{\sf hd}}, y{:}\istrue{\susp{\sf gen\_string}}}$};
\draw [->,decorate, 
decoration={snake,amplitude=.3mm,segment length=3mm,post length=1mm}] 
(-2.7,2) -- (-2.7,-1.3); 
\draw (-2.55,-1.3) node{$_*$};
\draw (-2.7,1.65) node[left]{$T$};
\draw (-2.7,-1.6) node{$\Delta' \mathstrut$};
%
\draw (-2.2,1.65) node{$=$};
%
\draw [->,densely dotted,decorate, 
decoration={snake,amplitude=.3mm,segment length=3mm,post length=1mm}] 
(2.2,2) -- (2.2,1.3); 
\draw (2.2,1.65) node[left]{$T'$};
\draw [->,densely dotted,decorate, 
decoration={snake,amplitude=.3mm,segment length=3mm,post length=1mm}] 
(2.2,.7) -- (2.2,0); 
\draw (2.2,.35) node[left]{$\trstep{()}{{\sf stack/done}\,x'}$};
\draw [->,densely dotted,decorate, 
decoration={snake,amplitude=.3mm,segment length=3mm,post length=1mm}] 
(2.2,-.6) -- (2.2,-1.3); 
\draw (2.2,-.95) node[left]{$\trstep{()}{{\sf string/done}\,y'}$};
\draw (2.35,1.3) node{$_*$};
%\draw (2.2,2.3) node{$\Delta$};
\draw (1.9,1) node[right]{$\tackon{\Theta'}{x'{:}\istrue{\susp{\sf gen\_stack}},
    h{:}\istrue{\susp{\sf hd}}, y'{:}\istrue{\susp{\sf gen\_string}}}$};
\draw (1.9,-.3) node[right]{$\tackon{\Theta'}{
    h{:}\istrue{\susp{\sf hd}}, y'{:}\istrue{\susp{\sf gen\_string}}}$};
\draw (1.9,-1.6) node[right]{$\Delta' = \tackon{\Theta'}{h{:}\istrue{\susp{\sf hd}}} \mathstrut$};
\end{tikzpicture} 
\end{center}
\end{lemma}

\begin{proof}
By induction on the structure of $T$ and case analysis on the first
steps in $T$. Up to concurrent equality, there are five possibilities:
\begin{itemize}
\item $T = (\trstep{()}{{\sf stack/done}\,x}; \trstep{()}{{\sf
      string/done}\,y})$. Immediate, letting $T' = \emptytrace$.
\item $T = (\trstep{x_1,x_2}{{\sf stack/left}\,x}; T'')$. By the
  induction hypothesis (where the new frame incorporates
  $x_1{:}\istrue{\susp{\sf L}}$), we have $T'' = (T''';
  \trstep{()}{{\sf stack/done}\,x}; \trstep{()}{{\sf
      string/done}\,y})$. Let $T' = (\trstep{x_1,x_2}{{\sf
      stack/left}\,x}; T''')$.
\item $T = (\trstep{y_1,y_2}{{\sf string/left}\,y}; T'')$. By the
  induction hypothesis (where the new frame incorporates
  $y_1{:}\istrue{\susp{\sf L}}$), we have $T'' = (T''';
  \trstep{()}{{\sf stack/done}\,x}; \trstep{()}{{\sf
      string/done}\,y})$. Let $T' = (\trstep{x_1,x_2}{{\sf
      string/left}\,y}; T''')$.
\item $T = (\trstep{y_1,y_2}{{\sf string/right}\,y}; T'')$. By the
  induction hypothesis (where the new frame incorporates
  $y_2{:}\istrue{\susp{\sf R}}$), we have $T'' = (T''';
  \trstep{()}{{\sf stack/done}\,x}; \trstep{()}{{\sf
      string/done}\,y})$. Let $T' = (\trstep{x_1,x_2}{{\sf
      string/right}\,y}; T''')$.
\item $T = (S; T'')$, where $x$ and $y$ are not free in $S$. 
      By the induction hypothesis, we have
      $T'' = (T''';
       \trstep{()}{{\sf stack/done}\,x}; \trstep{()}{{\sf
       string/done}\,y})$. Let $T' = (S; T''')$. (This case will not arise
      in the way we use this lemma, but the statement of the theorem 
      leaves open the possibility that there are other nonterminals
      in $\Theta$.)
\end{itemize}
This completes the proof. 
\end{proof}

A corollary of this lemma is that if 
$T :: (g{:}\istrue{\susp{\sf gen}}) \leadsto^*_{\Sigma_{\it Gen}} \Delta$ and
  $\restrictsig{\Delta}{\Sigma_{\it PDA}}$, then $T = (T';
  \trstep{()}{{\sf stack/done}\,x}; \trstep{()}{{\sf
      string/done}\,y})$ -- modulo concurrent equality, naturally -- 
where $T' ::
  (g{:}\istrue{\susp{\sf gen}}) \leadsto^*_{\Sigma_{\it Gen}}
  \tackon{\Theta}{x'{:}\istrue{\susp{\sf gen\_stack}},
    h{:}\istrue{\susp{\sf hd}}, y'{:}\istrue{\susp{\sf gen\_string}}}$
  and $\Delta = \tackon{\Theta}{h{:}\istrue{\susp{\sf hd}}}$. To prove
the corollary, we observe
that $T = (\trstep{x,h,r}{{\sf state}\,g\,}; T'')$ and apply the lemma
to $T''$. 

\bigskip
\begin{lemma} The following all hold:
\begin{itemize}
\item If $T :: (g{:}\istrue{\susp{\sf gen}}) \leadsto^*_{\Sigma_{\it Gen}} 
       \tackon{\Theta}{x_1{:}\istrue{\susp{\sf L}}, 
          x_2{:}\istrue{\susp{\sf gen\_stack}}}$, \\then 
$T = (T'; \trstep{x_1,x_2}{{\sf stack/left}\,x'})$ for some $x'$.
\item If $T :: (g{:}\istrue{\susp{\sf gen}}) \leadsto^*_{\Sigma_{\it Gen}} 
       \tackon{\Theta}{y_1{:}\istrue{\susp{\sf gen\_string}},
           y_2{:}\istrue{\susp{\sf L}}}$, \\then 
$T = (T'; \trstep{y_1,y_2}{{\sf string/left}\,y'})$ for some $y'$.
\item If $T :: (g{:}\istrue{\susp{\sf gen}}) \leadsto^*_{\Sigma_{\it Gen}} 
       \tackon{\Theta}{y_1{:}\istrue{\susp{\sf gen\_string}},
           y_2{:}\istrue{\susp{\sf R}}}$, \\then 
$T = (T'; \trstep{y_1,y_2}{{\sf string/right}\,y'})$ for some $y'$.
\end{itemize}
To give the last of the three statements as a picture:
\begin{center}
\begin{tikzpicture} 
\draw [->,decorate, 
decoration={snake,amplitude=.3mm,segment length=3mm,post length=1mm}] 
(-4.7,2) -- (-4.7,0); 
\draw (-4.55,0) node{$_*$};
\draw (-4.7,1.65) node[left]{$T$};
\draw (-4.7,2.3) node{$g{:}\istrue{\susp{\sf gen}}$};
\draw (-4.7,-.3) node{$\tackon{\Theta'}{y_1{:}\istrue{\susp{\sf gen\_string}},
           y_2{:}\istrue{\susp{\sf R}}}$};
%
\draw (-4.2,1.65) node{$=$};
%
\draw [->,densely dotted,decorate, 
decoration={snake,amplitude=.3mm,segment length=3mm,post length=1mm}] 
(2.2,2) -- (2.2,1.3); 
\draw (2.35,1.3) node{$_*$};
\draw (2.2,1.65) node[left]{$T'$};
\draw [->,densely dotted,decorate, 
decoration={snake,amplitude=.3mm,segment length=3mm,post length=1mm}] 
(2.2,.7) -- (2.2,0); 
\draw (2.2,2.3) node{$g{:}\istrue{\susp{\sf gen}}$};
\draw (2.2,.35) node[left]{$\trstep{y_1,y_2}{{\sf string/right}\,y'}$};
\draw (2.2,1) node{$\tackon{\Theta'}{y'{:}\istrue{\susp{\sf gen\_string}}}$};
\draw (2.2,-.3) node{$\tackon{\Theta'}{y_1{:}\istrue{\susp{\sf gen\_string}},
           y_2{:}\istrue{\susp{\sf R}}}$};
\end{tikzpicture} 
\end{center}
\end{lemma}
\begin{proof}
The proofs are all by induction on the structure of $T$ and case
analysis on the last steps in $T$; we will prove the last statement, as
the other two are similar. Up to concurrent equality, there are two
possibilities:
\begin{itemize}
\item $T = (T'; \trstep{y_1, y_2}{{\sf string/right}\,y'})$ -- Immediate.
\item $T = (T''; S)$, where $y_1$ and $y_2$ are not among the output variables $S{^\bullet}$. By the 
induction hypothesis, $T'' = (T'''; \trstep{y_1,y_2}{{\sf string/right}\,y'})$.
Let $T' = (T''; S)$. 
\end{itemize}
This completes the proof. 
\end{proof}

Note that we do not consider any cases where 
$T = (T'; \trstep{y_1',y_2}{{\sf string/right}\,y'})$ (for $y_1 \neq y_1'$),
$T = (T'; \trstep{y_1',y_2}{{\sf string/right}\,y'})$ (for $y_2 \neq y_2'$),
or (critically) where 
$T = (T'; \trstep{y_1,y_2'}{{\sf string/left}\,y'})$. There is no 
way for any of these traces to have the correct type, which makes
the resulting case analysis quite simple. 

\begin{proof}[Proof of Theorem~\ref{thm:pda-preservation} (Preservation)]
By case analysis on the structure of $S$. 

\bigskip
\noindent
{\bf Case 1:} $S = \trstep{x',h'}{{\sf push}\,(\tfuser{h}{y})}$,
which means that we are given the following 
generative trace in $\Sigma_{\it Gen}$:
\begin{align*}
& \qquad ({g{:}\istrue{\susp{\sf gen}}})\\
& T\\
& \qquad \tackon{\Theta}{h{:}\istrue{\susp{\sf hd}}, ~~
                   y{:}\istrue{\susp{\sf L}}}
\intertext{and we must construct a trace 
$({g{:}\istrue{\susp{\sf gen}}}) \leadsto^*_{\Sigma_{\it Gen}} 
\tackon{\Theta}{x'{:}\istrue{\susp{\sf L}},
                   h'{:}\istrue{\susp{\sf hd}}}$. Changing
$h$ to $h'$ is just renaming a bound variable, so we have}
& \qquad 
({g{:}\istrue{\susp{\sf gen}}})\\
& T'\\
& \qquad \tackon{\Theta}{h'{:}\istrue{\susp{\sf hd}}, ~~
                   y{:}\istrue{\susp{\sf L}}}
\intertext{
The corollary 
to the first inversion lemma above on $T'$ gives us}
T' = & \qquad 
({g{:}\istrue{\susp{\sf gen}}})\\
& T'';\\
& \qquad \tackon{\Theta}{
                   x_g{:}\istrue{\susp{\sf gen\_stack}}, ~~
                   h'{:}\istrue{\susp{\sf hd}}, ~~
                   y_g{:}\istrue{\susp{\sf gen\_string}}, ~~
                   y{:}\istrue{\susp{\sf L}}}\\
& \trstep{()}{{\sf stack/done}\,x_g};\\
& \trstep{()}{{\sf string/done}\,y_g}\\
& \qquad \tackon{\Theta}{h'{:}\istrue{\susp{\sf hd}}, ~~
                   y{:}\istrue{\susp{\sf L}}}
\intertext{The second inversion lemma (second part) on $T''$ gives us}
T'' = & \qquad 
({g{:}\istrue{\susp{\sf gen}}})\\
& T''';\\
& \qquad \tackon{\Theta}{
                   x_g{:}\istrue{\susp{\sf gen\_stack}}, ~~
                   h'{:}\istrue{\susp{\sf hd}}, ~~
                   y_g'{:}\istrue{\susp{\sf gen\_string}}}\\
& \trstep{y_g,y}{{\sf string/left}\,y_g'};\\
& \qquad \tackon{\Theta}{
                   x_g{:}\istrue{\susp{\sf gen\_stack}}, ~~
                   h'{:}\istrue{\susp{\sf hd}}, ~~
                   y_g{:}\istrue{\susp{\sf gen\_string}}, ~~
                   y{:}\istrue{\susp{\sf L}}}\\
\intertext{Now, we can construct the trace we need using $T'''$:}
& \qquad 
({g{:}\istrue{\susp{\sf gen}}})\\
& T''';\\
& \qquad \tackon{\Theta}{
                   x_g{:}\istrue{\susp{\sf gen\_stack}}, ~~
                   h'{:}\istrue{\susp{\sf hd}}, ~~
                   y_g'{:}\istrue{\susp{\sf gen\_string}}}\\
& \trstep{x', x_g'}{{\sf stack/left}\,x_g};\\
& \qquad \tackon{\Theta}{
                   x'{:}\istrue{\susp{\sf L}}, ~~
                   x_g'{:}\istrue{\susp{\sf gen\_stack}}, ~~
                   h'{:}\istrue{\susp{\sf hd}}, ~~
                   y_g'{:}\istrue{\susp{\sf gen\_string}}}\\
& \trstep{()}{{\sf stack/done}\,x_g'};\\
& \trstep{()}{{\sf string/done}\,y_g'}\\
& \qquad \tackon{\Theta}{
                   x'{:}\istrue{\susp{\sf L}}, ~~
                   h'{:}\istrue{\susp{\sf hd}}}
\end{align*}

\bigskip
\noindent
{\bf Case 2:} $S = \trstep{h'}{{\sf pop}\,(\tfuser{x}{\tfuser{h}{y}})}$,
which means that we are given the following 
generative trace in $\Sigma_{\it Gen}$:
\begin{align*}
& \qquad ({g{:}\istrue{\susp{\sf gen}}})\\
& T\\
& \qquad \tackon{\Theta}{x{:}\istrue{\susp{\sf L}}, ~~
                   h{:}\istrue{\susp{\sf hd}}, ~~
                   y{:}\istrue{\susp{\sf R}}}
\intertext{and we must construct a trace 
$({g{:}\istrue{\susp{\sf gen}}}) \leadsto^*_{\Sigma_{\it Gen}} 
\tackon{\Theta}{h'{:}\istrue{\susp{\sf hd}}}$. Changing
$h$ to $h'$ is just renaming a bound variable, so we have}
& \qquad ({g{:}\istrue{\susp{\sf gen}}})\\
& T'\\
& \qquad \tackon{\Theta}{x{:}\istrue{\susp{\sf L}}, ~~
                   h'{:}\istrue{\susp{\sf hd}}, ~~
                   y{:}\istrue{\susp{\sf R}}}
\intertext{The corollary to the first inversion lemma above on $T'$ gives us}
T' = & \qquad ({g{:}\istrue{\susp{\sf gen}}})\\
& T'';\\
& \qquad \tackon{\Theta}{x{:}\istrue{\susp{\sf L}}, ~~
                   x_g{:}\istrue{\susp{\sf gen\_stack}}, ~~
                   h'{:}\istrue{\susp{\sf hd}}, ~~
                   y_g{:}\istrue{\susp{\sf gen\_string}}, ~~
                   y{:}\istrue{\susp{\sf R}}}\\
& \trstep{()}{{\sf stack/done}\,x_g};\\
& \trstep{()}{{\sf string/done}\,y_g}\\
& \qquad \tackon{\Theta}{x{:}\istrue{\susp{\sf L}}, ~~
                   h'{:}\istrue{\susp{\sf hd}}, ~~
                   y{:}\istrue{\susp{\sf R}}}
\intertext{The second inversion lemma (first part) on $T''$ gives us}
T'' = & \qquad ({g{:}\istrue{\susp{\sf gen}}})\\
& T''';\\
& \qquad \tackon{\Theta}{x_g'{:}\istrue{\susp{\sf gen\_stack}}, ~~
                   h'{:}\istrue{\susp{\sf hd}}, ~~
                   y_g{:}\istrue{\susp{\sf gen\_string}}, ~~
                   y{:}\istrue{\susp{\sf R}}}\\
& \trstep{x, x_g}{{\sf stack/left}\,x_g'};\\
& \qquad \tackon{\Theta}{x{:}\istrue{\susp{\sf L}}, ~~
                   x_g{:}\istrue{\susp{\sf gen\_stack}}, ~~
                   h'{:}\istrue{\susp{\sf hd}}, ~~
                   y_g{:}\istrue{\susp{\sf gen\_string}}, ~~
                   y{:}\istrue{\susp{\sf R}}}\\
\intertext{The second inversion lemma (third part) on $T'''$ gives us}
T''' = & \qquad ({g{:}\istrue{\susp{\sf gen}}})\\
& T'''';\\
& \qquad \tackon{\Theta}{x_g'{:}\istrue{\susp{\sf gen\_stack}}, ~~
                   h'{:}\istrue{\susp{\sf hd}}, ~~
                   y_g'{:}\istrue{\susp{\sf gen\_string}}}\\
& \trstep{y_g, y}{{\sf string/right}\,y_g'};\\
& \qquad \tackon{\Theta}{x_g'{:}\istrue{\susp{\sf gen\_stack}}, ~~
                   h'{:}\istrue{\susp{\sf hd}}, ~~
                   y_g{:}\istrue{\susp{\sf gen\_string}}, ~~
                   y{:}\istrue{\susp{\sf R}}}\\
\intertext{Now, we can construct the trace we need using $T''''$:}
& \qquad ({g{:}\istrue{\susp{\sf gen}}})\\
& T'''';\\
& \qquad \tackon{\Theta}{x_g'{:}\istrue{\susp{\sf gen\_stack}}, ~~
                   h'{:}\istrue{\susp{\sf hd}}, ~~
                   y_g'{:}\istrue{\susp{\sf gen\_string}}}\\
& \trstep{()}{{\sf stack/done}\,x_g'};\\
& \trstep{()}{{\sf string/done}\,y_g'}\\
& \qquad \tackon{\Theta}{h'{:}\istrue{\susp{\sf hd}}}
\end{align*}

\noindent
These two cases represent the only two synthetic transitions that are possible
under the signature $\Sigma_{\it PDA}$, so we are done.
\end{proof}

Proving that generation under a generative signature like $\Sigma_{\it
  Gen}$ is invariant under transitions in a signature like
$\Sigma_{\it PDA}$ is something we will consider further
in Chapter~\ref{chapter-gen}.  
All such proofs essentially follow the structure of
Theorem~\ref{thm:pda-preservation}. First, we enumerate the synthetic
transitions associated with a given signature. Second, in each of those
cases, we use the type of the synthetic transition to perform
inversion on the structure of the given generative trace.
Third, we construct a generative
trace that establishes the fact that the invariant was preserved.

\subsection{Adequacy of the transition system}
\label{sec:pda-adequacy}

The hard work of adequacy is established by the preservation theorem; 
the actual adequacy theorem is just an enumeration in both directions.

\bigskip
\begin{theorem}[Adequacy]\label{thm:pda-adequacy}
  $\ctxinterp{k \rhd s} \leadsto_{\Sigma_{\it PDA}} \ctxinterp{k'
    \rhd s'}$ if and only if $\obj{k \rhd s} \mapsto \obj{k' \rhd s'}$.
\end{theorem}

\begin{proof}
  Both directions can be established by case analysis on the structure
  of $\obj{k}$ and $\obj{s}$.
\end{proof}

As an immediate corollary of this theorem and preservation
(Theorem~\ref{thm:pda-preservation}), we have the stronger adequacy
property that $\ctxinterp{k \rhd s} \leadsto_{\Sigma_{\it PDA}}
\Delta'$, then $\Delta' = \ctxinterp{k' \rhd s'}$ for some $\obj{k}$
and $\obj{s'}$ such that $\obj{k \rhd s} \mapsto \obj{k' \rhd s'}$.
In our two-dimensional notation, the complete discussion of adequacy
for \sls~is captured by the following picture:

\begin{center}
\begin{tikzpicture} 
\draw (.8,2.3) node{$(x{:}\istrue{\susp{\sf gen}})$};
\draw [->,decorate, 
decoration={snake,amplitude=.3mm,segment length=3mm,post length=1mm}] 
(0.8,2) -- (.8,.8); 
\draw (.9,.8) node{$_*$};
\draw (.3,1.4) node{$\Sigma_{\it Gen}$};
\draw (.8,.5) node{$\Delta$};
\draw [thick,dash pattern = on 2.82842842712mm off 2mm,decorate,decoration={saw,amplitude=2mm,segment length=2mm}] 
(.4,0) -- (1.2,0); 
\draw (.8,-.3) node{$\Delta$};
\draw (1.1,-.7) node{\begin{turn}{-45}$\Leftrightarrow$\end{turn}};
\draw (1.8,-1) node{$\obj{k \rhd s}$};
%
\draw (4.8,2.3) node{$(x{:}\istrue{\susp{\sf gen}})$};
\draw [->,densely dotted,decorate, 
decoration={snake,amplitude=.3mm,segment length=3mm,post length=1mm}] 
(4.8,2) -- (4.8,.8); 
\draw (4.9,.8) node{$_*$};
\draw (4.3,1.4) node{$\Sigma_{\it Gen}$};
\draw (4.8,.5) node{$\Delta'$};
\draw (2.8, 0) node{$\Sigma_{\it PDA}$};
\draw [thick,dash pattern = on 0.677mm off .4mm on 0.676142375mm off .4mm on 0.676142375mm off 2mm,decorate,decoration={saw,amplitude=2mm,segment length=2mm}] 
(4.4,0) -- (5.2,0); 
\draw (4.8,-.3) node{$\Delta'$};
\draw (5.1,-.7) node{\begin{turn}{-45}$\Leftrightarrow$\end{turn}};
\draw (5.8,-1) node{$\obj{k' \rhd s'}$};
%
\draw [->,decorate, 
decoration={snake,amplitude=.3mm,segment length=3mm,post length=1mm}] 
(1.2,-.3) -- (4.4,-.3); 
\draw [|->] (2.5,-1.04) -- (5,-1.04);;
\end{tikzpicture} 
\end{center}

\section{The \sls~implementation}
\label{sec:prototype}

The prototype implementation of \sls~contains a parser and typechecker
for the SLS language, and is available from
\url{https://github.com/robsimmons/sls}. Code that is checked by this
prototype implementation will appear frequently in the rest of this
thesis, always in a \verb|fixed-width font|.

\begin{figure}
\newcommand{\thingamajig}{=}
\begin{align*}
{\downarrow}A^- & \thingamajig \mbox{\Verb|A|} 
 & {\ocircle}A^+ & \thingamajig \mbox{\Verb|\{A\}|}
 & \lf{\lambda a.t} & \thingamajig \mbox{{\texttt{\char`\\}}\Verb|a.t|}
\\
{\gnab}A^- & \thingamajig \mbox{\Verb|\$A|}
 & A^+ \lefti B^- & \thingamajig \mbox{\Verb|A >-> B|}
 & \lf{{\sf foo}\,t_1\ldots t_n} & \thingamajig \mbox{\Verb|foo t1...tn|}
\\
{!}A^- & \thingamajig \mbox{\Verb|!A|}
 & A^+ \righti B^- & \thingamajig \mbox{\Verb|A ->> B|}
\\
\one & \thingamajig \mbox{\Verb|one|}
 & A^- \with B^- & \thingamajig \mbox{\Verb|A \& B|}
 & \Pi\lf{a}{:}\tau.\nu & \thingamajig \mbox{\Verb|Pi x.nu|}
\\
A^+ \fuse B^+ & \thingamajig \mbox{\Verb|A * B|}
 & \forall \lf{a}.\tau. A^- & \thingamajig \mbox{\Verb|All x.A|}
 & \tau \rightarrow \nu & \thingamajig{\mbox{\Verb|tau -> nu|}}
\\
\exists \lf{a}.\tau. A^+ & \thingamajig \mbox{\Verb|Exists x.A|}
 & {\gnab}A^- \lefti B^- & \thingamajig \mbox{\Verb|A -o B|}
 & {\sf bar}\,\lf{t_1 \ldots t_n} & \thingamajig{\mbox{\Verb|bar t1...tn|}}
\\
\lf{t} \doteq \lf{s} & \thingamajig \mbox{\Verb|t == s|}
 & {!}A^- \lefti B^- & \thingamajig \mbox{\Verb|A -> B|}
\end{align*}
\caption{Mathematical and ASCII representations of propositions,
  terms, and classifiers}
\label{fig:translate-types}
\end{figure}


The checked \sls~code differs slightly from mathematical
\sls~specifications in a few ways -- the translation between the
mathematical notation we use for \sls~propositions and the ASCII
representation used in the implementation is outlined in
Figure~\ref{fig:translate-types}.  Following CLF and the Celf
implementation, we write the lax modality ${\ocircle}A$ in ASCII as
\verb|{A}| -- recall that in Section~\ref{sec:slsframework} we
introduced the $\{ A^+ \}$ notation from CLF as a synonym for
Fairtlough and Mendler's ${\ocircle}A^+$.  The exponential ${\gnab}A$
doesn't have an ASCII representation, so we write \verb|$A| when $A$
is mobile. Upshifts and downshifts are always inferred: this means
that we can't write down ${\uparrow}{\downarrow}A$ or
${\downarrow}{\uparrow}A$, but neither of these \ollll~propositions
are part of the \sls~fragment anyway.

The \sls~implementation also supports conventional abbreviations for
arrows that we won't use in mathematical notation: ${\gnab}A^- \lefti
B^-$ can be written as \verb|A -o B| or \verb|$A >-> B| in the
\sls~implementation, and ${!}A^- \lefti B^-$ can be written as
\verb|A -> B| or \verb|!A >-> B|.  This final proposition is
ambiguous, because \verb|X -> Y| can be an abbreviation for ${!}X
\lefti Y$ or $\Pi \lf{a}{:}X. Y$, but \sls~can figure out whether the
proposition or classifier was intended by analyzing the structure of
\verb|Y|. Also note that we could have just as easily made
\verb|A -o B| an abbreviation for \verb|$A -o B|, but we had to pick
one and the choice absolutely doesn't matter.  All arrows can also be
written backwards: \verb|B <-< A| is equivalent to \verb|A >-> B|,
\verb|B o- A| is equivalent to \verb|A -o B|, and so on. Also
following traditional conventions, upper-case variables that are free
in a rule will be treated as implicitly quantified. Therefore, the
line \bigskip
\begin{verbatim}
rule: foo X <- (bar Y -> baz Z).
\end{verbatim}
\bigskip
will be reconstructed as the \sls~declaration 
\[{\sf rule} : \forall\lf{Y}{:}\tau_1.\,\forall\lf{Z}{:}\tau_2.\,\forall\lf{X}{:}\tau_3.\,{!}({!}{\sf bar}\,\lf{Y} \lefti {\sf baz}\,\lf{Z}) \lefti {\sf foo}\,\lf{X}\] 
%
where the implementation infers the types $\tau_1$, $\tau_2$, and
$\tau_3$ appropriately from the declarations of the negative
predicates ${\sf foo}$, ${\sf bar}$, and ${\sf baz}$.

Another significant piece of syntactic sugar introduced for 
the sake of readability is less conventional, if only because
positive atomic propositions are not conventional. If \verb|P| is a
persistent atomic proposition, we can optionally write \verb|!P|
wherever \verb|P| is expected, and if \verb|P| is a linear atomic
proposition, we can write \verb|$P| wherever \verb|P| is
expected. This means that if ${\sf a}$, ${\sf b}$, and ${\sf c}$ are
(respectively) ordered, linear, and persistent positive atomic
propositions, we can write the positive proposition ${\sf a} \fuse
{\sf b} \fuse {\sf c}$ in the \sls~implementation as
\verb|(a * b * c)|, \verb|(a * $b * c)|, \verb|(a * b * !c)|, or
\verb|(a * $b * !c)|. Without these annotations, it is difficult to
tell at a glance which propositions are ordered, linear, or persistent
when a signature uses more than one proposition. When all of these
optional annotations are included, the rules in a signature that uses
positive atomic propositions look the same as rules in a signature
that uses the pseudo-positive negative atomic propositions described
in Section~\ref{sec:pseudopositive}. 


In the code examples given in the remainder of this thesis, we will
use these optional annotations in a consistent way.  We will omit the
optional \verb|$A| annotations only in specifications with no ordered
atomic propositions, and we will omit the optional \verb|!A|
annotations in specifications with no ordered or linear atomic
propositions. This makes the mixture of different exponentials
explicit while avoiding the need for rules like
\verb|($a * $b * $c >-> {$d * $e})| when specifications are entirely
linear (and likewise when specifications are entirely persistent).

\section{Logic programming}
\label{sec:framework-logicprog}

One logic programming interpretation of CLF was explored by the
Lollimon implementation \cite{lopez05monadic} and adapted by the Celf
implementation
\cite{schacknielsen08celf,schacknielsen11implementing}. Logic
programming interpretations of \sls~are not a focus this thesis, but
we will touch on a few points in this section.

Logic programming is important because it provides us with operational
intuitions about the intended behavior of the systems we specify in
\sls. One specific set of these intuitions will form the basis of the
operationalization transformations on \sls~specifications considered
in Chapter~\ref{chapter-absmachine}. 
Additionally, logic programming intuitions are relevant
because they motivated the design of \sls, in particular the
presentation of the concurrent fragment in terms of partial, rather
than complete, proofs. We discuss this point in
Section~\ref{sec:framework-logicprog-trace}.

\subsection{Deductive computation and backward chaining}
\label{sec:framework-logicprog-deductive}
\label{sec:framework-modes}

Deductive computation in \sls~is the search for {\it complete} proofs
of sequents of the form $\foc{\Psi}{\Delta}{\istrue{\susp{p^-}}}$.  A
common form of deductive computation is {\it goal-directed search}, or
what Andreoli calls the {\it proof construction paradigm}
\cite{andreoli01focussing}.
In \sls, goal-directed search for the proof of a sequent
$\foc{\Psi}{\Delta}{\istrue{\susp{p^-}}}$ can only proceed by focusing
on a proposition like ${\downarrow}p_n^- \lefti \ldots \lefti
{\downarrow}p_1^- \lefti p^-$ which has a head $p^-$ that matches the
succedent. This replaces the goal sequent
$\foc{\Psi}{\Delta}{\istrue{\susp{p^-}}}$ with $n$ subgoals:
$\foc{\Psi}{\Delta_1}{\istrue{\susp{p_1^-}}}$ \ldots
$\foc{\Psi}{\Delta_n}{\istrue{\susp{p_n^-}}}$, where $\Delta$ matches
$\Delta_1,\ldots,\Delta_n$.

When goal-directed search only deals with the unproved subgoals of a
single coherent derivation at a time, it is called {\it backward
  chaining}, because we're working backwards from the goal we want to
prove.\footnote{The alternative is to try and derive the same sequent
  in multiple ways simultaneously, succeeding whenever some way of
  proving the sequent is discovered. Unlike backward chaining, this
  strategy of breadth-first search is complete: if a proof exists, it
  will be found.  Backward chaining as we define it is only
  nondeterministically or partially complete, because it can fail to
  terminate when a proof exists. We will call this alternative to
  backtracking {\it breadth-first theorem proving}, as it amounts to
  taking a breadth-first, instead of depth-first, view of the
  so-called {\it failure continuation} \cite{pfenning06backtracking}.}
The term {\it top-down logic programming} is also used, and refers to
the fact that, in the concrete syntax of Prolog, the rule
${\downarrow}p_n^- \lefti \ldots \lefti {\downarrow}p_1^- \lefti p^-$
would be written with $p^-$ on the first line, $p_1^-$ on the second,
etc. This is exactly backwards from a proof-construction perspective,
as we think of backward chaining as building partial proofs from the
bottom up, the root towards the leaves, so we will avoid this
terminology.

The backward-chaining interpretation of intuitionistic logics dates
back to the work by Miller et al.~on uniform proofs
\cite{miller91uniform}.  An even older concept, Clark's {\it
  negation-as-failure} \cite{clark87negation}, is based on a {\it
  partial completeness} criteria for logic programming interpreters.
Partial correctness demands that if the
interpreter reports that it has found a proof of a goal-directed
sequent, such a proof should exist. Partial completeness, on the other
hand, demands that if the interpreter gives up up on finding a proof,
no proof should exist. (The interpreter is also allowed to run forever
without succeeding or giving up.)  Partial completeness requires {\it
  backtracking} in backward-chaining search: if we we try to prove
$\foc{\Psi}{\Delta}{\istrue{\susp{p^-}}}$ by focusing on a particular
proposition and one of the resulting subgoals fails to be provable, we
have to consider any other propositions that could have been used to
prove the sequent before giving up. Backtracking can be extremely
powerful in certain cases and incredibly expensive in others, and so
most logic programming languages have an escape hatch that modifies or
limits backtracking at the user's discretion, such as the Prolog cut
(no relation to the admissible rule ${\it cut}$) or Twelf's
deterministic declarations. Non-backtracking goal-oriented deductive
computation is called {\it flat resolution} \cite{aitkaci99warrens}.

One feature of backward chaining and goal directed search is that it
usually allows for terms that are not completely specified -- these
unspecified pieces are are traditionally called {\it logic
  variables}. Because LF variables are also ``logic variables,''
the literature on $\lambda$Prolog and Twelf calls unspecified
pieces of terms {\it existential variables}, 
but as they bear no relation to the variables introduced
by the left rule for $\exists \lf{a}{:}\tau. A^+$, 
that terminology is also unhelpful
here. Consider the following \sls~signature:
\begin{align*}
 \Sigma_{\it Add} = \cdot, 
~&{\sf nat} : {\sf type}, 
~~\lf{\sf z} : {\sf nat}, 
~~\lf{\sf s} : {\sf nat} \rightarrow {\sf nat},\\
~&{\sf plus} : {\sf nat} \rightarrow {\sf nat} \rightarrow {\sf nat} 
                 \rightarrow {\sf prop},\\
~&{\sf plus/z} : \forall \lf{N}{:}{\sf nat}.\,
({\sf plus}\,\lf{\sf z}\,\lf{N}\,\lf{N}),\\
~&{\sf plus/s} : \forall \lf{N}{:}{\sf nat}.\, 
                 \forall \lf{M}{:}{\sf nat}.\, 
                 \forall \lf{P}{:}{\sf nat}.\,
{!}({\sf plus}\,\lf{N}\,\lf{M}\,\lf{P})
\lefti ({\sf plus}\,({\sf s}\,\lf{N})\,\lf{M}\,({\sf s}\,\lf{P}))
\end{align*}
In addition to searching for a proof of ${\sf plus}\,\lf{\sf
  (s\,z)}\,\lf{\sf (s\,z)}\,\lf{\sf (s\,(s\,z))}$ (which will succeed,
as $1 + 1 = 2$) or searching for a proof of ${\sf plus}\,\lf{\sf
  (s\,z)}\,\lf{\sf (s\,z)}\,\lf{\sf (s\,(s\,(s\,z)))}$ (which will
fail, as $1 + 1 \neq 3$), we can use goal-oriented deductive
computation to search for ${\sf plus}\,\lf{\sf (s\,z)}\,\lf{\sf
  (s\,z)}\,X$, where $X$ represents an initially unspecified term.
This search will succeed, reporting that $X = \lf{\sf
  (s\,(s\,z))}$. Unification is generally used in backward-chaining
logic programming languages as a technique for implementing partially
unspecified terms, but this implementation technique should not be
confused with our use of unification-based equality $\lf{t} \doteq
\lf{s}$ as a proposition in \sls.

We say that ${\sf plus}$ in the signature above is a {\it well-moded}
predicate with {\it mode} $({\sf plus}\,{+}\,{+}\,{-})$, because
whenever we perform deductive computation to derive $({\sf
  plus}\,\lf{n}\,\lf{m}\,\lf{p})$ where $\lf{n}$ and $\lf{m}$ are
fully specified, any unspecified portion of $\lf{p}$ must be fully
specified in any completed derivation. Well-moded predicates can be
treated as nondeterministic partial functions from their inputs (the
indices marked ``${+}$'' in the mode) to their outputs (the indices
marked ``${-}$'' in the mode). A predicate can sometimes be given more
than one mode: $({\sf plus}\,{+}\,{-}\,{+})$ is a valid mode for ${\sf
  plus}$, but $({\sf plus}\,{+}\,{-}\,{-})$ is not.

The implementation of backward chaining in substructural logic has
been explored by Hodas \cite{hodas94logic}, Polakow
\cite{polakow00linear,polakow01ordered}, Armel\'in and Pym
\cite{armelin01bunched}, and others. Efficient implementation of these
languages is complicated by the problem of {\it resource
  management}. In linear logic proof search, it would be technically
correct but highly inefficient to perform proof search by enumerating
the ways that a context can be split and then backtracking over each
possible split. Resource management allows the interpreter to avoid
this potentially exponential backtracking, but describing resource
management and proving it correct, especially for richer substructural
logics, can be complex and subtle \cite{cervesato00efficient}.

The term {\it deductive computation} is meant to be interpreted very
broadly, and goal-directed search is not the only form of deductive
computation. Another paradigm for deductive computation is the {\it
  inverse method}, where the interpreter attempts to prove a sequent
$\foc{\Psi}{\Delta}{\istrue{\susp{p^-}}}$ by creating and growing
database of sequents that are derivable, attempting to build the
appropriate derivation from the leaves down. The inverse method is
generally associated with theorem proving and not logic
programming. However, Chaudhuri, Pfenning, and Price have shown that
that deductive computation with the inverse method in a focused linear
logic can simulate both backward chaining and forward chaining
(considered below) for persistent Horn-clause logic programs
\cite{chaudhuri10logical}. 

\begin{figure}
\begin{tikzpicture}
\draw (0,10) node{~};
\draw (8,10.45) node{\bf Deductive computation};
\draw (8,10) node{Search for complete derivations $\foc{\Psi}{\Delta}{U}$};
\draw [->] (7.5,9.5) -- (6.5,8.5); 
\draw (7.2,9.3) node[left]{\it maintains sets of};
\draw (6.8,8.9) node[left]{\it subgoal sequents};
\draw [->] (8.5,9.5) -- (9.5,8.5); 
\draw (8.9,9.3) node[right]{\it maintains sets of};
\draw (9.2,8.9) node[right]{\it derivable sequents};
%
\draw (6,8) node{\bf goal-directed search};
\draw [->] (5.5,7.5) -- (4.5,6.5); 
\draw (5,7.1) node[left]{\it depth-first};
\draw [->] (6.5,7.5) -- (7.5,6.5); 
\draw (7,7.1) node[right]{\it breadth-first};
%
\draw (4,6) node{\bf backward chaining};
\draw [->] (3.5,5.5) -- (2.5,4.5); 
\draw (3,5.1) node[left]{\it backtracking};
\draw [->] (4.5,5.5) -- (5.5,4.5); 
\draw (5,5.1) node[right]{\it committed-choice};
%
\draw (2,4) node{\bf backward chaining};
%
\draw (6,4) node{\bf flat resolution};
%
\draw (9.6,6) node{\bf breadth-first theorem proving};
%
\draw (11.6,8) node{\bf inverse method theorem proving};
%
% \draw (13,10.45) node{\bf Trace computation};
% \draw (13,10) node{Search for partial proofs 
%    $(\Psi; \Delta) \leadsto^* (\Psi'; \Delta')$};
% \draw [->] (12.5,9.5) -- (11.5,8.5); 
% \draw [->] (13.5,9.5) -- (14.5,8.5); 
%
\end{tikzpicture}
\caption{A rough taxonomy of deductive computation}
\label{fig:computation-taxonomy}
\end{figure}

Figure~\ref{fig:computation-taxonomy} gives an taxonomy (incomplete
and imperfect) of the forms of deductive computation mentioned in this
section. Note that, while we will generally use {\it backward
  chaining} to describe backtracking search, backward chaining does
not always imply full backtracking and partial completeness. This
illustration, and the preceding discussion, leaves out many important
categories, especially tabled logic programming, and many potentially
relevant implementation choices, such as breath-first versus
depth-first or parallel exploration of the success continuation.


\subsection{Concurrent computation}
\label{sec:framework-logicprog-trace}


Concurrent computation is the search for {\it partial} proofs of
sequents. As the name suggests, in \sls~concurrent computation is
associated with the search for partial proofs of the judgment
$\islax{A^+}$, which correspond to traces $(\Psi;
\Delta) \leadsto^* (\Psi'; \Delta')$. 

The paradigm we will primarily associate with concurrent computation
is {\it forward chaining}, which implies that we take an initial
process state $(\Psi;\Delta)$ and allow it to evolve freely by the
application of synthetic transitions. Additional conditions can be
imposed on forward chaining: for instance, synthetic transitions like
$(\Delta, x{:}\ispers{\susp{p^+_\mpers}}) \leadsto (\Delta,
x{:}\ispers{\susp{p^+_\mpers}}, y{:}\ispers{\susp{p^+_\mpers}})$ that
do not meaningfully change the state can be excluded (if a persistent
proposition already exists, two copies of that proposition don't add
anything).\footnote{Incidentally, Lollimon implements this restriction
  and Celf does not.} Forward chaining with this restriction in a
purely-persistent logic is strongly associated with the Datalog
language and its implementations; we will refer to forward chaining in
persistent logics as {\it saturating logic programming} in 
Chapter~\ref{chapter-approx}.
Forward chaining does not always deal with partially-unspecified
terms; when persistent logic programming languages support forward
chaining with partially-unspecified terms variables, it is called {\it
  hyperresolution} \cite{fermuller01resolution}.

The presence of ephemeral or ordered resources in substructural logic
means that a process state may evolve in multiple
mutually-incompatible ways. {\it Committed choice} is a version of
forward chaining that never goes back and reconsiders alternative
evolutions from the initial state. Just as the default interpretation
of backward chaining includes backtracking, we will consider the
default interpretation of forward chaining to be committed choice,
following \cite{lopez05monadic}.  An alternate interpretation of
forward chaining would consider multiple evolutionary paths, which is
a version of {\it exhaustive search}.  Trace computation that works backwards
from a final state instead of forward from an initial state can also
be considered, and {\it planning} can be seen as specifying both the
initial and final process states and trying to extrapolate a trace
between them by working in both directions.

Outside of this work and Saurin's work on Ludics programming
\cite{saurin08towards}, there is not much work on explicitly
characterizing and searching for partial proofs in substructural
logics.\footnote{As such, ``concurrent computation,'' while
  appropriate for \sls, may or may not prove to be a good name for the
  general paradigm.}  Other forms of computation can be characterized
as trace computation, however.  Multiset rewriting and languages like
GAMMA can be partially or completely understood in terms of forward
chaining in linear logic \cite{cervesato09relating,paola96linear}, and
the ordered aspects of \sls~allow it to capture fragments of rewriting
logic. Rewriting logic, and in particular the Maude implementation of
rewriting logic \cite{clavel11ltl}, implements the committed choice
interpretation and exhaustive search interpretations, as well as a
{\it model checking} interpretation that characterize sets of process
states or traces using logical formulas. Constraint handling
rules \cite{betz10complete} and concurrent constraint programming
\cite{jagadeesan05testing} are other logic programming models can be 
characterized as forms of concurrent computation.


% not consider unspecified variables and is largely free of the resource
% management problems that appear in backward-chaining deductive
% computation for substructural logics. In previous work, we designed a
% forward chaining interpreter for linear logic that admits abstract
% reasoning about the asymptotic complexity of logical specifications
% \cite{simmons08linear}, but this is outside the scope of this thesis.



\subsection{Integrating deductive and trace computation}

In the logic programming interpretation of CLF used by Lollimon and
Celf, backtracking backward chaining is associated with the deductive
fragment, and committed-choice forward chaining is associated with the
lax modality. We will refer to an adaptation of the Lollimon/Celf
semantics to \sls~as LCI (``Lollimon/Celf Interpreter'') for
brevity in this section.

Forward chaining and backward chaining have an uneasy relationship in
LCI. To see why, consider the following \sls~signature:
\begin{align*}
 \Sigma_{\it Demo} = \cdot, 
~&{\sf posA} : {\sf prop}\,{\sf ord}, 
~~{\sf posB} : {\sf prop}\,{\sf ord}, 
~~{\sf posC} : {\sf prop}\,{\sf ord}, 
~~{\sf negD} : {\sf prop},\\
~&{\sf fwdruleAB} : {\sf posA} \lefti {\ocircle}{\sf posB},\\
~&{\sf fwdruleAC} : {\sf posA} \lefti {\ocircle}{\sf posC},\\
~&{\sf bwdrule} : ({\sf posA} \lefti {\ocircle}{\sf posB}) \lefti {\sf negD}
\end{align*}

In an empty context, there is only one derivation of ${\sf negD}$
under this signature: it is represented by the proof term ${\sf
  bwdrule}\,(\lambda x.\,\tlet{\trstep{y}{{\sf
      fwdruleAB}\,x}}{y})$. The partially complete interpretation of
backward chaining stipulates that an interpreter tasked with finding a
proof of ${\sf negD}$ should either find this proof or never
terminate, but LCI only admits this interpretation for purely
deductive proofs. To see why, consider backward-chaining search
attempting to prove ${\sf negD}$ in a closed context.  This can only
be done with the rule ${\sf bwdrule}$, generating the subgoal ${\sf
  posA} \lefti {\ocircle}{\sf posB}$.  At this point, LCI will switch
from backward chaining to forward chaining and attempt to satisfy this
subgoal by constructing a trace $(x{:}\istrue{\susp{\sf posA}})
\leadsto (y{:}\istrue{\susp{\sf posB}})$.

There are {\it two} nontrivial traces in this signature starting from
the process state $(x{:}\istrue{\susp{\sf posA}})$ -- the first is
$(\trstep{y}{{\sf fwdruleAB}\,x}) :: (x{:}\istrue{\susp{\sf posA}})
\leadsto (y{:}\istrue{\susp{\sf posB}})$, and the second is
$(\trstep{y}{{\sf fwdruleAC}\,x})::(x{:}\istrue{\susp{\sf posA}})
\leadsto (y{:}\istrue{\susp{\sf posC}})$.  Forward chaining can
plausibly come up with either one, and if it happens to derive the
second one, the subgoal fails. LCI then tries to backtrack to find
other rules that can prove the conclusion ${\sf negD}$, but there are
none, so LCI will report that it failed to prove ${\sf negD}$.

This example indicates that it is difficult to make backward chaining
(in its default backtracking form) reliant on committed-choice forward
chaining (in its default committed-choice form) in the style of
Lollimon or Celf. Either we can restrict forward chaining to confluent
systems (excluding $\Sigma_{\it Demo}$) or else we can give up on the
usual partially complete interpretation of backward chaining.  In the
other direction, however, it is entirely natural to make forward
chaining dependent upon backward chaining. The fragment of CLF that
encodes this kind of computation was labeled the {\it semantic
  effects} fragment by DeYoung \cite{deyoung09reasoning}. At the
logical level, the semantic effects fragment of \sls~removes the right
rule for ${\ocircle}A^+$, which corresponds to the proof term
$\tlet{T}{V}$.  As discussed in
Section~\ref{sec:framework-concurrent}, let-expressions are the only point
where traces are included into the language of
deductive terms.

\section{Design decisions}
\label{sec:designdecisions}

Aside from ordered propositions, there are several significant
differences between the framework \sls~presented in this chapter and
the existing logical framework CLF, including the presence of positive
atomic propositions, the introduction of traces as an explicit
notation for partial proofs, the restriction of the term language to
LF, and the presence of equality $\lf{t} \doteq \lf{s}$ as a
proposition. In this section, we will discuss design choices that were
made in terms of each of these features, their effects, and what
choices could have been made differently.

\subsection{Pseudo-positive atoms}
\label{sec:pseudopositive}

Unlike \sls, the CLF framework does not include positive atomic
propositions. Positive atomic propositions make it easy to
characterize the synthetic transitions associated with a particular
rule. For example, if ${\sf foo}$, ${\sf bar}$, and ${\sf baz}$ are
all linear atomic propositions, then the presence of a rule ${\sf
  somerule} : \left({\sf foo} \fuse {\sf bar} \lefti {\ocircle}{\sf
    baz}\right)$ in the signature is associated with synthetic
transitions of the form
%
$(\Psi; \matchconj{\Delta}{\matchconj{x{:}\iseph{\susp{\sf
        foo}}}{y{:}\iseph{\susp{\sf bar}}}})
 \leadsto
 (\Psi; \mkconj{\Delta}{z{:}\iseph{\susp{\sf baz}}})$.
%
The presence of the
rule $\sf somerule$ enables steps of this form, and every step made by
focusing on the rule has this form.

CLF has no positive propositions, so the closest analogue that we can
consider is where ${\sf foo}$, ${\sf bar}$, and ${\sf baz}$ are
negative propositions, and the rule ${\gnab}{\sf foo} \fuse
{\gnab}{\sf bar} \lefti \ocircle({\gnab}{\sf baz})$ appears in the
signature. Such a rule is associated with synthetic transitions of the
form
%
$(\Psi; \matchconj{\Delta}{\matchconj{\Delta_1}{\Delta_2}}) \leadsto
(\Psi; \mkconj{\Delta}{z{:}\istrue{{\sf baz}}})$ such that
$\foc{\Psi}{\restrictto{\Delta_1}{\meph}}{\istrue{\susp{\sf foo}}}$
and $\foc{\Psi}{\restrictto{\Delta_2}{\meph}}{\istrue{\susp{\sf
      bar}}}$. In \sls, it is a relatively simple syntactic criterion
to enforce that a sequent like $\foc{\Psi}{\Delta_1}{\istrue{\susp{\sf
      foo}}}$ can only be derived if $\Delta_1$ matches
$x{:}\sf{foo}$; we must simply ensure that there are no propositions
of the form $\ldots \lefti {\sf foo}$ or $\ldots \righti {\sf foo}$ in
the signature or context. (In fact, this is essentially the
\sls~version of the subordination criteria that allows us to conclude
that an LF type was only inhabited by variables in 
Section~\ref{sec:slsframework}.)  Note that, in full \ollll, this task
would not be so easy: we might prove $\istrue{\susp{\sf foo}}$
indirectly by forward chaining. This is one reason why association of
traces with the lax modality is so important!

When it is the case that  $\foc{\Psi}{\Delta_1}{\istrue{\susp{\sf
      foo}}}$ can only be derived if $\Delta_1$ matches
$x{:}\sf{foo}$, we can
associate the rule ${\gnab}{\sf foo} \fuse
{\gnab}{\sf bar} \lefti \ocircle({\downarrow}({\gnab}{\sf baz}))$
with the synthetic transition $(\Psi;
\matchconj{\Delta}{\matchconj{x{:}{\islvl{\sf foo}}}{y{:}{\islvl{\sf
        bar}'}}}) \leadsto (\Psi; \mkconj{\Delta}{z{:}\iseph{\susp{\sf
      baz}}})$ under the condition that neither $\mlvl$ or $\mlvl'$
are $\mtrue$.
Negative atomic propositions that can only be concluded when they are
the sole member of the context, like ${\sf foo}$ and ${\sf bar}$ in
this example, can be called {\it
  pseudo-positive}. Pseudo-positive atoms can actually be used a bit
more generally than \sls's positive atomic propositions. A positive
atomic proposition is necessarily associated with one of the three
judgments $\mtrue$, $\meph$, or $\mpers$, but pseudo-positive
propositions can associate with any of the contexts. This,
incidentally, gives pseudo-positive atoms in CLF or \sls~the flavor of
positive atomic propositions under Andreoli's atom optimization
(Section~\ref{sec:atomopt}).

It is, of course, possible to consistently associate particular
pseudo-positive propositions with particular modalities, which means
that pseudo-positive propositions can subsume the positive
propositions of \sls. The tradeoff between positive and
pseudo-positive propositions could be resolved either way. By
including positive atomic propositions, we made \sls~more complicated,
but in a local way -- we needed a few more kinds and a few more
rules. On the other hand, if we used pseudo-positive propositions, the
notion of synthetic transitions would be intertwined with the
subordination-like analysis that enforces their correct usage.

\subsection{The need for traces}
\label{sec:whytraces}

One of the most important differences between \sls~and its
predecessors, especially CLF, is that traces are treated as
first-class syntactic objects. This allows us to talk about 
partial proofs and thereby encode our earlier 
money-store-battery-robot example as a trace with this type:
\begin{align*}
& \left(
 x{:}\iseph{\susp{\sf 6bucks}}, ~~
 f{:}\iseph{({\sf battery} \lefti {\ocircle}{\sf robot})}, ~~
 g{:}\ispers{({\sf 6bucks} \lefti {\ocircle}{\sf battery})}
\right)
\\
\leadsto^* &
\left(
 z{:}\iseph{\susp{\sf robot}}, ~~
 g{:}\ispers{({\sf 6bucks} \lefti {\ocircle}{\sf battery})}
\right)
\end{align*}
It is also possible to translate the example from Chapter~\ref{chapter-foc}
as a {\it complete} proof of the following proposition:
\[
  {\sf 6bucks} 
      \fuse {\gnab}({\sf battery} \lefti {\ocircle}{\sf robot})
      \fuse {!}({\sf 6bucks} \lefti {\ocircle}{\sf battery})
     \lefti
     {\ocircle}{\sf robot}
\]

Generally speaking, we can try to represent a trace $T :: (\Psi;
\Delta) \leadsto^* (\Psi'; \Delta')$ as a closed deductive proof
$\lambda P.\,\tlet{T}{V}$ of the proposition $(\exists
\Psi.\,{\fuse}\Delta) \lefti {\ocircle}(\exists
\Psi'.\,{\fuse}\Delta)$,\footnote{The notation ${\fuse}{\Delta}$ fuses
  together all the propositions in the context. For example, if
  $\Delta = w{:}\iseph{\susp{p^+_\meph}} \fuse x{:}\istrue{A^-},
  y{:}\iseph{B^-}, z{:}\ispers{C^-}$, then ${\fuse}{\Delta} = p^+_\meph
  \fuse {\downarrow}A^- \fuse {\gnab}B^- \fuse {!}C^-$. The notation
  $\exists \Psi. A^+$ turns all the bindings in the context $\Psi =
  \lf{a_1}{:}\tau_1,\ldots,\lf{a_n}{:}\tau_n$ into existential
  bindings $\exists \lf{a_1}{:}\tau_1\ldots\exists
  \lf{a_n}{:}\tau_n.A^+$.}  where the pattern $P$ re-creates the initial
process state $(\Psi; \Delta)$ and all the components of the final
state are captured in the value $V$.  The problem with this approach
is that the final proposition is under no particular obligation to
faithfully
capture the structure of the final process state. This can be seen in
the example above: to actually capture the structure of the final
process state, we should have concluded ${\sf robot} \fuse {!}({\sf
  6bucks} \lefti {\ocircle}{\sf battery})$ instead of simply ${\sf
  robot}$. It is also possible to conclude any of the following:
\smallskip
\begin{enumerate}
\item ${\sf robot} \fuse {!}({\sf 6bucks} \lefti {\ocircle}{\sf
  battery}) \fuse {!}({\sf 6bucks} \lefti {\ocircle}{\sf
  battery})$, or 
\item ${\sf robot} \fuse {\downarrow}({\sf 6bucks} \lefti {\ocircle}{\sf
  battery}) \fuse {\gnab}({\sf 6bucks} \lefti {\ocircle}{\sf
  battery})$, or even
\item ${\sf robot}  \fuse {\gnab}({\sf 6bucks} \fuse {!}({\sf battery} \lefti {\ocircle}{\sf robot} \}) \lefti {\ocircle}{\sf robot})
  \fuse {\downarrow}({\sf robot}
\lefti {\ocircle}{\sf robot})$.
\end{enumerate}
\smallskip 
%
The problem with encoding traces as complete proofs, then, is that
values cannot be forced to 
precisely capture the structure of contexts, especially when there are
no variables or persistent propositions. Cervesato and Scedrov
approach this problem by severely restricting the logic and changing
the interpretation of the existential quantifier so that it acts like
a nominal quantifier on the right \cite{cervesato09relating}. The
introduction of traces allows us to avoid similar restrictions in
\sls.

Despite traces being proper syntactic objects, they are not
first-class concepts in the theory: they are derived from focused
\ollll~terms and interpreted as partial proofs. Because hereditary
substitution, identity expansion, and focalization are only defined on
complete \ollll~proofs, these theorems and operations only apply by
analogy to the deductive fragment of \sls; they do not apply to
traces.  In joint work with Deng and Cervesato, we considered a
presentation of logic that treats process states and traces as
first-class concepts and reformulates the usual properties of cut and
identity in terms of coinductive simulation relations on process
states \cite{deng12relating}. We hope that this work will eventually
lead to a better understanding of traces, but the gap remains quite
large.

% In \cite{deng12relating}, we presented the {\it logical preorder} as a
% relation $\Delta_1 \preceq \Delta_2$ between propositional process states
% that holds whenever, for all $\Theta$ and $U$, we have that
% $\tackon{\Theta}{\Delta_1} \vdash U$ implies $\tackon{\Theta}{\Delta_2}
% \vdash U$. An elegant property, {\it harmony}, relates the logical 
% preorder to cut admissibility and identity expansion. 


% \subsection{A logic of traces}

% Traces in \sls~are syntactic objects. They are not, however,
% first-class objects in the theory: they are derived from focused
% \ollll~terms and explained as partial proofs. Because hereditary
% substitution, identity expansion, and focalization are only defined on
% complete \ollll~proofs, these theorems and operations only apply by
% analogy to the deductive fragment \sls; they do not apply to traces.

% In \cite{deng12relating}, we presented the {\it logical preorder} as a
% relation $\Delta_1 \preceq \Delta_2$ between propositional process states
% that holds whenever, for all $\Theta$ and $U$, we have that
% $\tackon{\Theta}{\Delta_1} \vdash U$ implies $\tackon{\Theta}{\Delta_2}
% \vdash U$. An elegant property, {\it harmony}, relates the logical 
% preorder to cut admissibility and identity expansion. 

\subsection{LF as a term language}
\label{sec:why-not-fully-dependent}

The decision to use LF as a first-order domain of quantification
rather than using a fully-dependent system is based on several
considerations. First and foremost, this choice was sufficient for the
purposes of this thesis. In fact, for the purposes of this thesis, we
could have used an even simpler term language of simply-typed LF
\cite{pfenning08church}. Two other logic programming interpretations
of \sls-like frameworks, Lollimon \cite{lopez05monadic} and Ollibot
\cite{pfenning09substructural}, are in fact based on simply-typed term
languages. Canonical LF and Spine Form LF are, at this point,
sufficiently well understood that the additional overhead of fully
dependently-typed terms is not a significant burden, and there are
many examples beyond the scope of this thesis where dependent types are
useful.

On a theoretical level, it is a significant simplification when we
restrict ourselves to {\it any} typed term language with a reasonable
notion of equality and simultaneous substitution. The conceptual
priority in this chapter is clear: Section~\ref{sec:sls-termlanguage}
describes object terms, Section~\ref{sec:slsframework} describes proof
terms as a fragment of focused \ollll, and
Section~\ref{sec:framework-concurrenteq} describes a coarser
equivalence on proof terms, concurrent equality. If the domain of
first-order of quantification was \sls~terms, these three
considerations would be mutually dependent -- we would need to
characterize concurrent equality before presenting the logic
itself. For the purposes of showing that a logical framework can be
carved out from a focused logic -- the central thesis of this and the
previous two chapters -- it is easiest to break this circular
dependency. We conjecture that this complication is no great obstacle,
but this thesis avoids the issue.

On a practical level, there are advantages to using a well-understood
term language. The \sls~prototype implementation
(Section~\ref{sec:prototype}) uses the mature type reconstruction
engine of Twelf to reconstruct LF terms. Schack-Nielsen's
implementation of type reconstruction for Celf is complicated by the
requirements of dealing with type reconstruction for a substructural
term language, a consideration that is orthogonal to this
thesis \cite{schacknielsen08celf}. 

Finally, it is not clear that the addition of full CLF-like dependency
comes with great expressive benefit. 
% Even in LF and Twelf, many
% interesting specifications could be encoded in a two-level version of
% the language: a simply-typed object term language and a
% dependently-typed proof term language with first-order quantification
% over object terms. This restriction is sufficient for settings such as
% Harper's comprehensive survey of programming language design
% \cite{harper12practical},\footnote{Harper's metatheory also extends LF
%   by drawing a distinction between standard variables and nominal
%   parameters, but this is an orthogonal point.} and it is built in to
% the educational proof assistant SASyLF \cite{aldrich08sasylf}. 
In LF and Twelf, the ability to use full dependent types is critical
in part because it allows us to express {\it metatheorems} -- theorems
about the programming languages and logics we have encoded, like
progress and preservation for a programming language or cut
admissibility for a logic. Substructural logical frameworks like LLF
and CLF, in contrast, have not been successful in capturing
metatheorems with dependent types. Instead, metatheorems about
substructural logics have thus far generally been performed in logical
frameworks based on persistent logics. Crary proved theorems about linear
logics and languages in LF using the technique of explicit contexts
\cite{crary10higher}. Reed was able to prove cut admissibility for
linear logic and preservation for the LLF encoding of Mini-ML in HLF,
a persistent extension to LF that uses an equational theory to capture
the structure of substructural contexts \cite{reed09hybrid}.

% But in substructural logical frameworks like Linear LF, full
% dependency has been found to be {\it insufficient} for expressing
% metatheorems, which motivated the development of Hybrid LF as a
% framework for writing metatheorems about LF \cite{reed09hybrid}. The
% implementation of Hybrid LF effectively creates a stratification like
% \sls's -- full LF as an object term language, a linear logical
% framework with first-order quantification over object language terms,
% and a hybrid language that can inspect both LF object terms and linear
% proof terms.



% \subsection{Variations on concurrent equality}


% The interactions between unification and concurrent equality are
% delicate, and we do not claim that the answers we give here are
% final. In Section~\ref{sec:independency}, we motivated both the
% independency requirement that $\emptyset = {\bullet}S_1 \cap
% {\ast}S_2$ and the requirement that $\emptyset = {\ast}S_1 \cap
% {\bullet}S_2$ by giving ill-typed counterexamples. The violation of
% either condition does not, in general, imply that $S_2; S_1$ will be
% ill-typed, however. This indicates that it might be possible to give a
% more precise condition that admits a coarser notion of concurrent
% equality.

% For both
% conditions, however, there are steps $S_1$ and $S_2$ where the
% $S_1; S_2$ and $S_2; S_1$ are both well-tyled traces even though
% one of these independency requirements is not satisfied.


% coincide with the equivalence  \sls~

% ${a^+} \simplearrow (a^+ \simplearrow {\uparrow}b^+) \simplearrow 
%   ({\downarrow}{\uparrow}b^+ \simplearrow c^-) \simplearrow c^-$. 

% It is not obvious that our treatment of the interaction between 
% unification and concurrent equivalence is the right one. 

% \subsection{Concurrent equality and multifocusing}

% Concurrent equality is related to the equivalence relation induced by
% {\it multifocusing} \cite{chaudhuri08canonical}. Multifocusing is a
% concept, 

% One reason multifocusing is 

%  that has only been carefully explored in classical linear
% logic; the central change is that the rules which begins a focusing
% phase (in our presentation of MELL there were three: ${\it focus_L}$,
% ${\it focus_R}$, and ${\it copy}$) are allowed to simultaneously pull
% other propositions into focus.  As an illustration, if we reuse our
% notation from Section~\ref{sec:linnote} we can present the following
% plausible candidates for the multifocus rules in an intuitionistic
% system:
% \[
% \infer[{\it focus}_L]
% {\mildseq{\Gamma}{\Delta / A_1^-, \ldots, A_n^- }{U}}
% {n > 1
%  &
%  \mildseq{\Gamma}{\Delta, [A_1^-], \ldots, [A_n^-]}{U}}
% \quad
% \infer[{\it focus}_R]
% {\mildseq{\Gamma}{\Delta / A_1^-, \ldots, A_n^-}{C^+}}
% {n \geq 1
%  &
%  \mildseq{\Gamma}{\Delta, [A_1^-], \ldots, [A_n^-]}{[C^+]}}
% \]
% Multifocusing, however,
% appears to provide an even coarser notion of equivalence on focused
% proofs than concurrent equality does. In particular, the two
% distinct focusing proofs below are not concurrently equal: the proof
% on the right succeeds at proving $\langle c^- \rangle$ in one step,
% but leaves a subgoal in which $b^+$ is proved indirectly, whereas the
% proof at the right first transitions from having $\langle a^+ \rangle$
% and $a^+ \lolli {\uparrow} b^+$ resources to having a $\langle b^+
% \rangle$ resource, and only then proves $\langle c^- \rangle$, leaving
% a subgoal in which $b^+$ is proved directly.
% \[
% \infer
% {\mildseq{\cdot}
%   {~~
%    \langle a^+ \rangle, ~
%    a^+ \lolli {\uparrow}b^+, ~
%    {\downarrow}{\uparrow}b^+ \lolli c^-
%    ~~}
%   {~~\langle c^- \rangle}}
% {\infer
% {\mildseq{\cdot}
%   {~~
%    \langle a^+ \rangle, ~
%    a^+ \lolli {\uparrow}b^+
%    ~~}
%   {b^+}}
% {\infer
% {\mildseq{\cdot}
%   {~~
%    \langle b^+ \rangle
%    ~~}
%   {b^+}}
% {}}}
% \deduce{\mathstrut}
% {\deduce{\mathstrut}
% {\mbox{\it vs.}\mathstrut}}
% \infer
% {\mildseq{\cdot}
%   {~~
%    \langle a^+ \rangle, ~
%    a^+ \lolli {\uparrow}b^+, ~
%    {\downarrow}{\uparrow}b^+ \lolli c^-
%    ~~}
%   {~~\langle c^- \rangle}}
% {\infer
% {\mildseq{\cdot}
%   {~~
%    \langle b^+ \rangle, ~
%    {\downarrow}{\uparrow}b^+ \lolli c^-
%    ~~}
%   {~~\langle c^- \rangle}}
% {\infer
% {\mildseq{\cdot}
%   {~~
%    \langle b^+ \rangle
%    ~~}
%   {~~b^+}}
% {}}}
% \]
% Despite the lack of a full account of intuitionistic multifocusing, we
% can observe that the analogue of this sequent in classical linear
% logic has only one multifocused proof, and it is reasonable to
% conjecture that an account of multifocusing for intuitionistic logic
% would also relate these proofs. In classical linear logic,
% multifocusing offers a very fundamental normal form: any two proofs
% that can be made equal by locally permuting inference rules have the
% same multifocused proof.

% CLF's restricted form of concurrent equality will be sufficient for
% the logical framework in Chapter~\ref{chapter-framework}. 
% In fact, for the fragment of the
% the logic in Chapter~\ref{chapter-order} 
% that comprises our logical framework in Chapter~\ref{chapter-framework},
% I conjecture that concurrent equality and the equality given by
% multifocusing coincide.\footnote{This obviously means that the example
%   above will be outside the logical fragment that comprises the logical
%   framework.}  This conjecture is obviously difficult to make precise,
% much less prove, without a general theory of multifocusing in
% intuitionistic logic.


% \subsection{A warning about normalization}
% \label{sec:warning}

% In our earlier discussion of hereditary substitution and canonical
% forms in Section~\ref{sec:linlogicalframeworks}, we mentioned that the
% normalization theorem provided by hereditary substitution was weaker
% than the so-called weak normalization theorem for LF. That is because
% the weak normalization theorem says that any well-typed term can be
% converted into a canonical ($\beta$-normal and $\eta$-long) term by a
% particular series of $\beta$ and $\eta$ conversions. It is
% self-evident, by this statement of the theorem, that the resulting
% canonical term is equivalent to the original term.

% On the other hand, when we use hereditary substitution in the obvious
% way to obtain a Canonical LF term from an arbitrary non-canonical LF
% term, we gain {\it no guarantees} about the relationship between the
% non-canonical LF term and the Canonical LF term. The statement of the
% theorem does not preclude taking a $\beta$-normal, $\eta$-long LF term
% (like $\lambda x. \lambda y. x$ of type $p \rightarrow p \rightarrow
% p$ for some atomic type $p$) into a structurally different Canonical
% LF term (like $\lambda x. \lambda y. y$, which also has type $p
% \rightarrow p \rightarrow p$). It is possible to gain such a guarantee
% for LF, as Martens and Crary have shown in unpublished work
% \cite{martens11mechanizing}, but this result is a non-trivial statement
% about the constructive content of the normalization theorem. 

% In our setting, we should be concerned that we might take a focused
% proof, turn it into an unfocused proof by the obvious de-focalization
% procedure (the constructive content of
% Theorem~\ref{thm:linfocsound}), and then turn it back into a focused
% proof by focalization (the constructive content of
% Theorem~\ref{thm:linfoccomplete}) only to obtain a proof that was not
% identical or even related. This is not at all a merely hypothetical
% concern. We can run the mechanized structural focalization result from
% \cite{simmons11structural} on a persistent proposition,
% %
%    $a^+ \simplearrow 
%    {\downarrow}(a^+ \simplearrow {\uparrow}b^+) \simplearrow
%    {\downarrow}({\downarrow}{\uparrow}b^+ \simplearrow c^-) \simplearrow
%    c^-$, 
% %
% which is similar to the example from
% Section~\ref{sec:linconcurrenteq}.  In persistent logic (as in
% linear logic) that proposition has two focused propositions that
% are probably multifocusing equivalent (given a reasonable intuitionistic
% notion of multifocusing) but that are not concurrently equivalent
% under the proposed definition of concurrent equality. 
% However, if we take the focused proof that focuses 
% first on $a^+ \simplearrow {\uparrow}b^+$, transform it into an unfocused 
% proof, and then re-focus it, we will get the proof that focuses 
% first on ${\downarrow}{\uparrow}b^+ \simplearrow c^-$. Focalization,
% in other words, is not a partial inverse of de-focalization in the structural
% focalization development, except maybe modulo the (as yet undefined)
% equivalence relation established by multifocusing. 

% This example illustrates why we must be careful, but it is not a fatal
% flaw for two reasons. The first reason is the aforementioned
% conjecture that, for the restricted logical fragment defined in
% Chapter~\ref{chapter-framework} 
% as the basis of our logical framework, the focalizations of
% two proofs are concurrently equal if and only if the original proofs
% are convertible by local permutations of rules, the same condition
% that multifocusing satisfies. If this conjecture holds, it ought to be
% the case that, modulo this coarser equivalence, focalization {\it is}
% a partial inverse of de-focalization. Second, what is really at stake
% here is our ability to write down non-normal proofs in a logical
% framework that then normalizes them -- which is what the Twelf
% implementation of LF and the Celf implementation of CLF do -- with the
% confidence that we can look at a non-normal proof and know its
% corresponding canonical form. In this thesis, we will be content to
% work throughout with focused proofs and their analogues, so we can
% afford to leave questions about convertability and weak normalization
% to future work.



\part{Logical correspondence and abstraction}

% On logical correspondence
\chapter{On logical correspondence}

In Part 1, we defined \sls, the logical framework of substructural
logical specifications.

\section{Logical transformation: compilation}

\subsection{Tail-recursion}

\subsection{Parallelism}

\section{Logical transformation: defunctionalization}

\section{Logical transformation: factoring}

Example: exceptions

\section{Exploring the richer fragment}

\subsection{Mutable storage}
\label{sec:mutable-storage}

No check for pointer inequality! This is a fundamental restriction of
the fact that we're using existential quantificaiton rather than some
form of nominal quantification. (Hack due to Favonia and Bob, personal
communication.)

\subsection{Call-by-need}

\subsection{Environment semantics}

\subsection{Looking back at natural semantics}
\label{sec:enriching-natsem}

\section{Partial transformation}

\subsection{Evaluation contexts}

\subsection{Temporal logic}

The natural semantics of \rowan~are not, on a superficial level,
significantly more complex than other natural semantics. However, it
turns out that the usual set of techniques for adding state to a
natural semantics break down, and discussing a \rowan-like logic with
state remained a challenge for many years.\robnote{Figure out from
  Rowan what the recent work he told you about was.} Through the
logical correspondance, it is easy to see why: the natural SSOS
specification of \rowan~integrates both concurrent and deductive
reasoning in an arbitrarily nested way. In fact, Figure XXX is the
only SLS specification in this thesis that exhibits this form of
recursive dependency between concurrent and deductive reasoning.  In
particular, the \rowan~specification is way out of the image of the
extended natural semantics we considered in
Section~\ref{sec:enriching-natsem}. The natural encoding in state lies
in the ambient substructural context of a concurrent computation, but
that ambient computation cannot properly enter into a deductive
sub-computation. If we tried to add state to \rowan~the same way we
added it in Section~\ref{sec:mutable-storage}, the entire store
would effectively leave scope whenever computation considered
the subterm $e$ of ${\sf next}(e)$. That consideration happens
as deductive reasoning, not as concurrent reasoning!

 it is the only we
will consider in this thesis that has with property.

It's hard to include state in temporal logic! But the logical correspondence
helps us understand why: the natural SSOS specification of 

% Ordered abstract machines
\chapter{Ordered abstract machines}
\label{chapter-absmachine}

This chapter centers around two transformations on logical
specifications.  Taken together, the operationalization transformation
(Section~\ref{sec:operationalization}), and the defunctionalization
transformation (Section~\ref{sec:defunctionalization}) allow us to
establish the logical correspondence between the deductive SLS
specification of a natural semantics and the concurrent SLS
specification of an abstract machine.

Natural semantics specifications are common in the literature,
and are also easy to encode in either the deductive fragment of
\sls~or in a purely deductive logical framework like LF.  We will
continue to use the natural semantics specification of call-by-value
evaluation for the lambda calculus as our running example:
\[
\infer[{\sf ev/lam}]
{\lambda x. e \Downarrow \lambda x. e \mathstrut}
{}
\quad
\infer[{\sf ev/app}]
{e_1\,e_2 \Downarrow v \mathstrut}
{e_1 \Downarrow \lambda x.e
 &
 e_2 \Downarrow v_2
 &
 [v_2/x]e \Downarrow v \mathstrut}
\]

Abstract machine semantics are less prevalent than natural
semantics. The most well-known is almost certainly Landin's SECD
machine \cite{landin64mechanical}, though the abstract machine is much
more similar to Danvy's SC machine from \cite{danvy03rational} and
Harper's $\mathcal K\{{\sf nat}{\rightharpoonup}\}$ system from
\cite[Chapter 27]{harper12practical}.  The abstract machine semantics
that we will show to be has two states $s$. The state $s = k \rhd e$
represents the expression $e$ being evaluated on top of the stack $k$,
and the state $s = k \lhd v$ represents the value $v$ being returned
to the stack $k$. Stacks $k$ are sequences of frames $f$
with the form $((\ldots({\sf halt}; f_1); \ldots); f_n)$, and each
frame $f$ either has the form $\Box\,e_2$ (an application frame
waiting for an evaluated function to be returned to it) or the form
$(\lambda x.e)\,\Box$ (an application frame with an evaluated function
waiting for an evaluated value to be returned to it). Given states,
stacks, and frames, we can define a ``classical'' abstract machine for
call-by-value evaluation of the lambda calculus as a transition system
with four transition rules:
\begin{align*}
{\sf absmachine/lam}{:} & ~~ k \rhd \lambda x.e ~ \mapsto ~ k \lhd \lambda x.e
\\
{\sf absmachine/app}{:} & ~~ k \rhd e_1\,e_2 ~ \mapsto ~ (k; \Box\,e_2) \rhd e_1
\\
{\sf absmachine/app1}{:} & ~~ 
  (k; \Box\,e_2) \lhd \lambda x.e ~ \mapsto ~ (k; (\lambda x.e)\,\Box) \rhd e_2
\\
{\sf absmachine/app2}{:} & ~~
  (k; (\lambda x.e)\,\Box) \lhd v_2 ~ \mapsto ~ k \rhd [v_2/x]e
\end{align*}

The operational intuition for these rules is precisely the same as the
operational intuition for the rewriting rules given in
Section~\ref{sec:intro-ssos}. This is not coincidental: the
\sls~specification from the introduction adequately encodes the
transition system $s \mapsto s'$ defined above, a point that we will
make precise in Section~\ref{sec:nat-ssos-adequacy}. The
\sls~specification from the introduction is {\it also} the result of
applying the operationalization and defunctionalization
transformations to the \sls~encoding of the natural semantics given
above, so the these two transformations combined with the adequacy
arguments at either end constitute a logical correspondence between
natural semantics and abstract machines. 

As discussed in Section~\ref{sec:the-point-is-modular-extension}, it
is interesting to put existing specification styles into logical
correspondence, but that is not our main reason for being interested
in the logical correspondence. Instead, we are primarily interested in
exploring the set of programming language features that can be
modularly integrated into a transformed \sls~specification that could
not be integrated into a natural semantics specification.  In
Section~\ref{sec:richer-ordered-abstract} we explore a selection of
these features, including mutable storage, call-by-need evaluation,
and recoverable failure.

\section{Logical transformation: operationalization}
\label{sec:operationalization}

The intuition behind operationalization is rather simple: we examine
the behavior of a deductive computation and then encode that
operational intuition as a concurrent computation.  Before presenting
the general transformation, we will motivate this transformation using
our natural semantics specification of call-by-value-evaluation. 

The definition of $e \Downarrow v$ is moded with $e$ as an input and
$v$ as an output, so it is meaningful to talk about being given a
particular expression $e$ and using deductive computation to search
for a $v$ such that $e \Downarrow v$ is derivable.  Consider a
recursive search procedure implementing this particular deductive
computation:
\begin{itemize}
\item
      If $e = \lambda x. e'$, 
      it is possible to derive 
      $\lambda x. e' \Downarrow \lambda x. e'$
      with the rule ${\sf ev/lam}$.
\item
       If $e = e_1\,e_2$,
       attempt to derive 
       $e_1\,e_2 \Downarrow v$
       using the rule ${\sf ev/app}$ by doing the following:
    \begin{enumerate}
    \item Search for a $v_1$ such that 
          $e_1 \Downarrow v_1$ is derivable.
    \item Assert that $v_1 = \lambda x.e'$ for some
          $e'$; fail if it is not.
    \item Search for a $v_2$ such that 
          $e_2 \Downarrow v_2$ is derivable.
    \item Search for a $v$ such that 
          $[v_2/x]e \Downarrow v$ is derivable.
    \end{enumerate}
% \item
%       If $e = e_1 \arb e_2$,
%       attempt to derive $e_1 \arb e_2 \Downarrow v$ using
%       the rule ${\sf ev/choose1}$ by searching for a 
%       $v$ such that $e_1 \Downarrow v$ is derivable.
% \item
%       If $e = e_1 \arb e_2$,
%       attempt to derive $e_1 \arb e_2 \Downarrow v$ using 
%       the rule ${\sf ev/choose2}$ by searching for a 
%       $v$ such that $e_2 \Downarrow v$ is derivable.  
\end{itemize}
%
The goal of the operationalization transformation is to implement this
deductive computation as a concurrent computation. The first step in
doing so is to introduce two new ordered atomic propositions.  The
proposition ${\sf eval}\,\interp{e}$ is the starting point, indicating
that we want to search for a $v$ such that $e \Downarrow v$, and the
proposition ${\sf retn}\,\interp{v}$ indicates the successful
completion of this procedure. Therefore, searching for a $v$ such that
$e \Downarrow v$ is derivable will be analogous to building a trace $T
:: x_e{:}\susp{{\sf eval}\,\interp{e}} \leadsto^* x_v{:}\susp{{\sf
    retn}\,\interp{e}}$ with concurrent computation.

Representing the first case is straightforward: if we are evaluating
$\lambda x.e$, then we have succeeded and can return $\lambda x.e$. 
This is encoded in the rule ${\sf ev/lam}$. 
\[
{\sf ev/lam} : {\sf eval}\,({\sf lam}\,\lambda x.\,E\,x)
   \lefti \{ {\sf retn}\,({\sf lam}\,\lambda x.\,E\,x) \}
\]
Because the second rule involves both recursion and multiple subgoals,
we will generalize our picture of the process state to allow us to store a
stack of unfinished work in the ordered context, growing out to the
right. Our new understanding, then, is that contexts either have the
form $x{:}\susp{{\sf eval}\,\interp{e}}, \Delta$ or the form $x{:}\susp{{\sf
  retn}\,\interp{v}}, \Delta$. In the process of concurrently
computing a trace $x_e{:}\susp{{\sf eval}\,\interp{e_1\,e_2}}, \Delta
\leadsto^* x_r{:}\susp{{\sf retn}\,\interp{v}}, \Delta$, each of the
recursive calls to the search procedure will involve a sub-trace of the
form
%
\[x_e{:}\susp{{\sf eval}\,\interp{e'}}, y{:}\istrue{A^-}, \Delta
  \leadsto^*
  x_r{:}\susp{{\sf retn}\,\interp{v'}}, y{:}\istrue{A^-}, \Delta\]
%
where $A^-$ is a negative proposition that is prepared to interact
with the final ${\sf retn}\,\interp{v'}$ proposition to kickstart the
rest of the computation.

It's helpful to work backwards: in the fourth step, we have found
$E\,x = \interp{e}$ (where $e$ potentially has $x$ free) and $V_2 =
\interp{v_2}$, and the recursive call is to ${\sf
  eval}\,\interp{[v_2/x]e}$, which is the same thing as ${\sf
  eval}\,(E\,V_2)$. If the recursive call successfully returns, the
context will contain a suspended atomic proposition of the form ${\sf
  retn}\,V$ where $V = \interp{v}$, and the search procedure as a
whole is complete: the answer is $v$.  Thus, the negative proposition
that implements the continuation can be written as $(\forall V. {\sf
  retn}\,V \lefti \{ {\sf retn}\,V \})$. The positive proposition that
will create this sub-computation can be written as follows:
\begin{align*}
{\it Step_4}(E,V_2) & \equiv {\sf eval}\,(E\,V_2) 
\fuse {\downarrow}(\forall V.\, {\sf retn}\,V \lefti \{ {\sf retn}\,V \})
%
\intertext{Moving backwards, in the third step we have a $E_2 =
  \interp{e_2}$ that we were given and $E\,x = \interp{e}$ that we
  have computed. The recursive call is to ${\sf
    eval}\,\interp{e_2}$, and assuming that it completes, we need
  to begin the fourth step. The positive proposition that will 
  create this sub-computation can be written as follows:}
%
{\it Step_3}(E_2,E) & \equiv {\sf eval}\,E_2 
\fuse {\downarrow}(\forall V_2.\,
  {\sf retn}\,V_2 \lefti \{ {\it Step_4}(E,V_2) \})
%
\intertext{Finally, the first two steps can be handled together. We have
$E_1 = \interp{e_1}$ and $E_2 = \interp{e_2}$; the recursive
call is to ${\sf eval}\,\interp{e_1}$. Once the
recursive call completes, we can enforce that the returned value has
the form $\interp{\lambda x.e}$ before proceeding
to the continuation.}
{\it Step_{1,2}}(E_1, E_2) & \equiv {\sf eval}\,E_1
\fuse {\downarrow}(\forall E.\, {\sf retn}\,({\sf lam}\,\lambda x.\,E\,x)
\lefti \{ {\it Step_3}(E_2, E)\})
\end{align*}
Thus, the rule implementing this entire portion of the search
procedure is 
\[
\forall E_1.\,\forall E_2.\,
{\sf eval}\,({\sf app}\,E_1\,E_2) \righti \{ {\it
  Step_{1,2}}(E_1, E_2) \}
\]
The \sls~encoding of our example natural semantics is shown in
Figure~\ref{fig:example-transform-cbv} alongside the transformed
specification, which has the form of an ordered abstract machine
semantics, though it is different than the ordered abstract machine
semantics presented in the introduction. We say the specification
above is {\it higher-order}, as ${\sf ev/app}$ is a rule that, when it
participates in a transition, produces a new rule $(\forall E.\,{\sf
  retn}\,({\sf lam}\,\lambda x.\,E\,x) \lefti \{ \ldots \})$ that
lives in the context. The ordered abstract machine semantics from the
introduction was {\it first-order}, because the head $\{ A^+ \}$ of
every concurrent rule contains only positive atomic propositions.  We
discuss the defunctionalization transformation, which allows us to
derive first-order specifications from specifications that are
higher-order in this way, in Section~\ref{sec:defunctionalization}
below.

\begin{figure}
\begin{minipage}[b]{0.36\linewidth}
\fvset{fontsize=\small,boxwidth=auto}
\VerbatimInput{sls/cbv-ev.sls}
\end{minipage}
\hspace{0.5cm}
\begin{minipage}[b]{0.64\linewidth}
\fvset{fontsize=\small,boxwidth=auto}
\VerbatimInput{sls/cbv-ev-ssos.sls}
\end{minipage}
\caption{A natural semantics for CBV (left) and the corresponding (higher-order)
  ordered abstract machine (right).}
\label{fig:example-transform-cbv}
\end{figure}

The intuitive connection between natural semantics specifications and
concurrent specifications has been explored previously and
independently by Schack-Nielsen \cite{schacknielsen07induction} and by
Cruz and Favonia \cite{cruz12parallel}; Schack-Nielsen proves the
equivalence of the two specifications, whereas Cruz and Favonia used
the connection informally. The contribution of this section is to
describe a general transformation (of which
Figure~\ref{fig:example-transform-cbv} is one instance) and to prove
the transformation correct in general. 

In Section~\ref{sec:trans-subset} we will present the subset of
specifications that our operationalization transformation handles, and
in Section~\ref{sec:trans-basic} we present the most basic form of the
transformation.  In
Sections~\ref{sec:trans-tail}~and~\ref{sec:trans-par} we extend the
basic transformation to be both tail-recursion optimizing and
parallelism-enabling. Finally, in
Section~\ref{sec:operationalization-correct}, we establish the
correctness of the overall transformation.

\subsection{Transformable signatures}
\label{sec:trans-subset}

The starting point for the operationalization transformation is a
deductive signature that is well-moded in the sense described in
Section~\ref{sec:framework-modes}. Every declared negative predicate
will either remain defined by deductive proofs (we write the atomic
propositions built with these predicates as $p_d^-$, $d$ for
deductive) or will be transformed so that it is concurrently defined
(we write the atomic propositions built with these predicates as
$p_c^-$, $c$ for concurrent).

For the purposes of describing and proving the correctness of the
operationalization transformation, we will assume that all transformed
atomic propositions $p_d^-$ have two arguments where the first
argument is moded as an input and the second is an output. That is,
their predicates are declared as follows:
\begin{align*}
& {\sf a} : \tau_1 \rightarrow \tau_2 \rightarrow {\sf prop}.\\
& {\sf \#mode~a~{+}~{-}}.
\end{align*}
Without dependency, two-place relations are sufficient for describing
$n$-place relations.\footnote{As an example, to handle addition on
  natural numbers, defined as a three-place relation ${\sf add} : {\sf
    nat} \rightarrow {\sf nat} \rightarrow {\sf nat} \rightarrow {\sf
    type}$ with its usual mode (${\sf add}~{+}~{+}~{-}$), we define a
  unique type ${\sf add\_in}$ with one binary constructor ${\sf
    add\_c} : {\sf nat} \rightarrow {\sf nat} \rightarrow {\sf
    add\_in}$. Then we can declare (${\sf add'} : {\sf add\_in}
  \rightarrow {\sf nat} \rightarrow {\sf type}$) with mode (${\sf
    add'}~{+}~{-}$).}  It should be possible to handle dependent
predicates (that is, those with declarations of the form ${\sf a} :
\Pi x{:}\tau_1.\,\tau_2(x) \rightarrow {\sf type}$), but we will not do
so here.

The restriction on signatures furthermore enforces that all rules must
be of the form ${\sf r} : C$ or ${\sf r} : D$, where $C$ and $E$ are
refinements of the negative propositions of \sls~that are defined as
follows:
\begin{align*}
C & ::= p^-_{c} 
    \mid \forall x{:}\tau.\, C
    \mid p^+_\mpers \lefti C
    \mid {!}p^-_c \lefti C
    \mid {!}G \lefti C \\
D & ::= p^-_{d}
    \mid \forall x{:}\tau.\, D
    \mid p^+_\mpers \lefti D
    \mid {!}p^-_c \lefti D
    \mid {!}G \lefti C \\
G & ::= p^-_d 
    \mid \forall x{:}\tau.\, G
    \mid p^+_\mpers \lefti G
    \mid {!}D \lefti G
\end{align*}
If {\it all} propositions are to remain deductive, then the
propositions $p^-_c$ and $C$ are irrelevant, and this restriction
describes all persistent, deductive specifications -- essentially, any
signature that could be executed by the standard logic programming
interpretation of LF \cite{pfenning98elf}. On the other hand, if all
propositions are to be transformed, then the propositions $p^-_d$ and
$D$ are irrelevant and this restriction amounts to restricting
rules to the Horn fragment.

All propositions $C$ are equivalent (at the level of synthetic
inference rules) to propositions of the form $\forall
\overline{x_0}\ldots \forall \overline{x_n}.\,A^+_n \lefti \ldots
\lefti A^+_1 \lefti {\sf a}\,t_{0}\,t_{n-1}$, where the $\forall
\overline{x_i}$ are shorthand for a series of universal quantifiers
$\forall {x_{i1}}{:}{\tau_{i1}} \ldots \forall {x_{\it
    ik}}{:}{\tau_{\it ik}}$ and where each variable in
$\overline{x_i}$ does not appear in $t_0$ (unless $i = 0$) nor in any
$A^+_j$ with $j < i$ but does appear in $A^+_i$ (or $t_0$ if $i =
0$). Therefore, when we consider moded proof search, the variables
bound in $\underline{x_0}$ are all fixed by the query and those bound
in the other $\underline{x_i}$ are all fixed by the output position of
the $i^{\rm th}$ premise.

\subsection{Basic transformation}
\label{sec:trans-basic}

The operationalization transformation $\transop{\Sigma}$
operates on SLS signatures $\Sigma$ that have the form described in the
previous section. We
will first give the transformation on signatures; the transformation
of rule declarations ${\sf r} : C$ is the key case.

Each two-place predicate ${\sf a}$ that we plan to operationalize gets
turned into two one-place predicates ${\sf eval\_a}$ and ${\sf
  retn\_a}$.  We will write $\opsubst{X}$ for the operation of
substituting all occurrences of $p^-_c = {\sf a}\,t_1\,t_2$ with
$({\sf eval\_a}\,t_1 \lefti \{ {\sf retn\_a}\,t_2 \})$ in $X$. This
substitution operation is used on propositions, contexts, and frames;
it appears in the transformation of rules ${\sf r} : D$ below.

\begin{itemize}
\item $\transop{\cdot} = \cdot$
\item $\transop{\Sigma, {\sf a} : \tau_1 \rightarrow \tau_2
    \rightarrow {\sf prop}} = \transop{\Sigma}, ~ {\sf eval\_a} :
  \tau_1 \rightarrow {\sf prop\,ord}, ~ {\sf retn\_a} : \tau_2
  \rightarrow {\sf prop\,ord}$ \\ {\it (if $\sf a$ is one of the
    predicates that we are translating)}
\item $\transop{\Sigma, {\sf a} : K} = \transop{\Sigma}, ~ {\sf a}
  : K$ {\it (otherwise)}
\item $\transop{\Sigma, {\sf c} : \tau} = \transop{\Sigma}, ~ {\sf
    c} : \tau$ 
\item $\transop{\Sigma, {\sf r} : C} = \transop{\Sigma}, ~ {\sf r}
  : \forall \overline{x_0}.\, {\sf eval\_a}\,t_0 \lefti \llbracket A^+_1,
  \ldots, A^+_n \rrbracket (t_{n+1}, {\sf id})$ \\ {\it (where $C$ is
    equivalent to $\forall \overline{x_0}\ldots \forall
    \overline{x_n}.\, A^+_n \lefti \ldots \lefti A^+_1 \lefti {\sf
      a}\,t_{0}\,t_{n+1}$)}
\item $\transop{\Sigma, {\sf r} : D} = \transop{\Sigma}, ~ {\sf r}
  : \opsubst{D}$
\end{itemize}

The transformation of a proposition $C$ of the form $\forall
\overline{x_0}\ldots \forall \overline{x_n}.\,A^+_n \lefti \ldots
\lefti A^+_1 \lefti {\sf a}\,t_{0}\,t_{n+1}$ involves the definition
$\opbasic{A^+_i,\ldots,A^+_n}{t_{n+1}}{\sigma}$, where $\sigma$
substitutes only for variables in $\overline{x_j}$ where $j < i$. The
function is defined inductively on the length of the sequence
$A^+_i,\ldots,A^+_n$.

\begin{itemize}
\item $\opbasic{}{t_{n+1}}{\sigma} = \{ {\sf retn\_a}\,(\sigma{t_{n+1}}) \}$
\item $\opbasic{p^+_\mpers,A^+_{i+1},\ldots,A^+_n}{t_{n+1}}{\sigma} 
  = \forall \overline{x_i}.\, (\sigma{p^+_\mpers}) \lefti \opbasic{A^+_{i+1},\ldots,A^+_n}{t_{n+1}}{\sigma}$
\item $\opbasic{{!}p^-_c,A^+_{i+1},\ldots,A^+_n}{t_{n+1}}{\sigma}$
  \\
  $~ \qquad = \{ {\sf eval\_b}\,({\sigma}t^{\it in}_i) \fuse
  (\forall\overline{x_i}.\, {\sf retn\_b}\,(\sigma{t^{\it out}_i})
  \lefti \opbasic{A^+_{i+1},\ldots,A^+_n}{t_{n+1}}{\sigma}) \}$\\
  {\it (where $p^-_c$ is ${\sf b}\,t^{\it in}_i\,t^{\it out}_i$)}
\item $\opbasic{{!}G,A^+_{i+1},\ldots,A^+_n}{t_{n+1}}{\sigma} = \forall
  \overline{x_i}.\, {!}(\sigma\opsubst{G}) \lefti
  \opbasic{A^+_{i+1},\ldots,A^+_n}{t_{n+1}}{\sigma}$
\end{itemize}

\noindent
This operation is slightly more general than it needs to be to
describe the transformation on signatures, because the substitution
$\sigma$ will always just be the identity substitution ${\sf id}$.
Non-identity substitutions arise during the proof of correctness, which
is why we introduced them here.

We have already given an example of the this basic operationalization
transformation, as Figure~\ref{fig:example-transform-cbv} is an
instance of this transformation.

\subsection{Tail-recursion}
\label{sec:trans-tail}

Consider again our motivating example, the procedure for that takes
expressions $e$ and searches for expressions $v$ such that $e
\Downarrow v$ is derivable. If we were to implement that procedure as
a functional program, the procedure would be {\it tail-recursive}. In
the procedure that handles the case when $e = e_1\,e_2$, the last step
is that the search procedure is invoked recursively. If and when that
callee returns $v$, then the caller will also return $v$.

Tail-recursion is significant in functional programming because
tail-recursive calls can be implemented without allocating a stack
frame: when a compiler makes this more efficient choice, we say it is
performing {\it tail-recursion optimization}.\footnote{Or {\it tail-call optimization}, as a tail-recursive function call is just a
  specific instance of a tail call.} An analogous opportunity for
tail-recursion optimization also arises in our logical compilation
procedure. In our motivating example, the last step in the $e_1\,e_2$
case was operationlized as a positive proposition of the form ${\sf
  eval}\,(E\, V) \fuse (\forall v.\,{\sf retn}\,v \lefti \{ {\sf
  retn}\,v \})$. In a successful search, the process state 
\[ x{:}{\sf
  eval}\,(E\, V), y{:}\istrue{(\forall v.\,{\sf retn}\,v \lefti \{
  {\sf retn}\,v \})}, \Delta\]
will concurrently compute until the
state 
\[ x'{:}{\sf retn}\,V, y{:}\istrue{(\forall v.\,{\sf retn}\,v \lefti
  \{ {\sf retn}\,v \})}, \Delta\] is reached, at which point the next
step \[y'{:}{\sf retn}\,V, \Delta\] is reached in one step by focusing
on $y$. 

If we operationalize the last step in the $e_1\,e_2$ case as ${\sf
  eval}\,(E\,V)$ instead of as ${\sf eval}\,(E\, V) \fuse (\forall
v.\,{\sf retn}\,v \lefti \{ {\sf retn}\,v \})$, we will reach the same
final state with one less transition. The tail-recursion optimizing
version of the operationalization transformation creates concurrent
computations that avoid these useless steps.

We cannot perform tail recursion in general. The obvious reason for
this to be the case is when the output of the last subgoal is
different from the output of the goal. For example, the rule ${\sf r}
: \forall{x}.\,\forall{y}.\,{!}{\sf a}\,x\,y \lefti {\sf a}\,({\sf
  c}\,x)\,({\sf c}\,y)$, will translate to
\[ {\sf r} : \forall{x}.\,{\sf eval\_a}\,({\sf c}\,x) \lefti \{ {\sf
  eval\_a}\,x \fuse (\forall y.\, {\sf retn\_a}\,y \lefti \{ {\sf
  retn\_a}\,({\sf c}\,y) \} ) \} \] There is no opportunity for
tail-recursion optimization, because the output of the last search
procedure, $t^{\it out}_n = y$, is different than the value returned
down the stack, $t_{n+1} = {\sf c}\,y$. This case corresponds to
functional programs that cannot be tail-call optimized.

More subtly, we cannot even eliminate all cases where $t^{\it out}_n =
t_{n+1}$ unless these terms are {\it fully general}. We say that
$t_{n+1}$ with type $\tau$ is fully general if all of its free
variables are in $\overline{x_n}$ (and therefore not fixed by the
input of any other premise) and if, for any variable-free term $t'$ of
type $\tau$, there exists a substitution $\sigma$ such that $t =
{\sigma}t_{n+1}$. The simplest example way to do this is to force
$t_{n+1} = t^{\it out}_n = y$ where $y = \overline{x_n}$.\footnote{It
  is also possible to have a fully general $t_{n+1} = {\sf
    c}\,y_1\,y_2$ if, for instance, ${\sf c}$ has type $\tau_1
  \rightarrow \tau_2 \rightarrow {\sf foo}$ and there are no other
  constructors of type ${\sf foo}$. However, we also have to check
  that there are no other first-order variables in $\Psi$ with types
  like $\tau_3 \rightarrow {\sf foo}$ that could be used to make other
  terms of type ${\sf foo}$. The technology to handle this, worlds
  checking and subordination analysis, is well-understood and surveyed
  elsewhere \cite{harper07mechanizing}, but this is tangential to the
  current discussion.} This condition doesn't have an analogue in
functional programming, because it corresponds to the possibility that
moded deduction computation can perform pattern matching on {\it
  outputs} and fail if the pattern match fails.

The tail-recursive procedure can be described by adding a new 
case to the definition of 
$\opbasic{A^+_i,\ldots,A^+_n}{t_{n+1}}{\sigma}$:

\begin{itemize}
\item $\opbasic{{!}{\sf a}\,t^{\it in}_n\,t_{n+1}}{t_{n+1}}{\sigma} 
  = \{{\sf eval\_a}\,({\sigma}{t^{\it in}_n})\}$
\\
  {\it (where $t_{n+1}$ is fully general)}
\end{itemize}
This case overlaps with the third case of the definition given
in Section~\ref{sec:trans-basic}, which indicates that tail-recursion
optimization can be applied or not in a nondeterministic manner.

\begin{figure}
\fvset{fontsize=\small,boxwidth=auto}
\VerbatimInput{sls/cbv-ev-ssos-tail.sls}
\caption{A higher-order ordered abstract machine semantics for the CBV
  evaluation.}
\label{fig:cbv-ev-ssos-tail}
\end{figure}

\subsubsection{Example}

Operationalizing the natural semantics from
\ref{fig:example-transform-cbv} results in the ordered abstract
machine in Figure~\ref{fig:cbv-ev-ssos-tail}.  A more dramatic
illustration of tail-call optimization can be given if we consider a
big-step evaluation function that is based on a small-step structural
operational semantics (SOS) specification. In SOS specifications,
single-step evaluation is defined as the two-place relation ${\sf
  step} : {\sf exp} \rightarrow {\sf exp} \rightarrow {\sf prop}$
(moded ${\sf exp}\,{+}\,{-}$) that makes use of the helper judgment
${\sf value} : {\sf exp} \rightarrow {\sf prop}$ (moded ${\sf
  value}\,{+}$). We will not define these propositions here, but we do
so later on in Section~\ref{sec:evaluationcontexts}.

Given the definition of ${\sf step}\,\interp{e}\,\interp{e'}$, it is
easy to define big-step evaluation ${\sf ev}\,\interp{e}\,\interp{v}$
as a series of small steps:

\smallskip
\fvset{fontsize=\small,boxwidth=auto}
\VerbatimInput{sls/cbv-sos-steps.sls}
\smallskip

\begin{figure}
\begin{minipage}[b]{0.55\linewidth}
\fvset{fontsize=\small,boxwidth=auto}
\VerbatimInput{sls/cbv-sos-proc2.sls}
\end{minipage}
\hspace{0.5cm}
\begin{minipage}[b]{0.45\linewidth}
\fvset{fontsize=\small,boxwidth=auto}
\VerbatimInput{sls/cbv-sos-proc.sls}
\end{minipage}
\caption{The transformation of a trivial big-step semantics, both
  without (left) and with (right) tail-recursion optimization.}
\label{fig:sos-tailrecursion}
\end{figure}

Running this specification through the operationalization
transformation and only operationalizing the ${\sf ev}$ predicate
results in what I consider to be the most boring substructural
operational semantics specification, shown in
Figure~\ref{fig:sos-tailrecursion} both without the tail-recursion
optimization (left) and with the tail-recursion optimization (right).

\begin{figure}
\begin{align*}
& x_1{:}{\sf eval}\,\interp{e_1} \\
\leadsto ~ & x_2{:}{\sf eval}\,\interp{e_2}, 
  y_1{:}\istrue{(\forall v. {\sf retn}\,v \lefti \{ {\sf retn}\,v \})} \\
\leadsto ~ & x_3{:}{\sf eval}\,\interp{e_3}, 
  y_2{:}\istrue{(\forall v. {\sf retn}\,v \lefti \{ {\sf retn}\,v \})}, 
  y_1{:}\istrue{(\forall v. {\sf retn}\,v \lefti \{ {\sf retn}\,v \})} \\
\leadsto ~ & \cdots\\
\leadsto ~ & x_n{:}{\sf eval}\,\interp{v}, 
  y_{n-1}{:}\istrue{(\forall v. {\sf retn}\,v \lefti \{ {\sf retn}\,v \})}, 
  \cdots,
  y_1{:}\istrue{(\forall v. {\sf retn}\,v \lefti \{ {\sf retn}\,v \})} \\
\leadsto ~ & z_n{:}{\sf retn}\,\interp{v}, 
  y_{n-1}{:}\istrue{(\forall v. {\sf retn}\,v \lefti \{ {\sf retn}\,v \})}, 
  \cdots,
  y_1{:}\istrue{(\forall v. {\sf retn}\,v \lefti \{ {\sf retn}\,v \})} \\
\leadsto ~ & \cdots \\
\leadsto ~ & z_3{:}{\sf retn}\,\interp{v}, 
  y_2{:}\istrue{(\forall v. {\sf retn}\,v \lefti \{ {\sf retn}\,v \})}, 
  y_1{:}\istrue{(\forall v. {\sf retn}\,v \lefti \{ {\sf retn}\,v \})} \\
\leadsto ~ & z_2{:}{\sf retn}\,\interp{v}, 
  y_1{:}\istrue{(\forall v. {\sf retn}\,v \lefti \{ {\sf retn}\,v \})} \\
\leadsto ~ & z_1{:}{\sf retn}\,\interp{v}
\end{align*}
\caption{Example trace with the non-tail-recursion-optimized
  semantics in Figure~\ref{fig:sos-tailrecursion}}
\label{fig:example-proc-non-tail-recursive-trace}
\end{figure}

The tail-recursion optimized translation is definitely superior for
this example. Concurrent proofs for the non-tail-recursion-optimized
specification build up an enormous stack of useless copies of the
proposition $(\forall v. {\sf retn}\,v \lefti \{ {\sf retn}\,v \})$,
as shown in Figure~\ref{fig:example-proc-non-tail-recursive-trace}.
In contrast, the tail-recursion optimized version on the right hand
side of Figure~\ref{fig:sos-tailrecursion} takes half as many steps,
and each step is smaller, simpler, and the overall trace does a better
job of actually capturing the linear computation that is actually
involved in describing a big-step semantics using a small-step
structural operational semantics:
\[
x_1{:}{\sf eval}\,\interp{e_1} 
 ~\leadsto~
x_2{:}{\sf eval}\,\interp{e_2}
 ~\leadsto~
x_3{:}{\sf eval}\,\interp{e_3}
 ~\leadsto~ \cdots ~\leadsto~
x_n{:}{\sf eval}\,\interp{v}
 ~\leadsto~ 
z{:}{\sf retn}\,\interp{v}
\]

\subsection{Parallelism}
\label{sec:trans-par}

Both the basic transformation and the tail-recursive transformation
are sequential: if $x{:}{\sf eval}\,\interp{e} \leadsto^* \Delta$,
then the process state $\Delta$ contains at most one proposition ${\sf
  eval}\,\interp{e'}$ or ${\sf retn}\,\interp{v}$ that can potentially
be a part of any further transition. Put differently, the first two
operationalization transformations express deductive computation as a
concurrent computation that does not exhibit concurrency (sequential
computation being a special case of concurrent computation).

Sometimes, this is what we want: in
Section~\ref{sec:nat-ssos-adequacy} we will see that the sequential
tail-recursion-optimized abstract machine is what we want to
adequately represent the traditional on-paper abstract machines for
the call-by-value lambda calculus. In general, however, when distinct
subgoals do not have input-output dependencies (that is, when none of
subgoal $i$'s outputs are inputs to subgoal $i+1$), deductive computation
can search for subgoal $i$ and $i+1$ simultaneously, and this can 
be represented in the operationalization transformation.

In the previous transformations, our process states were structured
such that every negative proposition $A^-$ was waiting on a single
${\sf retn}$ to be computed to its left; at that point, the negative
proposition could be focused on, which effectively invokes the
continuation stored in that negative proposition. If we ignore the
first-order structure of the concurrent computation, the intermediate
states of look like this:
\[
  (\mbox{subgoal 1}), y{:}\istrue{({\sf retn} \lefti {\it cont})}
\]
Note that $(\mbox{subgoal 1})$ is intended to represent some nonempty
sequence of ordered propositions, not a single proposition. With the
parallelism-enabling transformation, subgoal 1 can even be performing
parallel search for its own subgoals:
\[
 (\mbox{subgoal 1.1}), (\mbox{subgoal 1.2}), 
   y_1{:}\istrue{({\sf retn}_{1.1} \fuse {\sf retn}_{1.2} \lefti {\it cont}_1)}, 
   y{:}\istrue{({\sf retn} \lefti {\it cont})}
\]
The two subcomputations $(\mbox{subgoal 1.1})$ and $(\mbox{subgoal
  1.2})$ are next to one another in the ordered context, but the
structure of transformed specifications ensures that the only way they
can interact is if they both finish (becoming $z_{1.1}{:}\susp{{\sf
    retn}_{1.1}}$ and $z_{1.2}{:}\susp{{\sf retn}_{1.2}}$), which will
allow us to focus on $y_1$ and begin working on the continuation ${\it
  cont}_1$. The principle at work is the same one that says that postfix
notations like Reverse Polish notation are unambiguous: there's always
only one way to reconstruct the tree of subgoals. 

To allow for the transformed programs to have parallelism, we again
add a new case to the function that transforms propositions $C$ in the
signature.  In this case, the new case will actually subsume the old
case that dealt with sequences of the form ${!}p_c^-,
A^+_{i+1},\ldots,A^+_n$; that old case is now an instance of the
general case where $i = j$. 

\begin{itemize}
\item $\opbasic{{!}p^-_{ci},\ldots,{!}p^-_{cj},A^+_{j+1},\ldots,A^+_n}{t_{n+1}}{\sigma}$
  \\
  $~ \qquad = \{ {\sf eval\_bi}\,({\sigma}t^{\it in}_i) 
                    \fuse \ldots \fuse
                 {\sf eval\_bj}\,({\sigma}t^{\it in}_j)$
  \\
  $~ \qquad \qquad (\forall\overline{x_i}\ldots\forall\overline{x_j}.\, 
     {\sf retn\_bi}\,(\sigma{t^{\it out}_i})
     \fuse \ldots \fuse 
     {\sf retn\_bj}\,(\sigma{t^{\it out}_j})$
  \\
  $~ \qquad \qquad \quad
   \lefti \opbasic{A^+_{j+1},\ldots,A^+_n}{t_{n+1}}{\sigma}) \}$\\
  {\it (where
   $p^-_{ck}$ is ${\sf bk}\,t^{\it in}_k\,t^{\it out}_k$ 
   and $FV(t_k^{\it in}) \notin (\overline{x_i} \cup \ldots \cup \overline{x_j})$ 
   for $i \leq k \leq j$)}
\end{itemize}

\noindent
Note that the second side condition on the free variables of inputs is
necessary if the resulting term is to be well-scoped, and is trivially 
satisfied in the sequential case where $i = j$. 

\begin{figure}
\fvset{fontsize=\small,boxwidth=auto}
\VerbatimInput{sls/cbv-ev-ssos-par.sls}
\caption{The parallel, tail-recursion optimized ordered abstract machine for
 call-by-value evaluation.}
\label{fig:cbv-ev-ssos-par}
\end{figure}

The result of running the natural semantics from
Figure~\ref{fig:example-transform-cbv} through the parallel and
tail-recursion optimizing ordered abstract machine is shown in
Figure~\ref{fig:cbv-ev-ssos-par}; it represents that we can
search for the subgoals $e_1 \Downarrow \lambda x.e$ and
$e_2 \Downarrow v_2$ in parallel. We cannot, of course, run either
of these subgoals in parallel with the third subgoal 
$[v_2/x]e \Downarrow v$ because the input $[v_2/x]e$ mentions the outputs
of both of the previous subgoals. 

\subsection{Correctness}
\label{sec:operationalization-correct}

The correctness of the basic, tail-recursion-optimizing, and parallel
transformations all follow from the correctness of the parallel
transformation; because the transformation is nondeterministic, the
previously presented transformations are just instances of this most
general one.

\bigskip
\begin{theorem}[No effect on the LF fragment]
  $\Psi \vdash_\Sigma t : \tau$ if and only if $\Psi
  \vdash_{\transop{\Sigma}} t : \tau$.
\end{theorem}

\begin{proof}
Straightforward induction in both directions; the transformation 
leaves the LF-relevant part of the signature unchanged.
\end{proof}

\begin{theorem}[Soundness of operationalization]
XXX Soundness
\end{theorem}

\begin{proof}
XXX Proof
\end{proof}

\begin{theorem}[Completeness of operationalization]
If all propositions in $\Gamma$ have the form 
$x{:}D$ or $z{:}\susp{p^+_{\mpers}}$, then
\begin{enumerate}
\item  
If $\slss{\Sigma}{\Psi}{\Gamma}{\susp{p^-_d}}$,
then $\slss{\transop{\Sigma}}{\Psi}{\opsubst{\Gamma}}{\susp{p^-_d}}$.
\item  
If $\slss{\Sigma}{\Psi}{\Gamma, [D]}{\susp{p^-_d}}$,
then $\slss{\transop{\Sigma}}{\Psi}{\opsubst{\Gamma}, [\opsubst{D}]}{\susp{p^-_d}}$.
\item  
If $\slss{\Sigma}{\Psi}{\Gamma}{G}$,
then $\slss{\transop{\Sigma}}{\Psi}{\opsubst{\Gamma}}{\opsubst{G}}$.
\item
If $\Delta$ matches $\frameoff{\Theta}{\Gamma}$ 
and $\slss{\Sigma}{\Psi}{\Gamma}{\susp{p^-_c}}$
(where $p^-_c = {\sf a}\,t\,s$),\\
then
$(\Psi; \tackon{\opsubst{\Theta}}{x{:}\susp{{\sf eval\_a}\,t}}) 
  \leadsto^*_{\transop{\Sigma}}
 (\Psi; \tackon{\opsubst{\Theta}}{y{:}\susp{{\sf retn\_a}\,s}})$.
\end{enumerate}
\end{theorem}

\begin{proof}
Mutual induction on the size 
of the input derivation.

The first three parts are straightforward. In part 1, we have
$\slst{\Sigma}{\Psi}{\Gamma}{\tfocusl{h}{\Sp}}{\susp{p^-_d}}$ where
either $h = x$ and $x{:}D \in \Gamma$ or else $h = {\sf r}$ and ${\sf
  c}{:}D \in \Sigma$. In either case the necessary result is
$\tfocusl{h}{\Sp'}$, where we get $\Sp'$ from the induction hypothesis
(part 2) on $\Sp$.

In part 2, we proceed by case analysis on the proposition $D$ in focus. 
The only interesting case is where $D = {!}p^-_c \lefti D'$
\begin{itemize}
\item If $D = p_d^-$, $\Sp = \tnil$ and $\tnil$ gives the desired result.

\item If $D = \forall x{:}\tau.\,D'$ or $D = p^+_{\sf
    pers} \lefti D'$, then $\Sp = (\tforalll{t}{\Sp'})$ 
  or $\Sp = (\tappl{z}{\Sp'})$ (respectively). The necessary result is
  $(\tforalll{t}{\Sp''})$ 
  or $(\tappl{z}{\Sp''})$ (respectively) where we get $\Sp''$ from the
  induction hypothesis (part 2) on $\Sp'$. 

\item If $D = {!}p^-_c \lefti D'$ and $p^-_c = {\sf a}\,t_1\,t_2$, then 
  $\Sp = (\tappl{\tbangr{\tetan{N}}}{\Sp'})$
  and $\opsubst{D} = {!}({\sf eval}\,t_1 \lefti \{ {\sf retn}\,t_2 \}) \lefti \opsubst{D'}$.

  \begin{tabbing}
  $\slst{\Sigma}{\Psi}{\Gamma}{N}{\susp{p^-_c}}$
  \` (given)
  \\
  $\slst{\Sigma}{\Psi}{\Gamma, [D]}{\Sp'}{\susp{p^-_d}}$
  \` (given)
  \\
  $T :: (\Psi; \opsubst{\Gamma}, x{:}{\sf eval_a}\,t_1)
    \leadsto^*_{\transop{E}} (\Psi; \opsubst{\Gamma}, y{:}{\sf eval_a}\,t_2)$
  \` (ind. hyp. (part 2) on $N$)
  \\
  $\slst{\transop{\Sigma}}{\Psi}{\Gamma, [\opsubst{D'}]}{\Sp''}{\susp{p_d^-}}$
  \` (ind. hyp. (part 2) on $\Sp'$)
  \\
  $\slst{\transop{\Sigma}}{\Psi}{\opsubst{\Gamma}}
    {\tlaml{\tetap{x}{\,\tlet{T}{y}}}}
    {{\sf eval\_a}\,t_1 \lefti \{ {\sf retn\_a}\,t_2 \}}$
  \` (construction)
  \\
  $\slst{\transop{\Sigma}}{\Psi}{\opsubst{\Gamma}, [\opsubst{D}]}
    {\tappl{\tbangr{(\tlaml{\tetap{x}{\,\tlet{T}{y}}})}}{\Sp'}}
    {\susp{p^-_d}}$
  \` (construction)
  \end{tabbing}
\item If $D = {!}G \lefti D'$, then $\Sp =
  (\tappl{\tbangr{\tetan{N}}}{\Sp'})$. The necessary result is
  $(\tappl{\tbangr{\tetan{N'}}}{\Sp''})$; we get $N'$ from the
  induction hypothesis (part 3) on $N$ and get $\Sp''$ from the induction
  hypothesis (part 2) on $\Sp'$.
\end{itemize}

The cases of part 3 follow the same pattern as the ones from part 2, but
without the interesting case (which is excluded from the refinement $G$);
the result follows by the induction hypothesis (part 3 or part 1). 

In part 4, we have $\slst{\Sigma}{\Psi}{\Gamma}{\tfocusl{\sf
    c}{\Sp}}{\susp{p^-_d}}$, where ${\sf r}{:}C \in \Sigma$ and the
proposition $C$ is equivalent to
$\forall{\overline{x_0}}\ldots\forall{\overline{x_n}}.\, A^+_n \lefti
\ldots \lefti A^+_1 \lefti {\sf a}\,t_0\,t_{n+1}$ as described in
Section~\ref{sec:trans-basic}. This means we can decompose $\Sp$ to
get $\sigma_i = (\overline{s_0}/\overline{x_0},\ldots,
\overline{s_i}/\overline{x_i})$ (for some terms $\overline{s_0} \ldots
\overline{s_i}$ that have the correct type in $\Psi$) and a value
$\slst{\Sigma}{\Psi}{\Gamma}{V_i}{[\sigma_i{A^+_i}]}$ for each $0 \leq
i \leq n$. We also have $t = \sigma_0{t_0}$ and $s = \sigma_n{t_{n+1}}$.

Because 
${\sf r}{:}\forall\overline{x_0}.\,{\sf eval\_a}\,t_0 \lefti \opbasic{A_1, \ldots, A_n}{t_{n+1}}{{\sf id}} \in \transop{\Sigma}$, by left-focusing
on that constant
it suffices to show that there is a $\Sp'$ such that 
$\slst{\transop\Sigma}{\Psi}{\Gamma,[
\opbasic{A_1^+, \ldots, A_n^+}{t_{n+1}}{\sigma_0}
]}{\Sp'}{\susp{\{C^+\}}}$ and a trace of the form
$T :: (\Psi; \tackon{\opsubst{\Theta}}{C^+}) \leadsto^*_{\transop{\Sigma}}
 (\Psi; \tackon{\opsubst{\Theta}}{y{:}{\sf retn\_a}\,})$. We will prove this
by induction on the length of the trace; the general statement is that
there is a $\Sp'$ such that 
$\slst{\transop\Sigma}{\Psi}{\Gamma,[
\opbasic{A_i^+, \ldots, A_n^+}{t_{n+1}}{\sigma_{i-1}}
]}{\Sp'}{\susp{\{C^+\}}}$ and a trace of the form
$T :: (\Psi; \tackon{\opsubst{\Theta}}{C^+}) \leadsto^*_{\transop{\Sigma}}
 (\Psi; \tackon{\opsubst{\Theta}}{y{:}{\sf retn\_a}\,s})$. We proceed
by case analysis on the definition of the operationalization transformation:
\begin{itemize}
\item $\opbasic{}{t_{n+1}}{\sigma_n} = \{ {\sf retn\_a}\,(\sigma_{n}{t_{n+1}}) \}$

  \bigskip
  This is a base case: $\Sp' = \tnil$ and because $\sigma_n{t_{n+1}} = s$, 
  $(\Psi; \tackon{\opsubst{\Theta}}{{\sf retn\_a}\,(\sigma_{n}{t_{n+1}})})$ decomposes
  to $(\Psi; 
  \tackon{\opsubst{\Theta}}{y{:}{\susp{{\sf retn\_a}\,(\sigma_{n}{t_{n+1}})}}})$
  in an inversion phase.
  \bigskip

\item $\opbasic{{!}{\sf a}\,t^{\it in}_n\,t_{n+1}}{t_{n+1}}{\sigma_{n-1}} 
  = \{{\sf eval\_a}\,(\sigma_{n-1}{t^{\it in}_n})\}$

  \bigskip
  We are given a value 
  $\slst{\Sigma}{\Psi}{\Gamma}{\tbangr{N}}
   {[{\bang}{\sf a}\,\sigma_n{t^{\it in}_n}\,\sigma_n{t_{n+1}}]}$;
  observe that $\sigma_{n-1}{t_n^{\it in}} = \sigma_n{t_n^{\it in}}$

  \smallskip
  This is also a base case of the inner induction: $\Sp' = \tnil$ and
  $(\Psi; 
  \tackon{\opsubst{\Theta}}
  {{\sf eval\_a}\,(\sigma_{n}{t^{\it in}_n})})$
  decomposes to 
  $(\Psi; 
  \tackon{\opsubst{\Theta}}
  {x_{n}{:}{\susp{{\sf eval\_a}\,(\sigma_{n}{t^{\it in}_n})}}})$.
  We must demonstrate a trace the rest of the way to
  $(\Psi; \tackon{\opsubst{\Theta}}{y{:}{\sf retn\_a}\,s})$. Because 
  $s = \sigma_n{t_{n+1}}$, this is established by the 
  outer induction hypothesis (part 4) on $N$. 
  \bigskip

\item $\opbasic{p^+_\mpers,A^+_{i+1},\ldots,A^+_n}{t_{n+1}}{\sigma_{i-1}} 
  = \forall \overline{x_i}.\, \sigma_{i-1}{p^+}_\mpers \lefti \opbasic{A^+_i,\ldots,A^+_n}{t_{n+1}}{\sigma_{i-1}}$

  \begin{tabbing}
  $\slst{\Sigma}{\Psi}{\Gamma}{z}{[\sigma_i{p^+_\mpers}]}$
  \` (given) 
  \\
  $\sigma_i = (\sigma_{i-1}, \overline{s_i}/\overline{x_i})$.
  \` (definition of $\sigma_i$)
  \\
  $\slst{\transop{\Sigma}}{\Psi}{\Gamma, [\opbasic{A^+_i,\ldots,A^+_n}{t_{n+1}}{\sigma_{i}}]}{\Sp'}{\susp{\{ C^+ \} }}$
  \` (by inner ind. hyp.)
  \\
  $T :: (\Psi; \tackon{\opsubst{\Theta}}{C^+}) \leadsto^*_{\transop{\Sigma}}
   (\Psi; \tackon{\opsubst{\Theta}}{y{:}{\sf retn\_a}\,s})$
  \` (by inner ind. hyp.)
  \\
  $\slst{\transop{\Sigma}}{\Psi}
    {\Gamma, [\forall \overline{x_i}.\, \sigma_{i-1}{p^+}_\mpers 
                \lefti \opbasic{A^+_i,\ldots,A^+_n}{t_{n+1}}{\sigma_{i-1}}]}
    {\left(\tforalll{\overline{s_i}}{\tappl{z}{\Sp'}}\right)}{\susp{\{ C^+ \}}}$
  \\ 
  \` (construction)
  \end{tabbing}

\item $\opbasic{{!}p^-_{ci},\ldots,{!}p^-_{cj},A^+_{j+1},\ldots,A^+_n}{t_{n+1}}{\sigma_{i-1}}$
  \\
  $~ \qquad = \{ {\sf eval\_bi}\,({\sigma_{i-1}}t^{\it in}_i) 
                    \fuse  \ldots \fuse
                 {\sf eval\_bj}\,({\sigma_{i-1}}t^{\it in}_j)$
  \\
  $~ \qquad \qquad (\forall\overline{x_i}\ldots\forall\overline{x_j}.\, 
     {\sf retn\_bi}\,(\sigma_{i-1}{t^{\it out}_i})
     \fuse \ldots \fuse 
     {\sf retn\_bj}\,(\sigma_{i-1}{t^{\it out}_j})$
  \\
  $~ \qquad \qquad \quad
   \lefti \opbasic{A^+_{j+1},\ldots,A^+_n}{t_{n+1}}{\sigma_{i-1}}) \}$\\
  {\it (where
   $p^-_{ck}$ is ${\sf bk}\,t^{\it in}_k\,t^{\it out}_k$ 
   and $FV(t_k^{\it in}) \notin (\overline{x_i} \cup \ldots \cup \overline{x_j})$ 
   for $i \leq k \leq j$)}

  \bigskip
  Let $\Sp = \tnil$. It then suffices to show that there is a trace
\begin{align*}
    &(\Psi, \opsubst{\Theta} \tackonstart
        x_i{:}\susp{{\sf eval\_bi}\,({\sigma_{i-1}}t^{\it in}_i)}, \ldots,
    x_j{:}\susp{{\sf eval\_bj}\,({\sigma_{i-1}}t^{\it in}_j)},
  \\
  & \qquad\qquad y_{ij}{:}(\forall\overline{x_i}\ldots\forall\overline{x_j}.\, 
     {\sf retn\_bi}\,(\sigma_{i-1}{t^{\it out}_i})
     \fuse \ldots \fuse 
     {\sf retn\_bj}\,(\sigma_{i-1}{t^{\it out}_j})\\
  & \qquad\qquad\qquad
      \lefti \opbasic{A^+_{j+1},\ldots,A^+_n}{t_{n+1}}{\sigma_{i-1}})\,\mtrue
       \tackonstop)
  \\
  & \quad \leadsto^*_{\transop{\Sigma}} 
     (\Psi; \tackon{\opsubst{\Theta}}{y{:}\susp{{\sf retn\_a}\,s}})
\end{align*}

  \begin{tabbing}
  $\slst{\Sigma}{\Psi}{\Gamma}{\tbangr{N_k}}{[{!}{{\sf bk}\,({\sigma_k}t_k^{\it in})\,({\sigma_k}t_k^{\it out})}]}$ \quad $(i \leq k \leq j)$
  \` (given) 
  \\
  $\slst{\Sigma}{\Psi}{\Gamma}{\tbangr{N_k}}{[{!}{{\sf bk}\,({\sigma_{i-1}}t_k^{\it in})\,({\sigma_j}t_k^{\it out})}]}$ \quad $(i \leq k \leq j)$
  \` (condition on translation, defn. of $\sigma_k$)
  \\
  $T :: (\Psi$\=$, \opsubst{\Theta} \tackonstart
        x_i{:}\susp{{\sf eval\_bi}\,({\sigma_{i-1}}t^{\it in}_i)}, \ldots,
    x_j{:}\susp{{\sf eval\_bj}\,({\sigma_{i-1}}t^{\it in}_j)},$\\
  \>$~ \qquad y_{ij}{:}(\forall\overline{x_i}\ldots\forall\overline{x_j}.\, 
     {\sf retn\_bi}\,(\sigma_{i-1}{t^{\it out}_i})
     \fuse \ldots \fuse 
     {\sf retn\_bj}\,(\sigma_{i-1}{t^{\it out}_j})$\\
  \>$~ \qquad\qquad
      \lefti \opbasic{A^+_{j+1},\ldots,A^+_n}{t_{n+1}}{\sigma_{i-1}})\,\mtrue
       \tackonstop)$\\
  $~ \qquad \leadsto^*_{\transop{\Sigma}} 
        (\Psi$\=$, \opsubst{\Theta} \tackonstart
        y_i{:}\susp{{\sf retn\_bi}\,({\sigma_{j}}t^{\it in}_i)}, \ldots,
    y_j{:}\susp{{\sf eval\_bj}\,({\sigma_{j}}t^{\it in}_j)},$\\
  \>$~ \qquad y_{ij}{:}(\forall\overline{x_i}\ldots\forall\overline{x_j}.\, 
     {\sf retn\_bi}\,(\sigma_{i-1}{t^{\it out}_i})
     \fuse \ldots \fuse 
     {\sf retn\_bj}\,(\sigma_{i-1}{t^{\it out}_j})$\\
  \>$~ \qquad\qquad
      \lefti \opbasic{A^+_{j+1},\ldots,A^+_n}{t_{n+1}}{\sigma_{i-1}})\,\mtrue
       \tackonstop)$\\
  \` (by outer ind. hyp. (part 4) on each of the $N_k$ in turn)
  \\
  $\slst{\transop{\Sigma}}{\Psi}
     {\Gamma,[\opbasic{A^+_{j+1},\ldots,A^+_n}{t_{n+1}}{\sigma_{j}}]}
     {\Sp'}{\susp{\{ C^+ \}}}$  \` (by inner ind. hyp.)
  \\
  $T' :: (\Psi, \tackon{\opsubst{\Theta}}{C^+}) \leadsto^*_{\transop{\Sigma}}
        (\Psi, \tackon{\opsubst{\Theta}}{y{:}\susp{{\sf retn\_a}\,s}})$ 
   \` (by inner ind. hyp.)
  \end{tabbing}

  The construction 
  $\left(T; \tstep{\mkpat{C^+}}{y_{ij}}{(\tforalll{\overline{s_i}\ldots\overline{s_j}}{\tappl{(\tfuser{y_i}{\tfuser{\ldots}{y_j}})}{\Sp'}})}; T'\right) $
  is then a trace of the correct type.
  \bigskip

\item $\opbasic{{!}G,A^+_{i+1},\ldots,A^+_n}{t_{n+1}}{\sigma} = \forall
  \overline{x_i}.\, {!}\sigma\opsubst{G} \lefti
  \opbasic{A^+_i,\ldots,A^+_n}{t_{n+1}}{\sigma}$

  \begin{tabbing}
  $\slst{\Sigma}{\Psi}{\Gamma}{\tbangr{N}}{[{!}\sigma_i{G}]}$
  \` (given) 
  \\
  $\slst{\transop{\Sigma}}{\Psi}{\opsubst{\Gamma}}{N'}{\sigma_i{\opsubst{G}}}$
  \` (by outer ind. hyp. (part 3)  on $N$) 
  \\
  $\sigma_i = (\sigma_{i-1}, \overline{s_i}/\overline{x_i})$.
  \` (definition of $\sigma_i$)
  \\
  $\slst{\transop{\Sigma}}{\Psi}{\Gamma, [\opbasic{A^+_i,\ldots,A^+_n}{t_{n+1}}{\sigma_{i}}]}{\Sp'}{\susp{\{ C^+ \} }}$
  \` (by inner ind. hyp.)
  \\
  $T :: (\Psi; \tackon{\opsubst{\Theta}}{C^+}) \leadsto^*_{\transop{\Sigma}}
   (\Psi; \tackon{\opsubst{\Theta}}{y{:}{\sf retn\_a}\,s})$
  \` (by inner ind. hyp.)
  \\
  $\slst{\transop{\Sigma}}{\Psi}
    {\Gamma, [\forall \overline{x_i}.\, {!}(\sigma_{i-1}{G})
                \lefti \opbasic{A^+_i,\ldots,A^+_n}{t_{n+1}}{\sigma_{i-1}}]}
    {\left(\tforalll{\overline{s_i}}{\tappl{\tbangr{N}}{\Sp'}}\right)}{\susp{\{ C^+ \}}}$
  \\ 
  \` (construction)
  \end{tabbing}
\end{itemize}

\noindent
This completes the inner induction in the fourth part, and hence
the proof.
\end{proof}

\section{Logical transformation: defunctionalization}
\label{sec:defunctionalization}

Defunctionalization is a procedure for turning higher-order
concurrent \sls~specifications into first-order concurrent
\sls~specifications. It is based on the following intuitions:
if $A^-$ is a closed negative proposition
of the form $\forall \overline{x}.\,A^+_1 \lefti \{ A^+_2 \}$
and we have a single-step transition 
$(\Psi; \tackon{\Theta}{y{:}\istrue{A^-}}) 
 \leadsto_{\Sigma} 
 (\Psi; \Delta')$
in an \sls~specification (witnessed by the step 
$(\tstep{\mkpat{A^+_2}}{y}{(\tforalll{\overline{t}}{\tappl{V}{\tnil}})})$), 
then we can define an augmented signature
\begin{align*}
\Sigma' = ~ & \Sigma, 
\\    ~~ & {\sf cont} : {\sf prop\,ord}, 
\\    ~~ & {\sf run\_cont} : \forall{\overline x}.\,p^+_\mtrue \fuse {\sf cont} \lefti \{ A^+ \}
\end{align*}
and it is the case that 
$(\Psi; \tackon{\Theta}{y{:}\susp{\sf cont}}) 
 \leadsto_{\Sigma'} 
 (\Psi; \Delta')$
as well; this new transition is witnessed by the step
$(\tstep{\mkpat{A^+}}{\sf run\_cont}{(\tappl{(\tfuser{V}{y})}{\tnil})})$.

More generally, if we are allowed to extend the signature and $A^-$
falls into the very specific form we have
specified,\footnote{Obviously, the restriction to propositions $A^-$
  of the form $\forall \overline{x}.\,p^+_\mtrue \lefti \{ A^+ \}$ is
  overly specific and designed to apply specifically to the output of
  operationalization, but we will not consider a generalization here.
  Conceptually, it is not complicated to consider a similar operation
  on other propositions, but it is difficult to elegantly describe the
  more general transformation due our use of ordered logic, and we do
  not currently need the more general transformation.}  we can create
a new ordered atomic proposition to do a negative proposition's
job. As long as $\Delta = \tackon{\Theta}{x{:}\susp{\sf cont}}$ and
${\sf cont}$ does not appear in $\Theta$, then $[{\downarrow}A^- /{\sf
  cont}]\Delta \leadsto_\Sigma [{\downarrow}A^- /{\sf cont}]\Delta'$
if and only if $\Delta \leadsto_{\Sigma'} \Delta'$. \footnote{Recall
  from Section~\ref{sec:framework-substprop} that we treat
  %
  $[{\downarrow}A^-/{\sf cont}](\tackon{\Theta}{z{:}\istrue{\susp{\sf cont}}})$
  %
  as being equal to the context in which we {\it first} perform the
  straightforward substitution, giving us
  $(\tackon{\Theta}{z{:}\istrue{{\downarrow}A^-}})$, and then {\it
    second} apply invertible rules, giving us
  $(\tackon{\Theta}{z'{:}\istrue{A^-}})$.} 

We need not restrict ${\sf cont}$ to just a single appearance
suspended in the process state, however. Multiple instances of ${\sf
  cont}$ can appear in the process state without a problem.  It is
similarly unproblematic for ${\sf cont}$ to appear in the monadic head
of some other rule in the process state, as the appearance of an
ordered atomic proposition in a monadic head will not effect the
existence of any transition, but may cause the ordered atomic
proposition to become a suspended ordered proposition in the process
state after the transition. 

By the same reasoning, it is similarly
unproblematic for ${\sf cont}$ to appear in the head of a rule in the
signature.  Therefore, we can replace propositions in the monadic heads
of rules in the signature, like this one:
\begin{align*}
& \Sigma, \\
& {\sf r} : {\sf a} \lefti \{ {\sf b} \fuse {\uparrow}({\sf c} \lefti \{ {\sf d} \fuse {\uparrow}({\sf e} \lefti \{ {\sf f} \}) \}) \}
\intertext{to produce a signature that looks like this:}
& \Sigma, \\
& {\sf cont1} : {\sf prop\,ord}, \\
& {\sf r1} : {\sf c} \fuse {\sf cont1} \lefti \{ {\sf d} \fuse {\uparrow}({\sf e} \lefti \{ {\sf f} \}) \}, \\
& {\sf r} : {\sf a} \lefti \{ {\sf b} \fuse {\sf cont1} \}
\intertext{and the process can be iterated to obtain 
a fully first-order signature:}
& \Sigma, \\
& {\sf cont2} : {\sf prop\,ord}, \\
& {\sf r2} : {\sf e} \fuse {\sf cont2} \lefti \{ {\sf f} \}, \\
& {\sf cont1} : {\sf prop\,ord}, \\
& {\sf r1} : {\sf c} \fuse {\sf cont1} \lefti \{ {\sf d} \fuse {\sf cont2} \}, \\
& {\sf r} : {\sf a} \lefti \{ {\sf b} \fuse {\sf cont1} \}
\end{align*}
This propositional transformation is similar to the one proposed by
Miller in~\cite{miller02higherorder}, where the new propositions were
introduced to hide the internal states of processes.

We can go further and allow $A^-$ to contain free variables if
$A^- = [t_1/y_1]\ldots[t_m/x_m]B^-$ where $B^- = \forall
\overline{x}.\,B_1^+ \lefti \{ B_2^+ \}$ has only the variables
$\overline{y} = y_1\ldots y_m$ free.  In this more general case, we
can revise the signature as follows:
\begin{align*}
\Sigma'' = ~ & \Sigma,
\\    ~~ & {\sf cont} : 
       \Pi x_y{:}\tau_1\ldots \Pi y_m{:}\tau_m.\, {\sf prop\,ord},
\\    ~~ & {\sf run\_cont} : \forall \overline{x}.\,\forall \overline{y}.\,
       p^+_\mtrue \fuse {\sf cont}\,\overline{y} \lefti \{ B^+ \}
\end{align*}
With this revision, we maintain that
%
$[{\downarrow}B^-/{\sf cont}\,\overline{x}]\Delta \leadsto_{\Sigma}
[{\downarrow}B^-/{\sf cont}\,\overline{x}]\Delta'$ if and only if
$\Delta \leadsto_{\Sigma''} \Delta'$ (as long as propositions of the
form ${\sf cont}\,\overline{t}$ only appear suspended in the process
state or in the monadic heads of rules that appear in the process
state).\robnote{The process of proving this is mostly an issue of
  stating it precisely, which is a pain. I'd appreciate feedback as to
  whether this seems clear or whether I need to write out the detailed
  proof.}

The one twist we make to the defunctionalization transformation is
that, instead of introducing a new ordered atomic proposition ${\sf
  cont}\,\overline{t}$ for each iteration of the defunctionalization
procedure, we introduce a single type $({\sf frame} : {\sf type})$ and a
single atomic proposition $({\sf cont} : {\sf frame} \rightarrow {\sf
  prop\,ord})$. Then, each iteration of the defunctionalization
procedure produces a new constant with type $\Pi x_y{:}\tau_1\ldots
\Pi y_m{:}\tau_m.\, {\sf frame}$ instead of a new atomic proposition
with kind $\Pi x_y{:}\tau_1\ldots \Pi y_m{:}\tau_m.\, {\sf
  prop\,ord}$.  Operationally, these two approaches are equivalent,
though the approach using frames requires us to disallow variables
that can construct new terms of type ${\sf frame}$ from appearing in
the variable context $\Psi$.

\begin{figure}
\fvset{fontsize=\small,boxwidth=229pt}
\VerbatimInput{sls/cbv-ev-ssos-fun.sls}
\caption{A first-order ordered abstract machine semantics for CBV
  evaluation.}
\label{fig:cbv-ev-ssos-fun}
\end{figure}

Using defunctionalization procedure outlined above, we obtain the
first-order specification in Figure~\ref{fig:cbv-ev-ssos-fun} from the
higher-order specification in Figure~\ref{fig:cbv-ev-ssos-tail}, which
was in turn derived from the natural semantics for CBV evaluation by
operationalization with tail-recursion optimization.

% As long as 
% $({\sf cont}\,t_1\ldots t_n)$ only appears in $\Delta$ as a 
% suspended atomic proposition, then it is the case that
% $[{\downarrow}(B^-\,x_1\ldots x_n)
%     /{\sf cont}\,x_1\ldots x_n]\Delta 
%  \leadsto_\Sigma
%  [{\downarrow}(B^-\,x_1\ldots x_n)
%     /{\sf cont}\,x_1\ldots\,x_n]\Delta'$ 
% if and only if 
% %
% $\Delta \leadsto_{\Sigma''} \Delta'$.\footnote{Recall from
%   Section~\ref{sec:framework-substprop} that
%   $[{\downarrow}(B^-\,x_1\ldots x_n)/({\sf cont}\,x_1\ldots
%   x_n)](z{:}\susp{{\sf cont}\,t_1\ldots t_n})$ as being equal to the
%   context in which we substitue and then apply invertible rules, i.e.
%   $z{:}\istrue{B^-\,t_1\ldots t_n}$}

% In addition to allowing these newly introduced 
% atomic propositions to appear suspended in the context, it is not 
% a problem to allow them to appear in the heads of monadic clauses. 
% This means that we can 

%  monadic
% clauses, there is no 

% The defunctionalization transformation then applies the same reasoning
% to signatures: if a proposition ${\downarrow}A^-$ appears in the monadic
% head of some rule, 

%  $A^- = B^-\,t_1\,t_2\,t_3$

% This is even
% true if $A^-$ has free variables: we can always define a closed
% $B^- : $




\section{Adequacy with abstract machines}
\label{sec:nat-ssos-adequacy}

I claim that the four-rule abstract machine specification given at the
beginning of this chapter is adequately represented by the derived
\sls~specification in Figure~\ref{fig:cbv-ev-ssos-fun}. For terms and
for deductive computations, adequacy is a well-understood concept: we
know what it means to define an adequate encoding function $\interp{e}
= t$ from ``on-paper'' terms $e$ with (potentially) variables
$x_1,\ldots,x_n$ free to LF terms $t$ where $x_1{:}{\sf
  exp},\ldots,x_n{:}{\sf exp} \vdash t : {\sf exp}$, and we know what
it means to adequately encode the judgment $e \Downarrow v$ as a
negative atomic \sls~proposition ${\sf ev}\,\interp{e}\,\interp{v}$
and to encode derivations of this judgment to \sls~terms $N$ where
$\slst{\Sigma}{\cdot}{\cdot}{N}{\susp{{\sf
      ev}\,\interp{e}\,\interp{v}}}$
\cite{harper93framework,harper07mechanizing}. What does it mean to
adequately represent machine states as process states (that is,
substructural contexts) and to encode a transition system as a 
concurrent \sls~specification? 

The answer given in the literature by Cervesato et
al.~\cite{cervesato02concurrent} and by
Schack-Nielsen~\cite{schacknielsen07induction} has three steps. The
first step is to, define an interpretation function from states $s$
and stacks $k$ to process states $\Delta$, so that, for example, the
state
\[
((\ldots({\sf halt}; \Box\,e_1)\ldots); (\lambda x.e_n)\,\Box) \lhd v
\]
is interpreted as the process state
\[
y{:}\susp{{\sf retn}\,\interp{v}}, ~~
x_n{:}\susp{{\sf cont}\,({\sf app2}\,\lambda x.\interp{e_n})}, ~~
\ldots, ~~
x_1{:}\susp{{\sf cont}\,({\sf app1}\,\interp{e_1})}, ~~
\]
The second step is to, prove a preservation-like adequacy theorem. Let
$\Sigma\ref{fig:cbv-ev-ssos-fun}$ be the signature from
Figure~\ref{fig:cbv-ev-ssos-fun}: we show that if state $s$ is
interpreted and $\Delta$ and $\Delta
\leadsto_{\Sigma\ref{fig:cbv-ev-ssos-fun}} \Delta'$, then there is a
state $s'$ such that $s'$ is interpreted as $\Delta'$. Then we can
prove the main adequacy result: that the interpretation of state $s$
steps to the interpretation of state $s'$ if and only if $s \mapsto
s'$.

I believe that the approach to adequacy given in previous work is
unsatisfactory because the interpretation of process states into
contexts is 1-to-1 but not onto (and therefore not invertible).  This
means that there is no {\it internal} notion of what it means for a
process state to encode a state $s$ or a stack $k$. By analogy,
``having type ${\sf exp}$'' captures what it means for an LF term
encode an expression and ``having type ${\sf
  ev}\,\interp{e}\,\interp{v}$'' captures what it means for an
\sls~term to encode a derivation of $e \Downarrow v$.

In this section, we will present a different three-part approach that
addresses this perceived deficiency. First, we create a signature
$\Sigma\sf gen$ that encodes well-formed states: the $\Delta$ such
that $x{:}\susp{\sf gen} \leadsto^*_{\Sigma\sf gen} \Delta$ and ${\sf
  gen} \notin \Delta$ are in a bijection with the states $s$
(Section~\ref{sec:nat-ssos-adequacy-gen}). This gives us the internal
notion of what it means to encode a process state, which is what we
were previously lacking. Second, we prove the preservation-like
property from before. The difference is that this can now be stated
formally as a property of \sls~specifications: if $x{:}\susp{\sf gen}
\leadsto^*_{\Sigma\sf gen} \Delta$ and $\Delta
\leadsto_{\Sigma\ref{fig:cbv-ev-ssos-fun}} \Delta'$, then $x{:}
\susp{\sf gen} \leadsto^*_{\Sigma\sf gen} \Delta'$
(Section~\ref{sec:nat-ssos-adequacy-pres}). The structure of this
theorem is critical, a point that we will consider in greater depth in
Part III of this thesis. Finally, the third step is the same as it was
in other approaches: we prove that the interpretation of state $s$
steps to the interpretation of state $s'$ if and only if $s \mapsto s'$.

\subsection{Adequacy of states}
\label{sec:nat-ssos-adequacy-gen}

Our first goal is to describe a signature $\Sigma\sf gen$ with the
property that if $x{:}\susp{\sf gen} \leadsto^*_{\Sigma\sf gen}
\Delta$ and ${\sf gen} \notin \Delta$ then $\Delta$. A well-formed
process state represting a state $k \rhd e$ has the form
\[
y{:}\susp{{\sf eval}\,\interp{e}}, ~~
x_n{:}\susp{{\sf cont}\,\interp{f_n}}, ~~
\ldots, ~~
x_1{:}\susp{{\sf cont}\,\interp{f_1}}
\]
where $\interp{\Box\,e_2} = {\sf app1}\,\interp{e_2}$ and
$\interp{(\lambda x.e)\,\Box} = {\sf app2}\,(\lambda.\interp{e})$. 
A well-formed process state representing a state $k \lhd e$ has 
the same form, but with a suspended ${\sf retn}\,\interp{v}$ instead
of ${\sf eval}\,\interp{e}$. 

The simplest \sls~signature that encodes this structure essentially
has the structure of a Chomsky normal form describing well-formed
contexts, with two unary productions and one unary production.

\smallskip
\fvset{fontsize=\small,boxwidth=229pt}
\VerbatimInput{sls/cbv-ev-ssos-gen.sls}
\smallskip

\noindent In addition to the four declarations above, the full
signature $\Sigma\sf gen$ includes all the type, proposition, and
constant declarations from Figure~\ref{fig:cbv-ev-ssos-fun}, but none
of the rules.

Note that this specification most definitely is {\it not} well-moded.
{\it Generative signatures} such as this one are not generally moded,
and we don't think about traces under these signatures as 
necessarily being concurrent computations in the same way we think
about ordered abstract machines being concurrent computations. That is,
rather than thinking of traces in these signatures being produced by
different computations, such as the computational content of the adequacy
theorem:

\bigskip
\begin{theorem}[Adequacy of states]~
\label{thm:adequacy-states}
\begin{itemize}
\item There is a bijection (up to the renaming of variables in the context) 
  between states $s$ and contexts $\Delta$ such that
  $x{:}\susp{\sf gen} \leadsto^*_{\Sigma\sf gen} \Delta$ where 
  ${\sf gen} \notin \Delta$.
\item There is a bijection (up to the renaming of variables in the context) 
  between stacks $k$ and frames $\Theta$ such that $x{:}\susp{\sf
    gen} \leadsto^*_{\Sigma\sf gen} \tackon{\Theta}{x'{:}{\sf gen}}$.
\end{itemize}
\end{theorem}

\begin{proof}
We will give only the two translation from the ``on paper''
semantic artifacts (states $s$ and stacks $k$) to traces:
\begin{itemize}
\item $\interp{s},$ which outputs
a trace $T$ with type $x{:}\susp{\sf gen} \leadsto_{\Sigma\sf gen} \Delta$
where ${\sf gen} \not\in \Delta$, and 
\item $\interp{k}$, which outputs
both a trace $T$ with type 
$x{:}\susp{\sf gen}
  \leadsto_{\Sigma\sf gen} \tackon{\Theta}{x'{:}\susp{\sf gen}}$ where
${\sf gen} \not\in \Delta$ and the variable name $x'$ of the resulting
${\sf gen}$ proposition (which may be the same as $x$). Rather than 
representing this output explicitly, we just assume it is always
named $x'$ in the definition below.  
\end{itemize}
Note that both functions builds contexts only indirectly by building 
traces; similarly, the inverses of these functions are defined by induction
on the structure of traces, not on the structure of contexts.
\begin{tabbing}
~~ \= \qquad\quad\qquad \= $~ :: ~$ \=\kill
\> $\interp{k \rhd e} = \interp{k}; 
     \tstep{z}{\sf gen/eval}{(\tforalll{\interp{e}}{(\tappl{x'}{\tnil})})}$
\\ \>\> $~ :: ~$ 
  \> $x{:}\susp{\sf gen} 
       \leadsto^*_{\Sigma\sf gen} \tackon{\Theta}{x'{:}\susp{\sf gen}} 
       \leadsto_{\Sigma\sf gen} 
          \tackon{\Theta}{z{:}\susp{{\sf eval}\,\interp{e}}}$
\\[4pt]
\> $\interp{k \lhd v} = \interp{k}; 
     \tstep{z}{\sf gen/retn}{(\tforalll{\interp{v}}{(\tappl{x'}{\tnil})})}$
\\ \>\> $~ :: ~$ 
  \> $x{:}{\sf gen} 
       \leadsto^*_{\Sigma\sf gen} \tackon{\Theta}{x'{:}\susp{\sf gen}} 
       \leadsto_{\Sigma\sf gen} 
          \tackon{\Theta}{z{:}\susp{{\sf retn}\,\interp{v}}}$
\\[4pt]
\> $\interp{{\sf halt}} = \emptytrace$
\> $~ :: ~$
  \> $x{:}\susp{\sf gen} \leadsto^* x{:}\susp{\sf gen}$
\\[4pt]
\> $\interp{k; \Box\,e_2} = \interp{k}; \tstep{z, x''}{\sf gen/cont}
     {(\tforalll{{\sf app1}\,\interp{e_2}}{(\tappl{x'}{\tnil})})}$
\\ \>\> $~ :: ~$
 \> $x{:}\susp{\sf gen}
       \leadsto^*_{\Sigma\sf gen} \tackon{\Theta}{x'{:}\susp{\sf gen}}
       \leadsto_{\Sigma\sf gen} \tackon{\Theta}
            {\mkconj
               {z{:}\susp{{\sf cont}\,({\sf app1}\,\interp{e_2})}}
               {x''{:}\susp{\sf gen}}}$
\\[4pt]
\> $\interp{k; (\lambda x.e)\,\Box} = \interp{k}; \tstep{z, x''}{\sf gen/cont}
     {(\tforalll{{\sf app1}\,\interp{e_2}}{(\tappl{x'}{\tnil})})}$
\\ \>\> $~ :: ~$
 \> $x{:}\susp{\sf gen}
       \leadsto^*_{\Sigma\sf gen} \tackon{\Theta}{x'{:}\susp{\sf gen}}
       \leadsto_{\Sigma\sf gen} \tackon{\Theta}
            {\mkconj
               {z{:}\susp{{\sf cont}\,({\sf app2}\,\lambda x.\interp{e})}}
               {x''{:}\susp{\sf gen}}}$
\end{tabbing}
To complete the theorem, it is necessary to show that the two encoding
functions are one-to-one and onto. This can be done by demonstrating
the existence of a function $\interp{T}^{-1}_s = s$ from traces $T$
with type $x{:}\susp{\sf gen} \leadsto_{\Sigma\sf gen} \Delta$ where
${\sf gen} \notin \Delta$ to states $s$ and a function
$\interp{T}^{-1}_k = k$ from traces $T$ with type $x{:}\susp{\sf gen}
\leadsto_{\Sigma\sf gen} \tackon{\Theta}{x'{:}{\sf gen}}$ (where $x$
and $x'$ may be the same) to stacks $k$ and then showing that
the functions compose to the identity in both directions. In this case,
that proof is tedious but straightforward.
\end{proof}

Note that two traces $T :: x{:}\susp{\sf gen} \leadsto^*_{\Sigma\sf
  gen} \Delta$ and $T' :: x{:}\susp{\sf gen} \leadsto^*_{\Sigma\sf
  gen} \Delta'$ are distinct if and only if the contexts $\Delta$ and
$\Delta'$ are distinct. Therefore, we can equivalently see adequacy as
a bijection between traces and abstract machine states $s$ or as a
bijection between contexts and abstract machine states $s$. In a
situation where this 1-to-1 correspondence between states and traces
did not exist (because two traces generated the same context), it is
not entirely clear whether it would be preferable to define adquecy in
terms of contexts or in terms of traces.

\subsection{Preservation}
\label{sec:nat-ssos-adequacy-pres}

Before we prove that the concurrent system from
Figure~\ref{fig:cbv-ev-ssos-fun} adequately represents the transition
system from the beginning of the chapter, we must show that our
criteria for context well-formedness is actually preserved by the
concurrent computations in Figure~\ref{fig:cbv-ev-ssos-fun}. This is
part of the adequacy argument, but because we state it in terms of the
generative signature $\Sigma\sf gen$, it is also a reasonable
standalone theorem entirely about of \sls~specifications. We will
return to theorems of this form in Part III of this thesis.

\bigskip
\begin{theorem}[Generation by $\Sigma\sf gen$ is invariant under
 $\Sigma\ref{fig:cbv-ev-ssos-fun}$]\label{thm:adequate-pres}~\\
  If $x{:}\susp{\sf gen} \leadsto^*_{\Sigma\sf gen} \Delta$ and
  $\Delta \leadsto_{\Sigma\ref{fig:cbv-ev-ssos-fun}} \Delta'$, then
  $x{:} \susp{\sf gen} \leadsto^*_{\Sigma\sf gen} \Delta'$
\end{theorem}

\begin{proof}
  Primarily by enumeration of the possible synthetic transitions of
  $\Sigma\ref{fig:cbv-ev-ssos-fun}$ and secondarily by case analysis
  on the structure of the trace $T :: x{:}\susp{\sf gen}
  \leadsto^*_{\Sigma\sf gen} \Delta$.

  \begin{itemize}
  \item $\tstep{z}{\sf ev/lam}{(\tforalll{\lambda x.e\,x}
                                {(\tappl{y}{\tnil})})}$

    \qquad $:: \frameoff{\Theta}
                 {y{:}\susp{{\sf eval}\,({\sf lam}\,\lambda x.e\,x)}}
               \leadsto
               \tackon{\Theta}
                 {z{:}\susp{{\sf retn}\,({\sf lam}\,\lambda x.e\,x)}} $

    \medskip

    $T = T'; \tstep{y}{\sf gen/eval}{(\tforalll{{\sf lam}\,\lambda x.e\,x}{\tappl{x'}{\tnil}})}$,\\
    so we construct\\
    $T'; \tstep{z}{\sf gen/retn}{(\tforalll{{\sf lam}\,\lambda x.e\,x}{\tappl{x'}{\tnil}})}$

    \medskip

  \item $\tstep{z_1, z_2}{\sf ev/app}{(\tforalll{e_1}
                                {\tforalll{e_2}{(\tappl{y}{\tnil})}})}$

    \qquad $:: \frameoff{\Theta}
                 {y{:}\susp{{\sf eval}\,({\sf app}\,e_1\,e_2)}}
               \leadsto
               \tackon{\Theta}
                 {\mkconj
                  {z_1{:}\susp{{\sf eval}\,e_1}}
                  {z_2{:}\susp{{\sf cont}\,({\sf app1}\,e_2)}}} $

    \medskip

    $T = T'; \tstep{y}{\sf gen/eval}{(\tforalll{{\sf app}\,e_1\,e_2}{\tappl{x'}{\tnil}})}$,\\
    so we construct\\
    $T'; 
     \tstep{z', z_2}{\sf gen/cont}{(\tforalll{{\sf app1}\,e_2}{(\tappl{x'}{\tnil})})};
     \tstep{z_1}{\sf gen/eval}{(\tforalll{e_1}{\tappl{z'}{\tnil}})}$

    \medskip


  \item $\tstep{z_1, z_2}{\sf ev/app1}{(\tforalll{\lambda x.e\,x}
                       {\tforalll{e_2}{(\tappl{\tfuser{y_1}{y_2}}{\tnil})}})}$

    \qquad $:: \frameoff{\Theta}
                 {\matchconj
                  {y_1{:}\susp{{\sf retn}\,({\sf lam}\,\lambda x.e\,x)}}
                  {y_2{:}\susp{{\sf cont}\,({\sf app1}\,e_2)}}}$

    \qquad\qquad
               $\leadsto
               \tackon{\Theta}
                 {\mkconj
                  {z_1{:}\susp{{\sf eval}\,e_2}}
                  {z_2{:}\susp{{\sf cont}\,({\sf app2}\,(\lambda x.e\,x))}}} $

    \medskip

    $T =
     T'; 
     \tstep{y', y_2}{\sf gen/cont}{(\tforalll{{\sf app1}\,e_2}{\tappl{x'}{\tnil}})};
     \tstep{y_1}{\sf gen/retn}{(\tforalll{{\sf lam}\,\lambda x.e\,x}{(\tappl{y'}{\tnil})})}$,\\
    so we construct\\
    $T'; 
     \tstep{z', z_2}{\sf gen/cont}{(\tforalll{{\sf app2}\,\lambda x.e\,x}{(\tappl{x'}{\tnil})})};
     \tstep{z_1}{\sf gen/eval}{(\tforalll{e_2}{\tappl{z'}{\tnil}})}$

    \medskip

  \item $\tstep{z}{\sf ev/app2}{(\tforalll{v_2}
                       {\tforalll{\lambda x.e\,x}
                         {(\tappl{\tfuser{y_1}{y_2}}{\tnil})}})}$

    \qquad $:: \frameoff{\Theta}
                 {\matchconj
                  {y_1{:}\susp{{\sf retn}\,v_2}}
                  {y_2{:}\susp{{\sf cont}\,({\sf app2}\,\lambda x.e\,x)}}}
               \leadsto
               \tackon{\Theta}
                 {z{:}\susp{{\sf eval}\,(e\,v_2)}} $

    \medskip

    $T =
     T'; 
     \tstep{y', y_2}{\sf gen/cont}{(\tforalll{{\sf app2}\,\lambda x.e\,x}{\tappl{x'}{\tnil}})};
     \tstep{y_1}{\sf gen/retn}{(\tforalll{v_2}{(\tappl{y'}{\tnil})})}$,\\
    so we construct\\
    $T'; 
     \tstep{z}{\sf gen/eval}{(\tforalll{e\,v_2}{\tappl{x'}{\tnil}})}$

    \medskip

  \end{itemize}

\noindent
This completes the proof. 
\end{proof}

It is straightfoward to see that the primary pattern match in this
proof covers all the cases. We postpone, for now, the less obvious
discussion of how we ensure that the secondary case analyses on
generative traces covered all the cases.\robnote{Fill in a forward
  reference above with a concrete reference when one exists}

\subsection{Adequacy of the transition system}
\label{sec:nat-ssos-adequacy-absmachine}

The most interesting part of the adequacy proof was showing that
formation by generative signature $\Sigma\sf gen$ was an invariant of
$\Sigma\ref{fig:cbv-ev-ssos-fun}$. With that property established, the
final step is as straightforward as 

\bigskip
\begin{theorem}[Adequacy of the transition system]
$s \mapsto s'$ if and only if there exist $\Delta$ and $\Delta'$
such that
$\Delta \leadsto_{\Sigma\ref{fig:cbv-ev-ssos-fun}} \Delta'$,
$\interp{s} :: x{:}\susp{\sf gen} \leadsto^*_{\Sigma\sf gen} \Delta$, and
$\interp{s'} :: x{:}\susp{\sf gen} \leadsto^*_{\Sigma\sf gen} \Delta'$. 
\end{theorem}

\begin{proof} The proof is by straightforward case analysis and
  construction; we will give the case associated with ${\sf ev/app}$
  in both directions.

  The forward direction prooceeds by case analysis over the definition
  of the transition system from the beginning of the chapter.  For
  instance, if $k \rhd e_1\,e_2 \mapsto (k; \Box\,e_2) \rhd e_1$ by
  rule ${\sf absmachine/app}$ then we can form (by
  Theorem~\ref{thm:adequacy-states}) the following traces:
  \begin{align*}
  \interp{k \rhd e_1\,e_2} 
  & :: x{:}\susp{\sf gen} \leadsto^*_{\Sigma\sf gen}
       \tackon{\Theta}
        {y{:}\susp{{\sf eval}\,({\sf app}\,\interp{e_1}\,\interp{e_2})}}
\\
  \interp{(k; \Box\,e_2) \rhd e_1} 
  & :: x{:}\susp{\sf gen} \leadsto^*_{\Sigma\sf gen}
       \tackon{\Theta}
        {\mkconj{z_1{:}\susp{{\sf eval}\,\interp{e_1}}}
         {z_2{:}\susp{{\sf cont}\,({\sf app1}\,\interp{e_2})}}}
  \end{align*}
  It is then possible to construct the required step:
  \begin{align*} 
  &\tstep{z_1, z_2}{\sf ev/app}{(\tforalll{\interp{e_1}}{\tforalll{\interp{e_2}}{(\tappl{y}{\tnil})}})}
  \\
  &\qquad\qquad :: \tackon{\Theta}
        {y{:}\susp{{\sf eval}\,({\sf app}\,\interp{e_1}\,\interp{e_2})}}
     \leadsto_{\Sigma\ref{fig:cbv-ev-ssos-fun}}
     \tackon{\Theta}
        {\mkconj{z_1{:}\susp{{\sf eval}\,\interp{e_1}}}
         {z_2{:}\susp{{\sf cont}\,({\sf app1}\,\interp{e_2})}}}
  \end{align*}

  In the backward direciton, we are given a step in the dynamic 
  semantics, such as the one above, as well as the two traces 
  \begin{align*}
  T_1
  & :: x{:}\susp{\sf gen} \leadsto^*_{\Sigma\sf gen}
       \tackon{\Theta}
        {y{:}\susp{{\sf eval}\,({\sf app}\,\interp{e_1}\,\interp{e_2})}}
\\
  T_2
  & :: x{:}\susp{\sf gen} \leadsto^*_{\Sigma\sf gen}
       \tackon{\Theta}
        {\mkconj{z_1{:}\susp{{\sf eval}\,\interp{e_1}}}
         {z_2{:}\susp{{\sf cont}\,({\sf app1}\,\interp{e_2})}}}
  \end{align*}
  By the same case analysis on the structure of the trace that we performed
  in the preservation theorem (Theorem~\ref{thm:adequate-pres}) and
  , we 
  need to establish that \medskip \\
  $T_1 = T'; \tstep{y}{\sf gen/eval}{(\tforalll{\interp{e_1}}{\tforalll{\interp{e_2}}{(\tappl{x'}{\tnil})}})}$ and \\
  $T_2 = T'; \tstep{z', z_2}{\sf gen/cont}{(\tforalll{{\sf app1}\,\interp{e_2}}{(\tappl{x'}{\tnil})})}; \tstep{z_1}{\sf gen/eval}{(\tforalll{\interp{e_1}}{(\tappl{z'}{\tnil})})}$ \medskip\\
  % 
  where $T' :: x{:}\susp{\sf gen} \leadsto^*_{\Sigma\sf gen}
  \tackon{\Theta}{x'{:}{\sf gen}}$ in both cases. Therefore,
  Theorem~\ref{thm:adequacy-states} there is a stack $k$ such that
  $\interp{k} = T'$, $\interp{k \rhd e_1\,e_2} = T_1$, and
  $\interp{(k; \Box\,e_2) \rhd e_1} = T_2$.
  We conclude, then, by observing that 
  $k \rhd e_1\,e_2 \mapsto (k; \Box\,e_2) \rhd e_1$ by rule 
  ${\sf absmachine/app}$.
\end{proof}

\section{Exploring the richer fragment}
\label{sec:richer-ordered-abstract}



\subsection{Mutable storage}
\label{sec:mutable-storage}

\begin{figure}
\fvset{fontsize=\small,boxwidth=229pt}
\VerbatimInput{sls/ssos-mutable.sls}
\caption{SSOS semantics of mutable storage.}
\label{fig:ssos-mutable}
\end{figure}


\subsubsection{Existential angst} 

No check for pointer inequality! This is a fundamental restriction of
the fact that we're using existential quantification rather than some
form of nominal quantification. (Hack due to Favonia and Bob, personal
communication, but dates back earlier - was it one of Karl's papers?
Cheney cites it in nominal abstraction.)

\subsection{Call-by-need evaluation}

\subsection{Recoverable failure}

\subsubsection{Failures and the parallel translation}

\subsection{Environment semantics}

\subsection{Looking back at natural semantics}
\label{sec:enriching-natsem}

\section{Partial transformation}
\label{sec:othertransform}

\subsection{Evaluation contexts}
\label{sec:evaluationcontexts}

Thus far, we have considered big-step operational semantics and abstract
machines, neglecting the third great tradition of programming language
specification, {\it structural operational semantics}. Structural
operational semantics (SOS) define single-step evaluation inductively over
the structure of expressions; the SOS semantics for our running example
language is the following:
\[
\infer
{\lambda x.e\,{\sf value} \mathstrut}
{}
\quad
\infer
{e_1\,e_2 \mapsto e_1'\,e_2 \mathstrut}
{e_1 \mapsto e_1' \mathstrut}
\quad
\infer
{e_1\,e_2 \mapsto e_1\,e_2' \mathstrut}
{e_1\,{\sf value}
 &
 e_2 \mapsto e_2' \mathstrut}
\quad
\infer
{(\lambda x. e)v \mapsto [v/x]e \mathstrut}
{v\,{\sf value} \mathstrut}
\]
This inductive specification is adequately encoded on the left-hand
side of Figure~\ref{fig:cbv-sos}, along with the proposition \Verb|ev|
that describes a big-step operational semantics in terms of repeated
application of the small-step operational semantics.

\begin{figure}[tp]
\fvset{fontsize=\small,boxwidth=229pt}
\BVerbatimInput{sls/cbv-sos.sls}
\BVerbatimInput{sls/cbv-sos-eval.sls}
\caption{Small-step evaluation, and one corresponding abstract machine.}
\label{fig:cbv-sos}
\end{figure}

\fvset{fontsize=\small}

There are a couple of possibilities for how the 
One obvious way to proceed is to simply translate the big-step portion
of our semantics as encoded 


If we just translate the ${\sf ev}$ portion of the semantics (using
the tail-recursion optimizing translation), then we will get what is
probably fair to call the most boring possible substructural
operational semantics: 

\smallskip
\VerbatimInput{sls/cbv-sos-proc.sls}
\smallskip

\noindent
Under this semantics, the substructural context contains a single
resource, \Verb|eval-steps(E)|, which takes steps according to the
rules of the small-step structural operational semantics until a value
is reached, at which point the context contains \Verb|retn-steps(V)|.


\begin{figure}[t]
\VerbatimInput{sls/cbv-sos-defun.sls}
\caption{The defunctionalized abstract machine from Figure~\ref{fig:cbv-sos}.}
\label{fig:cbv-sos-defun}
\end{figure}

The interesting observations are to be had from the other direction: what if

\subsection{Temporal logic}

The natural semantics of \rowan~are not, on a superficial level,
significantly more complex than other natural semantics. However, it
turns out that the usual set of techniques for adding state to a
natural semantics break down, and discussing a \rowan-like logic with
state remained a challenge for many years.\robnote{Figure out from
  Rowan what the recent work he told you about was.} Through the
logical correspondence, it is easy to see why: the natural SSOS
specification of \rowan~integrates both concurrent and deductive
reasoning in an arbitrarily nested way. In fact, Figure XXX is the
only SLS specification in this thesis that exhibits this form of
recursive dependency between concurrent and deductive reasoning.  In
particular, the \rowan~specification is way out of the image of the
extended natural semantics we considered in
Section~\ref{sec:enriching-natsem}. The natural encoding in state lies
in the ambient substructural context of a concurrent computation, but
that ambient computation cannot properly enter into a deductive
sub-computation. If we tried to add state to \rowan~the same way we
added it in Section~\ref{sec:mutable-storage}, the entire store
would effectively leave scope whenever computation considered
the subterm $e$ of ${\sf next}(e)$. That consideration happens
as deductive reasoning, not as concurrent reasoning!

 it is the only we
will consider in this thesis that has with property.

It's hard to include state in temporal logic! But the logical correspondence
helps us understand why: the natural SSOS specification of 



\chapter{Destination-passing}

\section{Logical transformation: destination-adding}

\section{Alternate semantics for parallelism and exceptions}

\section{First-class continuations}

\section{Exploring the richer fragment}

\subsection{Process calculus}

\subsection{First-class continuations}

\section{Why not just destinations?}

Seeing as the destination-passing semantics is the most general form
of substructural operational semantics presentation, and that it
subsumes both the ordered abstract machine semantics, it is worth
addressing the question: why not do {\it all} our work as a
destination-passing semantics? We could! But just as our goal in the
modular specification of programming languages is to make sure that
the semantics of call-by-need evaluation doesn't infect the
description of the semantics of

(Illustrate a hypothetical language development: natural numbers,
booleans, functions are specified with natural semantics, parallel
pairs, mutable state, and exceptions are specified with ordered
abstract machine semantics, and the pair/exception interface,
continations, and recursive suspensions are specified with
destination-passing semantics.)


\chapter{Linear logical approximation}

\section{Logical transformation: approximation}

\section{Control flow analysis}

\section{Alias analysis}

\part{Reasoning about substructural logical specifications}

\chapter{Case studies}

% \chapter{Programming with canonical forms}

% \newcommand{\F}[1]{\ensuremath{F({#1})}}
% \newcommand{\G}[1]{\ensuremath{G(\textcolor{TrueBlue}{#1})}}
% \newcommand{\upX}[1]{\ensuremath{{\uparrow}\textcolor{ValidBlue}{#1}}}
% \newcommand{\downX}[1]{\ensuremath{{\downarrow}\textcolor{ValidRed}{#1}}}
% \newcommand{\upA}[1]{\ensuremath{{\uparrow}\textcolor{TrueBlue}{#1}}}
% \newcommand{\downA}[1]{\ensuremath{{\downarrow}\textcolor{TrueRed}{#1}}}

% \newcommand{\valid}[1]{\ensuremath{{\downarrow}\textcolor{ValidBlue}{{#1}\,\mathit{valid}}}}
% \newcommand{\true}[1]{\ensuremath{{\downarrow}\textcolor{TrueBlue}{{#1}\,\mathit{true}}}}

% \newcommand{\ajseq}[2]{\ensuremath{\mathstrut{#1} \vdash {#2}}}
% \newcommand{\ajinv}[3]{\ajseq{{#1}; {#2}}{\textcolor{ValidRed}{#3}}}
% \newcommand{\ajrfoc}[2]{\ajseq{{#1}}{[\textcolor{ValidBlue}{#2}]}}
% \newcommand{\ajlfoc}[3]{\ajseq{{#1} [{#2}]}{\textcolor{ValidRed}{#3}}}
% \newcommand{\ajAseq}[3]{\ensuremath{\mathstrut{#1} \vdash {#2}}}
% \newcommand{\ajAinv}[4]{\ajseq{{#1}; {#2}}{\textcolor{ValidRed}{#3}}}
% \newcommand{\ajArfoc}[3]{\ajseq{{#1}}{[\textcolor{ValidBlue}{#2}]}}
% \newcommand{\ajAlfoc}[4]{\ajseq{{#1} [{#2}]}{\textcolor{ValidRed}{#3}}}
% \newcommand{\ajXseq}[2]{\ensuremath{\mathstrut{#1} \vdash {#2}}}
% \newcommand{\ajXinv}[3]{\ajseq{{#1}; {#2}}{\textcolor{ValidRed}{#3}}}
% \newcommand{\ajXrfoc}[2]{\ajseq{{#1}}{[\textcolor{ValidBlue}{#2}]}}
% \newcommand{\ajXlfoc}[3]{\ajseq{{#1} [{#2}]}{\textcolor{ValidRed}{#3}}}

% \begin{figure}
% \fbox{\ajXrfoc{\Gamma}{X^+}}
% \[
% \infer[x^+_R]
% {\ajrfoc{\valid{x^+}}{x^+}}
% {}
% \qquad
% \infer[{\downarrow}_{XR}]
% {\ajXrfoc{\Gamma}{\downX{X^-}}}
% {\ajXinv{\Gamma}{\cdot}{X^-}}
% \qquad
% \infer[G_R]
% {\ajXrfoc{\Gamma}{\G{A}}}
% {\ajArfoc{\Gamma}{\cdot}{A}}
% \]

% \fbox{\ajArfoc{\Gamma}{\Delta}{A^+}}
% \[
% \infer[a^+_R]
% {\ajArfoc{\cdot}{a^+}{a^+}}
% {}
% \qquad
% \infer[{\downarrow}_{AR}]
% {\ajArfoc{\Gamma}{\Delta}{\downA{A^-}}}
% {\ajAinv{\Gamma}{\Delta}{\cdot}{A^-}}
% \]

% \caption{Focused adjoint logic}
% \end{figure}

\chapter{Conclusion}

%\appendix
%\include{appendix}

\backmatter

%\renewcommand{\baselinestretch}{1.0}\normalsize

% By default \bibsection is \chapter*, but we really want this to show
% up in the table of contents and pdf bookmarks.
\renewcommand{\bibsection}{\chapter{\bibname}}
%\newcommand{\bibpreamble}{This text goes between the ``Bibliography''
%  header and the actual list of references}
\bibliographystyle{alpha}
\bibliography{ref} %your bib file

\end{document}
