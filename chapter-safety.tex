\chapter{Safety for substructural specifications}
\label{chapter-safety}

In Chapter~\ref{chapter-gen}, we showed how the preservation theorem
could be established for a wide variety of SSOS semantics, both
ordered abstract machines and destination-passing style semantics.
The methodology of generative invariants we espoused goes
significantly beyond previous work on proving preservation for
operational semantics specifications in substructural logic.  Neither
Linear LF encodings Pfenning, Cervesato, and Reed
\cite{cervesato02linear,reed09hybrid}, nor the Ordered LF encodings of
Felty and Momigliano \cite{felty12hybrid}, discussed preservation for
concurrent specifications or for first-class continuations. 

More fundamentally, however, this previous work does not even provide
a {\it language} for talking about progress theorems, the critical
companion of preservation theorems in reasoning about language safety.
These previous approaches were universally based on complete
derivations. These have the flavor of derivations in a big-step
semantics, and it is difficult to talk about progress for big-step
operational semantics. The purpose of this chapter is to establish that
\sls's traces and steps, which correspond to partial proofs, provide
a suitable basis for stating progress theorems (and therefore language
safety theorems) and for proving 

\section{Progress}

In the case of ordered abstract machines, we want to prove that $T ::
(x_0{:}\istrue{\susp{{\sf gen}\,\lf{{\it tp}_0}}})
\leadsto^*_{\siggenordertp} \Delta$ and $\restrictsig{\Delta}{}$ imply
one of three possibilities:

\smallskip
\begin{enumerate}
\item ${\Delta}{} \leadsto \Delta'$ under
the signature from Figure~\ref{fig:gen-order-prog}, 
\item $\Delta = y{:}\istrue{\susp{{\sf retn}\,\lf{v}}}$, or
\item $\Delta = y{:}\istrue{\susp{{\sf error}}}$.
\end{enumerate}
\smallskip

\noindent
This is exactly the form of a traditional progress theorem: if a
process state is well typed, it either takes a step under the dynamic
semantics or is a final state (terminating with an error or returning
a value).

The presence of parallel evaluation in Figure~\ref{fig:gen-order-prog}
necessitates that we generalize our induction hypothesis. The
statement above is a straightforward corollary of
Theorem~\ref{thm:prog-ordertp} below.

\bigskip
\begin{theorem}[Progress for ordered abstract machines]\label{thm:prog-ordertp}
If $T :: \tackon{\Theta}{x{:}\istrue{\susp{{\sf gen}\,\lf{{\it tp}}}}}
\leadsto_{\siggenordertp} \Delta$ and $\restrictsig{\Delta}{}$, then
either $\Delta \leadsto \Delta'$ under the signature from 
Figure~\ref{fig:gen-order-prog} for some $\Delta'$, or else 
$T = \left(T_1; \trstep{y}{{\sf gen/retn}\,\lf{{\it tp}}\,\lf{v}\,x}; T_2\right)$, or else 
$T = \left(T_1; \trstep{y}{{\sf gen/error}\,\lf{{\it tp}}\,x}; T_2\right)$.
\end{theorem}

\begin{proof}
By induction and case analysis on the first steps of $T$. We cannot 
have $T = \emptytrace$, because we cannot apply restriction to a context
with nonterminals like ${\sf gen}\,\lf{{\it tp}}$ in it. 
So $T = S; T'$, and either $x \notin {^\bullet}S$ or $x \in {^\bullet}S$. 

If $x \notin {^\bullet}S$, then $T' ::
\tackon{\Theta'}{x{:}\istrue{\susp{{\sf gen}\,\lf{{\it tp}}}}}
\leadsto_{\siggenordertp} \Delta$ and we can succeed by immediate appeal to the
induction hypothesis.

If $x \in {^\bullet}S$, then we perform case analysis on the possible
transitions enabled by $\siggenordertp$:

\medskip
\begin{itemize}
\item $S = \trstep{y}{{\sf gen/eval}\,\lf{e}\,\lf{{\it
        tp}}\,{(\tfuser{x}{\tbangr{N}})}}$
  where $\cdot \vdash N : \isconc{{\sf of}\,\lf{e}\,\lf{{\it tp}}}$.

  Because ${\sf eval}$ is a terminal, $\Delta = \tackon{\Theta'}{y{:}\istrue{\susp{{\sf eval}\,\lf{e}}}}$, and we proceed by case
  analysis on $N$ to show that the derivation can always take a step.

\medskip
\item $S = \trstep{y}{{\sf gen/retn}\,\lf{v}\,\lf{{\it tp}}\,(\tfuser{x}{\tfuser{\tbangr{N}}{\tbangr{N_v}}})}$ -- succeed immediately.

\medskip
\item $S = \trstep{y}{{\sf gen/error}\,\lf{{\it tp}}\,{x}}$ -- succeed immediately.

\medskip
\item $S = \trstep{y_1', y_2}{{\sf gen/cont}\,\lf{{\it tp}}\,\lf{f}\,\lf{{\it tp}'}\,(\tfuser{x}{\tbangr{N}})}$ where $\cdot \vdash N : \isconc{{\sf off}\,\lf{f}\,\lf{{\it tp}'}\,\lf{{\it tp}}}$.

Invoke the i.h.~on  
$T' :  \tackon{\Theta}{y_1'{:}\istrue{\susp{{\sf gen}\,\lf{{\it tp}'}}}, ~ 
                        y_2{:}\istrue{\susp{{\sf cont}\,\lf{f}}}}
\leadsto_{\siggenordertp} \Delta$, and then perform case analysis on the result
to prove that $\Delta \leadsto \Delta'$:

\begin{itemize}
\item If $\Delta \leadsto \Delta'$, then we're done. 
\item If $T' = \left(T_1'; \trstep{y_1}{{\sf gen/retn}\,\lf{{\it tp}}\,\lf{v}\,(\tfuser{y_1'}{\tfuser{\tbangr{N'}}{\tbangr{N_v'}}})}; T_2'\right)$,
\\
then because ${\sf retn}$ and ${\sf cont}$ are 
terminals, $\Delta = \tackon{\Theta'}{
         y_1{:}\istrue{\susp{{\sf retn}\,\lf{v}}}, ~
         y_2{:}\istrue{\susp{{\sf cont}\,\lf{f}}}}$, and we 
proceed by
simultaneous case analysis on $N$, $N'$, and $N_v'$ (canonical forms lemma). 
\item If $T' = \left(T_1'; \trstep{y_1}{{\sf gen/error}\,\lf{{\it tp}}\,y_1'}; T_2'\right)$,
\\
then because ${\sf error}$ and ${\sf cont}$ are terminals,
$\Delta = \tackon{\Theta'}{
         y_1{:}\istrue{\susp{{\sf error}}}, ~
         y_2{:}\istrue{\susp{{\sf cont}\,\lf{f}}}}$, and we have
$\trstep{z}{{\sf ev/error}\,\lf{f}\,(\tfuser{y_1}{y_2})} :: \Delta \leadsto \tackon{\Theta'}{z{:}\istrue{\susp{{\sf error}}}}$.

\end{itemize}

\medskip
\item $S = \trstep{y_1', y_2', y_3}{{\sf gen/cont2}\,\lf{{\it tp}}\,\lf{f}\,\lf{{\it tp}_1}\,\lf{{\it tp}_2}\,(\tfuser{x}{\tbangr{N}})}$
where $\cdot \vdash N : \isconc{{\sf off2}\,\lf{f}\,\lf{{\it tp}_1}\, \lf{{\it tp}_2}\,\lf{{\it tp}}}$.


Invoke the i.h.~twice on $T' :
\tackon{\Theta}{y_1'{:}\istrue{\susp{{\sf gen}\,\lf{{\it tp}_1}}}, ~
  y_2'{:}\istrue{\susp{{\sf gen}\,\lf{{\it tp}_2}}}, ~
  y_3{:}\istrue{\susp{{\sf cont2}\,\lf{f}}}}$, once to see what happens
to $y_1'$, and another time to see what happens to $y_2'$,
and then perform case analysis on the result to prove that
$\Delta \leadsto \Delta'$:

\begin{itemize}
\item If either invocation returns the first disjunctive possibility,
  that $\Delta \leadsto \Delta'$, then we're done.

\item
If both invocations return the second disjunctive possibility, then
 $T'$ contain two steps
$\trstep{y_1}{{\sf gen/retn}\,\lf{{\it tp}_1}\,\lf{v_1}\,(\tfuser{y_1'}{\tfuser{\tbangr{N_1}}{\tbangr{N_{v1}}}})}$ and \\
$\trstep{y_2}{{\sf gen/retn}\,\lf{{\it tp}_2}\,\lf{v_2}\,(\tfuser{y_2'}{\tfuser{\tbangr{N_2}}{\tbangr{N_{v2}}}})}$. Because ${\sf retn}$ and ${\sf cont}$
are terminals, \\$\Delta = \tackon{\Theta'}{y_1{:}\istrue{\susp{{\sf retn}\,\lf{v_1}}}, ~ y_2{:}\istrue{\susp{{\sf retn}\,\lf{v_2}}}, ~ y_3{:}\istrue{\susp{{\sf cont2}\,\lf{f}}}}$, and we proceed
by simultaneous case analysis on $N$, $N_1$, $N_{v1}$, $N_2$, and $N_{v2}$
(canonical forms lemma). 

\item In all the remaining cases, one of the subcomputations becomes
  an error and the other one becomes another error or a returned
  value.  In any of these cases, $\Delta \leadsto \Delta'$ by one of
  the rules ${\sf ev/errret}$, ${\sf ev/reterr}$, or by ${\sf
    ev/errerr}$.

\end{itemize}

\medskip
\item $S = \trstep{y_1', y_2}{{\sf gen/handle}\,\lf{{\it tp}}\,\lf{e_2}\,(\tfuser{x}{\tbangr{N}})}$.

Invoke the i.h.~on  
$T' :  \tackon{\Theta}{y_1'{:}\istrue{\susp{{\sf gen}\,\lf{{\it tp}'}}}, ~ 
                        y_2{:}\istrue{\susp{{\sf handle}\,\lf{e_2}}}}
\leadsto_{\siggenordertp} \Delta$, and then perform case analysis on the result
to prove that $\Delta \leadsto \Delta'$:

\begin{itemize}
\item If $\Delta \leadsto \Delta'$, then we're done. 
\item If $T' = \left(T_1'; \trstep{y_1}{{\sf gen/retn}\,\lf{{\it tp}}\,\lf{v}\,(\tfuser{y_1'}{\tfuser{\tbangr{N'}}{\tbangr{N_v'}}})}; T_2'\right)$,
\\
then because ${\sf retn}$ and ${\sf cont}$ are 
terminals, $\Delta = \tackon{\Theta'}{
         y_1{:}\istrue{\susp{{\sf retn}\,\lf{v}}}, ~
         y_2{:}\istrue{\susp{{\sf cont}\,\lf{f}}}}$, and 
we have $\trstep{z}{{\sf ev/catcha}\,\lf{v}\,\lf{e_2}\,(\tfuser{y_1}{y_2})}
:: \Delta \leadsto \tackon{\Theta'}{z{:}\istrue{\susp{{\sf retn}\,\lf{v}}}}$.

\item If $T' = \left(T_1'; \trstep{y_1}{{\sf gen/error}\,\lf{{\it tp}}\,y_1'}; T_2'\right)$,
\\
then because ${\sf error}$ and ${\sf cont}$ are terminals,
$\Delta = \tackon{\Theta'}{
         y_1{:}\istrue{\susp{{\sf error}}}, ~
         y_2{:}\istrue{\susp{{\sf handle}\,\lf{e}}}}$, and we have
$\trstep{z}{{\sf ev/catchb}\,\lf{e_2}\,(\tfuser{y_1}{y_2})} :: \Delta \leadsto \tackon{\Theta'}{z{:}\istrue{\susp{{\sf eval}\,\lf{e_2}}}}$.

\end{itemize}

\end{itemize}
\end{proof}

\section{Safety}