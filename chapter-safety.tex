\chapter{Safety for substructural specifications}
\label{chapter-safety}

In Chapter~\ref{chapter-gen}, we showed how the preservation theorem
could be established for a wide variety of SSOS semantics, both
ordered abstract machines and destination-passing style semantics.
The methodology of generative invariants we espoused goes
significantly beyond previous work on type preservation for
operational semantics specifications in substructural logic.  Neither
Linear LF encodings by Pfenning, Cervesato, and Reed
\cite{cervesato02linear,reed09hybrid}, nor the Ordered LF encodings of
Felty and Momigliano \cite{felty12hybrid}, discussed preservation for
concurrent specifications or for first-class continuations. 

More fundamentally, however, this previous work does not even provide
a {\it language} for talking about progress theorems, the critical
companion of type preservation theorems.  These previous approaches
were universally based on complete derivations. These have the flavor
of derivations in a big-step semantics, and it is difficult or even
impossible to talk about progress for such specifications. The purpose
of this chapter is to establish that the \sls~framework's traces $T$
and steps $S$, which correspond to partial proofs, provide a suitable
basis for stating progress theorems (and therefore language safety
theorems) and for proving these theorems.

We do not discuss progress and safety for the full range of
specifications from Part~II or Chapter~\ref{chapter-gen},
however. Instead, we will just discuss progress for two examples: the
ordered abstract machine specification with parallelism and failure
used as an example in Figure~\ref{fig:gen-order-prog}, and the
extension of this specification with mutable storage. The rest is left
for future work, though we claim that these two examples serve to get
across all the concepts necessary to prove progress theorems for SSOS
specifications. Ultimately, it is not only possible to prove progress
and safety theorems using SSOS specifications in \sls; it's also
reasonably straightforward.

\section{Backwards and forwards through traces}

In the last chapter, we worked on traces exclusively by induction and
case analysis on the {\it last} steps of a generative trace.  This
form of case analysis and induction on the last steps of a trace can
also be used to prove progress for sequential SSOS specifications, and
it is actually necessary to prove progress for SSOS specifications
with first-class continuations (discussed in
Section~\ref{sec:dest-continuations} and Section~\ref{sec:gen-letcc})
in this way, though we leave the details of this argument as future
work. However, for the ordered abstract machine example from
Figure~\ref{fig:gen-order-prog}, the other direction is more natural:
we work by induction and case analysis on the {\it first} steps of a
generative trace. The branching structure introduced by parallel
continuation frames (that is, ordered propositions ${\sf
  cont2}\,\lf{f}$) is what makes it more natural to work from the
beginning of a generative trace, rather than the end.

The proof of progress relies critically on one novel property: that
transitions in the generative trace do not tamper with terminals.
Formally, we need to know that if $\tackon{\Theta}{\Delta}
\leadsto^*_{\Sigma{\it Gen}} \Delta'$ under some generative signature
$\Sigma{\it Gen}$ and if $\Delta$ contains only terminals, then there
is some $\Theta'$ such that the final state $\Delta'$ matches
$\tackon{\Theta'}{\Delta}$. We will implicitly use this property in
most of the cases of the progress theorem below.

This property holds for all the generative signatures in
Chapter~\ref{chapter-gen}, but establishing this property for
generative signatures {\it in general} necessitates a further
restriction of what counts as a generative signature
(Definition~\ref{def:gensig}). To see why, let $\Delta =
(x_1{:}\istrue{\susp{{\sf retn}\,\lf{v}}}, x_2{:}\istrue{\susp{{\sf
      cont}\,\lf{f}}})$ and consider the generative rule $\forall
\lf{e}. \{ {\sf eval}\,\lf{e} \}$, which is allowed under
Definition~\ref{def:gensig}. This rule could ``break'' the context by
dropping an ordered ${\sf eval}\,\lf{e}$ proposition in between $x_1$
and $x_2$. A sufficient general condition for avoiding this problem in
general is to demand that any generative rule that produces ordered
atomic propositions mentions an ordered nonterminal as a premise.
(This is related to the property called {\it separation} in 
\cite{simmons08linear}.)

\section{Progress for ordered abstract machines}

The progress property is that, if $T ::
(x_0{:}\istrue{\susp{{\sf gen}\,\lf{{\it tp}_0}}})
\leadsto^*_{\siggenordertp} \Delta$ and $\restrictsig{\Delta}{}$, then
one of these three possibilities hold:

\smallskip
\begin{enumerate}
\item ${\Delta}{} \leadsto \Delta'$ under
the signature from Figure~\ref{fig:gen-order-prog}, 
\item $\Delta = y{:}\istrue{\susp{{\sf retn}\,\lf{v}}}$, where
  $\lf{v}$ is a value, or
\item $\Delta = y{:}\istrue{\susp{{\sf error}}}$.
\end{enumerate}
\smallskip

\noindent
This is exactly the form of a traditional progress theorem: if a
process state is well typed, it either takes a step under the dynamic
semantics or is a final state (terminating with an error or returning
a value).

The presence of parallel evaluation in Figure~\ref{fig:gen-order-prog}
necessitates that we generalize our induction hypothesis. The
statement above is a straightforward corollary of
Theorem~\ref{thm:prog-ordertp} below.

\bigskip
\begin{theorem}[Progress for ordered abstract machines]\label{thm:prog-ordertp}
If $T :: \tackon{\Theta}{x{:}\istrue{\susp{{\sf gen}\,\lf{{\it tp}}}}}
\leadsto_{\siggenordertp} \Delta$ and $\restrictsig{\Delta}{}$, then
either 
\begin{itemize}
\item 
$\Delta \leadsto \Delta'$ under the signature from 
Figure~\ref{fig:gen-order-prog} for some $\Delta'$, or else
\item 
$T = \left(T_1; \trstep{y}{{\sf gen/retn}\,\lf{{\it tp}}\,\lf{v}\,x}; T_2\right)$ and $\cdot \vdash N : \isconc{{\sf value}\,\lf{v}}$, or else 
\item
$T = \left(T_1; \trstep{y}{{\sf gen/error}\,\lf{{\it tp}}\,x}; T_2\right)$.
\end{itemize}
\end{theorem}

\bigskip

In the proof of Theorem~\ref{thm:prog-ordertp}, we will not consider
the details of the proof that already arise for traditional proofs of
progress for abstract machines. These missing details can be factored
into two lemmas. The first lemma is that if $\cdot \vdash N :
\ispers{{\sf of}\,\lf{e}\,\lf{{\it tp}}}$, then the process state
$\tackon{\Theta}{x{:}\istrue{\susp{{\sf eval}\,\lf{e}}}}$ can always
take a step; this lemma justifies the classification of ${\sf eval}$
as an {\it active} proposition as described in
Section~\ref{sec:dest-synch} and in
\cite{pfenning09substructural}. The second lemma is traditionally
called a {\it canonical forms} lemma: it verifies, by case analysis on
the structure of typing derivations and values, that well-typed
values returned to a well-typed frames can always take a step.

\begin{proof}
By induction and case analysis on the first steps of $T$. We cannot 
have $T = \emptytrace$, because we cannot apply restriction to a context
containing the nonterminal ${\sf gen}\,\lf{{\it tp}}$.
So $T = S; T'$, and either $x \notin {^\bullet}S$ or $x \in {^\bullet}S$. 

If $x \notin {^\bullet}S$, then $T' ::
\tackon{\Theta'}{x{:}\istrue{\susp{{\sf gen}\,\lf{{\it tp}}}}}
\leadsto_{\siggenordertp} \Delta$ and we can succeed by immediate appeal to the
induction hypothesis.

If $x \in {^\bullet}S$, then we perform case analysis on the possible
transitions enabled by $\siggenordertp$:

\begin{itemize}
\item $S = \trstep{y}{{\sf gen/eval}\,\lf{e}\,\lf{{\it
        tp}}\,{(\tfuser{x}{\tbangr{N}})}}$
  where $\cdot \vdash N : \isconc{{\sf of}\,\lf{e}\,\lf{{\it tp}}}$.

  Because ${\sf eval}$ is a terminal, $\Delta = \tackon{\Theta'}{y{:}\istrue{\susp{{\sf eval}\,\lf{e}}}}$, and we proceed by case
  analysis on $N$ to show that the derivation can always take a step
  (${\sf eval}$ is an active proposition).

\medskip
\item $S = \trstep{y}{{\sf gen/retn}\,\lf{{\it tp}}\,\lf{v}\,(\tfuser{x}{\tfuser{\tbangr{N}}{\tbangr{N_v}}})}$ -- succeed immediately.

\medskip
\item $S = \trstep{y}{{\sf gen/error}\,\lf{{\it tp}}\,{x}}$ -- succeed immediately.

\medskip
\item $S = \trstep{y_1', y_2}{{\sf gen/cont}\,\lf{{\it tp}}\,\lf{f}\,\lf{{\it tp}'}\,(\tfuser{x}{\tbangr{N}})}$ where $\cdot \vdash N : \isconc{{\sf off}\,\lf{f}\,\lf{{\it tp}'}\,\lf{{\it tp}}}$.

Invoke the i.h.~on  
$T' :  \tackon{\Theta}{y_1'{:}\istrue{\susp{{\sf gen}\,\lf{{\it tp}'}}}, ~ 
                        y_2{:}\istrue{\susp{{\sf cont}\,\lf{f}}}}
\leadsto_{\siggenordertp} \Delta$, and then perform case analysis on the result
to prove that $\Delta \leadsto \Delta'$:

\begin{itemize}
\item If $\Delta \leadsto \Delta'$, then we're done. 
\item If $T' = \left(T_1'; \trstep{y_1}{{\sf gen/retn}\,\lf{{\it tp}'}\,\lf{v}\,(\tfuser{y_1'}{\tfuser{\tbangr{N'}}{\tbangr{N_v'}}})}; T_2'\right)$,
\\
then because ${\sf retn}$ and ${\sf cont}$ are 
terminals, $\Delta = \tackon{\Theta'}{
         y_1{:}\istrue{\susp{{\sf retn}\,\lf{v}}}, ~
         y_2{:}\istrue{\susp{{\sf cont}\,\lf{f}}}}$, and we 
proceed by
simultaneous case analysis on $N$, $N'$, and $N_v'$ (canonical forms lemma). 
\item If $T' = \left(T_1'; \trstep{y_1}{{\sf gen/error}\,\lf{{\it tp}'}\,y_1'}; T_2'\right)$,
\\
then because ${\sf error}$ and ${\sf cont}$ are terminals,
$\Delta = \tackon{\Theta'}{
         y_1{:}\istrue{\susp{{\sf error}}}, ~
         y_2{:}\istrue{\susp{{\sf cont}\,\lf{f}}}}$, and we have
$\trstep{z}{{\sf ev/error}\,\lf{f}\,(\tfuser{y_1}{y_2})} :: \Delta \leadsto \tackon{\Theta'}{z{:}\istrue{\susp{{\sf error}}}}$.

\end{itemize}

\medskip
\item $S = \trstep{y_1', y_2', y_3}{{\sf gen/cont2}\,\lf{{\it tp}}\,\lf{f}\,\lf{{\it tp}_1}\,\lf{{\it tp}_2}\,(\tfuser{x}{\tbangr{N}})}$
where $\cdot \vdash N : \isconc{{\sf off2}\,\lf{f}\,\lf{{\it tp}_1}\, \lf{{\it tp}_2}\,\lf{{\it tp}}}$.


Invoke the i.h.~twice on $T' :
\tackon{\Theta}{y_1'{:}\istrue{\susp{{\sf gen}\,\lf{{\it tp}_1}}}, ~
  y_2'{:}\istrue{\susp{{\sf gen}\,\lf{{\it tp}_2}}}, ~
  y_3{:}\istrue{\susp{{\sf cont2}\,\lf{f}}}}$, once to see what happens
to $y_1'$, and another time to see what happens to $y_2'$,
and then perform case analysis on the result to prove that
$\Delta \leadsto \Delta'$:

\begin{itemize}
\item If either invocation returns the first disjunctive possibility,
  that $\Delta \leadsto \Delta'$, then we're done.

\item
If both invocations return the second disjunctive possibility, then
 $T'$ contain two steps
$\trstep{y_1}{{\sf gen/retn}\,\lf{{\it tp}_1}\,\lf{v_1}\,(\tfuser{y_1'}{\tfuser{\tbangr{N_1}}{\tbangr{N_{v1}}}})}$ and \\
$\trstep{y_2}{{\sf gen/retn}\,\lf{{\it tp}_2}\,\lf{v_2}\,(\tfuser{y_2'}{\tfuser{\tbangr{N_2}}{\tbangr{N_{v2}}}})}$. Because ${\sf retn}$ and ${\sf cont2}$
are terminals, \\$\Delta = \tackon{\Theta'}{y_1{:}\istrue{\susp{{\sf retn}\,\lf{v_1}}}, ~ y_2{:}\istrue{\susp{{\sf retn}\,\lf{v_2}}}, ~ y_3{:}\istrue{\susp{{\sf cont2}\,\lf{f}}}}$, and we proceed
by simultaneous case analysis on $N$, $N_1$, $N_{v1}$, $N_2$, and $N_{v2}$
(canonical forms lemma). 

\item In all the remaining cases, one of the subcomputations becomes
  an error and the other one becomes another error or a returned
  value.  In any of these cases, $\Delta \leadsto \Delta'$ by one of
  the rules ${\sf ev/errret}$, ${\sf ev/reterr}$, or by ${\sf
    ev/errerr}$.

\end{itemize}

\medskip
\item $S = \trstep{y_1', y_2}{{\sf gen/handle}\,\lf{{\it tp}}\,\lf{e_2}\,(\tfuser{x}{\tbangr{N}})}$.

Invoke the i.h.~on  
$T' :  \tackon{\Theta}{y_1'{:}\istrue{\susp{{\sf gen}\,\lf{{\it tp}'}}}, ~ 
                        y_2{:}\istrue{\susp{{\sf handle}\,\lf{e_2}}}}
\leadsto_{\siggenordertp} \Delta$, and then perform case analysis on the result
to prove that $\Delta \leadsto \Delta'$:

\begin{itemize}
\item If $\Delta \leadsto \Delta'$, then we're done. 
\item If $T' = \left(T_1'; \trstep{y_1}{{\sf gen/retn}\,\lf{{\it tp}'}\,\lf{v}\,(\tfuser{y_1'}{\tfuser{\tbangr{N'}}{\tbangr{N_v'}}})}; T_2'\right)$,
\\
then because ${\sf retn}$ and ${\sf cont}$ are 
terminals, $\Delta = \tackon{\Theta'}{
         y_1{:}\istrue{\susp{{\sf retn}\,\lf{v}}}, ~
         y_2{:}\istrue{\susp{{\sf cont}\,\lf{f}}}}$, and 
we have $\trstep{z}{{\sf ev/catcha}\,\lf{v}\,\lf{e_2}\,(\tfuser{y_1}{y_2})}
:: \Delta \leadsto \tackon{\Theta'}{z{:}\istrue{\susp{{\sf retn}\,\lf{v}}}}$.

\item If $T' = \left(T_1'; \trstep{y_1}{{\sf gen/error}\,\lf{{\it tp}'}\,y_1'}; T_2'\right)$,
\\
then because ${\sf error}$ and ${\sf cont}$ are terminals,
$\Delta = \tackon{\Theta'}{
         y_1{:}\istrue{\susp{{\sf error}}}, ~
         y_2{:}\istrue{\susp{{\sf handle}\,\lf{e}}}}$, and we have
$\trstep{z}{{\sf ev/catchb}\,\lf{e_2}\,(\tfuser{y_1}{y_2})} :: \Delta \leadsto \tackon{\Theta'}{z{:}\istrue{\susp{{\sf eval}\,\lf{e_2}}}}$.

\end{itemize}

\end{itemize}

\noindent
This covers all possible first steps in the trace $T$, and thus completes
the proof.
\end{proof}

\section{Progress with mutable storage}
\label{sec:progess-mutable}

Developing progress proofs to for stateful specifications requires a
property that is the flip side of unique index sets
(Definition~\ref{def:uniqueindexset},
Section~\ref{sec:uniqueness-gendests}). Unique index sets require that there
will be only ever be {\it at most one} proposition of a certain form, 
and the dual property, {\it assured index sets}, require that there is always
{\it at least one} proposition of a certain form. 

\bigskip
\begin{definition}
  A set $S$ is an {\em assured index set} at a type $\tau$ under a
  generative signature $\Sigma$ and an initial state $(\Psi; \Delta)$
  if, whenever $(\Psi; \Delta) \leadsto^*_{\Sigma} (\Psi'; \Delta')$,
  then $\Psi \vdash_\Sigma \lf{t} : \tau$ implies that, for some ${\sf
    a}/i \in S$, $x{:}\islvl{\susp{{\sf a}\,\lf{t_1}\ldots\lf{t_n}}}
  \in \Delta'$ where $\lf{t_i} = \lf{t}$. 
\end{definition}
\bigskip

The set $\{ {\sf gencell}/1, {\sf cell}/1 \}$ is also a unique index
set under $\siggenstate$, and the same set for $\siggenstate$ at type
${\sf mutable\_loc}$.  The latter property is critical to finishing
the extension of Theorem~\ref{thm:prog-ordertp} proof in certain cases
where we invoke the canonical forms lemma. When we invoked the
canonical forms lemma in the ${\sf cont}$ branch of that proof, we
started with the knowledge that $\Delta = \tackon{\Theta'}{
  y_1{:}\istrue{\susp{{\sf retn}\,\lf{v}}}, ~ y_2{:}\istrue{\susp{{\sf
        cont}\,\lf{f}}}}$. Two new outcomes are introduced when we
introduce mutable state as discussed in
Section~\ref{sec:mutable-storage} and Section~\ref{sec:gen-state}.
The first is the possibility that $\lf{v} = \lf{{\sf loc}\,l}$ while
$\lf{f} = \lf{\sf get1}$, and the second is the possibility that
$\lf{f} = \lf{{\sf set2}\,l}$. In each case, we cannot proceed with
rule ${\sf ev/get1}$ or rule ${\sf ev/set2}$, respectively, unless we
also know that there is a variable binding $z{:}{\sf
  cell}\,\lf{l}\,\lf{v'}$ in $\Delta$. We know precisely this because
$\{ {\sf gencell}/1, {\sf cell}/1 \}$ is an assured index set, because
$\Psi \vdash \lf{l} {:} {\sf mutable\_loc}$, and because ${\sf
  gencell}$ propositions, as nonterminals, cannot appear in the
generated process state $\Delta$. Therefore, in the former case we can
produce a step $\trstep{y', z'}{{\sf
    ev/get1}\,\lf{l}\,\lf{v'}\,(\tfuser{y_1}{\tfuser{y_2}{z}})}$, and
in the later case we can produce a step $\trstep{y', z'}{{\sf
    ev/set2}\,\lf{v}\,\lf{l}\,\lf{v'}\,(\tfuser{y_1}{\tfuser{y_2}{z}})}$.

A similar use of assured index sets arises if we try to prove progress
for a specification using letcc (Section~\ref{sec:dest-continuations}
and Section~\ref{sec:gen-letcc}). We begin with a destination
$\lf{d}$ and need to use the fact that $\{ {\sf gendest}/1, {\sf
  cont}/2 \}$ is an assured index set to confirm that we have some
continuation frame to return to. 

\section{Safety}

We conclude by presenting the safety theorem for the ordered abstract
machine specification from Figure~\ref{fig:gen-order-prog}. This
theorem relates the encoding of the usual deductive formulation of the
typing judgment, ${\sf of}\,\lf{e}\,\lf{{\it tp}}$, to a progress
property state in terms of substructural process states.

\bigskip
\begin{theorem}[Safety for ordered abstract machines]~
If $T :: (x{:}\istrue{\susp{{\sf eval}\,e}}) \leadsto^* \Delta$
under the signature from Figure~\ref{fig:gen-order-prog} and
$\cdot \vdash N : {\sf of}\,\lf{e}\,\lf{{\it tp}}$,
then either 
\begin{itemize}
\item $\Delta \leadsto \Delta'$ under the signature from
Figure~\ref{fig:gen-order-prog} for some $\Delta'$, or else
\item $\Delta = (y{:}\istrue{\susp{{\sf retn}\,\lf{v}}})$ and $\cdot \vdash N : {\sf value}\,\lf{v}$, or else
\item $\Delta = (y{:}\istrue{\susp{{\sf error}}})$.
\end{itemize}
\end{theorem}

\begin{proof}

  First, by induction and case analysis on the last steps of $T$, we show
  that for all $\Delta'$ such that $T' :: (x{:}\istrue{\susp{{\sf
        eval}\,\lf{e}}}) \leadsto^* \Delta'$ under the signature
  from Figure~\ref{fig:gen-order-prog}, we can construct
  a generative trace $T_g :: (x_0{:}\istrue{\susp{{\sf
        gen}\,\lf{{\it tp}}}}) \leadsto^*_{\siggenordertp} \Delta'$:

  \bigskip
  \noindent
  {\bf Base case $T' = \emptytrace$}.

  \smallskip
  \noindent
  Construct
  $T_g = \trstep{x}{{\sf gen/eval}\,\lf{{\it tp}}\,\lf{e}\,(\tfuser{x_0}{\tbangr{N}})}:: (x_0{:}\istrue{\susp{{\sf gen}\,\lf{{\it tp}}}})
  \leadsto^*_{\siggenordertp} (x{:}\istrue{\susp{{\sf
        eval}\,\lf{e}}})$.

  \bigskip
  \noindent
  {\bf Inductive case $T' = T''; S$}, where $T'' :: (x_0{:}\istrue{\susp{{\sf gen}\,\lf{{\it tp}}}}) \leadsto^*_{\siggenordertp} \Delta''$ and $S :: \Delta'' \leadsto_{\siggenordertp} \Delta'$. 

  \smallskip
  \noindent
  By the induction hypothesis, we have $T_g' :: (x_0{:}\istrue{\susp{{\sf gen}\,\lf{{\it tp}}}}) \leadsto_{\siggenordertp}^*
  \Delta''$. By preservation (Theorem~\ref{thm:siggenordertp}) on 
  $T_g'$ and $S$, we have $T_g' :: (x_0{:}\istrue{\susp{{\sf gen}\,\lf{{\it tp}}}}) \leadsto_{\siggenordertp}^*
  \Delta'$.

  \bigskip
  \noindent
  This means, in particular, that we can construct 
  $T_g :: (x_0{:}\istrue{\susp{{\sf gen}\,\lf{{\it tp}}}}) \leadsto_{\siggenordertp}^*
  \Delta$. 

  \bigskip
  \noindent
  By the progress theorem (Theorem~\ref{thm:prog-ordertp}) on $T_g$, 
  there are three possibilities:
  \begin{itemize}
  \item If $\Delta \leadsto \Delta'$, then we're done.
  \item If $T_g :: (T_1; \trstep{y}{{\sf gen/retn}\,\lf{{\it tp}}\,\lf{v}\,(\tfuser{x_0}{\tfuser{\tbangr{N'}}{\tbangr{N_v'}}})}; T_2)$, then by a trivial case analysis on $T_1$ and $T_2$ we can conclude that both are empty and, therefore, that $\Delta = (y{:}\istrue{\susp{{\sf retn}\,\lf{v}}})$.
  \item If $T_g :: (T_1; \trstep{y}{{\sf gen/error}\,\lf{{\it tp}}\,x_0}; T_2)$, then by a trivial case analysis on $T_1$ and $T_2$ we can conclude that both are empty and, therefore, that $\Delta = (y{:}\istrue{\susp{{\sf error}}})$.
  \end{itemize}

  \noindent
  This concludes the proof of safety.
\end{proof}