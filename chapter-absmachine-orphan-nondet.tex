\section{Nondeterminism}
\label{sec:absmachine-nondeterminism}

The examples given in the previous section have given so far all deal
with call-by-value semantics for the untyped lambda calculus, which
has the property that any expression will either evaluate forever or
will eventually evaluate to a value $\lambda x. e$. One of the
advantages of the operationalization transformation, however, is that
it allows us to


\subsection{Arbitrary choice and failure}

For the purposes of illustration, we will extend the language of
expressions with a nondeterministic choice operator $e_1 \arb e_2$
(encoded in LF as ${\sf choose}\,\interp{e_1}\,\interp{e_2}$) and an
extra expression ${\sf junk}$ that does not evaluate to anything: this
ensures that stuck states are possible, as search for a $v$ such that
$(\lambda x.\,x)\,{\sf junk} \Downarrow v$ is derivable will fail.
The natural semantics rules for this extension and their
(tail-recursion optimized) operationalization are shown in
Figure~\ref{fig:ns-arb}; there are no rules for ${\sf junk}$, as it is
a stuck expression.

\begin{figure}[t]
\begin{minipage}[b]{0.45\linewidth}
\fvset{fontsize=\small,boxwidth=auto}
\VerbatimInput{sls/cbv-arb.sls}
\end{minipage}
\hspace{0.5cm}
\begin{minipage}[b]{0.55\linewidth}
\fvset{fontsize=\small,boxwidth=auto}
\VerbatimInput{sls/cbv-arb-ssos.sls}
\end{minipage}
\caption{Natural semantics for the lambda calculus (call-by-value).}
\label{fig:ns-arb}
\end{figure}

We need to think about the desired semantics of $(\lambda y.\,y\,y)
\arb ((\lambda x.\,x)\,{\sf junk})$ -- if the first subexpression
$(\lambda y.\,y\,y)$ is chosen for evaluation, then the expression
evaluates to a value, but if the second subexpression $(\lambda
x.\,x)\,{\sf junk}$ is chosen, then the subexpression will evaluate to
${\sf junk}$ and be effectively stuck. Our small-step intutitions
about language safety say that this is a problem we should avoid, but
natural semantics do not have an internal way of expressing that a
successful evaluation exists along some paths but not along others.
This is one of the ways in which our transformed specification is more
expressive than the original semantics: it's possible to say
$x{:}\susp{{\sf eval}\,\interp{(\lambda y.\,y\,y) \arb ((\lambda
    x.\,x)\,{\sf junk})}} \leadsto^* \Delta$ such that $\Delta \neq
y{:}\susp{{\sf retn}\,V}$ and $\Delta \not\leadsto$; we can't hope to
reason about the absense of stuck states if we cannot express
them. Our ability to reason about stuck concurrent computations is an
artifact the fact that \sls~allows us to talk about traces $T$ in
addition to talking about complete proofs.  The operationalization
transformation allows us to transfer this concept to natural semantics
specifications by way of translation.

\begin{figure}[t]
\fvset{fontsize=\small,boxwidth=auto}
\VerbatimInput{sls/cbv-bad-ite.sls}
\caption{Operationalization of the problematic natural semantics for {\it if}-statements.}
\label{fig:sls-bad-ite}
\end{figure}

\subsection{Conditionals and factoring}

Another extension that we can introduce complicates this view. A
common way of giving a natural semantics for {\it if}-statements is
through these two rules:
\[
\infer
{{\sf true} \Downarrow {\sf true} \mathstrut}
{}
\quad
\infer
{{\sf false} \Downarrow {\sf false} \mathstrut}
{}
\quad
\infer
{{\sf if}\,e\,{\sf then}\,e_t\,{\sf else}\,e_f \Downarrow v \mathstrut}
{e \Downarrow {\sf true}
 &
 e_t \Downarrow v \mathstrut}
\quad
\infer
{{\sf if}\,e\,{\sf then}\,e_t\,{\sf else}\,e_f \Downarrow v \mathstrut}
{e \Downarrow {\sf false}
 &
 e_f \Downarrow v \mathstrut}
\]
However, according to the straightfoward operationalization of these
rules (shown in Figure~\ref{fig:sls-bad-ite}), we would be forced,
when we first encounter an {\it if}-statement, to nondeterminstically
choose whether the conditional will evaluate to ${\sf true}$ or ${\sf
  false}$, just as we nondeterminstically chose one of the two
expressions $e_1 \arb e_2$. If we make the wrong choice (and, assuming
$e$ evaluates to anything at all, we always {\it can} make the wrong
choice) then the concurrent computation will get stuck. Superficially,
the following trace is indistinguishable from the stuck trace that
arises from the previous example that was nondeterminstically ${\sf
  junk}$, but there is nothing wrong with including {\it if}-statements
in the language!
\begin{align*}
 & x_1{:}\susp{{\sf eval}\,\interp{{\sf if}\,{\sf false}\,{\sf then}\,e_t\,{\sf else}\,e_f}}
\\
\leadsto ~ & 
x_2{:}\susp{{\sf eval}\,{\sf false}},
y_2{:}\istrue{({\sf retn}\,{\sf true} \lefti \{ {\sf eval}\,\interp{e_t}\})}
\\
\leadsto ~ &
x_3{:}\susp{{\sf retn}\,{\sf false}},
y_2{:}\istrue{({\sf retn}\,{\sf true} \lefti \{ {\sf eval}\,\interp{e_t}\})}
\\
\not\leadsto ~ & (!!!)
\end{align*}


\begin{figure}[t]
\fvset{fontsize=\small,boxwidth=auto}
\VerbatimInput{sls/cbv-ok-ite.sls}
\caption{Operationalization of the fixed, determinstic natural semantics for {\it if}-statements.}
\label{fig:sls-ok-ite}
\end{figure}

It is highly desirable to maintain our ability to talk about stuck
states through the natural semantics; the easiest way to do so is 
by modfying the natural semantics into a different natural semantics
that is equivalent but more operational:
\[
\infer
{{\sf pick}\,{\sf true}\,e_t\,e_f\,e_t\mathstrut}
{}
\quad
\infer
{{\sf pick}\,{\sf false}\,e_t\,e_f\,e_f\mathstrut}
{}
\quad
\infer
{{\sf true} \Downarrow {\sf true} \mathstrut}
{}
\quad
\infer
{{\sf false} \Downarrow {\sf false} \mathstrut}
{}
\quad
\infer
{{\sf if}\,e\,{\sf then}\,e_t\,{\sf else}\,e_f \Downarrow v' \mathstrut}
{e \Downarrow v
 & 
 {\sf pick}\,v\,e_t\,e_f\,e'\mathstrut
 & e' \Downarrow v'}
\] 
We can prove, with the existing metatheoretic machinery of Twelf, that this
natural semantics is equivalent to the previous one in terms of 
which judgments $e \Downarrow v$ are derivable, but the operationalization
of this version, given in Figure~\ref{fig:sls-ok-ite}, is free
of the previous version's problems.

These two different versions of the same rule are related by a
transformation called {\it factoring}. While factoring has been
expressed by Polakow as a transformation on functional programs in a
variant of the Delphin programming language
\cite{poswolsky03factoring}, I am unaware of any treatment of the
factoring transformation as a generally-correct {\it logical}
transformation on Prolog, $\lambda$Prolog, or Twelf specifications,
and I will not do so here.

\subsection{Flat resolution for deductive computation}

The consideration of non-backtracking search is an interesting one,
because it indicates that, {\it in the context of natural semantics},
we are interested in an interpretation of deductive computation that is
not the usual backtracking interpretation.\footnote{This committed
  choice form of deductive computation also goes by the name {\it flat
    resolution} \cite{aitkaci99warrens}.} It is natural, then, to
represent this computation as a concurrent computation, which is
naturally interpreted as committed choice. 

This works both ways, however: in Section~\ref{sec:evaluationcontexts}
we transform a small-step structural operational semantics to obtain
an SLS analouge of evaluation contexts. As a deductive comptuation,
SOS specifications are naturally backtracking, so the concurrent
computations we derive also need to be interpreted with backtracking
search if we want to preserve the operational behavior of the
specification.

