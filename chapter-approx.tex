\chapter{Linear logical approximation}
\label{chapter-approx}

The abstract interpretation of programs relates the exact semantics of
a programming language to a finite approximation of those semantics. A
general recipe for constructing a sound program analysis is to (1)
specify the operational semantics of the underlying programming
language via an interpreter, and (2) specify a terminating
approximation of the interpreter itself. This is the basic idea behind
{\it abstract interpretation} \cite{cousot77abstract}, which provides
techniques for constructing approximations (for example, by exhibiting
a Galois connection between concrete and abstract domains). The
correctness proof must establish the appropriate relationship between
the concrete and abstract computations and show that the abstract
computation terminations. We need to vary both the specification of
the operational semantics and the form of the approximation in order
to obtain various kinds of program analyses, sometimes with
considerable ingenuity.

In this chapter, which is mostly derived from \cite{simmons11logical},
we consider a new class of instances in the general schema of abstract
interpretation that is based on the approximation of SSOS
specifications in \sls. We apply logically justified techniques for
manipulating and approximating SSOS specifications to yield
approximations that are correct by construction. The resulting
persistent logical specifications can be interpreted as saturating
logic programs, generalizing proposals of McAllester and Ganzinger
\cite{mcallester02complexity,ganzinger02logical} to specifications
that use higher-order abstract syntax.

\section{Saturating logic programming}

Concurrent \sls~specifications where all positive atomic propositions
are persistent (and where all inclusions of negative propositions in
positive propositions -- if there are any -- have the form ${!}A$, not
${\downarrow}A^-$ or ${\gnab}A^-$) have a distinct logical and
operational character. Logically, by the discussion in
Section~\ref{sec:perm-fragments} we are justified in reading such
specifications as specifications in persistent intuitionistic logic or
persistent lax logic. Operationally, while persistent specifications
have an interpretation as transition systems, that interpretation is
not very useful. This is because if we can take a transition once --
for instance, using the rule ${\sf a} \lefti \{ {\sf b} \}$ to derive
the persistent atomic proposition ${\sf b}$ from the persistent atomic
proposition ${\sf a}$ -- none of the facts that enabled that
transition can be consumed, as all facts are persistent. Therefore, we
can continue to make the same transition indefinitely; in the
above-mentioned example, such transitions will derive multiple
redundant copies of ${\sf b}$.

The way we will understand the meaning of persistent and concurrent
\sls~specifications is in terms of {\it saturation}. A process state
$(\Psi; \Delta)$ is saturated relative to the signature $\Sigma$ if,
for any step $(\Psi; \Delta) \leadsto_\Sigma (\Psi'; \Delta')$, it is
the case that $\Psi$ and $\Psi'$ are the same, $x{:}\susp{p^+_\mpers}
\in \Delta'$ implies $x{:}\susp{p^+_\mpers} \in \Delta$, and
$x{:}\ispers{A^-} \in \Delta'$ implies $x{:}\ispers{A^-} \in
\Delta$. This means that a signatures with a rule that
  produces new variables by existential quantification, like ${\sf a}
  \lefti \{ \exists \lf{x}. {\sf b}\lf{(x)} \}$ has no saturated process states
  where ${\sf a}$ is present. Notions of saturation that can cope with
  existentially generated parameters are interesting, but are beyond
  the scope of this thesis.

A {\it minimal} saturated process state
is one with no duplicated propositions; we can compute a process state
from any saturated process state by removing duplicates. For purely
persistent specifications and process states, minimal saturated process
states are unique when they exist: if $(\Psi; \Delta)
\leadsto^*_\Sigma (\Psi_1; \Delta_1)$ and $(\Psi; \Delta)
\leadsto^*_\Sigma (\Psi_2; \Delta_2)$ and both $(\Psi_1; \Delta_1)$
and $(\Psi_2; \Delta_2)$ are saturated, then $(\Psi_1; \Delta_1)$
and $(\Psi_2; \Delta_2)$ have minimal process states that differ only
in the names of variables.

Furthermore, if a saturated process state exists for a given initial
process state, the minimal saturated process state can be computed by
the usual forward-chaining semantics where only transitions that
derive ${\it new}$ persistent atomic propositions or equalities $\lf{t}
\doteq \lf{s}$ are allowed. This forward-chaining logic programming
interpretation of persistent logic is extremely common; in fact, it is
what is commonly meant by ``forward-chaining logic programming.'' Just
as the term {\it persistent logic} was introduced in Chapter 2 to
distinguish what is traditionally referred to as intuitionistic logic
from intuitionistic ordered and linear logic, we will use the term
{\it saturating logic programming} to distinguish what is
traditionally referred to as forward-chaining logic programming from
the forward-chaining logic programming interpretation that makes sense
for ordered and linear logical specifications.

\section{Logical transformation: approximation}
\label{sec:abstraction}

Our approximation strategy is simple: a signature in an ordered or
linear logical specification can be approximated by making all atomic
propositions persistent, and a flat, persistent rule $\forall
\lf{\overline{x}}. \,A^+ \lefti \{ B^+ \}$ can be further approximated by
removing premises from $A^+$ and adding conclusions to $B^+$. Of
particular practical importance are added conclusions that equate the
parameters introduced by existential quantification with terms: all
parameters introduced by existential quantification must be dealt with
as a necessary condition for interpreting a persistent signature as a
saturating logic program.

First, we define what it means for a specification to be an approximate
version of another specification:

\bigskip
\begin{definition}\label{def:approxversion}
  A flat, concurrent, and persistent specification $\Sigma_a$ is an
  {\em approximate version} of another specification $\Sigma$ if every
  predicate ${\sf a} : \Pi \lf{x_1}{:}\tau_1 \ldots \Pi \lf{x_n}{:}\tau_n.\,
  {\sf prop}\,{\sf lvl}$ declared in $\Sigma$ has a corresponding
  predicate ${\sf a} : \Pi \lf{x_1}{:}\tau_1 \ldots \Pi \lf{x_n}{:}\tau_n.\,
  {\sf prop}\,{\sf pers}$ in $\Sigma_a$ and if for every rule ${\sf
    r} : \forall \overline{x}.\,A_1^+ \lefti \{ A_2^+ \}$ in $\Sigma$ there
  is a corresponding rule ${\sf r} : \forall \lf{\overline{x}}.\,B_1^+ \lefti
  \{ B_2^+ \}$ in $\Sigma_a$ such that:
  \begin{itemize}
  \item The existential quantifiers in $A_1^+$ and $A_2^+$ are
    identical to the existential quantifiers in $B_1^+$ and $B_2^+$
    (respectively),
  \item For each premise ($p^+_\mpers$ or $\lf{t} \doteq \lf{s}$) in $B^+_1$,
    the same premise appears in $A^+_1$, and 
  \item For each conclusion ($p^+_\mlvl$ or $\lf{t}
    \doteq \lf{s}$) in $A^+_2$, the same premise appears in $B^+_2$.
  \end{itemize}
\end{definition}
\bigskip

\noindent
While approximation is a program transformation, it is not a
deterministic one. Even the nondeterministic operationalization
transformation was just a bit nondeterministic, giving several options
for operationalizing any given deductive rule. The approximation
transformation, in contrast, needs explicit information from the user:
which premises should be removed, and what new conclusions should be
introduced? While there is value in actually implementing the
operationalization, defunctionalization, and destination-adding
transformations, applying approximation requires
intelligence. Borrowing a phrase from Danvy, approximation is a
candidate for ``mechanization by graduate student'' rather than
mechanization by computer.

Next, we give a definition of what it means for a state to be an 
approximate version (we use the word ``generalization'') of another state
or a family of states. 

\bigskip
\begin{definition}
  The persistent process state state $(\Psi_g; \Delta_g)$ is a {\em
    generalization} of the process state $(\Psi; \Delta)$ if there is
  a substitution $\Psi_g \vdash \lf{\sigma} : \Psi$ such that, for all
  suspended atomic propositions $p^+_\mlvl = {\sf a}\,\lf{t_1}\ldots \lf{t_n}$
  in $\Delta$, there exists a suspended persistent proposition
  $p^+_\mpers = {\sf a}\,\lf{(\sigma t_1)}\ldots\lf{(\sigma t_n)}$ in
  $\Delta_g$.
\end{definition}
\bigskip

One thing we might prove about the relationship between process states
and their generalizations is that generalizations can {\it simulate}
the process states they generalize: that is, that if $(\Psi_g;
\Delta_g)$ is a generalization of $(\Psi; \Delta)$ and $(\Psi; \Delta)
\leadsto_\Sigma (\Psi'; \Delta')$ then $(\Psi_g; \Delta_g)
\leadsto_{\Sigma_a} (\Psi'_g; \Delta'_g)$ where $(\Psi'_g; \Delta'_g)$
is a generalization of $(\Psi'; \Delta')$. This is true, but we're not
interested in generalization per se; rather, we're interested in a
stronger property, {\em abstraction}, that is defined in terms of
generalization:

\bigskip
\begin{definition}
A process state $(\Psi_a; \Delta_a)$ is an {\em abstraction} of 
$(\Psi_0; \Delta_0)$ under the signature $\Sigma$ if, for any trace
$(\Psi_0; \Delta_0) \leadsto^*_\Sigma (\Psi_n; \Delta_n)$, 
$(\Psi_a; \Delta_a)$ is a generalization of $(\Psi_n; \Delta_n)$. 
\end{definition}
\bigskip

An abstraction of the process state $(\Psi_0; \Delta_0)$ is therefore
a single process state that captures {\it all possible future
  behaviors} of the state $(\Psi_0; \Delta_0)$ because, for any atomic
proposition $p^+_\mlvl = {\sf a}\,\lf{t_1}\ldots \lf{t_n}$ that may be derived
by evolving $(\Psi_0; \Delta_0)$, there is a substitution $\lf\sigma$
such that ${\sf a}\,\lf{(\sigma t_1)}\ldots\lf{(\sigma t_n)}$ is already present
in the abstraction. The meta-approximation theorem relates this definition
of abstraction to the concept of approximate versions of programs as
specified by Definition~\ref{def:approxversion}.  

\bigskip
\begin{theorem}[Meta-approximation]\label{thm:metapprox}
  If $\Sigma_a$ is an approximate version of $\Sigma$, and if there is a
  state $(\Psi_0; \Delta_0)$ well-formed according to $\Sigma$, and if
  for some $\Psi_0' \vdash \lf\sigma : \Psi_0$ there is a trace
  $(\Psi_0'; \lf\sigma\Delta_0) \leadsto^*_{\Sigma_a} (\Psi_a; \Delta_a)$
  such that $(\Psi_a; \Delta_a)$ is a saturated process state, then
  $(\Psi_a; \Delta_a)$ is an abstraction of $(\Psi_0; \Delta_0)$.
\end{theorem}

\begin{proof}
As in \cite[Theorem 3]{simmons11logical}.
\end{proof}

The meaning of the meta-approximation theorem is that if (1) we can
approximate a specification and an initial state and (2) we can obtain
a saturated process state from that approximate specification
and approximate initial state, then the saturated process state captures
all possible future behaviors of the (non-approximate) initial state. 

\begin{figure}
\fvset{fontsize=\small,boxwidth=229pt}
\VerbatimInput{sls/pda-pers.sls}
\caption{Skolemized approximate version of the PDA specificaiton from
Figure~\ref{fig:pda-lin}}
\label{fig:pda-pers}
\end{figure}



\section{Using approximation}

The meta-approximation theorem gives us a way of building abstractions
from specifications and initial process states: we interpret the
approximate version of the program as a saturating logic program over
that initial state. If we can obtain a saturated process state using
the logic programming interpretation, it is an abstraction of the
initial process state. It is not always possible to obtain a saturated
process state using the logic programming interpretation, however:
rules like 
$\forall \lf{x}.\,{\sf a}\lf{(x)} \lefti \{ {\sf a}\lf{({\sf s}(x))} \}$ and 
$\forall \lf{x}.\,{\sf
  a}\lf{(x)} \lefti \{ \exists \lf{y}. {\sf a}\lf{(y)} \}$ 
lead to non-termination
when interpreted as saturating logic programs. Important classes 
of programs are known to terminate in all cases, such as those in the
so-called ``Datalog fragment'' where the only terms in the program
are variables and constants. However, we want to consider programs
that fall outside of this fragment, which means that we must 
reason explicitly about termination.

Consider the destination-passing PDA specification from
Figure~\ref{fig:pda-lin}. If we simply turn all linear predicates
persistent, then the ${\sf push}$ rule will lead to non-termination
because the head $\exists \lf{m}. {\sf stack}\,\lf{x}\,\lf{l}\,\lf{m} \fuse {\sf
  hd}\,\lf{m}\,\lf{r}$ introduces a new existential parameter $\lf{m}$. 
We can cope
by adding a new conclusion $\lf{m} \doteq \lf{t}$, but this means we 
have to pick a $\lf{t}$.
The most general starting point for selecting a $\lf{t}$ is to apply
Skolemization to the rule. By moving the existential quantifier for
$\lf{m}$ in front of the implicitly quantified $\lf{X}$, $\lf{L}$, $\lf{M}$, 
and $\lf{R}$, we
get a Skolem function $\lf{{\sf fm}\,X\,L\,M\,R}$ that takes four
arguments. Letting $\lf{t} = \lf{{\sf fm}\,X\,L\,M\,R}$ results in the
\sls~specification/logic program shown in
Figure~\ref{fig:pda-pers}.

Skolem functions provide a natural starting point for approximations,
even though the Skolem constant that arises directly from
Skolemization is usually more precise than we want. From the starting
point in Figure~\ref{fig:pda-pers}, however, we can define
approximations simply by instantiating the Skolem constant.  For
instance, we can equate the existentially generated destination in the
conclusion with the one given in the premise (letting $\lf{\sf fm} =
\lf{\lambda X. \lambda L. \lambda M. \lambda R.\, M}$). The result is
equivalent to this specification:

\smallskip
\fvset{fontsize=\small,boxwidth=229pt}
\VerbatimInput{sls/pda-pers-precise.sls}
\smallskip

\noindent This substitution yields a precise approximation
that exactly captures the behavior of the original PDA as a saturating
logic program. On the other hand, if we set $\lf{m}$ equal to $\lf{l}$ 
(letting $\lf{\sf fm} = \lf{\lambda X. \lambda L. \lambda M. \lambda R.\, L}$), 
the result is equivalent to this specification:

\smallskip
\fvset{fontsize=\small,boxwidth=229pt}
\VerbatimInput{sls/pda-pers-approx.sls}
\smallskip


If the initial process state contains a single atomic proposition
${\sf hd}\,\lf{d_0}\,\lf{d_1}$ in addition to all the ${\sf left}$ and ${\sf
  right}$ facts, then the two rules above maintain the invariant that,
as new facts are derived, the first argument of ${\sf hd}$ and the
second and third arguments of ${\sf stack}$ will {\it always} be
$\lf{d_0}$.  These arguments are therefore vestigial, like the extra
arguments to ${\sf eval}$ and ${\sf retn}$ discussed in
Section~\ref{sec:vestigial}, and we can remove them from the
approximate specification, resulting in the specification in
Figure~\ref{fig:pda-pers-approx2}. 

\begin{figure}[ht]
\fvset{fontsize=\small,boxwidth=229pt}
\VerbatimInput{sls/pda-pers-approx2.sls}
\caption{Approximated PDA specification}
\label{fig:pda-pers-approx2}
\end{figure}

This logical approximation of the
original PDA accepts every string where, for every form of bracket
$\lf{x}$, at least one left $\lf{x}$ appears before any of the right 
$\lf{x}$. The
string \obj{\mbox{\sf [ ] ] ] ( ( )}} 
would be accepted by this approximated PDA,
but the string \obj{\mbox{\sf ( ) ] [ [ ]}} 
would not, as the first right square
bracket appears before any left square bracket.


\section{Control flow analysis}
\label{sec:0cfa}

The initial process state for destination-passing SSOS specifications
generally has the form 
$(\lf{d}{:}{\sf dest}; x{:}\susp{{\sf eval}\,\lf{t}\,\lf{d}})$
for some program represented by the LF term $\lf{t} = \interp{e}$. 
This means that we
can use the meta-approximation result to derive abstractions from
initial expressions $\obj{e}$ using the saturating logic programming
interpretation of persistent SSOS specifications.

\begin{figure}
\fvset{fontsize=\small,boxwidth=229pt}
\VerbatimInput{sls/dest-env.sls}
\caption{Alternate environment semantics for CBV evaluation}
\label{fig:dest-env}
\end{figure}

The starting point for deriving a control flow analysis is the
environment semantics for call-by-value shown in
Figure~\ref{fig:dest-env}. It differs from the environment semantics
shown in Figure~\ref{fig:ssos-by-env} in three ways. First and
foremost, it is a destination-passing specification instead of an
ordered abstract machine specification, but that difference is
accounted for by the destination-adding transformation in Chapter~7.
A second difference is that the existentially generated parameter $\lf{x}$
associated with the persistent proposition ${\sf bind}\,\lf{x}\,\lf{v}$ is
introduced as late as possible (rule ${\sf ev/app2}$ in
Figure~\ref{fig:dest-env}), not as early as possible (rule ${\sf
  ev/app1}$ in Figure~\ref{fig:ssos-by-env}). The third difference is
that there is an extra frame $\lf{\sf app3}$ and an extra rule ${\sf
  ev/app3}$ that consumes such frames. The $\lf{\sf app3}$ frame is
an important part of the control flow analysis we derive, but in
\cite{simmons11logical} the addition of these frames was otherwise
unmotivated. Based on our discussion of the logical correspondence in
Chapter~6, we now have a principled account for this extra frame and
rule: it is precisely the pattern we get from operationalizing a
natural semantics {\it without} tail-recursion optimization and then
applying defunctionalization.

In order for us to approximate Figure~\ref{fig:dest-env} to derive a
finite control flow analysis, our first step is to deal with the
variables introduced by existential quantification. The variable $\lf{x}$
introduced in ${\sf ev/app2}$ will be equated with $\lf{{\sf var}(\lambda
x.E\,x)}$. The new constructor $\lf{\sf var}$ is a greatly simplified
Skolem function for $\lf{x}$ that only mentions the term $\lf{E}$ with LF type
${\sf exp} \rightarrow {\sf exp}$ -- the most general Skolem function
in this setting would have also been dependent on $\lf{V}$, $\lf{D}$, and
$\lf{D_2}$. Adding $\lf{x} \doteq \lf{{\sf var}(\lambda x.E\,x)}$ 
as a conclusion to
the rule ${\sf ev/lam2}$ effectively causes us to associate all
variables ever passed to an LF function with the {\it function into
  which that parameter was passed}.

\begin{figure}[t]
\fvset{fontsize=\small,boxwidth=229pt}
\VerbatimInput{sls/dest-cfa-1.sls}
\caption{A control-flow analysis derived from Figure~\ref{fig:dest-env}}
\label{fig:dest-cfa-1}
\end{figure}

The pattern above is an important one in the design of saturating
forward-chaining logic programs that use higher-order abstract syntax,
because it is a simple way of getting a handle on the subterms of a
higher-order term. When given a term $\lf{{\sf a}\,({\sf b}\,{\sf c}\,{\sf
  c})}$, it is clear that there are three distinct subterms: the entire
term, $\lf{{\sf b}\,{\sf c}\,{\sf c}}$, and $\lf{{\sf c}}$. Therefore, it is
meaningful to bound the size of a saturated process state using some
function that depends on the number of subterms of the original
term. But what are the subterms of $\lf{{\sf lam}\,(\lambda x.\,{\sf
  app}\,x\,x)}$?  Because we ensure that we {\it only} substitute terms
of the form $\lf{{\sf var}(\lambda x.e\,x)}$ into functions of the form
$\lf{\lambda x.e\,x}$ we can actually answer this question: there are three
distinct subterms of $\lf{{\sf lam}\,(\lambda x.\,{\sf app}\,x\,x)}$: the
entire term, $\lf{{\sf app}\,({\sf var}(\lambda x.\,{\sf
  app}\,x\,x))\,({\sf var}(\lambda x.\,{\sf app}\,x\,x))}$, and $\lf{{\sf
  var}(\lambda x.\,{\sf app}\,x\,x)}$.  The subterms of any closed term
$\lf{e}$ of LF type ${\sf exp}$ can be enumerated by running this
saturating logic program starting with the fact ${\sf subterms}\lf{(e)}$,
where ${\sf subterms}$ is a persistent positive proposition.
\begin{align*}
{\sf sub/lam} &: ~~
  \forall \lf{e}{:}{\sf exp} \rightarrow {\sf exp}.\,
    {\sf subterms}\lf{({\sf lam}(\lambda x.e\,x))} \lefti
      \{ {\sf subterms}\lf{(e({\sf var}(\lambda x.e\,x)))} \}
\\
{\sf sub/app} &: ~~
  \forall \lf{e_1}{:}{\sf exp}.\, \forall \lf{e_2}{:}{\sf exp}.\,
    {\sf subterms}\lf{({\sf app}\,e_1\,e_2)} \lefti
      \{ {\sf subterms}\,\lf{e_1} \fuse {\sf subterms}\,\lf{e_2} \}
\\
{\sf sub/var} &: ~~ 
  \forall \lf{e}{:}{\sf exp} \rightarrow {\sf exp}.\,
    {\sf subterms}\lf{({\sf var}(\lambda x.e\,x))} \lefti
      \{ {\sf subterms}\lf{(e({\sf var}(\lambda x.e\,x)))} \}
\end{align*}
The last rule, ${\sf sub/var}$, is redundant: if we ever derive a fact
of the form ${\sf subterms}\lf{({\sf var}(\lambda x.e\,x))}$, we know that
we previously derived ${\sf subterms}\lf{({\sf lam}(\lambda x.e\,x))}$ and
therefore, by the first rule, we have already derived ${\sf
  subterms}\lf{(e({\sf var}(\lambda x.e\,x))}$.



The discussion above pertains to the existentially generated variable
$\lf{x}$ in rule ${\sf ev/app2}$, but we still need some method for
handling destinations $\lf{d_1}$, $\lf{d_2}$, and $\lf{d_3}$ in ${\sf ev/app1}$,
${\sf ev/app2}$, and ${\sf ev/app3}$ (respectively). To this end, we
need to have in mind the question that we intend to answer with
control flow analysis. The primary question that such a flow analysis
is intended to answer is, ``for any given call site in the source
program, what are the functions that may be invoked at that
location?''\footnote{This kind of ``{\it may}-'' analysis, where the
  intention is to over-approximate the events that might happen, is
  the kind of analysis (as opposed to a ``{\it must}-'' analysis) that
  maps easily onto the meta-approximation theorem.} Call sites
correspond to expressions of the form $\lf{{\sf app}\,e_1\,e_2}$, and
functions are expressions of the form $\lf{{\sf lam}(\lambda x.e\,x)}$;
therefore, our next step is to equate the destinations introduced in
the $\lf{\sf app}$ rules with the expressions we are evaluating at this
point. One way to do this would be to introduce a new constructor
$\lf{\sf d}: {\sf exp} \rightarrow {\sf dest}$, but we can equivalently
conflate the two types to get the specification in
Figure~\ref{fig:dest-cfa-1}. 


The specification in Figure~\ref{fig:dest-cfa-1} has a point of
redundancy along the lines of the redundancy in our second PDA
approximation: the rules maintain the invariants that the two
arguments to ${\sf eval}\,\lf{e}\,\lf{d}$ are always the same. Therefore, the
second argument to ${\sf eval}$ can be treated as vestigial; by
removing that argument, we get a specification equivalent to
Figure~\ref{fig:dest-cfa-2}. That figure includes another
simplifications as well: instead of introducing expressions $\lf{d_1}$,
$\lf{d_2}$, and $\lf{d_3}$ by existential quantification just to equate them
with expressions $\lf{e_1}$, $\lf{e_2}$, and $\lf{e}$, 
we substitute in the equated
expressions where the respective destinations appeared in
Figure~\ref{fig:dest-cfa-1}. 

\begin{figure}[t]
\fvset{fontsize=\small,boxwidth=229pt}
\VerbatimInput{sls/dest-cfa-2.sls}
\caption{Simplification of Figure~\ref{fig:dest-cfa-1} that
  eliminates the redundant argument to ${\sf eval}$}
\label{fig:dest-cfa-2}
\end{figure}

This specification in Figure~\ref{fig:dest-cfa-2} is terminating when
interpreted as a saturating logic program because the rules only break
expressions $\lf{e}$ and values $\lf{v}$ 
into their subexpressions in the sense
we have described above. If the expressions in the original state of a
program have $n$ subterms, the program can derive no more than $n$ new
${\sf eval}$ facts, $n^2$ new ${\sf retn}$ facts, and $2n^2 + n$ ${\sf
  cont}$ facts. This analysis combined with the meta-approximation
theorem ensures that we have derived some sort of program analysis. To
see that this program analysis is a control flow analysis, note that
the third argument $\lf{e}$ to ${\sf comp}\,\lf{f}\,\lf{e'}\,\lf{e}$ 
is always a term
$\lf{{\sf app}\,e_1\,e_2}$ -- that is, a call site. The rule ${\sf
  ev/app2}$ starts evaluating the function $\lf{{\sf lam}(\lambda
x.e_0\,x)}$ and generates the fact ${\sf comp}\,\lf{\sf app3}\,\lf{(e({\sf
  var}(\lambda x.e_0\,x)))}\,\lf{e}$. This means that, in the course of
evaluating some initial expression $\lf{e_{\it init}}$, the function $\lf{{\sf
  lam}(\lambda x.e_0\,x)}$ may be called from the call site $\lf{e}$ only if
${\sf comp}\,\lf{\sf app3}\,\lf{(e({\sf var}(\lambda x.e_0\,x)))}\,\lf{e}$ appears
in a saturated process state that includes the persistent 
atomic proposition ${\sf eval}\lf{(e_{\it init})}$.

There is one important caveat to this analysis. If for some value $\obj{v}$
we consider the program $\interp{((\lambda x.x)\,(\lambda y.y))\,v}$,
we might expect a reasonable control flow analysis to notice that only
$\interp{\lambda y.y}$ is passed to the function $\interp{\lambda
  x.x}$ and that only $\obj{v}$ is passed to the function $\interp{\lambda
  y.y}$. Because of our use of higher-order abstract syntax, however,
$\interp{\lambda y.y}$ and $\interp{\lambda x.x}$ are alpha-equivalent
and therefore equal in the eyes of the logic programming
interpreter. This is not a problem with correctness, but it means that
our analysis may be less precise than expected. One solution would be
to add distinct labels to terms. Adding a label on the inside of every
lambda-abstraction would seem to suffice, and in any real example
labels would already be present in the form of source-code positions
or line numbers. The alias analysis presented next demonstrates the
use of such labels.

\begin{figure}
\fvset{fontsize=\small,boxwidth=229pt}
\VerbatimInput{sls/ssos-monadic.sls}
\caption{Semantics of functions in the simple monadic language}
\label{fig:ssos-monadic}
\end{figure}

\section{Alias analysis}

The control flow analysis above was derived from the SSOS
specification of a language that looked much like the Mini-ML-like
languages considered in Chapters~6~and~7.  In this section, we try to
derive an alias analysis that matches the interprocedural
object-oriented alias analysis presented as a logic program in
\cite[Chapter 12.4]{aho07compilers}.  When we try to reverse-engineer
a source language from which this alias analysis can be derived, we
get the monadic language presented below.

\subsection{Monadic language}

The language we consider differentiates atomic actions, which we will
call {\it expressions} (and encode in the LF type ${\sf exp}$) and
procedures or {\it commands} (which we encode in the LF type ${\sf
  cmd}$). There are only two commands $\obj{m}$ in our monadic language. The
first command, $\obj{{\sf ret}\,x}$, is a command that returns the value
bound to the variable $\obj{x}$ (rule ${\sf ev/ret}$ in
Figure~\ref{fig:ssos-monadic}). The second command, $\interp{{\sf
    bnd}^l\, x \leftarrow e \,{\sf in}\,m} = \lf{{\sf
  bnd}\,l\,\interp{e}\,\lambda x.\interp{m}}$, evaluates $\obj{e}$ to a
value, binds that value to the variable $\obj{x}$, and then evaluates
$\obj{m}$. Note the presence of $\obj{l}$ 
in the bind syntax; we will call it a
{\it label}, and we can think of it as a line number or source-code
position from the original program.

In the previous languages we have considered, values $\obj{v}$ were a
syntactic refinement of the expressions $\obj{e}$. In contrast, our monadic
language will differentiate the two: there are five expression forms
and three values that we will consider. An expression $\interp{\lambda
  x.e} = \lf{{\sf fun}\,\lambda x.\interp{e}}$ evaluates to a value
$\interp{\lambda^l x.e} = \lf{{\sf lam}\,l\,\lambda x.\interp{e}}$, where
the label $\obj{l}$ represents the source code position where the function
was bound. When we evaluate the command $\interp{{\sf
    bnd}^l \, y \leftarrow \lambda x.e \,{\sf in}\,m}$, the value
$\interp{\lambda^l x.e}$ gets bound to $\obj{y}$ in the body of the command
$\obj{m}$ (rule ${\sf ev/fun}$ in Figure~\ref{fig:ssos-monadic}).

The second expression form is a function call: $\interp{f\,x} = \lf{{\sf
  app}\,f\,x}$. To evaluate a function call, we expect a function value
(which is a command $\obj{m_0}$ with one free variable) to be bound to the
variable $\obj{f}$; we then store the rest of the current command on the
stack and evaluate the command $\obj{m_0}$ to a value. Note that the rule
${\sf ev/call}$ in Figure~\ref{fig:ssos-monadic} also stores the call
site's source-code location $\obj{l}$ on the stack frame. The reason for
storing a label here is that we need it for the alias
analysis. However, it is possible to independently motivate adding
these source-code positions to the operational semantics: for instance, it
would allow us to model the process of giving a stack trace when an
exception is raised. When the function we have called returns (rule
${\sf ev/call1}$ in Figure~\ref{fig:ssos-monadic}), we continue
evaluating the command that was stored on the control stack.

\begin{figure}
\fvset{fontsize=\small,boxwidth=229pt}
\VerbatimInput{sls/ssos-monadic2.sls}
\caption{Semantics of mutable pairs in the simple monadic language}
\label{fig:ssos-monadic2}
\end{figure}

The rules for mutable pairs are given in
Figure~\ref{fig:ssos-monadic2}. Evaluating the expression $\obj{\sf
  newpair}$ allocates a tuple with two fields $\obj{\sf fst}$ and 
$\obj{\sf
  snd}$ and yields a value $\obj{{\sf loc}\,l}$ referring to the tuple; both
fields in the tuple are initialized to the value $\obj{\sf null}$, and
each field is represented by a separate linear ${\sf cell}$ resource
(rule ${\sf ev/new}$). The expressions $\interp{x.{\sf fst}} = \lf{{\sf
  proj}\,x\,{\sf fst}}$ and $\interp{x.{\sf snd}} = \lf{{\sf proj}\,x\,{\sf
  snd}}$ expect a pair location to be bound to $\obj{x}$, and yield the value
stored in the appropriate field of the mutable pair (rule ${\sf
  ev/proj}$). The expressions $\interp{x.{\sf fst} := y} = \lf{{\sf
  set}\,x\,{\sf fst}\,y}$ and $\interp{x.{\sf snd} := y} = \lf{{\sf
  set}\,x\,{\sf snd}\,y}$ work much the same way. The difference is
that the former expressions do not change the accessed field's
contents, whereas the latter expressions replace the accessed field's
contents with the value bound to $\obj{y}$ (rule ${\sf ev/set}$).

This language specification bears some similarity to Harper's
Modernized Algol with free assignables \cite[Chapter
36]{harper12practical}. The {\it free assignables} addendum is
critical: SSOS specifications do not have a mechanism for enforcing
the stack discipline of Algol-like languages. The other major
difference is that Harper's version of Algol does not allow
expressions to access the state.  For our language to behave
similarly, the expressions $\obj{\sf newpair}$, $\obj{x.{\sf fst}}$, 
$\obj{x.{\sf
  snd}}$, $\obj{x.{\sf fst} := y}$, and 
$\obj{x.{\sf snd} := y}$ would need to be
commands and not expressions; we make them expressions only for the
convenience of a more regular presentation.

\begin{figure}
\fvset{fontsize=\small,boxwidth=229pt}
\VerbatimInput{sls/ssos-monadic-approx.sls}
\caption{Alias analysis for the simple mondaic language}
\label{fig:ssos-monadic-approx}
\end{figure}



\subsection{Approximation and alias analysis}


To approximate the semantics of our monadic language, we can follow the
methodology from before and turn the program persistent. A further
approximation is to remove the last premise from ${\sf ev/set}$, as
the meta-approximation theorem allows -- the only purpose of this
premise in Figure~\ref{fig:ssos-monadic2} was to consume the ephemeral
proposition ${\sf cell}\,\lf{l'}\,\lf{\it fld}\,\lf{v}$, and this is unnecessary
if ${\sf cell}$ is not an ephemeral predicate.  Having
made these two moves (turning all propositions persistent, and removing
a premise from ${\sf ev/set}$), we are left with three types of
existentially-generated variables that must be equated with concrete
terms in order for our semantics to be interpreted as a saturating
logic program:

\smallskip
\begin{itemize}
\item Variables $\lf{y}$, introduced by every rule except for ${\sf ev/ret}$,
\item Mutable locations $\lf{l}$, introduced by rule ${\sf ev/new}$, and 
\item Destinations $\lf{d}$; the only place where a destination is created
by the destination-adding transformation is in rule ${\sf ev/call}$.
\end{itemize}
\smallskip

Variables $\lf{y}$ are generated to be substituted into the body of some
command, so we could equate them with the Skolemized function body as
we did when deriving a control flow analysis example. Another option
comes from noting that, for any initial source program, every command
is associated with a particular source code location, so a simpler
alternative is just to equate the variable with that source code
location. This is why we stored labels on the stack: if we had not
done so, then the label $\obj{l}$ associated with $\obj{m}$ in the command
$\interp{{\sf bnd}^l\, x \leftarrow \lambda x.m_0 \,{\sf in}\,m}$ would
no longer be available when we needed it in rule ${\sf ev/call1}$.

We deal with mutable locations $\lf{l}$ is a similar manner: we equate them
with the label $\lf{l}$ representing the line where that cell was
generated.

There are multiple ways to deal with the destination $\lf{d_0}$ generated
in rule ${\sf ev/call}$. We want our analysis, like Aho et al.'s, to
be insensitive to control flow, so we will equate $\lf{d_0}$ with the label
$\lf{l_0}$ associated with the function we are calling.  If we instead
equated $\lf{d_0}$ with the label $\lf{l}$ associated with the call-site or with
the pair of the call site and the called function, the result would be
an analysis that is more sensitive to control flow.

The choices described above are reflected in
Figure~\ref{fig:ssos-monadic-approx}, which takes the additional step
of inlining uses of equality in the conclusions of rules. We can
invoke this specification as a program analysis by packaging a program
as a single command $\obj{m}$ and deriving a saturated process state from the
initial process state $(\lf{l_{\it init}}{:}{\sf loc}; x{:}\susp{{\sf
    eval}\,\interp{m}\,\lf{l_{\it init}}})$. 
The use of source-code position labels
makes the answers to some of the primary questions asked of an alias
analysis quite concise. For instance:

\smallskip
\begin{itemize}
\item {\it Might the first component of a pair created at label
    $\obj{l_1}$ ever reference a pair created at label $\obj{l_2}$?}
  Only if ${\sf cell}\,\lf{l_1}\,\lf{\sf fst}\,\lf{({\sf loc}\,l_2)}$
  appears in the saturated process state (and likewise for the second
  component).
\item {\it Might the first component of a pair created at label
    $\obj{l_1}$ ever reference the same object as the first component
    of a pair created at label $\obj{l_2}$?} Only if there is some
  $\lf{l'}$ such that ${\sf cell}\,\lf{l_1}\,\lf{\sf fst}\,\lf{({\sf
      loc}\,l')}$ and ${\sf cell}\,\lf{l_1}\,\lf{\sf fst}\,\lf{({\sf
      loc}\,l')}$ both appear in the saturated process state.
\end{itemize}

\section{Related work}
\label{sec:approximately-related}

The technical aspects of linear logical approximation are similar to
work done by Bozzano et al.~\cite{bozzano02effective,bozzano04model},
which was also based on the abstract interpretation of a logical
specification in linear logic.
They encode distributed systems and communication protocols in
a framework that is similar to the linear fragment of \sls~without
equality. Abstractions of those programs are then used to verify
properties of concurrent protocols that were encoded in the logic
\cite{bozzano02protocol}. 

There are a number of significant difference between our work and
Bozzano et al.'s, however. The style they use to encode protocols is
significantly different from any of the SSOS specification styles
presented in this thesis. They used a general purpose approximation,
which could therefore potentially be mechanized in the same way we
mechanized transformations like operationalization; in contrast, the
meta-approximation result described here captures a whole class of
approximations. Furthermore, Bozzano et al.'s methods are designed to
consider properties of a system as a whole, not static analyses of
individual inputs as is the case in our work.

Work by Might and Van Horn on abstracting abstract machines can be
seen as a parallel approach to our methodology in a very different
setting
\cite{might10resolving,might10abstract,might10abstracting}. Their
emphasis is on deriving a program approximation by approximating a
{\it functional} abstract interpreter for a programming language's
operational semantics. Their methodology is similar to ours in large
part because we are doing the same thing in a different setting,
deriving a program approximation by approximating a
destination-passing SSOS specification (which we could, in turn, have
derived from an ordered abstract machine by destination-adding).

Many of the steps that they suggest for approximating programs have
close analogues in our setting. For instance, their {\it
  store-allocated bindings} are analogous to the SSOS environment
semantics, and their {\it store-allocated continuations} -- which they
motivate by analogy to implementation techniques for functional
languages like SML/NJ -- are precisely the destinations that arise
naturally from the destination-adding transformation. The first
approximation step we take is forgetting about linearity in order to
obtain a (non-terminating) persistent logical specification. This step
is comparable to Might's first approximation step of ``throwing hats
on everything'' (named after the convention in abstract interpretation
of denoting the abstract version of a state space $\Sigma$ as
$\hat{\Sigma}$. The ``mysterious'' introduction of power domains that
this entails is, in our setting, a perfectly natural result of
relaxing the requirement that there be at most one persistent
proposition ${\sf bind}\,\lf{x}\,\lf{v}$ for every $\lf{x}$. 
As a final point of
comparison, the ``abstract allocation strategy'' discussed in
\cite{might10abstracting} is quite similar to our strategy of
introducing and then approximating Skolem functions as a means of
deriving a finite approximation. Our current discussion of Skolem
functions in Section~\ref{sec:0cfa} is partially inspired by the
relationship between our use of Skolemization and the discussion of
abstract allocation in \cite{might10abstracting}.

The independent discovery of a similar set of techniques for achieving
similar goals in such different settings (though both approaches were
to some degree inspired by Van Horn and Mairson's investigations of
the complexity of $k$-CFA \cite{vanhorn07relating}) is another
indication of the generality of both techniques, and the similarity
also suggests that the wide variety of approximations considered in
\cite{might10abstracting}, as well as the approximations of
object-oriented programming languages in \cite{might10abstract}, can
be adapted to this setting.
