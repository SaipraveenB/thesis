\chapter{Linear logical approximation}
\label{chapter-approx}

In this chapter, we will explore an approximation strategy based on
approximating ordered and linear logical specifications as persistent
logical specifications. 

Concurrent \sls~specifications where all positive atomic propositions
are persistent (and where all inclusions of negative propositions in
positive propositions -- if there are any -- have the form ${!}A$, not
${\downarrow}A^-$ or ${\gnab}A^-$) have a distinct logical and
operational character. Logically, by the discussion in
Section~\ref{sec:perm-fragments} we are justified in reading such
specifications as specifications in persistent intuitionstic logic or
persistent lax logic. Operationally, while persistent specifications
have an interpretation as transition systems, that interpretation is
not very useful. This is because if we can take a transition once --
for instance, using the rule ${\sf a} \lefti \{ {\sf b} \}$ to derive
the persistent atomic proposition ${\sf b}$ from the persistent atomic
proposition ${\sf a}$ -- none of the facts that enabled that
transition can be consumed, as all facts are presistent. Therefore, we
can continue to make the same transition indefinitely; in the
above-mentioned example, such transitions will derive multiple
redundant copies of ${\sf b}$.

The way we will understand the meaning of some persistent and
concurrent \sls~specifications is in terms of {\it saturation}. A
process state $(\Psi; \Delta)$ is saturated relative to the signature
$\Sigma$ if, for any step $(\Psi; \Delta) \leadsto_\Sigma (\Psi';
\Delta')$, it is the case that $\Psi = \Psi'$, $x{:}\susp{p^+_\mpers}
\in \Delta'$ implies $x{:}\susp{p^+_\mpers} \in \Delta$, and
$x{:}\ispers{A^-} \in \Delta'$ implies $x{:}\ispers{A^-} \in
\Delta$.\footnote{This means that a signatures with a rule that
  produces new variables by existential quantification, like ${\sf a}
  \lefti \{ \exists x. {\sf b}(x) \}$, can only have saturated process
  states in which that rule cannot fire at all. Notions of equality
  that can cope with some amount of existentially generated parameters
  are interesting, but are beyond the scope of this thesis.} A {\it
  minimal} saturated process state is one with no duplicated
propositions; we can compute a process state from any saturated
process state by removing duplicates. For purely persistent
specifications and process sates, minimal saturated process states are
unique when they exist: if $(\Psi; \Delta) \leadsto^*_\Sigma (\Psi_1;
\Delta_1)$ and $(\Psi; \Delta) \leadsto^*_\Sigma (\Psi_2; \Delta_2)$,
then the process states $(\Psi_1; \Delta_1)$ and $(\Psi_2; \Delta_2)$
may not be identitical, but they correspond to the same minimal process
states.

Furthermore, if a saturated process state exists for a given initial
process state, the minimal saturated process state can be computed by
the usual forward-chaining semantics where only transitions that
derive ${\it new}$ persistent atomic propositions or equalities $t
\doteq s$ are allowed. This forward-chaining logic programming
interpretation of persistent logic is extremely common; in fact, it is
what is commonly meant by ``foward-chaining logic programming.'' Just
as the term {\it persistent logic} was introduced in Chapter 2 to
distinguish what is traditionally referred to as intuitionstic logic
from the intuitionstic ordered and linear logic, we will use the term
{\it saturating logic programming} to distinguish what is
tranditionally referred to as foward-chaining logic programming from
the forward-chaining logic programming interpretation that makes sense
for for ordered and linear logical specifications.\robnote{If I want
  to talk a bit more about Linear Logical Algorithms in Chapter 5, I
  could say some more.}

\section{Logical transformation: approximation}
\label{sec:abstraction}

Our approximation strategy is simple: a signature in an ordered or
linear logical specification can be approximated by making all atomic
propositions persistent, and a flat, persistent rule $\forall
\overline{x}. A^+ \lefti \{ B^+ \}$ can be further approximated by
removing premises from $A^+$ and adding conclusions to $B^+$. Of
particular practical importance are added conclusions that equate the
parameters introduced by existential quantificaiton with terms: all
parameters introduced by existential quantification must be dealt with
as a necessary condition for interpreting a persistent signature as a
saturating logic program.

First, we define what it means for a specification to be an approximate
version of another specification:

\bigskip
\begin{definition}
  A flat, concurrent, and persistent specification $\Sigma_a$ is an
  {\em approximate version} of another specification $\Sigma$ if every
  predicate ${\sf a} : \Pi x_1{:}\tau_1 \ldots \Pi x_n{:}\tau_n.\,
  {\sf prop}\,{\sf lvl}$ deciared in $\Sigma$ has a corresponding
  predicate ${\sf a} : \Pi x_1{:}\tau_1 \ldots \Pi x_n{:}\tau_n.\,
  {\sf prop}\,{\sf pers}$ in $\Sigma_a$ and if, for every rule ${\sf
    r} : \forall \overline{x}.\,A_1^+ \lefti \{ A_2^+ \}$ in $\Sigma$ there
  is a corresponding rule ${\sf r} : \forall \overline{x}.\,B_1^+ \lefti
  \{ B_2^+ \}$ in $\Sigma_a$ such that:
  \begin{itemize}
  \item The existential quantifiers in $A_1^+$ and $A_2^+$ are
    identital to the existential quantifiers in $B_1^+$ and $B_2^+$
    (respectively),
  \item For each premise ($p^+_\mpers$ or $t \doteq s$) in $B^+_1$,
    the same premise appears in $A^+_1$, and 
  \item For each conclusion ($p^+$, $p^+_\meph$, $p^+_\mpers$, or $t
    \doteq s$) in $A^+_2$, the same premise appears in $B^+_2$.
  \end{itemize}
\end{definition}
\bigskip

Next, we give a definition of what it means for a state to be an 
approximate version (we use the word ``abstraction'') 

\section{Control flow analysis}

\section{Alias analysis}
