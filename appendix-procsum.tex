\chapter{Process states summary}
\label{appendix-procsum}

In this dissertation, we emphasize the use of {\it process states} to
describe the internal states of the evolving systems we are interested in.
This view is an extension of Miller's {\it processes-as-formula}
interpretation \cite{miller92pi,deng12relating}. Of course, a process
state is not a formula; Section~\ref{sec:whytraces} discusses our
emphasis on \sls~process states instead of \sls~formulas as the fundamental
representation of the internal states of evolving systems.

A process state as defined in Chapter~\ref{chapter-framework} has the
form $(\Psi; \Delta)_{\lf{\sigma}}$, though outside of
Chapter~\ref{chapter-framework} we never use the associated
substitution $\lf{\sigma}$, writing $(\Psi; \Delta)$ to indicate the
empty substitution $\lf{\sigma} = \lf{\cdot}$. The first-order context
$\Psi$, which is sometimes omitted, is also called the {\it LF
  context} in \sls~because Spine Form LF is the first-order term
language of \sls~(Section~\ref{sec:sls-termlanguage}). $\Delta$ is a
{\it substructural context}.

\section{Substructural contexts}

A substructural context (written as $\Delta$ and occasionally as $\Xi$)
is a sequence of variable bindings $x{:}\islvl{T}$ -- all the
variables $x$ bound in a context must be distinct. In \sls, $\mlvl$ is
either $\mtrue$ (ordered resources), $\meph$ (mobile resources, also
called ephemeral or linear resources), or $\mpers$ (persistent
resources). 

In stable process states, $T$ is usually a {\it suspended positive
  atomic proposition} $\susp{Q}$.  The permeability of a positive
atomic proposition (ordered, mobile/linear/ephemeral, or persistent)
is one of its intrinsic properties (Section~\ref{sec:permeable},
Section~\ref{sec:ordpolarprop}), so we can write $x{:}\susp{Q}$
instead of $x{:}\istrue{\susp{Q}}$, $x{:}\iseph{\susp{Q}}$, or
$x{:}\ispers{\susp{Q}}$ if the permeability of $Q$ is known from the
context. So the encoding of the string \obj{\mbox{{\sf [
      \textless~\textgreater~( [ ] )}}}, described in the introduction
as
\[
{\sf 
  left(sq) ~~
  left(an) ~~
  right(an) ~~
  left(pa) ~~
  left(sq) ~~
  right(sq) ~~
  right(pa)}
\]
is more properly described as
{\footnotesize \[ 
{
  x_1{:}\susp{\sf left(sq)}, ~~
  x_2{:}\susp{\sf left(an)}, ~~
  x_3{:}\susp{\sf right(an)}, ~~
  x_4{:}\susp{\sf left(pa)}, ~~
  x_5{:}\susp{\sf left(sq)}, ~~
  x_6{:}\susp{\sf right(sq)}, ~~
  x_7{:}\susp{\sf right(pa)}
}
\]}We write $x_1{:}{\susp{\sf left(sq)}}$ instead of
$x_1{:}\istrue{\susp{\sf left(sq)}}$ above, leaving implicit the fact that 
${\sf left}$ and ${\sf right}$ are ordered predicates.


It is also possible, in {\it nested} \sls~specifications
(Section~\ref{section-introtologicalcorrespondence},
Section~\ref{sec:operationalization}), to have variable bindings
$x{:}\istrue{A^-}$, $x{:}\iseph{A^-}$, and $x{:}\ispers{A^-}$. These
nested specifications act much like rules in the \sls~signature,
though mobile rules ($x{:}\iseph{A^-}$) can only be used one time, and
ordered rules ($x{:}\istrue{A^-}$) can only be used one time and only
in one particular part of the context (Figure~\ref{fig:ho-evo-ex}).

Chapter~\ref{chapter-order} treats substructural contexts strictly as
sequences, but in later chapters we treat substructural contexts in a
more relaxed fashion, allowing mobile/linear/ephemeral and persistent
variable bindings to be tacitly reordered relative to one another
other and relative to ordered propositions. In this relaxed reading,
$(x_1{:}\istrue{\susp{Q_1}}, x_2{:}\istrue{\susp{Q_2}})$ and
$(x_2{:}\istrue{\susp{Q_2}}, x_1{:}\istrue{\susp{Q_1}})$ are not
equivalent contexts but $(x_3{:}\ispers{\susp{Q_3}},
x_2{:}\istrue{\susp{Q_2}})$ and $(x_2{:}\istrue{\susp{Q_2}},
x_3{:}\ispers{\susp{Q_3}})$ are. % $x_3{:}\ispers{\susp{Q_3}},
% x_4{:}\iseph{\susp{Q_4}}$ and $x_4{:}\iseph{\susp{Q_4}},
% x_3{:}\ispers{\susp{Q_3}}$ are equivalent as well.

A {\it frame}
$\Theta$ represents a context with a hole in it. The notation
$\tackon{\Theta}{\Delta}$ tacks the substructural context $\Delta$
into the hole in $\Theta$; the context and the frame must have
disjoint variable domains for this to make sense. In
Chapter~\ref{chapter-order}, frames are interrupted sequences of variable
bindings
$x_1{:}\islvl{T_1},\ldots x_n{:}\islvl{T_n}, \Box,
x_{n+1}{:}\islvl{T_{n+1}}, \ldots x_m{:}\islvl{T_{m}}$, where the box
represents the hole. In later chapters, this is relaxed in
keeping with the relaxed treatment of contexts modulo reordering
of mobile and persistent variable bindings.

\section{Steps and traces}

Under focusing, a \sls~proposition can correspond to some number of
synthetic transitions (Section~\ref{sec:linsynthetic},
Section~\ref{sec:framework-concurrent}). The declaration ${\sf rule} :
Q_1 \fuse Q_2 \lefti \{ Q_3 \fuse Q_2 \}$\footnote{This is
  synonymous with the proposition $Q_1 \fuse Q_2 \lefti
  {\ocircle}(Q_3 \fuse Q_2)$ (Section~\ref{sec:slsframework}).}  in an
\sls~signature $\Sigma$, where $Q_1$ is ordered, $Q_2$ is mobile, and
$Q_3$ is persistent, is associated with this synthetic transition:
\[
\tackon{\Theta}{x_1{:}\istrue{\susp{Q_1}}, ~~ 
                x_2{:}\iseph{\susp{Q_2}}} 
 \leadsto_\Sigma
\tackon{\Theta}{y_1{:}\ispers{\susp{Q_3}}, ~~
                y_2{:}\iseph{\susp{Q_2}}}
\]
The variable bindings $x_1$ and $x_2$ no longer appear in
$\tackon{\Theta}{y_1{:}\ispers{\susp{Q_3}},
  y_2{:}\iseph{\susp{Q_2}}}$. The proof terms associated with
synthetic transitions are {\it steps}
(Section~\ref{sec:framework-concurrent}), and the step corresponding
to the synthetic transition above is written as $\trstep{y_1,
  y_2}{{\sf rule}\,(\tfuser{x_1}{x_2})}$. As described in
Section~\ref{sec:framework-concurrent}, we can relate the step to the
synthetic transition like this:
\[
\trstep{y_1, y_2}{{\sf
    rule}\,(\tfuser{x_1}{x_2})}
::
\tackon{\Theta}{x_1{:}\istrue{\susp{Q_1}}, ~~ 
                x_2{:}\iseph{\susp{Q_2}}} 
 \leadsto_\Sigma
\tackon{\Theta}{y_1{:}\ispers{\susp{Q_3}}, ~~
                y_2{:}\iseph{\susp{Q_2}}}
\]
As described in Section~\ref{sec:presentingtraces}, we can also use
a more Hoare-logic inspired notation:
\begin{align*}
&\qquad
\tackon{\Theta}{x_1{:}\istrue{\susp{Q_1}}, ~~ 
                x_2{:}\iseph{\susp{Q_2}}} 
\\
&\trstep{y_1, y_2}{{\sf
    rule}\,(\tfuser{x_1}{x_2})}
\\
&\qquad
\tackon{\Theta}{y_1{:}\ispers{\susp{Q_3}}, ~~
                y_2{:}\iseph{\susp{Q_2}}}
\end{align*}

The reflexive-transitive closure of $\leadsto_{\Sigma}$ is
$\leadsto^*_{\Sigma}$, and the proof terms witnessing these sequences
of synthetic transitions are {\it traces} $T ::= \emptytrace \mid S
\mid T; T$.  {\it Concurrent equality} (Section~\ref{sec:framework-concurrenteq}) is an equivalence relation on
traces that allows us to rearrange the steps $S_1 = \trstep{P_1}{R_1}$
and $S_2 = \trstep{P_2}{R_2}$ in a trace when the variables introduced
by $P_1$ (the output interface of $S_1$, written $S_1{^\bullet}$) are
not mentioned in $R_2$ (the input interface of $S_2$, written
${^\bullet}S_2$) and vice versa.

